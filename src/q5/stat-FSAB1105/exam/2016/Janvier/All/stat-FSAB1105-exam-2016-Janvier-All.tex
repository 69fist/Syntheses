\documentclass[en]{../../../../../../eplexam}

\usepackage{../../../../../../eplunits}
\usepackage{../../../../stat-FSAB1105}

\hypertitle{Probabilité et Statistiques}{5}{FSAB}{1105}{2016}{Janvier}
{}
{Anouar El Ghouch et Rainer von Sachs}

L'examen était en anglais et consistait en 5 questions dont la répartition des points était, respectivement: /3 , /7, /4, /2, /4, le tout /20.

%%%%%%%%%%
% QUESTION 1
%%%%%%%%%% 
\section{(/3)}
Soient deux variables aléatoires X et Y qui suivent respectivement une loi
\begin{itemize}
    \item[.] $X \sim \gammad(\alpha_1,\beta)$
    \item[.] $Y \sim \gammad(\alpha_2,\beta)$
\end{itemize}
\begin{enumerate}
    \item \begin{enumerate}
    \item Quelle est la joint density function de $(U,V)=(X+Y,\frac{X}{X+Y})$? Est-ce que $U$ et $V$ sont indépendants? Justifier
    \item Quelle est la distribution (laquelle?) de $U$ ? Quel est son nom? Quels sont ses parametres? Justifier
\end{enumerate}
\item Si $X$ est continu avec comme cumulative distribution function $F$, quelle est la distribution de $Y=-\ln{F(x)}$? Quel est son nom? Queles sont ses paramètres? Justifier
\end{enumerate}
\nosolution

%%%%%%%%%%
% QUESTION 2
%%%%%%%%%% 
\section{(/7)}
Soient $X_1,\ldots,X_n$, des valeurs aléatoires dont la probability density function est
\begin{equation}
  f(x)=\frac{\alpha}{\beta}x^{\alpha-1}\exp{(\frac{-x^\alpha}{\beta})} \text{   avec }x>0
\end{equation}
$\alpha,\beta>0$ et $\alpha$ connu et $\beta$ inconnu
\begin{enumerate}
    \item Soit $\pi \in (0,1)$ donné, trouver $q_\pi$, le $\pi$-quantile de $X$ en fonction de $\alpha, \beta$
    \item Quelle est la distribution de $X^\alpha$ ? Quel est son nom, ses paramètres? Justifier?
    \item Quel est $\widehat{\beta}$, l'estimateur de $\beta$ ? Que vaut $\mse(\widehat{\beta})$? Est que $\widehat{\beta}$ est ``consistent''? En déduire $\widehat{q}_\pi$, un estimateur ``consistent'' pour $q_\pi$. Justifier
    \item Utiliser la mgf pour montrer que $n\widehat{\beta}\sim \gammad(n,\beta)$. En déduire la distribution de $\dfrac{2n\widehat{\beta}}{\beta}$. Justifier
    \item On veut tester $H_0$  : $\beta\leq\beta_0$ VS $H_1$ : $\beta>\beta_0$. Suggérez un test statistic \footnote{``Suggest a statistical test''} and dérivez-en la $p$-value. Justifiez chaque étape
    \item Donnez un intervalle de confiance de \SI{95}{\%} pour $q_\pi$
    \item Quelle est la distribution asymptotique de $\widehat{q}_\pi$ ? Donnez l'intervalle de confiance asymptotique de $q_\pi$
\end{enumerate}
\nosolution

%%%%%%%%%%
% QUESTION 3
%%%%%%%%%% 
\section{(/4)}
Soit une distribution jointe:
\begin{equation}
    f(x,y)=\dfrac{3y}{4} \text{ ($0\leq y \leq x \leq 2$)}
\end{equation}
\begin{enumerate}
    \item $F(Y|X=x)$ ?
    \item $P(Y>1/2|X=3/2)$ et $V(Y|X=3/2)$ ?
    \item $P(X+Y<2)$ ?
    \item $V(X-2Y)$ ?
\end{enumerate}
\nosolution

%%%%%%%%%%
% QUESTION 4
%%%%%%%%%% 

Les deux questions suivantes sont faites de tête, une aide à la restitution est la bienvenue!

\section{(/2)}
Des composants électroniques sont placés en parralèle. Il y en a $n_1$ qui suivent une distribution exponentielle (en nombre d'années) $\sim \expo(\beta_1)$ et $n_2$ dont la distribution est une exponentielle (en nombre d'années) $\sim \expo(\beta_2)$ avec $n_1+n_2=n$ et $n>2$. Sachant que le système ne fonctionne plus si TOUS les composants ne fonctionnent plus, quelle est la probabilité que le système dans sa globalité tiendra plus de 10 ans? Exprimer votre réponse en terme $\beta_1$, $\beta_2$, $n_1$ et $n_2$.
\nosolution

%%%%%%%%%%
% QUESTION 5
%%%%%%%%%% 
\section{(/4)}
Une population de composants a une probabilité de dysfonctionnement de \SI{8}{\%}.
\begin{enumerate}
    \item Si $n=12$, quelle est la probabilité que plus de 2 composants aient un défaut?
    \item si $n=200$, quell est la probabilité que moins de 20 composants aient un défaut?
    \item On teste les composants l'un a à la suite de l'autre. Quelle est la probabilité que l'on doive en tester plus de 3 avant de trouver le premier dysfonctionnant?
    \item On teste les composants l'un a à la suite de l'autre. Quelle est la probabilité que l'on doive en tester plus de 5 avant d'en trouver 3 dysfonctionnant?
\end{enumerate}
\nosolution

\end{document}
