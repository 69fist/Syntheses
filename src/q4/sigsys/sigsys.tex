%analogique:
%------> temps
%^
%|
%|
%| continuum
\documentclass[11pt,a4paper]{article}

% French
\usepackage[utf8x]{inputenc}
\usepackage[frenchb]{babel}
\usepackage[T1]{fontenc}
\usepackage{lmodern}
\usepackage{ifthen}

% Color
% cfr http://en.wikibooks.org/wiki/LaTeX/Colors
\usepackage{color}
\usepackage[usenames,dvipsnames,svgnames,table]{xcolor}
\definecolor{dkgreen}{rgb}{0.25,0.7,0.35}
\definecolor{dkred}{rgb}{0.7,0,0}

% Floats and referencing
\newcommand{\sectionref}[1]{section~\ref{sec:#1}}
\newcommand{\annexeref}[1]{annexe~\ref{ann:#1}}
\newcommand{\figuref}[1]{figure~\ref{fig:#1}}
\newcommand{\tabref}[1]{table~\ref{tab:#1}}
\usepackage{xparse}
\NewDocumentEnvironment{myfig}{mm}
{\begin{figure}[!ht]\centering}
{\caption{#2}\label{fig:#1}\end{figure}}

% Listing
\usepackage{listings}
\lstset{
  numbers=left,
  numberstyle=\tiny\color{gray},
  basicstyle=\rm\small\ttfamily,
  keywordstyle=\bfseries\color{dkred},
  frame=single,
  commentstyle=\color{gray}=small,
  stringstyle=\color{dkgreen},
  %backgroundcolor=\color{gray!10},
  %tabsize=2,
  rulecolor=\color{black!30},
  %title=\lstname,
  breaklines=true,
  framextopmargin=2pt,
  framexbottommargin=2pt,
  extendedchars=true,
  inputencoding=utf8x
}

\newcommand{\matlab}{\textsc{Matlab}}
\newcommand{\octave}{\textsc{GNU/Octave}}
\newcommand{\qtoctave}{\textsc{QtOctave}}
\newcommand{\oz}{\textsc{Oz}}
\newcommand{\java}{\textsc{Java}}
\newcommand{\clang}{\textsc{C}}
\newcommand{\keyword}{mot clef}

% Math symbols
\usepackage{amsmath}
\usepackage{amssymb}
\usepackage{amsthm}
\DeclareMathOperator*{\argmin}{arg\,min}
\DeclareMathOperator*{\argmax}{arg\,max}

% Sets
\newcommand{\Z}{\mathbb{Z}}
\newcommand{\R}{\mathbb{R}}
\newcommand{\Rn}{\R^n}
\newcommand{\Rnn}{\R^{n \times n}}
\newcommand{\C}{\mathbb{C}}
\newcommand{\K}{\mathbb{K}}
\newcommand{\Kn}{\K^n}
\newcommand{\Knn}{\K^{n \times n}}

% Chemistry
\newcommand{\std}{\ensuremath{^{\circ}}}
\newcommand\ph{\ensuremath{\mathrm{pH}}}

% Theorem and definitions
\theoremstyle{definition}
\newtheorem{mydef}{Définition}
\newtheorem{mynota}[mydef]{Notation}
\newtheorem{myprop}[mydef]{Propriétés}
\newtheorem{myrem}[mydef]{Remarque}
\newtheorem{myform}[mydef]{Formules}
\newtheorem{mycorr}[mydef]{Corrolaire}
\newtheorem{mytheo}[mydef]{Théorème}
\newtheorem{mylem}[mydef]{Lemme}
\newtheorem{myexem}[mydef]{Exemple}
\newtheorem{myineg}[mydef]{Inégalité}

% Unit vectors
\usepackage{esint}
\usepackage{esvect}
\newcommand{\kmath}{k}
\newcommand{\xunit}{\hat{\imath}}
\newcommand{\yunit}{\hat{\jmath}}
\newcommand{\zunit}{\hat{\kmath}}

% rot & div & grad & lap
\DeclareMathOperator{\newdiv}{div}
\newcommand{\divn}[1]{\nabla \cdot #1}
\newcommand{\rotn}[1]{\nabla \times #1}
\newcommand{\grad}[1]{\nabla #1}
\newcommand{\gradn}[1]{\nabla #1}
\newcommand{\lap}[1]{\nabla^2 #1}


% Elec
\newcommand{\B}{\vec B}
\newcommand{\E}{\vec E}
\newcommand{\EMF}{\mathcal{E}}
\newcommand{\perm}{\varepsilon} % permittivity

\newcommand{\bigoh}{\mathcal{O}}
\newcommand\eqdef{\triangleq}

\DeclareMathOperator{\newdiff}{d} % use \dif instead
\newcommand{\dif}{\newdiff\!}
\newcommand{\fpart}[2]{\frac{\partial #1}{\partial #2}}
\newcommand{\ffpart}[2]{\frac{\partial^2 #1}{\partial #2^2}}
\newcommand{\fdpart}[3]{\frac{\partial^2 #1}{\partial #2\partial #3}}
\newcommand{\fdif}[2]{\frac{\dif #1}{\dif #2}}
\newcommand{\ffdif}[2]{\frac{\dif^2 #1}{\dif #2^2}}
\newcommand{\constant}{\ensuremath{\mathrm{cst}}}

% Numbers and units
\usepackage[squaren, Gray]{SIunits}
\usepackage{sistyle}
\usepackage[autolanguage]{numprint}
%\usepackage{numprint}
\newcommand\si[2]{\numprint[#2]{#1}}
\newcommand\np[1]{\numprint{#1}}

\newcommand\strong[1]{\textbf{#1}}
\newcommand{\annexe}{\part{Annexes}\appendix}

% Bibliography
\newcommand{\biblio}{\bibliographystyle{plain}\bibliography{biblio}}

\usepackage{fullpage}
% le `[e ]' rend le premier argument (#1) optionnel
% avec comme valeur par défaut `e `
\newcommand{\hypertitle}[7][e ]{
\usepackage{hyperref}
{\renewcommand{\and}{\unskip, }
\hypersetup{pdfauthor={#6},
            pdftitle={Synth\`ese d#1#2 Q#3 - L#4#5},
            pdfsubject={#2}}
}

\title{Synth\`ese d#1#2 Q#3 - L#4#5}
\author{#6}

\begin{document}

\ifthenelse{\isundefined{\skiptitlepage}}{
\begin{titlepage}
\maketitle

 \paragraph{Informations importantes}
   Ce document est grandement inspiré de l'excellent cours
   donné par #7 à l'EPL (École Polytechnique de Louvain),
   faculté de l'UCL (Université Catholique de Louvain).
   Il est écrit par les auteurs susnommés avec l'aide de tous
   les autres étudiants, la vôtre est donc la bienvenue.
   Il y a toujours moyen de l'améliorer, surtout si le cours
   change car la synthèse doit alors être modifiée en conséquence.
   On peut retrouver le code source à l'adresse suivante
   \begin{center}
     \url{https://github.com/Gp2mv3/Syntheses}.
   \end{center}
   On y trouve aussi le contenu du \texttt{README} qui contient de plus
   amples informations, vous êtes invité à le lire.

   Il y est indiqué que les questions, signalements d'erreurs,
   suggestions d'améliorations ou quelque discussion que ce soit
   relative au projet
   sont à spécifier de préférence à l'adresse suivante
   \begin{center}
     \url{https://github.com/Gp2mv3/Syntheses/issues}.
   \end{center}
   Ça permet à tout le monde de les voir, les commenter et agir
   en conséquence.
   Vous êtes d'ailleurs invité à participer aux discussions.

   Vous trouverez aussi des informations dans le wiki
   \begin{center}
     \url{https://github.com/Gp2mv3/Syntheses/wiki}.
   \end{center}
   comme le statut des synthèses pour chaque cours
   \begin{center}
     \url{https://github.com/Gp2mv3/Syntheses/wiki/Status}.
   \end{center}
   vous pouvez d'ailleurs remarquer qu'il en manque encore beaucoup,
   votre aide est la bienvenue.

   Pour contribuer au bug tracker et au wiki, il vous suffira de
   créer un compte sur Github.
   Pour interagir avec le code des synthèses,
   il vous faudra installer \LaTeX.
   Pour interagir directement avec le code sur Github,
   vous devez utiliser \texttt{git}.
   Si cela pose problème,
   nous sommes évidemment ouverts à des contributeurs envoyant leurs
   changements par mail ou n'importe quel autre moyen.
\end{titlepage}
}{}

\ifthenelse{\isundefined{\skiptableofcontents}}{
\tableofcontents
}{}
}


\usepackage{pgfplots}
\newcommand{\fourier}{\mathcal{F}}
\newcommand{\laplace}{\mathcal{L}}
\newcommand{\laplacu}{\mathcal{L}_u}
\newcommand{\transfz}{\mathcal{Z}}
\newcommand{\transzu}{\mathcal{Z}_u}
\newcommand{\roc}{\mathrm{ROC}}
\DeclareMathOperator{\cof}{cof}


\hypertitle{fr}{Signaux et syst\`emes}{4}{FSAB}{1106}
{Beno\^it Legat}
{Luc Vandendorpe et Vincent Wertz}

\lstset{language=MATLAB}

\paragraph{Prérequis}
Il sera supposé tout au long de cette synthèse que les
formules
\begin{align*}
  f & = \frac{1}{T} & \omega & = 2\pi f
\end{align*}
sont maitrisées et connues.
Elles seront utilisées abondamment sans le signaler.
La deuxième formule est d'ailleurs l'origine de beaucoup
de facteur $2\pi$ dans les formules car $\dif\omega = 2\pi \dif f$.

%  ____  _                   _
% / ___|(_) __ _ _ __   __ _| |___
% \___ \| |/ _` | '_ \ / _` | / __|
%  ___) | | (_| | | | | (_| | \__ \
% |____/|_|\__, |_| |_|\__,_|_|___/
%          |___/

\part{Signaux}
\section{Signaux}
Un signal peut-être vu soit de façon continue soit de façon discrète.
On notera les signaux continus avec les parenthèses (e.g. $x(t)$)
et les signaux discrets avec des crochets (e.g. $x[n]$).
Quand on notera juste $x$, ça signifiera que la propriété est vraie
en discret et en continu.

Pour les signaux discrets, on considère qu'on prend les points où on
évalue le signal à un intervalle régulier.
Du coup, on peut juste donner un entier $n$ en argument.
On nomme $T_s$, la période entre deux évaluations.
En effet, passer d'un signal continu $x(t)$ à un signal discret
$x[n] = x(nT_s)$ est appelé de l'échantillonnage qui en anglais se dit
``sampling''. D'où le $T_s$ qui réfère à la période de ``sampling''.
L'échantillonnage est vu plus en détail à la \sectionref{sampling}.

\subsection{Propriétés d'un signal}
Un signal peut être pair si
\[ x(-t) = x(t) \]
ou impair si
\[ x(-t) = -x(t). \]
%On peut décomposer un signal complexe
%\[ z(t) = x(t) + jy(t) \]
%Symétrique conjuguée
%\begin{align*}
%  z(t) = z^*(-t)\\
%  x(t) +jy(t) = x(-t) - jy(-t)
%\end{align*}
%\begin{align*}
%  x(t) = x_p(t) + x_i(t)\\
%  x(-t) = x_p(t) - x_i(t)\\
%  x(t) + x(-t) = 2x_p(t)\\
%  x(t) - x(-t) = 2x_i(t)
%\end{align*}

Un signal peut être périodique.
C'est à dire qu'il existe $T$ tel que
$\forall t$,
\[ x(t) = x(t + T). \]
Le plus petit $T$ respectant cela est appelé sa période.

Par exemple,
\[ x(t) = B\cos(\omega t + \phi) \]
est périodique de période $\frac{2\pi}{\omega}$.
\[ x[n] = A\cos(\Omega n + \Phi) \]
est périodique si $\exists k, N \in \mathbb{Z}$ tels que
$\Omega = 2\pi\frac{k}{N}$, la période est alors $N$ divisé
par le PGCD de $k$ et $N$.

\section{Transformée de Fourier}
Nous noterons la transformée de Fourier d'un signal $x$, $\fourier\{x\}$.

\subsection{Transformée de Fourier continue}
Toute fonction $x(t)$ peut être non seulement décrite par son continuum de
valeurs aux temps $t$ mais aussi par un continuum d'amplitudes à des fréquences
$f$.
On nomme la fonction qui donne l'amplitude de ces fréquences sa transformée
de Fourier.
On la note avec une majuscule et au lieu de lui donner en argument la fréquence
$f$, on lui donne $j\omega$ ce qui ne change pas grand chose car
$j\omega = j2\pi f$.
On a les relations suivantes
\begin{align}
  \label{eq:tfc}
  X(j\omega) & = \int_{-\infty}^\infty x(t) e^{-j\omega t} \dif t\\
  \label{eq:tfci}
  x(t) & = \frac{1}{2\pi} \int_{-\infty}^\infty X(j\omega) e^{j\omega t}
  \dif \omega.
\end{align}

Comme $|e^{-j\omega t}| = 1$, on remarque que l'intégrale
converge si $\int_{-\infty}^\infty |x(t)| \dif t < \infty$.

\subsection{Transformée de Fourier discrète}
\label{sec:fourier_discret}
En discret, on pourrait imaginer qu'on va appliquer la même formule en prenant
simplement $x(t) =  \sum_{n=-\infty}^\infty x[n] \delta(t-nT_s)$, c'est à dire
qu'on aurait, en notant $\Omega = \omega T_s$,
\begin{align}
  \label{eq:tfd}
  X(j\omega) & = \int_{-\infty}^\infty \sum_{n=-\infty}^\infty x[n]
  \delta(t-nT_s) e^{-j\omega T_sn} \dif t\\
  \nonumber
  & = \sum_{n=-\infty}^\infty x[n] e^{-j\Omega n}.
\end{align}
Seulement, on peut remarquer que
\begin{align*}
  X(j(\omega+2\pi/T_s)) & = \sum_{n=-\infty}^\infty x[n]
  e^{-j(\Omega+2\pi) n} \dif t\\
  & = \sum_{n=-\infty}^\infty x[n]
  e^{-j\Omega n} \dif t\\
  & = X(j\omega).
\end{align*}
La transformée de Fourier serait donc périodique de période $\frac{2\pi}{T_s}$.
Ça justifie d'ailleurs le tableau~\ref{tab:recap_fourier}.

En connaissance de cela, une transformée de Fourier discrète plus appropriée
a été créée
\begin{align*}
  X(e^{j\Omega}) & = \sum_{n=-\infty}^\infty x[n]e^{-j\Omega n}\\
  x[n] & = \frac{1}{2\pi}
  \int_{0}^{2\pi} X(e^{j\Omega})e^{j\Omega n} \dif \Omega.
\end{align*}
À nouveau, comme $|e^{-j\Omega t}| = 1$, on remarque que la somme
converge si $\sum_{n=-\infty}^\infty |x[n]| < \infty$.

On pourrait se demander pourquoi on note $X(e^{j\Omega})$ et pas $X(j\Omega)$.
Il se fait que c'est en fait beaucoup plus explicite de le noter ainsi.
En effet, on montre ici bien que c'est périodique de période $2\pi$
en $\Omega$ car, si on prend $\Omega + 2\pi$,
on obtient $X(e^{j(\Omega + 2\pi)}) = X(e^{j\Omega})$.

La raison pour laquelle on parle en terme de $\Omega$ et non de $\omega$ n'est pas
évidente à saisir.
Lorsqu'on a un signal $x[n]$, parler de $f_s$ ou $T_s$ est un peu superflu.
En effet, avec par exemple un signal $x(t) = \sin(2\pi f t)$.
On a $x[n] = \sin\left(2\pi \frac{f}{f_s} t\right)$.
On voit bien ici qu'un même signal $x[n]$ peut être produit par $(f,f_s) = (2,2)$
ou par $(f,f_s) = (4,4)$ par exemple.
Lorsqu'on calcule la tranformée de Fourier discrète, on veut une transformée de Fourier
indépendante de $f_s$.
Même si, lorsqu'on connait $f_s$, on peut facilement repasser à la transformée de Fourier
classique en remplaçant $\Omega$ par $\omega T_s$.
Pour ceux pour qui ça parait abstrait, voir la \sectionref{fft}.

\subsection{Série de Fourier continue}
\label{sec:sfc}
Si on a un signal périodique de période $T$, on peut faire encore mieux.
Au lieu de prendre toutes les fréquences possibles dans le domaine spectral,
on ne va prendre que celle dont l'exponentielle correspondante est
aussi périodique de période $T$.

Cela peut se justifier de la même manière que pour
la transformée de Fourier discrète.
En effet, on a vu à la \sectionref{fourier_discret}
que la transformée de Fourier d'un signal discret
donne un signal périodique.
On peut donc s'attendre à ce que la transformée de Fourier inverse d'un
signal discret soit un signal périodique.
Tout cela justifie à nouveau le tableau~\ref{tab:recap_fourier}.

Pour cela, rappelons nous que $e^{jk\omega_0t}$ est un signal périodique
de période $\frac{2\pi}{\omega_0}$.
En posant $\omega_0 = \frac{2\pi}{T}$, on a donc
\begin{align}
  \label{eq:sfc}
  X[k] & = \frac{1}{T}\int_0^T x(t) e^{-jk\omega_0t} \dif t\\
  \label{eq:sfci}
  x(t) & = \sum_{k=-\infty}^\infty X[k]e^{jk\omega_0t}.
\end{align}

On remarque que la série de fourier n'est pas un échantillonage de la transformée de fourier.
C'est dû au fait qu'ici on divise par $T$.
Si on tend $T$ vers l'infini, on voit qu'on a la transformée de fourier divisée par $T$.
On a alors l'infini sur l'infini.
Ça s'interprète assez facilement avec un exemple.
Si on prend $x(t) = \exp(j\pi t)$, on a $T = 2$ et donc $\omega_0 = \pi$,
on voit par \eqref{eq:tfci} que $X(j\omega) = 2\pi\delta(\omega - \pi)$ alors qu'on voit
avec \eqref{eq:sfci} que $X[k] = \delta[k - 1]$.
$X(j\pi)$ est infini alors que $X[1]$ vaut juste 1.

On voit aussi que l'avantage de diviser par la longueur de l'intégrale c'est
que $X[k]$ ne change pas si on intègre sur n'importe quel multiple de la période.
On pourrait par exemple intégrer de $0$ à $2T$ et diviser par $2T$, on aurait le même résultat.

\subsubsection{Série de Fourier d'une répétition de deltas}
\label{sec:deltas}
Soit un signal
\[ x(t) = \sum_{n=-\infty}^\infty \delta(t-nT), \]
sa série de Fourier est
\footnote{On intègre de $-T/2$ à $T/2$ car si on intègrait
de $0$ à $T$ ce ne saurait pas clairement si on prend $\delta(t)$ et
$\delta(t-T)$.
De toute façon, c'est périodique donc l'intégrale est la même.}
, par \eqref{eq:sfc},
\begin{align*}
  X[k] & = \frac{1}{T}\int_{-T/2}^{T/2} \sum_{n=-\infty}^\infty \delta(t-nT)
  e^{-2\pi jkt/T} \dif t\\
  & = \frac{1}{T}\int_{-T/2}^{T/2} \delta(t) e^{-2\pi jkt/T} \dif t\\
  & = \frac{1}{T} e^{-2\pi jk0/T}\\
  & = \frac{1}{T}.
\end{align*}

Du coup, par \eqref{eq:sfci},
\[ x(t) = \frac{1}{T}\sum_{k=-\infty}^\infty e^{2\pi jkt/T} \]
ce qui nous permet,
par la linéarité de la transformée de Fourier et en nous aidant du
tableau~\ref{tab:fourier_ex} et de la propriété de décalage en fréquentiel,
d'obtenir le résultat assez intéressant
qui nous sera utile dans la \sectionref{sampling}
\[ X(j\omega) = \frac{2\pi}{T}
\sum_{k=-\infty}^\infty \delta(\omega-2\pi k/T). \]

\subsection{Série de Fourier discrète}
\label{sec:sfd}
Un signal discret et périodique a donc,
selon le tableau~\ref{tab:recap_fourier}, une transformée
périodique car il est discret et discrète car il est périodique.

En considérant que $x[n]$ est de période $N$.
On combine les résultats de la transformée de Fourier discrète
et de la série de Fourier continue.
\begin{description}
  \item[transformée de Fourier discrète]
    La transformée de Fourier est périodique de période $\frac{2\pi}{T_s}$
    lorsque $x$ est discet avec une période de discrétisation $T_s$.
  \item[série de Fourier continue]
    La transformée de Fourier est discrète de période de discrétisation
    $\frac{2\pi}{T}$ lorsque $x$ est périodique de période $T$.
\end{description}
La transformée de Fourier est donc ici discrète et périodique.
$x$ est discret et de période $N$, sa période en continue est donc $NT_s$
car $T_s$ est la distance entre deux valeurs discrète.

La transformée de Fourier a donc une période de discrétisation de $\omega_0 = \frac{2\pi}{NT_s}$
et une période de $\frac{2\pi}{T_s}$.
En prenant $\Omega = \omega T_s$ pour la même raison que pour la transformée de Fourier discrète,
on a une signale avec une période de discrétisation de $\Omega_0 = \frac{2\pi}{N}$ et une période
de $2 \pi$.
La période de la transformée de Fourier en discret est donc $N$, comme celle du signal de départ!
C'est pour ça que lorsqu'on applique FFT sur un vecteur de longueur $N$ on reçoit un vecteur de longueur
$N$ également (voir \sectionref{fft}).
On a donc
\begin{align}
  \label{eq:sfd}
  X[k] & = \frac{1}{N} \sum_{n=0}^{N-1}x[n]e^{-jk\Omega_0n}\\
  \nonumber
  x[n] & = \sum_{k=0}^{N-1}X[k]e^{jk\Omega_0n}.
\end{align}

Il est amusant de s'imaginer la série de Fourier discrète
comme une interpolation des $N$ points d'une période par
une combinaison linéaire des $N$ fonctions $\phi_k(t) = e^{jk\Omega_0n}$.
Les coefficients de cette combinaison linéaire sont alors les $X[k]$.

\begin{table}
  \centering
  \begin{tabular}{|l|l|}
    \hline
    \textbf{Temporel} & \textbf{Fréquentiel}\\
    \hline
    Continu & Non-périodique\\
    \hline
    Discret & Périodique\\
    \hline
    Non-périodique & Continue\\
    \hline
    Périodique & Discret\\
    \hline
  \end{tabular}
  \caption{Tableau récapitulatif de la transformée de Fourier}
  \label{tab:recap_fourier}
\end{table}

\subsection{Fast Fourier Transform (FFT)}
\label{sec:fft}
Le but de l'algorithme FFT est de calculer une transformée de Fourier de façon numérique.
Le problème avec le numérique c'est qu'on ne sais pas s'occuper d'une infinité de points.
On doit donc travailler sur un interval et avec une version discrétisée de la fonction.
On effectue alors une série de Fourier discrète.
Malheureusement, la série de Fourier discrète suppose que la fonction soit périodique.
FFT ne va donc pas calculer la transformée de Fourier de la fonction mais la transformée
de Fourier de la fonction obtenue en répétant l'interval et l'image de la fonction sur cet interval.

L'algorithme prend donc en entrée les $N$ valeurs $x[0], \ldots, x[N-1]$ et calcule
$X[0], \ldots, X[N-1]$.
tels que
\[ X[k] = \sum_{n=0}^{N-1}x[n]e^{-jk\Omega_0n} \]
ou encore
\[ X[k] = \sum_{n=0}^{N-1}x[n]e^{-j\frac{2\pi}{N} kn}. \]
Pour obtenir la série de Fourier discrète,
il faut encore diviser ce résultat par $N$
comme on le voit avec l'équation~\eqref{eq:sfd}.

Comme on l'a vu à la \sectionref{sfc},
il ne faut pas interpréter les $X[k]$ comme la valeur de
la transformée de Fourier en $k\omega_0$ mais plutôt la présence
d'une composante $\exp(jk\omega_0 t)$ d'amplitude $X[k]$
dans $x(t)$.
Dans la transformée de Fourier, comme on l'a vu
à l'exemple de la \sectionref{sfc},
on a plutôt $2\pi X[k] \delta(\omega - k\omega_0)$.
Par exemple, pour $\cos(4\pi)$ on peut s'attendre à
une série de Fourier continue $\frac{\delta[k-1]+\delta[k+1]}{2}$
pour $T = 1/2$ ($\omega_0 = \frac{2\pi}{T} = 4\pi$)
et $\frac{\delta[k-2]+\delta[k+2]}{2}$
pour $T = 1$ ($\omega_0 = \frac{2\pi}{T} = 2\pi$).

Il faut bien comprendre également que l'algorithme FFT
est indépendant de $T_s$ et donc de $T = nT_s$.
On lui donne un vecteur de $N$ élément, il nous renvoit $N$
élément.
Il peut trouver $\Omega_0 = \frac{2\pi}{N}$ et utiliser
l'équation~\eqref{eq:sfd} sans se soucier de $T_s$.
On voit à la \sectionref{sfd} où le $T_s$ se simplifie.
À nous donc de nous souvenir quel $T_s$ on a utilisé.
On sait alors que $\omega_0 = \Omega_0 f_s$ et donc on peut
calculer
$X(j\omega) = 2\pi \sum_{k=0}^{N-1} X[k] \delta(\omega - k\omega_0)$.

On voit qu'une manière simple de calculer la FFT est directement
avec la formule.
C'est un algorithme en $\bigoh(N^2)$.
Les algorithmes utilisés en pratiques sont plus rapides ($\bigoh(N\log(N))$)
et même plus précis.
Voici un exemple d'utilisation de \lstinline|fft| avec \matlab{},
avec un cosinus de fréquence $2$
\begin{lstlisting}
N = 10;
t = (0:N-1) / N;
u = cos(2*pi*2*t);
U = fft(u) / N;
f = 0:n-1;
stem(f, U);
\end{lstlisting}
La ligne la moins évidente est la ligne 5 et la division par $N$.
Essayez de la comprendre avec les formules de $\omega_0$, etc, vues plus haut.
Pour avoir les fréquences de $-5$ à $4$ et non de $0$ à $9$,
il faut remplacer les lignes 4 et 5 par
\begin{lstlisting}
U = fftshift(fft(u)) / N;
f = (0:n-1) - n/2;
\end{lstlisting}

\subsection{Propriété de la transformée de Fourier}
La transformée de Fourier a pas mal de propriétés intéressantes.
Pour commencer, elle est linéaire, c'est à dire que
$\fourier\{ax + by\} = a\fourier\{x\} + b\fourier\{y\}$.
À ne pas confondre avec le changement d'échelle
\[ \fourier\{x(at)\} = \frac{1}{|a|}X(j\omega/a). \]

De plus,
\begin{itemize}
  \item Si un signal est réel,
    la partie réelle de sa transformée de Fourier est paire et
    la partie imaginaire est impaire;
  \item Si un signal est pair, sa transformée de Fourier est réelle et paire;
  \item Si un signal est impair,
    sa transformée de Fourier est imaginaire pure et impaire.
\end{itemize}

Pour des signaux \emph{non-périodiques} $x$ et $y$,
\begin{align}
  \nonumber
  \fourier\{x * y\} & = X Y\\
  \label{eq:prod2conv}
  \fourier\{xy\} & = \frac{1}{2\pi}X * Y
\end{align}
et pour des signaux \emph{périodiques} $x$ et $y$
où $X$ et $Y$ sont donc leur série de Fourier,
\begin{align*}
  %\fourier\{x * y\} & = X Y\\ % TODO is it true ?
  \fourier\{xy\} & = X * Y.
\end{align*}
La disparition du facteur $\frac{1}{2\pi}$ et du au facteur
$\frac{1}{T}$ de l'équation~\eqref{eq:sfc}.

Une propriété très intéressante pour les équations différentielles est que
\begin{align*}
  \fourier\left\{\fdif{x(t)}{t}\right\} & = j\omega X(j\omega)\\
  \fourier^{-1}\left\{\fdif{X(j\omega)}{\omega}\right\} & =
  -jtx(t).
\end{align*}

On a aussi un propriété de décalage en continu
\begin{align*}
  \fourier\{x(t-t_0)\} & = e^{-j\omega t_0}X(j\omega)\\
  \fourier^{-1}\{X(j(\omega-\nu))\} & = e^{j\nu t}x(t)
\end{align*}
et en discret
\begin{align*}
  \fourier\{x[n-k]\} & = e^{-j\Omega k} X(e^{j\Omega})\\
  \fourier^{-1}\{X(e^{j(\Omega-\nu)})\} & = e^{j\nu n}x[n]
\end{align*}
qui va nous être utile dans la résolution d'équations aux différences.

Une propriété intéressante pour trouver $U(j\omega)$ donné dans le
tableau~\ref{tab:fourier_ex} à l'aide de \eqref{eq:uintdelta} est que si
$y(t) = \int_{-\infty}^t x(\tau) \dif \tau$, alors
\[ Y(j\omega) = \frac{1}{j\omega}X(j\omega) + \pi X(j0)\delta(\omega). \]

La relation de Parseval permet de passer d'une certaine intégrale en temporel
à une intégrale en fréquentiel
\[ \int_{-\infty}^\infty |x(t)|^2 \dif t
= \frac{1}{2\pi} \int_{-\infty}^\infty |X(j\omega)|^2 \dif \omega. \]

\begin{table}
  \[
    \begin{array}{|c|c|l|}
      \hline
      x(t) & X(j\omega) & \text{condition (convergence)}\\
      \hline
      e^{-at}u(t) & \frac{1}{a + j\omega} & \Re(a) > 0\\
      \frac{\sin(Wt)}{\pi t} &
      \begin{cases}
        1, \text{si } |\omega| \leq W\\
        0, \text{sinon}
      \end{cases}
      &\\
      \frac{1}{2\pi} & \delta(\omega) &\\
      u(t) & \frac{1}{j\omega} + \pi \delta(\omega)&\\
      \hline
      x[n] & X(e^{j\Omega}) &\\
      \hline
      \alpha^n u[n] & \frac{1}{1-\alpha e^{-j\Omega}} & |\alpha| < 1\\
      \frac{\sin(Wn)}{\pi n} &
      \begin{cases}
        1, \text{si } |\Omega| \leq W\\
        0, \text{sinon}
      \end{cases}
      &\\
      \hline
    \end{array}
  \]
  \caption{Exemples de transformées de Fourier}
  \label{tab:fourier_ex}
\end{table}

\subsection{Échantillonnage}
\label{sec:sampling}
\subsubsection{Échantillonnage en temporel}
Lorsqu'on échantillonne un signal continu $x(t)$ en $x_s(t)$,
comme on a vu à la \sectionref{fourier_discret},
son spectre devient périodique.

En effet, échantillonner,
revient à multiplier en temporel par
\[ \sum_{n=-\infty}^{\infty} \delta(t-nT_s). \]
C'est à dire, comme on a vu dans la \sectionref{deltas} et
en utilisant \eqref{eq:prod2conv},
faire une convolution en fréquentiel par
\[ \frac{1}{2\pi} \times
\frac{2\pi}{T_s}\sum_{k=-\infty}^\infty\delta(\omega-2\pi k/T_s)
= \frac{1}{T_s}\sum_{k=-\infty}^\infty\delta(\omega-2\pi k/T_s). \]
On va donc prendre le spectre de $x(t)$ et on va le reproduire tous les
$2\pi/T_s$ en le divisant par $T_s$.
On remarque donc que si $X(j\omega)$ n'est différent de 0 que pour
$|\omega| \leq \omega_\mathrm{max}$,
le spectre a une largeur de $2\omega_\mathrm{max}$,
il faut donc que $\frac{2\pi}{T_s} > 2\omega_\mathrm{max}$ pour que les
reproductions de signal ne se chevauchent pas.
Comme $\omega_s = \frac{2\pi}{T_s}$, on a en fait $f_s > 2f_\mathrm{max}$.
C'est le \emph{théorème de Shannon}.

Pour reconstituer le signal de départ après échantillonnage,
il suffit de se débarrasser de ces reproductions en fréquentiel
et de remultiplier le signal par $T_s$.
Pour cela, si on a fait bien attention à
ce qu'il n'y ait pas de chevauchements,
il suffit de multiplier le signal en fréquentiel par une fenêtre
qui vaut $T_s$ pour $-\pi/T_s \leq \omega < \pi/T_s$ et 0 sinon;
ce qui fait une convolution par $\sin(\pi t/T_s)/(\pi t/T_s)$ en temporel.
On appelle ça un \emph{filtre passe-bas}.

Il peut arriver que bien que les conditions du théorème de Shannon n'aient
pas été respectées, on sache quand même récupérer le signal de départ.
Il faut pour cela qu'il ne se soit pas chevauché avec ses reproductions.
C'est le cas quand les fréquences sont localisés par exemple entre
$4\pi \leq \omega < 6\pi$.
On peut alors récupérer le signal de départ en multipliant
en fréquentiel par une fenêtre qui faut $T_s$ pour
$4\pi \leq \omega < 6\pi$ et 0 ailleurs.
C'est un \emph{filtre passe-bande}.

\paragraph{Périodicité temporelle du signal échantillonné}
Même si $x(t)$ est périodique,
$x_s(t)$ n'est pas nécessairement périodique.
Pour que $x_s(t)$ soit périodique, il faut et il suffit que
$T_s$ soit un multiple rationnel de la période de $x(t)$.

\subsubsection{TFD - Transformée de Fourier discrète}
Faire une transformée de Fourier discrète consiste simplement à faire une
transformée de Fourier normale et puis d'échantillonner en fréquentiel.
Ce qui, du coup, reproduit périodiquement le signal en temporel.

Cela fait fort penser à la série de Fourier, en effet,
si on échantillonne \eqref{eq:tfd} tous les $2\pi/N$, on arrive
à la même chose que \eqref{eq:sfd} modulo un facteur $1/N$.
En effet, on a
\[ X[k] = \frac{X(e^{ik2\pi/N})}{N}. \]

\section{Transformée de Laplace}
La transformée de Laplace est une généralisation de la transformée de Fourier
continue.

En effet, au lieu de se limiter à $j\omega$, on prend $s = \sigma + j\omega$.
Les formules deviennent
\begin{align}
  \label{eq:tl}
  X(s) & = \int_{-\infty}^\infty x(t) e^{-st} \dif t\\
  \nonumber
  x(t) & = \frac{1}{2\pi}\int_{-\infty}^\infty X(\sigma+j\omega)
  e^{(\sigma+j\omega)t} \dif \omega.
\end{align}

En comparant l'équation~\ref{eq:tl} à l'équation~\ref{eq:tfc}, on remarque que
$\laplace\{x(t)\} = \fourier\{x(t) e^{-\sigma t}\}$.
D'où la condition de convergence.
\[ \int_{-\infty}^\infty |x(t)|e^{-\sigma t} < \infty. \]
Contre toute attente, on a donc une convergence dépendante de $\sigma$, c'est
à dire la partie réelle de $s$.
L'avantage, c'est qu'on va trouver des transformées de Laplace pour des
signaux n'ayant pas de transformée de Fourier.
Par exemple, bien que $e^tu(t)$ n'a pas de transformée de Fourier,
on va pouvoir lui trouver une transformée de Laplace.
Le désavantage, c'est qu'elle aura une ``Region Of Convergence'' ($\roc$).

\subsection{Propriétés de la transformée de Laplace}
Comme pour Fourier, on a la linéarité
\[ \laplace\{ax(t)+by(t)\} = aX(s) + bY(s) \]
où $\roc_1 \cap \roc_2 \subseteq \roc$,
le décalage temporel
\[ \laplace\{x(t-t_0)\} = e^{-st_0}X(s) \]
où le $\roc$ ne change pas,
le décalage fréquentiel
\[ \laplace^{-1}\{X(s-s_0)\} = e^{s_0t}x(t) \]
où le $\roc$ devient $\roc + \Re(s_0)$,
la convolution
\[ \laplace\{x*y\} = XY, \]
la différentiation
\[ \laplace\left\{\fdif{x}{t}\right\} = sX(s), \]
l'intégration
\[ \laplace\left\{\int_{-\infty}^tx(\tau)\dif\tau\right\} = \frac{X(s)}{s} \]
et la différentiation en fréquence
\[ \laplace^{-1}\left\{\fdif{X(s)}{s}\right\} = -tx(t). \]

\subsection{Transformée de Laplace unilatérale}
\label{sec:laplacu}
La propriété sur la différentiabilité nous permet de résoudre les équations
différentielles aux données initiales nulles mais pour les autres,
il nous faudra définir une transformée de Laplace unilatérale
\begin{align*}
  X_u(s) & = \int_{0^-}^\infty x(t) e^{-st} \dif t.
\end{align*}

Pour cette transformée, on a
\begin{align*}
  \laplacu\left\{\fdif{x(t)}{t}\right\} & =
  sX_u(s) - x(0^-)\\
  \laplacu\left\{\ffdif{x(t)}{t}\right\} & =
  s^2X_u(s) - sx(0^-) - x'(0^-)
\end{align*}
et pour des systèmes stables et causaux,
\begin{align*}
  \lim_{s\to\infty} sX_u(s) & = x(0^+)\\
  \lim_{s\to 0} sX_u(s) & = x(\infty).
\end{align*}

\section{Transformée en $z$}
La transformée en $z$ est une généralisation de la transformée de Fourier
discrète.

La transformée en $z$ d'un signal discret est
\begin{align*}
  X(z) & = \sum_{k=-\infty}^{\infty} x[k]z^{-k}\\
  x[n] & =
  \frac{1}{2\pi}\int_{-\pi}^{\pi} X(re^{j\omega})(re^{j\omega})^n\dif \omega\\
  & = \frac{1}{2\pi j}\oint X(z) z^{n-1} \dif z.
\end{align*}
L'égalité entre les deux dernières intégrales
s'obtient directement avec le changement
de variable $z = re^{j\omega}$ d'où $\dif z = zj\dif \omega$.

Comme $\transfz(x[n]) = \fourier(x[n]r^n)$,
une condition suffisante pour qu'une transformée de Laplace converge est
\[ \sum_{n=-\infty}^\infty|x[n]r^{-n}| < \infty. \]
La convergence dépend donc de $r$.

\subsection{Propriétés de la transformée en $z$}
Comme pour Laplace, on a la linéarité
\[ \transfz\{ax[n]+by[n]\} = aX(z) + bY(z) \]
où $\roc_1 \cap \roc_2 \subseteq \roc$,
le décalage temporel
\[ \transfz\{x[n-n_0]\} = z^{-n_0}X(z) \]
où le $\roc$ ne change pas,
le changement d'échelle en fréquentiel
\[ \transfz^{-1}\{X(z/z_0)\} = z_0^nx[n] \]
où le $\roc$ devient $\roc |z_0|$,
la convolution
\[ \transfz\{x*y\} = XY, \]
et la différentiation en fréquence
\[ \transfz^{-1}\left\{z\fdif{X(z)}{z}\right\} = -nx[n]. \]

\subsection{Transformée en $z$ unilatérale}
Comme pour la \sectionref{laplacu}, on a besoin
d'une transformée en $z$ unilatérale pour résoudre les équations
différentielles à conditions initiales non-nulles
\begin{align*}
  X_u(z) & = \sum_{n=0}^\infty x[n] z^{-n}.
\end{align*}

Pour cette transformée, on a
\begin{align*}
  \transzu\{x[n-1]\} & =
  z^{-1}X_u(z) + x[-1]\\
  \transzu\{x[n-2]\} & =
  z^{-2}X_u(z) + z^{-1}x[-1] + x[-2]\\
  \transzu(x[n-k]) & =
  z^{-k}X_u(z) + z^{-k+1}x[-1] + \ldots + z^{-1}x[-k+1] + x[-k]
\end{align*}
et pour des systèmes stables et causaux,
\begin{align*}
  \lim_{|z|\to\infty} X_u(z) & = x[0^+]\\
  \lim_{z\to 1} (z-1)X_u(z) & = x[\infty].
\end{align*}

Attention le principe est le même qu'avec Laplace mais pas le signe.


% on est pas interrogé sur le développement en série de puissances

%  ____            _
% / ___| _   _ ___| |_ ___ _ __ ___  ___
% \___ \| | | / __| __/ _ \ '_ ` _ \/ __|
%  ___) | |_| \__ \ ||  __/ | | | | \__ \
% |____/ \__, |___/\__\___|_| |_| |_|___/
%        |___/

\part{Systèmes}
Un système prend un signal en entrée $x(t)$ et retourne un signal $y(t)$
en sortie.
On note $H\{x(t)\} = y(t)$.

\section{Système LTI}
Un système LTI est un système Linéaire à Temps Invariant.

\subsection{Invariance temporelle}
Il est à temps invariant.
Ça veut dire que si $H\{x(t)\} = y(t)$, $H\{x(t-t_0)\} = y(t-t_0)$.

\subsection{Linéarité}
Un système peut être linéaire, c'est à dire que
\[ H\{\sum a_ix_i(t)\} = \sum a_iH\{x_i(t)\} \]

\subsection{Réponse impulsionnelle}
On peut représenter un système LTI à l'aide de sa réponse impulsionnelle
définie par $h(t) = H\{\delta(t)\}$. On a
\[ y(t) = x(t) * h(t). \]

\subsection{Réponse indicielle}
La réponse indicielle d'un système LTI notée $s(t)$ est par définition
\begin{align*}
  s(t) & \eqdef h(t) * u(t)\\
  & = \int_{-\infty}^t h(\tau) \dif \tau &
  s[n] & = \sum_{-\infty}^n h[k]
\end{align*}
d'où
\begin{align*}
  h(t) & = \fdif{}{t}s(t) & h[n] & = s[n] - s[n-1].
\end{align*}

On peut aussi calculer la réponse indicielle à l'aide de $H(s)$
ou $H(z)$.
C'est à dire calculer
\begin{align*}
  S(s) & = H(s)U(s) & S(z) & = H(z)U(z).
\end{align*}
et puis faire la transformée inverse pour trouver $s(t)$ ou $s[n]$.

%Parallèle : $+$
%Série : $*$ commutatif car LTI

\subsection{Système LTI sans mémoire}
Un système LTI peut être sans mémoire,
cela signifie simplement que $y(t)$ ne dépend que de $x(t)$.
Comme le système est temporellement invariant,
il fait la même chose pour tous les $x(t)$.
Du coup, $\exists c$ tel que $y(t) = cx(t)$, d'où
\begin{align*}
  h[n] & = c \delta[n] & h(t) = c \delta(t).
\end{align*}

\subsection{Système LTI causal}
Un système est causal si la réponse impulsionnelle ne dépend
que des valeurs présentes et passées, mais pas des valeurs futures.
Cela revient à dire que
\begin{align*}
  h[n] & = 0 \, \forall n < 0  & h(t) = 0 \, \forall t < 0.
\end{align*}
En effet, comme
\[ y[n] = \sum_{-\infty}^{+\infty}x[k]h[n-k] \]
pour que le signal ne soit pas influencé par les $x[k]$ avec
$k > n$, il faut que le $h[n-k]$ correspondant soit 0.
Comme $k > n$, $n - k < 0$ il faut donc bien $h[k] = 0$ $\forall k < 0$.

\subsection{Système LTI inversible}
Pour qu'un sysème LTI soit inversible, il faut qu'il existe $h^{-1}$ tel que
$h^{-1}*h*x = x$.
Comme vu précédemment, le neutre de la convolution est $\delta$ il faut donc
qu'il existe $h^{-1}$ tel que $h^{-1} * h = \delta$.

%\subsection{Équations différentielles}
%Time Invariant car coefficients indépendants du temps.
%Linear car $y$ et $y'$ etc. apparaissent de façon linéaire

\section{Fonction de transfert}
\subsection{Transformée en $z$ dans un système LTI}
Supposons qu'on doive calculer l'image d'un signal discret par un système LTI.
On peut éviter de calculer le produit de convolution grâce à la fonction
de transfert
\[ H(z) = \sum_{k=-\infty}^{\infty} h[k] z^{-k}. \]
On remarque d'ailleurs que $H(z)$ est la transformée en $z$ de $h$.

En effet, on remarque que l'image de $z^n$ est
\begin{align*}
  \sum_{k=-\infty}^{\infty} h(k) z^{n-k} &
  = z^{n} \sum_{k=-\infty}^{\infty} h[k] z^{-k}\\
  & = z^n H(z)
\end{align*}

Dès lors, lorsqu'on veut calculer l'image d'un signal $x[n]$. Il suffit de
prendre sa transformée de Laplace $X(z)$ dont l'image est $Y(z) = X(z) H(z)$
et puis d'appliquer la transformée en $z$ inverse sur $Y(z)$ pour trouver
$y[n]$.

C'est pareil pour les signaux continus avec Laplace.

\subsection{Diagramme de Bode}
Une aide peut être trouvé dans~\cite[pp.~251-253]{astrom2010feedback} ou dans~\cite{cheever2013bode} qui contient de bons exemples.
Soit une fonction de transfert $H(s)$.
Le diagramme de Bode de $H(s)$ est le graphe de la fonction $20\log|H(s)|$.
Une approximation de ce diagramme peut être très facilement dessiné à la main
à l'aide de l'équation de $H(s)$ exprimé sous la forme
\[ H(s) = K \frac{\sum_{i=0}^mb_is^i}{\sum_{i=0}^na_is^i}. \]

Il nous suffit de factoriser le numérateur et le dénominateur
jusqu'à avoir 4 types de facteurs.
\begin{itemize}
  \item Une racine nulle
    \[ s; \]
  \item Une racine simple réelle
    \[ \frac{s}{\omega_0} + 1; \]
  \item Deux racines conjuguées,
    \[ \left(\frac{s}{\omega_0}\right)^2 +
    2\zeta\frac{s}{\omega_0} + 1; \]
  \item Une racine double réelle
    \[ \left(\frac{s}{\omega_0} + 1\right)^2. \]
\end{itemize}
Il est important d'avoir à chaque fois la bonne forme pour les facteurs
car ça modifie $K$.
Par exemple, pour la racine simple réelle, il ne faut pas garder
$(s + \omega_0)$ mais le transformer en $(s/\omega_0 + 1)$.

En suite, comme on trace le graphe d'un logarithme,
le produit de ses facteurs va devenir leur somme.
En effet, par exemple,
\[ 20\log\left|2\frac{s}{s + 1}\right| =
20\log 2 + 20\log|s| - 20\log|s+1|. \]

Il nous suffit donc de prendre le graphe de chacun de nos constituants
et de les additionner.
Les graphes seront en dB par decade, les graphes de chacun de nos
composants sont donnés par le tableau~\ref{tab:bode}.
On remarque que le graphe d'un des éléments au dénominateur est simplement
l'opposé de celui au numérateur.
\begin{table}
  \centering
  \begin{tabular}{|l|l|}
    \hline
    \emph{Type} & \emph{Graphe}\\
    \hline
    Contante $K$ & $20\log|K|$\\
    \hline
    Zéro nul & $\si{20}{dB\per decade}$ passant par $\si{0}{dB}$ pour
    $\omega=1$\\
    \hline
    Pôle nul & $\si{-20}{dB\per decade}$ passant par $\si{0}{dB}$ pour
    $\omega=1$\\
    \hline
    Zéro réel & $\si{20}{dB\per decade}$ à partir de $\omega = \omega_0$\\
    \hline
    Pôle réel & $\si{-20}{dB\per decade}$ à partir de $\omega = \omega_0$\\
    \hline
    Zéros conjugués &
    $\si{40}{dB\per decade}$ à partir de $\omega = \omega_0$.\\
    & Il faut aussi rajouter un pic à $\omega_0$ d'amplitude
    $20\log(2\zeta)$.\\
    \hline
    Pôles conjugués &
    $\si{-40}{dB\per decade}$ à partir de $\omega = \omega_0$.\\
    &
    Il faut aussi rajouter un pic à $\omega_0$ d'amplitude
    $-20\log(2\zeta)$.\\
    \hline
    Double zéro réel &
    $\si{40}{dB\per decade}$ à partir de $\omega = \omega_0$\\
    \hline
    Double pôle réel &
    $\si{-40}{dB\per decade}$ à partir de $\omega = \omega_0$\\
    \hline
  \end{tabular}
  \caption{Diagramme de bode pour les différents facteurs de $H(s)$.
  ``À partir de'' signifie que c'est \si{0}{dB} avant et puis
  que ça part de $\si{0}{dB}$.}
  \label{tab:bode}
\end{table}

\begin{myexem}
  \label{ex:bode}
  Par exemple, si
  \[ H(s) = -100\frac{s}{s^3+12s^2+21s+10} =
  -10\frac{s}{(s+1)^2\left(\frac{s}{10}+1\right)}, \]
  on a
  \begin{itemize}
    \item une constante de $-10$, ce qui fait
      $\si{20}{dB}$;
    \item un zéro nul ce qui fait une droit qui passe par $(0,1)$
      avec une pente de $\si{20}{dB\per decade}$;
    \item un pôle réel à $s = -10$ ce qui fait
      $\si{-20}{dB\per decade}$ à partir de $\omega = 10$;
    \item un double pôle réel à $s = -1$ ce qui fait
      $\si{-40}{dB\per decade}$ à partir de $\omega = 1$.
  \end{itemize}
  %Le graphe obtenu est montré par la figure~\ref{fig:bodeexem}
  %\begin{figure}
    %\centering
    %\begin{tikzpicture}
      %\begin{semilogxaxis}[xmin=1e-2,xmax=1e3,xlabel=$\omega$, ylabel=dB,
        %ymin=-140,ymax=100,smooth]
        %\addplot {20};
        %\addlegendentry{$d = R$}
        %\addplot {20*log10(x)};
        %\addlegendentry{$d = 2R$}
      %\end{semilogxaxis}
    %\end{tikzpicture}
    %\caption{Diagramme de bode de $H(s)$ dans l'exemple~\ref{ex:bode}}
    %\label{fig:deqr}
  %\end{figure}
\end{myexem}

\section{Schéma bloc et représentation d'état}
Le schéma bloc et la représentation d'état constituent la représentation
interne d'un système, il y en a en général plusieurs possibles pour une
même représentation entrée-sortie.

Un schéma bloc est difficile à appréhender si on essaie d'y appliquer
notre logique des circuits électriques.
En effet, oubliez la loi de Kirchhoff sur les courants
$I_1 + I_2 + \ldots + I_n = 0$, dans un schéma bloc, c'est plutôt
$I_1 = I_2 = \ldots = I_n = 0$.

Un schéma bloc est constitué d'une entrée $x$, d'une sortie $y$ et
d'états internes $q_i$.
Il est parsemé de blocs $\int$ en continu (resp. $D$ en discret) tels que
la sortie du bloc est l'intégrale (resp. le décalage $D\{x[n]\} = x[n-1]$) du
signal d'entrée.
On met généralement les $q_i$ à la sortie des blocs.

On peut décrire le schéma bloc à l'aide de sa représentation d'état qui
en continu est de la forme
\begin{align*}
  q'(t) & = Aq(t) + Bx(t)\\
  y(t) & = Cq(t) + Dx(t)
\end{align*}
et en discret est de la forme
\begin{align*}
  q[n+1] & = Aq[n] + Bx[n]\\
  y[n] & = Cq[n] + Dx[n].
\end{align*}

On remarque que lorsqu'on a mis les $q_i$ aux sorties des blocs,
les $q_i[n+1]$ et les $q'(t)$ sont simplement l'entrée des blocs.
La première équation s'écrit donc simplement en regardant ce qui rentre
dans les blocs.
La seconde équation s'écrit en regardant ce qui sort du système en $y$.

\subsection{Changement de représentation interne}
Lorsqu'on change de représentation interne, on peut obtenir
la nouvelle représentation d'état en exprimant le changement de représentation
comme un changement de variable en $q$, $\tilde{q} = Tq$.
On a alors
\begin{align*}
  \tilde{q} & = TAT^{-1}\tilde{q} + TBx\\
  y & = CT^{-1}\tilde{q} + Dx.
\end{align*}
Ce qui n'est sûrement pas à retenir par coeur tellement c'est immédiat.
Il suffit de remplacer $q$ par $T^{-1}\tilde{q}$ dans la représentation d'état
en $q$ puis de multiplier la première équation par $T$.

On remarquera quand même que cela nous permet d'obtenir facilement une
transformation du système qui rendra $\tilde{A}$ diagonale.
En effet, trouver $D$ et $Q$ tels que $A = QDQ^{-1}$ est une opération connue,
c'est la diagonalisation.
Pour avoir $\tilde{A}$ diagonale,
il suffit donc de trouver $Q$ en diagonalisant $A$
et de faire un changement de variable $\tilde{q} = Tq$ avec $T = Q^{-1}$.

\subsection{De représentation d'état à fonction de transfert}
On peut passer de la représentation d'état de temporel à fréquentiel
en gardant les mêmes matrices.

En continu, on a
\begin{align*}
  sQ(s) & = AQ(s) + BX(s)\\
  Y(s) & = CQ(s) + DX(s)
\end{align*}
et on trouve alors
\[ H(s) = \frac{Y(s)}{X(s)} = C(sI-A)^{-1}B + D. \]
En discret, c'est pareil
\begin{align*}
  zQ(z) & = AQ(z) + BX(z)\\
  Y(z) & = CQ(z) + DX(z)
\end{align*}
et on trouve alors
\[ H(z) = \frac{Y(z)}{X(z)} = C(zI-A)^{-1}B + D. \]

Le plus simple pour inverser en pratique est à l'aide de
\[ C(zI-A)^{-1}B = \frac{1}{\det(zI-A)} C\cof(zI-A)^TB. \]
L'astuce est de bien regarder les coefficients de $C$ et de $B$
qui sont nuls pour éviter de calculer trop de cofacteurs.
Le calcul n'est alors pas si fastidieux.

\section{Équations différentielles - équations aux dérivées partielles}
Elle constitue la représentation entrée-sortie d'un système contrairement
au schéma bloc ou à la représentation d'état qui constitue une représentation
interne.

Tout comme dans la résolution standard,
on parle de solution homogène et solution particulière,
on va parler de réponse libre et de réponse forcée.
\begin{description}
  \item[Réponse libre]
    Condition initiales spécifiées et entrée nulle.
    C'est un système linéaire \emph{mais pas à temps invariant}.
    Il faut donc faire attention à ne pas utiliser les transformées
    unilatérales.
  \item[Réponse forcée]
    Condition initiales nulles et entrée spécifiée.
    C'est un système LTI.
\end{description}

\section{Commandabilité et observabilité}
Cette section traitera du cas du système continu mais c'est pareil en discret.
\subsection{Commandabilité}
\begin{description}
  \item[État commandable]
    Un état $q_0$ est commandable si,
    pour un état initial $q(0) = q_0$,
    il existe un $T$ et un $x$ tels qu'on ait $q(T) = 0$.
  \item[État accessible]
    Un état $\bar{q}$ est dit accessible si,
    pour un état initial $q(0) = 0$,
    il existe un $T$ et un $x$ tels qu'on ait $q(T) = \bar{q}$.
  \item[Système complètement commandable]
    Un système est complètement commandable si tous ses états sont
    commandables.
\end{description}

Seulement, l'ensemble des états commandables est l'espace engendré
par les colonnes de la matrice de commandabilité
(voir l'annexe~\ref{ann:command} pour une démonstration)
\[ \begin{pmatrix}B & AB & \cdots & A^{n-1}B\end{pmatrix}. \]
Du coup, un système est complètement commandable si cette matrice est de
plein rang.


\subsection{Observabilité}
\begin{description}
  \item[État non observable]
    Un état $q_0$ est non observable si,
    pour un état initial $q(0) = q_0$,
    pour tout $T > 0$, si $x(t) = 0$ $\forall 0 \leq t \leq T$, alors
    $y(t) = 0$ $\forall 0 \leq t \leq T$.
  \item[Système complètement observable]
    Un système est complètement observable si aucun de ses états est
    non observable.
\end{description}

À nouveau, l'ensemble des états non observables est le $\ker$ de
\[ \begin{pmatrix}C\\CA\\\cdots\\CA^{n-1}\end{pmatrix}. \]
Du coup, un système est complètement observable si cette matrice est de
plein rang.

On remarque que pour une certaine fonction de transfert,
on peut avoir une multitude de schéma blocs ou de représentations d'états
différents.
Mais la fonction de transfert n'utilise
que la partie observable et commandable.
Si un $q_i$ n'est pas observable, ou est non commandable
(c'est à dire, pour la première composante de $q$ par exemple, que l'état
$q = (1, 0, \ldots, 0)$ n'est pas observable ou est non commandable),
il n'apparaitra pas
dans la fonction de transfert.

\section{Stabilité}
\subsection{Stabilité BIBO}
Un système est BIBO stable si,
lorsque $|x[n]| \leq M_x < \infty$ $\forall n$,
on a aussi $|y[n]| \leq M_y < \infty$ $\forall n$ (pareil en continu).
Seulement, comme
\begin{align*}
  |y[n]| & = \left|\sum_{k=-\infty}^{\infty}h[k]x[n-k]\right|\\
         & \leq M_x \sum_{k=-\infty}^{\infty}|h[k]|,
\end{align*}
une condition \emph{suffisante} pour qu'un système soit BIBO stable est que
\begin{align*}
  \sum_{k=-\infty}^\infty |h[k]| & < \infty &
  \int_{t=-\infty}^\infty |h(t)| \dif t & < \infty
\end{align*}
C'est la même condition que celle de l'existence de la transformée de Fourier
de $h$.

Comme la transformée de Fourier
est la transformée en $z$ pour $r = 1$,
si la tranformée en $z$ a le cercle unité dans son ROC, on sait
que le système est BIBO stable.

Comme la transformée de Fourier est la transformée de Laplace pour $\Re(z) = 0$,
si l'axe imaginaire appartient au ROC, on sait
que le système est BIBO stable.

Pour un système \emph{causal},
\begin{itemize}
  \item pour les systèmes continus, il faut que les pôles soient
    strictement à gauche de l'axe imaginaire, $\Re(s) < 0$;
  \item pour les systèmes discrets, il faut soit que les pôles soient
    strictement à l'intérieur du cercle unité, $|z| < 1$.
\end{itemize}
Un système marginalement stable mais pas asymptotiquement stable
qui n'a pas d'annulation pôle-zéro n'est donc pas BIBO stable.

En effet, si le système est causal, il faut que $h[n] = 0$ $\forall n < 0$,
il faut donc du $u[n]$ et pas du $u[-n-1]$, la formule à utiliser donne
alors un ROC de $|z| > |\lambda|$.
S'il existe un $|\lambda| \geq 1$,
on a $|z| > 1$ et le cercle unité n'est pas
dans le ROC.
Par contre si tous les $\lambda < 1$, $|z| = 1$ est dans l'intersection
des ROC $|z| > |\lambda|$.

\subsection{Stabilité interne}
Un point $(\bar{q}, \bar{x})$ est un équilibre si $A\bar{q} + B\bar{x} = 0$.
Cet équilibre est \emph{stable} si un bruit suffisamment faible
va toujours rester dans une marge et est \emph{attractif} si
un bruit suffisamment faible va toujours se tasser à 0.
Un équilibre \emph{stable} et \emph{attractif}
est dit \emph{asymptotiquement stable}.

On remarque que $(0,0)$ est un équilibre et que tout équilibre peut se
ramener à $(0,0)$ par changement de variable.

Du coup, pour qu'un système ait tous ses équilibres stables,
il faut et il suffit que $(0,0)$ soit stable.
On peut donc, au lieu de parler d'équilibre stable,
directement parler de système stable.
Tous les équilibres ont la même nature,
on peut donc parler du système.
Dire qu'un système est stable
signifie que tous les équilibres du système sont stables.

Soient $\lambda_i$ les valeurs propres de $A$.
En continu,
\begin{itemize}
  \item Si $\forall i$ $\Re(\lambda_i) < 0$, le système est asymptotiquement
    stable;
  \item Si $\exists i$ $\Re(\lambda_i) > 0$, le système est instable;
  \item Si $\forall i$ $\Re(\lambda_i) \leq 0$ et que $\forall i$ tel que
    $\Re(\lambda_i) = 0$, $m_g(\lambda_i) = m_a(\lambda_i)$,
    le système est marginalement stable;
  \item Si $\forall i$ $\Re(\lambda_i) \leq 0$ et que $\exists i$ tel que
    $\Re(\lambda_i) = 0$, où $m_g(\lambda_i) < m_a(\lambda_i)$,
    le système est instable.
\end{itemize}

% TUYAU!!!
Et en discret,
\begin{itemize}
  \item Si $\forall i$ $|\lambda_i| < 1$, le système est asymptotiquement
    stable;
  \item Si $\exists i$ $|\lambda_i| > 1$, le système est instable;
  \item Si $\forall i$ $|\lambda_i| \leq 1$ et que $\forall i$ tel que
    $|\lambda_i| = 1$, $m_g(\lambda_i) = m_a(\lambda_i)$,
    le système est marginalement stable;
  \item Si $\forall i$ $|\lambda_i| \leq 1$ et que $\exists i$ tel que
    $|\lambda_i| = 1$, où $m_g(\lambda_i) < m_a(\lambda_i)$,
    le système est instable.
\end{itemize}

\subsection{Stabilité interne et externe}
% TUYAU!!!
Pour la stabilité interne, on regarde tous les états.
Si un des états est instable, le système est instable.

Comme la fonction de transfert ne regarde que
la partie observable et commandable,
on pourrait avoir une fonction de transfert
BIBO stable pour un système qui est
lui instable car dans la relation entrée sortie,
on n'a pas vu la composante instable.
Cela correspond à l'annulation du pôle zéro dans l'expression de $H(s)$.

Si on a la stabilité interne, alors on a la stabilité BIBO.
Car si mon système entier est stable,
alors la partie qui est observable et commandable est stable aussi.

Si on a une stabilité BIBO et une instabilité interne,
alors on a une annulation de zéro,
et donc une perte de commandabilité ou d'observabilité
car si tout était stable et commandable,
on aurait eu une instabilité BIBO aussi.

%Par exemple, pour
%\begin{align*}
  %q_1' & = -2q_2\\
  %q_2' & = q_1 + 2q_2  3q_3 + x\\
  %q_3' & = q_1 + 6q_2 - 7q_3 + x
%\end{align*}
%
%\[ C = \begin{pmatrix}0 & -1 & 2\\1 & -1 & -1\\1 & -1 & -1\end{pmatrix} \]
%base $\mathcal{C}(C)^\perp = (0;1;-1)$
%donc posons $z_1 = q_1$, $z_2 = q_2$ et $z_3 = q_2 - q_3$.

\subsection{Critère de Routh Hurwitz}
Soit une fonction
\[ F(s) = a_ns^n + a_{n-1}s^{n-1} + \ldots + a_1s + a_0. \]
On peut construire le tableau suivant
\begin{align*}
  &a_n && a_{n-2} && a_{n-4} && \cdots\\
  &a_{n-1} && a_{n-3} && a_{n-5} && \cdots\\
  &b_1=\frac{a_{n-1}a_{n-2}-a_na_{n-3}}{a_{n-1}} &&
  b_2=\frac{a_{n-1}a_{n-4}-a_na_{n-5}}{a_{n-1}} && \cdots\\
  &c_1=\frac{b_1a_{n-3}-a_{n-1}b_{2}}{b_1} && \cdots\\
  \vdots
\end{align*}
Ce tableau crée en fait un triangles en ne laissant que des zéros
sous lui ainsi qu'à sa droite.

On sait ensuite que le nombre de racines à partie réelle positive
est le nombre de changements de signe des coefficients de la première
colonne de ce tableau (si ces coefficients sont tous non-nuls).

\subsection{Feedback}
Soit un système LTI avec un gain de fonction de transfert $G(s)$ à la suite
de $H$.
On a du coup $Y = GH(X - Y)$ d'où $(1+GH)Y = GHX$ et donc
\[ T(s) = \frac{GH}{1+GH}. \]

Il faut cependant éviter d'avoir une sensibilité trop forte
\[ \frac{\Delta T/T}{\Delta G/G} =
\frac{G}{T}\frac{H(1 + GH) - GHH}{(1+GH)^2}
= \frac{G(1+GH)}{GH}\frac{H}{(1+GH)^2}
= \frac{1}{1+GH}. \]

Il faut donc un $GH/(1+GH)$ proche de 1 et un $1/(1+GH)$ petit.

En effet, si on a une perturbation $v$ à la sortie de $G$, on a
\[ Y = \frac{GH}{1+GH}X + \frac{1}{1+GH}v \approx X. \]

Il faut aussi faire attention à ne pas déstabiliser le système avec
le feedback.

Par exemple, si
\begin{align*}
  H(s) & = K & G(s) = \frac{1}{1+s\tau}
\end{align*}
où $\tau > 0$, on a
\[ T(s) = \frac{K}{K + 1 + s\tau}. \]
On n'avait pas de problème de stabilité avec $-\frac{1}{\tau}$ car sa partie
réelle était strictement négative.
Mais on a maintenant une racine de $-\frac{1+K}{\tau}$,
Il faut donc que $K \geq -1$ pour rester stable.

%Pour une poly du second degré, tous les coef positif est une ssi de stabilité.
%Pour les + que second, c'est de nécessaire.

\annexe
\section{Produit de convolution}
Le produit de convolution est un produit qui se note $*$ et qui se fait
soit entre deux fonctions discrètes $a[n]$ et $b[n]$
\[ a[n] * b[n] = \sum_{k=-\infty}^{+\infty} a[k]b[n-k], \]
soit entre deux fonctions continues $a(t)$ et $b(t)$
\[ a(t) * b(t) = \int_{\tau=-\infty}^{+\infty} a(\tau)b(t-\tau) \dif \tau. \]

\subsection{Propriétés}
Le produit de convolution a beaucoup de propriétés en commun avec le produit
normal~: il est associatif, commutatif et distributif avec l'addition.

L'absorbant est aussi 0. Cependant le neutre n'est pas 1 mais bien
$\delta(t)$ (resp. $\delta[n]$) en continu (resp. en discret).
Plus généralement, le $\delta$ permet de décaler un signal
\[ x(t) * \delta(t - t_0) = x(t - t_0). \]

\section{Échelon et Impulsion}
\subsection{Échelon}
Un échelon se note $u(t)$ et est défini par
\[ u(t) = \begin{cases}
    0 & t < 0\\
    1 & t > 0
\end{cases} \]
en continu et
\[ u[n] = \begin{cases}
    0 & n < 0\\
    1 & n \geq 0
\end{cases} \]
en discret.

\subsection{Impulsion}
\label{app:dirac}
Une impulsion de Dirac est une fonction avec singularité en 0.
Elle se note $\delta(t)$ et est définie par
\begin{equation}
  \label{eq:dirac1}
  \delta(t) =
  \begin{cases}
    \infty, & t = 0\\
    0, & t \neq 0
  \end{cases}
\end{equation}
et
\begin{equation}
  \label{eq:dirac2}
  \int_{-\infty}^{\infty} \delta(t) \dif t = 1.
\end{equation}

Il peut paraître étrange que l'intégrale soit finie alors que
$\delta(0) = \infty$.
Seulement, rappelez-vous qu'une intégrale est une aire et que
l'aire d'un Dirac n'est pas entièrement définie par \eqref{eq:dirac1}
car celle-ci impose que l'aire vaut $0 \cdot \infty$, qui est un cas
d'indétermination.
\eqref{eq:dirac2} permet de lever cette indétermination.

\subsection{Relations intéressantes}
\begin{align}
  \nonumber
  \delta(t) & = \fdif{u(t)}{t}\\
  \label{eq:uintdelta}
  u(t) & = \int_{-\infty}^t \delta(t) \dif t\\
  \nonumber
  f(t_0) & = \int_{-\infty}^\infty f(t)\delta(t-t_0) \dif t\\
  \label{eq:infinf00}
  f(t) \delta(t-t_0) & = f(t_0)\delta(t-t_0)
\end{align}
La \eqref{eq:infinf00} est vraie car si on a $t \neq t_0$, on a $0 = 0$
et si on a $t = t_0$, on a $\infty = \infty$.

\section{Lien entre matrice de commandabilité et commandabilité}
\label{ann:command}
Soit un état $q_0$, on a
\begin{align*}
  q[1] & = A q_0 + B x[0]\\
  \vdots & = \vdots\\
  q[n] & = A^nq_0 + A^{n-1}Bx[0] + A^{n-2}Bx[1] + \ldots + Bx[n-1]\\
\end{align*}
Si on veut $q[n] = 0$, on doit avoir
\[ -A^nq_0 = A^{n-1}Bx[0] + \ldots + Bx[n-1]. \]
Pour que le système soit commandable,
il faut donc que pour tout $q_0$, $-A^nq_0$ soit une combinaison
linéaire des colonnes de $\begin{pmatrix}A^{n-1}B & \cdots & B\end{pmatrix}$.
Il suffit donc % faut ?
que cette matrice soit de rang plein.

% Our job here is done! (And yet another successfully completed mission.)
% Thérèse & Lena

\biblio

\end{document}
