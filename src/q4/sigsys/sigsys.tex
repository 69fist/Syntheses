%analogique:
%------> temps
%^
%|
%|
%| continuum
\documentclass[11pt,a4paper]{article}

% French
\usepackage[utf8x]{inputenc}
\usepackage[frenchb]{babel}
\usepackage[T1]{fontenc}
\usepackage{lmodern}
\usepackage{ifthen}

% Color
% cfr http://en.wikibooks.org/wiki/LaTeX/Colors
\usepackage{color}
\usepackage[usenames,dvipsnames,svgnames,table]{xcolor}
\definecolor{dkgreen}{rgb}{0.25,0.7,0.35}
\definecolor{dkred}{rgb}{0.7,0,0}

% Floats and referencing
\newcommand{\sectionref}[1]{section~\ref{sec:#1}}
\newcommand{\annexeref}[1]{annexe~\ref{ann:#1}}
\newcommand{\figuref}[1]{figure~\ref{fig:#1}}
\newcommand{\tabref}[1]{table~\ref{tab:#1}}
\usepackage{xparse}
\NewDocumentEnvironment{myfig}{mm}
{\begin{figure}[!ht]\centering}
{\caption{#2}\label{fig:#1}\end{figure}}

% Listing
\usepackage{listings}
\lstset{
  numbers=left,
  numberstyle=\tiny\color{gray},
  basicstyle=\rm\small\ttfamily,
  keywordstyle=\bfseries\color{dkred},
  frame=single,
  commentstyle=\color{gray}=small,
  stringstyle=\color{dkgreen},
  %backgroundcolor=\color{gray!10},
  %tabsize=2,
  rulecolor=\color{black!30},
  %title=\lstname,
  breaklines=true,
  framextopmargin=2pt,
  framexbottommargin=2pt,
  extendedchars=true,
  inputencoding=utf8x
}

\newcommand{\matlab}{\textsc{Matlab}}
\newcommand{\octave}{\textsc{GNU/Octave}}
\newcommand{\qtoctave}{\textsc{QtOctave}}
\newcommand{\oz}{\textsc{Oz}}
\newcommand{\java}{\textsc{Java}}
\newcommand{\clang}{\textsc{C}}
\newcommand{\keyword}{mot clef}

% Math symbols
\usepackage{amsmath}
\usepackage{amssymb}
\usepackage{amsthm}
\DeclareMathOperator*{\argmin}{arg\,min}
\DeclareMathOperator*{\argmax}{arg\,max}

% Sets
\newcommand{\Z}{\mathbb{Z}}
\newcommand{\R}{\mathbb{R}}
\newcommand{\Rn}{\R^n}
\newcommand{\Rnn}{\R^{n \times n}}
\newcommand{\C}{\mathbb{C}}
\newcommand{\K}{\mathbb{K}}
\newcommand{\Kn}{\K^n}
\newcommand{\Knn}{\K^{n \times n}}

% Chemistry
\newcommand{\std}{\ensuremath{^{\circ}}}
\newcommand\ph{\ensuremath{\mathrm{pH}}}

% Theorem and definitions
\theoremstyle{definition}
\newtheorem{mydef}{Définition}
\newtheorem{mynota}[mydef]{Notation}
\newtheorem{myprop}[mydef]{Propriétés}
\newtheorem{myrem}[mydef]{Remarque}
\newtheorem{myform}[mydef]{Formules}
\newtheorem{mycorr}[mydef]{Corrolaire}
\newtheorem{mytheo}[mydef]{Théorème}
\newtheorem{mylem}[mydef]{Lemme}
\newtheorem{myexem}[mydef]{Exemple}
\newtheorem{myineg}[mydef]{Inégalité}

% Unit vectors
\usepackage{esint}
\usepackage{esvect}
\newcommand{\kmath}{k}
\newcommand{\xunit}{\hat{\imath}}
\newcommand{\yunit}{\hat{\jmath}}
\newcommand{\zunit}{\hat{\kmath}}

% rot & div & grad & lap
\DeclareMathOperator{\newdiv}{div}
\newcommand{\divn}[1]{\nabla \cdot #1}
\newcommand{\rotn}[1]{\nabla \times #1}
\newcommand{\grad}[1]{\nabla #1}
\newcommand{\gradn}[1]{\nabla #1}
\newcommand{\lap}[1]{\nabla^2 #1}


% Elec
\newcommand{\B}{\vec B}
\newcommand{\E}{\vec E}
\newcommand{\EMF}{\mathcal{E}}
\newcommand{\perm}{\varepsilon} % permittivity

\newcommand{\bigoh}{\mathcal{O}}
\newcommand\eqdef{\triangleq}

\DeclareMathOperator{\newdiff}{d} % use \dif instead
\newcommand{\dif}{\newdiff\!}
\newcommand{\fpart}[2]{\frac{\partial #1}{\partial #2}}
\newcommand{\ffpart}[2]{\frac{\partial^2 #1}{\partial #2^2}}
\newcommand{\fdpart}[3]{\frac{\partial^2 #1}{\partial #2\partial #3}}
\newcommand{\fdif}[2]{\frac{\dif #1}{\dif #2}}
\newcommand{\ffdif}[2]{\frac{\dif^2 #1}{\dif #2^2}}
\newcommand{\constant}{\ensuremath{\mathrm{cst}}}

% Numbers and units
\usepackage[squaren, Gray]{SIunits}
\usepackage{sistyle}
\usepackage[autolanguage]{numprint}
%\usepackage{numprint}
\newcommand\si[2]{\numprint[#2]{#1}}
\newcommand\np[1]{\numprint{#1}}

\newcommand\strong[1]{\textbf{#1}}
\newcommand{\annexe}{\part{Annexes}\appendix}

% Bibliography
\newcommand{\biblio}{\bibliographystyle{plain}\bibliography{biblio}}

\usepackage{fullpage}
% le `[e ]' rend le premier argument (#1) optionnel
% avec comme valeur par défaut `e `
\newcommand{\hypertitle}[7][e ]{
\usepackage{hyperref}
{\renewcommand{\and}{\unskip, }
\hypersetup{pdfauthor={#6},
            pdftitle={Synth\`ese d#1#2 Q#3 - L#4#5},
            pdfsubject={#2}}
}

\title{Synth\`ese d#1#2 Q#3 - L#4#5}
\author{#6}

\begin{document}

\ifthenelse{\isundefined{\skiptitlepage}}{
\begin{titlepage}
\maketitle

 \paragraph{Informations importantes}
   Ce document est grandement inspiré de l'excellent cours
   donné par #7 à l'EPL (École Polytechnique de Louvain),
   faculté de l'UCL (Université Catholique de Louvain).
   Il est écrit par les auteurs susnommés avec l'aide de tous
   les autres étudiants, la vôtre est donc la bienvenue.
   Il y a toujours moyen de l'améliorer, surtout si le cours
   change car la synthèse doit alors être modifiée en conséquence.
   On peut retrouver le code source à l'adresse suivante
   \begin{center}
     \url{https://github.com/Gp2mv3/Syntheses}.
   \end{center}
   On y trouve aussi le contenu du \texttt{README} qui contient de plus
   amples informations, vous êtes invité à le lire.

   Il y est indiqué que les questions, signalements d'erreurs,
   suggestions d'améliorations ou quelque discussion que ce soit
   relative au projet
   sont à spécifier de préférence à l'adresse suivante
   \begin{center}
     \url{https://github.com/Gp2mv3/Syntheses/issues}.
   \end{center}
   Ça permet à tout le monde de les voir, les commenter et agir
   en conséquence.
   Vous êtes d'ailleurs invité à participer aux discussions.

   Vous trouverez aussi des informations dans le wiki
   \begin{center}
     \url{https://github.com/Gp2mv3/Syntheses/wiki}.
   \end{center}
   comme le statut des synthèses pour chaque cours
   \begin{center}
     \url{https://github.com/Gp2mv3/Syntheses/wiki/Status}.
   \end{center}
   vous pouvez d'ailleurs remarquer qu'il en manque encore beaucoup,
   votre aide est la bienvenue.

   Pour contribuer au bug tracker et au wiki, il vous suffira de
   créer un compte sur Github.
   Pour interagir avec le code des synthèses,
   il vous faudra installer \LaTeX.
   Pour interagir directement avec le code sur Github,
   vous devez utiliser \texttt{git}.
   Si cela pose problème,
   nous sommes évidemment ouverts à des contributeurs envoyant leurs
   changements par mail ou n'importe quel autre moyen.
\end{titlepage}
}{}

\ifthenelse{\isundefined{\skiptableofcontents}}{
\tableofcontents
}{}
}


\newcommand{\fourier}{\mathcal{F}}
\newcommand{\laplace}{\mathcal{L}}
\newcommand{\laplacu}{\mathcal{F}_u}
\newcommand{\transfz}{\mathcal{Z}}
\newcommand{\transzu}{\mathcal{Z}_u}
\newcommand{\roc}{\mathrm{ROC}}

\hypertitle{Signaux et syst\`emes}{4}{FSAB}{1106}{Benoît Legat}{Benoît Legat}


%  ____  _                   _
% / ___|(_) __ _ _ __   __ _| |___
% \___ \| |/ _` | '_ \ / _` | / __|
%  ___) | | (_| | | | | (_| | \__ \
% |____/|_|\__, |_| |_|\__,_|_|___/
%          |___/

\part{Signaux}
\section{Signaux}
Un signal peut-être vu soit de façon continue soit de façon discrète.
On notera les signaux continu avec les parenthèses (e.g. $x(t)$)
et les signaux discrets avec des crochets (e.g. $x[n]$).
Quand on notera juste $x$, ça signifiera que la propriété est vraie
en discret et en continu.

Pour les signaux discrets, on considère qu'on prend les points où on
évalue le signal à un interval régulier.
Du coup, on peut juste donner un entier $n$ en argument.
On nomme $T_s$, la période entre deux évaluation.
En effet, passer d'un signal continu $x(t)$ à un signal discret
$x[n] = x(nT_s)$ est appelé de l'échantionnage qui en anglais se dit
``sampling''. D'où le $T_s$ qui réfère à la période de ``sampling''.
L'échantillonage est vu plus en détail à la section~\ref{sec:sampling}.

\subsection{Propriétés d'un signal}
Un signal peut être pair si
\[ x(t) = x(-t) \]
ou impair si
\[ x(t) = -x(-t). \]
%On peut décomposer un signal complexe
%\[ z(t) = x(t) + jy(t) \]
%Symétrique conjuguée
%\begin{align*}
%  z(t) = z^*(-t)\\
%  x(t) +jy(t) = x(-t) - jy(-t)
%\end{align*}
%\begin{align*}
%  x(t) = x_p(t) + x_i(t)\\
%  x(-t) = x_p(t) - x_i(t)\\
%  x(t) + x(-t) = 2x_p(t)\\
%  x(t) - x(-t) = 2x_i(t)
%\end{align*}

Un signal peut être périodique.
C'est à dire qu'il existe $T$ tel que
$\forall t$,
\[ x(t) = x(t + T). \]
Le plus petit $T$ respectant cela est appelé sa période.

Par exemple,
\[ x(t) = B\cos(\omega t + \phi) \]
est périodique de période $\frac{2\pi}{\omega}$.
\[ x[n] = A\cos(\Omega n + \Phi) \]
est périodique si $\exists k, N \in \mathbb{Z}$ tels que
$\Omega = 2\pi\frac{k}{N}$, la période est alors $N$ divisé
par le PGCD de $k$ et $N$.

\section{Transformée de Fourier}
Nous noterons la transformée de Fourier d'un signal $x$, $\fourier\{x\}$.

\subsection{Transformée de Fourier continue}
Toute fonction $x(t)$ peut être non seulement décrit par son continuum de
valeurs aux temps $t$ mais aussi par un continuum d'amplitudes à des fréquences
$f$.
On nomme la fonction qui donne l'amplitude de ces fréquences sa transformée
de Fourier.
On la note avec une majuscule et au lieu de lui donner en argument la fréquence
$f$, on lui donne $j\omega$ ce qui ne change pas grand chose car
$j\omega = j2\pi f$.
On a les relation suivantes
\begin{align}
  \label{eq:tfc}
  X(j\omega) & = \int_{-\infty}^\infty x(t) e^{-j\omega t} \dif t\\
  \nonumber
  x(t) & = \frac{1}{2\pi} \int_{-\infty}^\infty X(j\omega) e^{j\omega t}
  \dif \omega.
\end{align}

Comme $|e^{-j\omega t}| = 1$, on remarque que l'intégrale
converge si $\int_{-\infty}^\infty |x(t)| \dif t < \infty$.

\subsection{Transformée de Fourier discrète}
\label{sec:fourier_discret}
En discret, on pourrait imaginer qu'on va appliquer la même formule en prenant
simplement $x(t) =  \sum_{n=-\infty}^\infty x[n] \delta(t-nT_s)$, c'est à dire
qu'on aurait, en notant $\Omega = \omega T_s$,
\begin{align}
  \label{eq:tfd}
  X(j\omega) & = \int_{-\infty}^\infty \sum_{n=-\infty}^\infty x[n]
  \delta(t-nT_s) e^{-j\omega T_sn} \dif t\\
  \nonumber
  & = \sum_{n=-\infty}^\infty x[n] e^{-j\Omega n}.
\end{align}
Seulement, on peut remarquer que
\begin{align*}
  X(j(\omega+2\pi/T_s)) & = \sum_{n=-\infty}^\infty x[n]
  e^{-j(\Omega+2\pi) n} \dif t\\
  & = \sum_{n=-\infty}^\infty x[n]
  e^{-j\Omega n} \dif t\\
  & = X(j\omega).
\end{align*}
La transformée de Fourier serait donc périodique de période $\frac{2\pi}{T_s}$.
Ça justifie d'ailleurs le tableau~\ref{tab:recap_fourier}.

En connaissance de cela, un transformée de Fourier discrète plus appropriée
a été créée
\begin{align*}
  X(e^{j\Omega}) & = \sum_{n=-\infty}^\infty x[n]e^{-j\Omega n}\\
  x[n] & = \frac{1}{2\pi}
  \int_{0}^{2\pi} X(e^{j\Omega})e^{j\Omega n} \dif \Omega.
\end{align*}
À nouveau, comme $|e^{-j\Omega t}| = 1$, on remarque que la somme
converge si $\sum_{n=-\infty}^\infty |x[n]| < \infty$.

On pourrait se demander pourquoi on note $X(e^{j\Omega})$ et pas $X(j\Omega)$.
Il se fait que c'est en fait beaucoup plus explicite de le noter ainsi.
En effet, on montre ici bien que c'est périodique de période $2\pi$
en $\Omega$ car, si on prend $\Omega + 2\pi$,
on obtient $X(e^{j(\Omega + 2\pi)}) = X(e^{j\Omega})$.

\subsection{Série de Fourier continue}
Si on a un signal périodique de période $T$, on peut faire encore mieux.
Au lieu de prendre toutes les fréquences possibles dans le domaine spectral,
on ne va prendre que celle donc l'exponentielle correspondante est
aussi périodique de période $T$.

Ça peut aussi se justifier de la même manière qu'on l'a fait pour
la transformée de Fourier discrète.
En effet, on a vu à la section~\ref{sec:fourier_discret}
que la transformée de Fourier d'un signal discret
donnait un signal périodique.
On peut donc s'attendre à ce que la transformée de Fourier inverse d'un
signal discret soit un signal périodique.
Tout cela justifie à nouveau le tableau~\ref{tab:recap_fourier}.

Pour cela, rappelons nous que $e^{jk\omega_0t}$ est un signal périodique
de période $\frac{2\pi}{\omega_0}$.
En posant $\omega_0 = \frac{2\pi}{T}$, on a donc
\begin{align}
  \label{eq:sfc}
  X[k] & = \frac{1}{T}\int_0^T x(t) e^{-jk\omega_0t} \dif t\\
  \label{eq:sfci}
  x(t) & = \sum_{k=-\infty}^\infty X[k]e^{jk\omega_0t}.
\end{align}

\subsubsection{Série de Fourier d'une répétition de deltas}
\label{sec:deltas}
Soit un signal
\[ x(t) = \sum_{n=-\infty}^\infty \delta(t-nT), \]
sa série de Fourier est
\footnote{On intègre de $-T/2$ à $T/2$ car si on intègre
de $0$ à $T$ ce n'est pas claire quand à ce qu'on garde de $\delta(t)$ et
$\delta(t-T)$.
De toute façon c'est périodique donc l'intégrale est la même.}
, par \eqref{eq:sfc},
\begin{align*}
  X[k] & = \frac{1}{T}\int_{-T/2}^{T/2} \sum_{n=-\infty}^\infty \delta(t-nT)
  e^{-2\pi jkt/T} \dif t\\
  & = \frac{1}{T}\int_{-T/2}^{T/2} \delta(t) e^{-2\pi jkt/T} \dif t\\
  & = \frac{1}{T} e^{-2\pi jk0/T}\\
  & = \frac{1}{T}.\\
\end{align*}

Du coup, par \eqref{eq:sfci},
\[ x(t) = \frac{1}{T}\sum_{k=-\infty}^\infty e^{2\pi jkt/T} \]
ce qui nous permet,
par la linéarité de la transformée de Fourier et en nous aidant du
tableau~\ref{tab:fourier_ex} et de la propriété de décalage en fréquentiel,
d'obtenire le résultat assez intéressant
qui nous sera utile dans la section~\ref{sec:sampling}
\[ X(j\omega) = \frac{2\pi}{T}
\sum_{k=-\infty}^\infty \delta(\omega-2\pi k/T). \]

\subsection{Série de Fourier discrète}
Un signal discret et périodique a donc,
selon de tableau~\ref{tab:recap_fourier}, une transformée
périodique car il est discret et discrète car il est périodique.

En considérant que $x[n]$ est de période $N$, on a donc,
avec $\Omega_0 = \frac{2\pi}{N}$,
\begin{align}
  \label{eq:sfd}
  X[k] & = \frac{1}{N} \sum_{k=0}^{N-1}x[n]e^{-jk\Omega_0n}\\
  \nonumber
  x[n] & = \sum_{k=0}^{N-1}X[k]e^{jk\Omega_0n}.
\end{align}

Il est amusant de s'imaginer la série de Fourier discrète
comme une interpolation des $N$ points d'une période par
une combinaison linéaire des $N$ fonctions $\phi_k(t) = e^{jk\Omega_0n}$.
Les coefficients de cette combinaison linéaire sont alors les $X[k]$.

\begin{table}
  \centering
  \begin{tabular}{|l|l|}
    \hline
    \textbf{Temps} & \textbf{Fréquence}\\
    \hline
    Continu & Non-périodique\\
    \hline
    Discret & Périodique\\
    \hline
    Non-périodique & Continue\\
    \hline
    Périodique & Discret\\
    \hline
  \end{tabular}
  \caption{Tableau récapitulatif de la transformée de Fourier}
  \label{tab:recap_fourier}
\end{table}

\subsection{Propriété de la transformée de Fourier}
La transformée de Fourier a pas mal de propriétés intéressantes.
Pour commencer, elle est linéaire, c'est à dire que
$\fourier\{ax + by\} = a\fourier\{x\} + b\fourier\{y\}$.
À ne pas confondre avec le changement d'échelle
\[ \fourier\{x(at)\} = \frac{1}{|a|}X(j\omega/a). \]

De plus,
\begin{itemize}
  \item Si un signal est réel, sa transformée de Fourier est paire et sa partie
imaginaire est impaire.
  \item Si un signal est pair, sa transformée de Fourier est réelle et paire.
  \item Si un signal est impair,
    sa transformée de Fourier est imaginaire pure et impaire.
\end{itemize}

Pour des signaux \emph{non-périodiques} $x$ et $y$,
\begin{align*}
  \fourier\{x * y\} & = X Y\\
  \fourier\{xy\} & = \frac{1}{2\pi}X * Y
\end{align*}
et pour des signaux \emph{périodiques} $x$ et $y$,
\begin{align*}
  %\fourier\{x * y\} & = X Y\\ % TODO is it true ?
  \fourier\{xy\} & = X * Y
\end{align*}

Une propriété très intéressante pour les équations différentielles est que
\begin{align*}
  \fourier\left\{\fdif{x(t)}{t}\right\} & = j\omega X(j\omega)\\
  \fourier^{-1}\left\{\fdif{X(j\omega)}{\omega}\right\} & =
  -jtx(t).
\end{align*}

On a aussi un propriété de décallage en continu
\begin{align*}
  \fourier\{x(t-t_0)\} & = e^{-j\omega t_0}X(j\omega)\\
  \fourier^{-1}\{X(j(\omega-\nu))\} & = e^{j\nu t}x(t)
\end{align*}
et en discret
\[ \fourier\{x[n-k]\} = e^{-j\Omega k} X(e^{j\Omega}) \]
qui va nous être utile dans la résolution d'équations aux différences.

Une propriété intéressante pour trouver $U(j\omega)$ donné dans le
tableau~\ref{tab:fourier_ex} à l'aide de \eqref{eq:uintdelta} est que si
$y(t) = \int_{-\infty}^t x(\tau) \dif \tau$, alors
\[ Y(j\omega) = \frac{1}{j\omega}X(j\omega) + \pi X(j0)\delta(\omega). \]

La relation de Parceval permet de passer d'une certaine intégrale en temporel
à une intégrale en fréquentiel
\[ \int_{-\infty}^\infty |x(t)|^2 \dif t
= \frac{1}{2\pi} \int_{-\infty}^\infty |X(j\omega)|^2 \dif \omega. \]

\begin{table}
  \[
    \begin{array}{|c|c|l|}
      \hline
      x(t) & X(j\omega) & \text{condition (convergence)}\\
      \hline
      e^{-at}u(t) & \frac{1}{a + j\omega} & \Re(a) > 0\\
      \frac{\sin(Wt)}{\pi t} &
      \begin{cases}
        1, \text{si } |\omega| \leq W\\
        0, \text{sinon}
      \end{cases}
      &\\
      \frac{1}{2\pi} & \delta(\omega) &\\
      u(t) & \frac{1}{j\omega} + \pi \delta(\omega)&\\
      \hline
      x[n] & X(e^{j\Omega}) &\\
      \hline
      \alpha^n u[n] & \frac{1}{1-\alpha e^{-j\Omega}} & |\alpha| < 1\\
      \frac{\sin(Wn)}{\pi n} &
      \begin{cases}
        1, \text{si } |\Omega| \leq W\\
        0, \text{sinon}
      \end{cases}
      &\\
      \hline
    \end{array}
  \]
  \caption{Exemples de transformées de Fourier}
  \label{tab:fourier_ex}
\end{table}

\subsection{Échantillonnage}
\label{sec:sampling}
\subsubsection{Échantillonnage en temporel}
Lorsqu'on échantionne un signal continu $x(t)$ en $x_e(t)$,
comme on a vu à la section~\ref{sec:fourier_discret},
son spectre devient périodique.

En effet, échantillonner,
c'est comme multiplier en temporel par
$\sum_{n=-\infty}^{\infty} \delta(t-nT_s)$.
C'est à dire, comme on a vu dans la section~\ref{sec:deltas},
faire un convolution en fréquentiel par
$\frac{2\pi}{T_s}\sum_{k=-\infty}^\infty\delta(\omega-2\pi k/T_s)$.
On va donc prendre le spectre de $x(t)$ et on va le reproduire tous les
$2\pi/T_s$.
On remarque donc que si $X(j\omega)$ n'est différent de 0 que pour
$|\omega| \leq \omega_\mathrm{max}$,
le spectre a une largeur de $2\omega_\mathrm{max}$,
il faut donc que $\frac{2\pi}{T_s} > 2\omega_\mathrm{max}$ pour que les
reproductions de signal ne se chevauchent pas.
Comme $\omega_s = \frac{2\pi}{T_s}$, on a en fait $f_s > 2f_\mathrm{max}$.
C'est le \emph{théorème de Shannon}.

Pour reconstituer le signal de départ après échantillonnage,
il suffit de se débarrasser de ces reproductions.
Pour cela, si on avait fait bien attention à
ce qu'il n'y ait pas de chevauchements,
il suffit de multiplier le signal en fréquentiel par un fenêtre
qui vaut $T$ pour $-\pi/T_s \leq \omega < \pi/T_s$ et 0 sinon.
% Pourquoi pas T/2pi car on a multiplié par 2pi/T...
Ce qui fait une convolution par $\sin(\pi t/T)/(\pi t/T)$ en temporel.

\subsubsection{TFD - Transformée de Fourier discrète}
Faire une transformée de Fourier discrète consiste simplement à faire une
transformée de Fourier normale et puis d'échantillonner en fréquentiel.
Ce qui, du coup, reproduit périodiquement le signal en temporel.

Ça fait fort penser à la série de Fourier, en effet,
si on échantillonne \eqref{eq:tfd} tous les $2\pi/N$, on arrive
à la même chose que \eqref{eq:sfd} modulo un facteur $1/N$.
En effet, on a
\[ X[k] = \frac{X(e^{ik2\pi/N})}{N}. \]

\section{Transformée de Laplace}
La transformée de Laplace est une généralisation de la transformée de Fourier
continue.

En effet, au lieu de se limiter à $j\omega$, on prend $s = \sigma + j\omega$.
Les formules deviennent
\begin{align}
  \label{eq:tl}
  X(s) & = \int_{-\infty}^\infty x(t) e^{-st} \dif t\\
  \nonumber
  x(t) & = \frac{1}{2\pi}\int_{-\infty}^\infty X(\sigma+j\omega)
  e^{(\sigma+j\omega)t} \dif \omega.
\end{align}

En comparant l'équation~\ref{eq:tl} à l'équation~\ref{eq:tfc}, on remarque que
$\laplace\{x(t)\} = \fourier\{x(t) e^{-\sigma t}\}$.
D'où la condition de convergence.
\[ \int_{-\infty}^\infty |x(t)|e^{-\sigma t} < \infty. \]
Contre toute attente, on a donc une convergence dépendante de $\sigma$, c'est
à dire la partie réelle de $s$.
L'avantage, c'est qu'on va trouver des transformée de Laplace pour des
signaux n'ayant pas de transformée de Fourier.
Par exemple, bien que $e^tu(t)$ n'avait pas de transformée de Fourier,
on va pouvoir lui trouver une transformée de Laplace.
Le désavantage, c'est qu'elle va avoir une ``Region Of Convergence'' ($\roc$).

\subsection{Propriétés de la transformée de Laplace}
Comme pour Fourier, on a la linéarité
\[ \laplace\{ax(t)+by(t)\} = aX(s) + bY(s) \]
où $\roc_1 \cap \roc_2 \subseteq \roc$,
le décalage temporel
\[ \laplace\{x(t-t_0)\} = e^{-st_0}X(s) \]
où le $\roc$ ne change pas,
le décalage fréquentiel
\[ \laplace^{-1}\{X(s-s_0)\} = e^{s_0t}x(t) \]
où le $\roc$ devient $\roc + \Re(s_0)$,
la convolution
\[ \laplace\{x*y\} = XY, \]
la différentiation
\[ \laplace\left\{\fdif{x}{t}\right\} = sX(s), \]
l'intégration
\[ \laplace\left\{\int_{-\infty}^tx(\tau)\dif\tau\right\} = \frac{X(s)}{s} \]
et la différentiation en fréquence
\[ \laplace^{-1}\left\{\fdif{X(s)}{s}\right\} = -tx(t). \]

\subsection{Transformée de Laplace unilatérale}
\label{sec:laplacu}
La propriété sur la différentiabilité nous permet de résoudre les équations
différentielles aux données initiales nulles mais pour les autres,
il nous faudra définir une transformée de Laplace unilatérale
\begin{align*}
  X_u(s) & = \int_{0^-}^\infty x(t) e^{-st} \dif t.
\end{align*}

Pour cette transformée, on a
\begin{align*}
  \laplacu\left\{\fdif{x(t)}{t}\right\} & =
  sX_u(s) - x(0^-)\\
  \laplacu\left\{\ffdif{x(t)}{t}\right\} & =
  s^2X_u(s) - sx(0^-) - x'(0^-)
\end{align*}
et pour des systèmes stables et causaux,
\begin{align*}
  \lim_{s\to\infty} sX_u(s) & = x(0^+)\\
  \lim_{s\to 0} sX_u(s) & = x(\infty).
\end{align*}

\section{Transformée en $z$}
La transformée en $z$ est une généralisation de la transformée de Fourier
discrète.

La transformée en $z$ d'un signal discret est
\begin{align*}
  X(z) & = \sum_{k=-\infty}^{\infty} x[k]z^{-k}\\
  x[n] & =
  \frac{1}{2\pi}\int_{-\pi}^{\pi} X(re^{j\omega})(re^{j\omega})^n\dif \omega\\
  & = \frac{1}{2\pi j}\oint X(z) z^{n-1} \dif z.
\end{align*}
L'égalité entre les deux dernières intégrales
s'obtient directement avec le changement
de variable $z = re^{j\omega}$ d'où $\dif z = zj\dif \omega$.

Comme $\transfz(x[n]) = \fourier(x[n]r^n)$,
une condition suffisante pour qu'une transformée de Laplace converge est
\[ \sum_{k=-\infty}^\infty|x[n]r^{-n}| < \infty. \]
La convergence dépend donc de $r$.

\subsection{Propriétés de la transformée en $z$}
Comme pour Laplace, on a la linéarité
\[ \transfz\{ax[n]+by[n]\} = aX(z) + bY(z) \]
où $\roc_1 \cap \roc_2 \subseteq \roc$,
le décalage temporel
\[ \transfz\{x[n-n_0]\} = z^{-n_0}X(z) \]
où le $\roc$ ne change pas,
le changement d'échelle en fréquentiel
\[ \transfz^{-1}\{X(z/z_0)\} = z_0^nx[n] \]
où le $\roc$ devient $\roc |z_0|$,
la convolution
\[ \transfz\{x*y\} = XY, \]
le décalage temporel
\[ \transfz\{x[n-1]\} = z^{-1}X(z), \]
et la différentiation en fréquence
\[ \transfz^{-1}\left\{z\fdif{X(z)}{z}\right\} = -nx[n]. \]

\subsection{Transformée de Laplace unilatérale}
Comme pour la section~\ref{sec:laplacu}, on a besoin
d'une transformée en $z$ unilatérale pour résoudre les équations
différentielles à conditions initiales non-nulles
\begin{align*}
  X_u(z) & = \sum_{n=0}^\infty x[n] z^{-k}.
\end{align*}

Pour cette transformée, on a
\begin{align*}
  \transzu\{x[n-1]\} & =
  z^{-1}X_u(z) + x[-1]\\
  \transzu\{x[n-2]\} & =
  z^{-2}X_u(z) + z^{-1}x[-1] + x[-2]\\
  \transzu(x[n-k]) & =
  z^{-k}X_u(z) + z^{-k+1}x[-1] + \ldots + z^{-1}x[-k+1] + x[-k]
\end{align*}
et pour des systèmes stables et causaux,
\begin{align*}
  \lim_{|z|\to\infty} X_u(z) & = x[0^+]\\
  \lim_{z\to 1} (z-1)X_u(z) & = x[\infty].
\end{align*}

Attention le principe est le même qu'avec Laplace mais pas le signe.


% on est pas interrogé sur le développement en série de puissances

%  ____            _
% / ___| _   _ ___| |_ ___ _ __ ___  ___
% \___ \| | | / __| __/ _ \ '_ ` _ \/ __|
%  ___) | |_| \__ \ ||  __/ | | | | \__ \
% |____/ \__, |___/\__\___|_| |_| |_|___/
%        |___/

\part{Systèmes}
Un système prend un signal en entrée $x(t)$ et retourne un signal $y(t)$
en sortie.
On note $H\{x(t)\} = y(t)$.

\section{Système LTI}
Un système LTI est un système Linéaire à Temps Invariant.

\subsection{Invariance temporelle}
Il est à temps invariant.
Ça veut dire que si $H\{x(t)\} = y(t)$, $H\{x(t-t_0)\} = y(t-t_0)$.

\subsection{Linéarité}
Un système peut être linéaire, c'est à dire que
\[ H\{\sum a_ix_i(t)\} = \sum a_iH\{x_i(t)\} \]

\subsection{Réponse impulsionnelle}
On peut représenter un système LTI à l'aide de sa réponse impulsionnelle
définie par $h(t) = H\{\delta(t)\}$. On a
\[ y(t) = x(t) * h(t). \]

\subsection{Réponse indicielle}
La réponse indicielle d'un système LTI notée $s(t)$ est par définition
\begin{align*}
  s(t) & \eqdef h(t) * u(t)\\
  & = \int_{-\infty}^t h(\tau) \dif \tau &
  s[n] & = \sum_{-\infty}^n h[k]
\end{align*}
d'où
\begin{align*}
  h(t) & = \fdif{}{t}s(t) & h[n] & = s[n] - s[n-1].
\end{align*}


%Parallèle : $+$
%Série : $*$ commutatif car LTI

\subsection{Système LTI sans mémoire}
Un système LTI peut être sans mémoire,
ça signifie simplement que $y(t)$ ne dépend que de $x(t)$.
Comme le système est temporellement invariant,
il fait la même chose pour tous les $x(t)$.
Du coup, $\exists c$ tel que $y(t) = cx(t)$, d'où
\begin{align*}
  h[n] & = c \delta[n] & h(t) = c \delta(t).
\end{align*}

\subsection{Système LTI causal}
Un système est causal si la réponse impulsionnelle ne dépend
que des valeurs présentes et passées.
Pas des valeurs futures.
Ce qui revient à dire que
\begin{align*}
  h[n] & = 0 \, \forall n < 0  & h(t) = 0 \, \forall t < 0.
\end{align*}
En effet, comme
\[ y[n] = \sum_{-\infty}^{+\infty}x[k]h[n-k] \]
pour que le signal ne soit pas influencé par les $x[k]$ avec
$k > n$, il faut que le $h[n-k]$ correspondant soit 0.
Comme $k > n$, $n - k < 0$ donc il faut bien $h[k] = 0$ $\forall k < 0$.

\subsection{Système LTI inversible}
Pour qu'un sysème LTI soit inversible, il vaut qu'il existe $h^{-1}$ tel que
$h^{-1}*h*x = x$.
Comme vu précédemment, le neutre de la convolution est $\delta$ donc il faut
qu'il existe $h^{-1}$ tel que $h^{-1} * h = \delta$.

%\subsection{Équations différentielles}
%Time Invariant car coefficients indépendants du temps.
%Linear car $y$ et $y'$ etc. apparaissent de façon linéaire

\section{Transformée en $z$ dans un système LTI}
Supposons qu'on doive calculer l'image d'un signal discret par un système LTI.
On peut éviter de calculer en produit de convolution grâce à la fonction
de transfert
\[ H(z) = \sum_{k=-\infty}^{\infty} h[k] z^{-k}. \]
On remarque d'ailleurs que $H(z)$ est la transformée en $z$ de $h$.

En effet, on remarque que l'image de $z^n$ est
\begin{align*}
  \sum_{k=-\infty}^{\infty} h(k) z^{n-k} &
  = z^{n} \sum_{k=-\infty}^{\infty} h[k] z^{-k}\\
  & = z^n H(z)
\end{align*}

Dès lors, lorsqu'on veut calculer l'image d'un signal $x[n]$. Il suffit de
prendre sa transformée de Laplace $X(z)$ dont l'image est $Y(z) = X(z) H(z)$
et puis d'appliquer la transformée en $z$ inverse sur $Y(z)$ pour trouver
$y[n]$.

C'est pareil pour les signaux continu avec Laplace.

\section{Schéma bloc et représentation d'état}
Le schéma bloc et la représentation d'état constituent la représenation
interne d'un système, il y en a en général plusieurs possibles pour une
même représentation entrée-sortie.

Un schéma bloc est difficile à appréhender si on essaie d'y appliquer
notre logique des circuits électriques.
En effet, oubliez la loi de Kirchhoff sur les courants
$I_1 + I_2 + \ldots + I_n = 0$, dans un schéma bloc, c'est plus
$I_1 = I_2 = \ldots = I_n = 0$.

Un schéma bloc est constitué d'une entrée $x$, d'une sortie $y$ et
d'états internes $q_i$.
Il est parsemé de blocs $\int$ en continu (resp. $D$ en discret) tels que
la sortie du bloc est l'intégrale (resp. le décallage $D\{x[n]\} = x[n-1]$) du
signal d'entrée.
On mets généralement les $q_i$ à la sortie des blocs.

On peut décrire le schéma bloc à l'aide de sa représentation d'état qui
en continu est de la forme
\begin{align*}
  q'(t) & = Aq(t) + Bx(t)\\
  y(t) & = Cq(t) + Dx(t)
\end{align*}
et en discret est de la forme
\begin{align*}
  q[n+1] & = Aq[n] + Bx[n]\\
  y[n] & = Cq[n] + Dx[n].
\end{align*}

On remarque que lorsqu'on a mis les $q_i$ aux sorties des blocs,
les $q_i[n+1]$ et les $q'(t)$ sont simplement l'entrée des blocs.
La première équation s'écrit donc simplement en regardant ce qui rentre
dans les blocs.
La seconde équation s'écrit en regardant ce qui sort du système en $y$.

\subsection{Changement de représentation interne}
Lorsqu'on change de représentation interne, on peut obtenir
la nouvelle représentation d'état en exprimant le changement de représentation
comme un changement de variable en $q$, $\tilde{q} = Tq$.
On a alors
\begin{align*}
  \tilde{q} & = TAT^{-1}\tilde{q} + TBx\\
  y & = CT^{-1}\tilde{q} + Dx.
\end{align*}
Ce qui n'est sûrement pas à retenir par coeur tellement c'est immédiat.
Il suffit de remplacer $q$ par $T^{-1}\tilde{q}$ dans la représentation d'état
en $q$ puis de multiplier la première équation par $T$.

On remarquera quand même que ça nous permet de facilement obtenir une
transformation du système qui rendra $\tilde{A}$ diagonale.
En effet, trouver $D$ et $Q$ tels que $A = QDQ^{-1}$ est une opération connue,
c'est la diagonalisation.
Pour se trouver avec $\tilde{A}$ diagonale,
il suffit donc de trouver $Q$ en diagonalisant $A$
et faire un changement de variable $\tilde{q} = Tq$ avec $T = Q^{-1}$.

\subsection{De représentation d'état à fonction de transfert}
On peut passer de la représentation d'état de temporel à fréquentiel
en gardant les même matrices.

En continu, on a
\begin{align*}
  sQ(s) & = AQ(s) + BX(s)\\
  Y(s) & = CQ(s) + DX(s)
\end{align*}
et on trouve alors
\[ H(s) = \frac{Y(s)}{X(s)} = C(sI-A)^{-1}B + D. \]
En discret, c'est pareil
\begin{align*}
  zQ(z) & = AQ(z) + BX(z)\\
  Y(z) & = CQ(z) + DX(z)
\end{align*}
et on trouve alors
\[ H(z) = \frac{Y(z)}{X(z)} = C(zI-A)^{-1}B + D. \]


\section{Équations différentielles - équations aux dérivées partielles}
Elle constitue la représentation entrée-sortie d'un système contrairement
au schéma bloc ou à la représentation d'état qui constitue une représenation
interne.

Tout comme dans la résolution standard,
on parle de solution homogène et solution particulière,
on va parler de réponse libre et de réponse forcée.
\begin{description}
  \item[Réponse libre]
    Condition initiales spécifiées et entrée nulle.
    C'est un système linéaire \emph{mais pas à temps invariant}.
    Il faut donc faire attention à ne pas utiliser les transformée
    unilatérales.
  \item[Réponse forcée]
    Condition initiales nulles et entrée spécifiée.
    C'est un système LTI.
\end{description}

\section{Commandabilité et observabilité}
Cette section traitera du cas du système continu mais c'est pareil en discret.
\subsection{Commandabilité}
\begin{description}
  \item[État commandable]
    Un état $q_0$ est commandable si,
    pour un état initial $q(0) = q_0$,
    il existe un $T$ et un $x$ tels qu'on ait $q(T) = 0$.
  \item[État accessible]
    Un état $\bar{q}$ est dit accessible si,
    pour un état initial $q(0) = 0$,
    il existe un $T$ et un $x$ tles qu'on ait $q(T) = \bar{q}$.
  \item[Système complètement commandable]
    Un système est complètement commandable si tous ses états sont
    commandables.
\end{description}

Seulement, l'ensemble des états commandables est l'espace engendré
par les colonnes de la matrice de commandabilité
(voir l'annexe~\ref{ann:command} pour une démonstration)
\[ \begin{pmatrix}B & AB & \cdots & A^{n-1}B\end{pmatrix}. \]
Du coup, un système est complètement commandable si cette matrice est de
plein rang.


\subsection{Observabilité}
\begin{description}
  \item[État non observable]
    Un état $q_0$ est non observable si,
    pour un état initial $q(0) = q_0$,
    pour tout $T > 0$, si $x(t) = 0$ $\forall 0 \leq t \leq T$, alors
    $y(t) = 0$ $\forall 0 \leq t \leq T$.
  \item[Système complètement observable]
    Un système est complètement observable si aucun de ses états n'est
    non observable.
\end{description}

À nouveau, l'ensemble des états non observables est le $\ker$ de
\[ \begin{pmatrix}C\\CA\\\cdots\\CA^{n-1}\end{pmatrix}. \]
Du coup, un système est complètement observable si cette matrice est de
plein rang.

On remarque que pour une certaine fonction de transfert,
on peut avoir une multitude de schéma blocs ou de représentations d'états
différents.
Mais la fonction de transfert n'utilise
que la partie observable et commandable.
Si un $q_i$ n'est pas observables, ou est non commandable
(c'est à dire, pour la première composant de $q$ par exemple, que l'état
$q = (1, 0, \ldots, 0)$ n'est pas observable ou est non commandable),
il n'apparaitra pas
dans la fonction de transfert.

\section{Stabilité}
\subsection{Stabilité BIBO}
Un système est BIBO stable si,
lorsque $|x[n]| \leq M_x < \infty$ $\forall n$,
on a aussi $|y[n]| \leq M_y < \infty$ $\forall n$ (pareil en continu).
Seulement, comme
\begin{align*}
  |y[n]| & = \left|\sum_{k=-\infty}^{\infty}h[k]x[n-k]\right|\\
         & \leq M_x \sum_{k=-\infty}^{\infty}|h[k]|,
\end{align*}
une condition \emph{suffisante} pour qu'un système soit BIBO stable est que
\begin{align*}
  \sum_{k=-\infty}^\infty |h[k]| & < \infty &
  \int_{t=-\infty}^\infty |h(t)| \dif t & < \infty
\end{align*}
C'est la même condition que celle de l'existence de la transformée de Fourier
de $h$.

Comme la transformée de Fourier
est la transformée en $z$ pour $r = 1$,
si la tranformée en $z$ a le cercle unité dans son ROC, on sait
que le système est BIBO stable.

Comme la transformée de Fourier est la transformée de Laplace pour $\Re(z) = 0$,
si l'axe imaginaire appartient au ROC, on sait
que le système est BIBO stable.

Pou un système causaux, il faut soit que les pôles soient
à l'intérieur du cercle unité pour les systèmes discrets soit
à gauche de l'axe imaginaire pour les systèmes continus.

\subsection{Stabilité interne}
Un point $(\bar{q}, \bar{x})$ est un équilibre si $A\bar{q} + B\bar{x} = 0$.
Cette équilibre est \emph{stable} si un bruit suffisament faible
va toujours rester dans une marge et est \emph{attractif} si
un bruit suffisament faible va toujours se tasser à 0.
Un équilibre \emph{stable} et \emph{attractif}
est dit \emph{asymptotiquement stable}.

On remarque que $(0,0)$ est un équilibre et que tout équilibre peut se
ramener à $(0,0)$ par changement de variable.

Du coup, pour qu'un système ait tous ces équilibres stables,
il faut et il suffit que $(0,0)$ soit stable.
On peut donc, au lieu de parler d'équilibre stable,
directement parler de système stable.
Car tous les équilibres ont la même nature,
on peut donc parler du système.
Dire qu'un système est stable,
ça signifie que tous les équilibres du système sont stables.

Soit $\lambda_i$ les valeurs propres de $A$.
En continu,
\begin{itemize}
  \item Si $\forall i$ $\Re(\lambda_i) < 0$, le système est asymptotiquement
    stable;
  \item Si $\exists i$ $\Re(\lambda_i) > 0$, le système est instable;
  \item Si $\forall i$ $\Re(\lambda_i) \leq 0$ et que $\forall i$ tel que
    $\Re(\lambda_i) = 0$, $m_g(\lambda_i) = m_a(\lambda_i)$,
    le système est marginalement stable;
  \item Si $\forall i$ $\Re(\lambda_i) \leq 0$ et que $\exists i$ tel que
    $\Re(\lambda_i) = 0$, où $m_g(\lambda_i) < m_a(\lambda_i)$,
    le système est instable.
\end{itemize}

% TUYAU!!!
Et en discret,
\begin{itemize}
  \item Si $\forall i$ $|\lambda_i| < 1$, le système est asymptotiquement
    stable;
  \item Si $\exists i$ $|\lambda_i| > 1$, le système est instable;
  \item Si $\forall i$ $|\lambda_i| \leq 1$ et que $\forall i$ tel que
    $|\lambda_i| = 1$, $m_g(\lambda_i) = m_a(\lambda_i)$,
    le système est marginalement stable;
  \item Si $\forall i$ $|\lambda_i| \leq 1$ et que $\exists i$ tel que
    $|\lambda_i| = 1$, où $m_g(\lambda_i) < m_a(\lambda_i)$,
    le système est instable.
\end{itemize}

\subsection{Stabilité interne et externe}
% TUYAU!!!
Pour la tabilité interne, on regarde tous les états,
si un est instable, c'est instable.

Comme la fonction de transfert ne regarde que
la partie observable et commandable,
on pourrait avoir une fonction de transfert
BIBO stable pour un système qui est
lui instable car dans la relation entrée sortie,
on a pas vu la composante instable.
Ça correspond à l'annulation pôle zéro dans l'expression de $H(s)$.

Si on a la stabilité interne, alors on a la stabilité BIBO.
Car si mon système entier est stable,
alors la partie qui est observable et commandable est stable aussi.

Si on a une stabilité BIBO et une instabilité interne,
alors on a une annulation de zéro,
et donc une perte de commandabilité ou d'observabilité
car si tout était stable et commandable,
on aurait eu une instabilité BIBO aussi.

%Par exemple, pour
%\begin{align*}
  %q_1' & = -2q_2\\
  %q_2' & = q_1 + 2q_2  3q_3 + x\\
  %q_3' & = q_1 + 6q_2 - 7q_3 + x
%\end{align*}
%
%\[ C = \begin{pmatrix}0 & -1 & 2\\1 & -1 & -1\\1 & -1 & -1\end{pmatrix} \]
%base $\mathcal{C}(C)^\perp = (0;1;-1)$
%donc posons $z_1 = q_1$, $z_2 = q_2$ et $z_3 = q_2 - q_3$.

\subsection{Feedback}
Soit un système LTI avec un gain de fonction de transfert $G(s)$ à la suite
de $H$.
On a du coup $Y = GH(X - Y)$ d'où $(1+GH)Y = GHX$ et donc
\[ T(s) = \frac{GH}{1+GH}. \]

Il faut cependant éviter d'avoir une sensibilité trop forte
\[ \frac{\Delta T/T}{\Delta G/G} =
\frac{G}{T}\frac{H(1 + HT) - GHH}{(1+GH)^2}
= \frac{G(1+GH)}{GH}\frac{H}{(1+GH)^2}
= \frac{1}{1+GH}. \]

Il faut donc un $GH/(1+GH)$ proche de 1 et un $1/(1+GH)$ petit.

En effet, si on a une perturbation $v$ à la sortie de $G$, on a
\[ Y = \frac{GH}{1+GH}X + \frac{1}{1+GH}v \approx X. \]

Il faut aussi faire attention de ne pas déstabiliser le système avec
le feedback.

Par exemple, si
\begin{align*}
  H(s) & = K & G(s) = \frac{1}{1+s\tau}
\end{align*}
où $\tau > 0$, on a
\[ T(s) = \frac{K}{K + 1 + s\tau}. \]
On avait pas de problème de stabilité avec $-\frac{1}{\tau}$ car sa partie
réelle était strictement négative.
Mais on a maitenant une racine de $-\frac{1+K}{\tau}$,
Il faut donc que $K \geq -1$ pour rester stable.

%Pour une poly du second degré, tous les coef positif est une ssi de stabilité.
%Pour les + que second, c'est de nécessaire.

\annexe
\section{Produit de convolution}
Le produit de convolution est un produit qui se note $*$ et qui se fait
soit entre deux fonction discrète $a[n]$ et $b[n]$
\[ a[n] * b[n] = \sum_{k=-\infty}^{+\infty} a[k]b[n-k], \]
soit entre deux fonction continues $a(t)$ et $b(t)$
\[ a(t) * b(t) = \int_{\tau=-\infty}^{+\infty} a(\tau)b(t-\tau) \dif \tau. \]

\subsection{Propriétés}
Le produit de convolution a beaucoup de propriétés en commun avec le produit
normal, il est associatif, commutatif et distributif avec l'addition.

L'absorbant est aussi 0 mais le neutre n'est pas 1 mais
$\delta(t)$ (resp. $\delta[n]$) en continu (resp. en discret).
Plus généralement, le $\delta$ permet de décaller un signal
\[ x(t) * \delta(t - t_0) = x(t - t_0). \]

\section{Échelon et Impulsion}
\subsection{Échelon}
Un échelon se note $u(t)$ et est défini par
\[ u(t) = \begin{cases}
    0 & t < 0\\
    1 & t \geq 0
\end{cases}. \]

\subsection{Impulsion}
\label{app:dirac}
Une impulsion de Dirac est une fonction avec singularité en 0.
Elle se note $\delta(t)$ et est définie par
\begin{equation}
  \label{eq:dirac1}
  \delta(t) =
  \begin{cases}
    \infty, & t = 0\\
    0, & t \neq 0
  \end{cases}
\end{equation}
et
\begin{equation}
  \label{eq:dirac2}
  \int_{-\infty}^{\infty} \delta(t) \dif t = 1.
\end{equation}

Ça peut paraitre bizarre que l'intégrale soit finie alors que
$\delta(0) = \infty$.
Seulement, rappelez vous qu'une intégrale c'est une aire et que
l'aire d'un Dirac n'est pas entièrement définie par \eqref{eq:dirac1}
car celle ci impose que l'aire vaut $0 \cdot \infty$ qui est un cas
d'indétermination.
\eqref{eq:dirac2} permet de lever cette indétermination.

\subsection{Relations intéressantes}
\begin{align}
  \nonumber
  \delta(t) & = \fdif{u(t)}{t}\\
  \label{eq:uintdelta}
  u(t) & = \int_{-\infty}^t \delta(t) \dif t\\
  \nonumber
  f(t_0) & = \int_{-\infty}^\infty f(t)\delta(t-t_0) \dif t\\
  \label{eq:infinf00}
  f(t) \delta(t-t_0) & = f(t_0)\delta(t-t_0)
\end{align}
La \eqref{eq:infinf00} est vrai car si on a $t \neq t_0$, on a $0 = 0$
et si on a $t = t_0$, on a $\infty = \infty$.

\section{Lien entre matrice de commandabilité et commandabilité}
\label{ann:command}
Soit un état $q_0$, on a
\begin{align*}
  q[1] & = A q_0 + B x[0]\\
  \vdots & = \vdots\\
  q[n] & = A^nq_0 + A^{n-1}Bx[0] + A^{n-2}Bx[1] + \ldots + Bx[n-1]\\
\end{align*}
Si on veut $q[n] = 0$, on doit avoir
\[ -A^nq_0 = A^{n-1}Bx[0] + \ldots + Bx[n-1]. \]
Pour que le système soit commandable,
il faut donc que pour tout $q_0$, $-A^nq_0$ soit une combinaison
linéaire des colonnes de $\begin{pmatrix}A^{n-1}B & \cdots & B\end{pmatrix}$.
Il suffit donc % faut ?
que cette matrice soit de rang plein.

\end{document}
