\documentclass[11pt,a4paper]{article}

% French
\usepackage[utf8x]{inputenc}
\usepackage[frenchb]{babel}
\usepackage[T1]{fontenc}
\usepackage{lmodern}
\usepackage{ifthen}

% Color
% cfr http://en.wikibooks.org/wiki/LaTeX/Colors
\usepackage{color}
\usepackage[usenames,dvipsnames,svgnames,table]{xcolor}
\definecolor{dkgreen}{rgb}{0.25,0.7,0.35}
\definecolor{dkred}{rgb}{0.7,0,0}

% Floats and referencing
\newcommand{\sectionref}[1]{section~\ref{sec:#1}}
\newcommand{\annexeref}[1]{annexe~\ref{ann:#1}}
\newcommand{\figuref}[1]{figure~\ref{fig:#1}}
\newcommand{\tabref}[1]{table~\ref{tab:#1}}
\usepackage{xparse}
\NewDocumentEnvironment{myfig}{mm}
{\begin{figure}[!ht]\centering}
{\caption{#2}\label{fig:#1}\end{figure}}

% Listing
\usepackage{listings}
\lstset{
  numbers=left,
  numberstyle=\tiny\color{gray},
  basicstyle=\rm\small\ttfamily,
  keywordstyle=\bfseries\color{dkred},
  frame=single,
  commentstyle=\color{gray}=small,
  stringstyle=\color{dkgreen},
  %backgroundcolor=\color{gray!10},
  %tabsize=2,
  rulecolor=\color{black!30},
  %title=\lstname,
  breaklines=true,
  framextopmargin=2pt,
  framexbottommargin=2pt,
  extendedchars=true,
  inputencoding=utf8x
}

\newcommand{\matlab}{\textsc{Matlab}}
\newcommand{\octave}{\textsc{GNU/Octave}}
\newcommand{\qtoctave}{\textsc{QtOctave}}
\newcommand{\oz}{\textsc{Oz}}
\newcommand{\java}{\textsc{Java}}
\newcommand{\clang}{\textsc{C}}
\newcommand{\keyword}{mot clef}

% Math symbols
\usepackage{amsmath}
\usepackage{amssymb}
\usepackage{amsthm}
\DeclareMathOperator*{\argmin}{arg\,min}
\DeclareMathOperator*{\argmax}{arg\,max}

% Sets
\newcommand{\Z}{\mathbb{Z}}
\newcommand{\R}{\mathbb{R}}
\newcommand{\Rn}{\R^n}
\newcommand{\Rnn}{\R^{n \times n}}
\newcommand{\C}{\mathbb{C}}
\newcommand{\K}{\mathbb{K}}
\newcommand{\Kn}{\K^n}
\newcommand{\Knn}{\K^{n \times n}}

% Chemistry
\newcommand{\std}{\ensuremath{^{\circ}}}
\newcommand\ph{\ensuremath{\mathrm{pH}}}

% Theorem and definitions
\theoremstyle{definition}
\newtheorem{mydef}{Définition}
\newtheorem{mynota}[mydef]{Notation}
\newtheorem{myprop}[mydef]{Propriétés}
\newtheorem{myrem}[mydef]{Remarque}
\newtheorem{myform}[mydef]{Formules}
\newtheorem{mycorr}[mydef]{Corrolaire}
\newtheorem{mytheo}[mydef]{Théorème}
\newtheorem{mylem}[mydef]{Lemme}
\newtheorem{myexem}[mydef]{Exemple}
\newtheorem{myineg}[mydef]{Inégalité}

% Unit vectors
\usepackage{esint}
\usepackage{esvect}
\newcommand{\kmath}{k}
\newcommand{\xunit}{\hat{\imath}}
\newcommand{\yunit}{\hat{\jmath}}
\newcommand{\zunit}{\hat{\kmath}}

% rot & div & grad & lap
\DeclareMathOperator{\newdiv}{div}
\newcommand{\divn}[1]{\nabla \cdot #1}
\newcommand{\rotn}[1]{\nabla \times #1}
\newcommand{\grad}[1]{\nabla #1}
\newcommand{\gradn}[1]{\nabla #1}
\newcommand{\lap}[1]{\nabla^2 #1}


% Elec
\newcommand{\B}{\vec B}
\newcommand{\E}{\vec E}
\newcommand{\EMF}{\mathcal{E}}
\newcommand{\perm}{\varepsilon} % permittivity

\newcommand{\bigoh}{\mathcal{O}}
\newcommand\eqdef{\triangleq}

\DeclareMathOperator{\newdiff}{d} % use \dif instead
\newcommand{\dif}{\newdiff\!}
\newcommand{\fpart}[2]{\frac{\partial #1}{\partial #2}}
\newcommand{\ffpart}[2]{\frac{\partial^2 #1}{\partial #2^2}}
\newcommand{\fdpart}[3]{\frac{\partial^2 #1}{\partial #2\partial #3}}
\newcommand{\fdif}[2]{\frac{\dif #1}{\dif #2}}
\newcommand{\ffdif}[2]{\frac{\dif^2 #1}{\dif #2^2}}
\newcommand{\constant}{\ensuremath{\mathrm{cst}}}

% Numbers and units
\usepackage[squaren, Gray]{SIunits}
\usepackage{sistyle}
\usepackage[autolanguage]{numprint}
%\usepackage{numprint}
\newcommand\si[2]{\numprint[#2]{#1}}
\newcommand\np[1]{\numprint{#1}}

\newcommand\strong[1]{\textbf{#1}}
\newcommand{\annexe}{\part{Annexes}\appendix}

% Bibliography
\newcommand{\biblio}{\bibliographystyle{plain}\bibliography{biblio}}

\usepackage{fullpage}
% le `[e ]' rend le premier argument (#1) optionnel
% avec comme valeur par défaut `e `
\newcommand{\hypertitle}[7][e ]{
\usepackage{hyperref}
{\renewcommand{\and}{\unskip, }
\hypersetup{pdfauthor={#6},
            pdftitle={Synth\`ese d#1#2 Q#3 - L#4#5},
            pdfsubject={#2}}
}

\title{Synth\`ese d#1#2 Q#3 - L#4#5}
\author{#6}

\begin{document}

\ifthenelse{\isundefined{\skiptitlepage}}{
\begin{titlepage}
\maketitle

 \paragraph{Informations importantes}
   Ce document est grandement inspiré de l'excellent cours
   donné par #7 à l'EPL (École Polytechnique de Louvain),
   faculté de l'UCL (Université Catholique de Louvain).
   Il est écrit par les auteurs susnommés avec l'aide de tous
   les autres étudiants, la vôtre est donc la bienvenue.
   Il y a toujours moyen de l'améliorer, surtout si le cours
   change car la synthèse doit alors être modifiée en conséquence.
   On peut retrouver le code source à l'adresse suivante
   \begin{center}
     \url{https://github.com/Gp2mv3/Syntheses}.
   \end{center}
   On y trouve aussi le contenu du \texttt{README} qui contient de plus
   amples informations, vous êtes invité à le lire.

   Il y est indiqué que les questions, signalements d'erreurs,
   suggestions d'améliorations ou quelque discussion que ce soit
   relative au projet
   sont à spécifier de préférence à l'adresse suivante
   \begin{center}
     \url{https://github.com/Gp2mv3/Syntheses/issues}.
   \end{center}
   Ça permet à tout le monde de les voir, les commenter et agir
   en conséquence.
   Vous êtes d'ailleurs invité à participer aux discussions.

   Vous trouverez aussi des informations dans le wiki
   \begin{center}
     \url{https://github.com/Gp2mv3/Syntheses/wiki}.
   \end{center}
   comme le statut des synthèses pour chaque cours
   \begin{center}
     \url{https://github.com/Gp2mv3/Syntheses/wiki/Status}.
   \end{center}
   vous pouvez d'ailleurs remarquer qu'il en manque encore beaucoup,
   votre aide est la bienvenue.

   Pour contribuer au bug tracker et au wiki, il vous suffira de
   créer un compte sur Github.
   Pour interagir avec le code des synthèses,
   il vous faudra installer \LaTeX.
   Pour interagir directement avec le code sur Github,
   vous devez utiliser \texttt{git}.
   Si cela pose problème,
   nous sommes évidemment ouverts à des contributeurs envoyant leurs
   changements par mail ou n'importe quel autre moyen.
\end{titlepage}
}{}

\ifthenelse{\isundefined{\skiptableofcontents}}{
\tableofcontents
}{}
}


\lstset{language={C}}

\hypertitle{Syst\`emes informatiques}{4}{SINF}{1252}
{Beno\^it Legat}{Beno\^it Legat}

\part{Le language \clang{}}
\section{Normes}
Depuis sa création par Dennis Ritchie,
le \clang{} a pas mal évolué.
Il existe plusieurs normes dont la norme \clang99 qui permet
de faire pas mal de nouvelles choses.

Pour demander à \lstinline[language=bash]|gcc|
de compiler du code avec \clang99,
il faut lui donner l'argument \lstinline[language=bash]|--std=c99|.

\section{Les pointeurs}
En \clang{}, les variables sont passées par \emph{valeur},
pour partager une variable sans la copier, on utilise les pointeurs.
Un pointeur contient l'adresse \emph{virtuelle} d'un endroit en mémoire.
\begin{lstlisting}
int n = 0;
int *i = &n;
printf("%d == n\n", *i, n); // imprime 0 == 0
\end{lstlisting}
On doit aussi donner ce vers quoi point l'adresse (ici, un \lstinline|int|)
pour permettre au compilateur d'effectuer de l'arithmétique de pointeur.
En effet, lorsqu'on fait
\begin{lstlisting}
int *p = 0x10;
p++;
\end{lstlisting}
\lstinline|p| n'est pas incrémenté de 1 mais de la taille d'un
\lstinline|int|, c'est à dire \lstinline|sizeof(int)|.

Par convention, \lstinline|NULL| est la valeur d'un pointeur qui
ne pointe pas encore vers de la mémoire allouée.
Il donc mieux toujours donner cette valeur à un pointeur qui ne pointe
pas vers quelque chose de telle sorte que
\begin{lstlisting}
if (p != NULL) {
  free(p);
}
\end{lstlisting}
ne libère jamais de la mémoire non-allouée

\subsection{Le heap}
Les pointeurs permettent aussi de s'allouer de la mémoire dans la \emph{heap}.
La heap est un endroit de la mémoire où on peut s'allouer dynamiquement
de la mémoire.
C'est à dire s'allouer un nombre de bytes
qu'on ne connait pas à la compilation.

Pour cela, on utilise la fonction \lstinline|malloc| qui alloue de la mémoire
et la fonction \lstinline|free| qui la libère.
En effet, si on ne libère jamais la mémoire qu'on s'alloue dans la heap,
elle risque d'être remplie et \lstinline|malloc| renverra alors
\lstinline|NULL|.
\lstinline|malloc| renvoie un pointeur de type \lstinline|void*|
pour rester générique.
Il faut le caster dans le type voulu.
\begin{lstlisting}
int n;
printf("Combien voulez-vous de int (%d bytes par int) ? : ",
       sizeof(int));
scanf("%d", &n);
// On s'alloue n int consecutifs dans la heap
int *p = (int*) malloc(n * sizeof(int));
if (p == NULL) {
  // n est negatif, nul ou trop grand
  printf("Vous avez ete un peut trop optimiste il me semble :)\n");
}
// utilisation de p
// On n'oublie pas de liberer la memoire
free(p);
p = NULL;
\end{lstlisting}

\lstinline|malloc| ne met pas de valeur dans la mémoire qu'elle alloue
et \lstinline|free| ne change pas la valeur de \lstinline|p|
(d'ailleurs il en est incapable car on ne lui a pas passé un pointeur
vers \lstinline|p|).

Il existe aussi la fonction \lstinline|calloc| qui contrairement à
\lstinline|malloc| initialise tous les bytes à 0.
\lstinline|calloc| est donc moins rapide que \lstinline|malloc| mais
met tous les bytes à 0 de manière très efficace.
\begin{lstlisting}
#define N 42
int *p = (int*) malloc(N * sizeof(int));
for (int i = 0; i < N; i++) {
  p[i] = 0;
}
// est plus lent que
int *p = (int*) calloc(N, sizeof(int));
\end{lstlisting}

\section{Structures}
On peut créer des structures qui des une sorte de classes sans méthodes.
On accède aux éléments à l'aide de \lstinline|.|, \lstinline|a.b|
est l'élément au label \lstinline|b| dans la variable \lstinline|a|.
On a un raccourcis pour \lstinline|(*a).b| qui est \lstinline|a->b|.
\begin{lstlisting}
struct point {
  int x;
  int y;
}
int move (struct point *p, struct point delta) {
  (*p).x += delta.x;
  p->y += delta.y;
}
int main () {
  struct point d;
  d.x = 1;
  d.y = 2;
  struct point *triangle = (struct point *) malloc(3 * sizeof(struct point));
  printf("Before\n");
  for (int i = 0; i < 3; i++) {
    triangle[i].x = i;
    triangle[i].x = i+1;
    printf("(%d, %d)\n", triangle[i].x, triangle[i].y);
  }
  printf("After\n");
  for (int i = 0; i < 3; i++) {
    move(triangle[i], d);
    printf("(%d, %d)\n", triangle[i].x, triangle[i].y);
  }
  free(triangle);
  return 0;
}
\end{lstlisting}

\section{Fonctions}
En \clang{}, les variables sont passées par \emph{valeur}.
Il n'y a jamais de passage par référence comme en \java{} pour les objets
ou comme pour le \oz{}.
Par exemple,
\begin{lstlisting}
void inc1 (int a) {
  a++;
}
void inc2 (int *b) {
  (*b)++;
}
void foo () {
  int n = 0;
  inc1(n);
  printf("%d\n", n); // affiche 0
  inc2(&n);
  printf("%d\n", n); // affiche 1
}
\end{lstlisting}
comme on fait un passage par valeur,
\lstinline|a| vaut 0 et changer \lstinline|a| ne change pas \lstinline|n|.
Par contre, \lstinline|b| contient l'adresse de \lstinline|n| donc modifier
\lstinline|*b| modifie bien \lstinline|n|.
C'est pour ça que pour \lstinline|scanf|,
il faut passer l'adresse de la variable qui doit stocker le résultat.

\section{Opérateurs}
\subsection{Opérateurs logiques}
Les opérateurs de bits s'exécutent sur chaque bits.
On a
\begin{center}
  \begin{tabular}{cccc}
    \clang{} bit & \clang{} \lstinline|bool| & Logique & Appelation\\
    \& & \&\& & $\land$ & and\\
    | & || & $\lor$ & or\\
    $\hat{}$ &  & $\oplus$ & xor
  \end{tabular}
\end{center}

\section{Portée des variables}
En \clang{}, il n'y a pas de classe,
les variables sont soient \emph{locales}, soient \emph{globales}.

\subsection{Variables globales}
Les variables globales sont initialisées à 0 ou NULL pour les pointeurs.
Elles sont visibles partout.
Elles sont déclarées à l'extérieur de toute fonction.

\subsection{Variables locales}
Elles ne sont pas initialisées.
Elles sont visibles uniquement à l'intérieur de la fonction à l'intérieur
de laquelle elles sont créées.
Si elles ont le même nom qu'une variable globales,
la variable globale ne sera pas disponible à l'intérieur de la fonction.
Contrairement aux variables globales, ces variables ne sont pas
initialisées.

\subsection{Variables de boucle}
À partir de la norme \clang99,
on peut déclarer des variables à l'intérieur
de n'importe quel bloc
\begin{lstlisting}
int main () {
  {
    int i = 0;
  }
  i = 1; // ne compile pas
  return 0;
}
\end{lstlisting}
ce qui permet d'écrire
\begin{lstlisting}
int main () {
  int sum = 0;
  for (int i = 0; i < 5; i++) {
    sum += i;
  }
  return 0;
}
\end{lstlisting}

%\begin{lstlisting}
%char *string;
%int* v = (int *) string;
%\end{lstlisting}

%\section{Arguments}
%Variables d'env.
%Arguments.
%Pour beaucoup d'OS, ils sont dans un endroit au dessus de la pile de taille
%fixe. Il y a une limite sur la longueur de tous les args et pas sur leur
%nombre souvent.
%
%struct:
%regarde le plus long et leur donne tous ça
%ex:
%char : 1
%int : 4
%short : 2
%4 + 4 + 4 = 12
%char : 1
%long : 8
%short : 2
%8 + 8 + 8 = 24
%char : 1
%long[3] : 8
%short : 2
%8 + 3*8 + 8 = 40
%char : 1
%char : 1
%short : 2
%4 WTF :o
%32 bits - paquets de 4
%64 bits - paquets de 8

%int pid = fork();
%-> -1 pour père => fils existe pas
%-> >0 -> père et fils a pid = pid
%-> 0 -> fils


%\n ne flush pas tout le temps !!!
%flush est plus sûr. Crazy :o

\end{document}
