\documentclass[11pt,a4paper]{article}

% French
\usepackage[utf8x]{inputenc}
\usepackage[frenchb]{babel}
\usepackage[T1]{fontenc}
\usepackage{lmodern}
\usepackage{ifthen}

% Color
% cfr http://en.wikibooks.org/wiki/LaTeX/Colors
\usepackage{color}
\usepackage[usenames,dvipsnames,svgnames,table]{xcolor}
\definecolor{dkgreen}{rgb}{0.25,0.7,0.35}
\definecolor{dkred}{rgb}{0.7,0,0}

% Floats and referencing
\newcommand{\sectionref}[1]{section~\ref{sec:#1}}
\newcommand{\annexeref}[1]{annexe~\ref{ann:#1}}
\newcommand{\figuref}[1]{figure~\ref{fig:#1}}
\newcommand{\tabref}[1]{table~\ref{tab:#1}}
\usepackage{xparse}
\NewDocumentEnvironment{myfig}{mm}
{\begin{figure}[!ht]\centering}
{\caption{#2}\label{fig:#1}\end{figure}}

% Listing
\usepackage{listings}
\lstset{
  numbers=left,
  numberstyle=\tiny\color{gray},
  basicstyle=\rm\small\ttfamily,
  keywordstyle=\bfseries\color{dkred},
  frame=single,
  commentstyle=\color{gray}=small,
  stringstyle=\color{dkgreen},
  %backgroundcolor=\color{gray!10},
  %tabsize=2,
  rulecolor=\color{black!30},
  %title=\lstname,
  breaklines=true,
  framextopmargin=2pt,
  framexbottommargin=2pt,
  extendedchars=true,
  inputencoding=utf8x
}

\newcommand{\matlab}{\textsc{Matlab}}
\newcommand{\octave}{\textsc{GNU/Octave}}
\newcommand{\qtoctave}{\textsc{QtOctave}}
\newcommand{\oz}{\textsc{Oz}}
\newcommand{\java}{\textsc{Java}}
\newcommand{\clang}{\textsc{C}}
\newcommand{\keyword}{mot clef}

% Math symbols
\usepackage{amsmath}
\usepackage{amssymb}
\usepackage{amsthm}
\DeclareMathOperator*{\argmin}{arg\,min}
\DeclareMathOperator*{\argmax}{arg\,max}

% Sets
\newcommand{\Z}{\mathbb{Z}}
\newcommand{\R}{\mathbb{R}}
\newcommand{\Rn}{\R^n}
\newcommand{\Rnn}{\R^{n \times n}}
\newcommand{\C}{\mathbb{C}}
\newcommand{\K}{\mathbb{K}}
\newcommand{\Kn}{\K^n}
\newcommand{\Knn}{\K^{n \times n}}

% Chemistry
\newcommand{\std}{\ensuremath{^{\circ}}}
\newcommand\ph{\ensuremath{\mathrm{pH}}}

% Theorem and definitions
\theoremstyle{definition}
\newtheorem{mydef}{Définition}
\newtheorem{mynota}[mydef]{Notation}
\newtheorem{myprop}[mydef]{Propriétés}
\newtheorem{myrem}[mydef]{Remarque}
\newtheorem{myform}[mydef]{Formules}
\newtheorem{mycorr}[mydef]{Corrolaire}
\newtheorem{mytheo}[mydef]{Théorème}
\newtheorem{mylem}[mydef]{Lemme}
\newtheorem{myexem}[mydef]{Exemple}
\newtheorem{myineg}[mydef]{Inégalité}

% Unit vectors
\usepackage{esint}
\usepackage{esvect}
\newcommand{\kmath}{k}
\newcommand{\xunit}{\hat{\imath}}
\newcommand{\yunit}{\hat{\jmath}}
\newcommand{\zunit}{\hat{\kmath}}

% rot & div & grad & lap
\DeclareMathOperator{\newdiv}{div}
\newcommand{\divn}[1]{\nabla \cdot #1}
\newcommand{\rotn}[1]{\nabla \times #1}
\newcommand{\grad}[1]{\nabla #1}
\newcommand{\gradn}[1]{\nabla #1}
\newcommand{\lap}[1]{\nabla^2 #1}


% Elec
\newcommand{\B}{\vec B}
\newcommand{\E}{\vec E}
\newcommand{\EMF}{\mathcal{E}}
\newcommand{\perm}{\varepsilon} % permittivity

\newcommand{\bigoh}{\mathcal{O}}
\newcommand\eqdef{\triangleq}

\DeclareMathOperator{\newdiff}{d} % use \dif instead
\newcommand{\dif}{\newdiff\!}
\newcommand{\fpart}[2]{\frac{\partial #1}{\partial #2}}
\newcommand{\ffpart}[2]{\frac{\partial^2 #1}{\partial #2^2}}
\newcommand{\fdpart}[3]{\frac{\partial^2 #1}{\partial #2\partial #3}}
\newcommand{\fdif}[2]{\frac{\dif #1}{\dif #2}}
\newcommand{\ffdif}[2]{\frac{\dif^2 #1}{\dif #2^2}}
\newcommand{\constant}{\ensuremath{\mathrm{cst}}}

% Numbers and units
\usepackage[squaren, Gray]{SIunits}
\usepackage{sistyle}
\usepackage[autolanguage]{numprint}
%\usepackage{numprint}
\newcommand\si[2]{\numprint[#2]{#1}}
\newcommand\np[1]{\numprint{#1}}

\newcommand\strong[1]{\textbf{#1}}
\newcommand{\annexe}{\part{Annexes}\appendix}

% Bibliography
\newcommand{\biblio}{\bibliographystyle{plain}\bibliography{biblio}}

\usepackage{fullpage}
% le `[e ]' rend le premier argument (#1) optionnel
% avec comme valeur par défaut `e `
\newcommand{\hypertitle}[7][e ]{
\usepackage{hyperref}
{\renewcommand{\and}{\unskip, }
\hypersetup{pdfauthor={#6},
            pdftitle={Synth\`ese d#1#2 Q#3 - L#4#5},
            pdfsubject={#2}}
}

\title{Synth\`ese d#1#2 Q#3 - L#4#5}
\author{#6}

\begin{document}

\ifthenelse{\isundefined{\skiptitlepage}}{
\begin{titlepage}
\maketitle

 \paragraph{Informations importantes}
   Ce document est grandement inspiré de l'excellent cours
   donné par #7 à l'EPL (École Polytechnique de Louvain),
   faculté de l'UCL (Université Catholique de Louvain).
   Il est écrit par les auteurs susnommés avec l'aide de tous
   les autres étudiants, la vôtre est donc la bienvenue.
   Il y a toujours moyen de l'améliorer, surtout si le cours
   change car la synthèse doit alors être modifiée en conséquence.
   On peut retrouver le code source à l'adresse suivante
   \begin{center}
     \url{https://github.com/Gp2mv3/Syntheses}.
   \end{center}
   On y trouve aussi le contenu du \texttt{README} qui contient de plus
   amples informations, vous êtes invité à le lire.

   Il y est indiqué que les questions, signalements d'erreurs,
   suggestions d'améliorations ou quelque discussion que ce soit
   relative au projet
   sont à spécifier de préférence à l'adresse suivante
   \begin{center}
     \url{https://github.com/Gp2mv3/Syntheses/issues}.
   \end{center}
   Ça permet à tout le monde de les voir, les commenter et agir
   en conséquence.
   Vous êtes d'ailleurs invité à participer aux discussions.

   Vous trouverez aussi des informations dans le wiki
   \begin{center}
     \url{https://github.com/Gp2mv3/Syntheses/wiki}.
   \end{center}
   comme le statut des synthèses pour chaque cours
   \begin{center}
     \url{https://github.com/Gp2mv3/Syntheses/wiki/Status}.
   \end{center}
   vous pouvez d'ailleurs remarquer qu'il en manque encore beaucoup,
   votre aide est la bienvenue.

   Pour contribuer au bug tracker et au wiki, il vous suffira de
   créer un compte sur Github.
   Pour interagir avec le code des synthèses,
   il vous faudra installer \LaTeX.
   Pour interagir directement avec le code sur Github,
   vous devez utiliser \texttt{git}.
   Si cela pose problème,
   nous sommes évidemment ouverts à des contributeurs envoyant leurs
   changements par mail ou n'importe quel autre moyen.
\end{titlepage}
}{}

\ifthenelse{\isundefined{\skiptableofcontents}}{
\tableofcontents
}{}
}

\usepackage{multirow}
\usepackage[version=3]{mhchem}

\usepackage{layout}

\hypertitle{Chimie organique}{4}{MAPR}{1230}
{Damien Hoedenaeken}
{Sophie Demoustier et Benjamin Elias}

\part{Notions fondamentales}
\section{Structure atomique, symbolisme de l'atome et isotope}
Le noyau est constitué de particules chargées positivement (protons) et de particules neutres (neutrons).
Le nuage électronique est constitué de particules chargées négativement (électrons).
Pour décrire un élément chimique, on a
\begin{itemize}
  \item son numéro atomique, symbolisé par la lettre Z, qui indique le nombre de protons compris dans le noyau.
    Un atome est électriquement neutre donc le nombre d'électrons est aussi Z.
  \item le nombre masse, symbolisé par la lettre A, qui correspond à la somme du nombre de neutrons et du nombre de protons.
\end{itemize}
Les isotopes sont des atomes d'un même élément qui possèdent le même nombre de protons, mais un nombre différent de neutrons.
\section{Orbitales atomiques}
Les orbitales atomiques consistent en des régions de l'espace autour du noyau atomique où la probabilité de trouver un électron est de 90$\%$.

Elles sont caractérisées par 4 nombres quantiques distincts
\begin{itemize}
  \item le nombre quantique principal n
  \item le nombre quantique secondaire l
    \begin{itemize}
      \item si l=0, ce sont des orbitales s sphériques.
      \item si l=1, ce sont des orbitales p bilobées perpendiculaires les unes par rapport aux autres.
    \end{itemize}
  \item le nombre quantique magnétique $m_l$
  \item le nombre quantique de spins $m_s$
\end{itemize}

\paragraph{Notion de Lewis d'un élément donné}
La notion de Lewis consiste à répartir les électrons de valences sur les 4 cotés du symbole chimique.

\section{Règle de l'octet}
En chimie, le chiffre 8 est le symbole de la stabilité puisqu'il faut huit électrons pour combler un niveau électronique.
Deux électrons dans l'orbitale s et six électrons dans l'orbitale  p.
Les atomes tenteront donc d'adopter la configuration électronique stable du gaz rare le plus près, et donc d'acquérir l'octet sur la couche de valence.
Il existe des exceptions lorsqu'on atteint les éléments des périodes plus élevées et où il y a la présence d'orbitales d.

L'atome qui cède ses électrons devient un cation.

L'atome qui capte des électrons devient un anion.


\section{Différents types de liaisons chimiques}
Avant de distinguer les liaisons chimiques covalentes des liaisons ioniques, on a recours à la différence d'électronégativité entre les atomes impliqués dans la liaison.
L'électronégativité est la tendance d'un atome à vouloir attirer vers lui les électrons d'une liaison chimique.
Les liaisons métalliques ne sont pas traitées dans cette section.
Pour que l'octet soit respecté, il y a parfois des liaisons doubles ou triples.

\subsection{La liaison covalente}

\paragraph{Liaison covalente non polaire}, elle implique un partage égal des électrons entre les atomes dans une liaison chimique.
Le nuage électronique caractéristique de la liaison est symétrique.
Si la différence d'électronégativité entre deux atomes se trouve entre 0 et 0.40, il s'agit d'une liaison covalente non polaire.
Et lorsque la différence égale zéro, la liaison est parfaitement non polaire.

\paragraph{Liaison covalente polaire}, elle est formée par des atomes qui se partagent inégalement les électrons.
Pour dire qu'une liaison est covalente polaire, il faut que la différence d'électronégativité entre les deux atomes qui la composent ait une valeur entre 0.4 et 1.7.

\subsection{La liaison ionique}
On parle plutôt ici d'un transfert complet d'électrons.
Lorsque la différence d'électronégativité entre les deux atomes est supérieure à 1.7, la liaison entre ces deux atomes possède 51$\%$ et plus de caractères ioniques.
Les deux ions demeurent à proximité en raison d'une attraction anion-cation que l'on appelle attraction électrostatique.
\section{Charge formelle}
La charge formelle fait le bilan du nombre total d'électrons qui appartiennent à un atome par rapport à son nombre d'électrons de valence.
\\
Charge formelle  = Nbe $e^-_{ valence }$ atome neutre - (Nbe $e^-_{non partagé}$  + Nbe de laisons).
\section{Hybridation}
Les orbitales hybrides seront de forme et d'énergie différentes de celles des orbitales atomiques de départ.
De plus, le niveau d'énergie de ces orbitales hybrides sera intermédiaire, plus élevé que l'orbitale s mais moins élevé qu'une orbitale p.

\subsection{Hybridation $SP^3$}
Le symbole $SP^3$ provient du nombre d'orbitales de chaque type impliquées dans l'hybridation.
On a donc une orbitale s et trois orbitales p qui se mélangent pour donner quatre orbitales $SP^3$.
Pour minimiser la répulsion, les quatre orbitales adoptent une géométrie tétraédrique avec un angle de $109.5\degres$.
Il n' y a que des liaisons sigma, et  ce type de recouvrement axial permet une rotation libre autour de la liaison.

\subsection{Hybridation $SP^2$}
Lorsqu'un atome est entouré de trois groupements d'électrons, l'hybridation d'une orbitale s et de deux orbitales p doit avoir lieu.
La combinaison de ces trois orbitales mène à la formation de trois orbitales $SP^2$ de même forme et de même énergie mais d'orientations différentes.
Pour minimiser la répulsion électronique, l'angle entre les orbitales est de $120\degres$.
Le quatrième électron de valence se trouve dans une orbitale p non hybridée et qui est perpendiculaire au trois autres orbitales.
Il y a une liaison sigma et une pi.
Pour le lien pi, il y a recouvrement latéral entre les deux orbitales p parallèles.
La rotation libre autour d'une double liaison est bloquée.

\subsection{Hybridation $SP$}
Lorsqu'un atome est entouré de deux groupements d'électrons, l'hybridation d'une orbitale s et d'une orbitale p doit avoir lieu.
La combinaison de ces deux orbitales mène à la formation de deux orbitales $SP$ de même forme et de même énergie mais d'orientations différentes.
Pour minimiser la répulsion électronique, l'angle entre les orbitales est de $180\degres$.
Ici, il y a une liaison sigma et deux liaisons pi.

\section{Attractions intermoléculaires}
Dans un échantillon, il y a beaucoup de molécules et elle interagissent continuellement entre elles grâce aux attractions intermoléculaires.

\subsection{Forces de Van de Waals}Elles peuvent être divisées en trois sections selon que les dipôles sont permanents, instantanés ou induits.

\paragraph{Forces de dispersion de London}
Elles sont générées lorsqu'il y a création de dipôles instantanés due au mouvement aléatoire des électrons.
Il s'ensuit une induction formant alors des dipôles induits.
Plus une molécule est grosse, plus elle est polarisable et plus les forces de dispersion de London seront fortes.
\paragraph{Les interactions de Debye}
Elles sont similaires à celles de London mais sont créées par un dipôle permanent d'une molécule polaire.
\paragraph{Les interactions de Keesom}
Elles ont lieu lorsque deux dipôles permanents interagissent entre eux par des forces électrostatiques.


\subsection{Ponts hydrogène}
Les ponts hydrogène sont en fait des interactions de Keesom spécifiques puisqu'il s'agit de dipôles forts et permanents.
En effet, pour que ce type d'interactions électrostatiques fort se produise, une molécule doit posséder un élément très électronégatif (N, O ou F) et une molécule voisine doit renfermer un hydrogène directement lié à un de ces éléments très électronégatifs.






\part{\'Ecriture spécifique de la chimie organique}
On définit aujourd'hui la chimie organique comme la chimie du carbone.
Par convention, on exclut toutefois des composés organiques les oxydes de carbone, les carbonates, les cyanures, les carbures et les solides covalents tels que le diamant et le graphite.

\section{\'Ecriture des formules structurales}
La représentation des molécules peut être diversifiée
\begin{description}
  \item[La forme moléculaire] ce type d'écriture indique les atomes présents dans la molécule ainsi que le nombre réel de chacun.
  \item[Les formules développées] elles mettent en évidence chacune des liaisons formées lors de l'union des atomes.
    Elles n'illustrent toutefois pas la véritable structure tridimensionnelle des molécules.
  \item[La formule simplifiée ou stylisée] dans ces formules, il est sous-entendu que chaque extrémité d'une ligne et chaque intersection entre deux segments comportent un atome de carbone.
\end{description}

\section{Classification selon la charpente moléculaire}
La chaine carbonée la plus longue porte le nom de chaine principale alors que les embranchements sont appelés des ramifications ou substituants.
\subsection{Composés acycliques} Les molécules organiques acycliques sont constituées de chaines d'atomes de carbone, mais elles n'ont aucun cycle.
La chaine principale de ce type de molécule peut porter, ou non, une ou des ramifications.
\subsection{Composés carbocycliques ou homocycliques} Les composés carbocycliques ou homocycliques contiennent des cycles constitués exclusivement d'atomes de carbone.
Le plus petit cycle carbocyclique est formé de trois atomes de carbone.
Les cycles à cinq ou six carbones sont les plus répandus.
\subsection{Composés hétérocycliques} Dans ces derniers, au moins un atome du cycle doit être un hétéroatome, soit un atome autre que le carbone ou l'hydrogène.
Les hétéroatomes les plus courants sont l'oxygène, l'azote et le soufre.
\section{Classification des composés organiques selon les fonctions}
Certains groupes présentent des propriétés chimiques spécifiques qui dépendent très peu du squelette carboné auquel ils sont fixés.
\\
\\
\begin{tabular}{|l|c|l|}
  \hline
  Absence de groupes fonctionnels & alcane & éthane \\
  \hline
  \multirow{3}*{Groupes fonctionnels des liaisons pi} & alcène & éthylène \\
  \cline{2-3}
  & alcyne & alcyne \\
  \cline{2-3}
  & composés aromatiques & benzène \\
  \hline
  liaison simple de type carbone-halogène & composés halogénés & chlorure de méthyle\\
  \hline
  \multirow{4}*{liaison simple de type carbone-oxygène } & alcool & éthanol \\
  \cline{2-3}
  & phénols & vaniline \\
  \cline{2-3}
  & éthers & éther diéthylique  \\
  \cline{2-3}
  & peroxydes & peroxyde d'hydrogène\\
  \hline
  liaison simple de type carbone-soufre & thiols & méthanethiol \\
  \hline
  \multirow{3}*{liaison simple de type carbone-azote} & amine primaire & \multirow{3}*{éthylamine} \\
  \cline{2-2}
  & amine secondaire & \\
  \cline{2-2}
  & amine tertiaire & \\
  \hline
  \multirow{5}*{liaisons multiples polaires de type carbone-oxygène} & aldéhydes & formaldéhyde \\
  \cline{2-3}
  & cétones & acétone \\
  \cline{2-3}
  & acide carboxylique &  acide acétique\\
  \cline{2-3}
  & esters & acétate d'éthyle  \\
  \cline{2-3}
  & anhydrides & anhydsride acétique  \\
  \hline
  liaisons multiples polaires de type carbone-halogène & halogénures d'acide & chlorure d'éthanoyle\\
  \hline
  \multirow{5}*{liaisons multiples polaires de type carbone-azote} & amide primaire &\multirow{3}*{ formamide} \\
  \cline{2-2}
  & amide secondaire & \\
  \cline{2-2}
  & amide tertiaire & \\
  \cline{2-3}
  & nitro & nitroglycérine \\
  \cline{2-3}
  & nitriles & acrylonitrile \\
  \hline
\end{tabular}
\\
\\

\section{Nomenclature des composés organiques}

Il existe un ordre de priorité dans la nomenclature des groupements fonctionnels.

\begin{tabular}{|l | l|}
  \hline
  \multicolumn{2}{|c|}{Ordre de priorité des groupement fonctionnels}\\
  \hline
  - Important & Alcanes et composés halogénés \\
              &   \'Ethers et thioéthers\\
              & Alcènes et alcynes \\
              & Amines \\
              & Alcools et thiols \\
              & Cétones \\
              & Aldéhydes \\
              & Nitriles \\
              & Amides \\
              & Esters \\
              & Halogénures d'acide \\
  + Important & Acides carboxyliques \\

  \hline
\end{tabular}



\subsection{Nomenclature des alcanes}
Puisqu'il s'agit d'alcanes, les noms des chaines principales des composés acycliques se terminent par le même suffixe, soit -ane.
Ainsi, le nombre de carbone de la chaine principale est désigné par le préfixe auquel on ajoute le suffixe -ane.

Pour la nomenclature des alcanes acycliques ramifiés, on fait
\begin{itemize}
  \item le numéro de position séparé par une virgule pour chaque ramification
  \item le préfixe multiplicatif + le nom de ramification (par ordre alphabétique, chaque ramification séparée par un tiret et la dernière ramification est fusionnée directement avec le nom de la chaine principale.
  \item de la chaine principale.

    \paragraph{Exemples }
    \begin{itemize}
      \item 2 - méthylpentane
      \item 3,4 - diéthyl-5-méthyldécane
    \end{itemize}
\end{itemize}

\paragraph{Nomenclature des alcanes cycliques}
Si la chaine principale liée au cycle contient plus de carbone que le cycle lui-même, celui-ci devient alors la ramification

\paragraph{Exemples}
\begin{itemize}
  \item 1 - éthyl-3-isopropylcyclopentane
  \item 4 - butyl-1-éthyl-2-méthylcyclohexane
  \item 1 - cyclobutylpentane
\end{itemize}

\paragraph{Nomenclature des composés halogénés}
Dans les composés halogénés, les halogènes sont traités comme s'ils étaient des substituants fixés au squelette carboné des alcanes.
On suit donc les mêmes règles que celles adoptées pour les ramifications des alcanes.

\paragraph{Exemples }
\begin{itemize}
  \item 1 - bromopropane
  \item 4 - bromo-1,2-diméthylcyclohexane
\end{itemize}

\subsection{Nomenclature des alcènes}
La chaine d'alcènes doit être la plus longue possible tout en incluant le maximum de liaisons doubles.
Le suffixe du nom de la chaine principale est -ène.
Un préfixe multiplicateur doit être ajouté au suffixe s'il y a plus d'une liaison double.
La numérotation de la chaine principale est réalisée en donnant le plus petit indice possible aux liaisons doubles puis aux substituants .

\paragraph{Exemples}
\begin{itemize}
  \item pent-2-ène
  \item 5,5-diméthylhex-2-ène
  \item hepta-2,5-diène
\end{itemize}

\subsection{Nomenclature des alcynes} La chaine principale doit inclure le maximum de liaisons multiples, tout en étant la plus longue  possible, même si elle ne renferme pas le nombre maximum de carbone.
Le suffixe du nom de la chaine principale est -yne.
Un préfixe multiplicateur doit être ajouté au suffixe s'il y a plus d'une liaison triple.
Lorsqu'il y a des liaisons doubles et triples, il faut leur accorder la même priorité et d'après l'ordre alphabétique, la terminaison de la chaine principale s'écrit comme suit -én suivi de -yne.

\paragraph{Exemples}
\begin{itemize}
  \item oct-3-yne
  \item hept-1-én-6-yne
  \item heta-1,3-dién-6-yne
\end{itemize}

\subsection{Nomenclature des éthers-oxydes} La fonction éther est celle d'un oxygène entre deux chaines de carbone.
Pour nommer l'éther, on doit déterminer la chaine la plus longue de part et d'autre de l'oxygène.
On utilise le suffixe -oxy et  -yloxy.

\paragraph{Exemples}
\begin{itemize}
  \item 2-éthoxy-3-méthylypentane
  \item 1,4 diméthoxypentane
  \item 6-chloro-4-pentyloxyhept-1-ène
  \item oxyde d'éthyle et de méthyle
\end{itemize}

\subsection{Nomenclature des alcools}
Les alcools sont des groupes fonctionnels qui ont priorité sont les groupes précédents et donc le plus petit indice doit être attribué au carbone portant le groupement -OH.
Dans le cas des alcools, le suffixe est -ol.

\paragraph{Exemples}
\begin{itemize}
  \item  propan-2-ol
  \item  prop-2-ène-1-ol
  \item  heptane-2,6-diol
  \item  2-methylheptan-3-ol
\end{itemize}


\subsection{Nomenclature des aldéhydes} La fonction terminale aldéhyde -CHO a priorité sur les alcools.
Ces derniers deviennent alors des substituants hydroxyl -OH.
Le suffixe est -al.

\paragraph{Exemples}
\begin{itemize}
  \item propanal
  \item but-3-énal
  \item pentanedial
\end{itemize}

\subsection{Nomenclature des cétones} La nomenclature est presque la même que celle des aldéhydes à une seule exception près, une cétone ne termine jamais une chaine principale.
Le préfixe est -one.
La priorité des cétones est moindre que celle des aldéhydes.

\paragraph{Exemples}
\begin{itemize}
  \item propane-2-one
  \item but-3-èn-2-one
\end{itemize}

\part{Isomèrie}

Les molécules dont la nature et le nombre d'atomes sont identiques mais dont la disposition des atomes entre eux est différente sont des isomères.
Il y a deux grandes structures, les isomères de structure et les stéréoisomères.

\section{Isomère de structure}
Les isomères de structure sont des composés qui partagent la même formule moléculaire, mais dont la formule développée est différente.
Les isomères de structure peuvent être subdivisés en deux catégories
\begin{description}
  \item[Les isomères de position] lorsque leurs groupements fonctionnels sont les mêmes bien que la formule développée soit différente.
  \item[Les isomères de fonction] lorsque leurs groupements fonctionnels sont différents.
\end{description}

\section{Détermination du degré d'insaturation}
Le degré d'instauration  correspond à la sommation du nombre de liaisons $\pi$ et de cycles présents dans une structure.
Pour parvenir à déterminer le degré d'insaturation d'une molécule, les quelques règles suivantes doivent être appliquées
\begin{itemize}
  \item Compter le nombre de carbone dans la molécule et déterminer le nombre d'hydrogène éventuellement présent s'il n'y avait aucune insaturation ni cycle.
  \item Soustraire le nombre réel d'hydrogène du nombre d'hydrogène calculé précédemment et  diviser ce nombre par deux.
  \item Si la molécule contient des hétéroatomes, le nombre réel d'hydrogène doit être modifié.
    \begin{itemize}
      \item La présence d'hétéroatomes monovalents implique qu'il faut ajouter un hydrogène au nombre réel d'hydrogène.
      \item La présence d'hétéroatomes divalents n'influe pas le degré d'insaturation.
      \item La présence d'hétéroatomes trivalents implique qu'il faut retirer un hydrogène du nombre réel d'hydrogène.
    \end{itemize}
\end{itemize}

\section{Représentation tridimensionnelle des hydrocarbures saturés}

\subsection{La conformation}
La conformation est la représentation tridimensionnelle due aux multiples rotations autour des axes des liaisons $\sigma$, les atomes peuvent occuper dans l'espace différentes positions à des moments précis.
Une infinité de conformations sont possibles pour une même molécule dessinée en trois dimensions.

Dans la conformation décalée, chaque liaison C-H de l'un des atomes de carbone forme un angle de $60\degres$ par rapport à la liaison C-H de l'autre carbone.
On parle d'angle dièdre.
Dans la conformation éclipsée,les liaisons C-H de l'atome de carbone de l'avant sont superposées avec les liaisons de l'atome de carbone de l'arrière.
L'angle de dièdre est donc de $0\degres$.
Les répulsions électroniques créées entre deux groupements sur les carbones de l'axe de liaison portent le nom d'encombrement stérique et quand elle se produit lorsque l'angle de dièdre est de $60\degres$, on appellera cette tension stérique, l'effet gauche.
La conformation décalée est plus stable tandis que la conformation éclipsée est la moins stable.

Les conformations correspondant à un minimum d'énergie potentielle, soit la conformation décalée partent le nom de conformère ou rotamère.


\section{Cycloalacanes et leurs conformations}
Les petits cycles et les très grands cycles sont peu abondants car il y a une tension de cycle qui lui confère une très grande réactivité.
Ils ont des petits/grands angles par rapport au $109.5\degres$ .
Un des cycles le plus abondant est le cyclohexane.
Si celui-ci était plan, ses angles internes seraient de $120\degres$ et donc la conformation favorisée est la conformation chaise où tous les angles C-C-C sont de $109.5\degres$ et tous les atomes hydrogène portés par les atomes de carbone adjacents sont décalés.
Dans la conformation chaise, les atomes d'hydrogène du cyclohexane sont soit axiaux soit équatoriaux.

Lors de l'interconversion chaise-chaise, les liaisons vers le bas d'une conformation demeurent vers le bas dans l'autre conformation et il en va de même pour les liaisons vers le haut.
Par contre, toutes les liaisons équatoriales dans une conformation deviennent des liaisons axiales dans une autre conformation et vice versa.

La \textbf{conformation favorisée} est celle dans laquelle le substituant volumineux est en position équatoriale.


\section{Stéréoisomérie}

La stéréochimie est le domaine de la chimie qui étudie les représentations tridimensionnelles des  molécules ainsi que les mécanismes de réaction en trois dimensions.

La stéréoisomérie se rapporte donc aux molécules dont la structure en deux dimensions est la même, mais dont l'arrangement spatial est différent.
Les stéréoisomères ne  s'interconvertissent que par un bris et une formation de liaisons covalentes.
La stéréochimie se subdivise en deux sous-groupes : les énantiomères et les isomères géométriques.

\subsection{Chiralité et énantiomère} Une molécule peut être chirale ou achirale.
Une structure chirale a la propriété d'être l'image spéculaire d'une autre molécule sans y être superposable.
Une molécule achirale ne possède pas cette propriété.
L'image spéculaire d'une molécule chirale et la molécule elle-même ne peuvent pas se superposer.
En revanche, une molécule achirale et son image spéculaire sont identiques et superposables.

Une paire de molécules apparentées, comme un objet et son image spéculaire non superposable, sont appelées des énantiomères.
Ces molécules ne possèdent pas de plan de symétrie.
Un mélange 50/50 de deux énantiomères est appelé un mélange racémique.

\subsection{Centre stéréogénique et atome de carbone stéréogénique}Un atome de carbone, hybridé $sp^3$, muni de quatre groupes différents est un atome de carbone stéréogénique (asymétrique).
Ce type de carbone est aussi appelé centre stéréogénique.

\subsection{Configuration absolue des carbones stéréogéniques et convention R-S} Pour chaque carbone stéréogénique $C^*$, deux configurations absolues sont toujours possibles.
Afin de déterminer la notation R ou S aux deux configurations possibles pour un carbone stéréogénique, il est important de déterminer adéquatement les quatre différents groupements entourant le $C^*$ et de les placer en priorité, 1 $\rightarrow$  2 $\rightarrow$ 3 $\rightarrow$ 4 selon les règles suivantes :
\begin{itemize}
  \item les atomes liés au carbone stéréogénique sont classés selon leur numéro atomique.
    L'atome portant le plus haut numéro est le plus important.
  \item si plusieurs atomes portent le même numéro atomique, alors il faut aller au $2^e$, au $3^e$, au $4^e$,etc atome jusqu'à l'obtention d'une différence.
  \item on traite les liaisons multiples comme si chacune était simple.
    Ainsi, la liaison multiple doit être traitée comme autant de liaisons simples et ce, de chaque coté de la liaison multiple.

\end{itemize}


Les numéros atomiques ne doivent jamais être additionnés !!!

L'oeil de l'observateur doit être situé du coté opposé au groupement ayant la plus petite priorité.
Celui-ci doit se situer derrière le plan.
Ensuite, les trois groupements restants doivent être joints selon l'ordre croissant de leur priorité.
Si la rotation se fait dans le sens horaire en joignant les numéros de 1 à 3, il s'agit de la configuration R.
Au contraire, si la rotation se fait dans le sens antihoraire, il s'agit de la configuration S.

Puisque les énantiomères sont des images spéculaires, les configurations absolues de leurs centres stéréogéniques sont toujours l'inverse l'une de l'autre.

\subsection{Lumière polarisée et activité optique}On place un échantillon dans le tube.
Si la substance est optiquement inactive, aucun changement ne se produit tandis que si une substance est optiquement active, elle dévie le plan de polarisation.
L'angle $\alpha$ est l'angle selon lequel le prisme doit être tourné.
Si l'analyseur doit être tourné vers la droite, dans le sens horaire, la substance optiquement active est définie comme dextrogyre sinon elle est lévogyre.

Quand on place une substance dans un polarimètre, on y introduit une très grande quantité de molécules.
Les molécules achirales sont optiquement inactives.
Cependant, dans le cas des molécules chirales, une paire d'énantiomères en proportion égale ne fera pas tourner un plan de polarisation.
En effet, lorsqu'on place un mélange 50/50 de deux énantiomères, l'activité optique sera nulle.
Ce mélange porte le nom de mélange racémique.

Lorsqu'une solution contient un seul des énantiomères, la solution est optiquement active.

\subsection{Propriétés des énantiomères} En général, la chiralité d'un objet joue un rôle très important quand cet objet interagit avec un autre objet chiral.
Les énantiomères partagent des propriétés identiques, comme les points de fusion, d'ébullition, la masse volumique, etc.
Mais ils présentent des propriétés optiques différentes.

\subsection{Projections de Fischer}
L'atome de carbone stéréogénique est omis et est simplement symbolisé par le point d'intersection entre les lignes horizontale et verticale.
La ligne horizontale relie le centre stéréogénique aux groupes qui se projettent hors du plan de la feuille, vers l'avant, alors que la ligne verticale lie les groupes qui s'enfoncent sous le plan de la feuille, vers l'arrière.
La chaine la plus longue de carbones doit être mise à la verticale en prenant soin de placer le carbone prioritaire en nomenclature au sommet.

\subsection{Relations des composés munis de plusieurs centres stéréogéniques : énantiomères, diastéréoisomères et composés méso} Dans le cas de molécules ayant plusieurs centres stéréogéniques, tous les centres doivent être inversés pour qu'il s'agisse d'une paire d'énantiomères.
Ce raisonnement ne peut toutefois être appliqué que si la molécule ne possède pas de plan de symétrie puisque dans un tel cas, un composé méso peut exister.

On emploie le terme de diastéréoisomère pour désigner deux structures qui sont des stéréoisomères, mais qui ne sont pas l'image spéculaire l'une de l'autre.
Ils ne sont pas image un l'un de l'autre dans un miroir.
Les diastéréoisomères peuvent différer par toutes les propriétés.

Si une molécule possède $n$ centres stéréogéniques différents, il existe au maximum $2^n$ stéréoisomères différents.

Le composé méso est un seul composé qui comporte plusieurs centres stéréogéniques, mais qui possède un plan de symétrie.


\section{Isomères géométriques}

L'isomérie géométrique est un type de stéréoisomérie possible si et seulement si la rotation est bloquée ou limitée à l'intérieur d'une molécule.
De plus, il faut que deux carbones du cycle, ou les deux carbones de la liaison double, soient porteurs de deux groupements différents.
\subsection{Isomères géométriques de type cis-trans}

On parle d'un isomère cis si et seulement si les groupements les plus volumineux portés par les deux carbones différents du plan du cycle ou les deux carbones de l'alcène sont placés du même coté du plan.
\`A l'inverse, s'ils sont situés de part et d'autre du plan, il s'agit d'un isomère trans.
\subsection{Isomères géométriques et la convention E-Z} Ici, la stratégie consiste à couper perpendiculairement la double liaison et à considérer alors les groupements différents liés sur chacun des carbones de la liaison double.
Il faut comparer les deux groupements liés au carbone de gauche et déterminer la priorité.
La priorité va à l'atome qui a le numéro atomique le plus élevé.
Le même raisonnement est réalisé au carbone de droite.

Si les deux groupes dont la priorité est la plus élevée sont situés de part et d'autre de la liaison double, le préfixe E est employé.
En revanche, s'ils se trouvent du même côté de la liaison double, on emploie le préfixe Z.

\part{Réactivité chimique}

\section{\'Equation chimique et introduction aux mécanismes réactionnels}
Un mécanisme réactionnel est une représentation des mouvements des électrons impliqués dans les transformations qui surviennent au cours de la réaction chimique.
Le mouvement des électrons est représenté à l'aide de flèches courbes de mécanisme.
Pour une flèche courbe, il y a deux électrons en jeu et pour une flèche courbe à demi-pointe, il y a 1 électron en jeu.

\paragraph{Pour rappel :}
\[ \nu_{a} \ce{A} + \nu_{b} \ce{B}
\leftrightharpoons \nu_{x} \ce{X} + \nu_{y} \ce{Y} \]



Pour une réactions en chimie organique :
$K_c =\frac{[X]^{\nu_x}[Y]^{\nu_y}}{[A]^{\nu_a}[B]^{\nu_b}}$.

Si > 1 la réaction est déplacé vers la droite.

Si < 1 la réaction est déplacé vers la gauche.

Tout modification d'un facteur influant sur les conditions d'équilibre d'un système le force à évoluer dans le sens qui réduit ou contrecarre l'effet de  ce changement.

Ajout de A = équilibre $\longrightarrow$


Ajout de X = équilibre $\longleftarrow$


Enlever un des produits = équilibre  $\longrightarrow$

Mais K ne change pas !!!!


\section{Quatre grandes catégories de réactions en chimie organique}
Les quatre grandes catégories de réactions chimiques organiques sont basées sur la manière dont les substrats sont transformés en produits.
On observe des réactions

\begin{itemize}
  \item \textbf{d'addition}
  \item \textbf{d'élimination}
  \item \textbf{de substitution}
  \item \textbf{de réarrangement}
\end{itemize}

\subsection{Réaction d'addition}
Elle implique qu'un ou plusieurs réactifs s'additionne(nt) sur le substrat pour donner un produit final.
Les substrats sont insaturés .
Ces réactions engendrent donc des produits finaux ayant perdu la ou les liaison(s) $\pi $ présente(s) dans le substrat à la suite de l'ajout d'un ou de deux réactifs .

\subsection{Réaction d'élimination}
Elle est l'inverse de la réaction d'addition.
Le substrat subit une perte d'atomes d'une portion de sa structure moléculaire.
Il s'en suit la formation d'une ou de plusieurs liaisons $\pi$.
Elle se déroule souvent en présence de chaleur.

\subsection{Réaction de substitution}
Dans celle-ci, un atome ou groupe d'atomes sur le substrat est remplacé par un autre atome ou groupement d'atomes.
Il existe des réactions de substitution radicalaire, de substitution électrophile et de substitution nucléophile.

\subsection{Réarrangement}
C'est la restructuration d'une molécule.
Il n'y a ni gain, ni perte d'atome.
Le substrat et le produit final sont des isomères de structure.

\section{Diagrammes énergétiques}

\subsection{Réactions simples effectuées en une seule étape} Si les produits sont plus stables que les réactifs de départ, la réaction est dite exothermique, ayant une variation d'entropie de réaction négative.
Au contraire, si les produits sont moins stables que les réactifs de départ, la réaction est dite endothermique et présentera une enthalpie positive.

L'état de transition est une zone haute en énergie où les liaisons sont en train de se briser et de nouvelles se forment.
L'énergie d'activation est l'énergie minimale que les molécules entrant en collision doivent posséder afin de rendre les collisions efficaces et qu'il y ait transformation des réactifs en produit.
Plus l'énergie est élevée, plus la réaction est difficile et s'effectue lentement.

Des catalyseurs chimiques sont souvent utilisés pour diminuer cette barrière énergétique.
Un catalyseur agit sur la vitesse et jamais sur K ou la position de l'équilibre.

Un autre facteur qui influencent la vitesse est la température :
$$ k = Ae^{\frac{-E_a}{RT}} $$

\subsection{Réactions plus complexes effectuées en deux ou plusieurs étapes} Ici, il y a plusieurs intermédiaires réactionnels.
Un intermédiaire réactionnel est peu stable.
L'étape la plus lente, soit l'étape déterminante est celle dont l'état de transition est le plus haut en énergie.
\section{Intermédiaires réactionnels }


Trois types d'intermédiaires réactionnels sont fréquemment rencontrés suivant les modes de rupture des liaisons au cours de réactions chimiques
\begin{description}
  \item [les carbocations ($C^+$)]
  \item [les carbonions ($C:^-$)]
  \item [les radicaux libres ($C\cdot$)]
\end{description}
Les intermédiaires réactionnels peuvent être caractérisés comme étant tertiaires, secondaires, primaires ou nullaires.
Cela dépend du nombre de carbone directement lié à l'atome de carbone considéré.

\subsection{Carbocations}
Ils sont des intermédiaires réactionnels où un carbone porte une charge positive, ayant subi la perte d'un de ses électrons de valence.
Ils sont très réactifs et réagissent avec des espèces riches en électrons.

Le carbone positif d'un carbocation est hybridé $sp^2$.
Il confère une géométrie triangulaire plane.
Les carbocations sont le résultat d'une rupture hétérolytique d'une liaison chimique.
Ce type de rupture implique que l'un des atomes  conserve les deux électrons et donc il y a formation d'un cation et d'un anion.
Par conséquent, pour mener à la formation d'un carbocation , il faut que l'atome de carbone soit lié à un atome plus électronégatif que lui.

\subsection{Carbanions}
Ils sont des intermédiaires réactionnels dans lesquels un carbone est chargé négativement, ayant reçu un électron supplémentaire sur sa couche de valence.
Il est hybridé $sp^3$.
Il est hautement réactif, il veut donner son électron pour retrouver sa forme neutre.
Les carbanions sont également le résultat de la rupture hétérolytique d'une liaison chimique.
Par contre, pour le carbanion, il faut plutôt que le carbone soit l'atome le plus électronégatif des deux atomes constituant la liaison.

\subsection{Radicaux libres}
Ils sont des espèces chimiques neutres, mais très réactives, dans lesquelles un atome de carbone possède un électron célibataire.
L'atome de carbone d'un radical libre est hybridé $sp^2$.
Les radicaux libres sont le résultat
d'une rupture homolytique d'une liaison chimique.
Ce type de rupture peut avoir lieu dans le cas des liaisons non polaires ou très faiblement polaires.
\section{Acides et bases de Lewis}

Les acides de Lewis sont des accepteurs de doublets d'électrons libres alors que les bases de Lewis sont des donneurs de doublets libres.
La constante d'acidité $K_a$ est une indication du degré d'ionisation d'un acide donné.
Plus un acide est fort, plus son ionisation en solution aqueuse est grande et plus sa constante d'acidité $K_a$ est élevée (spn $pK_a$ sera faible) .
$pK_a + pK_b = 14 $ .
$$ HA + H_2O \rightleftharpoons A^- + H_3O^+ \,\,\,\,\,\,\,\,\,\,\,\, K_a = \frac{[H_3O^+][A^-]}{[HA]} \,\,\,\,\,\,\,\,\,\,\,\, pK_a = -\log K_a $$
$$ B + H_2O \rightleftharpoons BH^+ + OH^- \,\,\,\,\,\,\,\,\,\,\,\, K_b = \frac{[OH^+][BH^+]}{[B]} \,\,\,\,\,\,\,\,\,\,\,\, pK_b = -\log K_b $$
\subsection{Facteurs qui influencent l'acidité et la basicité}

La force dépend de plusieurs facteurs
\begin{description}
  \item[De la taille des atomes]

    Plus la taille augmente, plus l'acidité augmente.
  \item[Hybridation]

    Plus le caractère ``s'' augmente, plus l'acidité augmente.
  \item[\'Electronégativité]

    Plus l'électronégativité augmente, plus l'acidité augmente.
  \item[Mésomérie / délocalisation électronique]

    Plus les délocalisations augmentent , plus l'acidité augmente.
\end{description}


\section{Catégories des réactifs}

\subsection{\'Electrophile}
Il est une espèce chimique pauvre en électrons et qui cherche donc à attirer et à recevoir des doublets d'électrons au cours d'une réaction chimique.
\begin{itemize}
  \item Les molécules dont l'atome central possède un octet incomplet sont de puissants électrophiles puisqu'elles sont déficientes en électrons.
  \item Les espèces chimiques chargées positivement mais qui possèdent leur octet.
    Elles cherchent à recevoir des électrons pour retrouver leur stabilité et leur neutralité.
    Plus un élément est électronégatif, plus son caractère électrophile est grand.
  \item Les molécules polaires portant des charges partielles.
    Les réactions menées par ce type d'électrophile sont généralement moins rapides et moins exothermiques.

\end{itemize}

\subsection{Nucléophiles}
Ils sont des réactifs qui aiment les sites positifs d'un composé.
Ce sont des espèces riches en électrons.
Les nucléophiles peuvent porter une charge négative ou être neutres, possédant des doublets libres d'électrons ou des électrons $pi$.

Les électrophiles sont en fait des acides selon la théorie de Lewis, les nucléophiles sont des bases de Lewis.

La nucléophilie est le terme employé pour caractériser l'effet des nucléophiles sur la vitesse des réactions chimiques de substitution nucléophile.
Elle dépend de plusieurs facteurs dont les plus importants sont
\begin{description}
  \item[Force du nucléophile en fonction de l'électronégativité]

    Plus les électrons du nucléophile sont retenus par un atome électronégatif, plus ils auront de la difficulté à effectuer l'attaque sur l'électrophile.
  \item[Force du nucléophile en fonction de la concentration de sa charge]

    Un ion portant une charge négative sur un atome donné sera plus réactif par rapport à une molécule renfermant un doublet d'électrons libres sur le même élément.

  \item[Force du nucléophile en fonction de sa taille]

    Un nucléophile encombré réagira moins rapidement qu'un petit nucléophile puisqu'il y aura plus de tension stérique à l'état de transition qui est alors moins stable.

  \item[Force du nucléophile en fonction de sa polarisabilité]

    Plus le nucléophile est polarisable, meilleur est ce dernier, puisque ses électrons  sont moins bien retenus par le noyau atomique.
    La polarisabilité croît en général avec la masse atomique, avec le numéro atomique.

  \item[Force du nucléophile en fonction de sa solvation]

    La solvation est le procédé par lequel un solvant entoure une particule de soluté pour la solubiliser.
    On dira que le soluté est solvaté.
    Plus un nucléophile est stabilisé par les attractions intermoléculaires avec le solvant, moins il sera en mesure d'attaquer l'électrophile et il sera donc moins réactif.

  \item[Force du nucléophile en fonction de la force de liaison formée]

    Les nucléophiles formant des liaisons stables avec le carbone sont donc meilleurs nucléophiles que ceux formant des liaisons faibles.

\end{description}


\section{Effets électroniques}

\subsection{Effet inductif} L'effet inductif est créé par une déformation du nuage électronique d'une liaison au sein d'un composé chimique.
L'effet inductif peut être attractif ou répulsif.


\subsection{Impacts concrets de l'effet inductif}

\begin{description}

  \item[Pour l'acidité des composés]

    \begin{itemize}
      \item effet inductif du Cl par rapport au H, et donc l'acide est plus fort.
      \item effet répulsif du $CH_3$ par rapport au H, et donc l'acide est moins fort.
    \end{itemize}


  \item[Pour la basicité des composés], les effets inductifs répulsifs favorisent le partage du doublet et la molécule est donc plus basique.

  \item[Pour la stabilisation des intermédiaires réactionnels]
    \begin{itemize}
      \item Pour les carbocations considérés comme des donneurs d'électrons en raison de l'effet répulsif.
        Le carbocation le plus stable est le carbocation tertiaire, puisque trois groupements électrodonneurs stabilisent davantage une charge positive que deux, un ou aucun groupement R.
      \item Pour les carbanions, ils sont plus stables quand ils sont peu substitués par les groupements alkyles R puisque les groupements R, qui sont des donneurs d'électrons par  l'effet inductif répulsif, augmentent encore la densité électronique sur le carbone porteur de la charge négative.
      \item Les radicaux libres ne respectent pas la règle de l'octet, tout comme les carbocations.
        Il leur manque un seul électron dans l'orbitale p.
        Ils sont stabilisés lorsqu'il sont entourés de plusieurs groupements R donneurs d'électrons par l'effet inductif répulsif.
    \end{itemize}
\end{description}

\subsection{Facteurs influant sur l'effet inductif}
Trois facteurs distincts doivent être considérés lors de l'étude de l'effet inductif
\begin{description}
  \item[\'Electronégativité de l'élément]

    L'élément le plus électronégatif accroit davantage le caractère acide de la molécule.

  \item[La distance]

    L'effet inductif s'atténue avec la distance, plus l'atome est loin de la liaison O-H de la fonction acide, plus l'acidité sera faible.

  \item[Nombre de groupements]

    Si un atome influence le caractère acide des molécules, alors plusieurs groupements amplifieront l'effet inductif.
\end{description}


\subsection{Effet mésomère}
L'effet(ou résonance) est le phénomène qui est la cause de la stabilité du benzène .
Sa molécule est plus stable, soit environ 151kJ/mol.
Le benzène possède une structure plane et chaque atome de carbone est situé à l'un des sommets d'un hexagone régulier.
La longueur des liaisons carbone-carbone est identique, soit 139pm.

Pour observer l'effet électronique de résonance, celle-ci n'aura d'abord lieu que s'il y a présence de doublets d'électrons libres, sur un atome ou sur un anion, d'électrons $\pi$ ou de charges positives.
La résonance ne peut se réaliser que lorsque les liaisons multiples sont conjuguées.
La conjugaison commence et se termine par une liaison multiple, une charge ou un doublet d'électrons libre au sein d'un composé.
\begin{description}
  \item[doublet d'électrons libre, liaison simple, liaison multiple]

    La délocalisation des électrons commence par le doublet d'électrons libre, puisqu'il s'agit de l'endroit le plus riche en électrons.
    Puis la répulsion électronique favorise le déplacement des électrons $\pi$ de la liaison multiple vers l'atome marquant la fin du système.
  \item[charge +, liaison simple, liaison multiple]

    Les carbocations sont toujours très réactifs puisque leur octet est incomplet.
    Cependant, ils peuvent être stabilisés par l'effet inductif des divers substituants ainsi que par la résonance.
    Il est possible de délocaliser la charge attribuée initialement à un atome particulier sur un plus grand nombre d'atomes et ainsi diminuer sa réactivité en le stabilisant.

  \item[Formes limites majeures et mineures de résonance]

    Il est important d'étudier la stabilité relative des formes limites de résonance
    \begin{itemize}
      \item majoritaire, respect de l'octet(pas de charge) $\leftrightarrow$  minoritaire, respect de l'octet(avec charge)
      \item majoritaire, carbocation tertiaire $\leftrightarrow$  minoritaire, carbocation primaire
    \end{itemize}

  \item[conjugaison avec seulement des électrons $\pi$]


    une liaison $\pi$  doit se rompre, ce qui crée alors les profils décrits précédemment.
    Plus la molécule est polarisée, plus la rupture sera facile à réaliser.

    La liaison la plus susceptible de se rompre est la liaison la plus polaire.

    S'il n'y a pas de liaison polaire, on compare la stabilité limite de résonance obtenue (carbocation tertiaire, secondaire, primaire) .

  \item[systèmes conjugués multiples]

    Il faut simplement traiter les différents systèmes séparément.
\end{description}




\part{Alcanes}
Les hydrocarbures sont les principaux constituants du pétrole et du gaz naturel.
Il existe des hydrocarbures saturés et des hydrocarbures insaturés dans lesquels on trouve les hydrocarbures aromatiques.
Les hydrocarbures saturés sont appelés alcanes.
\section{Structure des alcanes}
Tous les alcanes acycliques correspondent à la formule moléculaire générale $C_nH_{2n+2}$, où n est le nombre de carbone.
Les alcanes dont la chaine carbonée est non ramifiée sont dits alcanes linéaires.

Dans le cas des alcanes monocycliques, deux hydrogènes de l'alcane linéaire doivent être retirés.
Leur formule moléculaire est donc $C_nH_{2n}$.
On constate ainsi que les alcanes cycliques sont des hydrocarbures saturés, n'ayant pas de liaison $\pi$, mais possédant une formule semblable à  celle des hydrocarbures insaturés.



\section{Sources d'alcanes}

Les deux sources naturelles d'alcanes les plus importantes sont le pétrole et le gaz naturel.
Le pétrole est un mélange complexe de composés organiques à l'état liquide, dont un grand nombre ont des alcanes et cycloalcanes.
Le gaz naturel est essentiellement constitué de méthane ( 80$\%$) et d'éthane (20$\%$).


\section{Propriétés physiques des alcanes et attractions intermoléculaires}

Les alcanes sont insolubles dans l'eau comme ils sont apolaires.
Pour disperser les molécules d'alcanes à travers les molécules d'eau, il faut briser les ponts hydrogène établis entre les molécules d'eau, ce qui requiert une énergie considérable.
Les alcanes ne peuvent pas établir avec les molécules d'eau des interactions de force comparable aux ponts hydrogène.

Comparativement à la plupart des composés organiques, les alcanes présentent un point d'ébullition inférieur pour une même masse molaire.
Le point d'ébullition des alcanes s'élève avec la longueur de la chaine mais diminue lorsque celle-ci devient plus ramifiée ou acquiert une forme plus sphérique, les interactions intermoléculaires étant moins efficaces en raison d'une surface de contact plus petite.


\section{Réaction des alcanes}

Toutes les liaisons intermoléculaires des alcanes sont simples, covalentes, apolaires et fortes.
C'est pourquoi les alcanes sont relativement inertes et ne réagissent généralement pas avec la plupart des acides, bases et agents d'oxydation et de réduction.

\subsection{Oxydation et combustion; alcanes en tant que combustibles}Les alcanes servent principalement de combustibles.
Avec un excès d'oxygène, la réaction de combustion est complète, les alcanes brûlent pour produire du dioxyde de carbone et de l'eau.
Ces réactions dégagent alors une quantité considérable de chaleur et sont donc exothermiques.

La combustion est une réaction d'oxydation.
En cas de manque d'oxygène, la combustion est incomplète et on observe une oxydation partielle.

\subsection{Halogénation radicalaire de alcanes}

En présence de lumière ou à une température élevée, supérieure à $\si{300}{\celsius}$, une réaction exothermique se produit entre un mélange d'alcane et de chlore à l'état gazeux.
Des atomes de chlore remplacent alors un ou plusieurs  atomes d'hydrogène de l'alcane.
Il s'agit d'une réaction de substitution radicalaire.
Une réaction analogue se produit lorsqu'on utilise le brome.
La réaction d'halogénation ne se réalise pas avec
\begin{itemize}
  \item le réactif $I_2$ puisqu'elle est endothermique et donc énergiquement défavorable.
  \item le réactif $F_2$ puisqu'elle est très exothermique et il y a un risque d'explosion.
\end{itemize}

En présence d'un excès d'halogène, la réaction peut se poursuivre pour mener à des produits polyhalogénés.

Quand des alcanes supérieurs sont halogénés, le mélange de produits devient plus complexe, et les isomères individuels sont difficiles à  séparer et à purifier.
\begin{description}
  \item[Mécanisme de l'halogénation radicalaire en chaine]

    L'halogénation est une succession de réactions radicalaires.

    La première phase est celle de l'amorçage ou de l'initiation au cours de laquelle la molécule d'halogène se scinde en deux sous l'effet de la lumière ou de la chaleur.
    Il y a formation d'un radical.

    La deuxième phase de la réaction de l'halogénation est la propagation.
    Les atomes de chlore radicalaire formés lors de l'amorçage sont très réactifs, car leur couche de valence est incomplète.
    Ils peuvent soit se recombiner, soit en cas de collision avec une molécule d'alcane, arracher une molécule d'hydrogène pour former du chlorure d'hydrogène et un radical alkyle, $R \cdot $.
    Dans la phase de propagation, les deux réactions mènent toujours à la formation d'un radical, ce qui permet à la substitution radicalaire de se propager.

    La troisième et dernière phase de cette réaction de substitution radicalaire est la terminaison.
    Si deux radicaux se recombinent, la réaction en chaine s'arrête puisque ces réactions ne produisent aucun nouveau radical.

  \item[Régiosélectivité des réactions d'halogénation radicalaire]

    Puisque lors du mécanisme réactionnel, l'intermédiaire le plus stable est favorisé, l'abondance des produits sera directement affectée par la stabilité des intermédiaires.

\end{description}



En conclusion, pour la chloration, les produits finaux produits dépendent à la fois du phénomène de probabilité d'attaque et de la stabilité relative des intermédiaires réactionnels qui sont, en général, deux phénomènes opposés.
Par contre, pour la bromation, seule la stabilité relative des radicaux formés  au cours du mécanisme réactionnel a un effet sur les produits obtenus; la bromation est très sélective.

\part{Alcènes et alcynes}

\section{Définitions et classification}

Les hydrocarbures qui contiennent une liaison double carbone-carbone sont appelés des alcènes, ceux avec une triple liaison carbone-carbone sont des alcynes.
Leur formule générale lorsqu'ils sont non cycliques s'exprime comme suit :
\begin{itemize}
  \item pour l'alcène $C_nH_{2n}$
  \item pour l'alcyne $C_nH_{2n-2}$
\end{itemize}
Ces deux catégories sont des hydrocarbures insaturés.
Si deux liaisons doubles sont présentes dans un composé, on dira qu'il s'agit d'un diène.
Il existe aussi des catégorie de triènes, des tétraènes et même des polyènes.


\section{Caractéristiques de la liaison double}

Une liaison double consiste en un lien $\sigma$ et un lien $\pi$.
Les électrons $\pi$ sont plus disponibles que les électrons $\sigma$ et sont plus susceptibles d'interagir avec divers réactifs électrophiles.
Bien que la rotation soit libre autour des liaisons simples, elle est bloquée, dans des conditions normales, autour des liaisons doubles.
Ce blocage de la rotation autour de la liaison double amène la possibilité d'isomères géométriques pour certains alcènes substitués.


\section{Réactions chimiques des alcènes }
Comparativement aux alcanes, les alcènes peuvent subir une multitude de réactions chimiques.

\subsection{Réactions d'addition sur les alcènes}
Dans une réaction d'addition, la potion A du réactif A-B se fixe à l'un des atomes de carbone de la liaison double, alors que la portion B s'attache à l'autre carbone, ce qui crée un produit ne comportant que des liaisons simples.
La réaction globale est favorisée comme on brise un lien $\pi$ et un lien $\sigma$ pour former deux liens $\sigma$.

On peut subdiviser les réactions d'addition en

\paragraph{Réactions d'addition polaire}

La liaison du réactif A-B est covalente.
On sait que un alcène est un nucléophile, les électrons de la liaison $\pi$ attaque l'électrophile du réactif A-B ( $\delta^+$ ).
L'attaque entraine la rupture de la liaison $\sigma$ du réactif ainsi que la formation d'une nouvelle liaison $\sigma$ entre l'un des carbone du double lien et de l'atome A.
L'autre carbone est chargé positivement et donc un carbocation est formé.
Une espèce chargée négativement es formée $B^-$ qui devient nucléophile.
Il attaque le carbocation qui est très réactif.
La première étape est souvent l'étape lente.
on peut avoir des
\begin{description}
  \item[Addition d'acides]

    Les acides tels que les halogénures d'hydrogène (H-F, H-Cl, H-Br, H-I)
  \item[Addition de réactif asymétrique sur des alcènes asymétriques]
    deux ou plusieurs produit sont théoriquement possibles.
    Lorsqu'un réactif asymétrique s'additionne à un alcène asymétrique, la partie électropositive du réactif se lie à l'atome de carbone de la liaison double qui porte le plus grand nombre d'atomes d'hydrogène.
    Pour regarder si un produit de réaction est majoritaire ou minoritaire, il faut toujours regarder la stabilité de l'intermédiaire réactionnel, le carbocation tertiaire étant le plus stable suivi du secondaire puis du primaire.
    Une addition électrophile d'un réactif asymétrique sur un alcène asymétrique favorise la formation du carbocation le plus stable et par conséquent, forme majoritairement le produit découlant de ce carbocation.
  \item[Hydratation des alcènes en milieu acide]

    En présence d'un catalyseur acide, l'eau s'additionne sur les alcènes.
    Le produits de cette attaque de molécules d'eau sur une formation alcène mène à la formation d'alcools.

    L'acide lors de l'hydratation des alcènes agit comme catalyseur.
    Il est pas nécessaire d'avoir recours à une quantité stoechiométrique d'acide, une quantité catalytique suffit.

    Pour que la liaison soit brisée, le nucléophile doit être assez puissant pour attaquer un électrophile qui, lui même doit posséder un bon groupe partant.

    Si le groupe partant est une base faible, cela signifie que le doublet d'électron est stabilisé et qu'il s'agit d'un bon groupe partant.

    une réaction d'hydratation en milieu acide sur un alcène asymétrique entraine la formation de plus d'un produit.
    Le produit majoritaire est celui qui est issu de l'addition de la fonction alcool sur la carbone de l'alcène portant le moins d'atome d'hydrogène, soit le plus substitué.

  \item[Hydroboration]
    L'hydroboration est une réaction d'addition de type anti-Markovnikov.
    La réaction se fait de telle sorte que l'atome de bore s'additionne sur le carbone le moins substitué.
    L'hydroboration additionne l'hydrogène et le bore du même coté de la liaison double de la molécule.
    Il s'agit d'une addition syn.
    Cette particularité est d'autant plus importante s'il y a formation de centres stéréogéniques au cours de la réaction.

  \item[Addition d'halogénure d'hydrogène de type anti-Markovnikov]
    Pour avoir cette réaction, il faut ajouter du peroxyde ( en présence de lumière).
    L'addition d'halogénures d'hydrogène de type anti-Markovnikov est une réaction radicalaire.
    Elle implique donc les trois étapes classiques d'une réaction radicalaire, soit l'amorçage, la propagation et la terminaison.
    Le mécanisme de réaction implique tout d'abord la formation d'un radical à partir du peroxyde, suivie d'une attaque radicalaire sur l'halogénure d'hydrogène.
    Ce radical, à son tour, effectue une réaction radicalaire  sur la fonction alcène pour former majoritairement la radical le plus sable.
\end{description}

\paragraph{Réaction d'addition non polaire}

\begin{description}
  \item[Hydrogénation catalytique]

    L'hydrogénation catalytique est une réaction où l'hydrogène s'additionne sur les alcènes en présence d'un catalyseur approprié.
    La réaction d'hydrogénation est une addition syn, puisque les deux atomes d'hydrogène s'additionne, à partir du catalyseur, sur le même coté de la double liaison.

  \item[Halogénation]

    La réaction d'halogénation sur les alcènes implique une addition anti.
    Cela est dû à la formation d'un intermédiaire réactionnel particulier, l'ion bromonium ou l'ion chloronium, un ion ponté, au lieu du carbocation présent dans les réaction d'addition polaire.
    L'ion bromure et l'ion chlorure attaque toujours du coté opposé au  pont.
    De plus, notons que si les carbones possèdent des substituants différents, l'ion bromure attaque le carbone le plus substitué.
\end{description}


\subsection{Addition radicalaires et polymères}
Un polymère est une molécule volumineuse possédant en général une masse molaire élevée et composé de nombreuse unité identique.
La molécule la plus simple de ces unité est un monomère, et on appelle polymérisation le processus transformant un monomère en polymère.

La polymérisation radicalaire se fait en présence d'un catalyseur qui est un peroxyde organique.
Un radical catalyseur est formé et il s'additionne ensuite à la double liaison du monomère.

\subsection{Oxydation des alcènes}


\paragraph{Oxydation à l'aide de permanganate de potassium}

\begin{description}
  \item[Oxydation douce]

    L'addition des deux alcools sur les carbones de l'alcène est une addition syn puisqu'elle provient d'une réaction concertée menant à un cycle à cinq membre entre l'alcène et le permanganate de potassium.
    De ce fait, les deux oxygènes sont ajoutés du même coté de la liaison double dès la première étape du mécanisme réactionnel.
    L'oxydation douce n'offre pas un très bon rendement.
  \item[Oxydation forte]

    Ici, le permanganate de potassium doit être concentré et traité en milieu acide.
    De plus, l'oxydation doit se dérouler à température élevée.

\end{description}

\paragraph{Ozonolyse des alcènes}

Les alcènes réagissent rapidement et de manière quantitative à l'ozone, \ce{O_3}.
Le premier produit, le molozonide, provient d'une cycloaddition des atomes d'oxygènes de chaque extrémité de la molécule d'ozone à la liaison double carbone-carbone.
Par la suite, ce produit subit rapidement un réarrangement et devient un ozonide.
L'ozonolose est une réaction des alcènes à l'ozone qui mène à la formation de composés ayant des groupes carbonyles.
IL y a ozonolyse oxydante et l'ozonolyse réductrice.


\paragraph{Formation d'époxydes à partir d'alcènes}

Les époxydes sont des éthers cycliques à trois membres, dont l'hétéroatome est un oxygène.
Les peracides proviennent d'une combinaison des groupements fonctionnels peroxydes et acide.


\section{Caractéristiques de la triple liaison}

Ils présentent une géométrie linéaire et à cause de celle-ci, les liaisons triples ne sont pas présentes dans les petits cycles.
Les orbitales sp sont impliquée dans les liaisons $\sigma$  tandis que les orbitales p permettent la formation des liaisons $\pi$.
Une liaisons triple est constitué d'une seule liaison $\sigma$ et de deux liaisons $\pi$.


\section{Réactions chimiques des alcynes}

\paragraph{Réactions d'addition électrophile des alcynes}

\begin{description}
  \item[Addition des halogénures d'hydrogène et d'halogénation]

    Bien des réactions d'addition décrites pour les alcènes s'applique aussi aux alcynes, même si elles se déroulent en général beaucoup plus lentement.
    La liaisons triples est plus courte et est plus forte que la liaisons doubles, ce qui la rend moins réactive.

    Toutes les additions électrophiles polaires et non polaires décrites pour les alcènes sont possibles pour les alcynes.
    Par contre, ma réactivité est double avec les alcynes puisqu'il ont deux liaisons $\pi$.


  \item[Hydratation et hydroboration des alcynes]

    Ici, il existe une différence importante entre les alcènes et les alcynes.
    Pour les alcynes, il y a l'utilisation d'un catalyseur de mercure pour activer la réaction.
    L'énol subit une tautomérie,c'est-à-dire un réarrangement qui le transforme en une fonction cétone plus stable.
    Ce réarrangement porte le nom d'équilibre céto-énolitique ou tautomérie céto-énolitique.

  \item[Hydrogénation des alcynes]

    En présence d'un catalyseur de nickel ou de platine, les alcynes sont hydrogénés jusqu'à se transformer en alcanes.
    Cependant, l'ajout de catalyseur de palladium restreint l'addition d'hydrogène, de sorte qu'un seul équivalent molaire d'hydrogène s'additionne sur la triple liaison.

    Dans cette situation, le produit est un alcène.
    Les deux atomes d'hydrogène s'additionnent du même coté de la triple liaison à partir de la surface du catalyseur ( addition syn) grâce au même mécanisme que celui décrit pour les alcènes.

    Si l'on désire obtenir un alcène à partir d'un alcyne par addition anti, il est possible de réaliser un autre type de réduction contrôlée nécessitant un réactif, le sodium métallique dans l'ammoniac liquide.
    Le mécanisme est alors un mécanisme radicalaire.

\end{description}


\paragraph{Oxydations des alcynes}

Tout comme les alcènes, les alcynes peuvent être oxydés.
Ainsi le permanganate de potassium concentré en milieu acide mènera une fois de plus à la rupture de la liaison, mais produira cette fois uniquement des composée acides.

Il en va de même pour l'ozonolyse des alcynes qui, simplement traité avec de l'eau, offre aussi des composé acides.

\paragraph{Acidité des alcynes et élongation de la chaine carbone}

\begin{description}

  \item[Formation d'un sel alcynes]

    Un alcynes ayant un atome d'hydrogène sur un des carbones de la triple liaison est faiblement acide.
    Il pourra être arracher par une base très forte.

    Ce type de réaction se produit avec l'atome d'hydrogène attaché à l'atome de carbone d'une liaison triple plutôt que celui d'une double.
    \`A mesure que le caractère du carbone hybridé augmente, l'acidité de l'hydrogène lié à ce carbone augmente.

    Même si les alcynes sont acides, ils le sont beaucoup moins que l'eau.
    Les acétylures peuvent être hydrolysés en alcynes par l'eau.
    Il faudra donc éviter toute présence d'humidité lors des réactions visant  à allonger la chaine de carbone avec des sels d'alcynes.

  \item[\'Elongation de la chaine carbone]

    Une fois l'alcyne déprotoné, le sel d'alcyne résultant est un nucléophile pouvant servir, en synthèse organique, à une élongation de la chaine de carbones.


\end{description}


\part{Composés halogénés}

Les composés halogénés sont des structures organiques d'alcanes sur lesquelles un ou plusieurs atomes d'halogène remplace l'atome hydrogène.
Les composés halogénés sont classés en trois catégories, tout comme les intermédiaires réactionnels.
En effet, on trouve les composés halogéné primaire, secondaire et tertiaire.

\section{Propriétés physiques et réactivité générales}

La liaison C-X est une liaison covalente polarisée.
La présence de celle-ci dans les composés halogénés leur donne des propriétés physique différentes de celle des hydrocarbures correspondant.
Plus l'halogéné possède une masse molaire élevée et plus il est polarisable.
Et plus la masse molaire est élevée, plus la température d'ébullition est élevée, car les attractions intermoléculaires sont plus nombreuses et importantes.


La liaison covalente polaire C-X n'affecte pas seulement les propriétés physique des composés halogéné mais elle est aussi la cause de leur réactivité spécifique.
Dans cette liaison C-X, c'est l'halogène qui est l'élément le plus électronégatif et donc porteur dela charge partielle négative.
Le carbone, quant à lui, devient alors chargé partiellement positif.

La réactivité du substrat dépend de la force de la liaison C-X.
Plus celle-ci est forte, plus la réactivité est faible.
On aurait tendance à penser l'inverse mais la liaison C-X est le résultat d'un recouvrement des orbitales de l'atome de carbone et de l'atome d'halogène.
Plus la taille de l'orbitale de l'halogène augmente, c'est-à-dire que le nuage électronique autour de l'halogène devient de plus en plus diffus plus le recouvrement des orbitales diminuent et la liaison est moins forte et de plus en plus longue.


Deux types principaux de réactions chimiques impliquant le bris de la liaison C-X sont possibles avec des composés halogénés, soit la substitution et d'élimination.

\section{Substitution nucléophile ($S_N$)}

Une réaction de substitution nucléophile implique le remplacement d'un groupe partant par un nucléophile.

Si le nucléophile et le substrat sont neutres, le produit organique aura une charge positive.
Si le nucléophile est un ion négatif et que le substrat est neutre, le produit organique sera neutre.

\subsection{Mécanisme d'une substitution nucléophile d'ordre 2, $S_N2$}
Ce mécanisme se réalise en une seule étape.
Le nucléophile attaque le carbone du coté opposé au groupe partant pour établir ainsi une nouvelle liaison.
\`A l'état de transition, le nucléophile et le groupe partant sont tous deux partiellement attaché au carbone qui subit la substitution.
La loi de vitesse de se genre de réaction est $v = k[R-X]^1[Nu]^1$ (ordre global = 2).
Cela implique que le substrat et le réaction participe à l'étape limitante de la réaction.
Le mécanisme d'une réaction de type $_N2$ s'effectue alors en une seule étape où le nucléophile attaque le substrat.
Il s'agit d'un mécanisme bimolécualire où le nucléophile et le substrat sont impliqués dans le complexe activé de l'état de transition.
Si dans le substrat le carbone partant le GP est chiral, le $C^*$, il sera essentiel de faire une représentation tridimensionnelle du mécanisme de la réaction puisq'on observera une inversion de configuration.
La réactionest plus rapide quand le carbone du substrat portant l'halogène est primaire.
La raison de cette réactivité est l'encombrement stérique.

\subsection{Mécanisme d'une substitution nucléophile d'ordre 1, $S_N1$}
Le mécanisme $S_N1$ se réalise en deux étape.
La première étape lente est lente et c'est la liaison entre l'atome de carbone et le groupe partant qui se brise.
Le groupe partant emporte avec lui les électrons de la liaison et un carbocation se forme.
Au cours de la deuxième étape, qui est rapide, la carbocation se combine avec le nucléophile pour former le produit attendu.

La loi de la vitesse de réaction est $ v = k[R-X]'$ (ordre global 1), ce qui indique que seule le substrat est impliqué dans l'étape déterminante.
Il s'agit d'un mécanisme unimoléculaire : seul le substrat est impliqué dans le complexe activé à  l'état de transition lors de l'étape limitante.
Le mécanisme réactionnel s'effectue en deux étapes.

Puisque le mécanisme passe par un carbocation, il y a une perte d'activité optique, c'est-à-dire une racémisation, si l'atome de carbone qui le porte le groupe partant est chiral.
La réaction est plus rapide lorsque le carbone du substrat portant l'halogène est tertiaire.
La raison de ce phénomène est que les réactions se réalisent grâce à la formation d'un carbocation, qui est plus stable s'il est tertiaire.

\subsection{ Comparaison entre les mécanismes $S_N1$ et $S_N2$}

On va regarder la structure de l'halogènure, le solvant, le nucléophile et la stéréochimie


\begin{tabular}{|l|l|l|}
  \hline
  Variable & $S_N2$ & $SN_1$ \\
  \hline
  \textbf{Structure de l'halogénure} & & \\
  Primaire ou nullaire & courant & rare \\
  Secondaire & parfois & parfois  \\
  Tertiaire & Négligeable & Courant \\
  \hline
  \textbf{Solvant} & Les solvants polaires aprotique & Les solvants polaires protique \\
                   & augmentent la v de la réaction & augmentent la v de la réaction  \\
  \hline
  \textbf{Nucléophiles} & Vitesse dépend de [Nu] 	& Vitesse indépendante de [Nu]\\
                         & favorisé quand Nu fort & \\
  \hline
  \textbf{Stéréochimie} & Inversion de configuration & Racémisation \\
  \hline
\end{tabular}


\section{Déshydrohalogénation(réaction d'élimination)et mécanisme E2, E1}

Lorsqu'un composé halogéné, avec un atome d'hydrogène attaché à l'atome de carbone adjacent ( carbone $\beta$ ) au carbone portant l'halogène ( carbone $\alpha$), réagit avec un nucléophile, deux réaction concurrentes sont possible : la substitution et l'élimination.

Dans une réaction d'élimination le nucléophile agit comme une base et arrache le proton du carbone $\beta$, soit le carbone voisin de celui portant l'halogène, la carbone $\alpha$.
L'halogène du  carbone $\alpha$ et l'atome d'hydrogène ducarbone $\beta$ sont éliminés, et nouvelle liaison ( un lien $\pi$) est formée entre les deux carbone .

Les réactions d'élimination se produise en général à une température plus élevée que les réactions du substitution.

\subsection{Mécanisme d'élimination d'ordre 2 ( E2)}
Le mécanisme E2 se déroule en 1 seule étape et aucun intermédiaire réactionnel n'est formé.
Le nucléophile, qui tient lieu de base, arrache l'atome l'atome d'hydrogène porté par un carbone $\beta$.
En même temps, le groupe partant se libère et une liaison double est formée.
Il s'agit donc d'un mécanisme bimoléculaire où la base et le substrat se retrouve impliqué dans le complexe activé.
La loi de vitesse est $v = k[R-X]^1[Nu]^1$ (ordre global = 2)

Une conformation coplanaire et anti de H et X est favorisée pendant une élimination de type E2.
Lorsqu'on obtient plusieurs produits lors d'une réaction d'élimination, le plus abondant sera l'alcène le plus substitué, soit l'alcène ayant le moins d'atome de carbone directement lié aux carbone de la double liaisons.
Il est possible d'obtenir des isomères  géométrique.
( cis-trans).

Les éliminations E2 sont favorisée à haute température  en présence d'une base forte et dans un solvant organique très peu ou pas polaire, ne favorisant pas l'ionisation et donc la formation d'un carbocation intermédiaire.
Tous les composé (primaires, secondaire, tertiaire) peuvent subir une élimination d'ordre 2.

\subsection{Mécanisme d'élimination d'ordre 1 (E1)}

Le mécanisme E1 est un mécanisme à deux étapes.
La première étape est lente et déterminante de la vitesse et est l'ionisation du substrat menant à la formation d'un carbocation.
\`A la deuxième étape, l'intermédiaire perd un hydrogène porté par un carbone $\beta$, ce qui produit un alcène.
Il s'agit d'un mécanisme unimoléculaire où le composé halogéné est le seul à être impliqué dans le complexe activé.
la loi de vitesse est  $ v = k[R-X]'$ (ordre global 1).

Les élimination E1 sont favorisée à haute température, lorsque le composé halogéné est tertiaire et en présence d'un solvant organique polaire aidant ainsi à l'ionisation et à la formation d'un carbocation.
Enfin le nucléophile se dot être une base, assez faible ou modérée, pour assurer le temps nécessaire à la formation du carbocation  et éviter,par conséquentn un mécanisme E2.

\subsection{ Comparaison entre les mécanismes E2 et E1}

On va regarder la structure de l'halogènure, le solvant, le nucléophile et la stéréochimie


\begin{tabular}{|l|l|l|}
  \hline
  Variable & E2 &  E1 \\
  \hline
  \textbf{Structure de l'halogénure} & & \\
  Primaire ou nullaire & courant & Négligeable\\
  Secondaire & courant & parfois  \\
  Tertiaire & courant & Courant \\
  \hline
  \textbf{Solvant} & E2 favorisé dans un & E1 favorisé dans un\\
                   & solvant faiblement polaire & solvant polaire  \\
  \hline
  \textbf{Nucléophiles} & Nu doit être une base forte et 	& Nu doit être une base  modérée\\
                         & concentrée & à faible\\
  \hline
  \textbf{Stéréochimie} & Inversion de configuration & Racémisation \\
  \hline
\end{tabular}


\section{Compétition entre les réactions de substitution  et d'élimination}

\subsection{Composés halogéné tertiaire}
\begin{description}
  \item[Substitution $S_N1$] uniquement en raison du grand encombrement stérique.
  \item[\'Elimination E1 ou E2] en fonction de la force du nuléophile.
\end{description}

C'est ainsi qu'en présence d'un nucléophile faible et d'un solvant polaire, il y a compétition entre les mécanise $S_N1$ et $E1$ .
Cependant le température joue un rôle important.
Plus elle est élevée, plus elle favorise un produit d'élimination.

Si on utilise une base forte et un solvant moins polaire, on favorise le mécanise E2, seule l'élimination survient avec comme unique produit, un alcène.


\subsection{Composé halogéné secondaire}

Les 4 mécanismes peuvent intervenir en fonction de la nature  de Nu et des conditions de réaction( la température et le solvant).

En général, la substitution est favorisée en présence d'un nucléophile puissant qui est pas une base forte.
($S_N2$)

ou un nucléophile faible dissous dans un solvant polaire  ($S_N1$).

Cependant, l'élimination est susceptible de se produire avec une base forte (E2).



\subsection{Composé halogéné primaire}

Dans le cas des composés halogéné primaire, les mécanismes ($S_N2$) et ($E2$) sont les seuls possibles, car l'ionisation d'un carbocation primaire ne se produit pas !!!

En présence de la plupart des  nucléophiles, les composés halogénés primaires aboutissent principalement à des produits de substitution ( $S_N2$) .
Seules les bases fortes et très encombrées vont favoriser une réaction E2.










\part{Composés aromatiques}
L'hydrocarbure $C_6H_6$ est appelé le benzène et constitue l'hydrocarbure parental d'une classe de composé particulièrement stable, appelés composé aromatique.
Cette stabilité est essentiellement due au phénomène interne de résonance de le cycle aromatique.

\section{Aromaticité}

Un composé est aromatique s'il
\begin{itemize}
  \item est tout d'abord cyclique
  \item a un système conjugué sur l'ensemble de cycle
  \item contient 4n + 2 électrons délocalisables ( électron $\pi$ ou des doublets d'électron libre) et si n n'est pas entier, le composé n'est pas aromatique.
  \item est plan
\end{itemize}


\section{Quelques observations sur le benzène}

\subsection{Symbole du benzène}
L'un d'eux est la structure de Kekulé, indiquant clairement la présence des six électrons $\pi$, et l'autre, un hexagone avec un cercle à l'intérieur, symbolisant des électrons du nuages $\pi$.

\subsection{Réactivité du benzène}

Selon sa formule moléculaire, $C_6H_6$, le benzène est un composé hautement insaturé.
Cependant, il ne comporte pas comme un composé insaturé typique.
Par exemple, il ne décolore pas les solutions de brome à l'opposé des alcènes et alcynes.

Cette faible réactivité est due à l'aromaticité des composés benzéniques.
En effet, la réaction n'a pas lieu, puisque le phénomène de résonance stabilise énormément les composé!!!


Le benzène réagit plutôt par substitution  électrophile.
Puisque les six atomes d'hydrogène du benzène sont chimiquement équivalents, un seul monobromobenzène ou monochlorobenzène est obtenu au cours de la réactions.
Quand le bromobenzène est traité de nouveau dans les même condition, trois dibromobnzène sont obtenus, soit le 1,2-dibromobenzène, le 1,3-dibromobenzène, et le 1,4-dibromobenzène.
On peut donc en déduire que tous les hydrogènes dans le bromobenzène ne sont plus chimiquement équivalent.



\section{Substitution électrophile aromatique}

Les réactions les plus courantes des composés aromatiques font intervenir une substitution des atomes d'hydrogènes par d'autres atomes ou groupes d'atomes.

\subsection{Mécanisme général de la substitution électrophile aromatique}
L'électrophile s'additionne sur le benzène lors de la première étape endothermique au cours de laquelle l'aromaticité est perdue.
Puis un proton situé sur l'atome ce carbone auquel est fixé l'électrophile est arraché à la seconde étape, qui est exothermique, ce qui rétablit l'aromaticité.

\subsection{Halogénation}

La chlore et le brome sont intégrés au cycle aromatique par une réaction mettant en jeu l'halogène et son halogénure ferrique correspondant en tant que catalyseur.

La fluoration et l'iodation directes des cycles aromatiques sont possibles, mais nécessitent des méthodes particulières qui ne seront pas traité ici.

\subsection{Nitration}
Au cours de la nitration aromatique, l'acide sulfurique, le catalyseur, cède un protons à l'acide nitrique, qui perd alors une molécule d'eau et génère l'ion nitonium, lequel renferme un atome d'azote chargé négativement.
\`A la première étape, l'acide nitrique agit à titre de base puisque l'acide sulfurique est un acide plus fort.
Comme dans la chloration et la bromation, cette première étape vise à préparer un électrophile capable d'effectuer une substitution sur le cycle aromatique L'ion nitronium, un électrophile puissant, peut subir une attaque du cycle aromatique.

\subsection{Sulfonation}
Ici on utilise de l'acide sulfurique concentré en présence de trioxyde de soufre ou de trioxyde de soufre protoné.
Les ions $^+SO_3H$ agissent comme électrophile lors de la réaction.

\subsection{Alkylation et acylation}
Dans cette réaction, l'élecrophile est un carbocation.
Celui-ci peut être formé par l'ajout d'un proton à un alcène ou par l'activation d'un halogène sur un composé halogéné, en ayant recours à un acide de Lewis.

L'acylation de Friedel-Crafts se déroule de manière similaire à l'alkylation.
L'électrophile est un cation acylium provenant d'un dérivé d'un acide carboxylique, habituellement un halogénure d'acide.
Cette réaction pratique est une voie de synthèse générale qui peut mener aux cétones aromatiques.

\section{Substituants activants et déactivants}

Les substituants d'un cycle aromatique ont un effet sur la vitesse de la substitution électrophile comparativement à celle du benzène.
Le substituants d'un cycle aromatique est dit activant s'il augmente la vitesse de la substitution par rapport à celle du benzène, et est désactivant s'il la diminue.

Si la vitesse de réaction dépend de l'attaque du cycle aromatique sur l'électrophile, alors le substituants qui fournissent des électrons au cycle augmenteront sa densité électrophile et accéléreront la réaction.
\begin{itemize}
  \item Les activants sont dits forts s'ils enrichissent le cycle aromatique par résonance et faibles s'ils le font exclusivement par un effet inductif répulsif.
  \item Un déactivants fort appauvrit le cycle aromatique par résonance alors qu'un déactivants faible l'appauvrit par effet inductif attractif seulement.
\end{itemize}



\section{Groupes orienteurs en ortho para, et en méta}

Les substituants déjà présents sur un cycle aromatique déterminent la position du substituants de deux façon différentes, et ce, pour toutes les réactions vue précédemment.

\begin{itemize}
  \item nitration du toluène produit principalement un mélange 0-, p-nitrotoluène.
  \item nitration du nitrobenzène réalisé dans des conditions similaires produit principalement l'isomère méta.
\end{itemize}

\subsection{Groupes orienteur en ortho et para }

Le méthyle est un groupe orienteur en ortho et para, puisque la réaction peut procéder par le carbocations intermédiaire le plus stable.
De la même manière, tous les autres groupes alkyles sont des orienteurs ortho et para.

L'atome fixé sur le cycle aromatique possède au moins un doublet d'électrons libre.
Ce doublet libre d'électron peut servir à stabiliser une charge positive adjacente.

Tous les groupes ayant au moins un doublet d'électron libre sur l'atome attaché au cycle sont orienteurs en ortho ou para.
incluant les halogènes.
Les donneurs sont activants et orientent la substitution électrophile aromatique en ortho ou para.

\subsection{Groupes orienteurs en méta}

Tous les groupes dans lesquels l'atome directement fixé au cycle aromatique est chargé positivement ou prend part à une liaison multiple avec un élément plus électronégatif seront orienteurs en méta.
Les capteurs sont des déactivants et orientent la substitution électrophile aromatique en méta.

\subsection{Particularité des substitutions électophiles aromatique des halogènes}
Tous les groupes orienteurs en méta sont des groupes déactivants.
\`A l'opposé, les groupes orienteurs en ortho et para cèdent généralement des électrons au cycle, par effet inductif ou par résonance, et sont, par conséquent, activants.
Dans le cas des halogènes, deux effets contradictoires sont à l'origine de la seule exception importante à ces règles.
Comme ils sont fortement électroattracteurs, les halogènes déactivent,  par effet inductif, le cycle benzénique; ils sont des déactivants faibles.
Cependant, ils possèdent des doublet d'électrons libres au même titre que les groupes orienteurs en ortho et para, ce qui permet une stabilisation du carbocation par une délocalisation additionnelle de la charge positive dans les formes limites de résonance.



\section{Importance des effets orienteurs en synthèse organique}
On voit que l'une des forme de résonance possède deux charges positive adjacentes, ce qui représente un arrangement des très peu favorisé.

Tous les groupes dans lesquels l'atome directement relié sur le cycle aromatique est chargé positivement ou prend par à une liaison multiple avec un élément plus électronégatif seront orienteurs en méta.
\\

\begin{tabular}{|c|l|}
  \hline
  Subsituent group (activing and ortho, para-directing)  &      Name of group \\
  \hline

  &  amino \\
  & \\
  &  hydroxy, alkoxy \\
  & \\
  & acylamino \\
  &  \\
  & \\
  &  alkyl \\
  & \\
  &  halo \\
  & \\
  \hline

  Subsituent group (deactiving and ortho, para-directing)  &      Name of group \\
  \hline

  &  acyl, carboxyl \\
  & \\
  & \\
  &  carboxamido, carboalkoxy \\
  & \\
  & \\
  & sulfonic acid \\
  &  \\
  & \\
  &  cyano \\
  & \\
  &  nitro \\
  & \\
  \hline
\end{tabular}




\part{Alcools, phénols et thiols}
\section{Classification des alcools}

Les alcools sont classifiés en groupes primaire, secondaire et tertiaire.

\section{Caractéristiques des alcools et phénols}
\subsection{Attractions intermoléculaire de type pont hydrogène}
Le point d'ébullition des alcools est beaucoup plus élevé que celui des éthers ou des hydrocarbures car ils établissent des attractions intermoléculaires de type pont hydrogène.
Leur force est de l'ordre de 20 à 40 kJ/mol.

Pour les petites molécules d'alcool, ceux-ci sont miscible dans l'eau due au pont hydrogène.

\subsection{Acidité des alcools et phénols}
Tout comme l'eau, ma plupart des alcools possèdent un très faible caractère acide.
L'effet inductif des groupements alkyles enrichit la densité électronique de l'oxygène et donc l'acide est moins fort.
L'effet attractif  des atomes de fluor diminue la densité électronique et favorise le bris de la liaison donc l'acide est plus fort.

Pour leur part, les phénols sont des acides plus fort que les alcools.
Ceux-ci sont des acides plus fort du à la résonance et que les ions éthanolate sont donc stabilisés.

\begin{itemize}
  \item préparation des ions alcoolate \\
    \\
    \\
  \item préparation des ions phénolate\\
    \\
    \\
\end{itemize}

\subsection{Basicité des alcools et phénols}
Les alcools et le phénols sont des substances amphotères (il peuvent agir comme acide faible et comme base faible).
On peut donc avoir la déshydratation des alcènes.
Cette protonation est la première étape de deux réactions importantes des alcools : la déshydratation en alcènes et la conversion en composés halogénés.

\section{Réactivité des alcools}

\subsection{Déshydratation des alcools en alcène}
Les alcools peuvent être déshydraté en présence d'un acide fort et par chauffage.

La déshydratation est une réaction d'élimination qui procède selon un mécanisme $E1$ ou $E2$ en fonction de la catégorie à laquelle appartient l'alcool.

\begin{itemize}
  \item Mécanisme $E1$ pour les alcools tertiaire
  \item Mécanisme $E2$ pour les alcools primaires
\end{itemize}

Comme chez les composé halogénés, il existe une compétition entre les mécanisme $E2$ et $S_N2$ chez les alcools primaire et secondaire.
Les alcools tertiaires, quand à eux, ne sont pas de bons nucléophile pour réaliser une $S_N1$ en raison de leur fort encombrement stérique.
Par conséquent le produit d'élimination sera beaucoup plus abondant.

La déshydratation de l'alcool peut parfois mener à la formation d'un mélange d'alcène.

Finalement, la déshydratation de l'alcool peut parfois mener à la formation de liaisons doubles.

\subsection{Formation des halogénés à partir des alcools}
Les alcools réagissent avec les halogénures d'hydogène et forment des composés halogénés selon une réactions de substitution nucléophile.
On obtient surtout des produit de substitution plutôt que d'élimination car les ions halogènure sont de bon nucléophiles.
Une fois encore, la vitesse de réaction et le mécanisme dépendent de la catégorie de l'alcool (primaire, secondaire, tertiaire).

\begin{itemize}
  \item Mécanisme $S_N1$ pour les alcools tertiaire
  \item Mécanisme $S_N2$ pour les alcools primaires
\end{itemize}

On peut avoir d'autre méthode pour la préparation de composés halogénés à partir d'alcool.
Par exemple, le chlorure de thionyle réagit avec les alcools pour produire des composés chlorés.

La réaction avec les halogénures de phosphore permet aussi de transformer les alcools en composé halogéné.

C'est deux dernières méthodes sont employées surtout avec les alcools primaires et secondaires parce que la réaction avec les halogénures d'hydrogène est lente.

\subsection{Oxydation des alcools en aldéhydes, en cétones et en acides carboxylique}
Les alcools munis d'au moins atome d'hydrogène fixé à l'atome le carbone portant le groupe hydroxyle, soit les carbone primaire et secondaire, peuvent être oxydés en composés carbonylés.
Les alcools primaires mènent à des aldéhydes qui peuvent être oxydés davantage en acides carboxylique.
Les alcools secondaire fournissent des cétones.



Les alcools tertiaire, qui ne possèdent aucun atome d'hydrogène sur l'atome de carbone portant le groupe hydoxyle ne se prêtent pas à ce type d'oxydation.


Le trioxydes de chrome, $CrO_3$, dissous dans une solution aqueuse  d'acide sulfurique est l'un des agents d'oxydation des alcools les plus couramment utilisés en laboratoire.
Cette solution oxydante parte le nom de réactif de Jones.
Il s'agit d'un oxydant fort et non sélectif effectuant une oxydation complète des alcools primaires en acides carboxyliques.

\section{Alcools munis de plusieurs groupes hydroxyle}

Les composés munis de deux groupes alcool adjacents sont des glycols.
Grâce à son aptitudes accrue à former des ponts hydrogènes, l'éthylène de glycol est miscible dans l'eau et possèdent un point d'ébullition exceptionnellement élevé pour sa masse molaire.



\section{ Comparaison entre les alcools et les phénols}
Les alcools et les phénols présentent beaucoup de propriétés similaires.
Cependant, alors qu'il est relativement facile de briser la liaison C-OH des alcools par catalyse acide, il est difficile de le faire chez les phénols.
Les cations phényles sont instables du point de vue énergétique et sont donc extrêmement difficiles à former.
Par conséquent les phénols ne peuvent pas  subir une substitution de leur groupe hydroxyle selon les mécanisme $S_N1$ t $S_N2$.

\section{Réactivité des phénols}

\subsection{Substitution électrophile aromatique des phénols}
Les phénols subissent une substitution électrophile aromatique dans des conditions très douces parce que le groupe hydroxyle est un groupe fortement activant.
Ainsi le phénols peut subir une nitration en solution aqueuse diluée d'acide nitrique.
L'ajout de catalyseur n'est pas nécessaire.


\subsection{Oxydation des phénols}
Les phénols peuvent être facilement oxydés.
\`A titre d'exemple, l'hydroquinone exposé à l'air ambiant pendant un certain temps se colore en raison du produit d'oxydation qui de forme.
Pour que les réactions d'oxydations des phénols puissent avoir lieu et former des quinones, on emploie souvent des sels de Frémy.


\section{Rôle antioxydant des phénols}
Les phénols sont des antioxydants qui empêchent l'air d'oxyder les substances.
Il y a la formation du radical phénoxyde à partir de la réaction d'un radical hydroxyde avec un phénol.
\section{Thiols et thiophénols}
Le soufre peut substituer l'oxygènes dans les composés organiques.
Le groupe -SH, appelé groupe sulfhydryle, est le groupe fonctionnel des thiols.


Les thiols sont plus acides que les alcools.
Ainsi le $pK_a$ de l'éthanethiol est de 10.6, tandis que celui de l'éthanol est de 15.9.
Pour cette raison, les thiols, contrairement aux alcools, réagissent avec des bases telles que l'hydroxyde se sodium en solution aqueuse pour former des ions thiolate.

Les thiols s'oxydent facilement en disulfure, des composés renfermant une liaison S-S, à l'aide d'agent oxydants doux comme le peroxyde d'hydrogène ou l'iode.
Le disulfuren diallylique est un disulfure naturel qui dégage une odeur d'ail.
\`A l'aide d'un agent réducteur, il est possible de procéder à la réaction inverse et de transformer les disulfure en thiols.


\part{\'Ether et époxydes}
Tous les éthers sont des composés constitués de deux groupes alkyles reliés par des liaisons $\sigma$ de part et d'autre d'un seul atome d'oxygène.
La forme générale des éther est R-O-R'.
\section{Propriétés physique des éthers}

Les éthers sont des composés incolores qui dégagent une odeur caractéristique.
Ils possèdent un point d'ébullition plus bas que les alcools pour un même nombre de carbone.
En fait, le point d'ébullition des éthers est pratiquement le même que celui de hydrocarbures correspondant dans lequel on aurait remplacé l'atome d'oxygène de l'éther par un groupe méthylène $-CH_2-$.
Les molécules d'éther ne forment pas des ponts hydrogène mais même si elles n'en forment pas  les unes avec les autres, elles peuvent néanmoins établir des pont hydrogène avec certains groupements fonctionnels dont, entre autres, les alcools.
En raison des attractions intermoléculaires possibles, les alcools et les éthers sont en générales solubles les uns avec les autres.

\section{\'Ethers en tant que solvant}

Les éthers sont des composés relativement inertes.
Normalement, ils ne réagissent ni avec les acides et les bases dilués ni avec les agents oxydants et réducteurs courant.
Ils ne réagissent pas non plus avec le sodium métallique, une propriété qui les distingue des alcools.
Une telles inertie générales et le fait que la plupart des composés organiques soient solubles dans les éthers font de ces derniers d'excellents solvants pour réaliser les réactions chimiques.


\section{Préparation des éthers}
\subsection{Préparation des éthers symétriques}
Le plus important éther commercial est l'éthoxyéthane.
On le prépare à partir d'éthanol et d'acide sulfurique.
Cette réaction se réalise à haute température et est efficaces seulement à partir d'alcool primaire.
La réaction se réalisant en présence de chaleur, il y a une compétition entre la formation d'un alcène ($E2$) et de la formation de l'éther ( $S_N2$).


Si  cette réaction s'effectue à partir d'un alcool tertiaire ou secondaire, le produit final est aussi un éther, mais celui-ci est obtenu selon un mécanisme $S_N1$.

\subsection{Préparation des éthers asymétrique}

La production commerciale du 2-méthoxy-2-méthylpropane a pris de l'importance au cours des dernière années.
Il est préparé en ajoutant du méthanol au 2-méthylpropène par catalyse acide.



\section{Réactivité des éthers}

Les éthers possèdent des doublets libres sur leur atome d'oxygène et sont, par conséquent des bases de Lewis.
Ils réagissent avec les protons des acides forts et avec les acides de Lewis.

\section{\'Epoxydes}

\subsection{Préparation des époxydes}
Les oxiranes couramment appelés époxydes sont des éthers  cycliques à trois atomes, dont l'un est un oxygènes.
L'époxydes commercial le plus important est l'oxiriane,produit par une oxydation à l'air ($O_2$) de l'éthène catalysée par l'argent.

\subsection{Réactivité des époxydes}
En raison de la tension d'angle dans un cycle à trois atomes, les époxydes sont beaucoup plus réactif que les éthers acyclique et aboutissent à des produits dérivant d'une ouverture du cycle.


La première étape du mécanisme de cette réaction est, tout comme pour les éthers acycliques, une protonation réversible de l'atome d'oxygène et l'époxyde.
Au cours de la deuxième étape, une substitution nucléophile de type $S_N2$ a lieu sur un des carbones d l'époxydes, l'eau étant un nucléophile.
Puis la perte du proton conduit à la formation du diol vicinal.
Dans le cas des époxydes symétriques, le nucléophile attaque l'un ou l'autre carbone.
Comme les ions bromonium, l'époxyde encombre une face d'attaque et oblige ainsi le nucléophile à réaliser son attaque du coté opposé au groupe partant.
Par conséquent, le produit obtenu est anti.



Nous avons déjà noté que les mauvais groupes partant comme le groupe hydroxyle -OH ne peuvent être expulsés sans être préalablement  transformés en bons groupes partants (par une protonation).
Cependant, il faut spécifier que la forte tension d'angle du cycle régnant dans un époxyde lui confère des caractéristiques particulières.
En effet de forts nucléophiles permettent son ouverture en milieu basique sans protonation préliminaire.



\section{\'Ethers acyclique}
Il existe aussi des éthers cycliques de plus grande taille que les époxydes à trois atomes.
Les plus courant ont des cycle constitué de 5 ou 6 atomes.







\part{Réactivité des aldéhydes et cétones}

\section{Aldéhydes et cétones}
Les aldéhydes et les cétones se caractérisent par la présence d'un groupe carbonyle ( C=O) au sein de leur structure.

\section{Propriétés physiques des aldéhydes et des cétones}

La polarisation de la liaison C=O explique les propriétés physiques des composés carbonylés.
Ainsi, le point d'ébullition de composés carbonylés est supérieur à celui des hydrocarbures, à cause de la présence d'interactions dipôle-dipôle entre les molécules, mais inférieur à celui des alcools de masse molaire similaire à cause de l'absence de ponts hydrogène entre les molécules.
Les composés carbonylé de petite masse molaire sont hydrosoluble.
Même si elles peuvent pas faire de liaisons hydrogène, elles peuvent faire des liaisons O-H.

\section{Réaction des aldéhydes et cétones : addition nucléophile}

\subsection{Nucléophile sur le groupe carbonyle}

Le mécanisme général d'une attaque nucléophile sur le groupe carbonyle des aldéhydes et des cétones peut s'effectuer avec un nucléophile très puissant, ou par catalyse acide en présence d'un nucléophile faible.

En général, les cétones réagissent moins aux nucléophile que les aldéhydes.
La première raison est d'ordre stérique.
L'atome de carbone carbonyle est effet plus encombré dans les cétones que dans les aldéhydes.
La deuxième raison est d'ordre électronique.
En effet, les groupes alkyles sont généralement électrodonneurs.
Par conséquent, ils ont tendances à atténuer la charge partielle positive de l'atome de carbone carbonylique, ce qui diminue sa réactivité vis-à-vis des nuléophiles.

\subsection{Addition de nucléophile oxygéné}

\begin{description}

  \item[L'addition d'eau].
    Chez la plupart des aldéhydes et des cétones, il est impossible d'isoler les hydrates puisque ces derniers perdent aisément une molécule d'eau et se retransforment en composés carbonylé, la constante d'équilibre étant largement inférieur à 1 à température ambiante.
    Les hydrates du méthanal et du tricholoroétanal sont des EXCEPTIONS.

  \item[L'addition d'alcool].
    Cela doit se passer en présence d'un catalyseur car les alcools sont de faibles nucléophile oxygénés.
    Le produit formé lors de l'addition d'un alcool sur un aldéhydes est un hémiacétal, dans lequel le même atome de carbone parte à la fois les fonctions alcool et éther.
    Le mécanisme se fait en trois étape, tout d'abord, la catalyseur acide protone l'atome d'oxygène.
    Ensuite, l'atome d'oxygène de l'alcool attaque l'atome de carbone du groupe carbonyle, Enfin, à la suite de la déprotonation de l'oxygène de l'alcool, l'hémiacétal se forme et l'ancien atome de carbone du groupe carbonyle devient un carbone stéréogénique, si la molécule originale n'est pas le méthanal.
\end{description}
Les fonctions aldéhydes et alcool impliqué dans la formation de l'hémiacétal peuvent également faire partie de la même molécules : l'addition nucléophile de l'alcool sera alors intermoléculaire, et l'hémiacétal formé sera cyclique.
Les composés dans lesquels le groupe hydroxyle est à  4 ou 5 atomes de carbone distance de la fonction aldéhyde ont tendance à former des hémiacétals cyclique car le cycle (5 ou 6), de part leur taille, est relativement exempt de tension.


En présence d'un excès d'alcool, la réaction se poursuit : le groupe hydroxyle (-OH) de l'hémiacétal est remplacé par un groupe alkoxyle  (-OR), il y a création d'un acétal, dans lequel le même atome de carbone porte simultanément deux fonctions éthers.
On peut favorisé la créations des produits lorsqu'on  utilise un excès d'alcool ou lorsqu'on retire continuellement l'eau du milieu réactionnel.


Les cétones réagissent également avec les alcools pour former d'abord des hemicétals, puis en présence d'un excès d'alcool, des cétals, selon le mêmes  étapes de mécanisme que celles des aldéhydes.

Une application est la protection du groupe carbonyle par addition Nu d'éthane-1,2-diol en milieu acide.

\subsection{Addition de nucléophiles azoté}

L'ammoniac, les amines, et certains composés apparentés comprennent un atome d'azote doté d'une paire d'électrons libres et peuvent ainsi tenir lieu de nucléophiles azotés faibles vis-à-vis de l'atome de carbone carbonylique des aldéhydes et des cétones.
Ainsi, la réaction d'addition en milieu acide d'une amine primaire sur des aldéhydes et des cétones conduit, par exemple, à la formation d'une imine.
Les imines jouent un role d'intermédiaire important dans certaines réactions biochimiques.

D'autre dérivés de l'ammoniac renferment un groupe $-NH_2$ réagissant avec les aldéhydes et les cétones selon des conditions semblables à celle des amines primaires.
Ces réactions produisent presque toujours des dérivés cristallins.
L'effet de résonance permet d'expliquer la plus grande stabilité des dérivés contenant un hétéroatome attaché à l'atome d'azote du C=N.


\section{Réactions des aldéhydes et des cétones : énolisation}

\subsection{Tautomérie des formes cétonique et énolique}
On appelle tautomérie céto-énolique  un type d'isomérie de structure, présent dans tous les aldéhydes et cétones ayant au moins un atome d'hydrogène sur l'atome de carbone adjacent au groupement carbonyle.
Les tautomères sont des isomères de structure et non les formes limites de résonances d'un composés.
Elles n'ont pas la même stabilité : la plupart des aldéhydes et des cétones sipmles existent surtout sous la forme cétonique.
Elle est plus stable par plus de 75kJ/mol.



\subsection{Acidité des aldéhydes et des cétones : création d'un ion énolate}

Un anion énolate est formé par élimination d'un atome H en $\alpha $ d'un aldéhyde ou d'une cétone.
L'atome d'hydrogène $\alpha$ d'un composés carbonylé est plus labile qu'un atome d'hydrogène d'un groupe alkyle.
Le fait de placer un groupe carboyle à proximité d'un atome d'hydrogène d'un groupe R entraine une augmentation spectaculaire de l'acidité du composé  d'un facteur $10^{30} $ !! Ils sont donc  des acides presque aussi fort que les alcools car :
\begin{itemize}
  \item Le groupement carboyle exerce un effet inductif attractif qui diminue la densité électronique du lien unissant l'atome d'hydrogène $ \alpha$ et l'atome de  carbone .
  \item L'anion résultant, appelé anion énolate, est stabilisé par résonance, et sa charge négative est répartie entre l'atome de carbone $ \alpha $ et l'atome d'oxygène du groupe carboyle.

\end{itemize}



\subsection{Condensation aldolique}

C'est une réaction des aldéhydes et des cétones qui implique le couplage d'un énol ( nucléophile) avec un dérivé carbonylé ( électrophile).
La condensation aldolique (un produit à la fois aldéhyde et un alcool) de l'éthanal procède par un mécanisme en trois étapes.
Tout d'abord, la base arrache un hydrogène $\alpha$ pour produire un ion énolate.
Ensuite, cet anion attaque l'atome de carbone carbonylique d'une autre molécule d'éthanal, ce qui conduit à la formation d'une nouvelle liaison C-C et à l'apparition d'un nouveau carbone stéréogénique.
Enfin, l'ion aldolate accepte un proton du solvant et régénère ainsi l'ion hydroxyde nécessaire pour la $1^{ière}$ étape.


Le chauffage des aldols en milieu acide conduit à la perte d'une molécule d'eau, et la double liaison est crée préférentiellement entre les atomes de carbone $\alpha$ et $\beta $ puisqu'il y a formation d'un système conjugué.
Avec l'éthanal, réactif de départ, on obtient ainsi un mélange des deux isomères géométrique ( Z et E) du but-2-énal.




Les condensations mixtes entre les aldéhydes de natures différentes et avec des cétones sont également possibles, mais il faut alors judicieusement choisir les substrats.

\section{Réactifs des aldéhydes et des cétones : oxydation et réduction}

\subsection{Réduction}
On produit faciement les aldéhydes et les cétones respectivement en alcools primaires et secondaires à l'aide d'hydrures métallique.
Les hydrures métalliques les plus couramment utilisés pour réduire les aldéhydes et les cétones sont l'aluminohydrure de lithium et le borohydrure de sodium.
on préfère le borohydrure de sodium car il peut être utilisé en milieu aqueux et il est plus stable.
Cette stabilité particulière permet d'ailleurs de réduire un aldéhyde ou une cétone sans faire réagir d'autre fonctions contenant un groupement carbonyle, comme les esters, les amides, et les carboxyliques.
On dt alors que le réactif est chimiosélectif.

\subsection{Oxydation}
Les aldéhydes s'oxydent beaucoup plus facilement que les cétones.
Il y a une distinction aisée des aldéhydes et cétones par leur réactivité différentes lors de l'oxydation.
Par exemples, le test De Tollens ou test au miroir d'argent.



\part{Aides Carboxylique et dérivés}

\section{Nomenclature des acides carboxyliques et de leurs dérivés }

\subsection{Nomenclature des acides carboxyliques}

Les acides carboxyliques sont constitués d'un groupement carboxyle ($CO_2H$), formé de l'union d'un groupement carbonyle (C=0) et d'un groupement hydroxyle (OH).
Ce groupement carboxyle est lié à un atomes d'hydrogène ou  un atome de carbone.

\subsection{Nomenclature des halogénures d'acyle}
Un halogénure d'acyle est un acide carboxylique où l'on a remplacé le groupement hydroxyle par un halogène.
On le nomme en spécifiant le nom de l'halogénure, suivi de la préposition "de" et du nom de l'acide carboxylique, où la terminaison "-oïque" est remplacée par "-oyle".
Si le nom de l'acide se termine par carboxylique, celui-ci sera remplacé par carbonyle.

\subsection{Nomenclature des anhydrides}
Un anhydride d'acide carboxylique est un acide carboxylique où l'on a remplacé le groupement hydroxyle par un ester.
Les anhydrides sont très souvent formés à partir de la condensation de deux molécules d'acide carboxylique; cela explique que le nom d'un anhydride est obtenu en nommant l'acide correspondant, mais on remplaçant le terme "acide" par "anhydride".

\subsection{Nomenclature des esters}

Un ester est un acide carboxylique où l'on a remplacé le groupement hydroxyle par un éther.
Les esters tirent donc leurs noms des acides carboxylique correspondant.
La terminaison "-oïque" est remplacé par "-oate".
Les esters cycliques sont des lactones.

\subsection{Nomenclature des amides}

Un amide est un acide carboxylique où l'on a remplacé le groupement hydroxyle par un atome d'azote relié à des atomes d'hydrogène ou de carbone.
On change la terminaison "-oïque" par "-amide".

\subsection{Nomenclature des nitriles}
Un nitrile est un amide déshydraté ; il est donc nommé comme les amides, en remplaçant la terminaison "-amide" par "nirile".

\section{Propriétés physiques des acides carboxylique}

Tout comme les alcools, les acides carboxylique sont des composés polaires qui établissent des liaisons hydrogènes entre eux ou avec d'autre molécules.
Ces liaisons hydrogène sont cependant plus forte entre les molécule carboxyliquen qu'entre celles des alcools à cause d la présence d'un groupement carbonyle qui augmente la polarité globale de la molécule.
Cela implique la température d'ébullition et la solubilité dans l'eau plus grandes des acides carboxylique par rapport à celles des alcools.

\section{Caractère acide des acides carboxylique}

Les acides carboxylique réagissent avec de l'eau pour donner naissance à un anion carboxylate et à un anion hydronium.
Plus la constante d'acidité $K_a$  d'un acide un élevée, plus ce dernier est acide et donc plus l'ion carboxylate formé est stable.


\subsection{Effet de résonance}

Dans l'ion éthanolate (alcool), la charge négative se trouve sur l'unique atome d'oxygène.
Elle ne peut être délocalisée d'aucune façon, ce qui est un facteur déstabilisation.
En revanche, dans l'ion éthanoate, la charge négative est répartie de manière égale entre les deux atomes d'oxygènes par résonance, de sorte que chaque atome d'oxygène de l'ion éthanoalte ne porte que la moitié de la charge totale.

\subsection{Effet inductif}
Un autre facteur est l'effet inductif produit par le groupement avoisinant le groupe carboxyle.
Un effet inductif attractif réparti la charge sur plusieurs centre et stabilise l'anion formé, ce qui augmente l'acidité de l'acide carboxylique.

\section{Caractère basique des acides carboxylique}

Généralement, la protonation s'effectue sur l'oxygène de la fonction carbonyle ( l'oxygène le plus basique) de manière à fournir un cation stabilisé par résonance.

\section{Préparation des acides carboxylique}
{Ozonolyse oxydante des alcènes}

{Oxydation de chaine alkyles latérales sur des cycles aromatique}

{Addition d'organomagnésiens au dioxyde de carbone}

{Oxydation d'un alcool primaire ou d'un aldéhyde}


\section{Réactivité des acides carboxylique}
Le groupement carboxyle d'un acide carboxylique est formé de deux atomes portant une charge partielle positive, qui peuvent chacun être attaqué par un réactif nucléophile et de deux atomes portant une charge partielle négative, qui peuvent chacun attaquer un réactif électrophile.
Dans un acide carboxylique, la polarisation du lien O-H incitera à privilégier l'atome d'hydrogène comme site d'attaque.

\section{Réactions des acides carboxyliques}
\subsection{Réaction acide-base : formation d'un sel}
Les acides carboxyliques sont des acides faibles, qui peuvent donc réagir avec de base forte.
Le sel formé est beaucoup plus polaire que l'acide carboxylique de départ, donc, beaucoup plus soluble dans l'eau.

\subsection{Réaction avec un alcool ou un phénol}
La réaction d'un acide carboxylique avec un alcool en présence d'un catalyseur acide est une transformation fortement étudiée et s'appelle estérification de Fischer.
Cette transformation crée un ester et de l'eau.
C'est une réaction réversible de condensation qui nécessite, en général, de chauffer le milieu réactionnels à reflux pendant plusieurs heure.



La réaction commence par une protonation de l'atome d'oxygène carbonylique, ce qui augmente la charge positive de l'atome de carbone carboxylique et le rend plus réactif vis-à-vis des nucléophile.

L'alcool en tant que nécléophile, attaque ensuite l'atome de carbone carbonylique de l'acide protoné et crée in intermédiaire tétraédrique.
C'est au cours de cette réaction que se forme une nouvelle liaison C-O.

Ensuite se produit l'échange rapide d'un proton entre les deux types d'atomes d'oxygènes de cet intermédiaires et l'éjection d'une molécule d'eau lorsqu'un des groupements OH est protoné.
La déprotonation du produit formé  permet au catalyseur de se recréer.

On peut déplacer l'équilibre vers la production d'éther en éliminant de l'eau au fur et à mesure de sa formation ou en engagent u large excès d'un des deux réactifs.

\subsection{Réaction avec une amine : formation d'un amide}
La réaction d'un acide carboxylique avec une amine crée, après chauffage, une amide.
La rencontre initiale entre l'acide carboxylique et l'amine, une base faible, crée un sel qui se déshydrate sous l'effet de la chaleur.
On peut utiliser l'ammoniac ou une amine primaire ou secondaire mais pas tertiaire, puisque cette dernière ne dispose pas, sur l'atome d'azote, de l'atome d'hydrogène nécessaire à l'achèvement de la réaction.

\subsection{Réduction des acides carboxyliques : formation d'alcool}

Ici le bobrohydrure de sodium n'est pas assez puissant pour effectuer la transformation d'un acide carboxylique en alcool.
Le phénomène de résonance présent dans le groupement carboxyle diminue en effet, la caractère positif du carbone caronylique, ce qui réduit la réactivité de ce dernier face à un nucléophile.
Dans ce cas, il faut utilisé l'aluminohydrure de lithium dans un solvant aprotique.
Il ne réduit pas les doubles ou les triples liaisons C-C, ce qui permet de transformer sélectivement un groupe carboxyle en présence d'alcènes ou d'alcynes.


\section{Propriété physique des  dérivés des acides carboxylique}
Il y a une modification des propriétés physique et une variation de la réactivité de la molécule.

\subsection{\'Etat physique}

Contrairement aux chlorure d'acyle et aux anhydride, très réactifs, les amides et les esters sont très répandu dans la nature.
Cependant, de tous ces dérivés, ce sont les amides qui montrent, à masse molaire comparable, les température d'ébullition les plus élevées, s'ils possèdent au moins une liaison N-H.
Cela est dû à leurs liaisons hydrogènes, plus fortes encore qu'entre les molécules d'acide carboxylique à cause de l'effet de résonance plus significatif impliquant l'atome d'azote.

\subsection{Odeurs}

Ceux-ci dégagent des odeurs agréables comparés aux acides carboxyliques



\section{Réactivité des dérivés des acides carboxylique}

Tous les dérivés des acides carboxyliques peuvent être transformés en acides carboxyliques mais pas à la même vitesse.
Ce ordre de réactivité s'explique par trois facteurs, l'effet de résonance impliquant le groupement carbonyle, l'effet inductif de l'hétéroatome lié au groupement carbonyle et à la facilité d'expulsion d'un groupe partant.
Plus l'énergie de l'orbitale associée au doublet d'électron libre est grande, plus l'effet de résonances est important, diminuant la charge positive du carbone carbonyique.
( amide < ester < chlorure d'acyle).

\section{Réactions des dérivés des acides carboxyliques}

\subsection{Réactions avec l'eau : formation des acides carboxyliques ou de leurs sels}

On peut transformer tous les dérivés des acides carboxyliques en acides carboxyliques en les faisant réagir avec de l'eau.
C'est une réaction d'hydrolyse.




Cependant les esters et les amides moins réactif, ne se transforment en acides carboxyliques  qu'en utilisant un milieu milieu aqueux acide ou basique et en chauffant.
L'hydrolyse acide des esters crée un acide carboxylique  et un alcool alors que l'hydrolyse acide des amines crée un acide carboxylique un une amine.
L'hydrolyse acide des esters est une réaction réversible.
Le mécanisme de la réaction est, tout d'abord une protonation de l'oxygène carnoylique, attaque nucléophile de l'eau et création de l'intermédiaire tétraédrique, transfert de proton, expulsion du groupe partant et déprotonation.
Il est similaire à la formation des esters.



L'hydrolyse basique d'un ester un appelée sponification parce que la réaction est impliquée dans la fabrication de savon.
On obtient alors son sel, ce qui est également le cas lorsqu'on utilise le même processus sur un amide.
Dans les deux situations , le nucléophile $OH^-$ se lie au carbone carbonylique et crée un intermédiaire tétraédrique : cet intermédiaire expulse le groupe partant, qui devient dans la solution une base plus forte que l'ion carboxylique.

\subsection{Réaction avec les alcools ou les phénols : formations d'esters}

Tous les composés plus réactifs que les esters, soit les chlorures d'acyle, les anhydrides, les acides carboxyliques et, dans certaine situations particulières, les esters eux-mêmes régissent avec des alcools pour former des esters.
La réaction avec un chlorure d'acyle et un anhydride est habituellement menée en présence d'une base faible non nucléophile ( pyridine) pour capter le HCl libéré.
Les autres réactions nécessitent l'utilisation d'un catalyseur acide et se font un mécanisme similaire à celui de l'estérification des Fischer.


La transestérification est une transformation très utilisée en industrie : elle est possible lorsque l'alcool produit possède une température d'ébullition beaucoup moins élevée que celle de l'alcool utilisé comme réactif, ce qui permet son élimination par distillation.

\part{Lipides}

\section{Classification des lipides}
Ils se caractérisent par leur insolubilité dans l'eau.
On regroupe les lipides en deux grandes classes.
On trouve ainsi dans la nature des lipides saponifiable, contenant une ou deux fonctions esters et dans l'hydrolyse basique(saponification) crée des sels d'acides carboxylique à longue chaine(sels d'acides gras).
Cette catégorie renferme des glycérides, des cires et des phosphoglycérides.

Toutefois on trouve également des lipides non saponifiables, ne contenant pas de fonction ester : les terpènes, les stéroïdes et les prostaglandines.

\section{Lipides saponifiables}

\subsection{Structure des lipides saponifiables}

\paragraph{Triglycérides}

Parmi les lipides saponifiables, les graisses et les huiles sont des substances qui nous sont familières.
Les graisses sont principalement d'origine animale et les huiles sont principalement d'origine végétale.
Les graisses et les huiles sont des généralement constitués d'un mélange complexe de triglycérides.
La plupart des acides gras naturels sont non ramifiés et referment un nombre pair d'atome de carbone (compris entre 4 et 28).


\paragraph{Cires}

Les cires sont des monoesters formés par condensation d'un acide gras et d'un alcool saturé à longue chaine carbonée.
Les résidus acide gras et alcool d'une molécule de cire comportent habituellement de 12 à 34 atomes de carbone

\paragraph{Phospholipides}
Les lipides saponifiables dans lesquels on trouve de l'azote ou du phosphore sont appelés lipides complexes : on y trouve notamment les phospholipides et les sphingolipides.
Les phospholipides sont très abondants dans la nature, puisqu'ils constituent environ $40 \%$ de la masse de substance organique présentes dans les membranes cellulaires.

\subsection{Réaction des lipides saponifiables}

\paragraph{Hydrolyse acide}

L'hydrolyse acide d'un triglycéride implique celle des groupements fonctionnels esters de la molécule.
L'équilibre sera déplacé vers la droite si l'on veille à travailler avec un grand excès d'eau.

\paragraph{Hydrolyse basique}
Nous avons vu que le chauffage d'un ester en milieu aqueux basique crée un alcool et un sel d'acide carboxylique.
Le même type de réaction sur un triglycéride transforme celui-ci en glycérol et en sels d'acides gras.
Les sels d'acides gras à longue chaine formés par cette réaction ont une tête polaire et une queue apolaire : ce caractère amphiphile, déjà mentionné pour les phospholipides, leur permet d'agir comme des savons.

\section{Savons et détergents}

\subsection{Mode d'action des savons}
Les molécules de savon comportent une queue apolaire et lipophile, et une tête polaire hydrophile.
Quand on ajoute un mélange d'eau et de savon, on forme une dispersion colloïdale, et non une véritable solution.
Ces solutions de savon contiennent des agrégats de molécules se savon appelés micelles.

Pour détacher la saleté qui adhère à une mince pellicule d'huile à la surface de la peau ou des vêtements, les molécules de savon de la micelle émulsionnent les gouttelettes d'huile ou de graisse, puis les micelles s reforment en emprisonnant dans leur centre ces gouttelettes d'huile  ou de graisses.
les queues lipophiles des molécules de savon solubilisent le composant graisseux apolaire par des forces de London, alors que les têtes hydrophiles se projettent hors de la gouttelettes d'huile et se baignent dans l'eau.

\part{Amines}

\section{Classification et nomenclature des amines}

\subsection{Classification des amines}

Une amine est fondamentalement de l'ammoniac dans lequel on a remplacé au moins un atome d'hydrogène par un groupe R, alkyle ou aryle.
Pour des raisons pratiques, les amines sont dites primaires, secondaire ou tertiaire, selon que l'atome d'azote porte un, deux ou trois groupes carbonés R.

\subsection{Nomenclature des amines}
Une première méthode standard, utilise pour l'atome d'azote le suffixe "-amine", que l'on accole au nom de la chaine principale.
Dans le cas où la fonction amine n'est pas la fonction principale de la molécule, on utilise le préfixe "-amino"

\section{Structure et propriétés physiques des amines}
\subsection{Structure des amines}
L'atome azote des amines est trivalent et possèdent en outre une paire d'électron libres.
Les orbitales de l'azote sont donc hybridée $sp^3$ et placée selon les axes d'un tétraèdre autour de ce dernier : la présence du doublet libre implique que les trois substituants autour de l'atome d'azote adopteront une géométrie pyramidale.


\subsection{Propriétés physiques des amines}
Tout comme pour les autres composés organique, les amines ont une température d'ébullition qui croit avec l'augmentation de la longueur de la chaine portant le groupe fonctionnel.

Toutefois, la variation est plus irrégulière lorsqu'on remplace successivement chaque atome d'hydrogène sur l'atome d'azote d'une amine primaire par un substituant carboné.
Les amines primaires ont un point d'ébullition beaucoup plus élevé qui celui des alcanes de masse molaire comparable, mais il est toutefois moins élevé que celui des alcools pour une masse molaire comparable.

\section{Basicité des amines}

\subsection{Expression de la constante de basicité}

Les amines sont des analogues azotés des alcools.
La plus petite électronégativité de l'azote signifie que son doublet libre est retenu moins fortement, ce qui rend les amines beaucoup plus basique et nucléophie que les alcools.
Ansi, en solution aqueuse, la collision entre une molécule d'eau et une amine entraine le transfert d'un proton de l'une vers l'autre.


On exprime la basicité de l'amine en solution aqueuse par sa constante de basicité $K_b$ .

\subsection{Influence de l'effet inductif sur la basicité des amines}
Le remplacement d'un des atomes d'hydrogène de l'ammoniac par un groupement alkyle électrodonneur augmente la densité électronique sur l'atome d'azote, ce qui rend l'amine primaire beaucoup plus basique que l'ammoniac.
En général, les groupe électrodonneur augmentent les basicité des amines et les groupes électroattracteurs la réduisent.
Attention au amine tertiaire avec certain stabilisation de l'ion ammonium avec le solvant.



\subsection{Influence de l'effet mésomère sur la basicité des amines}

La présence d'un cycle aromatique adjacent à l'atome d'azote portant le doublet libre modifie considérablement la basicité de l'amine.
Ainsi le passage de la cyclohexanamine à l'aniline diminue la basicité de cette dernière par un facteur d'un million.
Cette remarquable différence est due à la délocalisation par effet mésomère du doublet libre dans l'aniline, phénomène impossible dans la cyclohxanamine.
La délocalisation du doublet libre diminue la densité électronique sur l'atome d'azote et, par le fait même, la basicité de l'amine.

Attention lorsque l'atome d'azote est hybridé $sp^2$ comme dans la pyrrole.


\subsection{Réaction avec un acide : formation d'un sel d'ammonium}
Les amines sont généralement basiques à moins qu'un effet de résonance ne vienne diminuer appréciablement la densité électronique sur l'atome d'azote, comme c'est le cas pour la pyrrole ou pour les amides.

De plus, si ces amines ne sont pas hydrosolubles, leur caractère basique nous permet de les séparer  des composés neutre ou acides.
Il s'agit, dans un premier temps, de mettre le mélange de substances à séparer en contact avec une solution aqueuse acide : les amines basiques seront alors transformées en sels d'ammonium.



\part{Les protides}
\section{Classification des protides}

Les protides constituent près de 50$\%$ de la masse sèche des cellules.
On les retrouve sous forme de protéines, de peptides et d'acides aminés.


\section{Nature et classification des acides aminés}

\subsection{Acides $\alpha$-aminés génétiquement codés}
L'analyse des acides aminés montre qu'ils comportent au moins un groupement fonctionnel acide carboxylique et un groupement fonctionnel amine situés sur le carbone adjacent ( $\alpha$) au groupement carboxyle :  ce sont donc ce qu'on appelle des acide $\alpha$-aminés.

\section{Propriétés acidobasiques des acides aminés}
Les acides aminés ont des propriétés fort remarquable qui les démarquent des acides carboxyliques et des amines dont ils sont issus.

Les acides carboxyliques sont des acides en solution aqueuse, alors que les amines sont basiques.
La présence simultanée de ces deux groupements fonctionnels rend l'acide aminé amphotérique.
c'est-à-dire qu'il
se comporte à la fois comme acide et comme une base.
Il se crée ainsi en solution une espèce particulière que l'on appelle un Zwitterion, ou sel interne, dans lequel le groupe amino est protoné et le groupe carboxyle a perdu son proton.

Pour chaque acide aminé, il existe toutefois un et un seul pH particulier et distinct, le point isoélectrique, pour lequel on ne rencontre effectivement en solution que la forme Zwitterion.


\section{Nature et classification des peptides et des protéines}
\subsection{Nature de la liaison peptides}
Dans la nature, les acides aminés sont le plus souvent rattachés les uns au autres pour former des peptides et des protéines.
Dans ces molécules, les acides aminés sont reliés les uns aux autres par des liaisons amides, appelées liaisons peptidiques, établies entre le groupe carboxyle d'un acide aminé et le groupe $\alpha$-aminés d'un autre acide aminé.

\section{Niveaux de structures}
\subsection{Structure primaire}
La structure primaires des polypeptides et des protéines correspond tout simplement à la séquence des résidus d'acides aminés de ces derniers.
On représente la structure primaire des polypeptides ou des protéines par une chaine de symboles à trois lettres désignant les résidus d'acides aminés, de l'acide aminé N-terminal à l'acide aminé C-terminal.


\subsection{Structures secondaires}
Les amides établissent facilement des liaisons hydrogène intermoléculaire impliquant le groupe carbonyle et le groupe N-H.
Dans cette structure,  il y a les forme en hélice $\alpha$ et les feuillets $\beta$.

\subsection{Structure tertiaires}
La grande majorité des polypeptides et des protéines existent sous forme globulaire, où l'on retrouve sur l'ensemble de la chaine polypeptidique les différentes structures secondaires.
Ce repliement de la chaine dans l'espace confère une certaine forme tridimensionnelle au polypeptide ou à la protéine.

\subsection{Structure quaternaire}
Certaine protéines de masse élevée existent sous la forme d'un assemblage de plusieurs sous unité.
L'amas qu'elles forment représente la structure quaternaire de la protéines Dans la cellule, l'agrégation permet aux portions apolaires situés à la surface de la protéines de ne pas être exposés au milieu aqueux.

\section{Préparation des peptides et des protéines}
Les acides aminés sont fondamentalement bifonctionnels.
Puisque la création de la structure primaire implique la formation de liaisons peptidiques dirigées, il faudra lier spécifiquement le groupement carboxyle d'un acide aminé au groupement amino d'un second acide aminé, sans faire intervenir les autres groupements carboxyle et amino.
L'utilisation de groupements protecteurs permet de réussir ce processus, que l'on peut résumer en trois étapes.
La protection des groupements fonctionnels non désignés pour la condensation dans chaque acide aminé, la condensation des groupements carboxyle et amine non protégés par formation du lien peptidique et, finalement, la déprotonation des groupements fonctionnels protégés.
\end{document}
