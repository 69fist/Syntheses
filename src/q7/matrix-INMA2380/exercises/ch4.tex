\section{Inertia and stability of matrices}
\exo{3}
\begin{solution}
  Using the exercise~1.6 and the exercise~1.15,
  \begin{align*}
    V^* (I_n \otimes A^* + A^T \otimes I_n) V
    & = ((U^*)^T \otimes U^*) (I_n \otimes A^* + A^T \otimes I_n) (U^T \otimes U)\\
    & = ((U^*)^T \otimes U^*) (I_n \otimes A^*) (U^T \otimes U)\\
    & \quad + ((U^*)^T \otimes U^*) (A^T \otimes I_n) (U^T \otimes U)\\
    & = ((U^*)^TU^T) \otimes (U^*A^*U) + ((U^*)^TA^TU^T) \otimes (U^*U)\\
    & = I_n \otimes A_S^* + A_S^T \otimes I_n.
  \end{align*}
  We know that since $U$ is unitary, $U^T$ is unitary too.
  With the help of the exercise~1.16, we conclude that $U^T \otimes U$ is unitary.
  Therefore,
  $I_n \otimes A^* + A^T \otimes I_n$ has same spectrum than
  $I_n \otimes A_S^* + A_S^T \otimes I_n$.
  We can now see that
  \begin{align*}
    A_S & =
    \begin{pmatrix}
      \lambda_1 &           &        & 0\\
      \times    & \lambda_2 &        & \\
      \vdots    & \ddots    &        & \\
      \times    & \cdots    & \times & \lambda_n
    \end{pmatrix}\\
    I_n \otimes A_S^* & =
    \begin{pmatrix}
      A_S^* &       &        & 0\\
            & A_S^* &        & \\
            &       & \ddots & \\
      0     &       &        & A_S^*
    \end{pmatrix}\\
    A_S^T \otimes I_n & =
    \begin{pmatrix}
      \lambda_1 I_n & \times I_n    & \cdots & \times I_n\\
                    & \lambda_2 I_n & \ddots & \vdots\\
                    &               & \ddots & \times I_n\\
      0             &               &        & \lambda_n I_n
    \end{pmatrix}
  \end{align*}
  $I_n \otimes A_S^* + A_S^T \otimes I_n$ is therefore
  upper diagonal matrices with the $in + j$th element of its
  diagonal being $\overline{\lambda_j} + \lambda_i$.
  Since it is upper diagonal, it is in its Schur form
  and the elements in its diagonal are its eigenvalues.
\end{solution}

\exo{3}
\begin{solution}
  Using the exercise~1.6 and the exercise~1.15,
  \begin{align*}
    V^* (A^T \otimes A^* + I_n \otimes I_n) V
    & = ((U^*)^T \otimes U^*) (A^T \otimes A^* + I_n \otimes I_n) (U^T \otimes U)\\
    & = ((U^*)^T \otimes U^*) (A^T \otimes A^*) (U^T \otimes U)\\
    & \quad + ((U^*)^T \otimes U^*) (I_n \otimes I_n) (U^T \otimes U)\\
    & = (((U^*)^TA^TU^T) \otimes (U^*A^*U) + (U^*)^TU^T) \otimes (U^*U)\\
    & = A_S^T \otimes A_S^* + I_n \otimes I_n.
  \end{align*}
  We know that since $U$ is unitary, $U^T$ is unitary too.
  With the help of the exercise~1.16, we conclude that $U^T \otimes U$ is unitary.
  Therefore,
  $A^T \otimes A^* + I_n \otimes I_n$ has same spectrum than
  $A_S^T \otimes A_S^* + I_n \otimes I_n$.
  We can now see that
  \begin{align*}
    A_S & =
    \begin{pmatrix}
      \lambda_1 &           &        & 0\\
      \times    & \lambda_2 &        & \\
      \vdots    & \ddots    &        & \\
      \times    & \cdots    & \times & \lambda_n
    \end{pmatrix}\\
    A_S^T \otimes A_S^* & =
    \begin{pmatrix}
      \lambda_1 A_S^* & \times A_S^*    & \cdots & \times A_S^*\\
                      & \lambda_2 A_S^* & \ddots & \vdots\\
                      &                 & \ddots & \times A_S^*\\
      0               &                 &        & \lambda_n A_S^*
    \end{pmatrix}
  \end{align*}
  $A_S^T \otimes A_S^* + I_n \otimes I_n$ is therefore
  upper diagonal with the $in + j$th element of its
  diagonal being $\overline{\lambda_j}\lambda_i + 1$.
  Since it is upper diagonal, it is in its Schur form
  and the elements in its diagonal are its eigenvalues.
\end{solution}

\exo{2}
\begin{solution}
  Let $A_s$ be the Schur form of $A$ and $B_S$ the lower Schur
  form of $B$.
  We have unitary $U_A, U_B$ such that
  $A = U_AA_SU_A^*$ and $B = U_BB_sU_B^*$.
  Let's define the dense matrices $\tilde{X}$ and $\tilde{C}$ such that
  $\tilde{X} = U_A\tilde{X}U_B^*$ and $C = U_A\tilde{C}U_B^*$, we have
  \begin{align*}
    (U_AA_SU_A^*)(U_A\tilde{X}U_B^*) + (U_A\tilde{X}U_B^*)(U_BB_sU_B^*)
    & = U_A\tilde{C}U_B^*\\
    U_AA_S\tilde{X}U_B^* + U_A\tilde{X}B_sU_B^* & = U_A\tilde{C}U_B^*\\
    A_S\tilde{X} + \tilde{X}B_s & = \tilde{C}\\
  \end{align*}
  Using $\vect(BPA) = (A^T \otimes B) \vect(P)$, we have
  \begin{align*}
    \vect(A_s\tilde{X} + \tilde{X}B_S) & = \vect(\tilde{C})\\
    \vect(A_s\tilde{X}I_n) + \vect(I_n\tilde{X}B_S) & = \vect(\tilde{C})\\
    (I_n \otimes A_s) \vect(X) + (B_S^T \otimes I_n) \vect(\tilde{X}) & = \vect(\tilde{C})\\
    (I_n \otimes A_s + B_S^T \otimes I_n) \vect(\tilde{X}) & = \vect(\tilde{C})
  \end{align*}
  where $(I_n \otimes A_s + B_S^T \otimes I_n)$ is upper diagonal
  since $A_s$ and $B_S^T$ are upper diagonal.
\end{solution}

\exo{2}
\begin{solution}
  For 4.7, let
  \[ \lambda = \argmin_{\lambda^* = j\omega} \sigma_{\mathrm{min}}(A - \lambda^*I). \]
  Let $C = A - \lambda I$.
  If $C = U^* \Sigma V$, we can take $\Delta = -u_n \sigma_n v_n^*$ which gives
  \[ C + \Delta = A - \lambda I + \Delta = (A + \Delta) - \lambda I \]
  of rank $n-1$.

  For 4.8, it is the same except that we take
  \[ \lambda = \argmin_{\lambda^* = \exp(j\omega)} \sigma_{\mathrm{min}}(A - \lambda^*I). \]
\end{solution}

\exo{0}
\nosolution
