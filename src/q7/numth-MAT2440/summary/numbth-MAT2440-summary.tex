\documentclass[en]{../../../eplsummary}

\makeatletter
\def\legendre@dash#1#2{\hb@xt@#1{%
  \kern-#2\p@
  \cleaders\hbox{\kern.5\p@
    \vrule\@height.2\p@\@depth.2\p@\@width\p@
    \kern.5\p@}\hfil
  \kern-#2\p@
  }}
\def\@legendre#1#2#3#4#5{\mathopen{}\left(
  \sbox\z@{$\genfrac{}{}{0pt}{#1}{#3#4}{#3#5}$}%
  \dimen@=\wd\z@
  \kern-\p@\vcenter{\box0}\kern-\dimen@\vcenter{\legendre@dash\dimen@{#2}}\kern-\p@
  \right)\mathclose{}}
\newcommand\legendre[2]{\mathchoice
  {\@legendre{0}{1}{}{#1}{#2}}
  {\@legendre{1}{.5}{\vphantom{1}}{#1}{#2}}
  {\@legendre{2}{0}{\vphantom{1}}{#1}{#2}}
  {\@legendre{3}{0}{\vphantom{1}}{#1}{#2}}
}
\def\dlegendre{\@legendre{0}{1}{}}
\def\tlegendre{\@legendre{1}{0.5}{\vphantom{1}}}
\makeatother

\hypertitle{Number Theory}{7}{MAT}{2440}
{Benoît Legat}
{Olivier Pereira, Jean-Pierre Tignol}

\section{Modular arithmetic}
\subsection{Foundations}
\begin{mytheo}[Fundamental theorem]
  All positive integer can be uniquely written as a product of prime numbers.

  Said differently, for all $n \in \mathbb{N}_0$ there are unique prime numbers $p_1 < p_2 < \ldots < p_k$
  and positive integers $e_1, e_2, \dots, e_k$ such that
  $$n = p_1^{e_1} \cdot p_2^{e_2} \cdot \ldots \cdot p_k^{e_k}.$$
\end{mytheo}

\begin{mytheo}
  If $n \in \mathbb{N}_0$ has the decomposition
  $$n = p_1^{e_1} \cdot p_2^{e_2} \cdot \ldots \cdot p_k^{e_k}$$
  then $n$ has exactly $(e_1+1)\cdot(e_2+1)\cdot \ldots \cdot (e_k+1)$ positive divisors.
\end{mytheo}

\begin{mytheo}[Bezout]
  Let $a,b \in \mathbb{N}_0$, and $c \in \mathbb{Z}$.
  There are $x, y \in \mathbb{Z}$ such that
  $$ ax+by=c$$
  iff $c$ is a multiple of $(a,b)$.
\end{mytheo}

\subsection{The ring $\Z/n\Z$}
Let
\[ x + n\Z \eqdef \{z + nk | n \in \Z\} \]
with the addition and multiplication
\begin{align*}
  (x + n\Z) + (y + n\Z) & = (x + y) + n\Z\\
  (x + n\Z) \cdot (y + n\Z) & = (xy) + n\Z.
\end{align*}
The set
\[ \Z/n\Z \eqdef \{x + n\Z | x \in \Z\} = \{x + n\Z | x \in \{0, \ldots, n-1\}\} \]
is a commutative ring.

We also define
\[ (\Z/n\Z)^\times \eqdef \{x + n\Z | (x,n) = 1\}. \]
$(\Z/n\Z)^\times$ is an abelian (commutative) group.

\subsection{Chinese Remainder Theorem (CRT)}
\begin{mytheo}[CRT]
  If $n_1, n_2, \dots, n_k > 1$ relatively primes numbers, and if $a_1, a_2, \dots, a_k$ are integers,
  then there is an unique $x$ modulo $n_1\cdot \ldots \cdot n_k$ such that
  \begin{align*}
    x & \equiv a_1 \pmod {n_1} \\
    x & \equiv a_2 \pmod {n_2} \\
      & \vdots  \\
    x & \equiv a_k \pmod {n_k}.
  \end{align*}

  Equivalently, the CRT states that the map
  \begin{align*}
    \Z/n_1 \cdots n_r\Z & \to (\Z/n_1\Z) \times \cdots \times (\Z/n_r\Z), &
    x + n_1 \cdots n_r\Z & \mapsto (x + n_1\Z, \ldots, x + n_r\Z)
  \end{align*}
  is a bijection.
\end{mytheo}
The solution is obtained by the principle of superposition.
It is
\[ \sum_{i=1}^k \left(\prod_{j \neq i}n_j\right) y_i \]
where $y_i$ is the solution of
\[ \left(\prod_{j \neq i}n_j\right)y_i \equiv b_i \pmod{n_i}. \]
We see that $y_i$ is unique modulo $n_i$ so
\[ \left(\prod_{j \neq i}n_j\right) y_i \]
is unique modulo $n_1 \cdot \cdots \cdot n_k$.

\subsubsection{Generalization}
If there is $n_i,n_j$ such that $(n_i,n_j) = r \neq 1$,
we can split the 2 equations in 4.
\begin{align*}
  x & \equiv a_i \pmod{n_i/r}\\
  x & \equiv a_i \pmod{r}\\
  x & \equiv a_j \pmod{r}\\
  x & \equiv a_j \pmod{n_jr}.
\end{align*}
\begin{itemize}
  \item If $a_i \equiv a_j \pmod{r}$,
    then we have 3 equations instead of 2 and $(n_i/r,n_j/r) = 1$.
    We can now check $(n_i/r,r)$ and $(n_j/r,r)$ to see if we have to split more.
    At most one of them is different than 1.
  \item Else, the system has no solution.
\end{itemize}

A more general approx is to do the decomposition of $n_i$ and to split it
in each of its prime factors.
If $n_i = \prod_{j=1}^{k_i} p_j^{e_j}$, we split the equation in
$k_i$ equations.

Then for each prime $p$, we have equations modulo $p, p^2, p^3, \ldots, p^{f_p}$.
We need to check whether there are contradictions before dropping equations in $p^j$ for $j < f_p$.
It can be done by the following algorithm:

For $j = f_p, \ldots, 2$, we check that all equations modulo $p^j$ are the same and then
add an equation modulo $p^{j-1}$ with the remainder of the equation modulo $p^j$
(e.g. from $x \equiv a \pmod{p^j}$, we add $x \equiv a \pmod{p^{j-1}}$).

Then we check that all the equations modulo $p$ are the same and drop them.

If one check fails, that means the system has no solution.

The $n_i$ of the remaining system are powers of different primes and are therefore relatively prime.

\subsection{Euler's totient function}
\begin{mydef}
  $\varphi(n)$ est défini comme l'ordre de $(\Z/n\Z)^\times$, i.e.
  \begin{align*}
    \varphi(n) & \eqdef |(\Z/n\Z)^\times|\\
               & = |\left\{ k \in \left\{1, 2, \ldots, n \right\} : (k, n) = 1\right\}|,
  \end{align*}
  and $\varphi(1) \eqdef 1$.
\end{mydef}

\begin{myprop}
  If $p$ is prime, then $\varphi(p^m) = p^{m-1}(p-1)$
\end{myprop}
\begin{myprop}
  If $(a,b)=1$, then $\varphi(ab) = \varphi(a)\cdot\varphi(b)$.
\end{myprop}

\begin{mycorr}
  If $n = p_1^{e_1} \cdot p_2^{e_2} \cdot \ldots \cdot p_k^{e_k}$,
  then $\varphi(n) = p_1^{e_1 - 1}(p_1 - 1) \cdot \ldots \cdot p_k^{e_k - 1}(p_k - 1)$
  \begin{align*}
    \varphi(n) & = p_1^{e_1 - 1}(p_1 - 1) \cdot \ldots \cdot p_k^{e_k - 1}(p_k - 1)\\
               & = n \cdot\left(1 - \frac{1}{p_1}\right)\cdot \ldots \cdot \left(1 - \frac{1}{p_k}\right)
  \end{align*}
\end{mycorr}

\begin{myprop}
  For all $n \in \Z$,
  \[ n = \sum_{d | n,d \geq 1} \varphi(d). \]
\end{myprop}

\begin{mycorr}
  $n$ is prime iff $\varphi(n) = n-1$.
\end{mycorr}

\begin{mytheo}[Euler-Fermat]
  If $(a,n) = 1$, then $$a^{\varphi(n)} \equiv 1 \pmod n$$ where $\varphi$ désigne la fonction indicatrice d'Euler.
\end{mytheo}

\begin{mycorr}[Fermat]
  If $a$ is not divisible by a prime $p$, then
  $$a^{p-1} \equiv 1 \pmod p.$$
\end{mycorr}

\subsubsection{Multiplicative order}
\begin{mydef}
  Let $a,n \in \Z$ with $n \geq 1$.
  If $(a,n)=1$, the multiplicative order of $a$ modulo $n$ $o_n(a)$
  is defined to be the least integer $e > 0$ such that $a^e \equiv 1 \pmod{n}$.
\end{mydef}

\begin{itemize}
  \item If $(a,n) = 1$, then $a^m \equiv 1 \pmod n$ with $m \in \mathbb{N}$ iff $o_n(a)$ divides $m$.
  \item If $(a,n) = 1$, $o_n(a)$ divides $\varphi(n)$.
  \item If $(a,n) = 1$, the numbers $a^0, a^1, a^2, \ldots, a^{o_n(a)-1}$ are all distincts modulo $n$.
\end{itemize}

\begin{mytheo}[Gauss]
  For every prime number $p$,
  there exists an integer $g$ such that $o_p(g) = p-1$.
  It is called a primitive root of $p$.
\end{mytheo}

We have actually $\varphi(d)$ numbers $g$ such that $o_p(g) = d$
if $d|(p-1)$
and once we have a number $g$ of order $d$
all the other number $h$ of order $d$ are given by
\[ \{h | (h + p\Z) \in (\Z/p\Z)^\times, o_p(h) = d\} = \{g^e | e \in \{0,\ldots, d-1\}, (e,d) = 1\}. \]

We also have
\[ (\Z/p\Z)^\times = \{g^e + p\Z |e = 0, \ldots, p-2, \}. \]

We can therefore solve
\[ x^a \equiv b \pmod{p} \]
using a primitive root $g$ of $p$.
We find $b'$ such that $2^{b'} = b$
and we pose $x = g^y$ such that
\[ g^{ay} \equiv g^{b'} \pmod{p} \]
which is equivalent to
\[ ay \equiv b' \pmod{o_p(g) = p-1} \]

\subsection{Quadratic residues}
In this section, we will only consider the odd primes $p$.
\begin{mydef}
  Let $p$ be an odd prime.
  The Legendre symbol of $a$ modulo $p$ is
  $\legendre{a}{p} \in \{-1,1\}$.

  We say that $a$ is a quadratic residue of $p$ if
  $\exists b$ such that $b^2 \equiv a \pmod{p}$,
  in that case $\legendre{a}{p} = 1$.
  Otherwise we say that $a$ is a quadratic nonresidue
  and $\legendre{a}{p} = -1$.
\end{mydef}

\begin{myprop}
  There are $\frac{p-1}{2}$ quadratic residues
  and $\frac{p-1}{2}$ quadratic nonresidues.
\end{myprop}

\begin{mydef}[Jacobi symbol]
  If $n > 1$ is odd and has the decomposition
  $$n = p_1^{e_1} \cdot p_2^{e_2} \cdot \ldots \cdot p_k^{e_k}$$
  then
  \[ \legendre{a}{n} \eqdef {\legendre{a}{p_1}}^{e_1} \cdot {\legendre{a}{p_2}}^{e_2} \cdot \cdots \cdot {\legendre{a}{p_k}}^{e_k}. \]
\end{mydef}
However, $\legendre{a}{n} = 1$ does \emph{not} mean that $a$ is a quadratic residue modulo $n$.
$\legendre{a}{n} = -1$ does however implies that $a$ is not a quadratic residue for some $p_i$
and therefore not for $n$ either.

We have the following properties for $p > 1$ odd prime, and $(a,p) = 1$
\begin{align*}
  \legendre{a}{p} & \equiv a^{\frac{p-1}{2}} \pmod{p}\\
  \legendre{a}{p} & = \prod_{r = 1}^{\frac{p-1}{2}} \varepsilon(ar)
\end{align*}
where
\[
  \varepsilon(x) =
  \begin{cases}
    1 & \text{if }x \equiv r \pmod{p}\text{ where } 1 \leq r \leq \frac{p-1}{2}\\
    -1 & \text{otherwise}
  \end{cases}
\]
and the following properties for $n > 1$ odd, and $(a,n) = 1$
\begin{align*}
  \legendre{ab}{n} & = \legendre{a}{n}\legendre{b}{n}\\
  \legendre{-1}{n} & = (-1)^{\frac{n-1}{2}}\\
                   & =
  \begin{cases}
    1 & \text{if }n \equiv 1 \pmod{4}\\
    -1 & \text{if }n \equiv 3 \pmod{4}
  \end{cases}\\
  \legendre{2}{n} & =
  \begin{cases}
    1 & \text{if }n \equiv \pm 1 \pmod{8}\\
    -1 & \text{if }n \equiv \pm 3 \pmod{8}.
  \end{cases}\\
  \legendre{a}{n}\legendre{n}{a}
  & = (-1)^{\frac{a-1}{2}\frac{n-1}{2}}.
\end{align*}

\section{$p$-adique numbers}
\begin{mydef}[$p$-adique integer]
  A sequence of integers $(a_i)_{i \in \mathbb{N}}$ is said to be
  $p$-\emph{coherent} if $a_n \equiv a_{n-1} \pmod{p^n}$
  and $(a_i)_{i \in \mathbb{N}}$ and $(b_i)_{i \in \mathbb{N}}$
  are called $p$-\emph{equivalent} if $a_n \equiv b_n \pmod{p^{n+1}}$
  for all $n \in \mathbb{N}$.

  A $p$-\emph{adic} \emph{integer} is a $p$-equivalence class of $p$-coherent sequences.
\end{mydef}
$\mathbb{Z}_p$ is the set of $p$-\emph{adic} integers.

Every $p$-coherent sequence is $p$-equivalent to a unique $p$-coherent sequence of the form
\[ (\alpha_0, \alpha_0 + \alpha_1 p, \alpha_0 + \alpha_1p + \alpha_2p, \ldots) \]
with $\alpha_i \in \{0, \ldots, p-1\}$.
It can be regarded as the canonical element of the $p$-equivalence class.
As a consequence, we use the notation
\[ \sum_{i=0}^\infty \alpha_i p^i \]
to denote this $p$-adique integer.

\begin{mydef}[Residue]
  For every $p$-adic integer $\alpha \in \mathbb{Z}_p$,
  we define its residue $\overline{\alpha} \in \mathbb{Z}/p\mathbb{Z}$ to be
  \[ \overline{\alpha} = a_0 + p\mathbb{Z} \]
  where $(a_i)_{i \in \mathbb{Z}}$ represents $\alpha$.
\end{mydef}

\begin{myprop}
  $\alpha \in \mathbb{Z}_p$ has an inverse in $\mathbb{Z}_p$ iff $\overline{\alpha} \neq 0$.
  We denote the set of invertible elements by $\mathbb{Z}_p^\times$.
\end{myprop}

Every element nonzero element of $\alpha = \mathbb{Z}_p$ can be written as $\alpha = p^nu$ for some $n \in \mathbb{N}$
and some $u$ in $\mathbb{Z}_p^\times$.

\begin{mydef}[$p$-adique number]
  A $p$-adique number is a fraction of $p$-adique integers.
\end{mydef}
Since the denominator may be not invertible, all $p$-adique integers are not $p$-adique numbers.

We have actually
\begin{align*}
  \mathbb{Z} \subset \mathbb{Z}_p & = \{p^nu | n \in \mathbb{N}, u \in \mathbb{Z}_p^\times\} \cup \{0\}\\
  \mathbb{Q} \subset \mathbb{Q}_p & = \{p^nu | n \in \mathbb{Z}, u \in \mathbb{Z}_p^\times\} \cup \{0\}.
\end{align*}
$\mathbb{Z}_p$ and $\mathbb{Q}_p$ are both uncountable.

\begin{myprop}
  $p^nu$ with $u \in \mathbb{Z}_p^\times$ is square iff $n$ is even and $\legendre{\overline{u}}{p} = 1$.
\end{myprop}

Let's consider the equation
\[ p^nux^2 + p^mwy^2 = z^2 \]
where $u,w \in \mathbb{Z}_p^\times$.

\begin{itemize}
  \item It has a nontrivial solution in $\mathbb{Z}_p$ (resp. $\mathbb{Z}$) iff it has a nontrivial solution in $\mathbb{Q}_p$ (resp. $\mathbb{Q}$).
  \item It has a nontrivial solution in $\mathbb{Q}_p$ iff
    \[ (-1)^{nm\frac{p-1}{2}} {\legendre{\overline{u}}{p}}^m {\legendre{\overline{w}}{p}}^n = 1. \]
\end{itemize}

\begin{mytheo}[The Hasse principle for ternary quadratic forms (Legendre)]
  Let $a,b \in \mathbb{Q}$ with $a, b \neq 0$.
  The following statements are equivalent:
  \begin{enumerate}
    \item the equation $ax^2 + by^2 = z^2$ has a nontrivial solution in $\mathbb{Q}$;
    \item the equation $ax^2 + by^2 = z^2$ has a nontrivial solution in $\mathbb{R}$
      and in $\mathbb{Q}_p$ for every prime $p$;
    \item the equation $ax^2 + by^2 = z^2$ has a nontrivial solution in $\mathbb{R}$
      and in $\mathbb{Q}_p$ for every prime $p \neq 2$;
  \end{enumerate}
\end{mytheo}

If we write the equation as
\[ \prod_i p_i^{n_i} x^2 + \prod_i p_i^{m_i} y^2 = z^2 \]
we can see that when $n_i$ and $m_i$ are even (e.g. $n_i = m_i = 0$),
it has a nontrivial solution.
Hence we only have a finite number of primes to check.

\section{Analytic function $\zeta$}
Let $\zeta : \mathbb{R}_{>1} \to \mathbb{R}$ be defined as
\begin{align*}
  \zeta(s)
  & = \sum_{n=1}^\infty \frac{1}{n^s}.
\end{align*}
We have
\[ \frac{1}{s-1} \leq \zeta(s) \leq \frac{s}{s-1} \]
and
\[ \lim_{s \stackrel{>}{\to} 1} \frac{\log \zeta(s)}{\log \frac{1}{s-1}} = 1 \]

Let $P = \{2, 3, 5, 7, \ldots\}$ be the set of positive prime numbers.
Euler found that
\[ \zeta(s) = \prod_{p \in P} \frac{1}{1-p^{-s}} \]
where
\[ \frac{1}{1-p^{-s}} = \sum_{i=0}^\infty \frac{1}{p^{is}} \]
since its a geometric sequence.

We can see that
\[ \lim_{s \stackrel{>}{\to} 1} \frac{\sum_{p \in P} \frac{1}{p^s}}{\log \frac{1}{s-1}} = 1 \]
which invites us to define \emph{analytical density} of a set $Q \subseteq P$ as
\[ d(Q) = \lim_{s \stackrel{>}{\to} 1} \frac{\sum_{p \in Q} \frac{1}{p^s}}{\log \frac{1}{s-1}}. \]
We have the following obvious properties:
\begin{itemize}
  \item $d(P) = 1$,
  \item If $Q_1 \cap Q_2 = \emptyset$, $d(Q_1 \cup Q_2) = d(Q_1) + d(Q_2)$,
  \item If $Q$ is finite, $d(Q) = 0$,
\end{itemize}
and the following theorem
\begin{mytheo}[Dirichlet]
  For $a, n \in \mathbb{N}$ \emph{relatively prime},
  \[ d(\{p \in P | p \equiv a \pmod{n}\} = \frac{1}{\varphi(n)}. \]
\end{mytheo}

\section{Computational number theory}
\subsection{Primality testing}
Let $n$ be a number for which we want to know whether it is prime or composite.

The idea is to pick a random $a \i \{1, \ldots, n-1\}$,
and run the test.
The test may provide us a proof that $n$ is composite.
If it does not, do the test again with another random $a$.

If $n$ composite is such that the test do not provide
a proof that it is composite for any $a$, we call it a exception.

\paragraph{Fermat pseudo primes of base $a$}
\[ a^p \equiv a \pmod{n}. \]
Exceptions are Carmichael numbers.

If $n$ is a Carmichael number iff
it has no prime factor and for all prime $p | n$,
$(p-1) | (n-1)$.

There is also
\[ a^{\frac{n-1}{2}} \equiv \pm 1 \pmod{n} \]
which has fewer exceptions.

\paragraph{Euler-Jacobi pseudo primes of base $a$}
\[ a^{\frac{n-1}{2}} \equiv \legendre{a}{n} \pmod{n} \]
which has no exceptions.
For a composite $n$, at least half $a$ provide
us a proof that $n$ is composite.

\paragraph{Miller-Rabin or Strong pseudo primes}
Let $n-1 = 2^st$.
\[ a^t \equiv 1 \pmod{n} \]
or $\exists k \in \{0, 1 \ldots, s-1\}$ such that
\[ s^{2^kt} \equiv -1 \pmod{n}. \]
It has no exception.
For a composite $n$, at least 3 quarter of $a$ provide
us a proof that $n$ is composite.

\subsection{Proving primality}
\paragraph{$n-1$ method}

\subsection{Factoring}
\paragraph{Trial division}
\paragraph{Fermat method}
\paragraph{Pollard $p-1$}
\paragraph{Pollard $\rho$}
Using Floyd's or Tortoise and Hare cycle detection algorithm: the one in the slide.

Using Brent's algorithm:

\end{document}
