\section{}
\subsection{}
\begin{solution}
  \begin{enumerate}
    \item
      Let $c \in \C$ and $m_1,m_2 \in \M$.
      we have
      \begin{align*}
        \Pr[C = c | M = m_b]
        & = \Pr[m_b + k \equiv c \pmod{n}]\\
        & = \Pr[k \equiv c - m_b \pmod{n}]\\
        & = \frac{1}{n}
      \end{align*}
      since $k$ is selected uniformly at random in $\K$.
      We have therefore
      \[
        \Pr[C = c | M = m_1] = \Pr[C = c | M = m_2]
      \]
      for every $c \in \C$ and $m_1,m_2 \in \M$.
    \item
      \begin{align*}
        \Pr[C = c]
        & = \sum_{m \in \M} \Pr[\Enc_k(M) = c | M = m] \Pr[M = m]\\
        & = \frac{1}{n} \sum_{m \in \M} \Pr[M = m]\\
        & = \frac{1}{n}.
      \end{align*}
    \item
      \begin{align*}
        \Pr[K = k | C = c]
        & = \Pr[c - M \equiv k \pmod{n}]\\
        & = \Pr[M \equiv c - k \pmod{n}]\\
        & = {n-1 \choose c-k} p^{c-k} (1-p)^{n-1-c+k}.
      \end{align*}
  \end{enumerate}
\end{solution}

\subsection{}
\begin{solution}
  \begin{enumerate}
    \item Let $p$ be a polynomial,
      let's take $q = 1000p$,
      since it is also a polynomial, we know
      that there exists $N$ such that for all $n \geq N$,
      \[ f(n) \leq \frac{1}{q(n)}. \]
      But that implies that
      \[ 1000 \cdot f(n) \leq \frac{1}{p(n)} \]
      so
      \[ g(n) \leq \frac{1}{p(n)}. \]
    \item Let $p(n) = a_0 + a_1 n + \cdots + a_dn^d$ be an
      arbitrary polynomial of arbitrary degree $d$.
      Let $N_1$ such that $N_1 > r$ for each root $r$ of $p$.
      We know that for $n \geq N_1$, the sign of $p$ is the sign of $a_d$.
      Of course, if $a_d < 0$, our job is impossible but we do not consider these cases.
      Since $p(n) > 0$, our equation is equivalent to
      \[ n^{\log(n)} \geq p(n) \]
      For $n \geq \max(N_1,1)$ we also have
      $p(n) \leq n^d \sum_{i=0}^d|a_i|$.
      Taking the logarithm on both side (we can do it since the logarithm is strictly increasing),
      we have
	%Why ? Can you explain this operation ?
      \[ \log^2(n) - d \log(n) - \log\sum_{i=0}^d|a_i| \geq 0 \]
      which is a second order polynomial in $\log(n)$.
      Let $r_1,r_2$ be its roots.
      We can take $N = \max(N_1,1,2^{r_1},2^{r_2})$.
  \end{enumerate}
\end{solution}

\subsection{}
\begin{solution}
  The complexity of an algorithm is always at least
  the size of its output.
  So there the best complexity we can achieve is $\lfloor\sqrt{n}\rfloor$.
  The size of the input is the number of bits of $n$.
  Let's says, it is $k$.
  The best complexity we can achieve is therefore
  $\lfloor2^{k/2}\rfloor$ which is not efficient (An algorithm is said to be efficient if it's executed in a polynomial amount of steps).
\end{solution}

\subsection{}
\begin{solution}
%Sending it once (in a vector) or with a loop is exactly the same, so I think only one definition is sufficient...
  Two definition can be proposed.
  The first one is the one given in the reference \cite[p.~84]{katz2007introduction}.

  Both are equally good since it can be proven they are equivalent to the definition of indisguishability of a \emph{single} encryption
  under CPA.
  Proving that if $\Pi$ has indisguishable \emph{multiple} encryption under CPA then it also has indisguishable \emph{single} encryption
  is trivial.
  The other way is quite tricky.
  However in public key cryptosystems, CPA is the same than EAV since $\A$ has the public key and can therefore oracle access to $\Enc$.
  There is therefore the same property in assymetric crypto for EAV than for symmetric crypto with CPA.
  This is stated by the \cite[theorem~10.10]{katz2007introduction} which is proven.
  The proof is very similar to the proof we have to make to show the equivalence so if you are in doubt, just check it out.

  \begin{enumerate}

    \item
      $\Pi := \langle\Gen, \Enc, \Dec\rangle$ has indisguishable \emph{multiple} encryption under a chosen-plaintext attack
      if $\forall$ PPT $\A$, $\exists \epsilon$:
      \[ \Pr[\PrivKmultcpa_{\A,\Pi}(n) = 1] \leq \frac{1}{2} + \epsilon(n), \]
      where $\PrivKmultcpa_{\A,\Pi}(n)$ is defined as follows.
      \begin{enumerate}
        \item Choose $k \leftarrow \Gen(1^n)$
        \item $\A$ is given oracle access to $\Enc_k(\cdot)$
        \item $\A$ outputs $M_0 = (m_0^1, \ldots, m_0^t)$, $M_1 = (m_1^1, \ldots, m_1^t)$
        \item Choose $b \leftarrow \{0,1\}$, and send $(\Enc_k(m_b^1, \ldots, \Enc_k(m_b^t))$ to $\A$
        \item $\A$ is again given oracle access to $\Enc_k(\cdot)$
        \item $\A$ output $b'$
        \item Define $\PrivKmultcpa_{\A,\Pi}(n) := 1$ iff $b = b'$
      \end{enumerate}
	

    \item
      $\Pi := \langle\Gen, \Enc, \Dec\rangle$ has indisguishable \emph{multiple} encryption under a chosen-plaintext attack
      if $\forall$ PPT $\A$, $\exists \epsilon$:
      \[ \Pr[\PrivKmultcpa_{\A,\Pi}(n) = 1] \leq \frac{1}{2} + \epsilon(n), \]
      where $\PrivKmultcpa_{\A,\Pi}(n)$ is defined as follows.
      \begin{enumerate}
        \item Choose $k \leftarrow \Gen(1^n)$
        \item $\A$ is given oracle access to $\Enc_k(\cdot)$
        \item Choose $b \leftarrow \{0,1\}$, and send $(\Enc_k(m_b^1, \ldots, \Enc_k(m_b^t))$ to $\A$
        \item For $i \in \{1, \ldots, t\}$
          \begin{enumerate}
            \item $\A$ outputs $(m_0^i, m_1^i)$
            \item Send $\Enc_k(m_b^i)$ to $\A$
            \item $\A$ is again given oracle access to $\Enc_k(\cdot)$
          \end{enumerate}
        \item $\A$ output $b'$
        \item Define $\PrivKmultcpa_{\A,\Pi}(n) := 1$ iff $b = b'$
      \end{enumerate}
  \end{enumerate}
\end{solution}

\subsection{}
\begin{solution}
  $\Pi'$ is a secure encryption scheme.
  $\Pi'$ is public, only the key is hidden from $\A$.
  Adding a 1 at the end will just give no information to $\A$.

  To prove it rigorously, we can prove that ``if $\Pi'$ is insecure then $\Pi$ is insecure''
  since it is the contraposition of ``if $\Pi$ is secure then $\Pi'$ is secure''.
  This proof methodology is called ``reduction''.

  Let $\A'$ be an efficient adversary for $\Pi'$.
  We need to build an efficient adversary $\A$ for $\Pi$.
  \begin{enumerate}
    \item $\A$ is given $1^n$ as input
    \item
      \begin{itemize}
        \item $\A'$ gives $1^n$ as input for $\A$
        \item $\A'$ outputs $m_0, m_1$
      \end{itemize}
    \item $\A$ outputs $m_0, m_1$
    \item $\A$ receives $c$ in response
    \item
      \begin{itemize}
        \item $\A$ gives $c_1$ to $\A'$ which is obtained by discarding the last bit of $c$
        \item $\A'$ outputs $b'$
      \end{itemize}
    \item $\A$ outputs $b'$
  \end{enumerate}
\end{solution}

\subsection{}
\begin{solution}
  There are $|\{0,1\}^n|^{|\{0,1\}|^n} = {2^n}^{2^n}$ function from $\{0,1\}^n$ to $\{0,1\}^n$.
  However, since there are only $2^n$ different $k$, $F_k$ can only be $2^n$ different functions.
  If the distinguisher $D^g$ is unbounded, he can just check the output of $g$ for every possible input and
  the for all $k \in \{0,1\}^n$, he can check if it has the same output of $g$.
  If it has the same output of $F_k$ for at least one $k$, then $D^g(1^n) = 1$, else $D^g(1^n) = 0$.
  More formally
  \[
    D^g(1^n) \triangleq
    \begin{cases}
      1 & \text{if }\exists k \in \{0,1\}^n, \forall m \in \{0,1\}^n, F_k(m) = g(m)\\
      0 & \text{otherwise.}
    \end{cases}
  \]
  We can see that
  \[ \Pr[D^{F_k}(1^n) = 1] = 1 \]
  for all $k \in \{0,1\}^n$.
  Since there could be $k_1,k_2$ such that $F_{k_1}(m) = F_{k_2}(m)$ for all $m \in \{0,1\}^n$,
  \[ |\{f : \{0,1\}^n \to \{0,1\}^n | \exists k \in \{0,1\}^n, \forall m \in \{0,1\}^n f(m) = F_k(m) \}| \leq 2^n. \]
  Therefore
  \[ \Pr[D^{f}(1^n) = 1] \leq \frac{2^n}{{2^n}^{2^n}} \leq {2^n}^{2^n-1}. \]
\end{solution}
