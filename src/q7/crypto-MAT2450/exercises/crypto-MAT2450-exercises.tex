\documentclass[en]{../../../eplexercises}

\usepackage{qtree}

\newcommand{\xor}{\oplus}

\newcommand{\A}{\mathcal{A}}
\newcommand{\D}{\mathcal{D}}
\renewcommand{\C}{\mathcal{C}}
\renewcommand{\K}{\mathcal{K}}
\newcommand{\M}{\mathcal{M}}
\newcommand{\G}{\mathcal{G}}
\DeclareMathOperator{\Gen}{\mathsf{Gen}}
\DeclareMathOperator{\Enc}{\mathsf{Enc}}
\DeclareMathOperator{\Dec}{\mathsf{Dec}}
\DeclareMathOperator{\Mac}{\mathsf{Mac}}
\DeclareMathOperator{\Vrfy}{\mathsf{Vrfy}}
\DeclareMathOperator{\Com}{\mathsf{Com}}
\DeclareMathOperator{\Open}{\mathsf{Open}}
\DeclareMathOperator{\Sign}{\mathsf{Sign}}
\DeclareMathOperator{\Sig}{\mathsf{Sig}}

\DeclareMathOperator{\PrivK}{PrivK}
\DeclareMathOperator{\MacForge}{MacForge}
\DeclareMathOperator{\Sigforge}{Sig-forge}
\DeclareMathOperator{\Invert}{Invert}
\DeclareMathOperator{\DDH}{DDH}
\DeclareMathOperator{\DLog}{DLog}
\newcommand{\PrivKeav}{\PrivK^{\text{eav}}}
\newcommand{\PrivKmultcpa}{\PrivK^{\text{multcpa}}}
\newcommand{\Sigforgeone}{\Sigforge^{\text{1--time}}}

\DeclareMathOperator{\Ima}{\mathsf{Im}}

\hypertitle{Cryptography}{7}{MAT}{2450}
{Benoît Legat}
{François Koeune, Olivier Pereira}

\section{}
\subsection{}
\begin{solution}
  \begin{enumerate}
    \item
      Let $c \in \C$ and $m_1,m_2 \in \M$.
      we have
      \begin{align*}
        \Pr[C = c | M = m_b]
        & = \Pr[m_b + k \equiv c \pmod{n}]\\
        & = \Pr[k \equiv c - m_b \pmod{n}]\\
        & = \frac{1}{n}
      \end{align*}
      since $k$ is selected uniformly at random in $\K$.
      We have therefore
      \[
        \Pr[C = c | M = m_1] = \Pr[C = c | M = m_2]
      \]
      for every $c \in \C$ and $m_1,m_2 \in \M$.
    \item
      \begin{align*}
        \Pr[C = c]
        & = \sum_{m \in \M} \Pr[\Enc_k(M) = c | M = m] \Pr[M = m]\\
        & = \frac{1}{n} \sum_{m \in \M} \Pr[M = m]\\
        & = \frac{1}{n}.
      \end{align*}
    \item
      \begin{align*}
        \Pr[K = k | C = c]
        & = \Pr[c - M \equiv k \pmod{n}]\\
        & = \Pr[M \equiv c - k \pmod{n}]\\
        & = {n-1 \choose c-k} p^{c-k} (1-p)^{n-1-c+k}.
      \end{align*}
  \end{enumerate}
\end{solution}

\subsection{}
\begin{solution}
  \begin{enumerate}
    \item Let $p$ be a polynomial,
      let's take $q = 1000p$,
      since it is also a polynomial, we know
      that there exists $N$ such that for all $n \geq N$,
      \[ f(n) \leq \frac{1}{q(n)}. \]
      But that implies that
      \[ 1000 \cdot f(n) \leq \frac{1}{p(n)} \]
      so
      \[ g(n) \leq \frac{1}{p(n)}. \]
    \item Let $p(n) = a_0 + a_1 n + \cdots + a_dn^d$ be an
      arbitrary polynomial of arbitrary degree $d$.
      Let $N_1$ such that $N_1 > r$ for each root $r$ of $p$.
      We know that for $n \geq N_1$, the sign of $p$ is the sign of $a_d$.
      Of course, if $a_d < 0$, our job is impossible but we do not consider these cases.
      Since $p(n) > 0$, our equation is equivalent to
      \[ n^{\log(n)} \geq p(n) \]
      For $n \geq \max(N_1,1)$ we also have
      $p(n) \leq n^d \sum_{i=0}^d|a_i|$.
      Taking the logarithm on both side (we can do it since the logarithm is strictly increasing),
      we have
	%Why ? Can you explain this operation ?
      \[ \log^2(n) - d \log(n) - \log\sum_{i=0}^d|a_i| \geq 0 \]
      which is a second order polynomial in $\log(n)$.
      Let $r_1,r_2$ be its roots.
      We can take $N = \max(N_1,1,2^{r_1},2^{r_2})$.
  \end{enumerate}
\end{solution}

\subsection{}
\begin{solution}
  The complexity of an algorithm is always at least
  the size of its output.
  So there the best complexity we can achieve is $\lfloor\sqrt{n}\rfloor$.
  The size of the input is the number of bits of $n$.
  Let's says, it is $k$.
  The best complexity we can achieve is therefore
  $\lfloor2^{k/2}\rfloor$ which is not efficient (An algorithm is said to be efficient if it's executed in a polynomial amount of steps).
\end{solution}

\subsection{}
\begin{solution}
%Sending it once (in a vector) or with a loop is exactly the same, so I think only one definition is sufficient...
  Two definition can be proposed.
  The first one is the one given in the reference \cite[p.~84]{katz2007introduction}.

  Both are equally good since it can be proven they are equivalent to the definition of indisguishability of a \emph{single} encryption
  under CPA.
  Proving that if $\Pi$ has indisguishable \emph{multiple} encryption under CPA then it also has indisguishable \emph{single} encryption
  is trivial.
  The other way is quite tricky.
  However in public key cryptosystems, CPA is the same than EAV since $\A$ has the public key and can therefore oracle access to $\Enc$.
  There is therefore the same property in assymetric crypto for EAV than for symmetric crypto with CPA.
  This is stated by the \cite[theorem~10.10]{katz2007introduction} which is proven.
  The proof is very similar to the proof we have to make to show the equivalence so if you are in doubt, just check it out.

  \begin{enumerate}

    \item
      $\Pi := \langle\Gen, \Enc, \Dec\rangle$ has indisguishable \emph{multiple} encryption under a chosen-plaintext attack
      if $\forall$ PPT $\A$, $\exists \epsilon$:
      \[ \Pr[\PrivKmultcpa_{\A,\Pi}(n) = 1] \leq \frac{1}{2} + \epsilon(n), \]
      where $\PrivKmultcpa_{\A,\Pi}(n)$ is defined as follows.
      \begin{enumerate}
        \item Choose $k \leftarrow \Gen(1^n)$
        \item $\A$ is given oracle access to $\Enc_k(\cdot)$
        \item $\A$ outputs $M_0 = (m_0^1, \ldots, m_0^t)$, $M_1 = (m_1^1, \ldots, m_1^t)$
        \item Choose $b \leftarrow \{0,1\}$, and send $(\Enc_k(m_b^1, \ldots, \Enc_k(m_b^t))$ to $\A$
        \item $\A$ is again given oracle access to $\Enc_k(\cdot)$
        \item $\A$ output $b'$
        \item Define $\PrivKmultcpa_{\A,\Pi}(n) := 1$ iff $b = b'$
      \end{enumerate}
	

    \item
      $\Pi := \langle\Gen, \Enc, \Dec\rangle$ has indisguishable \emph{multiple} encryption under a chosen-plaintext attack
      if $\forall$ PPT $\A$, $\exists \epsilon$:
      \[ \Pr[\PrivKmultcpa_{\A,\Pi}(n) = 1] \leq \frac{1}{2} + \epsilon(n), \]
      where $\PrivKmultcpa_{\A,\Pi}(n)$ is defined as follows.
      \begin{enumerate}
        \item Choose $k \leftarrow \Gen(1^n)$
        \item $\A$ is given oracle access to $\Enc_k(\cdot)$
        \item Choose $b \leftarrow \{0,1\}$, and send $(\Enc_k(m_b^1, \ldots, \Enc_k(m_b^t))$ to $\A$
        \item For $i \in \{1, \ldots, t\}$
          \begin{enumerate}
            \item $\A$ outputs $(m_0^i, m_1^i)$
            \item Send $\Enc_k(m_b^i)$ to $\A$
            \item $\A$ is again given oracle access to $\Enc_k(\cdot)$
          \end{enumerate}
        \item $\A$ output $b'$
        \item Define $\PrivKmultcpa_{\A,\Pi}(n) := 1$ iff $b = b'$
      \end{enumerate}
  \end{enumerate}
\end{solution}

\subsection{}
\begin{solution}
  $\Pi'$ is a secure encryption scheme.
  $\Pi'$ is public, only the key is hidden from $\A$.
  Adding a 1 at the end will just give no information to $\A$.

  To prove it rigorously, we can prove that ``if $\Pi'$ is insecure then $\Pi$ is insecure''
  since it is the contraposition of ``if $\Pi$ is secure then $\Pi'$ is secure''.
  This proof methodology is called ``reduction''.

  Let $\A'$ be an efficient adversary for $\Pi'$.
  We need to build an efficient adversary $\A$ for $\Pi$.
  \begin{enumerate}
    \item $\A$ is given $1^n$ as input
    \item
      \begin{itemize}
        \item $\A'$ gives $1^n$ as input for $\A$
        \item $\A'$ outputs $m_0, m_1$
      \end{itemize}
    \item $\A$ outputs $m_0, m_1$
    \item $\A$ receives $c$ in response
    \item
      \begin{itemize}
        \item $\A$ gives $c_1$ to $\A'$ which is obtained by discarding the last bit of $c$
        \item $\A'$ outputs $b'$
      \end{itemize}
    \item $\A$ outputs $b'$
  \end{enumerate}
\end{solution}

\subsection{}
\begin{solution}
  There are $|\{0,1\}^n|^{|\{0,1\}|^n} = {2^n}^{2^n}$ function from $\{0,1\}^n$ to $\{0,1\}^n$.
  However, since there are only $2^n$ different $k$, $F_k$ can only be $2^n$ different functions.
  If the distinguisher $D^g$ is unbounded, he can just check the output of $g$ for every possible input and
  the for all $k \in \{0,1\}^n$, he can check if it has the same output of $g$.
  If it has the same output of $F_k$ for at least one $k$, then $D^g(1^n) = 1$, else $D^g(1^n) = 0$.
  More formally
  \[
    D^g(1^n) \triangleq
    \begin{cases}
      1 & \text{if }\exists k \in \{0,1\}^n, \forall m \in \{0,1\}^n, F_k(m) = g(m)\\
      0 & \text{otherwise.}
    \end{cases}
  \]
  We can see that
  \[ \Pr[D^{F_k}(1^n) = 1] = 1 \]
  for all $k \in \{0,1\}^n$.
  Since there could be $k_1,k_2$ such that $F_{k_1}(m) = F_{k_2}(m)$ for all $m \in \{0,1\}^n$,
  \[ |\{f : \{0,1\}^n \to \{0,1\}^n | \exists k \in \{0,1\}^n, \forall m \in \{0,1\}^n f(m) = F_k(m) \}| \leq 2^n. \]
  Therefore
  \[ \Pr[D^{f}(1^n) = 1] \leq \frac{2^n}{{2^n}^{2^n}} \leq {2^n}^{2^n-1}. \]
\end{solution}


\section{}
\subsection{}
\begin{solution}
  \begin{itemize}
    \item
      Let $\tilde{\Pi} = \langle \tilde{\Gen}, \tilde{\Mac}, \tilde{\Vrfy} \rangle$, defined as:
      \begin{itemize}
        \item $\tilde{\Gen}$: chooses a random $f$.
        \item $\tilde{\Mac}$: on input $m$, outputs $f(m)$.
        \item $\tilde{\Vrfy}$: on input $(m,t)$, outputs $1$ iff $f(m) = t$.
      \end{itemize}

      Let's analyse the maximum value of $\Pr[\MacForge_{\A, \tilde{\Pi}}(n) = 1]$ for an adversary $\mathcal{A}$.
      If after $q$ different queries (it gains no info doing the same query twice),
      $m_1, \ldots, m_q$, $\A$ outputs $(m, t)$, what are its chances of success ?
      Let $f:\{0,1\}^n \to \{0,1\}^n$.
      There are $(2^n)^{2^n}$ different $f$ and we pick a random one uniformly.
      However, there are only $(2^n)^{2^n-q}$ experiments such that $\A$ have received $(m_i,t_i)$ for $i = 1, \ldots, q$ because
      there are $(2^n)^{2^n-q}$ $f$ such that $f(m_i) = t_i$ for $i = 1, \ldots, q$.
      We could be in any of them.
      Among them, only $(2^n)^{2^n-(q+1)}$ are such that $f(m) = t$.
      Since $f$ is selected uniformly, we have

      \begin{align*}
        \Pr[\MacForge_{\A, \tilde{\Pi}}(n) = 1]
        & = \Pr[f(m) = t | f(m_i) = t_i, \forall m = 1, \ldots, q]\\
        & = \frac{\Pr[f(m) = t, f(m_i) = t_i, \forall m = 1, \ldots, q]}{\Pr[f(m_i) = t_i, \forall m = 1, \ldots, q]}\\
        & = \frac{\frac{(2^n)^{2^n-(q+1)}}{(2^n)^{2^n}}}{\frac{(2^n)^{2^n-q}}{(2^n)^{2^n}}}\\
        & = \frac{(2^n)^{2^n-(q+1)}}{(2^n)^{2^n-q}}\\
        & = \frac{1}{2^n}.
      \end{align*}
      A shortcut would have been to argued that since, $f(m)$ is independent of the $f(m_i)$ so

      \begin{align*}
        \Pr[\MacForge_{\A, \tilde{\Pi}}(n) = 1]
        & = \Pr[f(m) = t | f(m_i) = t_i, \forall m = 1, \ldots, q]\\
        & = \Pr[f(m) = t]\\
        & = \frac{(2^n)^{2^n-1}}{(2^n)^{2^n}}\\
        & = \frac{1}{2^n}.
      \end{align*}

      It is quite surprising that instead of a upper bound
      on $\Pr[\MacForge_{\A, \tilde{\Pi}}(n) = 1]$
      depending on $\A$ (and reached for $\A$ super smart),
      it is actually independent of $\A$.
    \item
      Let's now suppose that we have an adversary $\A$
      that win with non-neglishible probability against a PRF $F$
      and show that under this assumption we can build
      a distinguisher $\D$ for $F$.

      $\D$ will simply take a function $g$ as input
      and run $\A$ using $g$ to create the tags.
      He has $g$ so he can see if $\A$ wins or lose.
      If $\A$ wins he outputs $1$ otherwise it outputs $0$.

      We know that if $g$ is a pseudo random function,
      $\Pr[\MacForge_{\A, \Pi_g} = 1] = \frac{1}{2^n}$
      and if $g$ is a PRF
      $\Pr[\MacForge_{\A, \Pi_g} = 1] = \eta(n)$
      where $\eta$ is nonnegligible.
      We have therefore
      \[
        |\Pr[\D^{F_k}(1^n) = 1] - \Pr[\D^{f}(1^n) = 1]|
        = \left|\eta(n) - \frac{1}{2^n}\right|
      \]
      which is non-negligible.

      % Not sure it works
%    \item
%      Another simpler solution is possible. Using the first Hint, we can say that if $F_k$ is a PRF, it has a maximum of $2^n$ possible outputs
%      where a truly random function has exactly $2^n$ outputs. So if we suppose $\Pi$ secure with a PRF, then $\tilde{\Pi}$ is also secure because
%      $\epsilon_{\tilde{\Pi}}  = \frac{1}{2^n} \geq \epsilon_{\Pi}$. We then can play the PRF game to prove the security of $\Pi$ with the second hint.
  \end{itemize}

\end{solution}

\subsection{}
\begin{solution}
  If we have the tag of $p$, which is (let's consider that $k$ and $p$ are $l$ bits long for simplicity)
  \[ t_p = H^s(k\|p) = h^s(h^s(h^s(h^s(IV \| k) \| p)) \| 2l) \]
  we can find the tag of $p\|2l\|w$ (where $w$ is $l$ bits long for simplicity)
  without knowing $k$ since we know $h^s$.
  It is
  \begin{align*}
    H^s(k\|p\|2l\|w)
    & = h^s(h^s(h^s(h^s(h^s(h^s(IV \| k) \| p)) \| 2l) \| w) \| 4l)\\
    & = h^s(h^s(H^s(k \| p) \| w) \| 4l)\\
    & = h^s(h^s(t_p \| w) \| 4l)
  \end{align*}
  Since $p\|2l\|w \neq p$, this gives us an existential forgery.
\end{solution}

\subsection{}
\begin{solution}
  From $L_0\|R_0$, we get $L_2\|R_2$
  where

  \begin{align*}
    L_2 & = L_0 \xor F_k(R_0)\\
    R_2 & = R_0 \xor F_k(L_0 \xor F_k(R_0)).
  \end{align*}
  When a distinguisher $\D$ sends $L_0\|R_0$ and receives
  $L_2\|R_2$ we can
  compute

  \begin{align*}
    L_2' & = L_0 \xor L_2\\
         & = L_0 \xor (L_0 \xor F_k(R_0))\\
         & = F_k(R_0)\\
    R_2' & = R_0 \xor R_2\\
         & = R_0 \xor (R_0 \xor F_k(L_0 \xor F_k(R_0)))\\
         & = F_k(L_0 \xor F_k(R_0)).
  \end{align*}
  The problem is that $L_2'$ does not depend on $L_0$.

  He can therefore send another request with the same $R_0$
  and a different $L_0$.
  If it gets the same $L_2'$,
  It has only 1 chance over $2^l$ (where $l$ is the number of bits of $L$) to receive the same $L_2'$ for a random permutation.

  If $\D$ decide to output 1 iff $L_2'$ is the same for the two request we have
  \[ |\Pr[\D^{F_k}(1^n) = 1] -\Pr[\D^f(1^n)]| = 1 - \frac{1}{2^n} \]
  which is clearly not negligible.
\end{solution}

\subsection{}
\begin{solution}
  Let say we have to split the message into $m$ messages of $n$ bits.
  Choosing $m_0 = M_0 \| M_0 \| \cdots \| M_0$ and $m_1 = M_1 \| M_2 \| \cdots \| M_m$ with $M_i \neq M_j$ for $i \neq j$,
  if $b = 0$, we will get $c = C_0 \| C_0 \| \cdots \| C_0$ for some $C_0 \in \C$ and $c = C_1 \| C_2 \| \cdots \| C_m$ for some $C_i \in \C$.

  An adversary $\A$ can output $b = 0$ iff all the $C_i$s are equals.
  We have
  \[ \Pr[\PrivKeav_{\A,\text{ECB}}(nm)] = \frac{1}{2} + \frac{1}{2} = 1 \]
  since the $C_i$ cannot be equal if $b = 1$ since we use a PRP.
  If two different $M_i$ were encrypted as same $C_i$, decryption wouldn't be possible.
\end{solution}

\subsection{}
\begin{solution}
  We will give 2 collisions for $f_1$ and $f_2$.

  We know that $E(k, \cdot)$ is surjective because it must be injective and $\M = \C$.
  Therefore the decryption exists for all $c \in \C$ ! We will use it for the second collision of $f_1$.

  For $f_1$, we have the 2 following collisions

  \begin{align*}
    f_1(D(E(y,x),y), E(y,x))
    & = E(E(y,x), D(E(y,x), y)) \xor E(y,x)\\
    & = y \xor E(y,x)\\
    & = E(y,x) \xor y\\
    & = f_1(x, y)\\
    f_1(D(0, E(y,x) \xor y), 0)
    & = E(0, D(0, E(y, x) \xor y)) \xor 0\\
    & = E(y, x) \xor y\\
    & = f_1(x, y).
  \end{align*}
  and for $f_2$ we have

  \begin{align*}
    f_2(x, E(x,x)) & = E(x,x) \xor E(x,x)\\
                   & = 0 & \forall x \in \{0,1\}^l\\
    f_2(y, E(x,x)) & = E(y,y) \xor E(x,x)\\
                   & = E(x,x) \xor E(y,y)\\
                   & = f_2(x, E(y,y)).
  \end{align*}
\end{solution}

\subsection{}
\begin{solution}
  \begin{itemize}
    \item
      Let's show that from a collision of $H_4$, we generate a collision for $H_2$
      which prove that $H_4$ is collision resistant since $H_2$ is so.
      Let's suppose that we have $x_1\|x_2 \neq y_1\|y_2$ are such that $H_4(x_1\|x_2) = H_4(y_1\|y_2)$.
      \begin{itemize}
        \item
          If $H_2(x_1) \| H_2(x_1 \xor x_2) \neq H_2(y_1) \| H_2(y_1 \xor y_2)$,
          we have a colision for $H_2$ since their image by $H_2$ is identical.
        \item
          If $H_2(x_1) \| H_2(x_1 \xor x_2) = H_2(y_1) \| H_2(y_1 \xor y_2)$,
          we have $H_2(x_1) = H_2(y_1)$ \emph{and} $H_2(x_1 \xor x_2) = H_2(y_1 \xor y_2)$.
          \begin{itemize}
            \item If $x_1 \neq y_1$, we have a collision for $x_2$ since $H_2(x_1) = H_2(y_1)$.
            \item If $x_1 = y_1$, then $x_2 \neq y_2$ since $x_1\|x_2 \neq y_1\|y_2$.
              Therefore $x_1 \xor x_2 \neq y_1 \xor y_2$ and we have collision on $H_2$.
          \end{itemize}
      \end{itemize}
    \item
      $H_6$ is not collision resistant since $H_6(x_1\|x_2\|x_3) = H_6((x_1 \xor w)\|(x_2 \xor w)\|(x_3 \xor w))$
      for all $w$.
      Indeed, since $\xor$ is associative and commutative,
      \begin{align*}
        & = H_6((x_1 \xor w)\|(x_2 \xor w)\|(x_3 \xor w))
        & = H_3(H2((x_1 \xor w) \xor (x_2 \xor w))\|H2((x_2 \xor w) \xor (x_3 \xor w))\|H2((x_3 \xor w) \xor (x_1 \xor w)))
        & = H_3(H2(x_1 \xor (w \xor w) \xor x_2)\|H2(x_2 \xor (w \xor w) \xor x_3)\|H2(x_3 \xor (w \xor w) \xor x_1)))
        & = H_3(H2(x_1 \xor x_2)\|H2(x_2 \xor x_3)\|H2(x_3 \xor x_1)))\\
        & = H_6(x_1\|x_2\|x_3).
      \end{align*}
  \end{itemize}
\end{solution}

\subsection{}
\begin{solution}
  \begin{enumerate}
    \item Encrypt all the DVDs with $K_\text{root}$ !
    \item At the beginning of the DVD, encrypt a random key $K$ twice (using $K_2$ and then $K_3$): $Enc_{K_2}(K)$ and $Enc_{K_3}(K)$.

      We then encrypt all the content $M$ of the DVD using $K$. (We suppose $K_1$ is $K_{\text{root}}$)
      $K_2$ and $K_3$ are the keys associated to the 2 childs of the root.
    \item
      For every node $i$ on the path from the root to $r$, we must add an encryption of $K$ with the key of its child that is not in the path.
      For example, if $r = 10$ and $n = 16$, we must include the bold keys,
      so we will include the encryption of $K_7$ in a DVD which is quite something.
      \begin{center}
        \Tree [.{$K_{\text{root}} = K_1$}
          [.{$\mathbf{K_2}$}
            [.{$K_4$}
              [.{$K_8$}
                [.{0} ]
                [.{1} ]
              ]
              [.{$K_9$}
                [.{2} ]
                [.{3} ]
              ]
            ]
            [.{$K_5$}
              [.{$K_{10}$}
                [.{4} ]
                [.{5} ]
              ]
              [.{$K_{11}$}
                [.{6} ]
                [.{7} ]
              ]
            ]
          ]
          [.{$K_3$}
            [.{$K_6$}
              [.{$\mathbf{K_{12}}$}
                [.{8} ]
                [.{9} ]
              ]
              [.{$K_{13}$}
                [.{10} ]
                [.{11} ]
              ]
            ]
            [.{$\mathbf{K_7}$}
              [.{$K_{14}$}
                [.{12} ]
                [.{13} ]
              ]
              [.{$K_{15}$}
                [.{14} ]
                [.{15} ]
              ]
            ]
          ]
        ]
      \end{center}
  \end{enumerate}
\end{solution}

\subsection{}
\begin{solution}
  \begin{enumerate}
    \item TODO
    \item $\A'$ should pick a random $r$ and send to its oracle the $d$ queries $r\|l\|i\|m_i$ for $i = 1, \ldots, d$.
      and output $\langle r, t_1, \ldots, t_d \rangle$ where $t_i$ is the answer of the oracle to its $i$th query.
    \item For $\Pi$ a forgery is $(m, \langle r, t_1, \ldots, t_d \rangle)$ where $m$ is not one of its previous query
      and $t_i = \Mac_k(r\|l\|i\|m_i)$ for $i = 1, \ldots, d$ where $l$ is the length of $m$ and $m = m_1\| \cdots \|m_d$.
      \begin{itemize}
        \item
          If none of its previous query has the same $r$ and $l$, $r\|l\|1\|m_1$ cannot be a query made by $\A'$ and
          $(r\|l\|1\|m_1, t_1)$ is an existential forgery for $\Pi'$ that $\A'$ can output.
        \item
          If 2 previous queries have both $r$ and $l$, then we do not necessarily have an existential forgery.
          However, since $r$ is picked at random ($\A'$ could cheat and make sure that the same $r$ is not picked twice)
          the probability (birthday paradox) of this to happen (if all $m$ have the same $l$ which is the worst case) is approximatively
          $\frac{q(n)^2}{2 \cdot 2^n}$ where $q(n)$ is the number of queries make by $\A$.
        \item
          If one unique previous query $m^j$ has this $r$ and $l$, since $m$ is not one of the previous query, there must be $i$
          such that $m_i \neq m_i^j$.
          We know therefore that $r\|l\|j\|m_j$ has never been queried by $\A'$ so $(r\|l\|j\|m_j, t_j)$ is an existential forgery for $\Pi$.
      \end{itemize}
    \item In conclusion, we have
      \begin{align*}
        \Pr[\MacForge_{\A,\Pi}(n) = 1]
        & \leq \Pr[\MacForge_{\A',\Pi'}(n) = 1] + \frac{q(n)}{2^{n+1}}\\
        \epsilon(n)
        & \leq \epsilon'(n) + \frac{q(n)}{2^{n+1}}
      \end{align*}
      so since $\epsilon'(n)$ and $\frac{q(n)}{2^{n+1}}$ are negligible,
      $\epsilon(n)$ is negligible.
  \end{enumerate}
\end{solution}

\subsection{}
\begin{solution}
  We build $H(m) = H_0(m)\|H_1(m)$.
  If we have $m_1 \neq m_2$ such that $H(m_1) = H(m_2)$ then $H_0(m_1) = H_0(m_2)$ and
  $H_1(m_1) = H_1(m_2)$ so the collision resistant hash function has a collision whichever it is.
  However, $H$ is no more a compression function and we cannot use Merkle-Damg\aa{}rd.

  The input of $H$ therefore cannot have arbitrary length but its input is twice the length of the output of $H_0$ and $H_1$
  so it is twice the size of a tag.

  The output of $\Mac_0$ and $\Mac_1$ are the size of a tag so we can use the tag
  $H(\Mac_0(k,m)\|\Mac_1(k,m))$ for our Hash-MAC scheme.
  If we are able to output an existential forgery $(m, t)$,
  since $H$ is collision resistant, that means that we have found
  $\Mac_0(k,m)\|\Mac_1(k,m)$ and therefore we have found an existantial forgery for both
  $\Mac_0$ \emph{and} $\Mac_1$ which is absurd since one of them is believed to be unforgeable.

  Our Hash-MAC scheme is therefore unforgeable.
\end{solution}

\subsection{}
\begin{solution}
  \begin{enumerate}
    \item When player $\ell$ receives $(i,c_0,c_1)$ he can compute its binary representation $\ell = \sum_{j=0}^{n-1} b_j2^j$
      and then computes
      \begin{align*}
        D(k_{i,b_i}, c_{b_i})
        & = D(k_{i,b_i}, E(k_{i,b_i}, m))\\
        & = m.
      \end{align*}
    \item The content provider can recover $b_j$ by feeding $P$ with $(i,E(k_{j,0},0),E(k_{j,1},1))$.
      He will then listen to what $P$ broadcast.
      $b_j$ will be what he will hear.
    \item $i \neq j$ and $i \xor j$ is not a power of two.
      That means that $i$ and $j$ differ for at least 2 bits.
      For one of them they will use the key of $i$ and for the other they will use the key of $j$.
      For example, if $i = 101_2$ and $j = 110_2$, they will use $k_{2,1}$, $k_{1,0}$ and $k_{0,0}$.
      That way the content provider will think that the index is $100_2$.
  \end{enumerate}
\end{solution}

\subsection{}
\begin{solution}
  Pour un message $m = m_1 \| m_2$,

  \[ \Enc_k(m_1, m_2) = (y_1, y_2, y_3)  =(p_k(m_1) , p_k(y_1 \xor m_2), pk(y_2 \xor r)) \]

  Le $R$ est random, et je pense que contrairement à ce que tu disais dans ta
  solution, il est fourni en texte clair en même temps que le cipher text
  (enfin bon de toute manière du point de vue de l'adversaire tu verras que
  ça ne change pas grand chose)

  Si on s'intéresse au ``game'' qui garantit la caractère CPA du scheme (cf slide 15 L3).
  Il suffit qu'un adversaire $\A$ choisisse $m_0$ et $m_1$ distinct, \emph{il peut demander
  leurs encryptions}.

  If $m_0 = m_W || m_X$ and $m_1 = m_Y || m_Z$

  \begin{align*}
    \Enc_k(m_W,m_X) = (p_k(mW), \ldots)\\
    \Enc_k(m_Y,m_Z) = (p_k(mY), \ldots).
  \end{align*}

  $\A$ envoit $m_0$ et $m_1$ au testeur
  Le testeur choisit $b$ et renvoie $Enc(m_b) = (y_1 ,y_2, y_3)$

  \begin{itemize}
    \item Si $y_1 = p_k(m_W)$ l'adversaire sait que $b = 0$
    \item Si $y_1 = p_k(m_Y)$ l'adversaire sait que $b = 1$
  \end{itemize}

  Le problème avec ce scheme d'encryption est qu'il n'est pas CPA secure car
  le facteur $r$ random est très mal appliqué, ce qui fait que le début du
  message encrypté qu'il renvoie est tout à fait déterministe et donc pas du
  tout CPA secure.
\end{solution}

\subsection{}
\begin{solution}
  \begin{enumerate}
    \item
      The problem with a MAC is that every party that can verify a MAC can also build it since verifying a MAC consist
      in recomputing the tag and comparing it the the received tag.
      All $B_i$ therefore need to have the key $k$ and can therefore send valid packets to other $B_j$.
    \item
      For $B_i$ to build all the MACs for $B_j$, he needs $S_i \subseteq S_j$.
      We therefore need that there is no $i \neq j$ such that $S_i \subseteq S_j$.
    \item We can simply take all $S_i$ having 2 elements.
      $S_i \subseteq S_j$ simplifies therefore in $S_i \neq S_j$.
      There are ${5 \choose 2} = \frac{5!}{2! \cdot 3!} = 10$ possible such sets.
      %Sure ? Because I think in this case B_k would be allowed to sign packets to other users haveing k_k in his set. No ?
  \end{enumerate}
\end{solution}


\section{}
\subsection{}
\begin{solution}
  \begin{enumerate}
    \item
      To have $m^{ed} \equiv m \pmod{N}$,
      it is sufficient to have $ed \equiv 1 \pmod{\varphi(N)}$.
      To have this, we need $e$ relatively prime to $\varphi(N)$.

      $\varphi(N) = (7-1)(11-1) = 60 = 2^235$
      so the smallest $e$ we can take is $7$.
    \item
      To find $d$ such that $ed \equiv 1 \pmod{60}$ we
      can use the extended Euclidean algorithm \footnote{A good explaination can be found at \url{https://www.youtube.com/watch?v=fz1vxq5ts5I}}

      \begin{align*}
        60 & = 8 \cdot 7 + 4\\
        7 & = 1 \cdot 4 + 3\\
        4 & = 1 \cdot 3 + 1
      \end{align*}
      so

      \begin{align*}
        1 & = 4 - 1 \cdot 3\\
          & = 2 \cdot 4 - 1 \cdot 7\\
          & = 2 \cdot 60 - 17 \cdot 7.
      \end{align*}
      We can take $-17 \equiv 43 \pmod{60}$ for $d$.
    \item
      We have

      \begin{align*}
        c & \equiv 75^7 \pmod{77}\\
          & \equiv (-2)^7 \pmod{77}\\
          & \equiv -128 \pmod{77}\\
          & \equiv 26 \pmod{77}.
      \end{align*}
    \item
      Let's just verify it modulo $7$ and $11$ which is equivalent
      according to the CRT

      \begin{align*}
        c^{43} & \equiv 26^{43} \pmod{7}\\
               & \equiv (-2)^{7 \cdot \varphi(7) + 1} \pmod{7}\\
               & \equiv -2 \pmod{7}\\
               & \equiv 75 \pmod{7}
      \end{align*}
      and

      \begin{align*}
        c^{43} & \equiv 26^{43} \pmod{11}\\
               & \equiv 4^{4 \cdot \varphi(11) + 3} \pmod{11}\\
               & \equiv 4^{3} \pmod{11}\\
               & \equiv 64 \pmod{11}\\
               & \equiv -2 \pmod{11}\\
               & \equiv 75 \pmod{11}.
      \end{align*}
  \end{enumerate}
\end{solution}

\subsection{}
\begin{solution}
  \begin{enumerate}
    \item
      \begin{enumerate}
        \item
          $\epsilon_7$ should be a multiple of $11$ so there is a $y$ such
          that $\epsilon_7 = 11y$.
          So we have

          \begin{align*}
            11y & \equiv 1 \pmod{7}\\
              y & \equiv (11)^{-1} \pmod{7}\\
                & \equiv 2 \pmod{7}
          \end{align*}
          which gives $\epsilon_7 \equiv 22 \pmod{77}$.
        \item
          $\epsilon_{11}$ should be a multiple of $7$ so there is a $y$ such
          that $\epsilon_{11} = 7y$.
          So we have

          \begin{align*}
            7y & \equiv 1 \pmod{11}\\
             y & \equiv 7^{-1} \pmod{11}\\
               & \equiv 8 \pmod{11}
          \end{align*}
          which gives $\epsilon_{11} \equiv 56 \pmod{77}$.
        \item
          By the superposition principle
          \begin{align*}
            5\epsilon_7 + 9\epsilon_{11} & \equiv 5 \pmod{7}\\
            5\epsilon_7 + 9\epsilon_{11} & \equiv 9 \pmod{11}
          \end{align*}
          so $x \equiv 5 \cdot 22 + 9 \cdot 56 \equiv 75 \pmod{77}$.
        \end{enumerate}
      \item
        We can take $x_p\epsilon_7 + x_q\epsilon_{11}$
        since by the superposition principle
        \begin{align*}
          x_p\epsilon_7 + x_q\epsilon_{11} & \equiv x_p \cdot 1 + x_q \cdot 0 \pmod{7}\\
                                           & \equiv x_p \pmod{7}\\
          x_p\epsilon_7 + x_q\epsilon_{11} & \equiv x_p \cdot 0 + x_q \cdot 1 \pmod{11}\\
                                           & \equiv x_q \pmod{11}.
        \end{align*}
      \item
        We can know that $\epsilon_p = qy$ for some $y$.
        So we have
        \begin{align*}
          qy & \equiv 1 \pmod{p}\\
           y & \equiv q^{-1} \pmod{p}\\
        \end{align*}
        which gives $\epsilon_7 \equiv qq^{-1} \pmod{pq}$
        where the inverse is taken modulo $p$.

        This is the same for $\epsilon_q$.
      \item
        \[ \psi^{-1} : \mathbb{Z}_p \times \mathbb{Z}_q \to \mathbb{Z}_n; (x_p, x_q) \to x_p\epsilon_p + x_q\epsilon_q. \]
    \end{enumerate}
\end{solution}

\subsection{}
\begin{solution}
  \begin{enumerate}
    \item
      It is a subset of $\mathbb{Z}_p^*$ since $0$
      cannot be the square of an element in $\mathbb{Z}_p^*$.

      It is a group since if $x_1$ and $x_2$ are quadratic residues,
      there exists $y_1,y_2$ such that
      \begin{align*}
        y_1^2 & \equiv x_1 \pmod{p}\\
        y_2^2 & \equiv x_2 \pmod{p}
      \end{align*}
      so
      \[ (y_1y_2)^2 \equiv x_1x_2 \pmod{p} \]
      and $x_1x_2$ is also a quadratic residue.
    \item
      Let's show that there is one square root between $0$ and $\frac{p-1}{2}$ and one between $\frac{p-1}{2}+1$ and $p-1$.
      Using a primitive root $g$ of $p$, and $\alpha,\beta$ such that $g^\alpha = a$ and $g^\beta = b$,
      \begin{align*}
        a^2 & \equiv b^2 \pmod{p}\\
        g^{2\alpha} & \equiv g^{2\beta} \pmod{p}\\
        2\alpha & \equiv 2\beta \pmod{p-1}\\
        \alpha & \equiv \beta \pmod{\frac{p-1}{2}}.
      \end{align*}
      Hence the uniqueness of the square root between each half.
      A quadratic residue has therefore either 1 or 2 roots.
      But if $a^2 \equiv x \pmod{p}$, we also have $(-a)^2 \equiv x \pmod{p}$ and $a \not\equiv -a \pmod{p}$ since $p$ is odd and $a \neq 0$.
      As a consequence, it cannot have only 1 root.
    \item
      If all quadratic residues has 2 roots.
      Since there is $p$ roots and $a$ cannot be the root of 2 different quadratic residues,
      there must be $\frac{p-1}{2}$ quadratic residues (since we do not consider 0).
      The probability is therefore $1/2$.
    \item
      If $x$ is a QR, there is $y$ such that $y^2 \equiv x \pmod{p}$.
      We have
      \begin{align*}
        x^{\frac{p+1}{2}}
        & \equiv x^{\frac{p-1}{2}}x \pmod{p}\\
        & \equiv y^{p-1}x \pmod{p}\\
        & \equiv x \pmod{p}
      \end{align*}
      The other one is $-x^{\frac{p+1}{4}}$.
      If both of them is a quadratic residue, then $-1 = \frac{-x^{\frac{p+1}{4}}}{x^{\frac{p+1}{4}}}$ is one too.
      We know that one of its square root is $(-1)^{\frac{p+1}{2}} = 1$ since $\frac{p+1}{2}$ is even which is absurd
      since $1^2 \not\equiv -1 \pmod{p}$.
    \item
      It is obvious that if $0 \leq a < p-1$ is even, $g^a$ is a QR.
      Since only half are QR, $g^a$ cannot be a QR if $a$ is odd.
      $g^a$ is therefore a QR iff $a$ is even.

      $g^{ab}$ is a QR if $a$ or $b$ are even or equivalently
      if $g^a$ or $g^b$ are QR.
      If $g^a$ and $g^b$ are odd, $h_b$ cannot be a QR.
      If it is we are sure that $b = 0$.

      Let build an adversary $\A$ that outputs 1 if $1$ if the parity of
      $ab$ is consistent with the parity of $a$ and $b$ and $0$ otherwise.
      \begin{itemize}
        \item if $b = 0$, we win iff $c$ has a different parity than $ab$, so the probability of winning $1/2$.
        \item if $b = 1$, we win.
      \end{itemize}
      Therefore
      \[ \Pr[\DDH_{\A,\G}(n) = 1] = \frac{1}{2} + \frac{1}{4} = \frac{3}{4}. \]
      DDH is solved since $\frac{1}{4}$ is non-negligible.
  \end{enumerate}
\end{solution}

\subsection{}
\begin{solution}
  \begin{enumerate}
    \item
      We have
      \[ \epsilon_7 + \epsilon_{11} \equiv 78 \equiv 1 \pmod{77} \]
      and
      \[ \varphi(1) = (1,1) = (1,0) + (0,1). \]
      We also have
      \[ \epsilon_7 \cdot \epsilon_{11} \equiv (2 \cdot 11) \cdot (8 \cdot 7) \equiv 16 \cdot 77 \equiv 0 \pmod{77} \]
      and
      \[ \varphi(0) = (0,0) = (1,0) \cdot (0,1). \]
    \item           
      We have: $\epsilon_p = q(q^{-1} \pmod{p}$ and $\epsilon_q = p(p^{-1} \pmod{q}$, so, $\epsilon_p + \epsilon_q = q(q^{-1} \pmod{p}) + p(p^{-1} \pmod{q})$ where p and q are relatively prime.
      

      And also $\epsilon_p + \epsilon_q = q u + p v$ where u and v are the inverses of p (resp q). We have $q u + p v = 1 \pmod{pq}$ (see Bézout theorem).
      We also have $\epsilon_p \epsilon_q = pq(q^{-1} \pmod{p})(p^{-1} \pmod{q}) = 0 \pmod{pq}$. Whe then have a generalization:


      Finally: $\varphi(1) = (1, 1)$ and $\varphi(0) = (0, 0)$.

    \item
      For the sum whe have: $x = k_1 p + x_p$ and $y = k_2 p + yp$, then whe have $x + y = p(k_1 + k_2) + (x_p + y_p)$.


      For the multiplication: $\varphi(x)\varphi(y) = (x_p, x_q)(y_p, y_q) = (x_p y_p, x_qy_q) = ((xy)_p, (xy)_q) = \varphi(xy)$.
  \end{enumerate}


  The decryption can take advantage of it since it knows $p$ and $q$.
\end{solution}

\subsection{}
\begin{solution}
  \begin{enumerate}
    \item
      If $x \in QR(n)$, there is $y$ such that $y^2 \equiv x \pmod{n}$.
      We have seen that $(y_p^2,y_q^2) = \varphi(y)^2 = \varphi(y^2) = \varphi(x)$
      so $x$ is also a QR modulo $p$ and $q$.

      If $x \in QR(p)$ and $x \in QR(q)$, there is $r_1,r_2$ such that
      $r_1^2 \equiv x_p \pmod{p}$ and $r_2^2 \equiv x_q \pmod{q}$.
      We have seen that
      $x = \varphi^{-1}(x_p,x_q) = \varphi^{-1}(r_1^2,r_2^2) = \varphi^{-1}(r_1,r_2)^2$
      so $x \in QR(n)$.

      If $F$ is a injection, it is also a bijection since it maps $QR(n)$ to itself.
      Let's prove that it is an injection.
      If $x^2 \equiv y^2 \pmod{p}$, $x_p^2 \equiv y_p^2 \pmod{p}$ so
      $(x_p-y_p)(x_p+y_p) \equiv 0 \pmod{p}$.
      Since $p$ is prime, we have either
      $(x_p-y_p) \equiv 0 \pmod{p}$
      or
      $(x_p+y_p) \equiv 0 \pmod{p}$
      The second one is impossible as we have seen in the exercise 3 since that would mean that $-1 \in QR(p)$
      which is absurd since $p \equiv 3 \pmod{4}$.
      We have therefore $x_p \equiv y_p \pmod{p}$,
      the same reasoning gives $x_q \equiv y_q \pmod{q}$ so $x \equiv y \pmod{n}$.
    \item
      Half the numbers are in $QR(p)$ and half are in $QR(q)$ so $1/4$ are in $QR(n)$.
      So each elements of $QR(n)$ has 4 roots.
      These are simple obtained by using the CRT on $\pm$ the square root modulo $p$
      and $\pm$ the square root modulo $q$.
      For example, with 36, we have $6$ for both modulo 7 and 11.
      The 4 square roots modulo $n$ are therefore $\varphi^{-1}(\pm 1, \pm 5)$.
      \begin{align*}
        \epsilon_7 + 5\epsilon_{11} & \equiv -6 \pmod{77}\\
        \epsilon_7 - 5\epsilon_{11} & \equiv -27 \pmod{77}\\
        -\epsilon_7 - 5\epsilon_{11} & \equiv 6 \pmod{77}\\
        -\epsilon_7 + 5\epsilon_{11} & \equiv 27 \pmod{77}.
      \end{align*}
      $F^{-1}(36)$ is the one that is in $QR(n)$ so in $QR(p)$ and $QR(q)$.
      We see that $1^2 \equiv 1 \pmod{7}$ and $4^2 \equiv 5 \pmod{16}$ so $F^{-1}(36) = -6$.
    \item
      \begin{enumerate}
        \item
          We would have $n | (x_0-x_1)(x_0+x_1)$.
          However, since $x_0 \not\equiv \pm x_1 \pmod{n}$,
          $n \nmid (x_0-x_1)$ and $n \nmid (x_0 + x_1)$.
          Therefore, $\gcd(x_0-x_1,n)$ cannot $n$ and cannot be 1 either
          since that would mean that $n | (x_0 + x_1)$.
          Therefore $\gcd(x_0-x_1,n)$ is $p$ or $q$.
        \item
          Let's pick a random $x_1$ and run the the preimage algorithm on $x_1^2$.
          If it outputs $x_0 = \pm x_1$, restart over.
          The probability of this working at each iteration is $1/2$ since there is exactly $4$ roots,
          we have $1/2$ chance to pick the one that is not outputted by the algo.
      \end{enumerate}
    \item
      For an input message $m$, take
      \[ c := m^2 \pmod{n}. \]
      For the decryption, since there is 4 roots, a mechanism should be provided for the Decryption to know which one it is.
      This is why RSA is more used in practice while it has the desadvantage of having no proof to be as unbreakable as factoring.

      It is not CPA secure since it is not randomized.
  \end{enumerate}
\end{solution}

\subsection{}
\begin{solution}
  \begin{enumerate}
    \item
      We know that
      \[ e_{eve}d_{eve} \equiv 1 \pmod(N) \]
      so $e_{eve}d_{eve}-1$ is a multiple of $N$.
    \item
      Let's pick a random $g \in \mathbb{Z}_N^*$ different from $\pm 1$.
      Let $k'$ such that $k = k'\phi(N)$.
      We know that $g^k \equiv 1 \pmod{N}$.
      Considering the sequence
      $g^k, g^{k/2}, g^{k/4}, g^{k/8}, \ldots, g^{k/2^{e_1}}$
      where $e_1$ is the exponent of $2$ in the prime decomposition of $k$.
      We know that
      \begin{align*}
        1
        & = (g^{k/2^{e_1}})^{2^{e_1}} \pmod{N}\\
        ((g^{k/2^{e_1}})^{2^{e_1}} - 1) & = 0 \pmod{N}\\
        ((g^{k/2^{e_1}})^{2^{e_1-1}} - 1)((g^{k/2^{e_1}})^{2^{e_1-1}} + 1) & = 0 \pmod{N}
      \end{align*}
      If $g^{k/2^{e_1}} \not\equiv \pm 1 \pmod{N}$,
      $N$ divides neither of the 2 factors and we can get $p$ or $q$
      by computing $\gcd((g^{k/2^{e_1}})^{2^{e_1-1}} - 1, N)$.

      If $g^{k/2^{e_1}} \equiv \pm 1 \pmod{N}$,
      we need to take another $g$ at random.

      Sadly, I have found that this does not necessarily work.
      If $p \equiv q \equiv 3 \pmod{4}$ for example (let's also say $k' = 1$ for simplicity),
      we have
      \begin{align*}
        g^{k} & \equiv 1 \pmod{N}\\
        g^{k/2}
        & \equiv g^{2\frac{p-1}{2}\frac{p-1}{2}} \pmod{N}\\
        & \equiv 1 \pmod{p}\\
        & \equiv 1 \pmod{q}\\
        & \equiv 1 \pmod{N}\\
      \end{align*}
      $g^{k/4}$ cannot be different from $-1$ or $1$ because that would mean that
      we have found another root for $1$ than $\pm 1$ modulo $p$ or $q$.
    \item
      I think I don't understand the question right.
      Charly only sees $N$, $e_{alice}$ and $e_{bob}$
      but $e_{alice}$ and $e_{bob}$ gives him no infomation.
      it is just number relatively prime to $N$,
      we could have computed them just from $N$.
  \end{enumerate}
\end{solution}


\section{}
\subsection{}
\begin{solution}
  \begin{description}
    \item[computationally hiding]
      It has to know whether $c \in \Ima(G)$ and whether $c \xor R \in \Ima(G)$.
      This is easy to do with unbounded power so it is not perfectly hiding.

      This is impossible to do in less than $2^{n-1}$ with a random function.
      If $\A(G,R)$ can beat computational hiding, we can build $\A'^g$ that gives $g^{s_1}$ to $\A$.
      If $\A$ outputs 0, $\A'$ says that $g$ is a PRF and if $\A$ outputs 1,
      that means that it is wrong so that may be because $g$ is a random function
      so $\A'$ says that $g$ is a random function.
    \item[computationally biding]
      It is not perfectly biding since we can loop over all $s$ of $\{0,1\}^n$ and find two $s_1,s_2$ such
      that $G(s_1) \xor R = G(s_2)$.
      This is $O(\sqrt{2^n})$ by the Birthday Paradox.

      It is however computationally binding.
      Let's suppose that $\A(G,R)$ can output $(c, (s_1,b_1), (s_2, b_2))$ such that
      $G(s_1) \xor \langle b_1,R \rangle = c = G(s_2) \xor \langle b_2,R \rangle$.
      This is impossible with a random function, the best we can do with random function is the technique described just below.

      We can therefore build a distinguisher $\D$ which will give the function it is given to $\A$.
      If succeeds, it concludes that it is a PRF, otherwise, it concludes that this is a random function.
  \end{description}
\end{solution}

\subsection{}
\begin{solution}
  \begin{enumerate}
    \item
      This scheme is perfectly hiding.
      $r$ is picked at random uniformly in $\mathbb{Z}_N^*$
      so $c(0) = r^{2}$ is uniformly distributed in $QR(N)$ since each element of $QR(N)$ has the same number of roots.
      $c(1) = mr^{2}$ is also uniformly distributed in $QR(N)$
      since this is a subgroup of $\mathbb{Z}_N^*$.
      Therefore both $c(0)$ and $c(1)$ are random elements uniformly distributed in $QR(N)$.
    \item
      Because it's perfectly hiding, it cannot be perfectly binding.
      If there is a PPT adversary $\A$ that can find with non-negligible probability
      $r_{0}$, $r_{1}$ such that $\Open(c, (r_{0}, 0))=0$ and
      $\Open(c, (r_{1}, 1))=1$. This would mean that
      \[ c(0)=c(1) \Leftrightarrow r_{0}^{2}=mr_{1}^{2} \Leftrightarrow m=(r_{0}/r_{1})^{2}. \]
      This means that from $\A$, we can build $\A'$ that find the square roots of $m$ with non-negligible probability.
      As seen from exercise~5 of APE3, this means that we can factor $N$.

      Therefore, under the assumption that factoring $N$ is hard, the scheme is computationally binding.
    \item
      If the commiter choose $m$, he has the factorisation of $N$ so it is not binding.
      As we have seen in exercise~3 of APE3, we can find the square root of $m$ modulo $p$ and $q$
      and using the CRT, we obtains its square root modulo $m$.
      It suffices now to take $r_0$ and $r_1$ such that $r_{0}/r_{1}$ is this square root which is easy (take for example $r_1 = 1$).
    \item
      If $m \not\in QR(N)$, it is not perfectly hiding since the other party than the commiter can
      check if $c \in QR(N)$.
      If it is, $b = 0$ and if it is not, $b = 1$.

      It is not computationally hiding either since the other party have the factorisation of $N$
      so he can easily check if $c \in QR(N)$ with a PPT algorithm.
    \item
      He can ask, once the commitment has been opened,
      one square root at the other party.
      If he can't give it or if it is not a square root of $m$, he knows that the other party has cheated.
    \item
      \begin{itemize}
        \item $\Gen(1^n)$ sets $pk$ as $(N,e,m)$, where
          \begin{itemize}
            \item $N$ is a RSA modulus generated by $\G_\mathrm{RSA}(1^n)$
            \item $e$ is a element of $\mathbb{Z}_{\phi(N)}^*$
            \item $m$ is a random element of $\mathbb{Z}_{N}^*$
          \end{itemize}
        \item $\Com_{pk}(b)$ with $b \in \{0,1\}$ provides $(c,d)$ where:
          \begin{itemize}
            \item $c = m^br$, $r$ is a random element of $\mathbb{Z}_N^*$
            \item $d := (r, b)$
          \end{itemize}
        \item $\Open_{pk}(c, d)$ outputs $b$ if it can recompute $c$ from $d$ and $pk$, or $\perp$ otherwise
      \end{itemize}
      \begin{enumerate}
        \item
          This scheme is perfectly hiding because $c$ is a random element of $\mathbb{Z}_N^*$.
          The difference here is that $e$ is relatively prime to $\phi(N)$ since $e \in \mathbb{Z}_{\phi(N)}^*$ (this was not the case of 2, since $\phi(N) = (p-1)(q-1)$ is even).
          Therefore, $e$ has an inverse $d$ modulo $\phi(N)$ and every $x \in \mathbb{Z}_N^*$ has a $e$th root which is $x^d$.
        \item
          If there is a PPT adversary $\A$ that can find with non-negligible probability
          $r_{0}$, $r_{1}$ such that $\Open(c, (r_{0}, 0))=0$ and
          $\Open(c, (r_{1}, 1))=1$. This would mean that
          \[ c(0)=c(1) \Leftrightarrow r_{0}^e=mr_{1}^e \Leftrightarrow m=(r_{0}/r_{1})^e. \]
          This means that from $\A$, we can build $\A'$ that find the $e$th roots of $m$ with non-negligible probability.
          This means that $\A'$ can solve RSA problem!

          Under the assumption that the RSA problem is hard, the scheme is computationally binding.
        \item
          If the commiter choose $m$, he has the factorisation of $N$ so it is not binding.
          The commiter can find $\phi(N)=(p-1)(q-1)$ and find the inverse $d$ of $e$ modulo $\phi(N)$
          and ``decrypt'' $m$ ``à la RSA''.
          It suffices now to take $r_0$ and $r_1$ such that $r_{0}/r_{1}$ is this ``decryption'' which is easy (take for example $r_1 = 1$).
        \item
          If $m$ is not in $\mathbb{Z}_N^*$, then the commiter can compute $\gcd(m, N)$ and find a factor of $N$, the scheme won't be binding.
          The other party can easily check if $c \in \mathbb{Z}_N^*$ so it won't be hiding either.
        \item
          He can simply check that $\gcd(m, N) = 1$.
      \end{enumerate}
  \end{enumerate}
\end{solution}

\subsection{}
\begin{solution}
  \begin{enumerate}
    \item
      We define $\langle \Gen, \Com, \Open \rangle$ as:
      \begin{itemize}
        \item $\Gen(1^n, 1^l)$ sets $pk$ as $(p,q,g)$ where $q > 2^l$ ang $g$ has order $q$ modulo $p$ (since $\phi(p)$ is even, that means that $g \in QR(p)$).
        \item $\Com_{pk}(x_1, \ldots, x_n)$ provides $(c,d)$ where:
          \begin{itemize}
            \item $c := g^r g_1^{x_1} g_2^{x_2} \cdots g_n^{x_n}$ (for a random $r \in \mathrm{Z}_q$)
            \item $d := r$
          \end{itemize}
        \item $\Open_{pk}(c,d)$ outputs $x_1, \ldots, x_n$ if it can recompute $c$ from $d$ and $pk$,
          or $\perp$ otherwise.
      \end{itemize}
      We can see that there are different possible $x_1, \ldots, x_n$ that are valid.
      If we fix $x_2, \ldots, x_n$,
      there is an $x_1$ such that $g_1^{x_1} = c / (g^r g_2^{x_2} \cdots g_n^{x_n})$
      so there is $q^n$ possible opening.
      However, it is not easy to find for a PPT algorithm.

      I should maybe have defined $d := (r, x_1, \ldots, x_n)$ because here it is weird because $\Open$ can have different outputs.
    \item
      For a random $\alpha$, we need to find it from $g^\alpha$.

      Pick a random $i^*$ and set $g_{i^*} = g^\alpha$ and for $i \neq i^*$, random $\alpha_i$ and $g_i = g^{\alpha_i}$.
      From $g^rg_1^{x_1} \cdots g_n^{x_n} = g^{r'}g_1^{x_1'} \cdots g_n^{x_n'}$ we get
      \[ r + \alpha_1 x_1 + \cdots + \alpha_n x_n \equiv r' + \alpha_1 x_1' + \cdots + \alpha_n x_n' \pmod{q} \]
      so we get
      \[ (x_{i^*} - x_{i^*}') \alpha \equiv r' - r + \sum_{i \neq i^*}^n (x_i' - x_i) \alpha_i \pmod{q} \]
      We know that at least one $x_i \neq x_i'$ so we have at least one chance out of $q$ that $x_{i^*} \not\equiv x_{i^*}' \pmod{q}$.
      If this is the case, we can find the invers of $(x_{i^*} - x_{i^*}')$ and solve the $\DLog$ problem.
    \item
      Thanks to $r$, the commitment is uniformly distributed in $G$.

      The size is $n$ times smaller.
  \end{enumerate}
\end{solution}

\subsection{}
\begin{solution}
  No, they should use a commitment scheme.
  As a commitment point of view, this scheme is perfectly hiding, bit it is not binding.
  For example, if Bob has sent $c_{\text{bob}} := \mathsf{abraveboy} \xor k_{\text{bob}}$ and sees that it is a girl,
  he can say that its key was $c_{\text{bob}} \xor \mathsf{acutegirl}$ and Alice will
  compute $c_{\text{bob}} \xor (c_{\text{bob}} \xor \mathsf{acutegirl}) = \mathsf{acutegirl}$ and think that he guessed right.
\end{solution}

\subsection{}
\begin{solution}
  Let's show that $\A$ can break CDH, we can build $\A'$ that can break DDH.
  \begin{itemize}
    \item $\A'$ is given $\mathbb{G}, q, g, (g^x, g^y, h_b)$.
    \item $\A'$ gives $\mathbb{G}, q, g, g^x, g^y$ to $\A$.
    \item $\A'$ receives $h$ from $\A$.
    \item If $h = h_b$, $\A'$ outputs 1, otherwise, it outputs 0.
  \end{itemize}
  Let's analyse our probability of breaking DDH
  \begin{itemize}
    \item
      If $\A$ succeeds in finding $h = g^{xy}$ the only think that could go wrong
      is if the random $z = xy$ (probability of $1/q$) when $b = 0$ (probability of $1/2$).
    \item
      Even if $\A$ loses and
      \begin{itemize}
        \item $b = 0$, we win if $g^z \neq h$ which happens with probability $1-1/q$;
        \item $b = 1$, we lose.
      \end{itemize}
  \end{itemize}
  In conclusion,

  \begin{align*}
    \Pr[\DDH_{\A',\G}(n) = 1]
    & = \Pr[\DDH_{\A',\G}(n) = 1]\left(1 - \frac{1}{2q}\right)\\
    & \quad + (1 - \Pr[\DDH_{\A',\G}(n) = 1])\left(\frac{1}{2}-\frac{1}{2q}\right)\\
    & = \frac{1}{2} + \frac{\Pr[\DDH_{\A',\G}(n) = 1]}{2} - \frac{1}{2q}
  \end{align*}
  where $\frac{\Pr[\DDH_{\A',\G}(n) = 1]}{2} - \frac{1}{2q}$ is non-negligible if $\Pr[\DDH_{\A',\G}(n) = 1]$
  is non-negligible.
\end{solution}


\section{}
\subsection{}
\begin{solution}
  Two solution have been proposed
  \begin{itemize}
    \item
      Let's show that if $f$ is not one way, then $\Pi$ is not existentially unforgeable.
      Let $\A$ be the inverter of $f$, we will build $\A'$ that builds an existential forgery with non-negligible probability.

      \begin{itemize}
        \item $\A'$ receives $pk = (y_0, y_1)$
        \item $\A'$ ask the signature of 0 and gets $\sigma$, he does not really care about it
        \item $\A'$ gives $y_1$ to $\A$ which outputs $x_1'$
        \item $\A'$ outputs $(1, x_1')$
      \end{itemize}
      Since $y_1$ is the image of a random $x_1$, we are exactly in the inverting experiment so $f(x_1') = y_1$
      whith probability $\Pr[\Invert_{\A,f}(n) = 1]$.

      We know that
      \[
        \Pr[\Sigforgeone_{\A',\Pi}(n) = 1] = \Pr[\Invert_{\A,f}(n) = 1]
      \]
    \item
      Let assume that $f$ is not one way function $y = f(x)$.
      Then, we can recover $x$ with a non negligible probability $\epsilon_x (n)$.

      So, $(y_0, y_1) \Rightarrow (x_0, x_1)$ with probabilities $(\epsilon_{x_0} (n), \epsilon_{x_1} (n))$.
      We cannot compute a pre-image by asking the oracle. So I output $(0, x_1)$ and $(1, x_2)$ as a forgery.

      The probability $\Pr[\Sigforge(y)] = 2\epsilon(n)$, non-negligible since f is not one way.
  \end{itemize}
\end{solution}

\subsection{}
\begin{solution}
  \begin{enumerate}
    \item
      A has $(i, \sigma(i))$ whith $\sigma (i) = f^{n-i} (x)$. We know (because $f$ is a permutation function) that:
      $$f(\sigma(i)) = f^{n-i+1}(x) = f^{n-(i-1)}(x) = \sigma(i-1)(x)$$
      Then it's possible to compute a valid forgery for every $j < i$. The scheme is then not a one time-signature.
    \item
      Need schema drawn at TP !! It's a lot simpler with it...

      $\Pr[Success (A_\sigma)] = \epsilon(\lambda))$, $\Pr[Abort] = \frac{n-k}{n}$, $\Pr[Success] = \frac{n-k}{n-m-1}$ then:
      $$ \Pr[Success(A_{owf})] = \epsilon(\lambda) \frac{n-k}{n} \frac{n-k}{n-m-1}$$

      If $\epsilon(\lambda)$ is not negligible, then the probability of success is not negligible.
    \item
      We have $s_k = (x, x')$, $p_k = (f^n(x), f^n(x'))$.
      Then $m \rightarrow \sigma = (f^{n-m}(x), f^m(x'))$.

  \end{enumerate}
\end{solution}

\subsection{}
\begin{solution}
  \begin{enumerate}
    \item
      $(r, s) = ([g^k \pmod{p}] \pmod{q}, [H(m) + xr]k^{-1} \pmod{q})$, then:
      $u_1 = H(m)s^{-1} \pmod{q}$,  $u_2 = rs^{-1} \pmod{q}$, $r = [g^{u_1} g^{u_2} \pmod{p}] \pmod{q}$, $y = g^x$

      $$\Rightarrow g^{u_1 + xu_2} = g^{s^{-1}(H(m) + rx)} = [g^k \pmod{p}] \pmod{q} = r$$
    \item
      $s = (H(m) + xr)k^{-1} \pmod{q}$, $s' = (H(m') + xr)k^{-1} \pmod{q}$ ($s \neq s'$ otherwise we have a collision).
      $s - s' = (H(m) - H(m'))k^{-1} \pmod{q}$, $k = \frac{H(m) - H(m')}{s - s'}$.
      $s = (H(m) + xr)k^{-1}$ so $\frac{sk - H(m)}{r} = x$ where x is the secret.
  \end{enumerate}
\end{solution}

\subsection{}
\begin{solution}
  \begin{enumerate}
    \item
      It is not a secure scheme.
      We can ask for the signature of $m_1$ which is $(\sigma_1, \sigma_2, s)$ and give the existential forgery
      $(m, (\sigma_1, \Mac_s(m), s))$.
      The signature $\sigma_1$ has been replayed
    \item
      It is a secure scheme.
      Let's prove that if $\A$ an produce an existential forgery,
      we can build one for either the regular signature scheme or the one-time signature scheme.

      $\A$ does the $q(n)$ queries $m_1, \ldots, m_{q(n)}$ and receive the signatures $(\sigma_{1;i}, \sigma_{2;i}, {vk'}_i)$.
      If it then outputs an existential forgery $(m, (\sigma_1, \sigma_2, vk'))$, let's consider 2 cases
      \begin{itemize}
        \item
          Either $vk' \neq {vk'}_i$ for all $i$.
          Does that mean that $\sigma_2$ is an existential forgery for the one-time signature scheme ?
          We could say ``\emph{Not necessarily since $sk',vk'$ are picked at random, with very bad luck, two same could have been picked.
          And even if they are different, thinking about Lamport, some $y$ could be the same between two keys !}''.
          However here we deal with \emph{any} one-time signing scheme so we cannot reason about Lamport at a lower level and find an invertion of $f$.

          Fortunately, this is absolutely not a problem.
          An adversary $\A'$ trying to find an existential forgery for the one-time signature scheme has access $\Sig$,
          it is allowed to pick new random $(sk', vk')$ and compute $\sigma_2$ for those.
          Since it is PPT, it only has the time to pick a polynomial number of those so it is not a problem.

          Let's do a reduction where it is clear that there is no problem
          \begin{enumerate}
            \item $\A'$ receives $vk'$ and pick a random $i^* \in \{0, \ldots, q\}$ (he tries to guess for which ${vk'}_i$ will be reused).
            \item $\A'$ gives $pk$ to $\A$
              \begin{itemize}
                \item When $\A$ sends $m_i$ for $i \neq i^*$, $\A'$ pick random $({sk'}_i, {vk'}_i)$ and
                  computes sends $(\Sign_{sk}({vk'}_i), \Sig_{{sk'}_i}(m_i), {vk'}_i)$ to $\A$.
                \item When $\A$ sends $m_i$ for $i = i^*$, $\A'$ asks its single query $m_i$ and receives $\sigma_{2;i} = \Sig_{sk'}(m_i)$.
                  He computes $\sigma_{1;i} = \Sign_{sk}({vk'}_i)$ and sends $(\sigma_{1;i}, \sigma_{2;i}, vk')$ to $\A$.
              \end{itemize}
            \item $\A$ tries outputs an existential forgery, if it is valid and $vk' = {vk'}_{i^*}$ (probability $1/q(n)$),
              $\A'$ also has an existential forgery and wins.
          \end{enumerate}
        \item
          Either $vk' \neq {vk'}_i$ for all $i$.
          That means that $\sigma_1$ is an existential forgery for the regular signature scheme.

          From an adversary $\A$ for this scheme, we
          can build an adversary $\A''$ for the regular signature scheme
          simply outputs $(m,\sigma_1)$ when $\A$ outputs its
          existantial forgery $(\sigma_1, \sigma_2, m)$
          hoping that $vk'$ is not the same as an previous ${vk'}_i$.
          In that case, it will win if $\A$ wins.
      \end{itemize}

      In summary, we have
      \begin{align*}
        \Pr[\Sigforge_{\A}(n) = 1]
        & = \Pr[\Sigforge_{\A}(n) = 1 \land (\exists i: vk' = {vk'}_i)]\\
        & \quad + \Pr[\Sigforge_{\A}(n) = 1 \land (\forall i: vk' \neq {vk'}_i)]\\
        & \leq \Pr[\Sigforge_{\A''}(n) = 1] + \frac{\Pr[\Sigforgeone_{\A'}(n) = 1]}{q(n)}
      \end{align*}
      which is non-negligible since $\Pr[\Sigforge_{\A''}(n) = 1]$ and $\Pr[\Sigforgeone_{\A'}(n) = 1]$ are negligible and $q(n)$ is polynomial.
  \end{enumerate}
\end{solution}

\subsection{}
\copypaste{4}{1}

\subsection{}
\copypaste{4}{2}

\subsection{}
\copypaste{4}{3}

\subsection{}
\copypaste{4}{4}

\subsection{}
\copypaste{4}{5}


\section{}
\subsection{}
\begin{solution}
  The material necessary for this exercise has not been seen this year (2014-2015).
\end{solution}

\subsection{}
\begin{solution}
  \begin{enumerate}
    \item I guess he means ``completeness''.
      In this case, it is rather obvious.
      He has $x$ then it can computes $s \equiv r - cx \pmod{q}$.
    \item From
      \begin{align}
        \label{eq:schnorr1}
        s   & \equiv r - c  x \pmod{q}\\
        \label{eq:schnorr2}
        s^* & \equiv r - c^*x \pmod{q}
      \end{align}
      we have $\eqref{eq:schnorr1} - \eqref{eq:schnorr2}$:
      \[ s - s^* \equiv -(c - c^*)x \pmod{q} \]
      hence
      \[ x \equiv -(s - s^*)(c - c^*)^{-1} \pmod{q}. \]
    \item
      Let's do a reduction: we will show that if there is an
      adversary $P^*$ that can convince $V$, we can build an adversary $\A$ that solve the $\DLog$ problem.

      Let $P^*$ be such that $P^*$ does not know $x$.
      Let's run it.
      When it gives $t$, let's save its state and then give him $e$.
      It will output $s$.
      We will now rewind it to the previous state and give him $e' \neq e$.
      With the $s'$ he outputs, we can find $x$.

      Under the assumption that $\DLog$ is hard, this scheme is therefore sound.
    \item
      Not, what is convincing is the fact that he have found $s$ \emph{after} we have chosen
      $s$.

      Here, he could have just chosen random $c, s$ and picked $t \equiv y^cg^s \pmod{p}$.
    \item
      The verifier has $\langle t,c,s \rangle$ in its transcript.
      We have just seen that it this can be easily computed without knowledge of $x$.
      A simulator can therefore easily output $\langle t,c,s \rangle$ without knowledge of $x$.

      However, it has only been proven that is is zero-knowledge
      if the verifier is ``Honest'' which means that he will pick $c$ at random.
      If he picks $c$ as a function of $t$, it's getting hard to find triple $\langle t,c,s \rangle$
      for the simulator since $c$ is a function of $t$ so we cannot compute $t$ after having chosen
      $c$ and $s$ at random.
      For example, a dishonest verifier could pick $c := \mathcal{H}(t)$ where $\mathcal{H}$ is a hash function.
    \item \emph{I don't understand the Hint ``$m$ is the message to sign'', what message ???}

      We have seen that $c$ is a function of $t$,
      it is not easy to find $\langle t,c,s \rangle$.
      Maybe if we show in the triple that $c$ is a function of $t$,
      we can prove that we know $x$.

      $P$ will randomly pick $r \in_R \mathbb{Z}_q$ and compute $t \equiv g^r \pmod{p}$.
      He will then compute $c = H(y\|t)$ that everybody can recompute to see that $c$ could not have been chosen \emph{a priori}.
      He can now use his knowledge of $x$ to find $s \equiv r - cx \pmod{q}$ and send $\langle t, c, s \rangle$.

      It is sound in the ROM (Random Oracle Model).
      Let's do a reduction to show it.
      If $P^*$ does not know $x$ and is able to convince a verifier,
      we can do the same reduction than 3. but placing ourself in the body of $H$ this time, not of $V$.
      \begin{itemize}
        \item $\A$ starts $P^*$ with $y$ and answers $H(y\|t)$ with random $e$ until $P^*$ does a valid triple $\langle t^*,c^*,s^* \rangle$.
        \item $\A$ starts $P^*$ with $y$ again and answers $H(y\|t')$ with a random $e'$ until $t' = t^*$ and $\langle t',c',s' \rangle$ is valid.
          There will be a probability $\frac{1}{q}$ that $e' = e$ so we are almost sure that we have $e' \neq e$
          and $\A$ we can solve $\DLog$ as explained earlier.
      \end{itemize}
      Sadly, since $P^*$ picks $r$ at random when computing $t$, it is not likely that we will have $t' = t^*$.
      We can as well pick candidates $x_c$ for $x$ randomly and try if $g^{x_c} \equiv y \pmod{p}$ to try to solve $\DLog$,
      $\A$ is not more efficient than a brute force.

      However since we are in thre ROM, we can ``manipulate randomness'' and make $r$ be chosen a little bit more ``often'',
      if you know what I mean.
  \end{enumerate}
\end{solution}


%\section{}
\subsection{}
\begin{solution}
  Here, $2^n$ tags are valid for a single message (since $r$ can be anything).
  The problem with this scheme is that the function $\xor: \{0,1\}^n \times \{0,1\}^n \to \{0,1\}^n$
  cannot be injective.
  We have for example, $m_1 \xor m_2 = (m_1 \xor w) \xor (m_2 \xor w)$.

  For this scheme, a valid tag of $m_1 \| \cdots \| m_l$ (using $r$)
  is therefore also a valid tag for $(m_1 \xor w) \| \cdots \| (m_l \xor w)$ (using $r \xor w$).

  Our adversary $\A$ will simply submit the query $m_1 \| \cdots \| m_l$ and receive the tag $t$.
  It will then submit the forgery $((m_1 \xor w) \| \cdots \| (m_l \xor w), t)$.
\end{solution}

\subsection{}
\begin{solution}
  The IV should absolutely not be predictible (the heartbleed bug was caused by the fact that the end of the previous encryption was used as the IV).
  A common practice is to use a predictible nonce and set the IV as $F_k(\mathsf{nonce})$.

  Here is an adversary that break CPA security in 2 queries.
  \begin{enumerate}
    \item Query $(0,1)$, we either get $(0,F_k(0 \xor 0)) = (0,F_k(0))$ if $b = 0$ or $(0,F_k(0 \xor 1)) = (0,F_k(1))$ if $b = 1$.
      We could now ask the encryption of 0 and 1 to know the value of $b$.
      We can actually computes $F_k(0)$ to spare a query.
    \item Query $(1,1)$ to get $(b, F_k(1 \xor 1)) = (b, F_k(0))$.
      Here we see the devastating property of the predictability of the IV.
      We have beend able to cancel it using the beginning of the message.
      The scheme loses therefore its randomness and CPA security is easy to break.
  \end{enumerate}
  With the second query, we know $F_k(0)$ and we can see if $b = 0$ or $1$ using this information and the first query.
\end{solution}

\subsection{}
\begin{solution}
  Let's suppose that $N_1 \neq N_2 \neq N_3$ (it should be in the exercise statement).
  Let's first check that $\gcd(N_1,N_2) = 1$, $\gcd(N_2,N_3) = 1$ and $\gcd(N_3,N_1) = 1$.
  If it is not the case, we can immediately factor one of the $N_i$ and find $\phi(N_i)$, $d_i$ and decrypt $c_i$.

  Let's now consider that $N_1, N_2$ and $N_3$ are relatively prime.

  We have
  \begin{align*}
    m^3 & \equiv c_1 \pmod{N_1}\\
    m^3 & \equiv c_2 \pmod{N_2}\\
    m^3 & \equiv c_3 \pmod{N_3}
  \end{align*}
  and using the CRT we can find $c$ such that $m^3 \equiv c \pmod{N_1N_2N_3}$.

  However, if we assume that $m \leq N_i$ for $i = 1, 2, 3$ to simplify as suggested,
  that means that $m^3 < N_1N_2N_3$ (equality would mean that $N_1=N_2=N_3$ which is not true).
  We have therefore simply the regular equation (no modulo)
  \[ m^3 = c \]
  which can be computed easily (we can solve it simply with binary search for example).
\end{solution}

\subsection{}
\begin{solution}
  We have
  \begin{align*}
    \Enc(m_1m_2)
    & = (m_1m_2)^E\\
    & = m_1^Em_2^E\\
    & = \Enc(m_1)\Enc(m_2).
  \end{align*}

  This version of RSA is not randomized so breaking its CPA or CCA secure is completely obvious since we can ask
  the decryption of the same message several times.

  However, this is not the case for decryption so if this version was random, we would have to find a smarter attack.
  Using $\Enc(m_1 m_2) = \Enc(m_1)\Enc(m_2)$, we know that $\Dec(\Enc(m_1 m_2)) = \Dec(\Enc(m_1)\Enc(m_2))$ so
  $m_1 m_2 = \Dec(\Enc(m_1)\Enc(m_2))$.
  Hence we can do the query $(m_0,m_1)$ (with $m_0 \neq m_1$), receive $c_b$ and then ask for an encryption of a random $c_2$ of $m_2$ hoping that $c_2 \neq 1$ (if it is not, we pick another $m_2$ and start over).
  Afterwards, we can ask for the decryption of $c_bc_2$ which is allowed if $c_2$ is not 1.
  We will get $m_bm_2$, it is now trivial to find $m_b$ and then $b$.
\end{solution}

\subsection{}
\copypaste{5}{4}


\bibliographystyle{plain}
\bibliography{biblio}

\end{document}
