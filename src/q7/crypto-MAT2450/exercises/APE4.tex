\section{}
\subsection{}
\begin{solution}
  \begin{description}
    \item[computationally hiding]
      It has to know whether $c \in \Ima(G)$ and whether $c \xor R \in \Ima(G)$.
      This is easy to do with unbounded power so it is not perfectly hiding.

      This is impossible to do in less than $2^{n-1}$ with a random function.
      If $\A(G,R)$ can beat computational hiding, we can build $\A'^g$ that gives $g^{s_1}$ to $\A$.
      If $\A$ outputs 0, $\A'$ says that $g$ is a PRF and if $\A$ outputs 1,
      that means that it is wrong so that may be because $g$ is a random function
      so $\A'$ says that $g$ is a random function.
    \item[computationally biding]
      It is not perfectly biding since we can loop over all $s$ of $\{0,1\}^n$ and find two $s_1,s_2$ such
      that $G(s_1) \xor R = G(s_2)$.
      This is $O(\sqrt{2^n})$ by the Birthday Paradox.

      It is however computationally binding.
      Let's suppose that $\A(G,R)$ can output $(c, (s_1,b_1), (s_2, b_2))$ such that
      $G(s_1) \xor \langle b_1,R \rangle = c = G(s_2) \xor \langle b_2,R \rangle$.
      This is impossible with a random function, the best we can do with random function is the technique described just below.

      We can therefore build a distinguisher $\D$ which will give the function it is given to $\A$.
      If succeeds, it concludes that it is a PRF, otherwise, it concludes that this is a random function.
  \end{description}
\end{solution}

\subsection{}
\begin{solution}
  \begin{enumerate}
    \item
      This scheme is perfectly hiding.
      $r$ is picked at random uniformly in $\mathbb{Z}_N^*$
      so $c(0) = r^{2}$ is uniformly distributed in $QR(N)$ since each element of $QR(N)$ has the same number of roots.
      $c(1) = mr^{2}$ is also uniformly distributed in $QR(N)$
      since this is a subgroup of $\mathbb{Z}_N^*$.
      Therefore both $c(0)$ and $c(1)$ are random elements uniformly distributed in $QR(N)$.
    \item
      Because it's perfectly hiding, it cannot be perfectly binding.
      If there is a PPT adversary $\A$ that can find with non-negligible probability
      $r_{0}$, $r_{1}$ such that $\Open(c, (r_{0}, 0))=0$ and
      $\Open(c, (r_{1}, 1))=1$. This would mean that
      \[ c(0)=c(1) \Leftrightarrow r_{0}^{2}=mr_{1}^{2} \Leftrightarrow m=(r_{0}/r_{1})^{2}. \]
      This means that from $\A$, we can build $\A'$ that find the square roots of $m$ with non-negligible probability.
      As seen from exercise~5 of APE3, this means that we can factor $N$.

      Therefore, under the assumption that factoring $N$ is hard, the scheme is computationally binding.
    \item
      If the commiter choose $m$, he has the factorisation of $N$ so it is not binding.
      As we have seen in exercise~3 of APE3, we can find the square root of $m$ modulo $p$ and $q$
      and using the CRT, we obtains its square root modulo $m$.
      It suffices now to take $r_0$ and $r_1$ such that $r_{0}/r_{1}$ is this square root which is easy (take for example $r_1 = 1$).
    \item
      If $m \not\in QR(N)$, it is not perfectly hiding since the other party than the commiter can
      check if $c \in QR(N)$.
      If it is, $b = 0$ and if it is not, $b = 1$.

      It is not computationally hiding either since the other party have the factorisation of $N$
      so he can easily check if $c \in QR(N)$ with a PPT algorithm.
    \item
      He can ask, once the commitment has been opened,
      one square root at the other party.
      If he can't give it or if it is not a square root of $m$, he knows that the other party has cheated.
    \item
      \begin{itemize}
        \item $\Gen(1^n)$ sets $pk$ as $(N,e,m)$, where
          \begin{itemize}
            \item $N$ is a RSA modulus generated by $\G_\mathrm{RSA}(1^n)$
            \item $e$ is a element of $\mathbb{Z}_{\phi(N)}^*$
            \item $m$ is a random element of $\mathbb{Z}_{N}^*$
          \end{itemize}
        \item $\Com_{pk}(b)$ with $b \in \{0,1\}$ provides $(c,d)$ where:
          \begin{itemize}
            \item $c = m^br$, $r$ is a random element of $\mathbb{Z}_N^*$
            \item $d := (r, b)$
          \end{itemize}
        \item $\Open_{pk}(c, d)$ outputs $b$ if it can recompute $c$ from $d$ and $pk$, or $\perp$ otherwise
      \end{itemize}
      \begin{enumerate}
        \item
          This scheme is perfectly hiding because $c$ is a random element of $\mathbb{Z}_N^*$.
          The difference here is that $e$ is relatively prime to $\phi(N)$ since $e \in \mathbb{Z}_{\phi(N)}^*$ (this was not the case of 2, since $\phi(N) = (p-1)(q-1)$ is even).
          Therefore, $e$ has an inverse $d$ modulo $\phi(N)$ and every $x \in \mathbb{Z}_N^*$ has a $e$th root which is $x^d$.
        \item
          If there is a PPT adversary $\A$ that can find with non-negligible probability
          $r_{0}$, $r_{1}$ such that $\Open(c, (r_{0}, 0))=0$ and
          $\Open(c, (r_{1}, 1))=1$. This would mean that
          \[ c(0)=c(1) \Leftrightarrow r_{0}^e=mr_{1}^e \Leftrightarrow m=(r_{0}/r_{1})^e. \]
          This means that from $\A$, we can build $\A'$ that find the $e$th roots of $m$ with non-negligible probability.
          This means that $\A'$ can solve RSA problem!

          Under the assumption that the RSA problem is hard, the scheme is computationally binding.
        \item
          If the commiter choose $m$, he has the factorisation of $N$ so it is not binding.
          The commiter can find $\phi(N)=(p-1)(q-1)$ and find the inverse $d$ of $e$ modulo $\phi(N)$
          and ``decrypt'' $m$ ``à la RSA''.
          It suffices now to take $r_0$ and $r_1$ such that $r_{0}/r_{1}$ is this ``decryption'' which is easy (take for example $r_1 = 1$).
        \item
          If $m$ is not in $\mathbb{Z}_N^*$, then the commiter can compute $\gcd(m, N)$ and find a factor of $N$, the scheme won't be binding.
          The other party can easily check if $c \in \mathbb{Z}_N^*$ so it won't be hiding either.
        \item
          He can simply check that $\gcd(m, N) = 1$.
      \end{enumerate}
  \end{enumerate}
\end{solution}

\subsection{}
\begin{solution}
  \begin{enumerate}
    \item
      We define $\langle \Gen, \Com, \Open \rangle$ as:
      \begin{itemize}
        \item $\Gen(1^n, 1^l)$ sets $pk$ as $(p,q,g)$ where $q > 2^l$ ang $g$ has order $q$ modulo $p$ (since $\phi(p)$ is even, that means that $g \in QR(p)$).
        \item $\Com_{pk}(x_1, \ldots, x_n)$ provides $(c,d)$ where:
          \begin{itemize}
            \item $c := g^r g_1^{x_1} g_2^{x_2} \cdots g_n^{x_n}$ (for a random $r \in \mathrm{Z}_q$)
            \item $d := r$
          \end{itemize}
        \item $\Open_{pk}(c,d)$ outputs $x_1, \ldots, x_n$ if it can recompute $c$ from $d$ and $pk$,
          or $\perp$ otherwise.
      \end{itemize}
      We can see that there are different possible $x_1, \ldots, x_n$ that are valid.
      If we fix $x_2, \ldots, x_n$,
      there is an $x_1$ such that $g_1^{x_1} = c / (g^r g_2^{x_2} \cdots g_n^{x_n})$
      so there is $q^n$ possible opening.
      However, it is not easy to find for a PPT algorithm.

      I should maybe have defined $d := (r, x_1, \ldots, x_n)$ because here it is weird because $\Open$ can have different outputs.
    \item
      For a random $\alpha$, we need to find it from $g^\alpha$.

      Pick a random $i^*$ and set $g_{i^*} = g^\alpha$ and for $i \neq i^*$, random $\alpha_i$ and $g_i = g^{\alpha_i}$.
      From $g^rg_1^{x_1} \cdots g_n^{x_n} = g^{r'}g_1^{x_1'} \cdots g_n^{x_n'}$ we get
      \[ r + \alpha_1 x_1 + \cdots + \alpha_n x_n \equiv r' + \alpha_1 x_1' + \cdots + \alpha_n x_n' \pmod{q} \]
      so we get
      \[ (x_{i^*} - x_{i^*}') \alpha \equiv r' - r + \sum_{i \neq i^*}^n (x_i' - x_i) \alpha_i \pmod{q} \]
      We know that at least one $x_i \neq x_i'$ so we have at least one chance out of $q$ that $x_{i^*} \not\equiv x_{i^*}' \pmod{q}$.
      If this is the case, we can find the invers of $(x_{i^*} - x_{i^*}')$ and solve the $\DLog$ problem.
    \item
      Thanks to $r$, the commitment is uniformly distributed in $G$.

      The size is $n$ times smaller.
  \end{enumerate}
\end{solution}

\subsection{}
\begin{solution}
  No, they should use a commitment scheme.
  As a commitment point of view, this scheme is perfectly hiding, bit it is not binding.
  For example, if Bob has sent $c_{\text{bob}} := \mathsf{abraveboy} \xor k_{\text{bob}}$ and sees that it is a girl,
  he can say that its key was $c_{\text{bob}} \xor \mathsf{acutegirl}$ and Alice will
  compute $c_{\text{bob}} \xor (c_{\text{bob}} \xor \mathsf{acutegirl}) = \mathsf{acutegirl}$ and think that he guessed right.
\end{solution}

\subsection{}
\begin{solution}
  Let's show that $\A$ can break CDH, we can build $\A'$ that can break DDH.
  \begin{itemize}
    \item $\A'$ is given $\mathbb{G}, q, g, (g^x, g^y, h_b)$.
    \item $\A'$ gives $\mathbb{G}, q, g, g^x, g^y$ to $\A$.
    \item $\A'$ receives $h$ from $\A$.
    \item If $h = h_b$, $\A'$ outputs 1, otherwise, it outputs 0.
  \end{itemize}
  Let's analyse our probability of breaking DDH
  \begin{itemize}
    \item
      If $\A$ succeeds in finding $h = g^{xy}$ the only think that could go wrong
      is if the random $z = xy$ (probability of $1/q$) when $b = 0$ (probability of $1/2$).
    \item
      Even if $\A$ loses and
      \begin{itemize}
        \item $b = 0$, we win if $g^z \neq h$ which happens with probability $1-1/q$;
        \item $b = 1$, we lose.
      \end{itemize}
  \end{itemize}
  In conclusion,

  \begin{align*}
    \Pr[\DDH_{\A',\G}(n) = 1]
    & = \Pr[\DDH_{\A',\G}(n) = 1]\left(1 - \frac{1}{2q}\right)\\
    & \quad + (1 - \Pr[\DDH_{\A',\G}(n) = 1])\left(\frac{1}{2}-\frac{1}{2q}\right)\\
    & = \frac{1}{2} + \frac{\Pr[\DDH_{\A',\G}(n) = 1]}{2} - \frac{1}{2q}
  \end{align*}
  where $\frac{\Pr[\DDH_{\A',\G}(n) = 1]}{2} - \frac{1}{2q}$ is non-negligible if $\Pr[\DDH_{\A',\G}(n) = 1]$
  is non-negligible.
\end{solution}
