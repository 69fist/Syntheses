\section{}
\subsection{}
\begin{solution}
  Here, $2^n$ tags are valid for a single message (since $r$ can be anything).
  The problem with this scheme is that the function $\xor: \{0,1\}^n \times \{0,1\}^n \to \{0,1\}^n$
  cannot be injective.
  We have for example, $m_1 \xor m_2 = (m_1 \xor w) \xor (m_2 \xor w)$.

  For this scheme, a valid tag of $m_1 \| \cdots \| m_l$ (using $r$)
  is therefore also a valid tag for $(m_1 \xor w) \| \cdots \| (m_l \xor w)$ (using $r \xor w$).

  Our adversary $\A$ will simply submit the query $m_1 \| \cdots \| m_l$ and receive the tag $t$.
  It will then submit the forgery $((m_1 \xor w) \| \cdots \| (m_l \xor w), t)$.
\end{solution}

\subsection{}
\begin{solution}
  The IV should absolutely not be predictible (the heartbleed bug was caused by the fact that the end of the previous encryption was used as the IV).
  A common practice is to use a predictible nonce and set the IV as $F_k(\mathsf{nonce})$.

  Here is an adversary that break CPA security in 2 queries.
  \begin{enumerate}
    \item Query $(0,1)$, we either get $(0,F_k(0 \xor 0)) = (0,F_k(0))$ if $b = 0$ or $(0,F_k(0 \xor 1)) = (0,F_k(1))$ if $b = 1$.
      We could now ask the encryption of 0 and 1 to know the value of $b$.
      We can actually computes $F_k(0)$ to spare a query.
    \item Query $(1,1)$ to get $(b, F_k(1 \xor 1)) = (b, F_k(0))$.
      Here we see the devastating property of the predictability of the IV.
      We have beend able to cancel it using the beginning of the message.
      The scheme loses therefore its randomness and CPA security is easy to break.
  \end{enumerate}
  With the second query, we know $F_k(0)$ and we can see if $b = 0$ or $1$ using this information and the first query.
\end{solution}

\subsection{}
\begin{solution}
  Let's suppose that $N_1 \neq N_2 \neq N_3$ (it should be in the exercise statement).
  Let's first check that $\gcd(N_1,N_2) = 1$, $\gcd(N_2,N_3) = 1$ and $\gcd(N_3,N_1) = 1$.
  If it is not the case, we can immediately factor one of the $N_i$ and find $\phi(N_i)$, $d_i$ and decrypt $c_i$.

  Let's now consider that $N_1, N_2$ and $N_3$ are relatively prime.

  We have
  \begin{align*}
    m^3 & \equiv c_1 \pmod{N_1}\\
    m^3 & \equiv c_2 \pmod{N_2}\\
    m^3 & \equiv c_3 \pmod{N_3}
  \end{align*}
  and using the CRT we can find $c$ such that $m^3 \equiv c \pmod{N_1N_2N_3}$.

  However, if we assume that $m \leq N_i$ for $i = 1, 2, 3$ to simplify as suggested,
  that means that $m^3 < N_1N_2N_3$ (equality would mean that $N_1=N_2=N_3$ which is not true).
  We have therefore simply the regular equation (no modulo)
  \[ m^3 = c \]
  which can be computed easily (we can solve it simply with binary search for example).
\end{solution}

\subsection{}
\begin{solution}
  We have
  \begin{align*}
    \Enc(m_1m_2)
    & = (m_1m_2)^E\\
    & = m_1^Em_2^E\\
    & = \Enc(m_1)\Enc(m_2).
  \end{align*}

  This version of RSA is not randomized so breaking its CPA or CCA secure is completely obvious since we can ask
  the decryption of the same message several times.

  However, this is not the case for decryption so if this version was random, we would have to find a smarter attack.
  Using $\Enc(m_1 m_2) = \Enc(m_1)\Enc(m_2)$, we know that $\Dec(\Enc(m_1 m_2)) = \Dec(\Enc(m_1)\Enc(m_2))$ so
  $m_1 m_2 = \Dec(\Enc(m_1)\Enc(m_2))$.
  Hence we can do the query $(m_0,m_1)$ (with $m_0 \neq m_1$), receive $c_b$ and then ask for an encryption of a random $c_2$ of $m_2$ hoping that $c_2 \neq 1$ (if it is not, we pick another $m_2$ and start over).
  Afterwards, we can ask for the decryption of $c_bc_2$ which is allowed if $c_2$ is not 1.
  We will get $m_bm_2$, it is now trivial to find $m_b$ and then $b$.
\end{solution}

\subsection{}
\copypaste{5}{4}
