
\documentclass[11pt]{article} % use larger type; default would be 10pt

\usepackage[utf8]{inputenc} % set input encoding (not needed with XeLaTeX)

\usepackage{geometry} % to change the page dimensions
\geometry{a4paper} % or letterpaper (US) or a5paper or....
% \geometry{margin=2in} % for example, change the margins to 2 inches all round
% \geometry{landscape} % set up the page for landscape
%   read geometry.pdf for detailed page layout information

\usepackage{graphicx} % support the \includegraphics command and options
\usepackage{multicol}
\usepackage{float}
% \usepackage[parfill]{parskip} % Activate to begin paragraphs with an empty line rather than an indent

%%% PACKAGES
\usepackage{booktabs} % for much better looking tables
\usepackage{array} % for better arrays (eg matrices) in maths
\usepackage{paralist} % very flexible & customisable lists (eg. enumerate/itemize, etc.)
\usepackage{verbatim} % adds environment for commenting out blocks of text & for better verbatim
\usepackage{subfig} % make it possible to include more than one captioned figure/table in a single float
% These packages are all incorporated in the memoir class to one degree or another...

%%% HEADERS & FOOTERS
\usepackage{fancyhdr} % This should be set AFTER setting up the page geometry
\pagestyle{fancy} % options: empty , plain , fancy
\renewcommand{\headrulewidth}{0pt} % customise the layout...
\lhead{}\chead{}\rhead{}
\lfoot{}\cfoot{\thepage}\rfoot{}

%%% SECTION TITLE APPEARANCE
\usepackage{sectsty}
\allsectionsfont{\sffamily\mdseries\upshape} % (See the fntguide.pdf for font help)
% (This matches ConTeXt defaults)

%%% ToC (table of contents) APPEARANCE
\usepackage[nottoc,notlof,notlot]{tocbibind} % Put the bibliography in the ToC
\usepackage[titles,subfigure]{tocloft} % Alter the style of the Table of Contents
\renewcommand{\cftsecfont}{\rmfamily\mdseries\upshape}
\renewcommand{\cftsecpagefont}{\rmfamily\mdseries\upshape} % No bold!

%%% END Article customizations

%%% The "real" document content comes below...

\title{Gestion des Ressources Humaines}
\author{Houtain Nicolas \and Kabasele Ndonda Gorby Nicolas}

\begin{document}

\maketitle

\section{Penser la réalité humaine de l'organisation}

10 principes généraux de sociologie des organisations.

\subsection{La réification}
\begin{description}
    \item[Definition]   :    Illusion   qui   consiste    à   concevoir
l'organisation  comme  une réalité  extérieur  et  supérieur à  ses
membres, susceptible de  poursuivre ses propres intêret  et d'exercer un
pouvoir sur ses membres.
\end{description}

L'organisation n'est  pas un  organisme à part  qui serait  en relation
avec  ses   membres,  les   seules  contraintes  sociales   sont  celles
qu'exercent les hommes (agent sociaux) entre eux.

\paragraph{ } \textit{On lui donne une pseudo réalité, des pouvoirs et
contraintes à une abstraction en oubliant qu'on l'a créé}


\subsection{Dépasser les raisonnements mécaniques} Il ne faut pas voir
les agents composant une organisation comme des êtres passifs soumis au
impulsion  de l'environnement.  (\textit{tel  que une  prime  = plus  de
motivation})

\textsc{Exemple} : culture de l'entreprise.

On est pas dans une logique de cause-effet simpliste (\textit{une
personne n'est pas simplement soumis à un ensemble de ``facteurs''
ayant un impact sur lui})


\subsection{Articuler l'acteur et le système}

Deux grands travers affaiblissent le management:
\begin{enumerate}
\item Centration excessive sur le systeme (agents vu comme des produit passifs de logiques structurelles),
\item Exaltation démesurée de l'acteur (en particulier du dirigeant vu comme omnipotent et omniscient \textit{management heroique}).
\end{enumerate}

\paragraph{Note:} la combinaison voit l'entreprise comme un airbus et le 
dirigeant le pilote. (\textit{Cet avion ne doit pas faire partie de 
malaisia airlines})

\subsection{Raisonner en termes de dualité}
Deux visions pour l'organisation:
\begin{itemize}
	\item Les actions et les interactions qui s'y passent.
    \item Les élèments constitutif de \textbf{configuration} plus large.
\end{itemize}

Ces deux points de vue sont à adopter pour analyser l'organisation dans
un  sens et  dans l'autre.  (\textit{Va-vient entre  le singulier  et le
globale}).

\paragraph{Systeme}   
Conception pour expliquer des relations de contraintes et
d'interdépendance (\textit{sans assigner de réalité propre}). Tout
systeme est structuré et peut être répété.

\paragraph{Structure} 
Ensemble des propriétés générales qui font qu'un système social se
reproduit de période en période. (\textit{Le jeu n'existe que parce
que on applique les règles}).

Elles sont \textbf{virtuelles} (\textit{cfr: Ceci est une pipe})

\begin{itemize}
    \item[$\to$] Les actions sont conditionnés par le système mais le
système résulte des actions.
\end{itemize}

\paragraph{Dualité  versus  Dualisme}  
\begin{description}
    \item[Dualisme] : coexistence de deux réalité
    \item[Dualité] (\textit{doit etre substituer au dualisme}) : réalité
        unique mais double en soi
\end{description}

La dualité de l'organisation vient des actions et interactions d'une
part et du système globale d'autre part.

\textit{Une entreprise est un ensemble d'humain, elle est donc
inévitablement conditionnée par les propriétés plus générales de la
société}


\subsection{Critique de l'atomisme}

\begin{description}
\item[Conception atomique] :l'atome de la vie sociale et économique $=$ \textit{l'individu psychologique}
\end{description}

Organisation devient alors une agrégation d'atome. Conception          
individualiste et rationnelle de l'être humain.                        

Différents type de conception proche sans être les même.
\begin{enumerate}
    \item Conception monadique 
    \item Utilitarisme 
    \item Décideur rationnel 
        ses objectifs)
\end{enumerate}

\subsubsection{Conception monadique}

\begin{description}
    \item[Postulat] les causes du comportements exclusivement à l'intérieur de l'esprit d'un individu (pulsions, habitudes, traits de personnalité,...).
    \item[Conséquence] Il suffirait de selectionner le candidats le plus compétent pour avoir le meilleur
    \item[Critique] tout être humain existe par son rapport à autrui et sa position dans le champs social (ex: \textit{la vie intérieur d'un aborigene n'est pas la même qu'un cadre américain}).
\end{description}

L'entreprise à comme premier produit du travail, le travailleur lui même
puisque ce qu'ils sont dans l'entreprise les détermine en partie.

\paragraph{Note:} A cette vision il faut substituer une concetion
interactionniste où l'unité d'analyse devient l'interaction.

\subsubsection{Utilitarisme}

\begin{description}
    \item[Postulat] Toute action se réduit à la recherche d'intérêt egoiste.
    \item[Critique] L'utilitarisme est la négation même du social, c'est
        évident que tout n'est pas que calcul (même si le calcul est
        inhérent à l'homme est ne peut être nier)
\end{description}

\paragraph{Note: } Cette vision normalise l'égoïsme et la recherche du
profit avant tout.

\subsubsection{Décideur rationnel}

\begin{description}
    \item[Postulat] L'individu est un décideur rationnel
    \item[Critique] Negation des processus cognitifs, affectifs et normatifs
\end{description}


\subsection{Perspective interactioniste}
On détermine les actions humaines en les rapportant aux relations
sociales. TODO exemple de l'enqueteur.

\subsection{Passage par la subjectivité de l'acteur}
On part de la subjectivité des agents pour expliquer leur conduite,
on part de la situation vécue pas de la situation objective
(\textit{empathie}).

\paragraph{Théorme de Thomas} 
Si les hommes tiennent un situation pour réelle, même si ce n'est pas
le cas, les conséquences le seront. (ex: des ramoneur) L'empathie est
une compétence sociale décisive dans la vie du travail.


\subsection{Sous l'ordre hierarchiques les réseaux}
		Pour comprendre l'organisation, il faut s'intéresser au relation non-hiérarchique, les réseaux.
\begin{description}
\item[Réseau] Un réseau est un ensemble non hiérarchisé et largement informel de relations entre des acteurs. Ces liens sont évolutifs.
\end{description}

Réseau $\neq$ coalition, les participants n'ont pas tous les même intérêts.

\subsection{Au fondement du social, la logique de don}
		Ce qui structure les réseaux ce n'est pas du donnant-donnant mais plutôt la logique du don. Dans le don, la norme de réprocité reste implicite. Il présuppose la confiance, qui se crée au fur à mesure. La plus grande de confiance est la divulgation d'un secret.

Le lien social est fondé sur un triple geste:
\begin{enumerate}
\item Donner
\item Recevoir
\item Rendre
\end{enumerate}

\subsection{La réalité est socialement construite}
		Situation socialement construite au travers d'une variété de processus de communication. 
		\begin{itemize}
			\item \textbf{Division du travail et cognition} L'attention sélective tient sa source dans la limitation cognitif. La vie sociale détermine les foyers d'attention donc on divise le travail.
			\item \textbf{Rôle des représentations} L'environnement est abstrait (on ne sait pas tout). Représentation comme un acte de  communication (on imagine ce qu'on nous raconte). La réalité est donc socialement construite.
			\begin{description}
				\item[Validation empirique] Confronter la représentation à la réalités des faits (ex:météo)
				\item[Validation sociale] Représentation qu'on ne peut pas confronter à la réalité 		
				(ex:statistique).
			\end{description}
		\end{itemize}
		Certains dirigeant sont aveuglement socialement construit (s'entoure de gens qu'ils pensent qu'ils sont 	
		géniaux).
		\paragraph{Culture de l'entreprise} Ensemble des processus de construction sociale de la réalité 
		partagée d'une organisation.
		\textbf{Le grand problème du managemement  c'est de faire croire que l'ont décrit la réalité alors 
		qu'on l'a préscrit}


\section{La gestion des ressource humaines et les fondements de la motivation sociale}
	\subsection{Théorie du contenu versus théorie du processus}
		\paragraph{Motivation} Engagement de la personne dans son travail qui, si le contexte le permet, va entrainer un surcroit de performance.

		Deux façon de théoriser la motivation:
		\begin{description}
			\item[Théorème contenu: ] Qu'est ce que la motivation?
			\item[Théorème du processus: ] Quelles sont les facteurs qui influence la motivation?
		\end{description}
	\subsection{Motivation réduite à une phénomène purement individuel}
		Erreur de penser que la motivation est purement individuel (ex:Fight team spirit). En fonction du
		rôle qui est attribué à un acteur, le rapport au monde de l'acteur est modifié (motivation rôle).
	\subsection{Motivation comme trait de personnalité ou trait culturel}
		C'est une erreur de penser que la motivation nait d'un trait de la personnalité et le besoin de la 
		personnalité. C'est une erreur car les individus ne sont pas des personnalités figées, elles sont 	
		influencés par l'entreprise.
	\subsection{Motivation et satisfaction}
		Aucune relation n'a été clairement émise ente la satisfaction et la motivation. On ne peut pas conclure 
		que la satisfaction est une cause de motivation. Ce sont des effets très différents. L'un est 
		l'engagement et l'autre est un jugement par rapport à une situation.
	\subsection{Confusion entre motivation et satisfaction}
	Les études ont montré qu'il est faux de croire que la satisfaction détermine la motifaction. Il faut faire la distinction entre accroitre la satisfaction et accroitre la motivation.
	\begin{description}
		\item[Satisfaction] on peut définir la \textit{satisfaction} comme toute représentation de soi au moyen d'une \textit{comparaison} favorable.
	\end{description}

	\subsection{Confusion entre motivation et l'implication}
		\begin{description}
		\item[Implication] Attachement au travail, identification à la fonction. Fait état de l'importance
		que prend le travail par rapport aux autres choses de la vie. \textbf{Implication pas forcément 
		nécessaire à la motivation}
		\end{description}

	\subsection{Déni du principe de nécessité}
		La performance en amont de la motivation est le résultat d'un mélange d'intérêt pour la tâche et le poste.

	\subsection{Lien méchanique motivation = performance}
		Toute les activités comportent des dimensions:
		\begin{enumerate}
			\item Quantitative (volume de travail)
			\item Qualitative (qualité du travail)
		\end{enumerate}
		La motivation n'influe pas forcément la performance individuelle et collective. Toute motivation 
		s'actualise dans un milieu social qui peut aussi bien la relayer, l'amplifier ou au contraire l'étouffer (ex: \textit{patron qui ne tiens pas compte de proposition de cadres motivés}). 
		Performance est un concept qui n'est pas trés claire.

	\subsection{Le modèle V.I.E}
		3 types de facteurs suceptibles d'influencer la propension à s'engager dans certains types d'action:
		\begin{description}
			\item[Expectation: ] La représentation par l'individu des capacités à s'engager. Savoir s'il est capable d'atteindre l'objectif primaire. 
			\item[Instrumentalité: ] La représentation du rapport entre l'action et le résultat. Lien raisonnable à sa performance et certains résultats potentiels associé à cette performance (typiquement la réaction de la hiérarchie).
			\item[Valence: ] Valeur accordé au résultat. Elle peut être positif(ex: augmentation de salaire) ou	négative (ex: risque de licenciement)
		\end{description}
		$M=V\times I \times E$, si une des variables est à 0, la motivation est à 0. 
		\paragraph{Expectation} L'expectation peut être diviser en deux catégorie, l'\textbf{expectation de soi}
		(1,2,3) et l'\textbf{expection matérielle}(4,5,6):
		\begin{multicols}{2}
			\begin{enumerate}
				\item Ressources affectives (ex:confiance en soi)
				\item Ressources cognitives (ex: langue parlé)
				\item Ressources physique (ex: endurance)
				\item Ressources financière
				\item Ressources matérielles et technique
				\item Ressource stratégiques (ex: employés disponibles)
			\end{enumerate}
		\end{multicols}

		Certaines expectation sont vécue subjectivement même si il y a des éléments objectifs (ex: \textit{objectivement je ne parle pas chinois (ressources cognitive), subjectivement je ne me sens pas la force de l'apprendre (ressources physique)}).

		Il est important pour la motivation de fixer des objectifs (effet de but). Mais ce n'est pas forcément en fixant des objectifs que les gens seront plus motivés. Les objectifs doivent suivre le modèle SMART:
		 Spécifique, Mesurable, Ambitieux, Réaliste, délimiter dans le Temps. Ils doivent aussi:
		 \begin{multicols}{2}
		 	\begin{itemize}
				\item S'inscrire dans l'analyse de fonctions (déscription de poste)
				\item Etre SMART
		 		\item Sous le contôle de l'employé et aisément compréhensible.
		 		\item Négociés et justifié
		 		\item Non contradictoire et prioritiser.
		 		\item Il faut qu'il y ait un dispositif du feedback.
		 		\item Contrôle d'atteinte des objectifs.
		 		\item Valoriser les objectif
		 	\end{itemize}
		 \end{multicols}

		\begin{figure}[H]
		\centering
		\begin{tabular}{|l|l|}
			\hline
			\textbf{Expectation matérielle} & \textbf{Expectation de soi} \\
			\hline
			Analyse des fonctions & Organisation du travail\\
			Délégation de pouvoir & Délégation de pouvoir \\
			Allocation de moyens et gestion du temps & Politique de mobilité interne\\
			& Politique de formation\\
			& Encadrement\\
			& Coaching\\
			& Utiliser le groupe\\
			\hline
		\end{tabular}
		\caption{Moyen à disposition du GRH pour augmenter l'expectation}
		\end{figure}
		 
		 \paragraph{Instrumentalité} La probablilité d'obtenir/d'éviter tel résultat en adoptant un
		 comportement donné. La perception du résultat peut être objective ou subjective (dépend de chaque 
		 individu et de sa perception). Cette perception dépend des caractéristiques objectives du milieu de 
		 travail.
		
		L'instrumentalité pose 4 types de problèmes:
		 \begin{itemize}
		 	\item Comment la performance est reconnue est définie a tout les niveaux? 
		 	\item Comment evaluer la performance pour povuoir la sanctionner positivement ou négativement?
		 	\item Chaque employé doit avoir accès à l'information (chacun doit connaitre les "règles du jeu").
		 	\item Dépend de la confiance en la personne (ex: chef sadique qui feint une promotion)
		 \end{itemize}
		 
		 \paragraph{Valence} Les individus varient les uns les autres par rapport la valeur qu'il donne à 
		 certaines rétributions. 2 types de récompenses (valence positive):
		 \begin{itemize}
		 	\item \textbf{intrinsèques} émane du travail en lui même.
		 	\item \textbf{extrinsèques} travail non comme une finalités mais un moyen pour obtenir autre 
		 	chose qui n'est pas directement lié à la fonction.
		 \end{itemize}
		 L'entreprise influence la valeur symbolique des accomplissement humains (ex: bout de métal). Mais les
		 valences peuvent aussi être négatif, des sanctions qui aimerait être évité ou des par nécessité. 
		 Différent type de de valence.
		 \begin{multicols}{2}
		 	\begin{itemize}
		 		\item affective ou morale
		 		\item individuelle ou collective
		 		\item intrinsèque ou extrinsèque
		 	\end{itemize}
		 \end{multicols}
		Le GRH influe sur les valences en choisissant le type de gratification et de sanctions pratiquées, maus aussi par la valorisation des résultats et la promotion ou non des formes d'implications collectives.
		 \paragraph{Remarque sur le systeme V.I.E}
		 \begin{itemize}
		 	\item On parle de la motivation par rapport à un objectif, mais pas la motivation en général.
		 	\item Il est incomplet, on ne se limite à un moment $t$. La motivation est changeante dans le temps.
		 	\item Il peut quand même servir comme base pour un diagnostic d'entreprise.
		 \end{itemize}


	\subsection{La démotivation}
		La démotivation ne se définit pas comme étant l'absence de motivation, il provient souvent de la baisse d'un des facteurs VIE.
		\begin{description}
			\item[Démotivation] processus destructeur dans la relation de la personne à son travail.
		\end{description} L'absence de motivation dans un 
		domaine peut venir du fait de pas avoir envie de s'eparpiller. Deux degré d'intensité de la motivation:
		\begin{itemize}
			\item \textbf{douce} affecte la valorisation de la situation, souvent une baisse de I (ex: injustice) 
			qui 	provoque un baisse de V.
			\item \textbf{dure} affecte l'image de l'individu, une baisse de E (ex: je suis bon à rien). Cette 
			démotivation est difficilement réversible.
		\end{itemize}





\section{Justice et équité dans la GRH}
	La justice a un impact important sur la motivation (sur les 3 variables).
	\paragraph{Justice organisationnelle}: Il n'est pas suffisant de s'interroger sur la conformité d'une décision (face à un code juridique ou à un réglement) pour établir si elle juste ou pas.\\
	3 types de justices: 
	\begin{description}
		\item[Justice distributive:] Distribution des ressoureces au sens large (ex:horaire, compliment,...)
		\item[Justice procédurale:] Perception de la justice des procédures (ex Pourquoi je n'ai pas eu la 
		promotion $\to$ voir règle)
		\item[Justice interactionnelle:] La façon dont les décideurs traitent leurs subordonnée et respectent 
		leurs engagements.
	\end{description}



	\subsection{Justice distributive}
		Ce base sur la \textbf{privation relative} c'est à dire la divergence entre ce qu'on reçoit et ce qu'on 
		penser qu'on devrait recevoir. Cette divergence se calcule de toute façon:
		\begin{itemize}
			\item \textbf{temporelle} Ce qu'on a reçu aujourd'hui par rapport à ce qu'on avait reçu avant.
			\item \textbf{sociale} Ce qu'on reçoit et ce que les autres ont reçus.
		\end{itemize}
		Il y a sentiment d'inégalité/ d'injustice  lorsque $\frac{R_s}{C_s}<\frac{R_a}{C_a}$ où R sont les 
		rétribution,C le contribution , s pour soi-mêm et a pour les autres. Face à cette injustice il y a 
		différentes réaction, comme diminuer sa contribution (démotivation), se plaindre... Ce genre de justice 
		s'applique aussi au sanction. Il faut que ce soit deux poids deux mesures.




	\subsection{Justice procédurale}
		Face à une situation inéquitable, quelqu'un peut lui trouver un caractère juste si les procédure sont 
		juste (si la mise en oeuvre tiens compte des avis, suggestions, ...). Les personnes ayant un certain contrôle sur la décision ou le processus ont un plus fort 
		sentiment de justice. Il y a 6 facteurs pour  que les acteurs impliqués trouvent une procédure juste:
		\begin{description}
			\item[Cohérence:] C'est la même règle pour tout le monde.
			\item[Impartialité: ] Ne tient pas compte des intérêts (économique,sociaux,...).
			\item[Précision: ] Décision fondées sur des infos claires et précises.
			\item[Adaptabilité: ] Possiblités de corriger les décisions inappropriées.
			\item[Représentativité: ] Chaque acteur est écouté et a son mot à dire.
			\item[Éthique: ] Les décisions respectes les valeurs morales de chacun.
		\end{description}




	\subsection{Justice interactionnelle}
		Composante de la justice procédurale, traite de la manière dont les individus sont traités. Derrière celles ci se cache un enjeu fondamentale de reconnaissance. Cette justice est souvent considérée comme une composante de la \textit{justice procédurale}.





\section{La communication}
	Principe même de toute vie organisée. Les organisations ne sont pas des machines dont les hommes serait
	les rouages. L'efficacité d'une entreprise tient plus a bon vouloir de ceux qui la compose qu'aux différentes
	règles et stratégies qu'elle établie. \\
	La communication est le fondement de l'organisation, elle représente un accomplissement quotidien.




	\subsection{Que veut dire communiquer?}
		Communiquer est une action qui visent a signifier certaines intention en vue de susciter une réaction 
		chez autrui. \textbf{Ici on ne s'intéresse qu'aux processus par lesquels un agent manifeste ses 
		intentions pour autrui et qu'autrui reconnait}. Il faut donc éviter la confusion entre comprendre ce que 
		l'autre fait et comprendre ce qu'il nous dit. Les supposition faite par la communication sont:
		\begin{itemize}
			\item La conscience de s'adresser à autrui.
			\item Un effort pour se faire comprendre.
			\item L'anticipation de certaines réactions chez l'autre. La communication part donc d'une 
			conception d'autrui.
		\end{itemize}


	\subsection{Information et incertitude}
		L'information est ce qu'un émetteur apprend à un récepteur qui l'ignorait avant et qui le sait après. La
		communication est l'interaction qui permet la transmission d'information. En communicant, on modifie 
		le monde vécu par autrui pour l'influencer. Cette modification est l'information.
		\paragraph{Le principe de réduction de l'incertitude} En informant autrui sur la relation avec lui, on 
		réduit l'indétermination qu'avait autrui par rapport à cette relation. C'est cette liaison entre information 
		et indetermination qui oriente la communication.\\
		La communication part d'une hypothèse sur l'état des connaissances de l'autres et que ce qui vaut 
		comme information pour l'un ne vaut pas nécessairement pour l'autre.

	\subsection{Les distances informationnelles}
		La distance définie ici est l'écart par rapport à la situation où l'échange se réalise de façon fluide
		sans problème de circulation de l'information. Il y a différents types de distances entre les participants 
		à la communication. 
		\paragraph{La distance géographique} La forme originelle du travail organisé suppose que tout les 
		travailleurs soit physiquement présent (co-présence). Avec les nouvelles technologies, cette supposition 
		n'est plus que partielle. Cette contrainte de proximité géographique n'a plus autant d'effet que avant.
		\paragraph{La distance sociale} C'est la distance qui sépare les agents selon leurs statuts. Les 
		différence des statuts dans une organisation renvoie au stratification de la société dans laquelle elle se 
		développe. En plus de cela, elle établie ses propres différence. Ces différences entravent la 
		communication.
		\paragraph{La distance formelle} La distance dans l'organigramme( qui peut parler à qui? qui peut 
		donner des ordres à qui?). La structure privilégie la communication vertical ascendantet et 
		descendante. Elle rend la communication latéral difficile.
		\paragraph{La distance culturelle} Les différences de langage,de conception et de valeurs entre des 	
		individus ou des groupes d'individus. Différentes raison créent cette distance (professions,milieu 
		d'origine,...). Par exemple les informaticiens ont un jargon différent des juristes ce qui peut entraver la 
		communication.
		\paragraph{La distance pratique} Elle trouve sa source dans les conditions matérielles et l'organisation
		du travail, au regard de la limitation des ressources d'attention et de traitement des agents humains. 
		Exemple:La présence de bruit ou l'impossibilité de trouver un local pour se réunir.
		\paragraph{La distance affective} La jalousie, la méfiance, la rancune,....Toute sorte de raisons qui 
		peuvent pousser des gens à ne pas échanger.\\
		Ces distances représentent des difficultés à communiquer. Il faut identifier les distances les plus 
		critiques et les réduire pour améliorer au mieux la communication au sein de l'organisation.


	\subsection{Les erreurs typiques}
		Quelques erreurs en matière de communication qui résulte de simplification abusive.
		\paragraph{La confusion du signal et de l'information} Penser que transmetter = communiquer.  La transmission n'est qu'un ensemble de signaux. Ils ne deviennent des informations qu'en fonction du processus d'interpertration effectuer par 
		le récepteur.

		\paragraph{La réduction du message au langage explicite}  Le locuteur signifie toujours bien plus que ce qu'il dit, et son allocutaire comprend bien plus que ce qui est dit. Cette erreur recouvre plusieurs conceptions discutables:
		\begin{itemize}
			\item Réduction de la signification de l'enoncé (ce que veut dire le locuteur et ce que comprend 
			l'allocutaire) sans tenir compte de la situation(intonations,attitudes,...)
			\item Négation du rôles des sous-entendu/messages implicites qui doivent être évoqué pour
		être compris.
			\item Réduction de la communication à l'utilisation d'un langage articulé (langue ou gestuelle). 
			Cela ne tient pas compte de la gestion des inférénces, c'est à dire, le processus par lesquels
			les agents influencent intenionnellement les inférences des autres. (ex: Être froid vis-a-vis de 
			quelqu'un pour lui exprimer un reproche) 
		\end{itemize} 
		On ne tient donc pas compte du décalage qu'il peut y avoir entre ce que l'on a dit et ce qui a été 
		compris.
		\paragraph{La croyance en l'efficacité propre du langage} Les gens de pouvoir qui surestiment le 
		pouvoir des mots et pense changer le monde sitôt qu'ils en changent sa représentation. Ce 
		fourvoiement conduit des dirigeants à penser qu'avec qu'en expliquant des stratégies , sans forcément 
		les mettre en place tout changera.\\
		On pense qu'en convaincant sont interlocuteurs d'agir autrement,il le fera. Cela est faux! Si on ordonne
		quelque chose à quelqu'un (ex:Travail mieux), il peut répondre oui de façon passive, sans forcément 
		avoir l'intention d'executer l'ordre. Il est plus judicieux d'installer l'autre dans les conditions qui vont le 
		pousser à decider lui même de faire l'action. 
		\paragraph{L'inflation de la communication} Cette erreur consiste à élargir le champs de 
		communication jusqu'à ce qui l'engloble la sphère de l'interprétable. La communication devient une 
		abstraction. Tout les problème deviennent résolvable par la communication.  Le champs de la recouvre 
		2 ensembles:
		\begin{itemize}
			\item L'ensemble A de tout les messages que l'agent adresse aux autres
			\item L'ensemble B des messages que l'agent interprète comme lui étant destiné. 
		\end{itemize}
		En élargissant le champs de communication, on agrandit l'ensemble B. Tout devient donc un message.
		 Il faut de penser que tout comportement ait valeur de message (par exemple un mec qui se gratte le 
		 crâne).
		 \paragraph{La réduction à la fonction informationnelle}
		 Erreur de penser que l'information n'est qu'une qu'un atome reproductible. Le language a trois fonctions fondamentale:
		\begin{enumerate}
		\item \textbf{Fonction constitutive:} créer un univers d'objets partagés, un espace public.  En transmettant un information, celle ci acquièrent une objectivité (un fait) et une resistance  (si moi j'oublie, l'autre sait).
		\item \textbf{Fonction identitaire:} \textit{en tant que X} je parle à Y \textit{en tant que Y}. Confirmation ou infirmation des définitions de soi et de l'autre ( ex: ''Donnez moi le dossier Z'' 
		 $\equiv$''Je suis le bosse vous êtes le subordonné)
		\item \textbf{Fonction emblèmatique:} signifie que en parlant, l'agent et l'interlocuteur affirme parfois
		 une appartenance collective. Un élément qui permet de distinguer un groupe d'un autre peut devenir
		 un emblème. Le langage permet de des possibilités de réunion et de 
		 différenciation(langue,accent,vocabulaire,...).
		\end{enumerate}
On voit donc que le langage n'a pas seul vocation à transmettre de l'information.

	\subsection{Les enjeux de la communication}
		Qu'est ce que l'on cherche en parlant?
		\paragraph{Les  enjeux phatiques} Établir, garder le contact et préserver la relation.
		\paragraph{Les enjeux pratiques} Susciter plus ou moins directement des actions des l'autre dans
		la poursuite de la coordination sociale.
		\paragraph{Les enjeux de reconnaissance}
		La reconnaissance un enjeu capital, 5 sphère de de reconnaissance auxquelles correspondent trois
		type de rapport à soi:
		\begin{itemize}
			\item La sphère de l'amour, l'expérience d'être aimé et la sécurité émotionnelle. Apporte de la 
			confiance en soi. (sphère du travail non imperméable à de tels liens.
			\item La sphére de la reconnaissance juridique, sphère d'être investi de de droits et de devoirs. 
			Apporte le respect de soi
			\item La sphère de l'estime sociale, la reconnaissance par autrui des qualités du sujet. Apporte
			l'estime de soi.
			\item La reconnaissance du statut social de la personne, en tant que roi,chef, expert,....
			\item La reconnaissance en temps que membre du groupe.
		\end{itemize}
		Il faut être reconnu par les autres pour se reconnaître soi-même.
		\paragraph{Les enjeux de légitimation} Légitimer des contraintes sociale,(ex: le père qui dit à son fils
		qu'il doit dormir tôt pour son bien)
		\paragraph{Les enjeux ontologiques} Préserver notre sens de la réalité, conforter nos conceptions des choses et confirmer nos convictions sur l'ordre du monde.


	\subsection{Le sens dépend du contexte}

	Lorsque l'on parle ce que l'on veut signifier dépend du contexte, c'est à dire de tout ce qui est en dehors du message lui-même, de ce qui est explicitement manifesté.

	La perception est toujours perception immédiate d'un tout organisé différent de l'addition des parties, on ne distingue les parties qu'en fonction du tout.

	\begin{description}
		\item[Situation de confusion] Situation où l'on ne parvient pas directement à reconnaître ce contexte donnateur de sens.
	\end{description}

	\paragraph{Cadrage et recadrage} Parfois l'émetteur va transmettre des représentations du contexte afin de guider le récepteur dans son travail.
	\begin{itemize}
		\item \textbf{Cadrage:} toute communication visant à éclairer le récepteur sur le contexte du message (ou de l'action) que celui-ci chercher à comprendre
		\item \textbf{Recadrage:} toute communcation visant à transformer  antérieurement "appliqué" au message (à l'action) par le récepteur. 
	\end{itemize}
	
	Nous passons beaucoup de temps à transmettre aux autres des \textbf{représentations contextuelles}

	\subsection{La ponctuation de la séquence des faits}

		Nous faisons toujours commencer à un certains moment l'interpretation d'un message, c'est ce qu'on appelle la ponctuation de la séquence des faits. Un désaccord sur cette dernière est base de nombreux conflits portant sur la relation (ex: c'est toi qui a commencé, non c'est toi!)

	\subsection{Interaction symétrique et complémentaire}

	Il y a deux grands types d'interactions
	\paragraph{Interaction complémentaire} le comportement de l'un des partenaires complète le comportement de l'autre et réciproquement (ex: autorité du chef = obéissance du subordonné).

	\paragraph{Interaction symétrique} les partenaires reproduisent leur ressemblance et confirment l'un à l'autre qu'ils sont semblables.

	Ici deux conséquences possibles
	\begin{enumerate}
		\item Feedback positif: chacun réagit en faisant "plus de la même chose", pour affirmer une différence à sont avantage
		\item Escalade complémentaire: maximisation de la différence entre partenaire (ex: tu me gueulle dessus, je ne fais rien donc tu me gueulle plus encore et je travaille moins encore).
	\end{enumerate}

	\subsection{Les deux niveaux de la communication}

	Il faut distinguer le \textit{contenu} et la \textit{relation}. 

	\begin{description}
		\item[Contenu] tout ce qui est évoqué dans l'échange verbal, les objets et les événements auxquels on fait référence dans le discours.
		\item[Relation] Tout ce qui est exprimé en dehors de l'objet représenté et qui contribue à définir la relation entre eux (ex: jugement, approbation, ...).
	\end{description}

	Il y a \textbf{toujours} un niveau de relation.

	\subsection{Le rôle du corps}

	Dans la communication corporelle, il faut distinguer le rôle du corps en corps comme \textit{message} (usage du corps dans la construction du message) et corps comme \textit{contexte} (attitude involontaire qui donne un contexte (ex: je tremble en parlant $\rightarrow$ j'ai peur)).


	\subsection{La communication inférentielle}

	Il y a deux manières de signifier quelque chose à quelqu'un.
	\begin{enumerate}
		\item Au moyen du language
		\item En amenant l'autre à inférer lui-même ce que nous voulons lui signifier.
	\end{enumerate}

	Cela dit, en se répétant des messages inférentiels entre dans la sphère du language (ex: un symbole utilisé plusieurs fois deviens un message évident).

	Les problèmes de communication inférentielles sont souvents:
	\begin{enumerate}
		\item Le récepteur reste sans remarquer l'intention de communication,
		\item Le récepteur se figure qu'il n'y a pas d'intention de communication,
		\item Le récepteur infère un message érroné.
	\end{enumerate}

	\subsection{Les paradoxes dans la communication}

	Une communication paradoxale est une communication qui contient en elle-même sa propre contradiction.

	\paragraph{Paradoxes internes aux messages} La contradiction réside dans la nature même du message, soit au niveau du contenu, soit entre le contenu et la relation.

	\paragraph{Paradoxes propres à la situation de communication} La contradiction oppose le message et la réaction attendue à ce message (je ne peux pas faire ce qu'on me demande, sinon je désobéi à ce qu'on me demande, \textit{injonction paradoxale}).

	\begin{description}
		\item[Injonction paradoxale] L'émetteur demande ou exige quelque chose que, par ce fait même, il rend impossible, plaçant l'autre dans une situation intenable puisqu'il est condamné à désobéir en obeissant.
 	\end{description}

	\paragraph{Métacommunication} La communication sur la communication. Métacommuniquer c'est prendre la relation pour contenu du message: verbaliser la relation vécue (ex: demander une explication sur ce qui viens d'être dit) et reconnaitre le vécu exprimé par l'autre (ex: expliciter ce que l'autre dit).


	\subsection{Conclusion: la rationalité dialogique}

	

\section{Le pouvoir}
\section{L'autorité et l'ordre de l'interaction}
	\subsection{L'autorité et les autres modes d'interaction}
	\subsection{L'autorité dans ses rapports aux modes d'interaction}
	\subsection{L'autorité comme légitimité}
	\subsection{Une forme d'autorité typique de notre modernité: le leadership}
	

	
	



\end{document}

