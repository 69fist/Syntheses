
\documentclass[11pt]{article} % use larger type; default would be 10pt

\usepackage[utf8]{inputenc} % set input encoding (not needed with XeLaTeX)

\usepackage{geometry} % to change the page dimensions
\geometry{a4paper} % or letterpaper (US) or a5paper or....
% \geometry{margin=2in} % for example, change the margins to 2 inches all round
% \geometry{landscape} % set up the page for landscape
%   read geometry.pdf for detailed page layout information

\usepackage{graphicx} % support the \includegraphics command and options
\usepackage{multicol}
% \usepackage[parfill]{parskip} % Activate to begin paragraphs with an empty line rather than an indent

%%% PACKAGES
\usepackage{booktabs} % for much better looking tables
\usepackage{array} % for better arrays (eg matrices) in maths
\usepackage{paralist} % very flexible & customisable lists (eg. enumerate/itemize, etc.)
\usepackage{verbatim} % adds environment for commenting out blocks of text & for better verbatim
\usepackage{subfig} % make it possible to include more than one captioned figure/table in a single float
% These packages are all incorporated in the memoir class to one degree or another...

%%% HEADERS & FOOTERS
\usepackage{fancyhdr} % This should be set AFTER setting up the page geometry
\pagestyle{fancy} % options: empty , plain , fancy
\renewcommand{\headrulewidth}{0pt} % customise the layout...
\lhead{}\chead{}\rhead{}
\lfoot{}\cfoot{\thepage}\rfoot{}

%%% SECTION TITLE APPEARANCE
\usepackage{sectsty}
\allsectionsfont{\sffamily\mdseries\upshape} % (See the fntguide.pdf for font help)
% (This matches ConTeXt defaults)

%%% ToC (table of contents) APPEARANCE
\usepackage[nottoc,notlof,notlot]{tocbibind} % Put the bibliography in the ToC
\usepackage[titles,subfigure]{tocloft} % Alter the style of the Table of Contents
\renewcommand{\cftsecfont}{\rmfamily\mdseries\upshape}
\renewcommand{\cftsecpagefont}{\rmfamily\mdseries\upshape} % No bold!

%%% END Article customizations

%%% The "real" document content comes below...

\title{Gestion des Ressources Humaines}
\author{Houtain Nicolas \and Kabasele Ndonda Gorby Nicolas}

\begin{document}
\maketitle

\section{Penser la réalité humaine de l'organisation}
	\subsection{La réitification}
		Illusion qui consiste à concevoir l'organisation comme une réalité supérieur.
		L'organisation comme un organisme à part qui serait en relation avec ses membres.
		Seul contraintes sont celles qu'exercent les hommes (agent sociaux) entre eux.
	\subsection{Dépenser les raisonnements mécaniques}
		On pense que les agents sont des êtres passifs soumis au impulsion de l'environnement (ex: culture de
		l'entreprise)
	\subsection{Articuler l'acteur et le système}
		Centration excessive sur le système $\Leftrightarrow$ agent produit passif des logiques structurelles 
		économiques et technologiques de l'entreprise. L'inverse est toute aussi vrai, exaltation démésurée de 
		l'acteur (ex: Dirigeant vu comme quelqu'un d'omniscient)
	\subsection{Raisonner en termes de dualité}
		Deux visions pour l'organisation:
		\begin{itemize}
			\item Les actions et les intéractions qui s'y passent.
			\item Les éléments constitutif de configuration plus large.
		\end{itemize}
		Ces deux points de vue sont à adopter pour analyser l'oragnisation (Va-vient entre le singulier et le 		
		globale).
		\paragraph{Systeme} Conception pour expliquer des relations de contraintes et d'interdépendance. 
		Tout systeme est structuré et peut être répété.
		\paragraph{Structure} Propriétés générales qui font qu'un système sociale se reproduit de période en 
		période.
		Les actions sont conditionnés par le système mais le système résulte des actions des hommes.
		\paragraph{dualité versus dualisme} le dualité est une réalité double mais chaque réalité est unique en 	
		soi. Le dualisme est la coexistence de réalité différentes. La dualité de l'organisation vient des actions
		et interactions d'une part et du système globale d'autre part.
	\subsection{Critique de l'atomisme}
		Atome de la vie sociale et économique est l'individu psychologique (pense qu'à moi le reste, 
		m'enfous). 
		Organisation devient alors une agrégation d'atome. Conception individualiste et rationnelle de l'être 
		humain. Différents type de conception proche sans être les même.
		\begin{description}
			\item [Conception monadique:] Humain comme individu isolé de son milieu, son comportement 
			vient de l'intérieur de lui (pulsation,personnalités,...). Problème:négation de l'essence des hommes
			dans leurs rapports à autrui.
			\item[Utilitarisme:] Les humains peuvent se réduire à la poursuite d'intêrét égoïste (maximiser son 
			utilité). Problème: version très cynique, logique de l'intérêt pour tout.(amitié,amour,...)
			\item Individu est un décideur rationnel (il fait des choix pour atteindre ces objectifs). Problème:
			surestime les capacités de transmission et de traitement de l'information de l'homme..
		\end{description}
	\subsection{Perspective interactioniste}
		On détérmine les actions humaines en les rapportant aux relations sociales. TODO exemple de 
		l'enqueteur.
	\subsection{Passage par la subjectivité de l'acteur}
		Le passage par l'interaction n'empêche pas de nier ce que les gens vivent. Pour comprendre leurs 
		conduites, il faut partir de leur subjectivité. Ne pas prioritiser la vision de l'observateur de la situation
		mais comment l'acteur la vie.
		\paragraph{Théorme de Thomas} Si les hommes tiennent un situation pour réelle, même si ce n'est
		pas le cas, les conséquences le seront.  (ex: des ramoneur)
		L'empathie est une compétence sociale décisive dans la vie du travail.
	\subsection{Sous l'ordre hierarchiques les réseaux}
		Pour comprendre l'organisation, il faut s'intéresser au relation non-hiérarchique, les réseaux.
		\paragraph{réseaux} Ensemble non-hierarchisé et largement informel de relation entre acteurs. Ces 
		liens sont évolutif. Différent des coalition car les agents ont des objectifs différents.
	\subsection{Au fondement du social, la logique de don}
		Ce qui structure les réseaux ce n'est pas du donnant-donnant mais plutôt la logique du don. Dans le
		don, la norme de réprocité reste implicite. Il présuppose la confiance, qui se crée au fur à mesure. La 
		plus grande de confiance est la divulgation d'un secret.
	\subsection{La réalité est socialement construite}
		Situation socialement construite au travers d'une variété de processus de communication. 
		\begin{itemize}
			\item \textbf{Division du travail et cognition} L'attention sélective tient sa source dans la limitation
			cognitif. La vie sociale détermine les foyers d'attention donc on divise le travail.
			\item \textbf{Rôle des représentations} L'environnement est abstrait (on ne sait pas tout). 
			Représentation comme un acte de  communication (on imagine ce qu'on nous raconte). La réalité 
			est donc socialement construite.
			\begin{description}
				\item[Validation empirique] Confronté la représentation à la réalités des faits (ex:météo)
				\item[Validation sociale] Représentation qu'on ne peut pas confronté à la réalité 		
				(ex:statistique).
			\end{description}
		\end{itemize}
		Certains dirigeant sont aveuglement socialement construit (s'entoure de gens qu'ils pensent qu'ils sont 	
		géniaux).
		\paragraph{Culture de l'entreprise} Ensemble des processus de construction sociale de la réalité 
		partagée d'une organisation.
		\textbf{Le grand problème du managemement  c'est de faire croire que l'ont décrit la réalité alors 
		qu'on l'a préscrit}
\section{La gestion des ressource humaines et les fondements de la motivation sociale}
	\subsection{Théorie du contenu versus théorie du processus}
		\paragraph{Motivation} Engagement de la personne dans son travail qui, si le contexte le permet, va
		entrainer un surcroit de performance.
		Deux façon de théoriser la motivation:
		\begin{description}
			\item[Théorème contenu: ] Qu'est ce que la motivation?
			\item[Théorème du processus: ] Quelles sont les facteurs qui influence la motivation?
		\end{description}
	\subsection{Motivation réduite à une phénomène purement individuel}
		Erreur de penser que la motivation est purement individuel (ex:Fight team spirit). En fonction du
		rôle qui est attribué à un acteur, le rapport au monde de l'acteur est modifié (motivation rôle).
	\subsection{Motivation comme trait de personnalité ou trait culturel}
		C'est une erreur de penser que la motivation nait d'un trait de la personnalité et le besoin de la 
		personnalité. C'est une erreur car les individus ne sont pas des personnalités figées, elles sont 	
		influencés par l'entreprise.
	\subsection{Motivation et satisfaction}
		Aucune relation n'a été clairement émise ente la satisfaction et la motivation. On ne peut pas conclure 
		que la satisfaction est une cause de motivation. Ce sont des effets très différents. L'un est 
		l'engagement et l'autre est un jugement par rapport à une situation.
	\subsection{Confusion entre motivation et l'implication}
		\paragraph{implication} Attachement au travail, identification à la fonction. Fait état de l'importance
		que prend le travail par rapport aux autres choses de la vie. \textbf{Implication pas forcément nécessaire à la motivation}
	\subsection{Déni du principe de nécéssité}
		La performance en amont de la motivation est le résultat d'un mélange d'intérêt.
	\subsection{Lien méchanique motivation = performance}
		La motivation n'influe pas forcément la performance individuelle et collective. Toute motivation 
		s'actualise dans un lilieu social qui peut aussi bien la relayer, l'amplifier ou au contraire l'étouffer. 
		Performance est un concept qui n'est pas trés claire.
	\subsection{Le modèle V.I.E}
		3 types de facteurs suceptibles d'influencer la propension à s'engager dans certains types d'action:
		\begin{description}
			\item[Expectation: ] La représentation des capacités à s'engager. Savoir s'il est capable d'atteindre 
			l'objectif primaire. 
			\item[Instrumentalité: ] La représentation du rappor entre l'action et le résultat. Lien raisonnable 
			à sa performance et certains résultats potentiels associéq à cette performance (typiquement la 
			réaction de la hiérarchie).
			\item[Valence: ] Valeur accordé au résultat. Elle peut être positif(ex: augmentation de salaire) ou
			négatif(ex:licenciement)
		\end{description}
		\paragraph{Expecation} L'expectation peut être diviser en deux catégorie, l'\textbf{expectation de soi}
		(1,2,3) et l'\textbf{expection matérielle}(4,5,6):
		\begin{multicols}{2}
			\begin{enumerate}
				\item Ressources affectives (ex:confiance en soi)
				\item Ressources cognitives (ex: langue parlé)
				\item Ressources physique
				\item Ressources financière
				\item Ressources matérielles et technique
				\item Ressource stratégiques
			\end{enumerate}
		\end{multicols}
		Il est important pour la motivation de fixer des objectifs (effet de but). Mais ce n'est pas forcément en 
		fixant des objectifs que les gens seront plus motivés. Les objectifs doivent suivre le modèle SMART:
		 Spécifique, Mesurable, Ambitieux, Réaliste, délimiter dans le Temps. Ils doivent aussi:
		 \begin{multicols}{2}
		 	\begin{itemize}
		 		\item Sous le contôle de l'emplyé et aisément compréhensible.
		 		\item Négociés et justifié
		 		\item Non contradictoire et prioritiser.
		 		\item Il faut qu'il y ait un dispositif du feedback.
		 		\item Contrôle d'atteinte des objectifs.
		 		\item Valoriser les objectif
		 	\end{itemize}
		 \end{multicols}
		 La GRH peut améliorer l'expectation par diffétents moyens comme le coaching et la formation.
		 
		 \paragraph{Instrumentalité} La probablilité d'obtenir/d'éviter tel résultat en adoptant un
		 comportement donné. La perception du résultat peut être objective ou subjective (dépend de chaque 
		 individu et de sa perception). Cette perception dépend des caractéristiques objectives du milieu de 
		 travail.
		 L'instrumenlatité pose 4 types de problèmes (Á EXPLIQUER):
		 \begin{itemize}
		 	\item Comment la performance est reconnue est définie a tout les niveaux.
		 	\item Comment cette performance est évaluée et sanctionnée.
		 	\item Chaque employé doit avoir accès à l'information.
		 	\item Dépend de la confiance en la per
		 \end{itemize}
\section{Justice et équité dans la GRH}
	\subsection{Justice distributive}
	\subsection{Justice procédurale}
	\subsection{Justice interactionnelle}
	\subsection{Déni du principe de nécéssité}
	\subsection{Lien méchanique motivation = performance}
	\subsection{Le modèle V.I.E}
\section{Justice et équité dans la GRH}
	\subsection{Justice distributive}
	\subsection{Justice procédurale}
	\subsection{Justice interactionnelle}

\end{document}
