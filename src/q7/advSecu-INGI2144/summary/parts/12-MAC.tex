
\subsection{Basics}
The goal of MAC is to prevents tampering with a message. MAC require a secret
key otherwise modification of the message cannot be done. 
\begin{itemize}
    \item[$\Rightarrow$] MAC ensures integrity, authentication but not confidentiality.
\end{itemize}

\subsection{Constructions}
Let $h$ be a hash function, $k$ a key, $m$ the message to be MACed, and $M$ the
computed MAC.
\begin{itemize}
    \item Prefix method: $ M = h(k||m) $
    \item Suffix method: $ M = h(m||k) $
    \item Envelop method:$ M = h(k||p||m||k) $
\end{itemize}

\subsubsection{Hash MAC scheme}
MAC uses hash function rather than block cipher because hash function are
generally faster.

\begin{center}
    \begin{tikzpicture}
        \node [rectangle, red, draw] (H) {h};

        \node [left=1.4cm of H, draw, rectangle] (HMAC) {HMAC(k, m)};

        \node [above=of H, C, o-] (C1) {$||$};
        \node [above right=0.0cm and 0.1cm of C1] (X1) {$\oplus$};
        \node [draw, rectangle,above right=1.4cm and 0.1cm of C1] (k) {$k$};
        \node [draw, rectangle,above left=1.4cm and 0.1cm of C1] (m) {$m$};
        \node [right=0.4cm of X1] (ipad) {\textsc{ipad}};

        \node [right=0.6cm of H, C, o-] (C2) {$||$};
        \node [above right=0.5cm and 1cm of C2] (X2) {$\oplus$};
        \node [left=0.4cm of X2] (opad) {\textsc{opad}};

        \path[->] (H) edge (HMAC)
        (C1) edge (H) 
        (C2) edge (H) 
        (ipad) edge (X1)
        (opad) edge (X2);

        \draw[->] (X2) |- (C2.20);
        \draw[->] (X1) |- (C1);
        \draw[->] (m) |- (C1);
        \draw[] (k.280) edge[->] node (tmp) {} (X1);
        \draw[->] (tmp.center) -| (X2);

        \draw[->] (H) |- (+0.5cm, -0.5cm)  -| (+1.5cm, -0.5cm) |-
        (C2.340);

        \node [below left= 0.4cm and -0.2cm of m] (A) {};
        \node [below = 0.1cm of H] (B) {};
        \node [right = -0.2cm of X2] (C) {};

        \node [draw, double, rectangle, fit={(A) (B) (C) }] (FF) {};
    \end{tikzpicture}
\end{center}

%TODO example 374

\subsubsection{Built upon CBC block-cipher}

It's based on the CBC mode of operation: the \textbf{MAC} is the final
block, intermediary blocks are thrown out and IV consists of zeros.

\begin{itemize}
    \item \textbf{CBC-MAC}

        \begin{tikzpicture}
            \newcommand{\n}{3}
            \foreach \nr in {1, ..., \n}{
                \node (C\nr)            at ({(\nr-\n)*2},0) {};
                \node (D\nr)[encrypt]   at ({(\nr-\n)*2},1.5) {$E$};
                \node (x\nr)       at ({(\nr-\n)*2},2.5) {$\oplus$};
                \node (M\nr)            at ({(\nr-\n)*2},3.5) {$M_\nr$};

                \node (K\nr)            at ({(\nr-\n)*2-1},1.5) {$K$};

                \draw[->,very thick] (x\nr) -- (D\nr);
                \draw[->,very thick] (M\nr) -- (x\nr);

                \draw[->,very thick] (K\nr) -- (D\nr);
            }

            \node (C3)            at ({(3-\n)*2},0) {MAC};

            \foreach \nr in {2, ..., \n}{
                \newcommand{\tmp}{\n-\nr}
                \pgfmathtruncatemacro{\tmp}{\nr-1}
                \draw[->,very thick] (D\tmp) -- ({(\n-\tmp)*-2},0.75) -|
                ({(\n-\tmp)*-2+0.75},0.75) |- ({(\n-\tmp)*-2+0.75},2) |- (x\nr);
            }

            \draw[->,very thick] (D\n) -- (C\n);
            \node (IV) at ({\n*-2+1},2.5) {$0$};
            \draw[->, very thick] (IV) -- (x1);
        \end{tikzpicture}

        \paragraph{Characteristic}
        \begin{itemize}
            \item The IV should not be random
            \item Slower than HMAC
            \item The MAC can be truncated to enforce security but this also reduces the
                key search space
        \end{itemize}

        \paragraph{Security}
        \begin{itemize}
            \item CBC-MAC is secure for messages of a fixed number of blocks assuming
                the block cipher is secure.
            \item Not secure with variable lengths
            \item CBC-MAC $(length(M)||M)$ to avoid truncation attacks
        \end{itemize}

    \item \textbf{Retail-MAC}: the last block is decrypted with key $k'$
        then re-encrypted with $k$. This reduces the threat of exhaustive key search

        \begin{tikzpicture}
            \newcommand{\n}{3}
            \foreach \nr in {1, ..., \n}{
                \node (C\nr)            at ({(\nr-\n)*2},0) {};
                \node (x\nr)       at ({(\nr-\n)*2},2.5) {$\oplus$};
                \node (M\nr)            at ({(\nr-\n)*2},3.5) {$M_\nr$};


                \draw[->,very thick] (x\nr) -- (D\nr);
                \draw[->,very thick] (M\nr) -- (x\nr);

                \draw[->,very thick] (K\nr) -- (D\nr);
            }

            \foreach \nr in {1, ..., 2}{
                \node (K\nr)            at ({(\nr-\n)*2-1},1.5) {$K$};
                \node (D\nr)[encrypt]   at ({(\nr-\n)*2},1.5) {$E$};
            }


                \node (K3)            at ({(3-\n)*2-1},1.5) {$K'$};
            \node (D3)[encrypt]   at ({(3-\n)*2},1.5) {$D$};
            \node (DB3)[encrypt]   at ({(3-\n)*2},0.5) {$E$};
            \node (KB3)   at ({(3-\n)*2+1},0.5) {$K$};

            \node (C3)            at ({(3-\n)*2},-0.5) {MAC};

            \foreach \nr in {2, ..., \n}{
                \newcommand{\tmp}{\n-\nr}
                \pgfmathtruncatemacro{\tmp}{\nr-1}
                \draw[->,very thick] (D\tmp) -- ({(\n-\tmp)*-2},0.75) -|
                ({(\n-\tmp)*-2+0.75},0.75) |- ({(\n-\tmp)*-2+0.75},2) |- (x\nr);
            }

            \draw[->,very thick] (DB\n) -- (C\n);
            \draw[->,very thick] (D\n) -- (DB\n);
            \draw[->,very thick] (KB\n) -- (DB\n);
            \node (IV) at ({\n*-2+1},2.5) {$0$};
            \draw[->, very thick] (IV) -- (x1);
        \end{tikzpicture}

        \begin{itemize}
            \item Encrypt $length(M)$ with K, yielding K' and use it as key for the
                MAC function
        \end{itemize}

    \item \textbf{CMAC}
        %TODO finish CMAC

        \begin{tabular}{cc}
            (a) Multiple of block size & (b) Not multiple of block size
            \\
        \begin{tikzpicture}
            \newcommand{\n}{3}
            \foreach \nr in {1, ..., \n}{
                \node (C\nr)            at ({(\nr-\n)*2},0) {};
                \node (x\nr)       at ({(\nr-\n)*2},2.5) {$\oplus$};
                \node (M\nr)            at ({(\nr-\n)*2},3.5) {$M_\nr$};

                \node (K\nr)            at ({(\nr-\n)*2-1},1.5) {$K$};
                \node (D\nr)[encrypt]   at ({(\nr-\n)*2},1.5) {$E$};

                \draw[->,very thick] (x\nr) -- (D\nr);
                \draw[->,very thick] (M\nr) -- (x\nr);

                \draw[->,very thick] (K\nr) -- (D\nr);
                
            }


            \node (C3)            at ({(3-\n)*2},0) {MAC};
            \node (A) at ({(3-\n)*2+1},2.5) {$K_1$};


            \foreach \nr in {2, ..., \n}{
                \newcommand{\tmp}{\n-\nr}
                \pgfmathtruncatemacro{\tmp}{\nr-1}
                \draw[->,very thick] (D\tmp) -- ({(\n-\tmp)*-2},0.75) -|
                ({(\n-\tmp)*-2+0.75},0.75) |- ({(\n-\tmp)*-2+0.75},2) |- (x\nr);
            }

            \draw[->,very thick] (D\n) -- (C\n);
            \draw[->,very thick] (A) -- (x\n);
            \node (IV) at ({\n*-2+1},2.5) {$0$};
            \draw[->, very thick] (IV) -- (x1);
        \end{tikzpicture}
        &
        \begin{tikzpicture}
            \newcommand{\n}{3}
            \foreach \nr in {1, ..., \n}{
                \node (C\nr)            at ({(\nr-\n)*2},0) {};
                \node (x\nr)       at ({(\nr-\n)*2},2.5) {$\oplus$};
                \node (M\nr)            at ({(\nr-\n)*2},3.5) {$M_\nr$};

                \node (K\nr)            at ({(\nr-\n)*2-1},1.5) {$K$};
                \node (D\nr)[encrypt]   at ({(\nr-\n)*2},1.5) {$E$};

                \draw[->,very thick] (x\nr) -- (D\nr);
                \draw[->,very thick] (M\nr) -- (x\nr);

                \draw[->,very thick] (K\nr) -- (D\nr);
                
            }


            \node (C3)            at ({(3-\n)*2},0) {MAC};
            \node (A) at ({(3-\n)*2+1},2.5) {$K_2$};


            \foreach \nr in {2, ..., \n}{
                \newcommand{\tmp}{\n-\nr}
                \pgfmathtruncatemacro{\tmp}{\nr-1}
                \draw[->,very thick] (D\tmp) -- ({(\n-\tmp)*-2},0.75) -|
                ({(\n-\tmp)*-2+0.75},0.75) |- ({(\n-\tmp)*-2+0.75},2) |- (x\nr);
            }

            \draw[->,very thick] (D\n) -- (C\n);
            \draw[->,very thick] (A) -- (x\n);
            \node (IV) at ({\n*-2+1},2.5) {$0$};
            \draw[->, very thick] (IV) -- (x1);
        \end{tikzpicture}
        \end{tabular}

        \paragraph{Generate subkeys}
        %TODO slide 379

\end{itemize}


\subsection{Encrypt and MAC}
Encryption does not ensure integrity but MAC does, so they must be combined.
From the \textbf{LESS} to the \textbf{STRONGER} security:
\begin{itemize}
    \item $ E_K(M) $
    \item Redundancy-then-Encrypt: $ E_K(M,R(M)) $
    \item Hash-then-Encrypt: $ E_K(M,h(M)) $
    \item Hash and Encrypt: $ E_K(M),h(M) $
    \item MAC and Encrypt: $ E_{h1(K)},MAC_{h2(K)}(M) $ (SSH)
    \item MAC-then-Encrypt: $ E_{h1(K)}(M,MAC_{h2(K)}(M)) $ (SSL)
    \item Encrypt-then-MAC\@: $ E_{h1(K)}(M),MAC_{h2(K)}(E_{h1(k)}(M)) $ (IPSec)
\end{itemize}
\paragraph{Note:} Do not hash concatenation of key and message to get a MAC\@.
Never use the same key for both encryption and MAC\@.
