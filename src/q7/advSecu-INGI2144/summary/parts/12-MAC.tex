
\subsection{Basics}
The goal of MAC is to prevents tampering with a message. MAC require a secret
key otherwise modification of the message cannot be done. 
\begin{itemize}
    \item[$\Rightarrow$] MAC ensures integrity, authentication but not confidentiality.
\end{itemize}

\subsection{Constructions (Naïve)}
Let $h$ be a hash function, $k$ a key, $m$ the message to be MACed, and $M$ the
computed MAC.
\begin{itemize}
    \item Prefix method: $ M = h(k||m) $
    \item Suffix method: $ M = h(m||k) $
    \item Envelop (weak) method:$ M = h(k||p||m||k) $
\end{itemize}

\subsubsection{Hash MAC scheme}
MAC uses hash function rather than block cipher because hash function are
generally faster.

\begin{tabular}{m{8cm}m{5cm}}
    \centering
    \begin{tikzpicture}
        \node [rectangle, red, draw] (H) {h};

        \node [left=1.4cm of H, draw, rectangle] (HMAC) {HMAC(k, m)};

        \node [above=of H, C, o-] (C1) {$||$};
        \node [above right=0.0cm and 0.1cm of C1] (X1) {$\oplus$};
        \node [draw, rectangle,above right=1.4cm and 0.1cm of C1] (k) {$k$};
        \node [draw, rectangle,above left=1.4cm and 0.1cm of C1] (m) {$m$};
        \node [right=0.4cm of X1] (ipad) {\textsc{ipad}};

        \node [right=0.6cm of H, C, o-] (C2) {$||$};
        \node [above right=0.5cm and 1cm of C2] (X2) {$\oplus$};
        \node [left=0.4cm of X2] (opad) {\textsc{opad}};

        \path[->] (H) edge (HMAC)
        (C1) edge (H) 
        (C2) edge (H) 
        (ipad) edge (X1)
        (opad) edge (X2);

        \draw[->] (X2) |- (C2.20);
        \draw[->] (X1) |- (C1);
        \draw[->] (m) |- (C1);
        \draw[] (k.280) edge[->] node (tmp) {} (X1);
        \draw[->] (tmp.center) -| (X2);

        \draw[->] (H) |- (+0.5cm, -0.5cm)  -| (+1.5cm, -0.5cm) |-
        (C2.340);

        \node [below left= 0.4cm and -0.2cm of m] (A) {};
        \node [below = 0.1cm of H] (B) {};
        \node [right = -0.2cm of X2] (C) {};

        \node [draw, double, rectangle, fit={(A) (B) (C) }] (FF) {};
    \end{tikzpicture}
    &
$$HMAC(k,m) = h\bigg(opad \oplus k \quad || \quad h\big(m||k\oplus ipad\big)\bigg)$$
\end{tabular}


%TODO example 374

\subsubsection{Built upon CBC block-cipher}

It's based on the CBC mode of operation: the \textbf{MAC} is the final
block, intermediary blocks are thrown out and IV consists of zeros.

\begin{itemize}
    \item \textbf{CBC-MAC}: slower than HMAC and \textbf{secure} for
        messages of a fixed number of blocks assuming the block cipher is secure.

        \begin{tabular}{m{6cm}m{10cm}}
        \begin{tikzpicture}
            \newcommand{\n}{3}
            \foreach \nr in {1, ..., \n}{
                \node (C\nr)            at ({(\nr-\n)*2},0) {};
                \node (D\nr)[encrypt]   at ({(\nr-\n)*2},1.5) {$E$};
                \node (x\nr)       at ({(\nr-\n)*2},2.5) {$\oplus$};
                \node (M\nr)            at ({(\nr-\n)*2},3.5) {$M_\nr$};

                \node (K\nr)            at ({(\nr-\n)*2-1},1.5) {$K$};

                \draw[->,very thick] (x\nr) -- (D\nr);
                \draw[->,very thick] (M\nr) -- (x\nr);

                \draw[->,very thick] (K\nr) -- (D\nr);
            }

            \node (C3)            at ({(3-\n)*2},0) {MAC};

            \foreach \nr in {2, ..., \n}{
                \newcommand{\tmp}{\n-\nr}
                \pgfmathtruncatemacro{\tmp}{\nr-1}
                \draw[->,very thick] (D\tmp) -- ({(\n-\tmp)*-2},0.75) -|
                ({(\n-\tmp)*-2+0.75},0.75) |- ({(\n-\tmp)*-2+0.75},2) |- (x\nr);
            }

            \draw[->,very thick] (D\n) -- (C\n);
            \node (IV) at ({\n*-2+1},2.5) {$0$};
            \draw[->, very thick] (IV) -- (x1);
        \end{tikzpicture}
        &
        \paragraph{Characteristic}
        \begin{itemize}
            \item The IV should not be random
            \item The MAC can be truncated to enforce security but this also reduces the
                key search space
        \end{itemize}

        \paragraph{Security}
        \begin{itemize}
            \item Not secure with variable lengths (only length with
                fixed number of blocks.)
            \item CBC-MAC $(length(M)||M)$ to avoid truncation attacks
        \end{itemize}
        \end{tabular}


    \item \textbf{Retail-MAC}: 

        \begin{tabular}{m{6cm}m{10cm}}
        \begin{tikzpicture}
            \newcommand{\n}{3}
            \foreach \nr in {1, ..., \n}{
                \node (C\nr)            at ({(\nr-\n)*2},0) {};
                \node (x\nr)       at ({(\nr-\n)*2},2.5) {$\oplus$};
                \node (M\nr)            at ({(\nr-\n)*2},3.5) {$M_\nr$};


                \draw[->,very thick] (x\nr) -- (D\nr);
                \draw[->,very thick] (M\nr) -- (x\nr);

                \draw[->,very thick] (K\nr) -- (D\nr);
            }

            \foreach \nr in {1, ..., 2}{
                \node (K\nr)            at ({(\nr-\n)*2-1},1.5) {$K$};
                \node (D\nr)[encrypt]   at ({(\nr-\n)*2},1.5) {$E$};
            }


                \node (K3)            at ({(3-\n)*2-1},1.5) {$K'$};
            \node (D3)[encrypt]   at ({(3-\n)*2},1.5) {$D$};
            \node (DB3)[encrypt]   at ({(3-\n)*2},0.5) {$E$};
            \node (KB3)   at ({(3-\n)*2+1},0.5) {$K$};

            \node (C3)            at ({(3-\n)*2},-0.5) {MAC};

            \foreach \nr in {2, ..., \n}{
                \newcommand{\tmp}{\n-\nr}
                \pgfmathtruncatemacro{\tmp}{\nr-1}
                \draw[->,very thick] (D\tmp) -- ({(\n-\tmp)*-2},0.75) -|
                ({(\n-\tmp)*-2+0.75},0.75) |- ({(\n-\tmp)*-2+0.75},2) |- (x\nr);
            }

            \draw[->,very thick] (DB\n) -- (C\n);
            \draw[->,very thick] (D\n) -- (DB\n);
            \draw[->,very thick] (KB\n) -- (DB\n);
            \node (IV) at ({\n*-2+1},2.5) {$0$};
            \draw[->, very thick] (IV) -- (x1);
        \end{tikzpicture}
        &
        \begin{enumerate}
        \item the last block is decrypted with key $k'$
        \item then re-encrypted with $k$. 
            \item[$\Rightarrow$]This reduces the threat of exhaustive key search
                \end{enumerate}
                $$$$
        \begin{itemize}
            \item Encrypt $length(M)$ with K, yielding K' and use it as key for the
                MAC function
        \end{itemize}
        \end{tabular}

    \item \textbf{CMAC} Generate $K_1$ and $K_2$ from $K$ 

        \begin{tabular}{cc}
            (a) Multiple of block size & (b) Not multiple of block size
            \\
        \begin{tikzpicture}
            \newcommand{\n}{3}
            \foreach \nr in {1, ..., \n}{
                \node (C\nr)            at ({(\nr-\n)*2},0) {};
                \node (x\nr)       at ({(\nr-\n)*2},2.5) {$\oplus$};

                \node (M\nr)            at ({(\nr-\n)*2},3.5) {$M_\nr$};
                \node (K\nr)            at ({(\nr-\n)*2-1},1.5) {$K$};
                \node (D\nr)[encrypt]   at ({(\nr-\n)*2},1.5) {$E$};

                \draw[->,very thick] (x\nr) -- (D\nr);
                \draw[->,very thick] (M\nr) -- (x\nr);

                \draw[->,very thick] (K\nr) -- (D\nr);
                
            }

            \node (C3)            at ({(3-\n)*2},-0.5) {MAC};
            \node (CC3)[encrypt]            at ({(3-\n)*2},0.5)
            {$MSB(Tlen)$};
            \node (A) at ({(3-\n)*2+1},2.5) {$K_1$};


            \foreach \nr in {2, ..., \n}{
                \newcommand{\tmp}{\n-\nr}
                \pgfmathtruncatemacro{\tmp}{\nr-1}
                \draw[->,very thick] (D\tmp) -- ({(\n-\tmp)*-2},0.75) -|
                ({(\n-\tmp)*-2+0.75},0.75) |- ({(\n-\tmp)*-2+0.75},2) |- (x\nr);
            }

            \draw[->,very thick] (D\n) -- (CC\n);
            \draw[->,very thick] (CC\n) -- (C\n);
            \draw[->,very thick] (A) -- (x\n);
            \node (IV) at ({\n*-2+1},2.5) {$0$};
            \draw[->, very thick] (IV) -- (x1);
        \end{tikzpicture}
        &
        \begin{tikzpicture}
            \newcommand{\n}{3}
            \foreach \nr in {1, ..., \n}{
                \node (C\nr)            at ({(\nr-\n)*2},0) {};
                \node (x\nr)       at ({(\nr-\n)*2},2.5) {$\oplus$};

                \node (K\nr)            at ({(\nr-\n)*2-1},1.5) {$K$};
                \node (D\nr)[encrypt]   at ({(\nr-\n)*2},1.5) {$E$};

                \draw[->,very thick] (x\nr) -- (D\nr);
                \draw[->,very thick] (M\nr) -- (x\nr);

                \draw[->,very thick] (K\nr) -- (D\nr);
                
            }

            \foreach \nr in {1, ..., 2}{
                \node (M\nr)            at ({(\nr-\n)*2},3.5) {$M_\nr$};
            }
                \node (M3)            at ({(3-\n)*2},3.5)
                {$M_3||100\cdots0$};

            \node (C3)            at ({(3-\n)*2},-0.5) {MAC};
            \node (CC3)[encrypt]            at ({(3-\n)*2},0.5)
            {$MSB(Tlen)$};
            \node (A) at ({(3-\n)*2+1},2.5) {$K_2$};


            \foreach \nr in {2, ..., \n}{
                \newcommand{\tmp}{\n-\nr}
                \pgfmathtruncatemacro{\tmp}{\nr-1}
                \draw[->,very thick] (D\tmp) -- ({(\n-\tmp)*-2},0.75) -|
                ({(\n-\tmp)*-2+0.75},0.75) |- ({(\n-\tmp)*-2+0.75},2) |- (x\nr);
            }

            \draw[->,very thick] (D\n) -- (CC\n);
            \draw[->,very thick] (CC\n) -- (C\n);
            \draw[->,very thick] (A) -- (x\n);
            \node (IV) at ({\n*-2+1},2.5) {$0$};
            \draw[->, very thick] (IV) -- (x1);
        \end{tikzpicture}
        \end{tabular}

\begin{lstlisting}[mathescape, caption=Generate subkeys]
Step 1. L := AES-128(K, const_Zero);
Step 2. if MSB(L) is equal to 0
            then K1 := L << 1;
            else K1 := (L << 1) XOR const_Rb;
Step 3. if MSB(K1) is equal to 0
            then K2 := K1 << 1;
            else K2 := (K1 << 1) XOR const_Rb;
Step 4. return K1, K2;
\end{lstlisting}

\end{itemize}

\subsubsection{Attack on CBC-MAC}
\begin{tabular}{m{1cm}m{14cm}}
    Let&
    \begin{itemize}
        \item E be the cipher in CBC mode
        \item $m_1$ (resp. $m_2$) a random value with $MAC(m_1) =  M_1$
            ($MAC(m_2 = M_2)$)
        \item $|m_1| = |m_2|=B =  \textrm{ block size of } E$
    \end{itemize}
\end{tabular}

\begin{itemize}
    \item \textbf{MAC of forged message}
        To find the MAC of a forged message without the knowledge of the
        key, let $m_3 = m_1 || (m_2 \oplus M_1)$.
        \begin{multicols}{2}
            \centering
        \begin{tikzpicture}
            \newcommand{\n}{2}
            \foreach \nr in {1, ..., \n}{
                \node (C\nr)            at ({(\nr-\n)*2},0) {};
                \node (D\nr)[encrypt]   at ({(\nr-\n)*2},1.5) {$E$};
                \node (x\nr)       at ({(\nr-\n)*2},2.5) {$\oplus$};
                \node (M\nr)            at ({(\nr-\n)*2},3.5) {};

                \node (K\nr)            at ({(\nr-\n)*2-1},1.5) {$K$};

                \draw[->,very thick] (x\nr) -- (D\nr);
                \draw[->,very thick] (M\nr) -- (x\nr);

                \draw[->,very thick] (K\nr) -- (D\nr);
            }

            \foreach \nr in {1, ..., 1}{
                \node (M\nr)            at ({(\nr-\n)*2},3.5) {$m_\nr$};
            }
            \node (M\n)            at ({(\n-\n)*2},3.5) {$m_\n \oplus
            M_1$ };

            \node (C2)            at ({(2-\n)*2},0) {MAC};

            \foreach \nr in {2, ..., \n}{
                \newcommand{\tmp}{\n-\nr}
                \pgfmathtruncatemacro{\tmp}{\nr-1}
                \draw[->,very thick] (D\tmp) -- ({(\n-\tmp)*-2},0.75) -|
                ({(\n-\tmp)*-2+0.75},0.75) |- ({(\n-\tmp)*-2+0.75},2) |- (x\nr);
            }

            \draw[->,very thick] (D\n) -- (C\n);
            \node (IV) at ({\n*-2+1},2.5) {$0$};
            \draw[->, very thick] (IV) -- (x1);
        \end{tikzpicture}
            \break
            \begin{align*}
                Mac(m_3) &= E\bigg(\underbrace{E(m_1)}_{=M_1} \oplus m_2
                \oplus M_1\bigg)\\
                               &= E (m_2)\\
                                     &=Mac (m_2)
            \end{align*}
        \end{multicols}

    \item \textbf{MAC collision}: Find 2 messages with the same MAC.
        \begin{enumerate}
            \item We have $B_1,B_2$ two arbitrary value such that $|B_1|=|B_2|=B$:
                \begin{eqnarray*}
                    m_3 &=& m_1 || B_1\\
                    m_4 &=& m_2 || B_2
        \end{eqnarray*}
        \item 
        \begin{tabular}{m{6cm}m{6cm}}
            \begin{eqnarray*}
                Mac(m_3) &=& E(\underbrace{E(m_1)}_{=M_1} \oplus B_1)\\
                         &=& E (M_1\oplus B_1)
            \end{eqnarray*}
            &
            \begin{eqnarray*}
                Mac(m_4) &=& E(\underbrace{E(m_2)}_{=M_2} \oplus B_2)\\
                         &=& E (M_2 \oplus B_2)
            \end{eqnarray*}
        \end{tabular}

        \item If $B_2 = M_1 \oplus M_2 \oplus B1\quad \Rightarrow \quad Mac(m_4) = Mac(m_3)$
            \end{enumerate}
\end{itemize}

\subsection{Encrypt and MAC}
Encryption does not ensure integrity but MAC does, so they must be combined.

\begin{tabular}{m{1cm}m{10cm}}
    \begin{tikzpicture}
        \draw (0,0) edge[<-, >=latex] node[left]
        {\rotatebox{90}{\textbf{STRONGER}}} (0, 3);
        \end{tikzpicture}
        &
\begin{eqnarray*}
                                          & &  E_K\big(M\big) \\
    \textrm{Redundancy-then-Encrypt}      &:&  E_K\big(M,R(M)\big)  \\
    \textrm{Hash-then-Encrypt}            &:&  E_K\big(M,h(M)\big)  \\
    \textrm{Hash and Encrypt}             &:&  E_K(M),\  h(M)  \\
    \textrm{MAC and Encrypt} \quad(SSH)   &:&  E_{h1(K)},\ MAC_{h2(K)}(M)  \\
    \textrm{MAC-then-Encrypt} \quad(SSL)  &:&
    E_{h1(K)}\big(M,\ MAC_{h2(K)}(M)\big)   \\
    \textrm{Encrypt-then-MAC}\quad (IPSec)&:&
    E_{h1(K)}(M),\ MAC_{h2(K)}\big(E_{h1(k)}(M)\big)\\
\end{eqnarray*}
\end{tabular}

\subsection{Conclusion}
\begin{itemize}
    \item Do not hash concatenation of key and message to get a MAC.
        Never use the same key for both encryption and MAC.
    \item MAC does not ensure confidentiality and encryption does not ensure integrity.
\end{itemize}
