
\subsection{Definition}

\begin{description}
    \item[Identification:] We get the identity of a party
    \item[Authentication:] We get the identity of a party and the proof that the
        identity is true. The two parties are the \textbf{verifier}
        (verifies the proof) and the \textbf{prover} (provides the proof)
    \item[Token:] A token is an object that someone can own. 
\end{description}

When designing a system, we need authentication or identification.
The magnetic cards cannot bring security by themselves.

\subsection{Token classification}
They can be classified according to technology used:

\begin{itemize}
    \item \textbf{Printed tokens} (Tickets, Barcode)
        \paragraph{Human readable Tokens}
       \label{par:humanreadableTokens}

        \begin{itemize}
            \item If ticket provided by \textbf{verifier} it's less easy to
                falsify 
                
                $\Rightarrow$ Security is based on the difficulty to
                find the paper.

            \item If ticket printed on by the customer easy to copy but
                not to forge.

                $\Rightarrow$ Forgery detected by signature-like
                mechanism and
                copy by using centralized verification system
        \end{itemize}

        \paragraph{Optically-readable Tokens}

        Data are represented such that it can be read by an optical
        machine. These token can be used for authentication.

    \item \textbf{Digital Memory tokens} (Magnetic strips cards, USB, \ldots)
        \paragraph{Magnetic stripe} Mostly \textsc{ISO-7811} compliant
        which contains 3 tracks (IATA, ABA, THRIFT).
        \begin{itemize}
            \item Not \textsc{self-content}: sensitive data stored in
                external database
            \item \textsc{self-content}: contains sensitive data
        \end{itemize}

    \item \textbf{Microcircuit-based tokens}  (Smart cards, RFID, \ldots)
        \begin{itemize}
            \item Classification done according to the
                calculation capabilities and the interface.
            \item Data can (1) never leave the card, (2) be accessible after an
                authentication or (3) be publicly accessible.
        \end{itemize}
        \paragraph{Types}
        \begin{itemize}
            \item Disconnected token such as device for bank
            \item Connected token such as visa card, USB, RFID
        \end{itemize}

\end{itemize}

