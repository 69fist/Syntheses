\subsection{Definition}

\begin{description}
	\item[Identification:] We get the identity of a party
	\item[Authentication:] We get the identity of a party and the proof that the
	identity is true. The two parties are the \textbf{verifier}
    (verifies the proof) and the \textbf{prover} (provides the proof)
\end{description}

A token is an object that someone can own. They can be classified according to
technology used:

\begin{description}
	\item[Printed Tokens:] Tickets, Barcode
	\item[Digital Memory:] Magnetic strips cards,USB,\ldots
	\item[Microcircuit-based:] Smart cards,RFID,\ldots
\end{description}

\subsection{Printed Tokens}

    \paragraph{Human readable Tokens}
    \label{par:humanreadableTokens}

    \begin{itemize}
            \item When the ticket is provided by the verifier,
            the security is based on the difficulty to find the paper.
            \item When ticket printed on by the customer, forgery and
            copy can easy be detected.
    \end{itemize}

    \paragraph{Optically-readable Tokens}

    Data are represented such that it can be read by an optical
    machine. These token can be used for authentication.

\subsection{Digital-memory Tokens}

    For the magnetic stripe, sensitive data can be stored in the
    card or in a external database. Most magnetic stripe cards
    are \textsc{ISO-7811} compliant and contains 3 tracks.

\subsection{Microcircuits Tokens}

\begin{itemize}
    \item Classification on these tokens is done according to the
        calculation capabilities and the interface.
    \item Data can never leave the card,can be accessible after an
    authenfication and can be publicly accessible.
\end{itemize}


