\subsection{Terminology}
\begin{description}
    \item[Cryptography] Science that design algorithms to ensure confidentiality
    of communication through insecure channels.
    \item[Confidentiality] Insurance that a given information cannot be accessed
    by unauthorized parties.
    \item[Cryptanalysis] Science that proves or disproves the security of
    cryptographic algorithm.
\end{description}
Break a cryptographic algortim means either:
\begin{itemize}
    \item Decrypting an encrypted message.
    \item Recovering the key of the cryptographic algorithm.
    \item Proving that a algorithm is less secure than what is claimed.
\end{itemize}

\paragraph{Attacker Model}
\begin{itemize}
    \item Adversary can be passive (ciphertext-only, known-plaintext) or
    \item active (chosen-plaintext,chosen-ciphertext)
\end{itemize}

\subsubsection{Encription Algorithm}
\begin{description}
    \item[Encryption algorithm] Algorithm that transforms an intelligible text
    into text that is unintelligible for non-authorized parties. The input is
    the plaintext and the output is the ciphertext.
\end{description}
Two categories of cryptographic algorithm:
\begin{itemize}
    \item Symmetric-key where the same key is used for both encryption and
    decryption.
    \item Asymmetric-key cryptography where there is a public key to encrypt and
    an private key to decrypt.
\end{itemize}

\subsection{Encryption}
%%TODO put it in table
\subsubsection{Symmetric-key Cryptography}
\begin{itemize}
    \item Based on bit or byte operations.
    \item Tend to be fast.
    \item Typical key-size: 128 bits
    \item Best attack is the exhaustive search.
\end{itemize}
Symmetric-key algorihtm can use Block ciphers or Stream ciphers. Block ciphers
acts on the plaintext in blocks. Stream ciphers acts on the plaintext one
symbol at a time.
\paragraph{Note:} DES is known to not be secure anymore.
\subsubsection{Asymmetric-key Cryptography}
\begin{itemize}
    \item Based on mathematical problems
    \item Best attack exploit the mathematical structure.
    \item Typical key size: 1024 bits.
    \item Encryptions and decryptions are slow.
\end{itemize}
\paragraph{Note:} RSA stands for Rivest-Shamir-Adleman
Certificate are used to identiry of the key owner. A Certificate authority
issue the certificate.

Parties can exchange their key or used Diffie-Hellman to agree on a shared key.

\subsection{Authentication and Integrity}
Entity authentication is asymmetric or symmmetric. Data authentication is
asymmetric or symmetric.

\subsection{Hash Function}
$$h:{0,1}*\rightarrow{0,1}^n$$
\begin{description}
    \item[First preimage resistance] Given a hash value y, it is infeasible to
    find m such that $h(m) = y$.
    \item[Second preimage resistance] Given a message $m_1$, it is infeasible
    to find a different message $m_2$ such that $h(m_1)=h(m_2)$.
    \item[Collision resistance] It is infeasible to find two different messages
    $m_1$ and $m_2$ such that $h(m_1)=h(m_2)$
    \item[Random oracle property] $h(m)$ is distinguishable from a random n-bit
    value.
\end{description}

\paragraph{Birthday Paradox} If we pick $\theta\sqrt{N}$ random numbers,
independently and uniformly, in {1,2,\ldots,N}, we get at least one number twice
with probability:
$$1-e^{-\frac{\theta^2}{2}}$$
With the Birthday paradox, we need about $\sqrt{2^n}$ hash operations to find a
collision with probability $\frac{1}{2}$
\paragraph{Message Authentication Code} Hash function that use a key.


