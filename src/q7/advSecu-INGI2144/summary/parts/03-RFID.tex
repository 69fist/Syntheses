\subsection{Definition}

\begin{description}
    \item[Radio Frequency IDentification:] Remotely retrieves data
    using devices called RFID tags through electromagnetic radiating
    waves.
    \item[RFID tags:] Small device containing a chip and an antenna to
    receive/respond to radio-frequency queries from an RFID reader/writer.
\end{description}
RFID tag can be low-capability device or porwerful contacless smartcard.
%%TODO Architecture slide 60?

\subsection{Daily Life Example}

\begin{itemize}
    \item Pet identification
    \item Localisation
    \item Book borrowing and inventories.
\end{itemize}

\subsection{Tags Characteristics}
\subsubsection{Power Source}
\begin{description}
    \item[Passive] Tags don't have any internal energy source. They
    get energy from the reader.
    \item[Active] Tags have a battery that is used both for internal
    calculations and transmission.
    \item[Semi-Passive] Tags only have energy for the calculation.
\end{description}


\subsubsection{Frequency Bands}
\begin{table}
    \centering
    \begin{tabular}{c|c|c}
        & Frequency & Range \\
        \hline
        Low Frequency &  124\text{-}135 kHz & centimeters \\
        High Frequency &  13.56 MHz & decimeters \\
        Ultra-High Frequency & 860\text{-}960 MHz & meters \\
    \end{tabular}
\end{table}

\subsubsection{Memory}
Tags have at least a few bits to store unique identifier UID\@.
\begin{itemize}
        \item UID size 32 to 128 bits.
        \item Usually, the UID is chosen by the manufacturer and cannot
        be changed by the user.
        \item EEPROM additional memory can be added (1KB or 70KB for passport)
\end{itemize}

\begin{description}
    \item[Tamper-resistance] A device is tamper-resistant if no adversary can
    get access to its protected memory by use of side-channel attacks.
    \item[Side-channel attack] An attack where the information is retrieved
    from physical implementation of a system (ex:Timing attack,physical attack).
\end{description}

\paragraph{Computation Capabilities}
The tags can perfom simple logic operation and symmetric/asymmetric
cryptographic. For symmetric cryptography, the microprocessor is not
necessarily needed.

\paragraph{Near Field Communication}
NFC is an extension of RFID, its main difference its that it offer
Peer-to-Peer connections between two (active) devices. NFC Data are exchanged
in a different format.

\subsection{Communication with the Tag}
For tags compliant with \textsc{ISO 7816}
\subsubsection{Memory}
Internal structure composed of files. There are two types of files:
\begin{itemize}
    \item Dedicated File (\textbf{DF})
    \item Elementary File (\textbf{EF}) that can be identified in two subtypes
    \begin{description}
        \item[Internal] Store information used by the card for management and
        control purpose.
        \item[External] Store information used exclusively by the outside world
    \end{description}
\end{itemize}
The Master File (\textbf{MF}) is a special \textbf{DF} and is the only mandatory
file. Each application is stored in a distinct \textbf{DF}.

Data inside an EF is stored in different formats:
\begin{description}
        \item[data unit] Smallest set of bits. Default value is 1 byte.
        \item[record] String of bytes which can be handled as a whole. Record
        can be referenced by a an 8-bit identifier.
        \item[date object] Structure of information which consists of a tag, a
        length and a value.
\end{description}

\paragraph{Communication}
The protocol used to exchange data is the Application Protocol Data Unit.


