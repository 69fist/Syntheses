\subsection{Adversary Means}
\begin{itemize}
    \item[Blocks]
\item\textbf{Eavesdropping} Adversary listens to the channel and hopes getting some
        useful data.
    \item\textbf{Skimming} Adversary queries the prover to get some useful information.
    \item\textbf{Modifying} Adversary modifies messages while they transit on the
        channel.
    \item\textbf{Injecting} Adversary injects messages on the channel.
    \item\textbf{Tampering} Adversary obtains content of the protected memory by
        physical means.
    \item\textbf{Exploiting} Adversary obtains information thanks to the system
        implementation.

    \item[Attacks]
    \item\textbf{Re-playing} Adversary replays a message she previously observed.
    \item\textbf{Pre-playing} Adversary plays the message she obtained.
    \item\textbf{Reflecting} Adversary reflects a reflect from a verifier so that
        answers itself.
    \item\textbf{Guessing} Adversary tries to guess the right answer or key w/o help of
        the prover.
    \item\textbf{Relaying} Adversary forwards the signal between ther verifier and the
        prover.
\end{itemize}

\subsubsection{Dolev-Yao model}
In the \textbf{Dolev-Yao} Model, adversary
is powerful but not all powerful: 
\begin{itemize}
    \item Cannot guess random numbers
    \item Cannot decrypt or create valid ciphertexts without correct secret.
    \item Cannot retrieve private keys from public information.
\end{itemize}

\subsubsection{Avoid replay and preplay}
\begin{itemize}
    \item Authentication based on the exchange of a password or hash of
        password in no secure
    \item[$\Rightarrow$]
        To avoid replay and preplay attack, one-time password or
        challenge/response protocol can be used.
\end{itemize}

%%TODO get schema from note

\subsection{One-time Passwords}
Password are generated with a one-way function.

\subsection{Challenge response}
Challenge response protocol can use different mechanism:
\begin{itemize}
    \item Timestamps where the challenge is based on clock.
    \item Sequence numbers where the challenge is based on hash on a
        one time password.
    \item Random numbers
\end{itemize}

\subsubsection{Challenge Response example}
\begin{itemize}
    \item \textbf{Bad CR} protocol
        \begin{eqnarray*}
            A \leftarrow B & r_B \\
            A \rightarrow B & E_k(r_B) \\
        \end{eqnarray*}

        \begin{itemize}
            \item An attacker can't get the secret key $k$ by
                observing protocol
            \item B can be sure than message comes from A because
                only A and B share $k$
            \item No replay because $r_B$ is a nonce
        \end{itemize}

        \paragraph{Bad}
        \begin{itemize}
            \item Attacker can replay $E_k(r_B)$ if $r_B$ appear twice
            \item If $r_B$ is predictible, the adversary can play
                the role of B (verifier) in front of A, sending him
                a random $r_B$ , then play the role of A (prover) in
                front of B.
            \item Chosen-plaintext attack possible because response
                from A is not randomized.
                \begin{enumerate}
                    \item Adversary can precompute all possible
                        answers
                    \item The adversary can require the
                        prover to sign a chosen value (with
                        public key CR)
                \end{enumerate}
        \end{itemize}

    \item \textbf{Medium CR} protocol
        \begin{eqnarray*}
            A \leftarrow B & r_B \\
            A \rightarrow B & E_k(r_A, r_B) \\
        \end{eqnarray*}

        \paragraph{Bad}
        \begin{itemize}
        \item A adversary can make a reflection attacks
            \begin{enumerate}
                \item Initiate a new protocol with B and send $r_B$
                \item B will response with $E_k(r_B', r_B)$ which is
                    precisely the answer the adversary must provide
                    to B to impersonate A
            \end{enumerate}
        \end{itemize}

    \item \textbf{Better CR} protocol
        \begin{eqnarray*}
            A \leftarrow B & r_B \\
            A \rightarrow B & E_k(r_A, r_B, B) \\
        \end{eqnarray*}

        \begin{itemize}
            \item Unilateral authentication protocol (A to B)
        \end{itemize}

    \item \textbf{King CR} protocol
        \begin{eqnarray*}
            A \leftarrow B & r_B \\
            A \rightarrow B & E_k(r_B, B), r_A \\
            A \leftarrow B & E_k(r_A, A) \\
        \end{eqnarray*}
\end{itemize}


\subsection{ISO 9798 Challenge/Response}
\begin{itemize}
    \item Mechanism~1: Unilateral authentication with timestamps
    \item Mechanism~2: Unilateral authentication with random numbers
    \item Mechanism~3: Mutual authentication with random numbers
\end{itemize}

\subsubsection{ISO 9798\text{-}2 based on Symmetric-Key Encryption}

\begin{eqnarray*}
    Mechanism~1 \quad & A  \rightarrow B &  E_k(t_A,B)\\
    \\
    Mechanism~2 \quad & A \leftarrow  B & r_B \\
                & A  \rightarrow B & E_k(r_B,B) \\
    \\
    Mechanism~3 \quad & A  \leftarrow B & r_B \\
                & A  \rightarrow B & E_k(r_A,r_B,B) \\
                & A  \leftarrow  B & E_k(r_B,r_A) \\
    \\
    Map1 (Mech~3 alternative) & A  \leftarrow B & r_B \\
                & A  \rightarrow B & E_k(A, B, r_B,r_A) \\
                & A  \leftarrow  B & E_k(\textcolor{red}{B},r_A) \\
    \\
    Map1.1 (Mech~3 alternative) & A  \leftarrow B & r_B \\
                & A  \rightarrow B & E_k(A, B, r_B,r_A) \\
                & A  \leftarrow  B & E_k(r_A) 
\end{eqnarray*}\\

\subsubsection{ISO 9798\text{-}4 based on Hash function}
\begin{eqnarray*}
    Mechanism-1 \quad & A \rightarrow B & H_k(t_A,B),t_A \\
    \\
    Mechanism-2 \quad & A \leftarrow B & r_B \\
                      & A \rightarrow B & H_k(r_B,B),B \\
    \\
    Mechanism-3 \quad & A \leftarrow B & r_B \\
                      & A \rightarrow B & H_k(r_A,r_B,B),r_A \\
                      & A \leftarrow B & H_k(r_B,r_A,A) 
\end{eqnarray*}

\subsubsection{ISO 9798\text{-}3 based on Public-Key Signature}
\begin{eqnarray*}
    Mechanism-1 \quad & A \rightarrow B & S_A(t_A,B),B,t_A,cert_A \\
    \\
    Mechanism-2 \quad & A \leftarrow B & r_B\\
                      & A \rightarrow B &
    S_A(r_A,r_B,B),B,r_A,cert_A \\
    \\
    Mechanism 3 \quad & A \leftarrow B & r_B \\
                      & A \rightarrow B &
    S_A(r_A,r_B,B),B,r_A,cert_A \\
    & A \leftarrow B & S_B(r_B, r_A, A), A, cert_B \\
    \\
    Public-key witness (Mech~3 alternative) & A \leftarrow B & h(r), B, P_A(r, B) \\
    & A \rightarrow B & r \\
    \\
    Public-key (Mech~3 alternative) & A \leftarrow B & h(r), B, P_A(r_B, B) \\
                                    & A \rightarrow B & P_B(r_B, r_A)\\
                                    & A \leftarrow B & r_A
\end{eqnarray*}

\paragraph{Note} 
\begin{itemize}
    \item $r$ is a witness, it proves that B knows r : this avoids a
        chosen-plaintext attack
    \item $P_A(m)$ means the
        encryption of the message $m$ with A’s public key.
\end{itemize}

\subsection{Authentication with Key Establshment}
During the phase of authentication phase, the two parties can exchange key
material.

ISO 11770\text{-}2 Symmetric-Key
\begin{itemize}
    \item Mechanism 4
        $$ A \leftarrow B \quad r_B $$
        $$ A \rightarrow B \quad E_k(r_B,B,k_1) $$
        The session key is $k_1$

    \item Mechanism 6
        $$ A \leftarrow B \quad r_B $$
        $$ A \rightarrow B \quad E_k(r_A,r_B,B,k_1) $$
        $$ A \leftarrow B \quad E_k(r_B,r_A,k_2) $$
        With a key-derivation function $f$, $f(k_1,k_2)$ is the session key.
\end{itemize}

Nedham\text{-}Schroeder Public-Key
\begin{itemize}
    \item Original Version
        $$ A \leftarrow B \quad P_A(k_1,B)  $$
        $$ A \rightarrow B \quad P_B(k_1,k_2) $$
        $$ A \leftarrow B P_A(k_2) $$
        $f(k_1,k_2)$ is the session key.
    \item Modified Version
        $$ A \leftarrow B \quad P_A(k_1,B,r_1) $$
        $$ A \rightarrow B \quad P_B(k_2,r_1,r_2) $$
        $$ A \leftarrow B \quad r_2 $$
\end{itemize}

ISO 11770\text{-} Asymmetric-Key
\begin{itemize}
    \item Mechanism 5
        $$ A \leftarrow B \quad r_B $$
        $$ A \rightarrow B \quad r_A,r_B,B,P_B(A,k_1),S_A(r_A,r_B,B,P_B(A,k_1))$$
        $$ A \leftarrow B \quad r_B,r_A,A,P_A(B,k_2),S_B,(r_B,r_A,A,P_A(B,k_2))$$
        Session key can be $f(k_1,k_2)$.
    \item Mechanims 6
        $$ A \leftarrow B \quad P_A(B,k_2,r_B) $$
        $$ A \rightarrow B \quad P_B(A,k_2,r_B,r_A) $$
        $$ A \leftarrow B \quad r_A $$
        Session key van be $f(k_1,k_2)$.
\end{itemize}
X.509
\begin{itemize}
    \item 2-Pass Mutual Authentication Protocol
        $$ A \leftarrow B \quad cert_B,D_B,S_B(D_B) $$
        $$ D_B = t_B,r_B,A,data*_1,P_A(k_1)* $$
        $$ A \rightarrow B \quad cert_A,D_A,S_A(D_A) $$
        $$ D_A = t_A,r_A,B,r_B,data_2,P_B(k_2)* $$
        $t_x$ defines a validity period and $r_x$ includes a sequential component.
        $f(k_1,k_2)$ is the session key.
    \item 3-Pass Mutual Authentication Protocol
        $$ A \leftarrow B \quad cert_B,D_B,S_B(D_B) $$
        $$ D_B = (t_B,r_b,A,data_1,P_A(k_1)*) $$
        $$ A \rightarrow B \quad cert_A,D_A,S_A(D_A) $$
        $$ D_A = (t_A,r_A,B,r_B,data*_2,P_B(k_2)*) $$
        $$ A \leftarrow B \quad r_A,A,S_B(r_A,A) $$
\end{itemize}

