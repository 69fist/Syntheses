\subsection{Adversary Means}
\begin{description}
    \item[Eavesdropping] Adversary listens to the channel and hopes getting some
    useful data.
    \item[Skimming] Adversary queries the prover to get some useful information.
    \item[Modifying] Adversary modifies messages while they transit on the
    channel.
    \item[Injecting] Adversary injects messages on the channel.
    \item[Tampering] Adversary obtains content of the protected memory by
    physical means.
    \item[Exploiting] Adversary obtains information thanks to the system
    implementation.
    \item[Re-playing] Adversary replays a message she previously observed.
    \item[Pre-playing] Adversary plays the message she obtained.
    \item[Reflecting] Adversary reflects a reflect from a verifier so that
    answers itself.
    \item[Guessing] Adversary tries to guess the right answer or key w/o help of
    the prover.
    \item[Relaying] Adversary forwards the signal between ther verifier and the
    prover.
\end{description}
In the \textbf{Dolev-Yao} Model, adversary
\begin{itemize}
    \item Cannot guess random numbers
    \item Cannot decrypt or create valid ciphertexts without correct secret.
    \item Cannot retrieve private keys from public information.
\end{itemize}
\paragraph{Note:} To avoid replay and preplay attack, one-time password or
challenge/response protocol can be used.

%%TODO get schema from note
\subsection{One-time Passwords}
Password are generated with a one-way function.
\subsection{Challenge response}
Challenge response protocol can use different mechanism:
\begin{enumerate}
    \item Unilateral (only one party is authenticated) authentication based on Timestamp
    \item Unilateral authentication based on Random Numbers
    \item Bilateral (both parties are authenticated) authentication based on Random Numbers
\end{enumerate}

%%Align all these
\subsection{ISO 9798 Challenge/Response}
ISO 9798\text{-}2 based on Symmetric-Key Encryption:
\begin{itemize}
    \item Mechanism 1
    		$$ A  \rightarrow B \quad E_k(t_A,B) $$
    \item Mechanism 2
    		$$ A  \leftarrow  B \quad r_B $$
    		$$ A  \rightarrow B \quad E_k(r_B,B) $$
    \item Mechanism 3
    		$$ A  \leftarrow B \quad r_B $$
    		$$ A  \rightarrow B \quad E_k(r_A,r_B,B) $$
    		$$ A  \leftarrow  B \quad E_k(r_n,r_A) $$
\end{itemize}
%%TODO VARIANT slide 162
ISO 9798\text{-}3 based on Public-Key Signature:
\begin{itemize}
    \item Mechanism 1
    $$ A \rightarrow B \quad S_A(t_A,B),B,t_A,cert_A $$
    \item Mechanism 2
    $$ A \leftarrow B \quad r_B$$
    $$ A \rightarrow B \quad S_A(r_A,r_B,B),B,r_A,cert_A $$
    \item Mechanism 3
    $$ A \leftarrow B \quad r_B $$
    $$ A \rightarrow B \quad S_A(r_A,r_B,B),B,r_A,cert_A $$
    $$ A \leftarrow B \quad S_B(r_B,r_A,A),A,cert_B $$
\end{itemize}
%%TODO VARIANT slide 174

ISO 9798\text{-}4 based on Hash function:
\begin{itemize}
    \item Mechanism 1
    $$ A \rightarrow B \quad H_k(t_A,B),t_A $$
    \item Mechanism 2
    $$ A \leftarrow B \quad r_B $$
    $$ A \rightarrow B \quad H_k(r_B,B),B $$
    \item Mechanism 3
    $$ A \leftarrow B \quad r_B $$
    $$ A \rightarrow B \quad H_k(r_A,r_B,B),r_A $$
    $$ A \leftarrow B \quad H_k(r_b,r_A,A) $$
\end{itemize}
\subsection{Authentication with Key Establishment}
During the phase of authentication phase, the two parties can exchange key
material.

\subsubsection{ISO 11770\text{-}2 Symmetric-Key}
\begin{itemize}
    \item Mechanism 4
    $$ A \leftarrow B \quad r_B $$
    $$ A \rightarrow B \quad E_k(r_B,B,k_1) $$
    The session key is $k_1$

    \item Mechanism 6
    $$ A \leftarrow B \quad r_B $$
    $$ A \rightarrow B \quad E_k(r_A,r_B,B,k_1) $$
    $$ A \leftarrow B \quad E_k(r_B,r_A,k_2) $$
    With a key-derivation function $f$, $f(k_1,k_2)$ is the session key.
\end{itemize}

\subsubsection{Nedham\text{-}Schroeder Public-Key}
\begin{itemize}
    \item Original Version
    $$ A \leftarrow B \quad P_A(k_1,B)  $$
    $$ A \rightarrow B \quad P_B(k_1,k_2) $$
    $$ A \leftarrow B P_A(k_2) $$
    $f(k_1,k_2)$ is the session key.
    \item Modified Version
    $$ A \leftarrow B \quad P_A(k_1,B,r_1) $$
    $$ A \rightarrow B \quad P_B(k_2,r_1,r_2) $$
    $$ A \leftarrow B \quad r_2 $$
\end{itemize}

\subsubsection{ISO 11770\text{-} Asymmetric-Key}
\begin{itemize}
    \item Mechanism 5
    $$ A \leftarrow B \quad r_B $$
    $$ A \rightarrow B \quad r_A,r_B,B,P_B(A,k_1),S_A(r_A,r_B,B,P_B(A,k_1))$$
    $$ A \leftarrow B \quad r_B,r_A,A,P_A(B,k_2),S_B,(r_B,r_A,A,P_A(B,k_2))$$
    Session key can be $f(k_1,k_2)$.
    \item Mechanims 6
    $$ A \leftarrow B \quad P_A(B,k_2,r_B) $$
    $$ A \rightarrow B \quad P_B(A,k_2,r_B,r_A) $$
    $$ A \leftarrow B \quad r_A $$
    Session key van be $f(k_1,k_2)$.
\end{itemize}
\subsubsection{X.509}
\begin{itemize}
    \item 2-Pass Mutual Authentication Protocol
    $$ A \leftarrow B \quad cert_B,D_B,S_B(D_B) $$
    $$ D_B = t_B,r_B,A,data*_1,P_A(k_1)* $$
    $$ A \rightarrow B \quad cert_A,D_A,S_A(D_A) $$
    $$ D_A = t_A,r_A,B,r_B,data_2,P_B(k_2)* $$
    $t_x$ defines a validity period and $r_x$ includes a sequential component.
    $f(k_1,k_2)$ is the session key.
    \item 3-Pass Mutual Authentication Protocol
    $$ A \leftarrow B \quad cert_B,D_B,S_B(D_B) $$
    $$ D_B = (t_B,r_b,A,data_1,P_A(k_1)*) $$
    $$ A \rightarrow B \quad cert_A,D_A,S_A(D_A) $$
    $$ D_A = (t_A,r_A,B,r_B,data*_2,P_B(k_2)*) $$
    $$ A \leftarrow B \quad r_A,A,S_B(r_A,A) $$
\end{itemize}

