\subsection{Adversary Means}
\begin{itemize}
    \item[Blocks]
\item\textbf{Eavesdropping} Adversary listens to the channel and hopes getting some
        useful data.
    \item\textbf{Skimming} Adversary queries the prover to get some useful information.
    \item\textbf{Modifying} Adversary modifies messages while they transit on the
        channel.
    \item\textbf{Injecting} Adversary injects messages on the channel.
    \item\textbf{Tampering} Adversary obtains content of the protected memory by
        physical means.
    \item\textbf{Exploiting} Adversary obtains information thanks to the system
        implementation.

    \item[Attacks]
    \item\textbf{Re-playing} Adversary replays a message she previously observed.
    \item\textbf{Pre-playing} Adversary skim the prover then plays the message she obtained.
    \item\textbf{Reflecting} Adversary reflects a reflect from a verifier so that
        answers itself.
    \item\textbf{Guessing} Adversary tries to guess the right answer or key w/o help of
        the prover.
    \item\textbf{Relaying} Adversary forwards the signal between ther verifier and the
        prover.
\end{itemize}

\subsubsection{Dolev-Yao model}
In the \textbf{Dolev-Yao} Model, adversary
is powerful but not all powerful: 
\begin{itemize}
    \item Cannot guess random numbers
    \item Cannot decrypt or create valid ciphertexts without correct secret.
    \item Cannot retrieve private keys from public information.
\end{itemize}

\subsubsection{Avoid replay and preplay}
\begin{itemize}
    \item Authentication based on the exchange of a password or hash of
        password in no secure
    \item[$\Rightarrow$]
        To avoid replay and preplay attack, one-time password or
        challenge/response protocol can be used.
\end{itemize}

\subsection{One-time Passwords}
Passwords are generated with a one-way function $h$.
\begin{enumerate}
	\item Picks a random $S_0$ 
	\begin{center}
	\begin{tikzpicture}
		\node at (0,0) (s0) {$S_0$};
		\node [right =of s0] (s1) {$S_1$};
		\node [right = of s1] (s2) {$S_2$};
		\node [right = of s2] (dots) {\ldots};
		\node [right = of dots] (sn) {$S_n$};
	
		\draw [->]  (s0) -- (s1) node [above,midway] {h};
		\draw [->]  (s1) -- (s2) node [above,midway] {h};
		\draw [->]	(s2) -- (dots) node [above,midway] {h};
		\draw [->]	(dots) -- (sn) node [above,midway] {h};
	\end{tikzpicture}
	\end{center}
	\item Verifier has $S_n$
	\begin{center}
	\begin{tikzpicture}
		\node at (0,0) (p) {P};
		\node [right = 3cm of p] (v) {V};
		
		\draw [->] (p) -- (v) node [above,midway] {$S_n$};
	\end{tikzpicture}
	\end{center}
	\item To authenticate
	\begin{center}
	\begin{tikzpicture}
		\node at (0,0) (p) {P};
		\node [right = 3cm of p] (v) {V};
		
		\draw [->] (p) -- (v) node [above,midway] {$S_{n-1}$};
	\end{tikzpicture} 
	\end{center}
\end{enumerate}
Verifier will check that $h(S_{n-1}) = S_n$, in that case the prover is the
legitimate one. Verifier must update its value to $S_{n-1}$.
\paragraph{Note} A replay attack is not possible because $S_n$ is
updated at every authentication. But a preplay is still possible (depends on the
timing).

\subsection{Challenge response}
Challenge response protocol can use different mechanism:

\begin{enumerate}
    \item Unilateral (only one party is authenticated) authentication
        based on Timestamps where the challenge is based on clock.
    \item Unilateral authentication based on Random Numbers where the
        challenge is based on hash on a
        one time password.
    \item Bilateral (both parties are authenticated) authentication based on Random Numbers
\end{enumerate}

\subsubsection{Challenge Response example}
\begin{itemize}
    \item \textbf{Bad CR} protocol
        \begin{eqnarray*}
            A \leftarrow B & r_B \\
            A \rightarrow B & E_k(r_B) \\
        \end{eqnarray*}

        \begin{itemize}
            \item An attacker can't get the secret key $k$ by
                observing protocol
            \item B can be sure than message comes from A because
                only A and B share $k$
            \item No replay because $r_B$ is a nonce
        \end{itemize}

        \paragraph{Bad}
        \begin{itemize}
            \item Attacker can replay $E_k(r_B)$ if $r_B$ appear twice
            \item If $r_B$ is predictible, the adversary can play
                the role of B (verifier) in front of A, sending him
                a random $r_B$ , then play the role of A (prover) in
                front of B.
            \item Chosen-plaintext attack possible because response
                from A is not randomized.
                \begin{enumerate}
                    \item Adversary can precompute all possible
                        answers
                    \item The adversary can require the
                        prover to sign a chosen value (with
                        public key CR)
                \end{enumerate}
        \end{itemize}

    \item \textbf{Medium CR} protocol
        \begin{eqnarray*}
            A \leftarrow B & r_B \\
            A \rightarrow B & E_k(r_A, r_B) \\
        \end{eqnarray*}

        \paragraph{Bad}
        \begin{itemize}
        \item A adversary can make a reflection attacks
            \begin{enumerate}
                \item Initiate a new protocol with B and send $r_B$
                \item B will response with $E_k(r_B', r_B)$ which is
                    precisely the answer the adversary must provide
                    to B to impersonate A
            \end{enumerate}
        \end{itemize}

    \item \textbf{Better CR} protocol
        \begin{eqnarray*}
            A \leftarrow B & r_B \\
            A \rightarrow B & E_k(r_A, r_B, B) \\
        \end{eqnarray*}

        \begin{itemize}
            \item Unilateral authentication protocol (A to B)
        \end{itemize}

    \item \textbf{Little Better CR} protocol
        \begin{eqnarray*}
            A \leftarrow B & r_B \\
            A \rightarrow B & E_k(r_B, B), r_A \\
            A \leftarrow B & E_k(r_A, A) \\
        \end{eqnarray*}
        
         \begin{itemize}
            \item This is an unilateral authentication applied twice, instead of a mutual
            authentication.
            \item B authenticated A and vice-versa.
            \item Depending of the implementation, B could not know wheter A
            authenticated it  (the two protocols are no linked)
        \end{itemize}
        
	\item \textbf{King CR} protocol
		\begin{eqnarray*}
			A \leftarrow B & r_B \\
			A \rightarrow B & E_k(r_B,r_A, B) \\
			A \leftarrow B & E_k(r_B,r_A,A) \\
		\end{eqnarray*}
\end{itemize}


\subsection{ISO 9798 Challenge/Response}
\begin{itemize}
    \item Mechanism~1: Unilateral authentication with timestamps
    \item Mechanism~2: Unilateral authentication with random numbers
    \item Mechanism~3: Mutual authentication with random numbers
\end{itemize}

\subsubsection{ISO 9798\text{-}2 based on Symmetric-Key Encryption}
\begin{eqnarray*}
    Mechanism~1 \quad & A  \rightarrow B &  E_k(t_A,B)\\
    \\
    Mechanism~2 \quad & A \leftarrow  B & r_B \\
                & A  \rightarrow B & E_k(r_B,B) \\
    \\
    Mechanism~3 \quad & A  \leftarrow B & r_B \\
                & A  \rightarrow B & E_k(r_A,r_B,B) \\
                & A  \leftarrow  B & E_k(r_B,r_A) \\
    \\
    Map1 \quad & A  \leftarrow B & r_B \\
                & A  \rightarrow B & E_k(A, B, r_B,r_A) \\
                & A  \leftarrow  B & E_k(\textcolor{red}{B},r_A) \\
    \\
    Map1.1 \quad & A  \leftarrow B & r_B \\
                & A  \rightarrow B & E_k(A, B, r_B,r_A) \\
                & A  \leftarrow  B & E_k(r_A) 
\end{eqnarray*}

\subsubsection{ISO 9798\text{-}4 based on Hash function}
\begin{eqnarray*}
    Mechanism~1 \quad & A \rightarrow B & H_k(t_A,B),t_A \\
    \\
    Mechanism~2 \quad & A \leftarrow B & r_B \\
                      & A \rightarrow B & H_k(r_B,B),B \\
    \\
    Mechanism~3 \quad & A \leftarrow B & r_B \\
                      & A \rightarrow B & H_k(r_A,r_B,B),r_A \\
                      & A \leftarrow B & H_k(r_B,r_A,A) 
\end{eqnarray*}

\subsubsection{ISO 9798\text{-}3 based on Public-Key Signature}
\begin{eqnarray*}
    Mechanism~1 \quad & A \rightarrow B & S_A(t_A,B),B,t_A,cert_A \\
    \\
    Mechanism~2 \quad & A \leftarrow B & r_B\\
                      & A \rightarrow B &
    S_A(r_A,r_B,B),B,r_A,cert_A \\
    \\
    Mechanism~3 \quad & A \leftarrow B & r_B \\
                      & A \rightarrow B &
    S_A(r_A,r_B,B),B,r_A,cert_A \\
    & A \leftarrow B & S_B(r_B, r_A, A), A, cert_B \\
    \\
    Pub-witness & A \leftarrow B & h(r), B, P_A(r, B) \\
    & A \rightarrow B & r \\
    \\
    Public-key & A \leftarrow B & P_A(r_B, B) \\
                                    & A \rightarrow B & P_B(r_B, r_A)\\
                                    & A \leftarrow B & r_A
\end{eqnarray*}

\paragraph{Note} 
\begin{itemize}
    \item $r$ is a witness, it proves that B knows r : this avoids a
        chosen-plaintext attack
    \item $P_A(m)$ means the
        encryption of the message $m$ with A’s public key.
\end{itemize}


\subsection{Authentication with Key Establishment}
During the authentication phase, the two parties can exchange key
material.

\begin{itemize}
    \item The \textbf{session key} is $k_1$ or a key-derivation function  
        $f(k_1,k_2)$ if there is $k_1$ and $k_2$
\end{itemize}

\subsubsection{ISO 11770\text{-}2 Symmetric-Key}
\begin{eqnarray*}
    Mechanism~4 \quad &  A \leftarrow B & r_B \\
                      & A \rightarrow B & E_k(r_B,B,k_1) \\
    \\
    Mechanism~6 \quad & A \leftarrow B & r_B \\
                      & A \rightarrow B & E_k(r_A,r_B,B,k_1) \\
                      & A \leftarrow B & E_k(r_B,r_A,k_2) \\
\end{eqnarray*}

\paragraph{Note}
Mechanism 6 used in basic access control of ePassport

\subsubsection{Nedham\text{-}Schroeder Public-Key}
\begin{eqnarray*}
    Original \quad &  A \leftarrow B & P_A(k_1,B)  \\
                   & A \rightarrow B &  P_B(k_1,k_2) \\
                   & A \leftarrow B & P_A(k_2) \\
    \\
    Modified \quad &  A \leftarrow B & P_A(k_1, B, r_1)  \\
                   & A \rightarrow B & P_B(k_2,r_1,r_2) \\
                   & A \leftarrow B & r_2 \\
\end{eqnarray*}

\subsubsection{ISO 11770\text{-}3 Asymmetric-Key}
\begin{eqnarray*}
    Mechanism~5 \quad &  A \leftarrow B & r_B \\
                      & A \rightarrow B & r_A,r_B,B, \quad P_B(A,k_1),
    \quad S_A(r_A,r_B,B,P_B(A,k_1))\\
    & A \leftarrow B &
    r_B,r_A,A, \quad P_A(B,k_2), \quad S_B,(r_B,r_A,A,P_A(B,k_2))\\
    \\
    Mechanism~6 \quad &  A \leftarrow B & P_A(B,k_2,r_B) \\
                      & A \rightarrow B & P_B(A,k_2,r_B,r_A) \\
                      & A \leftarrow B & r_A 
\end{eqnarray*}

\paragraph{Mechanism~5} 
\begin{itemize} 
        \item Session key can also be only $k_1$ or $k_2$.
In such a case, $P_A(B, k_2)$ can be omitted (resp. $P_B (A, k_1 )$).
\end{itemize}

\paragraph{Mechanism~6}
\begin{itemize}
    \item Only encryption is used (no signature) but required stronger
        encryption scheme
    \item Instead of use $f(k_1, k_2)$, $k_1$ may be used by A to encrypt messages for B and
        authenticate messages from B, and a similarly for $k_2$ .
\end{itemize}

\subsubsection{X.509}
Mutual authentication protocol

\begin{eqnarray*}
    2~Pass \quad &  A \leftarrow B & cert_B,D_B,S_B(D_B) 
                 \quad \textrm{ where } D_B = ( t_B,r_B,A,data*_1,P_A(k_1)*)\\
                 & A \rightarrow B & cert_A,D_A,S_A(D_A) 
                  \quad \textrm{ where  } D_A =
    (t_A,r_A,B,r_B,data_2,P_B(k_2)* )\\
    \\
    3~Pass \quad &  A \leftarrow B & cert_B,D_B,S_B(D_B) 
        \quad \textrm{ where  } D_B = (t_B,r_b,A,data_1,P_A(k_1)*) \\
        & A \rightarrow B & cert_A,D_A,S_A(D_A) 
        \quad \textrm{ where } D_A = (t_A,r_A,B,r_B,data*_2,P_B(k_2)*)
        \\
        & A \leftarrow B & r_A,A,S_B(r_A,A) 
\end{eqnarray*}

\begin{itemize}
    \item $t_x$ defines a validity period 
    \item  $r_x$ includes a sequential component
\end{itemize}

\paragraph{2~Pass} B does not sign any nonce sent by A

\subsection{Conclusion}
\begin{itemize}
    \item Challenge/response protocols are secure when used with care
    \item Because of the reflection attack, it is better to use an
        asymmetric form of the exchange
    \item General idea to get a good scheme: \textbf{2 nonces +
        identities}
\end{itemize}
