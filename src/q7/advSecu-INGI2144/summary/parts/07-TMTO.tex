\subsection{One-way Function} 

\begin{center}
    Function $h: A \rightarrow B$: 
    \begin{tabular}{l}
        easy to compute \\
        but hard to invert.
    \end{tabular}
\end{center}

One-way function are used to store password. To break password, an exhaustive
search can be used in two different ways.
\begin{table}[ht!]
    \centering
    \begin{tabular}{|c|c|c|}
        \hline
        & Online exhaustive search & Precalculated exhaustive search \\
        \hline
        Computation & $ |A|:=N $ & 0 \\
        Storage & 0 & N \\
        Precalculation & 0 & N \\
        \hline
    \end{tabular}
\end{table}

\subsection{Hellman's TMTO}
Hellman table is used to pre-calculation phase to speed up the online attack.

\subsubsection{Pre-calculation}
%$$ H: A \to B \quad the\ hash\ function $$
%$$ R: B \to A \quad the\ reduction\ function $$
%A number M of chains is generated from a arbitrary value in A\@:

\begin{tabular}{m{6cm}m{8cm}}
    \begin{tikzpicture}
        \node[draw, circle] (A1) {S1};
        \node[draw, circle] (A2) [below=of A1] {};
        \node[draw, circle] (A3) [below=of A2] {};
        \node[draw, circle] (A4) [below=of A3] {E1};

        \node[draw, circle] (B1) [below right=0.5cm and 3cm of A1] {};
        \node[draw, circle] (B2) [below=of B1] {};
        \node[draw, circle] (B3) [below=of B2] {};

        \node [draw, ellipse, fit={(A1) (A2) (A3) (A4)}] (F) {};
        \node [draw, ellipse, fit={(B1) (B2) (B3) }] (FF) {};

        \draw[->] (A1) edge node[above] {h} (B1);
        \draw[->] (B1) edge node[above] {R} (A2);

        \path[->] (A2) edge (B2)
        (B2) edge (A3)
        (A3) edge (B3)
        (B3) edge (A4);

    \end{tikzpicture}
    &
    \begin{eqnarray*}
        \textrm{Hash function } \quad & h &: A \to B \\
        \textrm{Reduction function } \quad & R &: B \to A 
    \end{eqnarray*}
    $$$$
    \begin{tikzpicture}
        \node[draw, circle] (A1) {S1};
        \node[draw, circle] (A2) [right=of A1] {};
        \node[draw, circle] (A3) [right=of A2] {};
        \node[draw, circle] (A4) [right=of A3] {};

        \node[draw, circle] (A5) [right=of A4] {};
        \node[draw, circle] (A6) [right=of A5] {};
        \node[draw, circle] (A7) [right=of A6] {E1};

        \path[bend left, ->] (A1) edge node[above] {h} (A2)
        (A2) edge node[above] {R} (A3)
        (A3) edge node[above] {h} (A4)

        (A5) edge node[above] {h} (A6)
        (A6) edge node[above] {R} (A7);

        \node[draw, circle] (B1) [below=of A1] {S2};
        \node[draw, circle] (B2) [right=of B1] {};
        \node[draw, circle] (B3) [right=of B2] {};
        \node[draw, circle] (B4) [right=of B3] {};

        \node[draw, circle] (B5) [right=of B4] {};
        \node[draw, circle] (B6) [right=of B5] {};
        \node[draw, circle] (B7) [right=of B6] {E2};

        \path[bend left, ->] (B1) edge node[above] {h} (B2)
        (B2) edge node[above] {R} (B3)
        (B3) edge node[above] {h} (B4)

        (B5) edge node[above] {h} (B6)
        (B6) edge node[above] {R} (B7);

    \end{tikzpicture}
    \\
\end{tabular}


\paragraph{Coverage and collisions}

During the generation of the chains, chains may collide between each other, to
avoid that, different reduction function must be used.
\subsubsection{Online attack}
%%TODO ADD from the book, image
The idea is to take the a value $h$ in B, use the reduction function to get in A
and check if the value is contains in the last column of the table match.
If they don't match, use two reduction function and redo the verification.
If the verification fails, redo with 3 reduction function and so on.\ if
no match is found then the attack fails.

When there is a match, redo the chains from the beginning and stop when the
current value is equal to $h$. The password is the value just before $h$
%%TODO false alarms

\subsection{Oechslin Tables}
Oechslin tables also called rainbow table. The main difference compared to
Hellman table is that it uses different reduction function per column such that:

\begin{itemize}
    \item If 2 chains collide in different columns, they don't merge.
    \item If 2 chains collide in same column, merge can be detected.
\end{itemize}

\paragraph{Theorem} The success probability of a table is bounded:
\begin{itemize}
    \item
        Given $t$ and a sufficiently large $N$, the expected maximum number of
        chains per perfect rainbow table without merge is:
        $$ m_{\max}(t)\approx\frac{2N}{t+1} $$
    \item
        Given $t$, for any problem of size $N$, the expected maximum probability of
        success of a single perfect rainbow table is:
        $$ P_{\max}(t)\approx 1 - (1-\frac{2}{t+1})^t $$
        which tends toward $ 1 - e^{-2}\approx 86\% $ when $t$ is large.

        %%TODO have we seen that Average cryptanalysis time ?
\end{itemize}

