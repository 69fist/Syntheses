
\subsection{Choice of the Block Cipher}
\begin{center}
    \begin{tabular}{|c|c|c|}
        \hline
        \textbf{Primitive} & \textbf{Legacy} & \textbf{Future} \\
        \hline
        AES & \textcolor{green!50!black}{v} &
        \textcolor{green!50!black}{v} \\
        Camelia & \textcolor{green!50!black}{v} &
        \textcolor{green!50!black}{v} \\
        \hline
        Three-key-3DS & \textcolor{green!50!black}{v} &
        \textcolor{red!50!black}{x} \\
        Two-key-3DS & \textcolor{green!50!black}{v} &
        \textcolor{red!50!black}{x} \\
        Kasumi & \textcolor{green!50!black}{v} &
        \textcolor{red!50!black}{x} \\
        Blowfish & \textcolor{green!50!black}{v} &
        \textcolor{red!50!black}{x} \\
        \hline
        DES & \textcolor{red!50!black}{x} & \textcolor{red!50!black}{x}\\
        \hline
    \end{tabular}
\end{center}

\subsection{Key Generation}
Some block ciphers suffer from poor keys. For example DES has 16 known poor
(weak and semi-weak) keys. %TODO add definition

\subsection{Mode of Operation} %TODO add schema
Mode of operation is used to define how to encrypt a message whose length
is larger than the cipher block size.
\subsubsection{Electronic codeBook}

\begin{itemize}
    \item Block oriented
    \item Require padding
    \item Do not require initialization vector.
    \item Parallel encryption possible and parallel decryption possible.
    \item Swapping blocks is undetectable
    \item Same plaintext blocks produce same ciphertext blocks
    \item Not suitable for long or strongly structured messages
\end{itemize}

\subsubsection{Cipher Block chaining}

\begin{itemize}
    \item Block oriented
    \item Require padding
    \item Require initialization vector
    \item Parallel encryption not possible but parallel decryption possible
    \item General-purpose block-oriented encryption
\end{itemize}

\subsubsection{Cipher feedback}

\begin{itemize}
    \item Stream-oriented
    \item Do not require padding
    \item Require initialization vector
    \item Parallel encryption not possible but decryption possible (if IV known)
    \item Decryption can be done with the encryption function
    (cannot be used with a PK cipher)
    \item General-purpose stream-oriented encryption
\end{itemize}

\subsubsection{Output feedback}

\begin{itemize}
    \item Stream-oriented
    \item Do not require padding
    \item Require initialization vector
    \item Prepocessing possible for encryption and decryptio (if IV known)
    \item Decryption can be done with encryption function
    (cannot be used with a PK cipher)
    \item Enable preprocessing
\end{itemize}

\subsubsection{Counter}

\begin{itemize}
    \item Stream-oriented
    \item Do not require padding
    \item Do not require initialization vector
    \item Preprocessing possible for encrytion and decryption
    (nonce/counter known)
    \item Decryption can be done with encryption function. (cannot not be used
    with a PK cipher)
    \item Same as OFB but randomizes the encryption with a counter value instead
    of some feedback from previous blocl
    \item Must have a different counter value for every plaintext block.
    \item Enable preprocessing, Random access.
\end{itemize}

\subsection{Initialization Vector}
Required with CBC,CFB,OFB %%TODO complete with note

\subsection{Padding}
\begin{itemize}
    \item Main aim of padding is to fill the last block of plaintext (No
    necessarily the shortest)
    \item Can be eithet bit vs byte-oriented
    \item Required with ECB and CBC
    \item Padding should be checked after decryption
    \item Padding increases the size of the plaintext
\end{itemize}

\subsubsection{ISO/IEC 9797}
\begin{itemize}
    \item Method 1: Append some 0-bits (possibly none) to the end of the message
    to fill the last block of plaintext. Problem is that it brings ambiguity if
    recipient does not know the length of the plaintext.

    \item Method 2: Put a 1-bit at the end of the plaintext, then pad with
    some 0-bits (possibly none) to fill the last block of plaintext. To avoid
    ambiguity, plaintext should always be padded even if the last block is
    already completed.
\end{itemize}

\subsubsection{ANSI X9.23}
Append some 0-bits to the message and the length (in bytes) of the padding.

\subsubsection{ISO/IEC 10126}
Append some random bits to the message and the length (in bytes) of the padding.

\subsubsection{PKCS 7}
Each added byte is the length (in bytes) of the padding.

\subsection{Cipher used as Hash Function}
A block cipher can be used in place of a hash function.
\paragraph{Davies-Meyer} $H_i=E_{m_i}(H_{i-1})\oplus H_{i-1}$ The $m_i$ are
blocks of the padded message to be hashed. This construction leads to fixed
points $E_m(h)\oplus h=h$.
\paragraph{Matyas-Meyer} $h_i = E_{g(H_{i-1})}(m_i)\oplus m_i$
\begin{itemize}
    \item $m_i$ are blocks of the padded message to be hashed.
    \item Function g is just to map the output size of H into the key size.
\end{itemize}


