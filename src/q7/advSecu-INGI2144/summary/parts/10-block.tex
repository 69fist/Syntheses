
\subsection{Choice of the Block Cipher}
\begin{center}
    \begin{tabular}{|c|c|c|}
        \hline
        \textbf{Primitive} & \textbf{Legacy} & \textbf{Future} \\
        \hline
        AES & \textcolor{green!50!black}{v} &
        \textcolor{green!50!black}{v} \\
        Camelia & \textcolor{green!50!black}{v} &
        \textcolor{green!50!black}{v} \\
        \hline
        Three-key-3DS & \textcolor{green!50!black}{v} &
        \textcolor{red!50!black}{x} \\
        Two-key-3DS & \textcolor{green!50!black}{v} &
        \textcolor{red!50!black}{x} \\
        Kasumi & \textcolor{green!50!black}{v} &
        \textcolor{red!50!black}{x} \\
        Blowfish & \textcolor{green!50!black}{v} &
        \textcolor{red!50!black}{x} \\
        \hline
        DES & \textcolor{red!50!black}{x} & \textcolor{red!50!black}{x}\\
        \hline
    \end{tabular}
\end{center}

\subsection{Characteristic block cipher}

\subsubsection{Key Generation}
Use a \textbf{good pseudo-random generator (PRG)}.

\paragraph{Poor keys}
Some block ciphers suffer from poor keys. For example DES has 16 known poor
(weak and semi-weak) keys. 

\begin{itemize}
	\item\textbf{Weak keys} : A key $k$ is weak if $\forall m$, $E_k(E_k(m)) = m$
	\item\textbf{Semi-weak keys}: A pair of key $(k_1,k_2)$ is semi-weak if $\forall m$,
	$E_{k_1}(E_{k_2}(m))=m$
\end{itemize}
E is an encryption algorighm.

\subsubsection{Mode of Operation} 
Mode of operation is used to define how to encrypt a message whose length
is larger than the cipher block size.

\begin{center}
    \begin{tabular}{|m{2.5cm}|m{1.5cm}|m{1.5cm}|m{1.5cm}|m{2cm}|m{2.5cm}|}
        \hline
        & \textbf{ECB}        & \textbf{CBC}        & \textbf{CFB}    & \textbf{OFB} & \textbf{Counter} \\
        \hline
        Oriented                   & Block      & Block      & Stream & Stream
        & Stream\\
        Padding                    & Require    & Require    & No & No & No\\
        IV                         & No         & Require    & Require &
        Require & No\\
        \hline
        Decryption done with encryption & No & No & Yes & Yes & Yes \\
        \hline
        Parallelization encryption & Yes        & No         & No &
        Preprocessing if IV know & Preprocessing if nonce/counter know\\
        \hline
        Parallelization decryption & Yes        & Yes        & if IV know &
        Preprocessing if IV know & Preprocessing if nonce/counter know\\
        \hline
    \end{tabular}
\end{center}

\tikzstyle{vertex}=[draw,fill=black!15,circle,minimum size=20pt,inner sep=0pt]
\tikzstyle{encrypt}=[draw,fill=black!15,rectangle,minimum size=20pt,inner sep=0pt]

%TODO: decryption
\begin{itemize}
    \item \textbf{Electronic codeBook (ECB)}: Not suitable for long or strongly structured messages
        \begin{itemize}
            \item Swapping blocks is undetectable
            \item Same plaintext blocks produce same ciphertext blocks
        \end{itemize}

        \begin{tabular}{cm{1.5cm}c}
        \begin{tikzpicture}
            \newcommand{\n}{3}
            \foreach \nr in {1, ..., \n}{
                \node (C\nr) at ({(\nr-\n)*2}, 0) {$C_\nr$};
                \node (M\nr) at ({(\nr-\n)*2}, 2) {$M_\nr$};
                \node (E\nr)[encrypt] at ({(\nr-\n)*2},1) {$E$};
                \node (K\nr) at ({(\nr-\n)*2-1},1) {$K$};

                \draw[->,very thick] (M\nr) -- (E\nr);
                \draw[->,very thick] (K\nr) -- (E\nr);
                \draw[->,very thick] (E\nr) -- (C\nr);
            }
        \end{tikzpicture}
        & &
        \begin{tikzpicture}
            \newcommand{\n}{3}
            \foreach \nr in {1, ..., \n}{
                \node (C\nr) at ({(\nr-\n)*2}, 0) {$M_\nr$};
                \node (M\nr) at ({(\nr-\n)*2}, 2) {$C_\nr$};
                \node (E\nr)[encrypt] at ({(\nr-\n)*2},1) {$D$};
                \node (K\nr) at ({(\nr-\n)*2-1},1) {$K$};

                \draw[->,very thick] (M\nr) -- (E\nr);
                \draw[->,very thick] (K\nr) -- (E\nr);
                \draw[->,very thick] (E\nr) -- (C\nr);
            }
        \end{tikzpicture}
        \end{tabular}

    \item \textbf{Cipher Block chaining (CBC)}: General-purpose block-oriented encryption

        \begin{tabular}{cm{1.5cm}c}
        \begin{tikzpicture}
            \newcommand{\n}{3}
            \foreach \nr in {1, ..., \n}{
                \node (C\nr)            at ({(\nr-\n)*2},0) {$C_\nr$};
                \node (D\nr)[encrypt]   at ({(\nr-\n)*2},1.5) {$E$};
                \node (x\nr)       at ({(\nr-\n)*2},2.5) {$\oplus$};
                \node (M\nr)            at ({(\nr-\n)*2},3.5) {$M_\nr$};

                \node (K\nr)            at ({(\nr-\n)*2-1},1.5) {$K$};

                \draw[->,very thick] (D\nr) -- (C\nr);
                \draw[->,very thick] (x\nr) -- (D\nr);
                \draw[->,very thick] (M\nr) -- (x\nr);

                \draw[->,very thick] (K\nr) -- (D\nr);
            }

            \foreach \nr in {2, ..., \n}{
                \pgfmathtruncatemacro{\tmp}{\nr-1}
                \draw[->,very thick] ({(\n-\tmp)*-2},0.75) -|
                ({(\n-\tmp)*-2+0.75},0.75) |- ({(\n-\tmp)*-2+0.75},2) |- (x\nr);
            }

            \node (IV) at ({\n*-2+1},2.5) {$IV$};
            \draw[->, very thick] (IV) -- (x1);
        \end{tikzpicture}
        & &
        \begin{tikzpicture}
            \newcommand{\n}{3}
            \foreach \nr in {1, ..., \n}{
                \node (C\nr)            at ({(\nr-\n)*2},0) {$M_\nr$};
                \node (D\nr)[encrypt]   at ({(\nr-\n)*2},2) {$D$};
                \node (x\nr)            at ({(\nr-\n)*2},1) {$\oplus$};
                \node (M\nr)            at ({(\nr-\n)*2},3.5) {$C_\nr$};

                \node (K\nr)            at ({(\nr-\n)*2-1},2) {$K$};

                \draw[->,very thick] (M\nr) -- (D\nr);
                \draw[->,very thick] (D\nr) -- (x\nr);
                \draw[->,very thick] (x\nr) -- (C\nr);

                \draw[->,very thick] (K\nr) -- (D\nr);
            }

            \foreach \nr in {2, ..., \n}{
                \pgfmathtruncatemacro{\tmp}{\nr-1}
                \draw[->,very thick] ({(\n-\tmp)*-2},2.75) -|
                ({(\n-\tmp)*-2+0.75},2.75) |- ({(\n-\tmp)*-2+0.75},2) |- (x\nr);
            }

            \node (IV) at ({\n*-2+1},1) {$IV$};
            \draw[->, very thick] (IV) -- (x1);
        \end{tikzpicture}
        \end{tabular}

    \item \textbf{Cipher feedback (CFB)}: General-purpose stream-oriented encryption

        \begin{tabular}{cm{1.5cm}c}
        \begin{tikzpicture}
            \newcommand{\n}{3}
            \foreach \nr in {1, ..., \n}{
                \node (C\nr)            at ({(\nr-\n)*2},0) {$C_\nr$};
                \node (x\nr)   at ({(\nr-\n)*2},1.5) {$\oplus$};
                \node (D\nr)[encrypt]       at ({(\nr-\n)*2-1},1.5) {$E$};
                \node (K\nr)            at ({(\nr-\n)*2-1},2.5) {$K$};
                \node (M\nr)            at ({(\nr-\n)*2},3) {$M_\nr$};

                \draw[->,very thick] (x\nr) -- (C\nr);
                \draw[->,very thick] (M\nr) -- (x\nr);
                \draw[->,very thick] (K\nr) -- (D\nr);
                \draw[->,very thick] (D\nr) -- (x\nr);
            }

            \foreach \nr in {2, ..., \n}{
                \pgfmathtruncatemacro{\tmp}{\nr-1}
                \draw[->,very thick] ({(\n-\tmp)*-2}, 0.75) --
                ({(\n-\nr)*-2-1}, 0.75) -- (D\nr);
            }

            \node (IV) at ({\n*-2+1},0.50) {$IV$};
            \draw[->, very thick] (IV) -- (D1);

        \end{tikzpicture}
        & &
        \begin{tikzpicture}
            \newcommand{\n}{3}
            \foreach \nr in {1, ..., \n}{
                \node (C\nr)            at ({(\nr-\n)*2},0) {$M_\nr$};
                \node (x\nr)            at ({(\nr-\n)*2},1.5) {$\oplus$};
                \node (D\nr)[encrypt]   at ({(\nr-\n)*2-1},1.5) {$E$};
                \node (K\nr)            at ({(\nr-\n)*2-1},0.5) {$K$};
                \node (M\nr)            at ({(\nr-\n)*2},3) {$C_\nr$};

                \draw[->,very thick] (x\nr) -- (C\nr);
                \draw[->,very thick] (M\nr) -- (x\nr);
                \draw[->,very thick] (K\nr) -- (D\nr);
                \draw[->,very thick] (D\nr) -- (x\nr);
            }

            \foreach \nr in {2, ..., \n}{
                \pgfmathtruncatemacro{\tmp}{\nr-1}
                \draw[->,very thick] ({(\n-\tmp)*-2}, 2.4) --
                ({(\n-\nr)*-2-1}, 2.4) -- (D\nr);
            }

            \node (IV) at ({\n*-2+1},2.50) {$IV$};
            \draw[->, very thick] (IV) -- (D1);

        \end{tikzpicture}
        \end{tabular}

    \item \textbf{Output feedback (OFB)}: Enable preprocessing

        \begin{tabular}{cm{1.5cm}c}
        \begin{tikzpicture}
            \newcommand{\n}{3}
            \foreach \nr in {1, ..., \n}{
                \node (C\nr)            at ({(\nr-\n)*2},0) {$C_\nr$};
                \node (D\nr)   at ({(\nr-\n)*2},1) {$\oplus$};
                \node (x\nr)[encrypt]       at ({(\nr-\n)*2},2.5) {$E$};
                \node (M\nr)            at ({(\nr-\n)*2},3.5) {$K$};

                \node (K\nr)            at ({(\nr-\n)*2-1},1) {$M_\nr$};

                \draw[->,very thick] (D\nr) -- (C\nr);
                \draw[->,very thick] (x\nr) -- (D\nr);
                \draw[->,very thick] (M\nr) -- (x\nr);

                \draw[->,very thick] (K\nr) -- (D\nr);
            }

            \foreach \nr in {2, ..., \n}{
                \pgfmathtruncatemacro{\tmp}{\nr-1}
                \draw[->,very thick] ({(\n-\tmp)*-2},1.75) -|
                ({(\n-\tmp)*-2+0.75},1.75) |- ({(\n-\tmp)*-2+0.75},2) |- (x\nr);
            }

            \node (IV) at ({\n*-2+1},2.5) {$IV$};
            \draw[->, very thick] (IV) -- (x1);

        \end{tikzpicture}
        & &
        \begin{tikzpicture}
            \newcommand{\n}{3}
            \foreach \nr in {1, ..., \n}{
                \node (C\nr)            at ({(\nr-\n)*2},0) {$M_\nr$};
                \node (D\nr)   at ({(\nr-\n)*2},1) {$\oplus$};
                \node (x\nr)[encrypt]       at ({(\nr-\n)*2},2.5) {$E$};
                \node (M\nr)            at ({(\nr-\n)*2},3.5) {$K$};

                \node (K\nr)            at ({(\nr-\n)*2-1},1) {$C_\nr$};

                \draw[->,very thick] (D\nr) -- (C\nr);
                \draw[->,very thick] (x\nr) -- (D\nr);
                \draw[->,very thick] (M\nr) -- (x\nr);

                \draw[->,very thick] (K\nr) -- (D\nr);
            }

            \foreach \nr in {2, ..., \n}{
                \pgfmathtruncatemacro{\tmp}{\nr-1}
                \draw[->,very thick] ({(\n-\tmp)*-2},1.75) -|
                ({(\n-\tmp)*-2+0.75},1.75) |- ({(\n-\tmp)*-2+0.75},2) |- (x\nr);
            }

            \node (IV) at ({\n*-2+1},2.5) {$IV$};
            \draw[->, very thick] (IV) -- (x1);

        \end{tikzpicture}
        \end{tabular}

    \item \textbf{Counter (CTR)}: Enable preprocessing and random access

        \begin{itemize}
            \item Same as OFB but randomizes the encryption with a
                \textbf{counter} value instead of some feedback from previous block
            \item Must have a different counter value for every plaintext block.
        \end{itemize}
        With CTR, a counter value should \textbf{never be used} twice with 
        the same secret key, accross all block of all messages
        \begin{itemize}
        	\item Method 1 : A global counter can be used for all messages, so the
        	counter is never re-initialized (cycling period must be long)
        	\item Method 2 : The counter is reinitialized for every message but a 
        	nonce is append to the counter. 
        \end{itemize}

        \begin{tabular}{cm{1.5cm}c}
        \begin{tikzpicture}
            \newcommand{\n}{3}
            \foreach \nr in {1, ..., \n}{
                \node (C\nr)            at ({(\nr-\n)*2},0) {$C_\nr$};
                \node (D\nr)   at ({(\nr-\n)*2},1) {$\oplus$};
                \node (x\nr)[encrypt]       at ({(\nr-\n)*2},2) {$E$};
                \node (M\nr)            at ({(\nr-\n)*2},3) {$nonce||\nr$};

                \node (K\nr)            at ({(\nr-\n)*2-1},1) {$M_\nr$};

                \draw[->,very thick] (D\nr) -- (C\nr);
                \draw[->,very thick] (x\nr) -- (D\nr);
                \draw[->,very thick] (M\nr) -- (x\nr);

                \draw[->,very thick] (K\nr) -- (D\nr);
            }

            \node (IV) at ({\n*-2+1},2) {$IV$};
            \draw[->, very thick] (IV) -- (x1);

        \end{tikzpicture}
        & &
        \begin{tikzpicture}
            \newcommand{\n}{3}
            \foreach \nr in {1, ..., \n}{
                \node (C\nr)            at ({(\nr-\n)*2},0) {$M_\nr$};
                \node (D\nr)   at ({(\nr-\n)*2},1) {$\oplus$};
                \node (x\nr)[encrypt]       at ({(\nr-\n)*2},2) {$E$};
                \node (M\nr)            at ({(\nr-\n)*2},3) {$nonce||\nr$};

                \node (K\nr)            at ({(\nr-\n)*2-1},1) {$C_\nr$};

                \draw[->,very thick] (D\nr) -- (C\nr);
                \draw[->,very thick] (x\nr) -- (D\nr);
                \draw[->,very thick] (M\nr) -- (x\nr);

                \draw[->,very thick] (K\nr) -- (D\nr);
            }

            \node (IV) at ({\n*-2+1},2) {$IV$};
            \draw[->, very thick] (IV) -- (x1);

        \end{tikzpicture}
        \end{tabular}
\end{itemize}

\subsubsection{Initialization Vector}
\begin{center}
    \begin{tabular}{|m{2.5cm}|m{1.5cm}|m{1.5cm}|m{1.5cm}|}
        \hline
        &  \textbf{CBC}        & \textbf{CFB}    & \textbf{OFB} \\
        \hline
        Confidentiality & No & No & No \\
        \hline
        Integrity & Required & Yes & Yes \\
        \hline
        Freshness & Yes & Yes & Yes\\
        \hline 
        Unpredictability & Required & Not issue & Not issue\\
        \hline
    \end{tabular}
\end{center}

\begin{itemize}
    \item \textbf{Confidentiality}
        Confidentiality is not needed, the IV simply act as a $C_0$, making IV confidential
        $\to$ other cipher block confidential.

    \item \textbf{Integrity}
        If IV is changed, part of the plaintext can be modified. With CBC, you can choose
        what the modified the first plaintext block will be. But CFB and
        OFB, plaintext is random

    \item \textbf{Freshness}
        Freshness must be ensured.
        \begin{itemize}
            \item \textbf{CBC:} If the an IV is used twice (wih same key), then the 2 plaintext
                provide same ciphertext up to the first difference between the two plaintext.

            \item \textbf{CFB:} If IV is used twice and if $M_1,C_1$ is known, then
                $M_2$ can be retrieved (given $C_2$). This true up to the first difference up
                to the first difference between the two plaintext.
                \begin{eqnarray*}
                    C_1 & = & M_1 \oplus E(IV)\\
                    C_2 & = & M_2 \oplus E(IV) \\
                    C_1 \oplus C_2 & = & M_1 \oplus M_2
                \end{eqnarray*}
            \item \textbf{OFB:} Same problem as CFB but all the blocks can be retrieved
                if the IV is used twice.
        \end{itemize}

    \item \textbf{Unpredictability}
        \begin{itemize}
            \item For CFB and OFB, as far as we know predicatbility is not an issue.
            \item With CBC, an adversary who has an idea about the plaintext M can
                use the encryptor as an oracle to check her guess G.
        \end{itemize}
        \begin{center}
            \begin{tabular}{m{4cm}m{8cm}}
                \begin{tikzpicture}
                    \node at (0,0) (xor) {$\oplus$};
                    \node [above = 0.5cm of xor] (msg) {M};
                    \node [left = 0.5cm of xor] (iv) {IV};
                    \node [below = 0.5cm of xor,encrypt] (e) {E};
                    \node [left = 0.5cm of e] (key) {K};
                    \node [below = 0.5cm of e] (cipher) {C};

                    \draw[->,very thick] (iv) -- (xor);
                    \draw[->,very thick] (msg) -- (xor);
                    \draw[->,very thick] (key) -- (e);
                    \draw[->,very thick] (xor) -- (e);
                    \draw[->,very thick] (e) -- (cipher);
                \end{tikzpicture}&
                    \begin{itemize}
                        \item Input : IV,C
                        \item Output : Guess G
                        \item Assumption : Adversary has access
                            to an encryptor oracle.
                    \end{itemize}
                \end{tabular}
            \end{center}
            If the IV is predictable, $IV_G$ is the prediction. Advesary request the oracle
            to encrypt $G\oplus IV \oplus IV_G$
            \begin{align}
                E(G\oplus IV \oplus IV_G) &= E((G\oplus IV \oplus IV_G ) \oplus IV_G) &&\text (CBC\ Mode) \notag\\
                                            &= E(G\oplus IV) \notag
            \end{align}
            If $E(G\oplus IV) = C \to G = M$.

        \item \textbf{IV generation}
            \begin{itemize}
                \item Using a counter ensure freshness
                \item Using a encrypted counter ensure freshness \textbf{and} unpredictability
                    (same key as the one to encrypt message)
            \end{itemize}
    \end{itemize}

\subsubsection{Padding}
\begin{itemize}
    \item Main aim of padding is to fill the last block of plaintext (No
        necessarily the shortest)
    \item Can be either bit vs byte-oriented
    \item Required with ECB and CBC
    \item Padding should be checked after decryption
    \item Padding increases the size of the plaintext
\end{itemize}

\paragraph{Methods}

\begin{itemize}

    \item \textbf{ISO/IEC 9797}
        \begin{itemize}
            \item Method 1: Append some 0-bits (possibly none) to fill
                the last block.
                
                $\Rightarrow$ \textbf{Ambiguous} because trailing 0-bits
                of the message cannot be distinguished from the padding!
        \end{itemize}
                \begin{center}
                    \begin{tabular}{|c|c|c|}
                        \cline{1-1} \cline{3-3}
                        DD DD DD DD DD DD DD DD & & DD DD
                        \textcolor{red!50!black}{00 00 00 00 00 00} \\
                        \cline{1-1} \cline{3-3}
                    \end{tabular}
                \end{center}
        \begin{itemize}
            \item Method 2: Put a 1-bit at the end of the plaintext, then pad with
                some 0-bits (possibly none) to fill the last block of plaintext. 
                
                $\Rightarrow$ To avoid ambiguity, plaintext should
                always be padded even if the last block is already
                completed.
        \end{itemize}
                \begin{center}
                    \begin{tabular}{|c|c|c|}
                        \cline{1-1} \cline{3-3}
                        DD DD DD DD DD DD DD DD & & DD DD
                        \textcolor{green!50!black}{80}
                        \textcolor{red!50!black}{00 00 00 00 00} \\
                        \cline{1-1} \cline{3-3}
                    \end{tabular}
                \end{center}
                $$80_{HEX} = 10000000_{BIN}$$

    \item \textbf{ANSI X9.23}: Append some 0-bits to the message and the
        length (in bytes) of the padding.
                \begin{center}
                    \begin{tabular}{|c|c|c|}
                        \cline{1-1} \cline{3-3}
                        DD DD DD DD DD DD DD DD & & DD DD
                        \textcolor{red!50!black}{00 00 00 00 00}
                        \textcolor{green!50!black}{06}\\
                        \cline{1-1} \cline{3-3}
                    \end{tabular}
                \end{center}

    \item \textbf{ISO/IEC 10126}: Append some random bits to the message
        and the length (in bytes) of the padding.
                \begin{center}
                    \begin{tabular}{|c|c|c|}
                        \cline{1-1} \cline{3-3}
                        DD DD DD DD DD DD DD DD & & DD DD
                        \textcolor{red!50!black}{9F 3B 45 10 23}
                        \textcolor{green!50!black}{06}\\
                        \cline{1-1} \cline{3-3}
                    \end{tabular}
                \end{center}

    \item \textbf{PKCS 7}: Each added byte is the length (in bytes) of the padding.
                \begin{center}
                    \begin{tabular}{|c|c|c|}
                        \cline{1-1} \cline{3-3}
                        DD DD DD DD DD DD DD DD & & DD DD
                        \textcolor{red!50!black}{06 06 06 06 06 06}\\
                        \cline{1-1} \cline{3-3}
                    \end{tabular}
                \end{center}
\end{itemize}

\subsection{Cipher used as Hash Function}
A block cipher can be used in place of a hash function.

\subsubsection{Davies-Meyer} 
$$H_i=E_{m_i}(H_{i-1})\oplus H_{i-1}$$ 

\begin{itemize}
    \item The $m_i$ are blocks of the padded message to be hashed. 
    \item[$\Rightarrow$] This
        construction leads to fixed points $E_m(h)\oplus h=h$.

        The Merkle-Damgard strengthening is used to avoid fixed points: the
        bit-length of the message is appended to its end.
\end{itemize}

\subsubsection{Matyas-Meyer} 
$$h_i = E_{g(H_{i-1})}(m_i)\oplus m_i$$

\begin{itemize}
    \item $m_i$ are blocks of the padded message to be hashed.
    \item Function $g$ is just to map the output size of $H$ into the key size.
\end{itemize}

