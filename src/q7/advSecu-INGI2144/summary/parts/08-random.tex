\subsection{Introduction}
\begin{description}
    \item[Random Numbers] Random numbets are sequence of statistically
    independent and unbiased numbers.
    \item[Random Bits] Random bits are a sequence of statistically independent
    and unbiased binary digits.
    \item[Uniform distribution] All the numbers in a given range must occur
    equally often.
    \item[Independence] Given the knowledge of the all previous numbers
    generated, you can not predict the next number.
\end{description}

\subsubsection{Entropy}
The word entropy is used to describe a measure of randomness, a description
of how hard a value is to guess.
\begin{description}
    \item Let us assume that $X$ is a discrete random variable on the sample
    space $ \Omega_n = \{\omega_1,\omega_2,\ldots,\omega_n\} $ with the
    probability distribution $P=\{ p_i,1 \leq i \leq n \}$. The entropy $H(X)$
    of X is defined by:
    $$ H(X)= - \sum_{i=1}^{n} p_i \log_{2}p_i $$
\end{description}

\paragraph{Common Uses} Randomness is needed for key generation, nonces,salt,
\ldots

\subsection{Generating Randomness}
\begin{description}
    \item[Random bit Generator] A random bit generator is a device or algorithm
    which outputs a sequence of statistically independent and unbiased binary
    digits.
\end{description}

\paragraph{Note} A random integer in the interval [0;n] can be obtained by
generating a random bit sequence of length $ \lfloor \log{}{n} \rfloor + 1 $
and converting it to an integer.

\subsection{Hardware Random Number Generators}
The source of entropy come the software (time,process ID, user) or the hardware
(mechanical noises,electrical noises).

\subsubsection{Random Number Generators}
\begin{description}
    \item[Non-deterministic RNGs/True RNGs] Each bit produced by a HRNG comes
    from the observation of an unpredictable physical process.
    \item[Deterministic RNGs/PRNG] Each bit is produced by a deterministic
    algorithm properly initialized.
\end{description}

\subsection{PRNGs in Theory}
\begin{description}
    \item[Pseudo Random Bit Generator] A PRBG is a deterministic algorithm
    which given a truly random binary sequence of length k, outputs a binary
    sequence of length $l \gg k $ which 'appears' to be random. The input is
    the \textbf{seed}, while the output is the psedorandom bit sequence.
    \item[PRNG-FSM] A PRNG is a FSM defined by: $x_{t+1} = f(x_t)$.\ f is the
    transition function and $x_0$ is the seed.
    \item[Period] The period T of the PRNG is the smallest integer such that:
    $x_{t+T} = x_t$
\end{description}

\subsubsection{Middle-square Method}
To generate a sequence of n-digit pseudorandom numbers:
\begin{itemize}
    \item A n-digit seed is created and squared
    \item Middle n digits of the result are output and refeed the process
\end{itemize}
Example wiht n = 6
\begin{tikzpicture}
    \node (firstseed){847209};
    \node (textseed)[below=of firstseed,yshift=1cm] {seed};
    \node (result)[below=of textseed]{717\textcolor{red}{763089}681};
    \node (seedsquare) [below=of result,yshift=1cm]{$seed^2$};
    \node (result2)[below=of seedsquare]{763089};

    \draw[->] (textseed) -- (result);
    \draw[->] (seedsquare) -- (result2);
\end{tikzpicture}

\begin{itemize}
    \item For a generator of n-digits numbers, the period cannot be longer than
    $10^n$.
    \item If the middle n digits are all zeroes, the generator then outputs
    zeroes forever.
    \item If the first half of the middle n digits are all zeroes, the
    subsequent values decrease to zero.
    \item There exists short periods.
\end{itemize}

\subsubsection{Linear Congruential Generator}
\begin{description}
    \item[Linear Congruential Generator] A LCG produces a pseudorandom sequence
    of numbers $x_1,x_2,x_3,\ldots$ according to the linear recurrence:
    $$ x_n = a\times x_{n-1} + b \quad \mod{m}\quad with\ n \geq 1 $$
    The integers a (multiplier), b (increment), and m (modulus) are parameters
    which characterize the generatorn while $x_0$ is the (secret) seed.
\end{description}

\subsubsection{Blum-Blum-Shub Generator}
\begin{itemize}
    \item Let $m=p\times q$ where p and q are both primes congruent to 3 modulo
    4, and $ \gcd{\phi(p)}{\phi(q)} $ is small.
    \item Let $x_0$ be a seed.
    \item We iterate $x_{n+1} := x^2_n\ \mod{m}$
    \item We output the parity bit of $x_n$ (n>0).
    %TODO Comment
\end{itemize}

\subsubsection{ANSI X9.17}

\begin{itemize}
    \item Input: a random (and secret) 64-bit seed $x_0$, integer m, and a
    DES E-D-E encryption key k.
    \item Output: m pseudorandom 64-bit string $x_1,x_2,\ldots,x_m$
    \begin{enumerate}
        \item $T_t = E_k(time())$
        \item $o_t = E_k(x_t \oplus T_t)$
        \item $x_{t+1} = E_k(T_t \oplus o_t)$
    \end{enumerate}
\end{itemize}

\subsubsection{LFSR}
\begin{figure}[ht!]
    \centering
    \begin{tikzpicture}
        \node [shape = rectangle](b3) {b3};
        \node [shape = rectangle, right=of b3] (b2) {b2};
        \node [shape = rectangle, right=of b2] (b1) {b1};
        \node [shape = rectangle, above=of b1] (xor) {$\oplus$};
        \node [shape = rectangle, right=of b1] (b0) {b0};
        \node [above of=b0](xoranchorr){};
        \node [above of=b3](xoranchorl){};

        \draw [->] (b3) -- (b2);
        \draw [->] (b2) -- (b1);
        \draw [->] (b1) -- (b0);
        \draw [->] (b1) -- (xor);
        \draw (b0) -- (xoranchorr);
        \draw [->] (xoranchorr) -- (xor);
        \draw (xor) -- (xoranchorl);
        \draw [->] (xoranchorl) -- (b3);
    \end{tikzpicture}
\end{figure}


