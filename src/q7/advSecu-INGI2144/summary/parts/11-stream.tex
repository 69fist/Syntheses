\subsection{Basics}
\begin{table}
    \centering
    \begin{tabular}{c|c|c}
        Primitive & Legacy & Future\\
        \hline
        Rabbit & Yes & Yes\\
        SNOW3G & Yes & Yes\\
        \hline
        Trivium & Yes & No\\
        SNOW2.0 & Yes & No\\
        \hline
        RC4 & No & No\\
    \end{tabular}
\end{table}

\paragraph{One-Time Pad} $C_i = M_i \oplus K_i\quad for\ i=1,2,3,\ldots$
\begin{itemize}
    \item Key must be generated randomly
    \item Key must be at least as long as the message
    \item Key must be used once
\end{itemize}
A stream ciphers deal with these drawbacks
\begin{itemize}
    \item Key!= Keystream, keystream is generated pseudo-randomly
    \item Key is shorter than the message
    \item Key cannot be used more than once: An IV is appended to the key to
    counter this problem.
\end{itemize}
As with block ciphers, stream ciphers do not ensure integrity (Associativity
and commutativity)

\begin{itemize}
    \item Synchronized mode
    \begin{itemize}
        \item Use a single long stream during the session
        \item A and B initialize the stream with the shared key
        \item A uses the first $|m_A|$ bits of the stream to encrypt message
        $m_A$
        \item B uses the next $|m_B|$ bits of the stream to encrypt message
        $m_B$
    \end{itemize}

    \item Unsynchronize mode
    \begin{itemize}
        \item The keystream is reinitialized for each new packet
        \item Initialization depends on two inputs k and IVi\@.
        \item k secret but constant key
        \item IV non-secret but changing initial vector
    \end{itemize}
\end{itemize}

\subsubsection{Linear Feedback Shift Registers}
\paragraph{Linear feedback shift register} LFSR\footnote{see section random}
of length L consists of L
stages numbered 0,1,\ldots,L-1, each capable of storing one bit and having one
input and one output; and a clock which controls the ovements of data.

During each init of time the following operations are performed:
\begin{itemize}
    \item The content of stage 0 is outputed and forms part of the output
    sequence.
    \item The content of stage i is moved to stage i-1 for each i,
    $1\leq i\leq L-1$
    \item The new content of stage L-1 is the feedback bit $s_j$ which is
    calculated by adding together modulo 2 the previous contents of a fixed
    subsets of stages 0,1,\ldots,L-1.
\end{itemize}
LFSR should never be used by itself as keystreamn generator but they have the
advantage of having low implementation costs. To destroy the linearity:
\begin{itemize}
    \item A nonlinear combining function on the outputs of several LFSRs
    \item A nonlinear filtering function on the contents of a single LFSR
    \item The output of one (or more) LFSRs to control the clock of one
    (or more) other LFSRs.
\end{itemize}

\subsection{RC4}
Use a secret key of length from 1 to 256 bytes.

\subsection{Conclusion}
Using a stream cipher is tricky as the security of stream ciphers is not
well-established, therefore it should be used carefully.


