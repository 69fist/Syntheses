\documentclass[en]{../../../eplsummary}

\usepackage{enumerate}
\usepackage{pifont}
\usepackage{diagbox}
\usepackage[normalem]{ulem}
\newcommand{\cmark}{\ding{51}}%
\newcommand{\xmark}{\ding{55}}%

\newcommand{\norm}[1]{\left\|#1\right\|}

\DeclareMathOperator{\epi}{epi}
\DeclareMathOperator{\conv}{conv}
\DeclareMathOperator{\dom}{conv}

\hypertitle{Optimization models and methods}{7}{INMA}{2471}
{Beno\^it Legat\and M\'elanie Sedda}
{François Glineur}

\section{Reminder}
\subsection{Convex set}
\begin{mydef}[Convex set]
  A set $X$ is convex if for all $x,y \in X$ and $\lambda \in [0,1]$,
  $\lambda x + (1-\lambda) y \in X$.
\end{mydef}

\begin{myprop}
  Let $X,Y$ be convex set.
  The intersection $X \cap Y$ and the cartesian product $X \times Y$ are convex
  but the union $X \cup Y$ is \emph{not} necessarily convex.
\end{myprop}

\subsection{Convex function}
\begin{mydef}[Epigraph]
  The epigraph of a function is defined as
  \[ \epi(f) = \{(x,t) \in \Rn \times \R | f(x) \leq t\}. \]
\end{mydef}

\begin{mydef}[Convex function]
  A function $f$ is convex iff $\epi(f)$ is convex.
\end{mydef}

\begin{myprop}
  Let $f$ and $g$ be convex functions.
  $\max(f,g)$ is convex but $\min(f,g)$ is \emph{not} necessarily convex.
  \begin{proof}
    \begin{align*}
      \epi(\max(f,g))
      & = \{(x,t) \in \Rn \times \R : \max(f(x),g(x)) \leq t\}\\
      & = \{(x,t) \in \Rn \times \R : f(x) \leq t \land g(x) \leq t\}\\
      & = \{(x,t) \in \Rn \times \R : f(x) \leq t\} \cap \{(x,t) \in \Rn \times \R : g(x) \leq t\}\\
      & = \epi(f) \cap \epi(g)\\
      \epi(\min(f,g))
      & = \{(x,t) \in \Rn \times \R : \min(f(x),g(x)) \leq t\}\\
      & = \{(x,t) \in \Rn \times \R : f(x) \leq t \lor g(x) \leq t\}\\
      & = \{(x,t) \in \Rn \times \R : f(x) \leq t\} \cup \{(x,t) \in \Rn \times \R : g(x) \leq t\}\\
      & = \epi(f) \cup \epi(g).
    \end{align*}
  \end{proof}
\end{myprop}

\begin{myprop}
  If $f$ is convex and $g$ is linear, $f \circ g$ is convex.
\end{myprop}

\section{Model Building}
\subsection{Building linear models}
\subsubsection{Objective function}
\begin{myprop}[Linear objective]
  \label{prop:linobj}
  Every optimization problem can be modeled with a linear objective.
  \begin{proof}
    Let
    \begin{align*}
      \min_{x \in \Rn} f(x)\\
      x & \in X
    \end{align*}
    be a modelisation of the problem.
    It can be reformulated as
    \begin{align*}
      \min_{x \in \Rn} t\\
      f(x) & \leq t\\
      x & \in X.
    \end{align*}
  \end{proof}
\end{myprop}

\begin{myrem}[Convex]
  If the original model was convex, $f$
  is convex and therefore, $f(x) \leq t \equiv \epi(f)$ defines % TODO define epi(f)
  a convex set.
  The intersection of convex sets is convex so the
  reformulation is convex too!
\end{myrem}

\begin{myrem}[Piecewise linear]
  If $f$ was piecewise linear, the reformulation is linear.
\end{myrem}

\subsubsection{Monotone transformation}
We can apply strictly monotone transformation to equations and/or the objective
(changing the inequality and/or the $\min$ in $\max$ if it is strictly decreasing).
The reformulated model is equivalent but can become convex or even linear !

\begin{myexem}[Geometric optimization]
  The problem
  \begin{align*}
    \min_{r \geq 0} r_1 r_2 r_3 r_4 r_5\\
    r_1r_2r_3 & \geq 1.05\\
    r_2r_3r_4 & \geq 1.10\\
    r_3r_4r_5 & \geq 1.15
  \end{align*}
  can be reformulated (since $\log$ is strictly increasing) as
  \begin{align*}
    \min_{r \geq 0} \log(r_1) + \log(r_2) + \log(r_3) + \log(r_4) + \log(r_5)\\
    \log(r_1) + \log(r_2 + \log(r_3) & \geq \log(1.05)\\
    \log(r_2) + \log(r_3 + \log(r_4) & \geq \log(1.10)\\
    \log(r_3) + \log(r_4 + \log(r_5) & \geq \log(1.15)
  \end{align*}
  which is linear in $x_i \eqdef \log(r_i)$.
\end{myexem}

\subsubsection{D.E.A Data Envelopment Analysis}
Let's analyse the non-convex problem
\begin{align*}
  \min_x \frac{c^T x + d}{f^T x + g}\\
  Ax & \leq b
\end{align*}
under the asumption $f^Tx+g > 0$ for all $x$ such that $Ax \leq b$.

Charnes and Cooper in \cite{charnes1985foundations} describes
how to reformulate this as a convex problem.
We introduce $t > 0$ and $y \in \in \mathbb{R}^n$
such that $x = y/t$.
The problem becomes
\begin{align*}
  \min_{y \in \mathbb{R}^n,t > 0} \frac{c^T y + dt}{f^T y + gt}\\
  Ay & \leq bt
\end{align*}
which is \emph{homogeneous} since for $\lambda > 0$,
$(y,t)$ and $(\lambda y, \lambda t)$ have the same objective
and $(y,t)$ is admissible iff $(\lambda y, \lambda t)$ is admissible).

We can therefore suppose that $f^T y + g t = 1$
(if not, we take $\lambda = \frac{1}{f^Ty + gt}$ and
$(y,t) \leftarrow (\lambda y, \lambda t)$), we have
\begin{align*}
  \min_x c^T y + dt\\
  Ay & \leq bt.
\end{align*}

\subsection{Building convex model}

\subsubsection{Convexification}
Let's suppose that we have a general (see property~\ref{prop:linobj}) problem
\begin{align*}
  \min_x c^T x\\
  x & \in X
\end{align*}

Its admissible domain is not necessarily convex but we can make it so while
keeping the \emph{same} optimal value.

\begin{mydef}[Convex hull]
  The convex hull $\conv(X)$ of a set $X$ is the smallest convex set containing $X$.
  It can be defined as the set of of all convex combinations of the points of $X$
  \[ \left\{\sum_{i=1}^k \lambda_ix_i | x_1, \ldots, x_k \in X \lambda \sum \lambda_i = 1 \land \lambda_i \geq 0\right\}. \]
\end{mydef}
From the simple observation
\begin{align*}
  c^T\left(\sum_{i=1}^k \lambda_ix_i\right)
  & = \sum_{i=1}^k \lambda_i c^Tx_i\\
  & \geq \min_i c^Tx_i
\end{align*}
we see that the convexified problem has the same optimal solution but its optimal
set is also a superset of the optimal set of the original problem.

Sadly this is very hard to do in general even if it is easy in some case.

\subsubsection{Strict convexity}
\begin{mydef}[Strict convexity]
  $f$ is strictly iff
  \[ f(\lambda x + (1-\lambda)y) < \lambda f(x) + (1-\lambda)f(y) \]
  for all $0 < \lambda < 1$ and for all $x,y$.
\end{mydef}

\begin{mytheo}
  The problem
  \begin{align*}
    \min_x f(x)\\
    x & \in X
  \end{align*}
  with $f$ stricly convex and $X$ convex has at most 1 optimal solution.
\end{mytheo}

If $\lap f(x)$ is positive definite for all $x$ then $f$ is strictly convex
but it is \emph{not} necessary.
For example $x^4$ is strictly convex but $12x^2$ is 0 for $x=0$ and
is therefore only positive semi-definite.

An usual trick to make a convex problem strictly convex is convexification
\begin{mytheo}[Regularization]
  If $f$ is convex then $f + \mu\|x\|^2$ is strictly convex for all $\mu > 0$.
\end{mytheo}

\subsubsection{Convex maximization}
Let's first notice that since a linear function is both convexe and concave,
a linear problem is both a convex minimizatoin \emph{and} a convex maximization.
It has therefore the properties of both.

\begin{mydef}[Extreme point]
  Let $X$ be a convex set.
  The point $x$ is said to be extreme if it does is not inside any segment formed
  by two distinct points of $X$.
\end{mydef}

\begin{mytheo}
  If the problem $\max_{x \in X} f(x)$ with $f$ and $X$ convex has an optimal solution,
  one of the optimal solution is an extreme point of $X$.
\end{mytheo}
This is a property that we already knew for linear optimization.
Now we see that it is actually a property of a bigger class of problems.

\subsubsection{Quasiconvexity}
\begin{mydef}[Quasiconvex]
  A function $f$ is quasiconvex if for all $\alpha$,
  its sub-level set
  \[ L_\alpha(f) = \{x \in \dom f | f(x) \leq \alpha\} \]
  is convex for all $\alpha$.
\end{mydef}

If $f$ is convex, $L_\alpha(f)$ is also convex for all $\alpha$
but the converse is not true (e.g. $\sqrt{|x|}$).

If $f$ is convex and $g$ is concave, $\frac{f}{g}$ is quasiconvex on $\{x | g(x) > 0\}$.
We see that $\frac{f}{g} \leq \alpha$ is equivalent to
$f - \alpha g \leq 0$ which is a convex set since $f - \alpha g$ is convex.
Here we see that it only works for $\alpha \geq 0$, since if $\alpha < 0$, $-\alpha g$ is concave.

\paragraph{Solving procedure}
If the objective $f$ is quasiconvex and the admissible set $X$ is convex then
we can find the optimal solution using bisection.
If the convex set
\begin{align*}
  f(x) & \leq \alpha\\
  x & \in X
\end{align*}
is empty, $\alpha$ is a lower bound.
Otherwise it is an upper bound.

Checking whether it is non-empty can be done by solving a convex problem
such as
\begin{align*}
  \min 0\\
  f(x) & \leq \alpha\\
  x & \in X
\end{align*}

% TODO relation restriction

\section{Alternative et duality}
\subsection{Theorems of the alternative}
\begin{mytheo}[Farkas' lemma]
  Only one of the two following systems has a solution
  \begin{align*}
    Ax & \leq b & A^T\lambda & = 0\\
       & & \lambda & \geq 0\\
       & & b^T\lambda & < 0.
  \end{align*}
  Only one of the two following systems has a solution
  \begin{align*}
    Ax & = b & A^T\lambda & \geq 0\\
     x & \geq 0 & b^T\lambda & < 0.
  \end{align*}
  Only one of the two following systems has a solution
  \begin{align*}
    Ax & = b & A^T\lambda & = 0\\
       & & b^T\lambda & \neq 0.
  \end{align*}
\end{mytheo}

\subsection{Duality and certificates}
Let's consider the problem primal (P)
\begin{align*}
  \max b^Ty\\
  Ay & \leq c
\end{align*}

If we want to prove that the optimal objective is $\theta$,
we have to find a proof that
\[ Ay \leq c \Rightarrow b^Ty \leq \theta. \]
Let's analyse the LHS
\begin{align*}
  a_1 y & \leq c_1\\
  \vdots \quad  & \quad\quad \vdots\\
  a_m y & \leq c_m
\end{align*}
For $x \geq 0$, this system implies the following system
\begin{align*}
  x_1 a_1 y & \leq x_1 c_1\\
  \vdots \quad  & \quad\quad \vdots\\
  x_m a_m y & \leq x_m c_m
\end{align*}
and also
\begin{align*}
  \sum_{i=1}^m x_m a_m y & \leq \sum_{i=1}^m x_m c_m\\
  x^T A y & \leq x^T c.
\end{align*}

We see that $\lambda$ is a certificate for $\theta$ if
\begin{align*}
  A^T x & = b\\
  c^T x & = \theta\\
  x & \geq 0.
\end{align*}

Of course, finding a certificate just gives us an upper bound.
Finding the certificate for the lowest upper bound is the optimization problem
\begin{align*}
  \min c^T x\\
  A^T x & = c\\
  x & \geq 0
\end{align*}
which is called the \emph{dual} (D) of (P).

Of course if we find an admissible couple $(y,x)$ for which $b^T y = \theta$ and
$c^T x = \theta$ we are sure that $\theta$ is the optimal value.
In this case $x$ is called the optimality certificate of $y$.

\subsubsection{Weak and strong duality}
\begin{mytheo}[Weak duality]
  For all admissible solution $y$ of the primal and all admissible solution $x$ of the dual,
  \[ b^Ty \leq c^Tx. \]
\end{mytheo}

\begin{mytheo}[Strong duality]
  Let $y^*$ be an optimal solution for (P).
  Then there exists an (optimal\footnote{The fact that this solution is optimal is a consequence of the weak duality}) solution $x^*$ to the problem (D) such that
  \[ c^Tx^* = b^Ty^*. \]
\end{mytheo}

\begin{mycorr}
  The different possibilites are summarized below
  \begin{center}
    \begin{tabular}{l|c|c|c}
      \diagbox{P}{D} & impossible & finite & unbounded\\
      \hline
      impossible & \checkmark & \xmark & \cmark\\
      \hline
      finite & \xmark & \cmark & \xmark\\
      \hline
      unbounded & \cmark & \xmark & \xmark\\
    \end{tabular}
  \end{center}
  \begin{itemize}
    \item If (P) is unbounded, (D) is impossible.
    \item If (D) is unbounded, (P) is impossible.
    \item If (P) is finite, (D) is finite.
    \item If (D) is finite, (P) is finite.
    \item \sout{If (P) is impossible, (D) is unbounded.}
    \item \sout{If (D) is impossible, (P) is unbounded.}
  \end{itemize}
  $\checkmark$ means that it is possible but is unstable
  in the sense that for a problem for which the dual and the primal is
  impossible, modifying a little bit $A$, $b$ or $c$ makes it fall in another category.

  In practice, we are never in this category except if the problem has
  been crafted on purpose.
\end{mycorr}

% TODO Analyse de sensibilité

\subsection{Robust optimization}
Let consider a linear problem
\begin{align*}
  \max b^T y\\
  a_i^Ty & \leq c_i & \forall i, a_i \in \mathcal{A_i} = \{a | C_ia \leq d_i\}.
\end{align*}

Each constraint is equivalent to
\[ \max_{a_i \in \mathcal{A}_i} y^Ta_i \leq c_i. \]
The LHS is a linear optimization problem in $a_i$ (not $y$ !).
By strong duality, we know that it is equivalent to
\begin{align*}
  \min d_i^Tx & \leq c_i\\
  C_i^T x & = y\\
  x & \geq 0
\end{align*}
or simply asking that there exists an $x$ such that
\begin{align*}
  d_i^Tx & \leq c_i\\
  C_i^T x & = y\\
  x & \geq 0
\end{align*}
but asking for the existence of such $x$ is done
by adding these 3 constraints in the model.

\section{Conic optimization}
For a cone $K$,
we define the parial order $\succeq_K$ as
\[ y \preceq x \iff x \succeq_K y \iff x - y \succeq_K \iff x - y \in K. \]
It is a \emph{partial} order since there can be $x,y$ such that
neither $x \succeq y$ nor $x \preceq y$.

\begin{mydef}[Convex cone]
  A convex cone is a set $K$ such that
  \begin{itemize}
    \item if $x \in K$ and $\lambda \in \mathbb{R}_+$, $\lambda x \in \preceq_K$
      (i.e. if $x \preceq_K y$, $\lambda x \preceq_K \lambda y$);
    \item $x, y \in K$, $x + y \in K$
      (i.e. if $x_1 \preceq_K y_1$ and $x_2 \preceq_K y_2$, $x_1 + x_2 \preceq y_1 + y_2$).
  \end{itemize}
\end{mydef}
It is interesting to compare it with the definition of a vector space.
The difference is that here we impose that $\lambda \geq 0$.
If we allowed $\lambda < 0$, it would mean that if $x \preceq y$, $-x \preceq -y$ and that wouldn't work.

\begin{mydef}[Regular cone]
  A regular cone is a convex cone $K$ such that
  \begin{itemize}
    \item $K$ is closed;
    \item $K$ has a non-empty interior;
    \item $K \cap (-K) = \{0\}$ (i.e. $K$ is \emph{pointed}).
  \end{itemize}
\end{mydef}
The fact that $K$ is pointed means that $x \preceq_K 0 \iff 0 \preceq_K -x$.
For example, the cone $\{(x,y) \in \mathbb{R}^2 | x \geq 0\}$ is not pointed.

\begin{mydef}[Lorentz cone]
  The Lorentz cone $\mathbb{L}^n$ is defined as
  \[ \mathbb{L}^n \eqdef \left\{(x_0, x_1, \ldots, x_n) \in \mathbb{R}^{n+1} | x_0 \geq \|(x_1, \ldots, x_n)\|_2 = \sqrt{x_1^2 + \cdots + x_n^2}\right\}. \]
\end{mydef}

\begin{myexem}[Ellipsoid]
  An ellipsoid is defined as $\{x|(x-c)^TE(x-c) \leq 1\}$ wher $c$ is the center
  and $E \succ 0$.
  Using Cholesky factorization, we have $E = LL^T$ and the condition becomes
  \begin{align*}
    \sqrt{(x-c)^TE(x-c)} & \leq 1\\
    \sqrt{(x-c)^TLL^T(x-c)} & \leq 1\\
    \|L^T(x-c)\|_2 & \leq 1\\
    \begin{pmatrix}
      1\\
      L^T(x-c)
    \end{pmatrix}
    & \in \mathbb{L}^n.
  \end{align*}
\end{myexem}

\biblio
%http://www2.isye.gatech.edu/~nemirovs/Lect_IPM.pdf

\annexe

\section{Preuve dualité forte}
Soient les problèmes
\begin{align*}
(P) \max b^T y&\\
A^Ty &\leq c\\
\end{align*}

\begin{align*}
(D) \min c^T x&\\
Ax &= b\\
x &\geq 0\\
\end{align*}

\paragraph{Théorème}
Si (P) admet une solution optimale $y^*$ alors (D) admet une solution
(optimale) $x^*$ telle que $b^T y^* = c^T x^*$.

\paragraph{Preuve\\}
\begin{enumerate}[a.]
  \item Par hypothèse, (P) est admissible donc le système $A^Ty \leq c$ admet
    des solutions. Le théorème de l'alternative implique donc que le
    système
    \[
      \left\{%
        \begin{array}{l}
          Ax=0\\
          c^Tx < 0\\
          x \geq 0
        \end{array}%
      \right.
    \]
    est impossible.\\

  \item Soit $\theta$ un paramètre réel. Considérons le système (I-$\theta$)
    \[
      \left\{%
        \begin{array}{l}
          b^T y \geq \theta \\
          A^T y \leq c
        \end{array}%
      \right.
    \]
    Ce système affirme qu'il existe au primal (P) une solution admissible
    de valeur supérieure ou égale à $\theta$, en d'autres termes $val(P) \geq
    \theta$.\\

  \item
    Ce système (I-$\theta$) peut se réécrire
    \[
      \left\{%
        \begin{array}{l}
          A^T y \leq c\\
          -b^T y \leq -\theta
        \end{array}%
      \right.
    \]
    ou encore
    $$\left( \begin{array}{c}
        A^T\\
        -b^T
    \end{array} \right) y \leq \left( \begin{array}{c}
        c\\
        -\theta
    \end{array} \right),$$
    dont l'alternative est (on note $x$ le vecteur multiplicateur des
    contraintes $A^T y \leq c$ et $\mu$ le multiplicateur scalaire de la dernière
    contrainte $-b^T y \leq -\theta$)
    \[
      \left\{%
        \begin{array}{l}
          Ax - b \mu = 0\\
          c^Tx - \theta \mu < 0\\
          x \geq 0\\
          \mu \geq 0
        \end{array}%
      \right.
    \]
    qu'on nommera (II-$\theta$).\\

  \item
    Dans le système (II-$\theta$), on peut prendre sans perte de généralité $\mu=1$.
    En effet, si $\mu=0$ est solution, on obtient
    \[
      \left\{%
        \begin{array}{l}
          Ax = 0\\
          c^T<0\\
          x \geq 0,
        \end{array}%
      \right.
    \]
    qui
    est pourtant impossible (voir point a.).
    Si $\mu>0$, on peut multiplier $x$ et $\mu$ par une constante ($1/\mu$ en
    l'occurrence) pour obtenir une autre solution admissible où $\mu=1$.
    Par conséquent, on a en réalité une alternative entre le système
    (I-$\theta$) et le système (II'-$\theta$) suivant
    \[
      \left\{%
        \begin{array}{l}
          Ax = b\\
          c^T x < \theta\\
          x \geq 0.
        \end{array}%
      \right.
    \].\\

  \item
    Le système
    \[
      \left\{%
        \begin{array}{l}
          Ax = b\\
          c^T x < \theta \\
          x \geq 0
        \end{array}%
      \right.
    \]
    affirme qu'il existe au dual (D) une solution de valeur inférieure à $\theta$, soit $val(D) <
    \theta$.\\

  \item
    Nous avons donc une alternative entre les deux propositions
    $val(P) \geq \theta$
    et
    $val(D) < \theta$
    (c'est ici qu'au cours j'avais conclu directement que $val(P)=val(D)$ ;
    l'argument qui suit est plus explicite).
    Si on choisit une valeur de $\theta$ égale à $b^T y^* + \epsilon$
    (où $\epsilon$ est un paramètre strictement positif et $b^T y^*$ est la valeur
    optimale du primal),
    on voit que la proposition $val(P) \geq \theta$ est clairement impossible,
    donc la proposition $val(D) < \theta$ est vraie.
    On a donc $val(D) < b^T y^* + \epsilon$ et ce pour tout $\epsilon>0$. En prenant la
    limite $\epsilon \rightarrow 0$ on voit qu'au final il faut forcément que $val(D) \leq b^T
    y^*$.\\

  \item
    Comme par ailleurs on savait que $val(D) \geq b^T y^*$ par dualité faible,
    on a bien établi l'égalité
    $val(D) = b^T y^*$, et donc l'absence de saut de dualité entre les
    valeurs optimales du primal et du dual.\\

    \textbf{NB}: On n'a pas formellement établi l'existence de la solution optimale
    duale $x^*$, mais seulement le fait que le coût optimal du dual est borné
    (et égal au coût optimal du primal). Cependant on sait (cf. cours
    d'opti 1) que tout problème linéaire atteint sa valeur optimale en au
    moins une solution, ce qui permet bien de conclure. A noter que cette
    propriété n'est pas forcément valable en optimisation convexe
    non-linéaire (considérer par exemple le problème
    \begin{align*}
      \min_x \frac{1}{x+1}&\\
      x \geq 0).
    \end{align*}
\end{enumerate}

\section{Optimisation de portefeuille}

Soient $n$ actifs dont les rendements moyens sont $r_i$ ($i=1,...,n$) et soit
une somme $S$ à investir. Le gain moyen maximal s'obtient en optimisant
\begin{align*}
\max \sum_i r_i x_i &\\
\sum x_i &= S\\
x &\geq 0,
\end{align*}
où $x_i$ est la somme à investir dans l'actif $i$. La solution optimale de ce problème consiste à placer l'entièreté de la somme $S$ dans l'actif possédant le rendement maximal (parmi tous les $r_i$). En cas d'ex-aequo on peut répartir arbitrairement. \\

L'inconvénient de cette stratégie est qu'elle est très risquée. Une
mesure naturelle du risque est la variance du portefeuille, qui vaut
précisément $x^T C x$, où $C$ est la matrice de variance-covariance
(estimée) des différents rendements. Markowitz suggère de maximiser une combinaison du gain moyen et de la
variance, selon
\begin{align*}
\max \sum_i r_i - \rho x^T C x &\\
\sum x_i &= S\\
x \geq 0
\end{align*}
(et il suggère d'analyser la "frontière" des solutions obtenues en
faisant varier $\rho$ entre $0$ et $+\infty$).\\

Une formulation conique s'obtient facilement en introduisant une
variable $v$ pour la variance
\begin{align*}
\max \sum_i r_i - \rho v &\\
\sum x_i &= S\\
 x &\geq 0\\
x^T C x &\leq v
\end{align*}
et la dernière contrainte $x^T C x \leq v $ s'exprime aisément à l'aide du
cône de Lorentz (cf. ensemble S9 de la séance - $C$ est bien semidéfinie
positive).

\subsection{Formulation puissance $3/2$}

Pour aller plus loin, voici comment modéliser une puissance $3/2$. Plus
précisément, on va montrer comment modéliser la contrainte $x^{3/2} \leq y$
pour une variable $x$ supposée positive (on vérifie facilement que c'est
un ensemble convexe, car épigraphe de la fonction convexe $x^{3/2}$).\\

On écrit les deux contraintes
$x^2 \leq y z$ et $z^2 \leq x$
qui sont toutes les deux exprimables à l'aide du cône de Lorentz (cf. séance).
On voit qu'elles sont équivalentes à $x^4 \leq y^2 z^2$ et $z^2 \leq x$ qui
reviennent à $x^4 \leq y^2 x$
qui revient bien à $x^3 \leq y^2$ ou $x^{3/2} \leq y$. On peut alors modéliser $||v||^{3/2}$ en posant $||v||\leq x$, ou même passer
à $(v^T C v)^{3/2}$.\\

A l'aide de cette technique on peut "modérer" l'influence de la
variance dans la fonction objectif (modulo quelques étapes de
reformulation que je ne détaille pas ici).

\section{Preuve que l'algorithme de suivi de chemin fonctionne}
Pour rappel, on a le problème initial $(P)$
\begin{align*}
\min_x c^T x&\\
x &\in X
\end{align*}
et le problème $(P_\mu)$ perturbé par la barrière $F(x)$, pour un paramètre $\mu>0$
$$\min_x \frac{c^T x}{\mu} + F(x)  =  \min_x f_\mu(x).$$
Le pas de Newton pour le problème perturbé s'écrit
$$n_\mu(x) = - (\nabla^2 f_\mu(x))^{-1} \nabla f_\mu(x)$$
et la mesure de proximité correspondante
$$\delta_\mu(x) = || n_\mu(x) ||_x = || \nabla f_\mu(x) ||^*_x.$$\\

Plaçons-nous à l'itération $k$ et posons
\begin{align*}
x &= x_k; \\
x^+&= x_{k+1};\\
\mu &= \mu_k; \\
\mu^+&= \mu_{k+1} = (1-\theta) \mu.
\end{align*}
On sait qu'on a au départ $\delta_\mu(x) < \tau$ et on voudrait prouver qu'après la mise à jour de $\mu$ à $\mu^+$ et le pas de Newton on a $\delta_{\mu^+}(x^+) < \tau$ avec donc $x^+ = x + n_{\mu^+}(x)$. On analyse d'abord la quantité intermédiaire $\delta_{\mu^+}(x)$ (après mise à jour de $\mu$, mais avant le pas de Newton). On a
$$\delta_{\mu^+}(x) = \norm{n_{\mu^+}(x)}_x = \norm{\nabla f_{\mu^+}(x)}^*_x = \norm{\frac{c}{\mu^+} + \nabla F(x)}^*_x.$$
Or, on peut écrire
$$\frac{c}{\mu^+} + \nabla F(x) = \frac{\mu}{\mu^+} \left(\frac{c}{\mu} + \nabla F(x)\right) + \left(1-\frac{\mu}{\mu^+}\right) \nabla F(x),$$
d'où
$$\delta_{\mu^+}(x) \leq \frac{\mu}{\mu^+} \norm{\frac{c}{\mu} + \nabla F(x)}^*_x + \left |1-\frac{\mu}{\mu^+}\right| \norm{\nabla F(x)}^*_x.$$
Notons que l'hypothèse que $F$ est une b.a.c. de paramètre $\nu$ fournit $\norm{\nabla F(x)}^*_x \leq \sqrt{\nu}$, tandis que $\delta_\mu(x) < \tau$ fournit la borne $\norm{\frac{c}{\mu} + \nabla F(x)}^*_x < \tau$ ce qui permet d'écrire, en utilisant le fait que $\frac{\mu}{\mu^+} = \frac{1}{1-\theta}$ et $ |1-\frac{\mu}{\mu^+}|=\frac{\theta}{1-\theta}$ (attention : la valeur absolue est nécessaire sur $1-\frac{\mu}{\mu^+}$, je l'avais oubliée au tableau), $$\delta_{\mu^+(x)} < \frac{\tau + \sqrt{\nu} \theta}{1-\theta}.$$

A ce stade, on utilise les valeurs concrètes $\tau=\frac{1}{4}$ et $\theta = \frac{1}{16\sqrt{\nu}}$. On utilise également le fait que $\nu \geq 1$ quelle que soit la barrière, d'où $1-\theta \geq 1-\frac{1}{16} = \frac{15}{16}$, d'où on tire $$\delta_{\mu^+}(x) < \frac{\frac{1}{4} + \frac{1}{16}}{\frac{15}{16}} = \frac{5/16}{15/16} = \frac{1}{3}.$$ On voit donc que la proximité au chemin central se détériore un peu $\left(\frac{1}{4} \rightarrow \frac{1}{3}\right)$ mais reste assez bonne avec la nouvelle valeur (plus petite) de $\mu$.\\

Maintenant, on applique un pas de Newton $n_{\mu^+}(x)$ (pour minimiser $f_{\mu^+}$), et le résultat de convergence quadratique nous indique que $$\delta_{\mu^+}(x^+) \leq \frac{\delta_{\mu^+}(x)}{(1-\delta_{\mu^+}(x))^2 } < \left(\frac{1/3}{2/3}\right)^2 = \frac{1}{4},$$ ce qui permet de montrer que le nouvel itéré a bien restauré sa proximité initiale ($\tau=\frac{1}{4}$) au par rapport au (nouveau point du) chemin central.\\

L'algorithme fonctionne donc correctement, et tout itéré $x_k$ vérifiera $\delta_{\mu_k}(x_k) < \tau$.

\end{document}
