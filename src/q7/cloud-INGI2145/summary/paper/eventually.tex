
Data inconsistency in large-scale reliable distributed systems has to be tolerated for two reasons: improving read and write performance under highly concurrent conditions; and handling partition cases where a majority model would render part of the system unavailable even though the nodes are up and running.

Whether or not inconsistencies are acceptable depends on the client application. In all cases the developer needs to be aware that consistency guarantees are provided by the storage systems and need to be taken into account when developing applications.



\begin{description}
    \item[Strong consistency] After the update completes, any subsequent access (by A, B, or C) will return the updated value.

    \item[Weak consistency] The system does not guarantee that subsequent accesses will return the updated value. A number of conditions need to be met before the value will be returned. The period between the update and the moment when it is guaranteed that any observer will always see the updated value is dubbed the inconsistency window.

    \item[Eventual consistency] This is a specific form of weak consistency; the storage system guarantees that if no new updates are made to the object, eventually all accesses will return the last updated value. If no failures occur, the maximum size of the inconsistency window can be determined based on factors such as communication delays, the load on the system, and the number of replicas involved in the replication scheme. The most popular system that implements eventual consistency is DNS (Domain Name System). Updates to a name are distributed according to a configured pattern and in combination with time-controlled caches; eventually, all clients will see the update.

    \item[The eventual consistency has a number of variations that are important to consider:]

    \item[Causal consistency] If process A has communicated to process B that it has updated a data item, a subsequent access by process B will return the updated value, and a write is guaranteed to supersede the earlier write. Access by process C that has no causal relationship to process A is subject to the normal eventual consistency rules.

    \item[Read-your-writes consistency] This is an important model where process A, after it has updated a data item, always accesses the updated value and will never see an older value. This is a special case of the causal consistency model.

    \item[Session consistency] This is a practical version of the previous model, where a process accesses the storage system in the context of a session. As long as the session exists, the system guarantees read-your-writes consistency. If the session terminates because of a certain failure scenario, a new session needs to be created and the guarantees do not overlap the sessions.

    \item[Monotonic read consistency] If a process has seen a particular value for the object, any subsequent accesses will never return any previous values.

    \item[Monotonic write consistency] In this case the system guarantees to serialize the writes by the same process. Systems that do not guarantee this level of consistency are notoriously hard to program.

\end{description}

A number of these properties can be combined. For example, one can get monotonic reads combined with session-level consistency. From a practical point of view these two properties (monotonic reads and read-your-writes) are most desirable in an eventual consistency system, but not always required. These two properties make it simpler for developers to build applications, while allowing the storage system to relax consistency and provide high availability.


