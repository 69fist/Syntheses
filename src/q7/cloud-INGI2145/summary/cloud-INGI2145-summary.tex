\documentclass[en]{../../../eplsummary}

\hypertitle{cloud-INGI2145}{7}{INGI}{2145}
{Houtain Nicolas}
{Canini Marco}

\section{Introduction}

\subsection{Computing at scale}

Cloud need for scalability because modern application require
huge amounts of processing and data. Cluster (room-sized) and datacenter 
(building-sized) can provide the resources needed.
They are composed of \textbf{rack} which is a aggregation of storage devices, many
nodes and switch to connect nodes together. Unfortunately, they are not perfect.
\begin{enumerate}
    \item Difficult to dimension because they must be provisioning for the peak load
    \item Expensive in hardware invest, expertise (ex: special software) and maintenance
    \item Difficult to scale because adding new machines is not easy
\end{enumerate}

\subsection{Cloud computing}

\begin{center}
\textit{Cloud computing is a model for enabling convenient,  
on-­demand network access to a shared pool of configurable  
computing resources (e.g., networks, servers, storage,  
applications, and services) that can be rapidly provisioned  
and released with minimal management effort or  
service provider interaction.}
\end{center}

$\rightarrow$ On-demand self service, Broad network access, Resource pooling, Rapid elasticity, Measured service.

\subsubsection{Model}
Cloud computing is a business models where everything is a service:
\begin{itemize}
    \item SaaS: Software as a service
    \item PaaS: Platform as a service
    \item IaaS: Infrastructure as a service
\end{itemize}

\subsubsection{Types}
There also have three types of cloud : 
\begin{itemize}
    \item Public: commercial commercial service open to almost anyone.
    \item Community: shared by several similar organization
    \item Private: shared within a organization
\end{itemize}
In this course we focus on public cloud.

\subsubsection{Applications}

Typically, applications that involve large amounts of computation,  storage,
bandwidth Especially when lots of resources are needed quickly or load varies
rapidly.


\subsubsection{Virtualization}
IS used to simulate multiple physical machine for the consumer with different
capabilities. It's powerful for security and isolation because VM cannot influence
other. In the other hand, performance is hard to predict because other VM run on
the same physical machine.

\subsubsection{Challenge}
\begin{tabular}{m{0.5\linewidth}m{0.5\linewidth}}
\begin{itemize}
    \item Availability
    \item Data lock-in (moving data)
    \item Data confidentiality and auditability 
    \item Data transfer bottlenecks
    \item Performance unpredictability for VM
\end{itemize}
&
\begin{itemize}
    \item Scalable storage
    \item Bugs in large distributed systems
    \item Scaling quickly
    \item Reputation fate sharing
    \item Software licensing
\end{itemize}
\end{tabular}

\section{Design for scale}

A system is scalable if it can easily adapt to  
increased (or reduced) demand.




\section{Paper}



\end{document}
