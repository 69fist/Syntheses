\documentclass[fr,license=none]{../../../eplsummary}

\usepackage{graphicx}
\usepackage{multicol}
\usepackage{float}
\usepackage{booktabs} % for much better looking tables
\usepackage{array} % for better arrays (eg matrices) in maths
\usepackage{paralist} % very flexible & customisable lists (eg. enumerate/itemize, etc.)
\usepackage{verbatim} % adds environment for commenting out blocks of text & for better verbatim

\hypertitle{Gestion des ressources humaines}{7}{LSMG}{2004}
{Nicolas Houtain\and Gorby Nicolas Ndonda Kabasele}{Alain Eraly}

\section{Penser la réalité humaine de l'organisation}

10 principes généraux de sociologie des organisations.

\subsection{La réification}
\begin{description}
    \item[Definition]   :    Illusion   qui   consiste    à   concevoir
l'organisation  comme  une réalité  extérieur  et  supérieur à  ses
membres, susceptible de  poursuivre ses propres intêret  et d'exercer un
pouvoir sur ses membres.
\end{description}

L'organisation n'est  pas un  organisme à part  qui serait  en relation
avec  ses   membres,  les   seules  contraintes  sociales   sont  celles
qu'exercent les hommes (agent sociaux) entre eux.

\paragraph{ } \textit{On lui donne une pseudo réalité, des pouvoirs et
contraintes à une abstraction en oubliant qu'on l'a créé}


\subsection{Dépasser les raisonnements mécaniques} Il ne faut pas voir
les agents composant une organisation comme des êtres passifs soumis au
impulsion  de l'environnement.  (\textit{tel  que une  prime  = plus  de
motivation})

\textsc{Exemple} : culture de l'entreprise.

On est pas dans une logique de cause-effet simpliste (\textit{une
personne n'est pas simplement soumis à un ensemble de ``facteurs''
ayant un impact sur lui})


\subsection{Articuler l'acteur et le système}

Deux grands travers affaiblissent le management:
\begin{enumerate}
\item Centration excessive sur le systeme (agents vu comme des produit passifs de logiques structurelles),
\item Exaltation démesurée de l'acteur (en particulier du dirigeant vu comme omnipotent et omniscient \textit{management heroique}).
\end{enumerate}

\paragraph{Note:} la combinaison voit l'entreprise comme un airbus et le
dirigeant le pilote. (\textit{Cet avion ne doit pas faire partie de
malaisia airlines})

\subsection{Raisonner en termes de dualité}
Deux visions pour l'organisation:
\begin{itemize}
	\item Les actions et les interactions qui s'y passent.
    \item Les élèments constitutif de \textbf{configuration} plus large.
\end{itemize}

Ces deux points de vue sont à adopter pour analyser l'organisation dans
un  sens et  dans l'autre.  (\textit{Va-vient entre  le singulier  et le
globale}).

\paragraph{Systeme}
Conception pour expliquer des relations de contraintes et
d'interdépendance (\textit{sans assigner de réalité propre}). Tout
systeme est structuré et peut être répété.

\paragraph{Structure}
Ensemble des propriétés générales qui font qu'un système social se
reproduit de période en période. (\textit{Le jeu n'existe que parce
que on applique les règles}).

Elles sont \textbf{virtuelles} (\textit{cfr: Ceci est une pipe})

\begin{itemize}
    \item[$\to$] Les actions sont conditionnés par le système mais le
système résulte des actions.
\end{itemize}

\paragraph{Dualité  versus  Dualisme}
\begin{description}
    \item[Dualisme] : coexistence de deux réalité
    \item[Dualité] (\textit{doit etre substituer au dualisme}) : réalité
        unique mais double en soi
\end{description}

La dualité de l'organisation vient des actions et interactions d'une
part et du système globale d'autre part.

\textit{Une entreprise est un ensemble d'humain, elle est donc
inévitablement conditionnée par les propriétés plus générales de la
société}


\subsection{Critique de l'atomisme}

\begin{description}
\item[Conception atomique] :l'atome de la vie sociale et économique $=$ \textit{l'individu psychologique}
\end{description}

Organisation devient alors une agrégation d'atome. Conception
individualiste et rationnelle de l'être humain.

Différents type de conception proche sans être les même.
\begin{enumerate}
    \item \textbf{Conception monadique} :
        \begin{description}
            \item[Postulat] les causes du comportements exclusivement à
            l'intérieur de l'esprit d'un individu (pulsions, habitudes,
            traits de personnalité,...).
            \item[Conséquence] Il suffirait de selectionner le
            candidats le plus compétent pour avoir le meilleur
            \item[Critique] tout être humain existe par son rapport à
            autrui et sa position dans le champs social (ex: \textit{la
            vie intérieur d'un aborigene n'est pas la même qu'un cadre
            américain}).
        \end{description}

        L'entreprise à comme premier produit du travail, le travailleur lui même
        puisque ce qu'ils sont dans l'entreprise les détermine en partie.

        \paragraph{Note:} A cette vision il faut substituer une concetion
        \textbf{interactionniste} où l'unité d'analyse devient l'interaction.

    \item \textbf{Utilitarisme} :
        \begin{description}
            \item[Postulat] Toute action se réduit à la recherche d'intérêt egoiste.
            \item[Critique] L'utilitarisme est la négation même du social, c'est
                évident que tout n'est pas que calcul (même si le calcul est
                inhérent à l'homme est ne peut être nier)
        \end{description}

        \paragraph{Note: } Cette vision normalise l'égoïsme et la recherche du
        profit avant tout. \textbf{cynisme absolu}

    \item \textbf{Rationnalité optimal} :

        \begin{description}
            \item[Postulat] L'individu est un décideur rationnel
            \item[Critique] Negation des processus cognitifs, affectifs et normatifs
        \end{description}

        \paragraph{} L'être humain est capable de rationalité limitée, dans le
        cas où il peut simplifier la réalité. (\textit{Achat de voiture
        avec un budget limité})
\end{enumerate}

\subsection{Perspective interactioniste}
On détermine les actions humaines en les rapportant aux relations
sociales.

\paragraph{Exemple}
Un enqueteur demande à un employé ``Es-tu satisfait?''
\begin{itemize}
    \item Monadique : La réponse est une description de son état
        intérieur
    \item Interactioniste : La réponse est une interaction. Si elle
        répond oui, on va se demande ``Pq construire l'image d'un
        employé satisfait?''
\end{itemize}


\subsection{Passage par la subjectivité de l'acteur}
On part de la subjectivité des agents pour expliquer leur conduite, et
non de la situation définie par l'observateur.
(\textit{Situation vécue pas de la situation objective $\to$ empathie})

\paragraph{ }L'objectivité requiert de partir de la subjectivité des
agents mais pas uniquement (car cette subjectivité est souvent une
vision partielle du système réel que construisent les agents)

(ex: des ramoneur avec un visage sale et un propre, c'est la
communication empathique)

\begin{itemize}
    \item \textbf{Empathie} : Capacité à comprendre l'autre en nous
        plaçant dans sa situation à lui et pas dans la
        notre.
\end{itemize}

\paragraph{Théorme de Thomas}

Si les hommes tiennent une situation pour réelle, même si ce n'est pas
le cas (\textit{par exemple le culte de la nation}), les conséquences
le seront.

L'empathie est une compétence sociale décisive dans la vie du travail.

\subsection{Sous l'ordre hierarchiques les réseaux}
Pour comprendre l'organisation, il faut pas s'intéresser que à
l'organigramme mais aussi s'intéresser au relation
non-hiérarchique, les réseaux.

\begin{description}
    \item[Réseau] Un réseau est un ensemble non hiérarchisé et
largement informel de relations entre des acteurs.

        Ces liens sont évolutifs et souvent peu formalisé.
\end{description}

$\to$ Réseau $\neq$ coalition, les participants n'ont pas tous les même intérêts.

\paragraph{Centralité des dirigeants} dans les réseaux est une dimension
fondamentale de leur pouvoir. (\textit{D'où les efforts pour placer des
fidèles}).

\subsection{Au fondement du social, la logique de don}
Réseaux $\neq$  modèle hierachique qui lui est traversé par les réseaux.

Ce qui structure les réseaux ce n'est pas du donnant-donnant
(\textit{Echange économique}) mais plutôt la
logique du don (\textit{qui est une forme sociale de réciprocité})

Dans le don, la norme de réprocité reste implicite et présuppose la
confiance, qui se crée au fur à mesure. La plus grande de confiance
est la divulgation d'un secret.


\paragraph{ }
Le lien social est fondé sur un triple geste: $<$Donner, Recevoir,
Rendre$>$.

\subsection{La réalité est socialement construite}

La situation est socialement construite au travers d'une variété de
processus de communication. (Exemple lien avec échec)

\begin{itemize}

 \item \textbf{Division du travail et cognition} L'attention sélective
tient sa source dans la limitation cognitif. La vie sociale comme la
division du travail dans l'entreprise détermine les foyers d'attention.
(\textit{Ces différences favorisent les différences d'interpretations})

$\to$ Pluralité de rationalité dans l'entreprises. (\textit{Les
    structures formelles deviennent des structures de raisonnements})

 \item \textbf{Rôle des représentations} L'environnement est abstrait
puisque l'on ne sait pas tout et constitué de représentations
construites et transmises par d'autre agent.
Représentation comme un acte de communication, donc une opération
d'influence sur autrui.

$\to$ La réalité est socialement construite, puisque l'on incorporent des
représenattions d'autrui (télévision, journaux,\ldots)

\begin{itemize}
    \item[Note:] Une représentation permet d'absober une incertitude de l'agent
    (portions de l'environnement comlpexe auquel l'agent n'a pas accès).

    Toutefois en transmettant ces représentation ceux qui les reçoivent les
    considère comme des \textbf{fait bruts} et non des représentations
    sociales.
\end{itemize}

	\begin{description}
		\item[Validation empirique] Confronter la représentation à la réalités des faits (ex:météo)
		\item[Validation sociale] Représentation qu'on ne peut pas confronter à la réalité
		(ex: statistique).
	\end{description}
\end{itemize}

\paragraph{Décision}
La qualité de la décision dépend moins des capacités cognitives des
décideux et d'heuristiques compliquées que de la qualité des
communications dans l'organisation avec son environnement
(\textit{écoute, pluralisme des canaux,\ldots})

\paragraph{Aveuglement}
Certains dirigeant sont aveuglement socialement construit en s'entourant
de collaborateur anxieux de plaire (et filtrant donc les informations
déplaisantes)

\paragraph{Culture de l'entreprise} Ensemble des processus de
construction sociale de la réalité partagée d'une organisation.
\textbf{Le grand problème du managemement c'est de faire croire que
l'ont décrit la réalité alors qu'on l'a préscrit}


\subsection{Conclusion}
On a encore et toujours affaire à des groupes humains et des relations
entre ces groupes.


\section{La gestion des ressource humaines et les fondements de la motivation sociale}

Motivation, implication et satisfaction au travail sont trois notions
clé donnant lieu a une infinité de variantes.

\subsection{Théorie du contenu versus théorie du processus}

\paragraph{Motivation} Engagement de la personne dans son travail qui,
si le contexte le permet, va entrainer un surcroit de performance.
(\textit{Propriété générale de la relation d'un individu à sa situation
de travail et non d'une caractéristique de l'individu})

\paragraph{ }
Deux façon de théoriser la motivation:
\begin{description}
 \item[Théorème contenu: ] Qu'est ce que la motivation?
 \item[Théorème du processus: ] Quelles sont les facteurs qui
influence la motivation?
\end{description}

\subsubsection{Erreurs typiques des practiciens}

\paragraph{Definitions} :
\begin{itemize}
 \item \textbf{Motivation} : tendance à s'investir (\textit{articule
         des facteurs de personnalité avec des facteurs
     interpersonneles et organisationnels})
 \item \textbf{Satisfaction} : représentation de soi au moyen d'une
     comparaison favorable
 \item \textbf{Implication} : Attachement au travail, identification à la
     fonction. Fait état de l'importance que prend le travail par rapport
     aux autres choses de la vie.
\end{itemize}

\paragraph{Erreurs} :
\begin{enumerate}
 \item \textbf{Motivation réduite à phénomène purement individuel} :
     Il suffit de regarder les phénomènes de groupe générateur d'une
     implication forte (foot,\ldots) qu'on appelle \textbf{Team
     spirit}.

     \begin{itemize}
         \item Motivation rôle : Toute personne qui endosse un rôle
             (\textit{souvent porteur de responsabilité}) modifie son
             rapport au monde.

         \item[$\to$] Des attentes de rôle (\textit{= exigences
             propres au rôle}) nous poussent à agir de telle ou telle
             manière.
     \end{itemize}

 \item \textbf{Motivation comme trait de personnalité/culturel} :
     Il suffit de se rendre compte que les individus ont des
     personnalités qui changent et elles sont même influencés par
     l'entreprise.

     \begin{itemize}
         \item[$\to$] On peut donc pas rentrer dans la logique où il
             suffit de recruter ``la bonne personnalité'' pour un
             post
     \end{itemize}

 \item \textbf{Motivation et satisfaction} :

     Pas de corélation, ni de lien de causalité entre satisfaction
     et motivation.

     Les situations de travail ont deux effets distincts :
     \begin{enumerate}
         \item Sur l'engagement
         \item Sur le jugement sur la situation qu'elle construit et
             communique
     \end{enumerate}

     \begin{itemize}
         \item[$\to$] Distinction entre les politiques qui ont pour
             effet d'accroite la satisfaction (\textit{Salaire plus
             élevé que les entreprises de même milieu}) et celle pour
             accroître la motivation.
     \end{itemize}

 \item \textbf{Motivation et implication} :
     Il est clair que l'implication favorise la motivation
     (\textit{puisque via son identification, l'employé considère les
     objectifs de l'entreprise/son département comme ses propres
     objectifs})

     Toutefois, l'implication n'est pas forcément nécessaire à la
     motivation, comme le chercheur en mathématique dont l'unique
     motivation est personnelle (\textit{amour des maths})

 \item \textbf{Déni du principe de nécessité}
     La performance (\textit{en amont de la motivation}) est
     le résultat d'un mélange variables d'intérêt pour la
     tâche et le poste, mais encore de \textbf{contraintes} et de
     possibilités d'action dans le cadre d'une développement de la
     carrière.

     \begin{itemize}
         \item[$\to$] Ce n'est pas un processus spontanné (de l'ordre
             de la passion) et la motivation au travail implique une
             \textbf{nécessité de la performance}
     \end{itemize}

     \textit{Par exemple, le mec qui est sur sa barque en hiver au
     milieu du lac, si il voit qu'elle prend l'eau il sera TRES motivé
     à ramer jusqu'à la rive}

 \item \textbf{Lien mécanique motivation = performance} : Toute
 les activités comportent des dimensions: \begin{enumerate} \item
 Quantitative (volume de travail) \item Qualitative (qualité du
 travail) \end{enumerate}

     La motivation n'influe pas forcément la performance individuelle
     et collective. Toute motivation s'actualise dans un milieu social
     qui peut aussi bien la relayer, l'amplifier ou au contraire
     l'étouffer (ex: \textit{patron qui ne tiens pas compte de
     proposition de cadres motivés}).

     \begin{itemize} \item[$\to$] Performance est un concept dur à
     définir même si il y a une intuition général de l'efficience
     car cette dernière implique de pouvoir tout monétarisé
     (\textit{mais quel valeur pour une formation?}) \end{itemize}

\end{enumerate}

\subsection{Le modèle V.I.E}

3 types de facteurs suceptibles d'influencer la propension à s'engager
dans certains types d'action:

\begin{description}
 \item[Expectation: ] La représentation par l'individu de sa propre
     capacité à engager une action donnée. (\textbf{Moyen, compétence et
     confiance en soi})
     \begin{itemize}
         \item[$\to$] Savoir si il croit qu'il a des chances
             raisonnables d'atteindre l'objectif primaire.
     \end{itemize}

 \item[Instrumentalité: ] La représentation du rapport entre l'action
et le résultat.

Lien raisonnable à sa performance et certains résultats potentiels
associé à cette performance (typiquement la réaction de la
hiérarchie).

 \item[Valence: ] Valeur accordé au résultat. Elle peut être
positif(ex: augmentation de salaire) ou négative (ex: risque de
licenciement)

\end{description}

$M=V\times I \times E, \quad$ si une des variables est à 0 alors la
motivation est à 0.

\subsubsection{Expectation}

\paragraph{\textbf{E $\bf \to$ Em, Em}}
L'expectation peut être diviser en deux catégorie:
\begin{itemize}
    \item \textbf{Expectation de soi} (Es)
	\begin{enumerate}
		\item Ressources affectives (ex:confiance en soi)
		\item Ressources cognitives (ex: langue parlé)
		\item Ressources physique (ex: endurance)
	\end{enumerate}

    \item \textbf{Expectation matérielle} (Em) qui est les moyens matériels
        dont dispose l'individu
	\begin{enumerate}
		\item Ressources financière
		\item Ressources matérielles et technique
		\item Ressource stratégiques (ex: employés disponibles)
	\end{enumerate}
\end{itemize}

\subparagraph{Subjectivité}
Certaines expectation sont vécue subjectivement même si il y a des
éléments objectifs : \textit{Objectivement je ne parle pas chinois
(savoir), subjectivement je ne me sens pas la force de
l'apprendre (savoir-faire)}.

\subparagraph{Confiance en soi}
L'image de soi est partiellement \textit{socialement construite} par
rapport au regard des autres sur nos échec/réussite.

\textit{La réussite favorise l'ambition (cfr boxeur)}


\subparagraph{Expectation facteur de la motivation} Il faut trouver le
juste équilibre pour ne pas démotiver l'employé avec des taches trop
complexe au trop simple (\textit{au dessus ou en dessous de l'expectation})

\paragraph{\textbf{E $\bf \to$ O}}

Il est important pour la motivation de fixer des objectifs
(\textbf{effet de but}). Permet un foyer d'attention, une référence
et pousse à la persévérance. De plus on peut régulièrement mesurer
\textit{l'état d'avancement}.

Il est important que l'employé s'approprie les objectifs et pas qu'on
lui dise \textit{``Voila tes objectifs''}

\begin{itemize}
    \item[$\to$] Le but offre une représentation synthétique des actions à
        entreprendre et permet d'exercer un contrôle effecace sur ses
        propres comportements.
\end{itemize}

Mais ce n'est pas forcément en fixant des objectifs que les gens seront
plus motivés. Les objectifs doivent :
\begin{itemize}
 \item S'inscrire dans l'analyse de fonctions (description de poste)
 \item Etre \textbf{SMART} (\textit{Spécifique, Mesurable,
     Ambitieux, Réaliste, délimiter dans le Temps})
 \item Sous le contôle de l'employé et aisément compréhensible.
 \item Négociés et justifié
 \item Non contradictoire et prioritiser.
 \item Il faut qu'il y ait un dispositif de feedback intermédiaires
 \item Contrôle d'atteinte des objectifs.
 \item Valoriser la réalisation des objectif
\end{itemize}

\paragraph{\textbf{Expectation et GRH}}
Moyen a disposition du GRH pour favoriser l'expectation :

\begin{center}
\begin{tabular}{|m{6cm}|m{8cm}|}
	\hline
	\textbf{Expectation matérielle} & \textbf{Expectation de soi} \\
	\hline
    \begin{enumerate}
        \item Analyse des fonctions : Entretiens régulier comportant la
            fixation d'objectifs ainsi que de feedback
        \item Délégation de pouvoir
        \item Allocation de moyens et gestions du temps
    \end{enumerate}
     &
     \begin{enumerate}
         \item Organisation du travail
         \item Délégation de pouvoir
         \item Politique de mobilité interne (\textit{pour développer
             des nouvelles compétences})
	     \item Politique de formation
	     \item Le type de leadership et le rôle pédagogique joué par
             l'encadrement
	     \item Coaching
	     \item Rôle du groupe (soutient, encoragement,\ldots)
     \end{enumerate}
     \\
	\hline
\end{tabular}
\end{center}

\subsubsection{Instrumentalité}
Une conduite est instrumentale si elle représente pour celui qui
l'accomplit un moyen d'atteindre un résultat.
\begin{itemize}
 \item La probablilité d'obtenir/d'éviter tel résultat en adoptant un
     comportement donné.

     Indice de probabilité : -1 (\textit{action evite un résultat})
     $\to$ +1 (\textit{relation forte entre performance et résultat})
\end{itemize}

La perception du résultat peut être objective ou subjective (dépend
de chaque individu et de sa perception). Cette perception dépend des
caractéristiques objectives du milieu de travail (\textit{recouvre
donc l'ensemble des dispositids qui établissent une liaison entre les
performances des employés et les réactions à ces performances})

\begin{itemize}
    \item[$\to$] L'être humain se compare, il veut que ce soit
        \textbf{équitable}
\end{itemize}

\paragraph{ }L'instrumentalité pose 4 types de \textbf{problèmes}:
 \begin{itemize}
	\item Comment la performance est reconnue et définie a tout les
        niveaux?
	\item Comment evaluer la performance pour pouvoir la sanctionner
        positivement ou négativement?
	\item Chaque employé doit avoir accès à l'information (chacun doit
        connaitre les "règles du jeu").
	\item Dépend de la confiance en la personne (ex: chef sadique qui
        feint une promotion)
 \end{itemize}

Sanctions et gratifications sont deux faces indissoluble de la
motivation.

\paragraph{Intrumentalisation et GRH}
Notons que pour le GRH, c'est bien le \textbf{lien} entre performance et réactions
qui l'interrèsse.

Par exemple :
\begin{itemize}
    \item Promotion, gratification,\ldots si ceux ci prennent en compte
        les performances passés des employés
\end{itemize}

\subparagraph{Conscience des résultats} arrive lorsqu'un système
réinjecte des résultats. Les gens prennent conscience de ce pourquoi ils
travaillent.

\subsubsection{Valence}
La valence est la valeur que représente un résultat pour un individu.
Bien entendu celle ci varie selon les individus.

\begin{enumerate}
    \item \textbf{Valence positive} :
        \begin{itemize}
           \item \textbf{récompense intrinsèques} : émane du travail en
               lui même (\textit{valorisation, intérêt du
               travail,\ldots})
           \item \textbf{récompense extrinsèques} : travail non comme
               une finalités mais un moyen pour obtenir autre chose qui
               n'est pas directement lié à la fonction.
               (\textit{augmentation de salaire, avantages
               sociaux,\ldots})
        \end{itemize}

        La hiérachie influence la valeur symbolique des accomplissemnts
        humains dans l'entreprise et donc la motivation.

    \item \textbf{Valence négative} : par exemple des sanctions ou des
        pressions participe à la motivation.
\end{enumerate}

\subparagraph{Note: } Les revers de la médaille (\textit{comme
augmentation de stress d'une promotion}) sont des valences négatives.

\paragraph{Type de de valence}

\begin{itemize}
    \item affective, morale ou altruiste
    \item individuelle ou collective
    \item intrinsèque ou extrinsèque
\end{itemize}

\paragraph{Valence et GRH}

Le GRH influe sur les valences en choisissant le type de gratification
et de sanctions pratiquées, mais aussi par la valorisation des
résultats et la promotion ou non des formes d'implications collectives.

\subparagraph{Implication} est indispensable pour que l'individu
s'identifie à un groupe et adopte les normes et valeurs constitutives de
ce groupe.

\subsubsection{Systeme V.I.E}
\paragraph{Remarques }:
\begin{enumerate}
   \item On parle de la motivation par rapport à un objectif, mais pas
       la motivation en général.
   \item Le système est incomplet si on se limite à un moment $t$. La
       motivation est changeante dans le temps.
   \item Il peut quand même servir comme base pour un diagnostic
       d'entreprise.
\end{enumerate}

\paragraph{Diminution dans VIE} :
\begin{itemize}
    \item[-] $\searrow$ V : Impunité, plus d'avenir
    \item[-] $\searrow$ I : Mon chef s'en fou, injustice
    \item[-] $\searrow$ E : On m'enlève les moyens, pas d'objectifs,
        pas de confiance en moi
\end{itemize}

\subsection{La démotivation}

\begin{description}
	\item[Démotivation] processus destructeur (\textit{pas uniquement
        une carence}) dans la relation de la personne à son travail.

        \begin{itemize}
            \item[$\bf\neq$] \textbf{non-motivation} (qui peut venir du fait de pas avoir
envie de s'eparpiller.)
        \end{itemize}

        Elle provient souvent de la baisse d'un des facteurs VIE.
\end{description}

\paragraph{Degré d'intensité de la démotivation} :
\begin{itemize}
	\item \textbf{douce} : affecte la valorisation de la situation,
        souvent une baisse de I (ex: injustice) qui provoque un baisse
        de V.

        \textit{``Je suis entrain d'échouer, mais je m'en fiche``},
        \textit{''Je n'ai pas reçu les équipements promis``}

        \begin{itemize}
            \item[$\to$] réaction intentionnelle de défense
        \end{itemize}

    \item \textbf{dure} affecte l'image de l'individu, une baisse de E
        (ex: je suis bon à rien).

        Cette démotivation est difficilement réversible.
\end{itemize}


\section{Justice et équité dans la GRH}

Le sentiment de justice influence V.I.E, et en particulier V,
ce qui a un impact important sur la motivation.

\subsection{Justice organisationnelle}

Il n'est pas suffisant de s'interroger sur la conformité d'une
décision, pratique ou politique (\textit{face à un code juridique ou à
un réglement}) pour établir si elle juste ou pas.

\subparagraph{Exemple} : Je veux que ce que recoivent mes collègues
soient juste avec ce que je reçoit selon mes efforts, sinon je vais me
démotiver

\begin{itemize}
    \item[$\to$] L'instrumentalité n'existe donc pas dans l'absolu mais
        par \textbf{comparaison sociale}
\end{itemize}

\paragraph{3 types de justices} :
\begin{itemize}
\item \textbf{Justice distributive:} Distribution des ressoureces au sens large
    (ex:horaire, compliment,...)
\item \textbf{Justice procédurale:} Perception de la justice des procédures (ex
    Pourquoi je n'ai pas eu la promotion $\to$ voir règle)
\item \textbf{Justice interactionnelle:} La façon dont les décideurs traitent
    leurs subordonnés et respectent leurs engagements.
\end{itemize}


\subsection{Justice distributive}

Fondé sur le principe de proportionnalité des échanges, le
\textbf{mérite}.

\paragraph{Privation relative} est la divergence
entre ce qu'on reçoit et ce qu'on penser qu'on devrait recevoir. Cette
divergence se calcule de toute façon:

\begin{itemize}
	\item \textbf{Temporelle} : Ce qu'on a reçu aujourd'hui par rapport à
        ce qu'on avait reçu avant.
	\item \textbf{Sociale} : Ce qu'on reçoit et ce que les autres ont
        reçus.
\end{itemize}

Il y a sentiment d'inégalité/ d'injustice lorsque
$$\frac{R_s}{C_s}<\frac{R_a}{C_a}$$ où R sont les rétribution, C les
contributions , s pour soi et a pour les autres.


\paragraph{Réaction face à une iniquité favorable} :
On se tait quand on se sent privilégier et on invente une histoire pour
justifier ce privilège

\paragraph{Réaction face à une iniquité défavorable} :
\begin{enumerate}
    \item Diminuer sa contribution (démotivation)
    \item Se plaindre
    \item Arreter de se comparer
    \item Faire un \textit{retrait} où il refoule toute forme
        d'implication au travail
\end{enumerate}

Bref, on essaye de rétablir l'équilibre en ne foutant rien.

\paragraph{ }Ce genre de justice s'applique aussi au sanction. Il faut que ce soit
deux poids deux mesures.


\subsection{Justice procédurale}

\paragraph{ }Face à une situation inéquitable, un individu peut
malgré tout conserver un sentiment de justice s'il a l'impression que
les procédure suivies sont justes.

\paragraph{ } Les personnes ayant un certain contrôle sur la décision
ou le processus ont un plus fort sentiment de justice.

\paragraph{Facteurs pour que les acteurs impliqués trouvent une
procédure juste} :

\begin{itemize}
	\item \textbf{Cohérence:} C'est la même règle pour tout le monde.
	\item \textbf{Impartialité: } Ne tient pas compte des intérêts (économique, sociaux,...).
	\item \textbf{Précision: } Décision fondées sur des infos claires et précises.
	\item \textbf{Adaptabilité: } Possiblités de corriger les décisions inappropriées.
	\item \textbf{Représentativité: } Chaque acteur est écouté et a son mot à dire.
	\item \textbf{Éthique: } Les décisions respectes les valeurs morales de chacun.
\end{itemize}


\subsection{Justice interactionnelle}

Composante de la justice procédurale, s'intèresse à la manière dont les
individus sont traités.

Derrière la justice interractionnel se cache un enjeu fondamentale
de reconnaissance (\textit{ce n'est pas que de la justification des
décisions et la possibilité pour les acteurs de peser sur ces
décisions})


\section{La communication}

Principe même de toute vie organisée. Les organisations ne sont
pas des machines dont les hommes serait les rouages. L'efficacité
d'une entreprise tient plus a bon vouloir de ceux qui la compose
qu'aux différentes règles et stratégies qu'elle établie. \\ La
communication est le fondement de l'organisation, elle représente un
accomplissement quotidien.


\subsection{Que veut dire communiquer?}

Communiquer est une action qui visent a signifier certaines intention en
vue de susciter une réaction chez autrui.

\begin{itemize}
 \item[$\to$] Ici on ne s'intéresse qu'aux processus par lesquels un agent
     manifeste ses intentions pour autrui et qu'autrui reconnait.

     Il faut donc éviter la confusion entre comprendre ce que l'autre
     fait et comprendre ce qu'il nous dit.
\end{itemize}

\paragraph{Suppositions de la communication} :
\begin{itemize}
 \item La conscience de s'adresser à autrui.
 \item Un effort pour se faire comprendre.
 \item L'anticipation de certaines réactions chez l'autre. La
     communication part donc d'une conception d'autrui.
\end{itemize}


\subsection{Information et incertitude}

L'information est ce qu'un émetteur apprend à un récepteur qui
l'ignorait avant et qu'il sait après.

La communication est l'interaction qui permet la transmission
d'information. L'information est une modification du monde vécu par
autrui pour l'influencer.

\paragraph{Le principe de réduction de l'incertitude}

En informant autrui sur la relation avec lui, on réduit
l'indétermination qu'avait autrui par rapport à cette relation. C'est
cette liaison entre information et indetermination qui oriente la
communication.

La communication part d'une hypothèse sur l'état des connaissances
de l'autres et ce qui vaut comme information pour l'un ne vaut pas
nécessairement pour l'autre.


\subsection{Les distances informationnelles}

La distance définie ici est l'écart par rapport à la situation où
l'échange se réalise de façon fluide sans problème de circulation
de l'information. Il y a différents types de distances entre les
participants à la communication.

\begin{itemize}

 \item \textbf{La distance géographique} : La forme originelle du
travail organisé suppose que tout les travailleurs soit physiquement
présent (co-présence).
    \begin{itemize}
        \item[$\to$] Avec les nouvelles technologies, cette
    supposition n'est plus que partielle. Cette contrainte de proximité
    géographique n'a plus autant d'effet que avant.
    \end{itemize}

    \paragraph{Note: } Une bonne structure, maximise les coordinations
    au sein des départements, et minimise les coordinations entres les
    départements.

 \item \textbf{La distance sociale} C'est la distance qui sépare
les agents selon leurs statuts. Les différence des statuts dans
une organisation renvoie au stratification de la société dans
laquelle elle se développe.
En plus de cela, elle établie ses propres différence.

Ces différences entravent la communication.

 \item \textbf{La distance formelle} La distance dans l'organigramme
(qui peut parler à qui? qui peut donner des ordres à qui?).

La structure privilégie la communication vertical ascendante et
descendante. Elle rend la communication latéral difficile.

 \item \textbf{La distance culturelle} Les différences de langage, de
conception et de valeurs entre des individus ou des groupes d'individus.

Différentes raison créent cette distance (professions, milieu
d'origine,...). \textit{Par exemple les informaticiens ont un jargon
différent des juristes ce qui peut entraver la communication.}

 \item \textbf{La distance pratique} Elle trouve sa source dans les
conditions matérielles et l'organisation du travail, au regard de
la limitation des ressources d'attention et de traitement des agents
humains. \textit{Exemple: La présence de bruit ou l'impossibilité de
trouver un local pour se réunir.}

\item \textbf{La distance affective} La jalousie, la méfiance, la
rancune,... Toute sorte de raisons qui peuvent pousser des gens à
ne pas échanger.

\paragraph{Note: } Vous êtes payer pour travailler, donc avoir de bonnes
relations. Vous n'êtes pas là pour avoir de la
rancoeur, pour ne pas digérer un mauvais coup. Vous devez prendre sur
vous.

\end{itemize}

Ces distances représentent des difficultés à communiquer. Il faut
identifier les distances les plus critiques et les réduire pour
améliorer au mieux la communication au sein de l'organisation.


\subsection{Les erreurs typiques}
Quelques erreurs en matière de communication qui résulte de
simplification abusive.

\begin{enumerate}
 \item \textbf{La confusion du signal et de la communication} : Penser que
    transmettre = communiquer.

    La transmission n'est qu'un ensemble de signaux. Ils ne deviennent
    des informations qu'après le processus d'interpertration effectuer
    par le récepteur.

    \paragraph{Exemple: }\textit{Envoyé un mail sans savoir si il l'a lu}

 \item \textbf{La réduction du message au langage explicite}

     Le locuteur signifie toujours bien plus que ce qu'il dit, et son
    allocutaire comprend bien plus que ce qui est dit.

    Cette erreur recouvre plusieurs conceptions discutables:

    \begin{itemize}
     \item Réduction de la signification de \textbf{l'enoncé} (ce que veut dire
     le locuteur et ce que comprend l'allocutaire) sans tenir compte de la
     situation(intonations, attitudes,...)

     \item Négation du rôles des sous-entendu/messages \textbf{implicites} qui
    doivent être évoqué pour être compris.

     \item Réduction de la communication à l'utilisation d'un langage
     articulé (langue ou gestuelle). Cela ne tient pas compte de la gestion
     des inférénces, c'est à dire, le processus par lesquels les agents
     influencent intenionnellement les \textbf{inférences} des autres. (ex: Être
     froid vis-a-vis de quelqu'un pour lui exprimer un reproche)
    \end{itemize}

    \begin{itemize}
        \item[$\to$] On ne tient donc pas compte du décalage qu'il peut y
    avoir entre ce que l'on a dit et ce qui a été (doit être) compris.
    \end{itemize}



\item \textbf{La croyance en l'efficacité propre du langage} :

    Les gens de pouvoir qui surestiment le pouvoir des mots et pense
    changer le monde sitôt qu'ils en changent sa représentation.
    \begin{itemize}
        \item[$\to$] Ce fourvoiement conduit des dirigeants à penser qu'en expliquant des
    stratégies, sans forcément les mettre en place, tout changera.
    \end{itemize}

    On pense qu'en convaincant sont interlocuteurs d'agir autrement,il
    le fera. Cela est faux! Si on ordonne quelque chose à quelqu'un
    (ex:Travail mieux), il peut répondre oui de façon passive, sans
    forcément avoir l'intention d'executer l'ordre.

    Il est plus judicieux d'installer l'autre dans les conditions qui
    vont le pousser à decider lui même de faire l'action.

    \paragraph{Exemple: } \textit{L'adolescent qui dit ``ouai'' mais qui
    s'en branle complétement}

\item \textbf{L'inflation de la communication}

    Cette erreur consiste à élargir le champs de communication
    jusqu'à ce qui l'engloble toute la sphère de l'interprétable.
    Tout les problème deviennent résolvable par la communication.

    \begin{itemize}
        \item[$\to$] La communication en devient alors une \textit{théorie général des
    systèmes}, tellement général qu'elle se condamme à l'abstraction.
    \end{itemize}

    Le champs de la communication recouvre en fait 2 ensembles:

    \begin{itemize}
        \item L'ensemble A de tout les messages que l'agent adresse aux autres
        \item L'ensemble B des messages que l'agent interprète comme lui étant destiné.
    \end{itemize}

    En élargissant le champs de communication, on agrandit l'ensemble B
    et tout devient un message. On finit par penser que tout comportement
    a une valeur de message (\textit{par exemple un mec qui se gratte le
    crâne}).

    \paragraph{Exemple : } \textit{Tout les problèmes peuvent être
    résoudre par la communication. (Chef psy) }

 \item \textbf{La réduction à la fonction informationnelle}

     Erreur de penser que l'information n'est qu'un atome reproductible,
     transmettre une information peut aussi permettre de créer une
     réalité partagé (\textbf{objet de discution}) à laquel on
     pourra se référer. (\textit{par exemple dire qu'on déteste son
     chef à un collègue})

     Le language a trois fonctions fondamentale:

    \begin{enumerate}
        \item \textbf{Fonction constitutive:} créer un univers
        d'objets partagés, un espace public. En transmettant un
        information, celle ci acquièrent une objectivité (un fait) et
        une resistance (si moi j'oublie, l'autre sait).

        \item \textbf{Fonction identitaire:} \textit{en tant que X} je
        parle à Y \textit{en tant que Y}. Confirmation ou infirmation
        des définitions de soi et de l'autre ( ex: ''Donnez moi le
        dossier Z'' $\equiv$ ''Je suis le bosse vous êtes le subordonné)

        \item \textbf{Fonction emblèmatique:} signifie que en
        parlant, l'agent et l'interlocuteur affirme parfois une
        appartenance collective. Un élément qui permet de
        distinguer un groupe d'un autre peut devenir un emblème.
        Le langage permet de des possibilités de réunion et de
        différenciation(langue, accent, vocabulaire,...).
    \end{enumerate}

    On voit donc que le langage n'a pas seul vocation à
    \textbf{transmettre de l'information}.

\end{enumerate}


\subsection{Les enjeux de la communication}
Qu'est ce que l'on cherche en parlant?

\begin{enumerate}

    \item \textbf{Les enjeux phatiques} : Établir, garder le contact et
          préserver la relation.

    \item \textbf{Les enjeux pratiques} : Susciter plus ou moins
    directement des actions de l'autre dans la poursuite de la
    coordination sociale. (\textit{Ta gueule, je dois me concentrer})

    \item \textbf{Les enjeux de reconnaissance} : La reconnaissance
    un enjeu capital. On distingue 5 reconnaissances :
    \begin{itemize}
        \item La sphère de l'\textbf{amour} : l'expérience d'être aimé et
        la sécurité émotionnelle apporte de la confiance en soi.
        (\textit{sphère du travail perméable à de tels liens})

        \item La sphére de la reconnaissance \textbf{juridique} : sphère
        d'être investi de de droits et de devoirs. Apporte le respect de
        soi

        \item La sphère de l'estime \textbf{sociale} : la
        reconnaissance par autrui des qualités du sujet. Apporte
        l'estime de soi.

        \item La reconnaissance du \textbf{statut} social de la personne : en tant
        que roi, chef, expert,\ldots

        \item La reconnaissance en temps que \textbf{membre} du groupe.

    \end{itemize}

    $\to$ Il faut être reconnu par les autres pour se reconnaître soi-même.

    \item \textbf{Les enjeux de légitimation} Légitimer des
    contraintes sociale,(ex: le père qui dit à son fils qu'il doit
    dormir tôt pour son bien)

     \item \textbf{Les enjeux ontologiques} Préserver notre sens de la
    réalité, conforter nos conceptions des choses et confirmer nos
    convictions sur l'ordre du monde.

\end{enumerate}


\subsection{Le sens dépend du contexte}

Lorsque l'on parle, ce que l'on veut signifier dépend du
\textbf{contexte} (\textit{c'est à dire de tout ce qui est en dehors du
message lui-même, de ce qui est explicitement manifesté})
\begin{itemize}
\item[$\to$] Tout échange est \textbf{contextualisé}
\end{itemize}

La perception est toujours perception immédiate d'un tout organisé
différent de l'addition des parties, on ne distingue les parties qu'en
fonction du tout.

\begin{description}
 \item[Situation de confusion] Situation où l'on ne parvient pas
directement à reconnaître ce contexte donnateur de sens.
\end{description}


\paragraph{Cadrage et recadrage} Parfois l'émetteur va transmettre
des représentations du contexte afin de guider le récepteur dans son
travail.

\begin{itemize}
 \item \textbf{Cadrage:} toute communication visant à éclairer le
récepteur sur le contexte du message (ou de l'action) que celui-ci

 \item \textbf{Recadrage:} toute communcation visant à transformer
antérieurement "appliqué" au message (à l'action) par le récepteur.
(\textit{Typiquement concevoir la situation autrement})
\end{itemize}

Nous passons beaucoup de temps à transmettre aux autres des
\textbf{représentations contextuelles} (\textit{par exemple : ``Je vous
préviens, je n'ai que 10minutes'' contextualise sa hate})

\subsection{La ponctuation de la séquence des faits}

\begin{itemize}
    \item \textbf{Ponctuation de la séquence des faits} : quand on
    interprète un message nous faisons toujours commencer à un
    certains moment l'interpretation.
    \item[$\to$]
    Un désaccord sur cette dernière est base de nombreux conflits
    portant sur la relation (ex: c'est toi qui a commencé, non c'est toi!)
\end{itemize}


\subsection{Interaction symétrique et complémentaire}
Une relation est le résultat complexe des réactions d'une personne aux
réactions de l'autre.
Il y a deux grands types d'interactions :

\begin{itemize}
    \item \textbf{Interaction complémentaire} le comportement
    de l'un des partenaires complète le comportement de l'autre
    et réciproquement (\textit{ex: autorité du chef = obéissance du
    subordonné}).

\item[$\to$] Escalade complémentaire : Enchaînement des
    réactions conduit à la maximisation de la différence entre les
    partenaires. (\textit{le chef assume plus que ce qu'il ne voulait})

    \item \textbf{Interaction symétrique} les partenaires reproduisent
    leur ressemblance et confirment l'un à l'autre qu'ils sont
    semblables.

\item[$\to$] Escalade symétrique : chacun réagit en faisant "plus de la
    même chose", pour affirmer une différence à son avantage

\end{itemize}

\subsection{Les deux niveaux de la communication}

Il faut distinguer le \textit{contenu} et la \textit{relation}.

\begin{description}
\item[Contenu] : tout ce qui est évoqué dans l'échange verbal,
les objets et les événements auxquels on fait référence dans le
discours.

\item[Relation] Tout ce qui est exprimé en dehors de l'objet
représenté et qui contribue à définir la relation entre eux (ex:
jugement, approbation, humour,\ldots).
\end{description}

Il y a \textbf{toujours} un niveau de relation.

\subsection{Le rôle du corps}

Dans la communication corporelle, il faut distinguer le rôle du corps
\begin{itemize}
    \item[-] comme \textit{message} : usage du corps dans la construction du
message
    \item[-] comme \textit{contexte} :attitude involontaire qui
donne un contexte (\textit{ex: je tremble en parlant $\rightarrow$ j'ai
peur}).
\end{itemize}


\subsection{La communication inférentielle}

Il y a deux manières de signifier quelque chose à quelqu'un.
\begin{enumerate}
    \item Au moyen du language
    \item En amenant l'autre à inférer lui-même ce que nous voulons lui
        signifier.
\end{enumerate}

\paragraph{Répétition} Cela dit, en se répétant, des messages inférentiels entre dans la
sphère du language (ex: un symbole utilisé plusieurs fois deviens un
message évident).


Les problèmes de communication inférentielles sont souvents:
\begin{enumerate}
    \item Le récepteur reste sans remarquer l'intention de communication,
    \item Le récepteur se figure qu'il n'y a pas d'intention de communication,
    \item Le récepteur infère un message érroné.
\end{enumerate}

\subsection{Les paradoxes dans la communication}

Une communication \textbf{paradoxale} est une communication qui contient
en elle-même sa propre contradiction. (\textit{Je mens})

\begin{itemize}

    \item \textbf{Paradoxes internes aux messages} : La contradiction
    réside dans la nature même du message, soit au niveau du contenu,
    soit entre le contenu et la relation.

    \item \textbf{Paradoxes propres à la situation de communication}:
    La contradiction oppose le message et la réaction attendue à
    ce message.

    (je ne peux pas faire ce qu'on me demande, sinon je
    désobéi à ce qu'on me demande, \textit{injonction paradoxale}).

\begin{description}

\item[Injonction paradoxale] L'émetteur demande ou exige quelque chose
que, par ce fait même, il rend impossible, plaçant l'autre dans une
situation intenable puisqu'il est condamné à désobéir en obeissant.

\end{description}
\end{itemize}

\subsection{Métacommunication} : La communication sur la communication.
La métacommunication n'est donc pas inhérente à la communication.

Métacommuniquer c'est prendre la relation pour contenu du message:
verbaliser la relation vécue (\textit{ex: demander une explication sur
ce qui viens d'être dit}) et reconnaitre le vécu exprimé par l'autre
(\textit{ex: expliciter ce que l'autre dit}).
\begin{itemize}
    \item[$\to$] Double accomplissement : verbalisation et
        reconnaissance mutuelle
\end{itemize}


La métacommunicaiton apparaît souvent comme l'instrument essentiel de la
résolution des tensions et des conflits puisqu'il prend la
\textbf{relation} comme contenu de l'échange. On s'engage donc dans
activité de description de ce rapport.

\paragraph{ } \textit{Nous réinjectons de la vérité dans notre relation}

\subsection{Conclusion: la rationalité dialogique}

L'efficacité des décisions dépendent de l'image faite de la
réalité. La réalité dépend de la communication car elle est
en partie socialement construite par les représentations des autres.

Les problèmes et les solutions sont construits à partir de
dialogue et de négociation $\to$ la rationalité \textbf{dialogique}.

\paragraph{Conditions pour amener à une situation idéale de
communication} :

\begin{itemize}
 \item \textbf{Objectivité} : Les agents se limitent à comparer leurs
 conceptions de la réalité par arguments d'autorités. Il faut
 délaisser la fonction identitaire et emblématique.

 \item \textbf{Bienveillance mutuelle} : Chaque participant bénéficie d'une
attitude bienveillante.

 \item \textbf{Intégrité} : Les intentions désavouées ne peuvent se
 cacher derrière des intentions désintéressé.

 Les positions d'intérêts ne faussent pas la conception de la
 réalité.

 \item \textbf{Accès à l'échange} : Tout ceux qui ont des infos pertinente
 par rapport à un problème peuvent accéder à l'espace public de la
 gestion de ce problème.
\end{itemize}

La rationnalité n'est donc pas un flash mais une recherche qui demande
de rassembler l'information, l'interpreter, définir le problème et
construire des solutions. (\textit{Ces processus s'opèrent par des
échanges, réunions, négociations,\ldots})


\section{Le pouvoir}

Le pouvoir désigne tout exercice intentionnel d'une contrainte dans une
relation humaine. C'est une propriété de l'interaction et non de la
personne. Il suppose:

\begin{itemize}
    \item Une situation d'interdépendance.
    \item Un asymétrie dans l'interdépendance.
    \item Une marge de manoeuvre liée au contrôle de l'incertitude.
    \item L'usage de l'incertitude comme ressource dans la relation pour
        exercer une contrainte sur autrui.
\end{itemize}

\paragraph{ } \textit{Le pouvoir est tout exercice d'une contrainte dans
une relation}

\subsection{Interdépendance comme contexte nécessaire au pouvoir}

\textbf{L'interdépendance} trouve ses sources dans l'organisation
matérielle des activités humaines (ex:division du travail).

Elle est le contexte nécessaire pour des relations de pouvoirs mais ce
n'est pas la même chose que le pouvoir en lui même!

\paragraph{Note: } Le pouvoir ne surgit pas forcément d'une relation
d'interdépendance.

\subsubsection{Type d'interdépendance :}
\begin{enumerate}
    \item Séquentiel
    \item Réciproque
    \item Ressource partagées (\textit{Une tartellette pour deux mecs
        affamé})
\end{enumerate}

\subsection{La diversité des ressources stratégiques}
On considère comme ressource toute possibilité d'action sur autrui.
Elles permettent à l'agent qui les détient de tirer profit de la
situation/interaction.

\paragraph{Différents types de ressources} :

\begin{itemize}
	\item Matérielles : l'argent ou la gestion du temps
	\item Normatives : les règles et les normes sociales
	\item Affective : le charisme et l'affection.
\end{itemize}

La sanction est ce qui définit le mieux la notion de ressources.

Nous parlons de \textbf{sanction potentielles} : ce n'est pas l'exercice
effectif de la force qui définit le pouvoir, seulement, la conscience
chez le dominé de la possibilité de sa mise en oeuvre.

\paragraph{Incertitude}
Le \textbf{contôle de l'incertitude} est le caractère d'
imprévisibilité, la capacité discrétionnaire qui suspend le dominé
au bon vouloir du dominant.

\begin{itemize}
\item[$\to$] La ressource stratégique est donc la conjonction de la dépendance et des incertitudes utilisées pour faire pression.
    \end{itemize}

\paragraph{Volonté stratégique}
Il faut aussi que l'on ai envie d'utiliser les ressources.

\subsection{Le pouvoir est une relation}
Le pouvoir est une propriété de toute relation humaine, et les
ressources du pouvoir son mobilités dans l'interaction.

\subsection{Le pouvoir est moyen plus souvent qu'une fin}

Le pouvoir est plus souvent une médiation nécessaire plus qu'une
motivation. Si on peut l'éviter on le fait. Ce n'est pas l'emprise qui
compte mais le résultat.

\paragraph{Le pouvoir intentionnelle} n'implique pas que :
\begin{itemize}
	\item L'auteur le voit comme un pouvoir sur autrui.
	\item Que son exercice soit délibéré.
	\item Qu'il soit rationnelle au sens d'une sélection réfléchie des moyens pour atteindre une fin.
\end{itemize}


\subsection{Le pouvoir n'est pas un statut}

Le statut est une valorisation différentielle, une reconnaissance
sociale (prestige). Le pouvoir n'est que l'exercice d'une force, il
n'offre pas forcément cette reconnnaissance sociale.

Le pouvoir, contrairement au statut, n'a pas nécessairement une échelle
de valeur!

\subsection{Le pouvoir n'est pas réductible à des intérêts personnels}
Les finalités auxquelles amène le pouvoir peuvent être des intérêts
collectifs (\textit{par exemple lors d'un licensiement quand le grh doit
choisir qui il licensie selon des directives. Il ne cherche pas à
satisfaire ses besoins propres})

\paragraph{Légitimiation} Tout les processus qui contribue à effacer
ou atténuer cette image d'un pouvoir cynique et égoïste.


\subsection{Le pouvoir n'est pas la règle formelle}
Le pouvoir est
\begin{itemize}
	\item Local et non global, car il se conçoit dans le cadre d'un relation.
	\item Dispersé et non concentré , car il caractérise tout interaction humaine et non
	seulement les rapport hiérarchique.
\end{itemize}

Pour ces raisons, il ne faut pas confondre la position formelle
avec la règle formelle.
%TODO

\paragraph{Formaliser}, c'est construire et utiliser une présentaion
normative (ex: règle de travail) dans des rapports sociaux.

\subsection{Le pouvoir est fongible}

Le pouvoir est fongible, c'est à dire qu'il se consomme dans l'échange
et ne peut se prolonger qu'en reconstruisant le cadre concret de cette
échange. (Si on joue tout ses atouts, on a plus). Pour contrer cela, on
tente le bluff.

\subsection{Le pouvoir est intrinséquement limité}

Puisqu'il mobilise des ressources, il est forcément limité. Il est
limité par l'étendue des ressources mobilisable et l'intensité de
l'interdépendance entre les agents.

\paragraph{Relation} Le pouvoir ne survit pas à la relation : la
dépendance dépend de cette relation, si elle est détruite, le pouvoir
l'est aussi. (\textit{Si le marchand de chocolat exerce trop de pouvoir
sur toi en te demander de payer plus, supprime ton besoin de
chocolat et tu as supprimé le pouvoir qu'il avait sur toi})

\paragraph{Autonomie} Le pouvoir ne produit pas l'autonomie mais
l'autonomie est une limite au pouvoir (si le gars est autonome, le
pouvoir aura moins d'impact)

\subsection{La communication stratégique}
Il faut communiquer pour faire connaitre les ressources dont l'on
dispose. Le pouvoir est donc un propriété de l'interaction.

\begin{itemize}
    \item[] \textbf{Communication stratégique} : l'ensemble des processus de
communication inhérent à l'exercice du pouvoir (communication sur les
ressources, les intentions, l'interdépendance).
\end{itemize}

Le modèle qui imagine un acteur rationnel stratège comme un agent qui
part de la logique de la situation et qui en reconstruit les stratégies
des acteurs. Ce modèle a plusieurs limite:

 \begin{itemize}
	\item Méconnnaissance du rôle du processus de communication.
	\item Privilégie les ressources matérielles aux ressources
        affectives, cognitives,\ldots
\end{itemize}


Il existe un décalage entre les ressources des agents telles qu'une
analyse peut les reconstituer et la réalité du pouvoir qu'exercent ces
agents les un sur les autres.  Il y a un autre décalage entre les
ressources et intentions du dominant et celle qu'imagine le dominé.

\paragraph{Orienter a son avantage}

La relation de pouvoir est toujours sous entendue par des croyances
relatives au contexte d'interdépendance, aux ressources et aux
intentions. (\textit{``Je tiens dans ma poche un pistolet'', ``Inutile
de fuir les issues sont bloquer'',\ldots})

$\to$ Ces croyances sont en parties construites, étayés et corroborées dans
l'échange.

\subsection{Le pouvoir comme lien affectif}

On peut obéir à quelqu'un pour éviter sa colère, sa déception. Mais
l'émergence du pouvoir au coeur du lien affectif n'est pas un lien
mécanique. L'amour est un contexte nécessaire au pouvoir mais n'amène
pas directement le pouvoir. Il faut 4 conditions en plus:

\begin{enumerate}
	\item Chez l'amoureux: croyance d'une possibilité de  réciprocité.
	\item Chez l'aimé: L'envie de tirer profit de la situation.
	\item Chez l'aimé: Une manière de jouer avec l'incertitude.
	\item Chez l'amoureux: L'acceptation de ce pouvoir (se laisser faire)
\end{enumerate}

\subsection{Un modèle de ressource organisationnelles du pouvoir
(CHIMERE)}

Les ressources qui naissent de l'organisation elles-mêmes.

\begin{itemize}

 \item \textbf{Coalition} Mettre en communs nos atouts avec celui des
 autres (l'union fait la force, ex:syndicat)

 \item \textbf{Hierarchie} La position hiérarchique est associé au
 pouvoir formelle de dicter ou de faire appliquer. $\to$ Contrôle de
 l'incertitude.

 \item \textbf{Information} Dans le cas on l'on détient une information
dont l'autre a besoin.

 \item \textbf{Maitrise des processus} Tirer profit du fait de controler
 certains processus (ex: allocation des nouveaux ordinater)

 \item \textbf{Expertise} Dès lors que l'on possède des compétences
 nécessaire à l'organisation. Être irremplaçable est une forme de
 pouvoir.

 \item \textbf{Réseau} Le simple plus fait de connaitre quelqu'un
 intérieur ou extérieur à l'organisation. Les autres doivent passer
 par moi pour le contacter (ex: je connais un ministre)

 \item \textbf{Expérience stratégique} L'art de mobiliser les
 ressources citées plus haut en matière de jeux de pouvoir. Les
 jeux de pouvoir s'apprennent et certains sont meilleur que d'autre.

 Ils demandent des compétences telles que pouvoir lire la situation
 (politique), définir un stratégie et la mettre en oeuvre ou encore
 la résistance nerveuse.
\end{itemize}

\subsection{Appliquer les principes de l'analyse stratégiques}

\paragraph{3 grandes étapes à l'analyse stratégique} :

\begin{itemize}
 \item Définir la nature des interdépendance et l'incertitude qui lie
les acteur entre eux.

 \item Identifier pour chaque acteurs les ressources mobilisées, les
 principales incertitudes controlées.

 \item Analyser les conduites et les attitudes des acteurs comme des
 stratégies (faire pression ou se protéger).
\end{itemize}

\section{L'autorité et l'ordre de l'interaction}

\paragraph{Qu'est ce que l'autoritité?}
\begin{itemize}
	\item La combinaison entre le pouvoir et la légitimité
        \begin{itemize}
            \item[$\to$] Trop général , par exemple
	un expert un informatique qui conseille un président sur un ordi n'a pas d'autorité sur lui.
        \end{itemize}

	\item La combinaison entre le pouvoir hierarchique et la légitimité.
		\begin{itemize}
			\item Le pouvoir est la contrainte et non la personne qui exerce la contrainte.
			\item Cette définition confond le rôle hiérarchique et le pouvoir hiérarchique. Un
			rôle ça s'adopte, l'autorité est un rôle mais l'inverse non (il y a des chefs sans autorité).
			\item Le pouvoir s'oppose à une certaine mesure de l'autorité (ex: un prof qui multiplie
			les sanctions pour avoir le silence.)
		\end{itemize}

    \item[$\Rightarrow$] Nous avons donc que l'autorité s'attache à
 une personne et qu'elle est associé à un rôle hiérarchique et à
 un pouvoir hiérarchique.
\end{itemize}

\begin{center}
    \textbf{L'autorité est la légitimité personnelle que l'on peut tirer
    de l'exercice d'une position hiérarchique ou plus exactement l'exercice
    légitime d'un rôle hiérarchique et d'un pouvoir dans le cadre de ce
    rôle.}
\end{center}

\paragraph{Note: }
\begin{enumerate}
    \item Parler de rôle hiérarchique ça sous entend que l'autorité
    fait l'objet d'attente normaltives (\textit{tel que donner des
    consignes, réprimander,\ldots})

    \item La légitimité d'un acteur précede souvent son ascension à
    un rôle.

    \item On parle de l'autorité pour évoquer la subordination et
    l'obéissance, autrement dit la perte d'autonomie du dominé
    (\textit{cette perte d'autonomie n'étant pas une humiliation})
\end{enumerate}

La question fondamentale de l'autorité n'est pas celle du pouvoir, mais
celle de l'exercice légitime d'un rôle hiérarchique.


\subsection{Obéir à une autorité n'est pas se soumettre à un pouvoir}
Le fils obéit à son père car il reconnait son autorité via
l'appartenance à un groupe (père-fils)

\paragraph{ } La soumission à un pouvoir engendre une humilitiaion. Il
y a un calcul, comme si un mec a un flingue pointé tu vas te soumettre
pour ne pas mourrir.

\subsection{Caractéristiques}
\begin{itemize}
    \item Le pouvoir n'est pas rôle, l'autorité oui.
    \item Il faut du temps pour acquérir de l'autorité qui peut se
        peudre très rapidement
    \item Le pouvoir est un échec de l'autorité (\textit{vision
        simpliste}) car on s'attend en tant qu'autorité à ce qu'il
        utilise son pouvoir
    \item Il faut toujours qu'il y ait une \textbf{communauté} pour
        qu'il y ait une figure d'autorité qui exerce un pouvoir
\end{itemize}

\subsection{Rôle d'autorité}
On décrit ce qu'il y a dans a peu près toutes les autorités :
\begin{itemize}
    \item Le chef représente le groupe à l'extérieur
    \item Représenter le groupe à l'intérieur, car le groupe est une
        construction intellectuel et le chef doit définir qui l'on est,
        ce que l'on fait,\ldots
    \item Transcender les intéret personnel
    \item Organiser la réflexion collective
    \item Maîtrise du temps
    \item Assumer la responsabilité du groupe
\end{itemize}

\subsection{Perte d'autorité avec les autres modes d'interaction}

Exemple: Un directeur qui a perdu son autorité, n'est plus respecté,
écouté et obéit.
Plusieurs réactions possibles:

\begin{itemize}

 \item \textbf{Surcout de pouvoir coercitif} : assortir les directives
de menaces et sanctions.
	\item \textbf{Échange contractuel} : essayer d'acheter la soumission
        des membres.
	\item \textbf{Échange de don} : s'appuyer sur la norme de
        réciprocité.
	\item \textbf{Persuasion} : persuader les autres du bien fondé des
        consignes.
\end{itemize}

\subsubsection{Pouvoir coercitif}

Compenser sa légitimité défaillante par des menaces dans la mesures où
sa position lui confère des \textit{ressources stratégiques}.

\paragraph{ }
On passe d'un \textbf{pouvoir normatif} où on mobilise les forces
des normes partagées à un \textbf{pouvoir coercitif} où on profite
d'une asymétrie dans une relation d'intérdépendance pour imposer sa
volonté à un autre.

\begin{itemize}
    \item[$\to$] Le glissement du pouvoir normatif au pouvoir coercitif
marque la limite de l'autorité.
\end{itemize}

Le pouvoir coercitif s'observe lorsque la légitimité de l'agent est
remise en question ou que les normes sont inapplicables ou contestées.
Si on observe des menaces trop sévère, sa légitimité sera encore
plus égratigner (on peut penser que c'est une vengance).

\paragraph{Note: }
Toute relation humaine est asymétrique.

Le pouvoir coercitif s'observe dans les relations humaines dés qu'il
y a des déséquilibres intentionnels. (\textit{un enfant tyrannique,
une femme menaçant son amant de le quitter si ils ne se marient
pas,\ldots})


\subsubsection{L'échange contractuel}

Ex: L'enseignant supprime une dissertation en échange du calme
dans la classe $\to$ Établir une convention. En faisant cela:

\begin{itemize}
 \item Elle ouvre un droit de réclamation et à des pénalités en cas
de non respect de la convention.

 \item L'échange présuppose un droit de liberté formelle, on peut
 accepter ou non la convention sans être pénalisé.
\end{itemize}

A la différence du pouvoir coercitif, les deux agents entre en
intéraction pour gagner (\textit{coercitif l'un entrait pour ne pas
perdre quelque chose})

Si les parties ne se mettent pas d'accord, on revient au point initial.

\paragraph{Via la coercition} L'échange contractuel précède souvent
le pouvoir coercitif, même si le pouvoir coercitif peut prendre la
forme d'un échange contractuel (ex: si tu fais bien promotion , sinon
sanction). La seul chose qui change est l'absence de liberté formelle.


\subsubsection{L'échange de don}

Le don n'est pas un contrat, il est irréductible à un échange
contractuel (\textit{la réciprocité est implicite}). Le chef tente de
solliciter le bien vouloir de ses employés et peut même être le
premier à offrir un don.

En faisant cela il rénonce à son autorité:

\begin{itemize}
 \item Il sollicite une faveur au lieu de l'ordonner.
 \item S'abaisse à travestir sa volonté en recourant à la
 manipulation (on faisant le don, il attend quelque chose)
\end{itemize}

$\to$ Il n'est plus le porte parole de la communauté mais une personne
qui entre en intéraction avec un employé.

\subsubsection{La persuasion}

Si ces décisions sont remises en question, il peut tenter de persuader
des biens fondées de celles-ci. Il devient alors un participant égal
aux autre soumis à la régles du meilleur argument.

$\to$ L'employé obéit à la réalité et non plus au chef

\begin{itemize}
    \item Si l'agent est meilleur que les autres en joute verbal, cela accroîtra
son autorité. Les autres reconnaitront ses capacités et cela diminuera
les doutes qu'ils émettaient par rapport à l'agent.

La persuasion deviendra donc de moins en moins nécessaire.

\item L'inverse serait catastrophique pour l'agent, c'est pour cela qu'on
utilise les sanctions pour éviter la discussion.
\end{itemize}


L'autorité s'arrête où la discution commence, car ce qui est soumit à la
réflexion collective échappe nécessaire au contrôle du supérieur
hiérarchique.

\subsection{L'autorité dans ses rapports aux modes d'interaction}

La perte d'autorité force le sujet (\textit{via les quatres réaction})
à s'immerger dans le sphère des échanges et se lier à ses semblables
(l'autorité n'est plus qu'une ressource).

\paragraph{Recadrage} Les modes d'interactions peuvent être utiliser
pour le recadrage des échanges (\textit{Faire passer un échange
de don pour un échange contractuel, ou formuler une coercition en
terme d'échange contractuel}), mais ils peuvent aussi servir à
\textbf{déligitimer}.


\subparagraph{ } Un reponsables dans l'exercice de ses fonctions
s'engagera, au gré des situations, dans les différentes modes
d'interactions.

L'autorité que l'on confère à une personne cesse d'exister avec les
autres communautés (ex: une réunion de présidents). \textit{C'est même
    propre à l'autorité d'être une médiation avec d'autres lieu et
puissance sur lequel il n'a pas d'autorité}.

A partir de ce moment, cet agent d'autorité (pour nous) est bien forcé
de nouer des interactions basés sur la coercition, l'échange
contractuel, le don et la persuasion avec les autres.

$\to$ Les capacités dont il fera preuve face à cette autre communauté
influeront l'autorité qu'il a dans sa communauté.

\subsection{L'autorité comme légitimité}

\begin{itemize}
    \item[-] La légitimité ne s'ajoute pas à la l'autorité, elle lui est
inhérente (autorité illégitime = domination).

\item[-] Dans ce contexte, la légitimité $\neq$ de légitime à mes/tes/ses
yeux. On parle de la légitimité d'un point de vue générale (pour un
témoin extérieur et impartial).
\end{itemize}

\paragraph{Les sources de la légitimité}

D'ou vient la légitimité de l'autorité?

\begin{itemize}

 \item \textbf{Légitimité personelle} C'est à dire les qualité, les
compétences, le pouvoir,....

Au contraire la moindre preuve d'ignoraince, incompétence,\ldots rend un
candidat illégitime.
 \item \textbf{Légitimité d'accesion} L'accès à la position c'est fait des
 les règles d'usage (un entretien, un vote électorale,...).

 \item \textbf{Légitimité de la position} Reconnaissance publique de la
 position (ex: le beau père qui n'est pas obéi par ses belles filles
 $\to$ 'T'es pas notre père'')

 \item \textbf{Légitimité de représention} Incarne un image valorisante du
 collectif. Raconte un bon récit de la communauté.

 \item \textbf{Légitimité de rôles et d'action} Rempli les différentes
 fonctions pratico-sociales et délivre des résultats attendu
 (ex:nouvel entraineur d'un équipe de foot qui perd tout le temps)

 \item \textbf{Légitimité de consentement} La légitimité
 par inférence, l'autorité fait l'objet de signe manifeste de
 consentement.
 $\to$ je suis nouveau dans un boite,
 je sais pas qui est le bosse mais comme les autres obéissent à un
 gars ça doit être lui.

 \textit{le consentement de tout le monde (ou d'une autre
 autorité à laquelle nous déléguons notre pouvoir de conférer la
 légitimié) est une légitimité}

 On peut donc affaiblir sa légitimié en remettant en cause l'évidence de
 sa légitimité

\end{itemize}

Si l'une de ces légitimités est remise en question, l'autorité
s'amenuise $\to$ Renforcement des autres modes d'intéraction.

\section{Une forme d'autorité typique de notre modernité: le leadership}

Le leadership est une forme d'autorité, il reprend la légitimité
personnelle, de réprésentation et celle de rôle et d'action. Il est
lié à des environnements de performance où il y a de la gestion du
changement et des valeurs démocratiques.

\subsection{Erreur sur le leadership}
\begin{itemize}

    \item{Confusion entre leadership et charisme}

    \item{Héroisation du leadership}, c'est à ça que vous devez
    ressembler! On place le leader sur un pied d'estale mais ce n'est
    qu'une illusion.

    \item{Leadership innée}, le leadership ça s'apprend, il suffit
juste de ce remetrre en question

\end{itemize}

\subsection{Valeur fondatrice du leadership}

\begin{itemize}

    \item \textbf{Finalité dans les processus} Il ne se contente pas de gérer
    pas les règles, la domination ou les échanges mais plutôt par la
    valeur de la finalité (ex: infirmière qui sauve des vies)

    \item \textbf{Égal respect} Respect dans les relations, pas d'air
condescendant.

    \item \textbf{Reconnaissance de la personne} Étre attentif vis à vis des
    individus et être attentionné (écouter ce qu'il dise)

    \item \textbf{Vérité avec un 'v' minuscule} Dire la vérité tout en
    respectant les autres Si il y a un problème il faut en parler
    mais pas dans un logique ''il n'y a que moi qui ai la réponse''.
    \textbf{Mais} Ne pas oublier la vérité qu'il y un chef et il faut
    lui obéir

    \item \textbf{Réflexivité} Demander un feedback sur sa performance. Il
    faut partir de la vision des autres pour se réprésenter dans la
    but de s'améliorer.

    \item \textbf{Équité} Juste rapport entre rétribution et contributiions.
    Entre ce que les gens font et leurs réaction $\to$ Créer
    l'instrumentalité.

    \item \textbf{Résponsabilité} Assumer ses résponsabilité tout en
résponsabilisant les autres.
\end{itemize}

\subsection{Conclusion}

\begin{itemize}
    \item Tout ces valeurs se tiennent les une les autres. Soit on prend
    une, on les prend toutes.
    \item Ce ne sont pas vraiment des principes morales mais plutôt des
    guides de comportement.
    \item Leadership n'est pas un ensemble de technique mais un projet
de vie.
\end{itemize}

\end{document}
