\documentclass[en]{../../../eplsummary}

\hypertitle{Artificial Intelligence}{7}{INGI}{2261}
{Nicolas Houtain \and Gorby Nicolas Ndonda Kabasele}
{Peter Van Roy}
$$$$

Attention : This summary is actually based on course note


\section{Course note}

(Need slide to understand)

\subsection{Teaser}

\subsubsection{Parallel and distributed computing}
\begin{itemize}
    \item Parallel computing : many node, optimiz performance, no
        failure
    \item[$\to$] Tightly coupled

    \item Distributed computing : many node in collaboration with
        \textcolor{red}{partial failure}
    \item[$\to$] Loosely coupled
\end{itemize}

\subsubsection{Consensus}
Atomic broadcast $\equiv$ consensus (proof slide 13)

\begin{enumerate}
    \item broadcast $\to$ consensus : on prend la première proposition
        reçue (On peut le faire gràce à l'ordre)
    \item Consensus $\to$ broadcast : on fait un consensus sur l'ordre à
        avoir
\end{enumerate}

Paxos est ce qui est le plus utilisé pour les consensus (TODO)

\begin{itemize}
    \item Asynchronous : Consensus with 0 node crashes
    \item Partially synchronous : consensus sit < $\frac{n}{2}$ crashes
    \item Synchronous : consensus with n-1 crashes
\end{itemize}

\paragraph{Asynchronous vs Synchronous}

Bound is simulated with a expect bound to be in partially synchronous.

\subsubsection{Failure detector}
Bound exist but we don't know the exact value because this bound can
change with time (if RTT increase for example)

\subsection{Formal models of distributed system}




\begin{thebibliography}{1}
\bibitem{wikiadmheur} http://en.wikipedia.org/wiki/Admissible\_heuristic, {\em Wikipedia}
%\bibitem{Propagation} P.G. Fontolliet, {\em Traité d'Electricité}, Volume XVIII, Ecole polytechnique fédéral de Lausanne, pp 72-73
\end{thebibliography}

\end{document}
