\documentclass[en]{../../../eplsummary}

\hypertitle{advSecu-lingi2144}{7}{lingi}{2144}
{Houtain Nicolas \and Kabasele Ndonda Gorby Nicolas}
{Gildas Avoine}

\section{Introduction}

\section{Vocabulary}

\begin{itemize}
    \item nonce: random number used only once
\end{itemize}


\section{Token-based Authentication}
\subsection{Definition}

\begin{description}
	\item[Identification:] We get the identity of a party
	\item[Authentication:] We get the identity of a party and the proof that the
	identity is true. The two parties are the \textbf{verifier}
    (verifies the proof) and the \textbf{prover} (provides the proof)
\end{description}

A token is an object that someone can own. They can be classified according to
technology used:

\begin{description}
	\item[Printed Tokens:] Tickets, Barcode
	\item[Digital Memory:] Magnetic strips cards,USB,\ldots
	\item[Microcircuit-based:] Smart cards,RFID,\ldots
\end{description}

\subsection{Printed Tokens}

    \paragraph{Human readable Tokens}
    \label{par:humanreadableTokens}

    \begin{itemize}
            \item When the ticket is provided by the verifier,
            the security is based on the difficulty to find the paper.
            \item When ticket printed on by the customer, forgery and
            copy can easy be detected.
    \end{itemize}

    \paragraph{Optically-readable Tokens}

    Data are represented such that it can be read by an optical
    machine. These token can be used for authentication.

\subsection{Digital-memory Tokens}

    For the magnetic stripe, sensitive data can be stored in the
    card or in a external database. Most magnetic stripe cards
    are \textsc{ISO-7811} compliant and contains 3 tracks.

\subsection{Microcircuits Tokens}

\begin{itemize}
    \item Classification on these tokens is done according to the
        calculation capabilities and the interface.
    \item Data can never leave the card,can be accessible after an
    authenfication and can be publicly accessible.
\end{itemize}

\section{RFID Primer}
\subsection{Definition}

\begin{description}
    \item[Radio Frequency IDentification:] Remotely retrieves data
    using devices called RFID tags through electromagnetic radiating
    waves.
    \item[RFID tags:] Small device containing a chip and an antenna to
    receive/respond to radio-frequency queries from an RFID reader/writer.
\end{description}
RFID tag can be low-capability device or porwerful contacless smartcard.
%%TODO Architecture slide 60?

\subsection{Daily Life Example}

\begin{itemize}
    \item Pet identification
    \item Localisation
    \item Book borrowing and inventories.
\end{itemize}

\subsection{Tags Characteristics}
\paragraph{Power Source}
\begin{description}
    \item[Passive] Tags don't have any internal energy source. They
    get energy from the reader.
    \item[Active] Tags have a battery that is used both for internal
    calculations and transmission.
    \item[Semi-Passive] Tags only have energy for the calculation.
\end{description}


\paragraph{Frequency Bands}
\begin{table}
    \centering
    \begin{tabular}{c|c|c}
        & Frequency & Range \\
        \hline
        Low Frequency &  124\text{-}135 kHz & centimeters \\
        High Frequency &  13.56 MHz & decimeters \\
        Ultra-High Frequency & 860\text{-}960 MHz & meters \\
    \end{tabular}
\end{table}

\paragraph{Memory}
Tags have at least a few bits to store unique identifier UID.
\begin{itemize}
        \item UID size 32 to 128 bits.
        \item Usually, the UID is chosen by the manufacturer and cannot
        be changed by the user.
        \item EEPROM additional memory can be added (1KB or 70KB for passport)

\end{itemize}
\end{document}
