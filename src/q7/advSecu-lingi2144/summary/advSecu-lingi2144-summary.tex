\documentclass[en]{../../../eplsummary}

\hypertitle{advSecu-lingi2144}{7}{lingi}{2144}
{Houtain Nicolas \and Kabasele Ndonda Gorby Nicolas}
{Gildas Avoine}

\section{Introduction}

\section{Vocabulary}

\begin{itemize}
    \item nonce: random number used only once
\end{itemize}


\section{Token-based Authentication}
\subsection{Definition}

\begin{description}
	\item[Identification:] We get the identity of a party
	\item[Authentication:] We get the identity of a party and the proof that the
	identity is true. The two parties are the \textbf{verifier}
    (verifies the proof) and the \textbf{prover} (provides the proof)
\end{description}

A token is an object that someone can own. They can be classified according to
technology used:

\begin{description}
	\item[Printed Tokens:] Tickets, Barcode
	\item[Digital Memory:] Magnetic strips cards,USB,\ldots
	\item[Microcircuit-based:] Smart cards,RFID,\ldots
\end{description}

\subsection{Printed Tokens}

    \paragraph{Human readable Tokens}
    \label{par:humanreadableTokens}

    \begin{itemize}
            \item When the ticket is provided by the verifier,
            the security is based on the difficulty to find the paper.
            \item When ticket printed on by the customer, forgery and
            copy can easy be detected.
    \end{itemize}

    \paragraph{Optically-readable Tokens}

    Data are represented such that it can be read by an optical
    machine. These token can be used for authentication.

\subsection{Digital-memory Tokens}

    For the magnetic stripe, sensitive data can be stored in the
    card or in a external database. Most magnetic stripe cards
    are \textsc{ISO-7811} compliant and contains 3 tracks.

\subsection{Microcircuits Tokens}

\begin{itemize}
    \item Classification on these tokens is done according to the
        calculation capabilities and the interface.
    \item Data can never leave the card,can be accessible after an
    authenfication and can be publicly accessible.
\end{itemize}

\section{RFID Primer}
\subsection{Definition}

\begin{description}
    \item[Radio Frequency IDentification:] Remotely retrieves data
    using devices called RFID tags through electromagnetic radiating
    waves.
    \item[RFID tags:] Small device containing a chip and an antenna to
    receive/respond to radio-frequency queries from an RFID reader/writer.
\end{description}
RFID tag can be low-capability device or porwerful contacless smartcard.
%%TODO Architecture slide 60?

\subsection{Daily Life Example}

\begin{itemize}
    \item Pet identification
    \item Localisation
    \item Book borrowing and inventories.
\end{itemize}

\subsection{Tags Characteristics}
\subsubsection{Power Source}
\begin{description}
    \item[Passive] Tags don't have any internal energy source. They
    get energy from the reader.
    \item[Active] Tags have a battery that is used both for internal
    calculations and transmission.
    \item[Semi-Passive] Tags only have energy for the calculation.
\end{description}


\subsubsection{Frequency Bands}
\begin{table}
    \centering
    \begin{tabular}{c|c|c}
        & Frequency & Range \\
        \hline
        Low Frequency &  124\text{-}135 kHz & centimeters \\
        High Frequency &  13.56 MHz & decimeters \\
        Ultra-High Frequency & 860\text{-}960 MHz & meters \\
    \end{tabular}
\end{table}

\subsubsection{Memory}
Tags have at least a few bits to store unique identifier UID\@.
\begin{itemize}
        \item UID size 32 to 128 bits.
        \item Usually, the UID is chosen by the manufacturer and cannot
        be changed by the user.
        \item EEPROM additional memory can be added (1KB or 70KB for passport)
\end{itemize}

\begin{description}
    \item[Tamper-resistance] A device is tamper-resistant if no adversary can
    get access to its protected memory by use of side-channel attacks.
    \item[Side-channel attack] An attack where the information is retrieved
    from physical implementation of a system (ex:Timing attack,physical attack).
\end{description}

\paragraph{Computation Capabilities}
The tags can perfom simple logic operation and symmetric/asymmetric
cryptographic. For symmetric cryptography, the microprocessor is not
necessarily needed.

\paragraph{Near Field Communication}
NFC is an extension of RFID, its main difference its that it offer
Peer-to-Peer connections between two (active) devices. NFC Data are exchanged
in a different format.

\subsection{Communication with the Tag}
For tags compliant with \textsc{ISO 7816}
\subsubsection{Memory}
Internal structure composed of files. There are two types of files:
\begin{itemize}
    \item Dedicated File (\textbf{DF})
    \item Elementary File (\textbf{EF}) that can be identified in two subtypes
    \begin{description}
        \item[Internal] Store information used by the card for management and
        control purpose.
        \item[External] Store information used exclusively by the outside world
    \end{description}
\end{itemize}
The Master File (\textbf{MF}) is a special \textbf{DF} and is the only mandatory
file. Each application is stored in a distinct \textbf{DF}.

Data inside an EF is stored in different formats:
\begin{description}
        \item[data unit] Smallest set of bits. Default value is 1 byte.
        \item[record] String of bytes which can be handled as a whole. Record
        can be referenced by a an 8-bit identifier.
        \item[date object] Structure of information which consists of a tag, a
        length and a value.
\end{description}

\paragraph{Communication}
The protocol used to exchange data is the Application Protocol Data Unit.

\section{Basic of Cryptography}
\subsection{Terminology}
\begin{description}
    \item[Cryptography] Science that design algorithms to ensure confidentiality
    of communication through insecure channels.
    \item[Confidentiality] Insurance that a given information cannot be accessed
    by unauthorized parties.
    \item[Cryptanalysis] Science that proves or disproves the security of
    cryptographic algorithm.
\end{description}
Break a cryptographic algortim means either:
\begin{itemize}
    \item Decrypting an encrypted message.
    \item Recovering the key of the cryptographic algorithm.
    \item Proving that a algorithm is less secure than what is claimed.
\end{itemize}

\paragraph{Attacker Model}
\begin{itemize}
    \item Adversary can be passive (ciphertext-only, known-plaintext) or
    \item active (chosen-plaintext,chosen-ciphertext)
\end{itemize}

\subsubsection{Encription Algorithm}
\begin{description}
    \item[Encryption algorithm] Algorithm that transforms an intelligible text
    into text that is unintelligible for non-authorized parties. The input is
    the plaintext and the output is the ciphertext.
\end{description}
Two categories of cryptographic algorithm:
\begin{itemize}
    \item Symmetric-key where the same key is used for both encryption and
    decryption.
    \item Asymmetric-key cryptography where there is a public key to encrypt and
    an private key to decrypt.
\end{itemize}

\subsection{Encryption}
%%TODO put it in table
\subsubsection{Symmetric-key Cryptography}
\begin{itemize}
    \item Based on bit or byte operations.
    \item Tend to be fast.
    \item Typical key-size: 128 bits
    \item Best attack is the exhaustive search.
\end{itemize}
Symmetric-key algorihtm can use Block ciphers or Stream ciphers. Block ciphers
acts on the plaintext in blocks. Stream ciphers acts on the plaintext one
symbol at a time.
\paragraph{Note:} DES is known to not be secure anymore.
\subsubsection{Asymmetric-key Cryptography}
\begin{itemize}
    \item Based on mathematical problems
    \item Best attack exploit the mathematical structure.
    \item Typical key size: 1024 bits.
    \item Encryptions and decryptions are slow.
\end{itemize}
\paragraph{Note:} RSA stands for Rivest-Shamir-Adleman
Certificate are used to identiry of the key owner. A Certificate authority
issue the certificate.

Parties can exchange their key or used Diffie-Hellman to agree on a shared key.

\subsection{Authentication and Integrity}
Entity authentication is asymmetric or symmmetric. Data authentication is
asymmetric or symmetric.

\subsection{Hash Function}
$$h:{0,1}*\rightarrow{0,1}^n$$
\begin{description}
    \item[First preimage resistance] Given a hash value y, it is infeasible to
    find m such that $h(m) = y$.
    \item[Second preimage resistance] Given a message $m_1$, it is infeasible
    to find a different message $m_2$ such that $h(m_1)=h(m_2)$.
    \item[Collision resistance] It is infeasible to find two different messages
    $m_1$ and $m_2$ such that $h(m_1)=h(m_2)$
    \item[Random oracle property] $h(m)$ is distinguishable from a random n-bit
    value.
\end{description}

\paragraph{Birthday Paradox} If we pick $\theta\sqrt{N}$ random numbers,
independently and uniformly, in {1,2,\ldots,N}, we get at least one number twice
with probability:
$$1-e^{-\frac{\theta^2}{2}}$$
With the Birthday paradox, we need about $\sqrt{2^n}$ hash operations to find a
collision with probability $\frac{1}{2}$
\paragraph{Message Authentication Code} Hash function that use a key.

\section{Authentication Protocols}
\subsection{Adversary Means}
\begin{description}
    \item[Eavesdropping] Adversary listens to the channel and hopes getting some
    useful data.
    \item[Skimming] Adversary queries the prover to get some useful information.
    \item[Modifying] Adversary modifies messages while they transit on the
    channel.
    \item[Injecting] Adversary injects messages on the channel.
    \item[Tampering] Adversary obtains content of the protected memory by
    physical means.
    \item[Exploiting] Adversary obtains information thanks to the system
    implementation.
    \item[Re-playing] Adversary replays a message she previously observed.
    \item[Pre-playing] Adversary plays the message she obtained.
    \item[Reflecting] Adversary reflects a reflect from a verifier so that
    answers itself.
    \item[Guessing] Adversary tries to guess the right answer or key w/o help of
    the prover.
    \item[Relaying] Adversary forwards the signal between ther verifier and the
    prover.
\end{description}
In the \textbf{Dolev-Yao} Model, adversary
\begin{itemize}
    \item Cannot guess random numbers
    \item Cannot decrypt or create valid ciphertexts without correct secret.
    \item Cannot retrieve private keys from public information.
\end{itemize}
\paragraph{Note:} To avoid replay and preplay attack, one-time password or
challenge/response protocol can be used.

%%TODO get schema from note
\subsection{One-time Passwords}
Password are generated with a one-way function.
\subsection{Challenge response}
Challenge response protocol can use different mechanism:
\begin{itemize}
    \item Timestamps where the challenge is based on clock.
    \item Sequence numbers where the challenge is based on hash on a
    one time password.
    \item Random numbers
\end{itemize}

%%Align all these
\subsection{ISO 9798 Challenge/Response}
ISO 9798\text{-}2 based on Symmetric-Key Encryption:
\begin{itemize}
    \item Mechanism-1
    		$$ A  \rightarrow B \quad E_k(t_A,B) $$
    \item Mechanism-2
    		$$ A  \leftarrow  B \quad r_B $$
    		$$ A  \rightarrow B \quad E_k(r_B,B) $$
    \item Mechanism-3
    		$$ A  \leftarrow B \quad r_B $$
    		$$ A  \rightarrow B \quad E_k(r_A,r_B,B) $$
    		$$ A  \leftarrow  B \quad E_k(r_n,r_A) $$
\end{itemize}
%%TODO VARIANT slide 162
ISO 9798\text{-}4 based on Hash function:
\begin{itemize}
    \item Mechanism-1
    $$ A \rightarrow B \quad H_k(t_A,B),t_A $$
    \item Mechanism-2
    $$ A \leftarrow B \quad r_B $$
    $$ A \rightarrow B \quad H_k(r_B,B),B $$
    \item Mechanism-3
    $$ A \leftarrow B \quad r_B $$
    $$ A \rightarrow B \quad H_k(r_A,r_B,B),r_A $$
    $$ A \leftarrow B \quad H_k(r_b,r_A,A) $$
\end{itemize}
ISO 9798\text{-}3 based on Public-Key Signature:
\begin{itemize}
    \item Mechanism-1
    $$ A \rightarrow B \quad S_A(t_A,B),B,t_A,cert_A $$
    \item Mechanism-2
    $$ A \leftarrow B \quad r_B$$
    $$ A \rightarrow B \quad S_A(r_A,r_B,B),B,r_A,cert_A $$
    \item Mechanism 3
    $$ A \leftarrow B \quad r_B $$
    $$ A \rightarrow B \quad S_A(r_A,r_B,B),B,r_A,cert_A $$
    $$ A \leftarrow B \quad S_B(r_B,r_A,A),A,cert_B $$
\end{itemize}
%%TODO VARIANT slide 174
\subsection{Authentication with Key Establshment}
During the phase of authentication phase, the two parties can exchange key
material.

ISO 11770\text{-}2 Symmetric-Key
\begin{itemize}
    \item Mechanism 4
    $$ A \leftarrow B \quad r_B $$
    $$ A \rightarrow B \quad E_k(r_B,B,k_1) $$
    The session key is $k_1$

    \item Mechanism 6
    $$ A \leftarrow B \quad r_B $$
    $$ A \rightarrow B \quad E_k(r_A,r_B,B,k_1) $$
    $$ A \leftarrow B \quad E_k(r_B,r_A,k_2) $$
    With a key-derivation function $f$, $f(k_1,k_2)$ is the session key.
\end{itemize}

Nedham\text{-}Schroeder Public-Key
\begin{itemize}
    \item Original Version
    $$ A \leftarrow B \quad P_A(k_1,B)  $$
    $$ A \rightarrow B \quad P_B(k_1,k_2) $$
    $$ A \leftarrow B P_A(k_2) $$
    $f(k_1,k_2)$ is the session key.
    \item Modified Version
    $$ A \leftarrow B \quad P_A(k_1,B,r_1) $$
    $$ A \rightarrow B \quad P_B(k_2,r_1,r_2) $$
    $$ A \leftarrow B \quad r_2 $$
\end{itemize}

ISO 11770\text{-} Asymmetric-Key
\begin{itemize}
    \item Mechanism 5
    $$ A \leftarrow B \quad r_B $$
    $$ A \rightarrow B \quad r_A,r_B,B,P_B(A,k_1),S_A(r_A,r_B,B,P_B(A,k_1))$$
    $$ A \leftarrow B \quad r_B,r_A,A,P_A(B,k_2),S_B,(r_B,r_A,A,P_A(B,k_2))$$
    Session key can be $f(k_1,k_2)$.
    \item Mechanims 6
    $$ A \leftarrow B \quad P_A(B,k_2,r_B) $$
    $$ A \rightarrow B \quad P_B(A,k_2,r_B,r_A) $$
    $$ A \leftarrow B \quad r_A $$
    Session key van be $f(k_1,k_2)$.
\end{itemize}
X.509
\begin{itemize}
    \item 2-Pass Mutual Authentication Protocol
    $$ A \leftarrow B \quad cert_B,D_B,S_B(D_B) $$
    $$ D_B = t_B,r_B,A,data*_1,P_A(k_1)* $$
    $$ A \rightarrow B \quad cert_A,D_A,S_A(D_A) $$
    $$ D_A = t_A,r_A,B,r_B,data_2,P_B(k_2)* $$
    $t_x$ defines a validity period and $r_x$ includes a sequential component.
    $f(k_1,k_2)$ is the session key.
    \item 3-Pass Mutual Authentication Protocol
    $$ A \leftarrow B \quad cert_B,D_B,S_B(D_B) $$
    $$ D_B = (t_B,r_b,A,data_1,P_A(k_1)*) $$
    $$ A \rightarrow B \quad cert_A,D_A,S_A(D_A) $$
    $$ D_A = (t_A,r_A,B,r_B,data*_2,P_B(k_2)*) $$
    $$ A \leftarrow B \quad r_A,A,S_B(r_A,A) $$
\end{itemize}
\end{document}
