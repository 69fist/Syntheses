\documentclass[fr]{../../../eplsummary}

\newcommand{\syn}{\mathsf{SYN}}
\newcommand{\ack}{\mathsf{ACK}}

\hypertitle{Computer System Security}{6}{INGI}{2347}
{Julien Bouilliez\and Julien Bergiers\and Quentin Cappart\and Thomas Haine\and Julien Henricot}
{Gildas Avoine}

\section{Introduction}
\paragraph{Hacker célèbre :}
\begin{itemize}
  \item Kevin Mitnick (telephone hack et IP Spoofing)
  \item Captain Krunch (appel longue distance gratuit)
  \item David L. Smith (worm Melissa)
\end{itemize}

\paragraph{Analyse de risque:}
\begin{itemize}
  \item Importance des actifs (V).
  \item Menaces potentielles (M).
  \item Évaluer la probabilité de menace (P).
  \item Nous devons parvenir à un risque résiduel raisonnable.
\end{itemize}

$$\mathsf{Risk} = \sum P(M_i) V_i$$

\section{Spam :}
\subsection{Généralités:}
\begin{mydef}[Brad Templetons]
  Je définis l'abus e-mail comme un e-mail qui répond aux les trois de critères suivants:
  \begin{enumerate}
    \item Il est non sollicitée.
    \item Il fait partie d'un ``mailing de masse''.
      (envois en nombre)
    \item L'expéditeur est un inconnu pour le destinataire.
      (Le destinataire n'a jamais eu de contact personnel volontaire avec l'expéditeur.)
  \end{enumerate}
\end{mydef}

\subsection{Conséquence du spam}
Infrastructure.
\begin{itemize}
  \item Bande passante.
  \item Coût lors du téléchargement des données limitées ou GPRS.
  \item Pare-feu, stockage de données, serveurs de messagerie, sauvegarde, etc.
\end{itemize}
Productivité.
\begin{itemize}
  \item Les utilisateurs sont perturbés au cours de leur travail.
  \item E-mails utiles perdues.
  \item Besoin de renouveler adresses e-mail.
  \item Virus peut être inclus dans le spam.
\end{itemize}
Lois.
\begin{itemize}
  \item Les employés contre la compagnie.
  \item Les utilisateurs contre les fournisseurs.
\end{itemize}
Modèle économique:
Acteurs.
\begin{itemize}
  \item Publicitaire.
  \item Spammer.
  \item Récupérateur d'adresse.
\end{itemize}
Sources de bénéfices.
\begin{itemize}
  \item Bénéfices des ventes, escroquerie.
  \item Partage des bénéfices.
  \item Nombre d'adresses dans la liste.
  \item Cliquez par clic.
  \item Obtenir plus d'adresses.
\end{itemize}

\subsection{Spamming techniques:}
\subsubsection{Méthode 1: utilisation de son propre serveur SMTP / ISP.}
Le protocole utilisé par les serveurs de messagerie sur Internet est SMTP (Simple protocole de
transfert de courrier), RFC 821, 1982.
Comme ce protocole n'utilise aucune authentification, il est facile de forger des mails.
Limitations :
Le serveur qui reçoit le courrier initialement note:
\begin{itemize}
  \item Le contenu de la commande HELO.
  \item L'adresse IP de l'expéditeur du courrier.
  \item Le temps de réception.
\end{itemize}

Avec ces données, il est souvent facile de trouver l'auteur de l'e-mail. Il est souvent facile d'obtenir les
mêmes résultats en seulement modifiant la configuration de votre programme de messagerie.
Pour être non détectable, il faut lancer la connexion telnet sur une machine qui ne log pas les
utilisateurs.
\subsubsection{Méthode 2: Abus de relais ouverts (Open Relay).}
Un seul message est déposé dans quelques serveurs SMTP avec des milliers d'adresses de destination
chacun. Les serveurs abusés envoient poliment sur une copie du message à chaque destination.
Éviter Open Relay:
Pour qu'un message soit accepté par le serveur de messagerie, l'adresse du destinataire ou de
l'expéditeur doivent appartenir au même domaine que le serveur.
Ou encore que seul les machines appartenant au même domaine que le serveur de messagerie puisse
transmettre des email avec un envoyeur du domaine, à celui-ci.

\subsubsection{Méthode 3: l'abus d'un compte webmail.}
Utilisez un script pour ouvrir de nombreux comptes webmail et envoyer du spam jusqu'a ce que les
comptes soient fermés.

\subsubsection{Méthode 4: pirater des ordinateurs personnels.}
Utilisez un virus qui infecte les ordinateurs personnels (les transformant en «bots»).
Utilisez un réseau de bots pour envoyer le spam.
Note : Botnet
Botnet = Un réseau de machines piratées contrôlées par un hacker.
Généralement, les bots se connectent à un Relay chat Internet et
attendent les commandes de leur maître.
Récupération d'adresse e-mail :
\begin{itemize}
  \item Acheter un fichier d'adresses email.
  \item Crawler (robot d'indexation) qui parcours le Web.
  \item Dictionary, la force brute pour deviner des adresses email valides pour un domaine donné.
  \item Hacking: attaquer une base de données.
  \item Virus, logiciels espions: envoyez le fichier de contacts.
  \item Hoax, chaîne de lettres.
\end{itemize}

\subsection{Fighting spam:}
\subsubsection{Protection: Filtres}
Logiciel de filtrage peut trier le spam des emails légitimes.
Les filtres peuvent être basées sur:
\begin{itemize}
  \item Contenu du message.
  \item Formats de message.
\end{itemize}
Filtres ne sont pas parfaits et ne peuvent pas éviter les faux positifs ou de faux négatifs.
exemple de filtre: SpamAssassin
\subsubsection{Filtre: listes noires}
Logiciel de filtrage (Ex. Spamhaus ) utilisant les listes SBL et XBL.
\begin{itemize}
  \item SBL: adresses IP des opérateurs de spam connus.
  \item XBL: adresses IP des systèmes détournés s'appuyant spams.
\end{itemize}
Les serveurs de messagerie peuvent utiliser cette base pour filtrer le trafic entrant.
Avantages:
\begin{itemize}
  \item Pas cher.
  \item Facile à mettre en pratique, mais nécessitent une liste noire centralisée pour que la technique soit efficace.
\end{itemize}
Inconvénient:
\begin{itemize}
  \item Beaucoup de faux négatifs.
  \item Gestionnaire de la liste noire doit être réactif.
\end{itemize}

\subsubsection{Filtre: listes blanches}
Au lieu de bloquer des e-mails, la réception n'est possible que si l'expéditeur (domaine, adresse IP, etc)
appartient à la liste blanche.
Avantages:
\begin{itemize}
  \item Pas cher.
  \item Facile à mettre en pratique.
\end{itemize}
Inconvénient:
\begin{itemize}
  \item Expéditeurs autorisés doivent être connus a priori.
  \item Beaucoup de faux positifs.
\end{itemize}

\subsubsection{Filtre: Spam Database}
Base de données maintenue sur un serveur centralisé. Le logiciel vérifie si le message reçu apparaît
dans la base de données.
Avantages:
\begin{itemize}
  \item La base de données est partagée.
  \item Peu de faux positif.
\end{itemize}
Inconvénient:
\begin{itemize}
  \item Ne pas détecter variante de spam.
  \item Exiger calcul et de bande passante.
  \item Beaucoup de faux négatifs.
\end{itemize}

\subsubsection{Filtre: Les listes grises}
L'idée de base est de bloquer un mail si le comportement du serveur de l'expéditeur est anormal.
Le serveur destinataire gère une base de données qui contient des «triplet» pour chaque courrier
entrant:
\begin{itemize}
  \item L'adresse IP de l'hôte de connexion.
  \item L'adresse de l'expéditeur de l'enveloppe.
  \item L'adresse du destinataire de l'enveloppe.
\end{itemize}
Lorsque le serveur reçoit un e-mail, il vérifie s'il appartient déjà à la base de données.
Si non, le courrier est greylisted pour une courte durée et un message d'erreur est renvoyé au serveur
expéditeur. Si le serveur de l'expéditeur est conforme à RFC2821, il va réessayer la transmission après
au moins 30 minutes. Le courrier greylisted est alors débloqué.
Avantages:
\begin{itemize}
  \item Pas ou peu de faux négatifs.
\end{itemize}
Inconvénient:
\begin{itemize}
  \item Les retards dus au greylisting.
\end{itemize}
\subsubsection{Solutions alternatives :}
\begin{description}
  \item[Enregistrer les ordinateurs / utilisateurs.]
    Désigner les machines autorisées à envoyer des e-mail avec une adresse d'expéditeur
    d'origine dans le domaine (une norme est SPF - Sender Policy Framework).
  \item[Challenge / Response.]
    L'utilisateur doit répondre à un défi afin d'être ajouté à la liste blanche.
  \item[Ajout coût aux e-mails.]
    L'ordinateur doit effectuer un calcul pour envoyer un e-mail.
\end{description}

\section{Malwares \& Exploits}
\subsection{Malwares}
\paragraph{Définition :}
\begin{description}
  \item[Virus] fragment de code qui se propage à l'aide d'autres programmes.
  \item[Worm] programme indépendant.
  \item[Cheval de Troie (Trojan)] programme utile qui contient un programme malveillant (ou lui-
même par extension).
  \item[Backdoor] accès caché à un ordinateur.
Le Programme permet d'accéder à distance à un système, sans que l'utilisateur le sache.
Ils sont installé par les chevaux de Troie et donc souvent classés comme tels.
  \item[Spyware] programme qui envoie les informations vous concernant à des tiers.
  \item[Adware] programme qui provoque l'apparition d'annonces sur votre écran ou des
    changements des résultats de votre recherche (manipule le navigateur)
  \item[Rootkit] programme qui masque la présence de programmes malveillants sur votre ordinateur.
    Il modifie votre système d'exploitation afin de cacher l'exécution des programmes,
    des fichiers ou des configurations.
\end{description}

\paragraph{La période classique}
Propagation passive à travers l'échange de disquettes. La propagation est lente et donc le virus à
besoin d'être efficace.
\begin{description}
  \item[Les virus Stealth]$ $
    \begin{description}
      \item[Simple] le virus comprime le fichier original, et crée un fichier infecté de la même taille.
      \item[Complexe] le virus modifie le système de manière à être invisible. Il modifie le fichier routines
        de lecture afin qu'ils ne révèlent pas le virus.
    \end{description}
  \item[Les virus polymorphes]
    Le virus se modifie après chaque infection de manière à rester indétectable.
\end{description}

\paragraph{La période moderne}
Les virus modernes utilisent Internet pour propager activement et peuvent infecter la planète en
quelques heures. Ils sont souvent, simple et facile à détecter. Cependant, ils sont efficace, car ils se
propagent beaucoup plus rapidement que le logiciel antivirus ne peut être mis à jour.

\subsection{Exploits on the web}
\paragraph{Exploits}
La plupart des logiciels contiennent des défauts.
Ces bugs peuvent être exploitées par des pirates.
Un `exploit' est une méthode ou un script qui permet d'exploiter les bugs.
Le plus intéressant sont exploits sur les serveurs.
Ils peuvent se faire à distance et les serveurs ont
souvent des privilèges plus élevés.

\subsubsection{Directory Traversal}
Les documents du serveur Web sont accessibles à partir d'une racine. Si le serveur ne vérifie pas l'URL,
nous pouvons accéder à d'autres fichiers.
Exemple classique: traversée de répertoire.
\begin{itemize}
  \item \verb|../../../../|
  \item Scripts faibles.
\end{itemize}
Pour éviter le Directory Traversal, certaines application web vérifie la chaîne : \verb|..| et \verb|../..| et \verb|/| dans l'URL.
Cependant, ces applications sont encore vulnérables au encodage ``pour cent'' de l'URL (\verb|?%2e%2e/|
qui est à \verb|../,?..%2F| qui est de \verb|../..|.),
mais aussi l'encodage UTF8, etc. La Sécurité de l'application
dépend du codage autorisé par le serveur web.

\subsubsection{Cross-Site Scripting}
Le «Cross-site» consiste à insérer un script dans une page Web.
Souvent, la script lui-même est situé
sur un autre serveur, c'est du "Cross-site scripting".
(Exemple : <script src=http://www.evilsite.com/hack.js> )
Dangers:
\begin{itemize}
  \item Réorientation de la session (par exemple, sur une copie du site original).
  \item Afficher de fausses informations.
  \item Afficher des formulaires de collecte de données (phishing).
  \item Vole des cookies.
\end{itemize}

\subsubsection{Phishing}
Le but est d'obtenir des informations et identification se faisant passer pour une entité digne de
confiance.
Le plus souvent par email, l'adresse ressemble à l'adresse valable.
Par exemple
\url{www.mybanck.com} ou
\url{http://www.mybank.com.example.com}.
On peut aussi avoir une image avec un lien hypertexte.

\subsubsection{SQL Injection}
Exploitation d'une faille de sécurité d'une application interagissant avec une base de données, en
injectant une requête SQL non prévue par le système et pouvant compromettre sa sécurité.

\subsection{Buffer Overflows}
\subsubsection{Exploit: Buffer Overflow}
Si un programme ne vérifie pas la quantité de données qu'il reçoit, il court le risque d'écraser la zone
mémoire contenant des variables, des codes ou des adresses de saut.
Connaissant bien l'architecture de la machine, nous pouvons fournir le code machine qui va être exécuter.
\subsubsection{Difficultés}
Les Buffer Overflow ne sont plus un problème de sécurité majeur.
Les programmes sont maintenant mieux protégés contre le Buffer Overflow et concevoir un Buffer overflows intelligents est complexes.
En effet, il est difficile de deviner la croissance de la pile et choisir adr et le programme injecté ne peut
pas avoir un \verb|\0| octet,
parce que cela marque la fin d'une chaîne de caractères peut empêcher la copie des octets restants.

\section{Network Attacks}
\subsection{Denial of Service}
\subsubsection{Ping of death}
La taille normale d'un Ping est de 64 octets (84 avec en-tête).
On envoie des paquets IP qui dépassent la taille légale maximale (65535 octets).
C'est l'une des premières attaque par déni de service.
Unix, Linux, Mac, Windows, imprimantes, routeurs et étaient vulnérables ($<1997$).
Rappel: TCP Handshake :
L'établissement d'une connexion TCP se fait par un handshaking en trois temps:
\begin{enumerate}
  \item Le client envoie un segment $\syn(x)$ au serveur, avec $x$ son sequence number.
  \item Le serveur lui répond par un segment $\syn+\ack(y,x+1)$, avec $y$ son sequence number.
  \item Le client confirme par un segment $\ack(x+1,y+1)$.
\end{enumerate}

\subsubsection{Syn-Flooding}
On envoit un grand nombre de demandes ouverture de connexion TCP ($\syn$).
À la réception du paquet $\syn$,
le serveur alloue de la mémoire nécessaire pour la connexion et l'inscrit dans une file d'attente de
connexions à moitié ouvertes. Cette situation n'ayant pas été prévu, le serveur ne peut plus accepter
de nouvelles connexions une fois que les débordements de file d'attente.
L'attaquant peut forger l'adresse source de ses paquets $\syn$ rester anonyme.
Les versions récentes des systèmes d'exploitation (Windows, Unix)
sont protégés contre de telles attaques.
Protections contre les $\syn$ Flooding :
\begin{itemize}
  \item Augmenter la taille de la file d'attente.
  \item Réduire délai pendant lequel le serveur attend un $\ack$.
  \item Dropper le plus ancien $\syn$ dans la file d'attente.
  \item Filtrage, par exemple sur les adresses IP.
  \item $\syn$-Cache: cache le $\syn$ et envoyer un $\syn$ / $\ack$. Si les $\ack$ arrive, une connexion complète
    est créée.
  \item $\syn$-Cookies: Une fois que la file d'attente de connexion est presque rempli, le serveur utilise
    les cookies $\syn$.
\end{itemize}


\subsubsection{Smurf Attack}
Le but est de noyer la cible à l'aide d'amplificateurs de circulation.
Cas typique: ICMP echo-request (ping).
\begin{itemize}
  \item Le pirate envoie un paquet de ping avec l'adresse cible
comme adresse source.
  \item La machine ``pings'' envoie sa réponse à la cible.
  \item Si le pirate envoie le paquet à une adresse de diffusion
    (broadcast) , toutes les machines du réseau répondront à
    la cible.
\end{itemize}
Protection:
\begin{itemize}
  \item Configurer les hôtes et les routeurs individuels afin de ne pas répondre aux requêtes ping à
    une adresse de broadcast.
  \item Configurer les routeurs afin de ne pas transmettre les paquets dirigés vers des adresses de
    broadcast.
\end{itemize}

\subsubsection{DDoS (Distributed Denial of Service)}
Pour augmenter l'efficacité du déni de service, les pirates vont pirater plusieurs machines et installer
des agents. Plusieurs machines maîtres contrôlent les agents. Le pirate envoie ensuite des commandes
aux maîtres qui, vont exécuter l'attaque ordonnée par les agents.
La puissance (bande passante) de l'attaque est multipliée par les agents.
Il est plus difficile de retracer les pirates (2 couches intermédiaires).
Puisque l'attaque provient de plusieurs sources, elle est beaucoup plus difficile à filtrer.

\subsection{Spoofing}
\subsubsection{IP Spoofing}
Dans certains cas, l'adresse IP source est utilisé pour autoriser une connexion.
Les routeurs et les pare-feu peuvent filtrer les paquets en fonction de leur source.
Certains programmes (rlogin, rsh) peuvent
autoriser certaines sources à se connecter sans authentification.
Il est facile d'usurper l'adresse source d'un paquet et d'abuser de la confiance de cette source.
La réponse à un message forgé est envoyé à l'adresse falsifiée.
Facile à utiliser avec les protocoles basés sur UDP.

\subsubsection{TCP/IP Spoofing}

\subsubsection{TCP/IP Spoofing Dans un LAN}

\subsubsection{TCP/IP Spoofing De l'extérieur}
Prédiction de l'ISN (Incremental Sequence Number):
La norme originale (RFC 793) exige que l'ISN est incrémenté une fois tous les quatre microsecondes.
Dans certaines implémentations TCP simples le prochain ISN peut être prédit.
La procédure du Hacker pour prédire l'ISN:
\begin{itemize}
  \item Il ouvre quelques liens authentiques (par exemple SMTP) pour obtenir les échantillons
    d'incrémentation d'ISN actuels.
  \item Il lance sa connexion forgée en utilisant le dernier ISN plus un incrément obtenu à partir de ces
    échantillons.
  \item Il peut lancer plusieurs connexions forgés avec des incréments différents en espérant qu'au
    moins un d'entre eux est correct.
\end{itemize}

\subsubsection{ARP Spoofing}
ARP = Address Resolution Protocol.
L'ARP est un protocole qui permet de trouver une adresse de la couche 2 (Ethernet) à partir d'une
adresse de couche 3 (IP). Il Très simple et non sécurisé:
\begin{itemize}
  \item Un client demande qui connaît l'adresse Ethernet 10.1.2.3?
  \item Une personne répond 10.1.2.3 a l'Adresse Ethernet 010203040506.
\end{itemize}
Il est facile de forger des réponses (même les non-sollicité) pour rediriger le trafic!

\subsection{Sniffing}
De nombreux protocoles utilisent l'authentification par texte claire (pas d'encodage).
Par l'écoute du traffic sur une partie du réseau, nous pouvons obtenir les noms d'utilisateur et mots de
passe. Un mot de passe permet d'accéder à un ordinateur distant à partir de laquelle nous pouvons
renifler à nouveau et obtenir des nouveaux mots de passe.

\subsection{Session Hijacking}
\subsubsection{Modem Session}
Le modem permet d'accéder à une ligne série (par ex. Accès à distance).
Un utilisateur peut supprimer la ligne sans quitter la session en ligne, la session du terminal reste alors
actif pendant un certain temps. L'utilisateur suivant (ou le hacker) qui se connecte au modem peut
trouver la session de l'utilisateur précédent et l'utiliser.

\subsubsection{TCP Session}
Si un pirate peut espionner sur une connexion TCP, il peut insérer un paquet TCP avec des numéros de
séquence corrects.
L'insertion d'un paquet supplémentaire dans une connexion TCP crée une avalanche de paquets:
\begin{itemize}
  \item La source, qui n'a jamais envoyé le paquet, n'est pas d'accord avec le numéro de séquence
    reconnue et émet un accusé de réception.
  \item La destination, qui a vu le paquet,
    insiste sur le numéro de séquence et envoie un accusé de réception.
\end{itemize}

\subsubsection{HTTP Session}
Le protocole HTTP n'a pas le concept d'une session.
Il est fait de requêtes/réponses indépendants.
Les sites d'e-commerce utilisent des moyens artificiels pour reconnaître les demandes appartenant à
une session:
\begin{itemize}
  \item Cookies.
  \item URL personnalisé.
\end{itemize}
Si le pirate peut espionner ces données, il peut créer des demandes qui feraient partie de la même session.

\section{Firewall}
\subsection{Introduction}
\subsubsection{Introduction: Firewalls}
Les pare-feu de réseau doivent empêcher la propagation d'une attaque, tout en permettant la
circulation souhaité de le traverser.
Un pare-feu peut être réalisé en une ou plusieurs composantes. Les pare-feu peuvent être logiciel ou
de matériel.

\subsubsection{Types de pare-feu}
\begin{itemize}
  \item Logiciel: Poste de travail standard avec un logiciel de pare-feu.
  \item Hardware: Boîte noire spécialisé (qui contient également des logiciels)
\end{itemize}
Logiciel vs Hardware
\begin{itemize}
  \item Pare-feu logiciels héritent de toutes les vulnérabilités du système d'exploitation sur lequel ils
    s'exécutent.
  \item Architectures de pare-feu logiciels sont bien connus, il est plus facile d'exploiter ses
    vulnérabilités (ex. débordement de la mémoire tampon).
  \item Pare-feu logiciels ont souvent une meilleure performance: ils bénéficient des progrès rapides
    et des prix bas dans le matériel PC.
\end{itemize}

\subsection{Basic Principles - The sevens Principle}
\begin{itemize}
  \item Least privileges:
    Chaque élément d'un système (utilisateur, logiciel) doit seulement avoir les droits minimaux
    nécessaires pour mener à bien sa tâche.
    Exemples:
    \begin{itemize}
      \item Les utilisateurs réguliers ne doivent pas être administrateurs.
      \item Les administrateurs doivent également utiliser des comptes d'utilisateurs réguliers.
      \item Un serveur Web s'exécute sous un compte non-privilégié.
    \end{itemize}
  \item Defense in depth:
    Plusieurs mesures de sécurité sont mieux qu'un seul.
    Exemple:
    \begin{itemize}
      \item Anti-virus sur les serveurs de messagerie et les ordinateurs de bureau.
      \item Nous avons aussi des machines sécurisées (configuration, patches) qui sont protégés par un
        pare-feu.
      \item Même si les connexions FTP sont bloqués par le pare-feu, les postes de travail ne doivent pas
        exécuter des serveurs FTP.
    \end{itemize}
  \item Choke point:
    Il est plus facile de contrôler la sécurité si toutes les données doit passer par un point donné.
    \begin{itemize}
      \item Les utilisateurs ne devraient pas être autorisés à se connecter aux modems de leurs machines.
      \item Interconnexions avec d'autres sociétés doivent passer par le pare-feu.
    \end{itemize}
  \item Weakest link:
    Le pare-feu est aussi sûr que son maillon le plus faible.
    \begin{itemize}
      \item C'est inutile de dépenser de l'argent pour protéger une partie de la FW si les autres parties ne
        sont pas protégés.
    \end{itemize}
  \item Deny by Default:
    Il est préférable d'interdire tout ce qui n'est pas explicitement autorisé plutôt que d'autoriser tout ce
    qui n'est pas explicitement interdit.
    \begin{itemize}
      \item Nous ne pouvons jamais savoir à l'avance toutes les menaces auxquelles nous seront exposés.
    \end{itemize}
  \item User participation:
    Un système de protection est efficace seulement si tous les utilisateurs appuient.
    \begin{itemize}
      \item Le but d'un pare-feu est d'autoriser tout ce qui est utile et en même temps d'éviter les dangers.
      \item Nous devons comprendre les besoins de l'utilisateur et assurez-vous que les raisons des
        restrictions sont bien comprises par eux.
    \end{itemize}
  \item Simplicity
    La plupart des problèmes de sécurité proviennent de l'erreur humaine.
    Dans un système simple:
    \begin{itemize}
      \item Le risque d'erreur est plus faible.
      \item Il est plus facile de vérifier son bon fonctionnement.
      \item Surtout dans l'évolution des réseaux.
      \item Surtout avec plusieurs administrateur.
    \end{itemize}
\end{itemize}

\subsection{FW Features}
\subsubsection{Stateless vs Stateful}
\begin{itemize}
  \item Sans mémoire (stateless): Ne se souvient pas que les paquets qu'il a déjà vu.
  \item Avec la mémoire (stateful): Garde une trace des paquets qui passent et peux reconstruire
    l'état de chaque connexion, ou même certains protocoles.
\end{itemize}

\subsubsection{Packet Analysis}
Un FW peut analyser les paquets pour vérifier leur format et leur contenu.
\begin{itemize}
  \item Permet l'élimination des paquets malformés (DOS, exploits, par exemple. Ping de la mort).
  \item Permet l'élimination des paquets qui ne correspondent pas à l'état actuel du protocole.
  \item Permet d'éliminer des paquets dont le contenu indésirable.
\end{itemize}

\subsubsection{Filtering}
Le filtrage permet de limiter le trafic à des services utiles. Il peut être basé sur plusieurs critères:
\begin{itemize}
  \item l'adresse IP src ou l'adresse IP dst.
  \item Protocoles (TCP, UDP, ICMP, ...) et les ports.
  \item Drapeaux et options (syn, ack, le type de message ICMP, ...).
\end{itemize}

\subsubsection{Network Address Translation (NAT)}
Etant donné que les adresses IP publiques sont rares, au lieu de réserver 256 adresses pour 100 postes
de travail, nous pouvons cacher ces 100 postes de travail derrière une seule adresse.

\paragraph{Principe de base}
\begin{itemize}
  \item Nous utilisons les adresses privées dans le réseau interne et un / plusieurs adresses publiques
    pour communiquer avec Internet.
  \item Quand un paquet quitte le réseau interne, nous remplaçons son adresse source par une
    adresse publique.
  \item Quand un paquet arrive de l'Internet, nous remplaçons la destination du public par une
    adresse privée
  \item Nous utilisons une table de traduction pour stocker les relations entre les adresses internes et
    externes.
\end{itemize}

\subsubsection{Propriétés de NAT Dynamique}
Lorsque deux connexions ne se différencient que par leur adresse interne, nous avons une collision).
\begin{itemize}
  \item Nous pouvons changer le port source (port et l'adresse Translation - PAT).
  \item Nous pouvons utiliser un pool d'adresses publiques et utiliser différents des adresses de
source.
\end{itemize}
NAT dynamique ne permet pas l'établissement entrant connexions (une bonne protection par défaut).
NAT statique:
Pour autoriser les connexions entrantes, nous devons définir certaines entrées statiques dans la table
de traduction. Typiquement, nous créons une entrée par protocole (SMTP, HTTP, ...). Différents ports
de la même adresse externe peut conduire à différentes adresses internes.
Problèmes et limites
\begin{itemize}
  \item TCP et UDP checksums
  \item Certains protocoles ne supportent pas les modifications de paquets: IPSec.
  \item Certains protocoles d'échanger leurs adresses IP
  \item FTP, car peut avoir des connexions dans les deux sens (serveur - client)
Avantages
  \item Moins adresses publiques, des coûts limités
  \item Facile à changer de fournisseur d'accès
  \item Facile à réorganiser le réseau interne
  \item Effet de protection automatique (réf. Static NAT)
  \item Masque de la structure du réseau interne
Authentication:
Le FW peut exiger une authentification afin d'obtenir une connexion.
  \item Outbound: permet de limiter l'accès à Internet uniquement pour les utilisateurs privilégiés.
  \item Entrant: permet d'autoriser l'accès aux ressources internes pour les employés en voyage.
Remote network access:
Le FW permet aux utilisateurs externes d'accéder au réseau local. Pour cela, l'utilisateur externe
établit une connexion cryptée (un tunnel) avec le FW et se retrouve comme s'il était à l'intérieur du
réseau local. La connexion peut se faire via Internet ou modem.
Encryption:
Un FW peut crypter ou décrypter des données qui traverse une zone moins sûr.
Logging:
Il est important d'être protégé mais il faut aussi savoir quand nous sommes attaqués et réagir en
conséquence.
  \item Les logs garder une trace de tentatives d'attaques.
  \item Ils permettent aussi de vérifier que les ports ou les destinations que nous autorisons sont
vraiment nécessaires (moindre privilège).

\subsection{5.4 Architectures}
Personal Firewall:
Le pare-feu personnel interdit d'abord toutes les connexions, et à chaque alarme, l'utilisateur peut
autoriser l'application à se connecter, pour une fois ou pour toujours.
NAT + filtering:
Configuration.
\begin{itemize}
  \item NAT dynamique pour toutes les machines internes.
  \item NAT statique pour tous les serveurs accessibles.
  \item Filtrage sortant et filtrage entrant.
Limites.
  \item Aucune analyse de contenu (virus) à partir d'Internet.
  \item Les connexions directes sur des serveurs internes (exploits, DoS).
Application.
  \item Faible niveau de sécurité.
  \item Pas un grand serveur Web public.
FW with demilitarized zone:
La zone démilitarisée (DMZ) n'est relié ni à Internet ni au réseau interne.
Configuration
  \item Machines internes peuvent seulement se connecter au proxy.
  \item Seul le proxy peut se connecter à Internet.
  \item Outbound NAT dynamique.
  \item Inbound NAT statique vers le proxy.
  \item Filtrage sortant et filtrage entrant.
Limitations (de l'exemple, pas du DMZ).
  \item Le pare-feu est un point critique.
  \item Tous les services passent par le même proxy, une vulnérabilité sur un seul service peut donner
accès à l'ensemble du trafic.
Application.
  \item Sécurité moyenne a besoin.
\end{itemize}

Sandwiched demilitarized zone (DMZ):
Configuration.
  \item Machines internes peuvent seulement se connecter aux poxies (un protocole par proxy).
  \item Seules les proxies peuvent se connecter à l'Internet.
  \item Pas de routage dans proxies.
  \item Outbound NAT dynamique, statique entrant.
  \item Filtrage sortant et le filtrage entrant.
Applications.
  \item Haute sécurité.
5.5 Filtering rules :
Règles de filtrage:
Les règles de filtrage sont spécifiées dans une liste.
Le pare-feu parcourt la liste jusqu'à ce qu'il trouve une règle qui s'applique et exécute alors l'action
spécifiée par la règle et se déplace sur le prochain paquet.
La dernière règle doit interdire tout ce qui n'a pas été autorisée.
Organisation des règles de filtrage:
L'ordre dans lequel les règles sont spécifiées est important.
Quand il ya beaucoup de règles, il est important de les organiser de manière systématique.
Méthode :
  \item Nous définissons un niveau de sécurité pour chaque zone.
  \item Nous règles du groupe par zones dans l'ordre de niveau de sécurité décroissant.
  \item Chaque groupe se compose de quatre parties:
o Autorisations explicites pour le trafic entrant.
o Interdiction générale pour le trafic entrant.
o Autorisations explicites pour le trafic sortant.
o Interdiction générale pour le trafic sortant.
Propriétés
  \item Pour chaque zone, il suffit de constater l'écoulement vers les zones moins sécurisées.
  \item Le flux vers des zones plus sûres ne peut plus être influencé (but de l'opération): «tout» se
réfère à des niveaux inférieurs.
  \item Une règle qui implique 2 zones apparaît dans le bloc relatif à la zone plus sûre.
  \item Le bloc liée à la dernière zone est vide.
  \item La dernière règle (any-any) ne doit pas être nécessaire. En activant les journaux sur cette règle,
nous pouvons détecter les erreurs éventuelles.
- 18 -
6. Proxies :
6.1 Proxies' features and benefits :
Le proxy est un intermédiaire entre le client et le serveur. Il fait office de serveur pour le client et de
client pour le serveur. Il empêche le réseau interne d'avoir une connexion directe avec Internet.
Un serveur proxy permet de réaliser :
  \item Un goulot d'étranglement (chokepoint).
  \item Un accès sur authentification : pour limiter l'accès à Internet à un certain nombre d'utilisateur.
  \item Un système de cache : pour avoir un taux de transfert plus rapide et économiser sa bande
passante, le proxy va garder une copie locale des documents téléchargés récemment.
  \item Un filtre sur le contenu des paquets : DNS ou URL blacklisté, détection de virus, filtre un mot
clé, une URL ou un MIME.
  \item Un reformatage du contenu des paquets : pour transformer une page dans le format des PDAs
et des Smartphones par exemple.
  \item Faire des tests sur un réseau sans devoir sortir de celui-ci (en passant par un proxy situé à
l'extérieur).
  \item Permet de surfer anonymement au vu du serveur final mais le proxy peut conserver des traces
du trafic et donc retrouver la source des requêtes.
6.2 Types of Proxies :
Transparent Proxy :
Le serveur proxy transparent ne requière pas de configuration au niveau du navigateur web du client.
Il est aussi appelé « intercepting proxy ». Tout le trafic avec le port 80 est automatiquement redirigez
vers le serveur proxy par le firewall. Il est possible de détecter un proxy transparent par l'adresse IP
(non NATé) ou encore le header http.
Avantage :
  \item Permet de forcer le client à utiliser le serveur proxy.
  \item Pas de configuration chez le client.
  \item Possibilité de faire du load balancing.
Limitation :
  \item Cela ne fonctionne pas pour tous les serveurs web (ceux n'utilisant pas les ports standards)
FTP Proxy:
Le protocole FTP établi deux connexions : une pour transmettre les actions à réaliser, une autre pour
transmettre les documents. Le choix du port source pour le transfert des documents peut se faire du
côté serveur (passif mode) ou du côté client (actif mode). Initialement, le protocole FTP n'a pas été fait
pour être utilisé à travers un proxy. Il existe quand même deux manières d'y parvenir :
1. FTP Proxy par HTTP:
Par le biais d'un navigateur web en indiquant l'URL (FTP://server.com/file.txt). Le client va
alors faire une requête HTTP vers le proxy qui va la convertir en une requête FTP pour ensuite
retourné le fichier sous la forme d'une réponse HTTP.
- 19 -
2. FTP Proxy par User@:
L'utilisation du user@ fonctionne comme une requête FTP standard. Pour indiquer au proxy
les informations de connexion, le client doit indiquer son login@server.com pour établir une
connexion avec le proxy. Celui-ci va ensuite établir une connexion entre lui et le serveur FTP
final. La connexion peut être active ou passive.
SMTP Proxy:
Il est tout à fait possible de configurer un serveur proxy pour le protocole SMTP car celui-ci est de type
hop by hop.
  \item Mail sortant : Le serveur proxy est désigné comme étant le serveur SMTP pour les mails
sortant.
  \item Mail entrant : Le serveur proxy doit être enregistré dans le serveur DNS comme étant le SMTP
pour ce domaine. Il doit être également configuré pour transmettre les mails au vrai serveur
SMTP.
DNS Proxy:
Tout comme le proxy SMTP, le proxy DNS transfert les requêtes vers le vrai serveur. Il est possible
d'implémenter un cache au sein du serveur proxy.
SOCKS Proxy:
Serveur proxy générique pour les connexions TCP et UDP. Il reçoit une connexion du client et en recrée
une vers le serveur. Le SOCKS proxy ne prend pas en compte le type de protocole, il ne fait que
transmettre les paquets.
HTTPS Proxy:
HTTPS est la version sécurisé du HTTP. Il n'est bien-sûr pas possible de faire un serveur HTTPS proxy
sécurisé car cela va à l'encontre du principe du HTTPS. Il joue alors simplement un rôle de transfert
entre le client et le serveur (un peu comme le SOCKS Proxy). Les ports disponibles sont souvent limités
aux ports 443 et 563. Pour passer à travers un firewall avec n'importe quel protocole, il suffit de
démarrer le serveur sur le port 443 et de passer par un proxy HTTPS.
- 20 -
6.3 Reverse Proxies :
Lorsque le serveur proxy se trouve du côté client, le client sait qu'il parle à un proxy et peut adapter
ses requêtes mais lorsque le serveur proxy se trouve du côté serveur, le client ne sait pas qu'il a affaire
à un proxy. Le serveur proxy doit donc se comporter exactement comme un serveur. Pour les
protocoles qui ne supportent pas le relaye (HTTP, FTP), le proxy ne peut relayer le paquet qu'à un seul
serveur.
Un proxy HTTP :
  \item Permet de filtrer les requêtes vers le server.
  \item Une authentification du client (impossible d'attaquer le serveur avant de s'être authentifié).
  \item Accélérer le serveur : utiliser un cache, déléguer au proxy les fonctions statice (accessibilité de
documents statices), Workload dispatcher (??).
Un proxy HTTPS :
  \item Un proxy HTTPS permet de décharger le serveur des calculs souvent gourment d'encryptage et
e décryptage. C'est donc le proxy qui s'en charge et le liaison proxy serveur se fait en HTTP.

Les serveurs proxy peuvent utiliser plusieurs protocoles différents pour un même paquet. Ainsi
un web mail peut recevoir une requête HTTPS et générer une requête IMAP vers le serveur
mail.

\section{IDS}
Systèmes de détection d'intrusion (IDS)
Les systèmes de détection d'intrusion (IDS) analysent le trafic réseau (réseau IDS, NIDS), généralement
en face du pare-feu et/ou les événements sur les serveurs (Host IDS, HIDS).
En cas d'attaque, ils lancer une alarme (SMS, courrier, etc) et permettent de reconfigurer le pare-feu
(filtrage de l'attaque) ou les serveurs.
L'analyse peut se faire en temps réel ou en analysant les logs.
IDS avec caractérisation du trafic (IDS with Traffic Characterization):
IDS qui réalise des statistiques sur le trafic.
Si la valeur va au-delà de ses limites habituelles, alors l'IDS suppose qu'il y a une attaque.
  \item Ce système peut reconnaître de nouvelles attaques.
  \item Il peut aussi ne pas les reconnaître ... (faux négatifs).
  \item Il peut voire attaques où il n'y a pas (faux positifs).
Le taux élevé auquel il génère des faux positifs rend ce type d'IDS impopulaire.
IDS basés sur les signatures (Signature-based IDS):
L'IDS dispose d'une base de données avec les attaques connues.
Il ne reconnaît pas de nouvelles attaques (doit être constamment mis à jour).
Les faux négatifs.
  \item Attaques manuelles peuvent avoir des variations qui ne sont pas détectés.
  \item Les signatures sont pas toujours précis.
Les faux positifs.
  \item L'IDS, dans la plus part des cas, ne sait pas si une tentative d'attaque a réussi.
  \item L'IDS ne sait pas si la cible de l'attaque est vulnérable (par exemple attaque Linux sur serveur
Windows).
Exemple d'IDS basés sur les signatures : Snort
IDS basé sur l'intégrité de l'host (Integrity-based Host IDS):
Tripwire est un exemple typique d'un HIDS avec une analyse différée.
  \item Il crée une signature numérique de tous les fichiers et répertoires qui ne doivent pas être
modifiés.
  \item Les signatures ne peuvent pas être modifiées par un pirate.
  \item Il compare régulièrement les fichiers et les signatures pour détecter d'éventuelles
modifications.
  \item Il génère une alarme lorsqu'il détecte une modification et peut restaurer automatiquement la
version originale du fichier.
Systèmes de prévention des intrusions (IPS):
Un IPS est un IDS qui réagit à une attaque.
  \item Niveau IP: Filtres l'adresse IP source dans le pare-feu (pour un temps).
  \item Niveau TCP: Envoie un spoofed paquet TCP 'reset' à la destination pour tuer la connexion.
  \item Niveau application: «corrige» une requête Web pour enlever les caractères spéciaux.

\section{Introduction of Cryptography}
\subsection{Vocabulary}
Cryptographie :
Définition (cryptographie):
  \item Origine: Science qui permet condentiality de communication à travers des canaux non
sécurisés.
  \item Aujourd'hui: Condentiality, l'authentification, l'intégrité des données (non-répudiation,
Disponibilité, Signature).
Définition (Condentiality, Secrecy)
  \item Assurance que l'information donnée ne peut pas être consulté par des tiers non autorisés.
Encryption Algorithm:
Définition (algorithme de chiffrement):
  \item Algorithme qui transforme un texte intelligible dans un texte qui est inintelligible pour les
parties non autorisées. Il est généralement paramétré avec une clé cryptographique.
Cryptographie asymétrique à clé:
  \item Pour encrypter, décrypter, l'algorithme de décryptage.
  \item Un cryptosystème = enc / dec algorithmes (+ génération de clé).
Cryptographie à clé symétrique:
  \item Pour chiffrer, déchiffrer.
  \item Un ciffer: algorithme de chiffrement à clé symétrique.
Plaintext \& Ciphertext:
Définition (Plaintext)
  \item Entrée d'un algorithme de chiffrement.
Définition (Cyphertext)
  \item Sortie d'un algorithme de chiffrement.
Définition (Cleartext)
  \item Les informations codées en utilisant un code public.
Cryptanalysis:
Définition (cryptanalyse)
  \item Science qui confirme ou infirme la sécurité d'un système de cryptographie.
Définition (Casser un système de chiffrement)
  \item Prouver l'insécurité d'un système de cryptographie.
- Décryptage (totalement ou partiellement) un texte chiffré donné.
- Récupérer la clé de la cryptographie.
- Prouver qu'un système de chiffrement est moins sûr que ce qui est revendiqué.
Définition (cryptologie)
  \item Sciences de la cryptographie et la cryptanalyse.
- 23 -
8.2 Symmetric-key vs Asymmetric-key Cryptography :
La cryptographie à clé symétrique:
  \item Même clé secrète est utilisée pour le chiffrement et le déchiffrement.
  \item Taille typique de la clé: 128 bits.
  \item Basé sur des opérations sur les bits ou octets (substitutions, permutations).
  \item Rapide dans les implémentations logicielles ou matérielles.
La cryptographie à clé asymétrique:
  \item Il s'agit d'une clé publique pour crypter et une clé privée pour décrypter.
  \item Taille typique de la clé: 1024 bits.
  \item Basé sur des problèmes mathématiques (FACT par exemple, DL)
  \item Cryptages et décryptages sont lents.
Comparaison Sym / Asym Cryptographie:
  \item Speed.
  \item Key-size.
  \item La distribution des clés.
8.3 Authentication :
Authentification d'entité:
  \item Asymétrique ou symétrique.
  \item Exemples: Challenge / Response, Zero-Knowledge.
Authentification des données:
  \item Asymétrique (signature) ou symétrique (contrôle d'intégrité).
  \item Exemples: RSA, MAC.
8.4 Cryptographic Hash Functions :
Idée intuitive:
$h: \{0,1\} * \to \{0,1\}^n.$
exemple: MD5, SHA-1.
Propriétés:
Définition (First preimage resistance)
  \item Compte tenu d'une valeur de hachage y, il est impossible de trouver m tel que h (m) = y.
Définition (Second preimage resistance)
  \item Étant donné un message m1, il est impossible de trouver un message différent, m2 tel que
h (m1) = h (m2).
Définition (Collision Resistance)
  \item Il est impossible à deux messages différents m1 et m2 tels que h (m1) = h (m2).
Définition (Random Oracle Property)
  \item h (m) est in-distinguable d'une valeur de n bits aléatoires.
- 24 -
Théorème du paradox de l'anniversaire:
Si nous choisissons $\omega$ N nombres aléatoires, indépendant et uniforme, dans {1,2,3, ..., N}, nous
obtenons au moins un chiffre deux fois avec une probabilité:
8.5 Algorithm and Key :
ES keys = 128 bits, 192 bits ou 256 bits.
DES keys = 64 bits (56 key bits + 12 salt bits).
AES block = 128 bits.
DES block = 64 bits.
SHA1 (output) = 160 bits.
MD5 (output) = 128 bits.
RSA keys = 1024 bits, 2048 bits ou 4096 bits.

\section{Certificates}
9.1 Basics :
Bases:
L'objectif d'un certificat est de relier une clé publique avec son propriétaire.
Le couple (clé publique, propriétaire) est signé par un tiers de confiance appelé autorité de
certification (CA). Pour vérifier la signature, la clé publique de le CA est nécessaire.
Le couple (clé publique du CA, CA) est auto-signé: certificat racine. L'authenticité du certificat racine
est fondamentale.
Certificats X.509:
Norme de l'Union internationale des télécommunications (UIT), sorti en 1988. Aussi IETF RFC-2459
(mis à jour).
Trois champs obligatoires.
  \item TBS certificat (TBS = " To Be Signed "): La charge utile du certificat (voir ci dessous).
  \item CA algorithme de signature: identificateur de l'algorithme cryptographique utilisé par l'AC pour
signer le certificat.
  \item Valeur de signature CA: Signature du certificat par le CA.
X.509: TBS certificat:
  \item Numéro de série: Numéro unique attribué par le CA à la certificat.
  \item Domaine de la société: Identifies l'entité qui a signé et délivré le certificat.
  \item Sujet: Identifies l'entité associée à la clé publique (O: organisation, C: pays, OU: Unité


d'organisation, CN: nom commun par exemple DNS, ST:. Etat, L: ville, etc Aucune adresse IP).
Validité: Pas avant, pas après.
Subject Public Key Info: La clé publique et identification de l'algorithme avec lequel la clé est
utilisée.
9.2 Obtaining and verifying a Certificate :
Comment obtenir un certificat
1.
 Le demandeur doit se présenter à la CA.
2.
 Le CA (physiquement) authentifie le demandeur.
3.
 Le CA demande au participant de générer des clés publiques / privées.
4.
 Le CA crée un certificat avec l'identité du demandeur, la clé publique, la date d'expiration, etc,
et la signature du CA.
5. Le CA fournit une copie de sa clé publique à la requérante.
6. Le demandeur peut propager son certificat.
Autorité d'enregistrement
  \item Le CA peut déléguer l'enregistrement d'un candidat à l'autorité d'enregistrement (RA).
  \item La RA n'a pas la clé privée de CA.
  \item Le CA approuve le RA pour authentifier les demandeurs.
  \item Après avoir authentifié le requérant, la RA permet au demandeur de générer une paire de clés

et envoie la clé publique de la CA pour créer le certificat.
Techniquement, le RA envoie un Certificate Signing Request signé (RSE) à l'AC.

Vérification d'un certificat
\begin{itemize}
  \item Vérifiez le chemin de certification.
  \item Vérifiez la période de validité.
  \item Vérifiez que le certificat n'est pas révoqué.
\end{itemize}
\section{Passwords}
\subsection{Basics \& Vulnerabilities}
Basics:
Les mots de passe ne sont jamais enregistrés en « clear », le risque de les récupérer par un hacker est
trop grand. Ils sont alors stockés sous la forme d'un hash. Le hash d'un mot de passe est irréversible.
Pour vérifier une authentification, il suffit de comparer le hash du login et le hash stocké sur le serveur.
h: fonction de hash
Avec un h(x) = y, il est impossible de retrouver x seulement avec y
Vulnerabilities1.2.3.4.5.6.7.8.9.:
Noter son mot de passe :
Cacher un post-it avec ses mots de passe en dessous du clavier, à côté du bureau, etc. Les gens
ont plus tendance à noter leur mot de passe lorsqu'ils sont obligés de le renouveler
régulièrement.
Les regards indiscrets :
Un regard par-dessus une épaule, une mini caméra bien placée. Les mots de passe n'ayant pas
de signification sont plus vulnérables à ce type d'attaque (pourquoi ?)
Arnaque (social engineering) :
Envoyer de faux mail demandant à l'utilisateur de renvoyer son mot de passe pour le valider
dans la base de données (plus de 1/3 des gens réponde).
Key logger, Rootkit :
Software ou Hardware qui enregistre toutes les touches du clavier de la victime. Pour contrer
cette attaque, on peut utiliser un clavier virtuel ou encore brouiller le mot de passe en
changeant la position du curseur lors de l'écriture du mot de passe.
Sniffer un réseau :
Certain protocole (POP, FTP, Telnet) ne chiffre pas les mots de passe qu'ils envoient, ils sont
donc visible si on analyse les paquets échangés sur le réseau.
Utiliser plusieurs fois un mot de passe :
Ne jamais utiliser le même mot de passe entre deux environnements (Windows, Linux). Ne
jamais utiliser un mot de passe reçu par mail. Utiliser différent niveau de sécurité.
Les traces :
Faire attention au gestionnaire de mot de passe (qui les enregistre) sur les ordinateurs public.
Faire attention au champ qui s'auto-remplit après avoir valider un formulaire où par erreur on
a entré son mot de passe dans le login.
Les attaques par essais :
Cette attaque consiste à faire un grand nombre d'essayes avec des mots de passe différent. On
distingue deux types d'attaque, les online attack et les offline attack (détaillé plus loin).

\subsection{Online attack}
Les online attack consistent à tester des mots de passe successivement à travers l'interface (web par
exemple) qui sépare le pirate et le serveur. Le serveur va alors indiquer au pirate si le mot de passe est
bon ou non et dans ce cas, essayer un autre (le serveur fonctionne comme une black box, une oracle).
Ce système est lent, mais ne nécessite pas de connaitre la fonction de hash.
Pour contrer ce genre d'attaque :
\begin{itemize}
  \item Ajouter un délai de réponse :Un délay est ajouté avant que le client (pirate ici) ne reçoit la
réponse du serveur.
  \item Notifier l'utilisateur (par mail) et bloquer son compte : Attention, cette mesure permet de faire
du déni de service. En effet, un pirate peut essayer plusieurs mot de passe random pour
bloquer le compte d'un utilisateur. L'utilisateur va alors contacter le service client et cela aura
également un coup pour celui-ci.
  \item Faire participer l'utilisateur : La machine du client doit faire un grand nombre d'opération
avant de pouvoir soumettre un mot de passe (trouver un « r » tell que h(login,password,r)
renvoie un nombre dont les derniers bits soit égaux à 0). Cela ralentit considérablement la
vitesse des attaques tout en maintenant un temps de calcule raisonnable pour une simple
connexion.
  \item Captcha : La plupart des attaques se font par des ordinateurs alors que la vraie connexion par
des humains. Il suffit donc de demander au client une réflexion simple pour un humain mais
compliqué pour un ordinateur.
\end{itemize}

\subsection{Offline attack}
Principe:
Les offline attack consiste à voler le fichier qui contient tous les hash, trouver la fonction de hash et
ensuite faire une attaque par dictionnaire, brute force, etc. En connaissant la fonction de hash, le
pirate peut comparer des mots de passe hashé au fichier récupérer sur le serveur.
Les attaques par dictionnaire :
Beaucoup de gens utilise des mots de passe issus du dictionnaire (plus simple à retenir) mais celui-ci
est fini (entre 150000 mots et 200000 mots) et est réduit donc considérément les possibilités. Un
dictionnaire déjà hashé peut accélérer l'attaque. Un ordinateur peut calculer entre 200 000 hash et 10
000 000 hash par second en fonction de la fonction de hash.
Brute-force :
Les attaques par brute-force consiste à essayer un grand nombre de possibilité même si celle-ci n'ont
pas obligatoirement un sens (« aaa » puis « bbb » puis « abb », etc). Cette attaque permet de trouver
les mots de passe abstrait sans signification qui ne se trouve pas dans un dictionnaire.
Heuristique attaque :
On peut combiner l'attaque par dictionnaire et l'attaque par brute force en appliquant certaines règles
souvent utilisé par les gens :
\begin{itemize}
  \item Convertir majuscule/minuscule (FRED/fred).
  \item Mettre la première lettre en majuscule (Fred).
  \item Inverser les lettres d'un mot (Derf).
  \item Dupliquer un mot (FredFred).
  \item Réfléchir un mot (Fredderf).
  \item Décaler les lettres vers la gauche ou la droite (Redf/Dfre).
  \item Remplacer certain caractère par des similaires (Fred).
  \item Ajouter X caractère au début ou à la fin du mot (aaaFred).
\end{itemize}

\subsection{Unix/Windows case}
La fonction de hash peut être basé sur le DES, MD5, Blowfish, SHA256, SHA512.
Unix (DES):
pw : le mot de passe (56 bits).
salt : un block généré aléatoirement (12bits)
zero : un block comprenant uniquement des zéros (64 bits).
DES(plaintext, key) : un block de 64 bits crypté.
 hash(pw) = 25 x DES(zero, pw+salt)
Le hash est stocké dans le fichier /etc/shadow sous la forme :
LOGIN:SALT(2 char)HASH(11 char):etc
Exemple: Smith:3Yr83xxCi/Ki2:12801:0:99999:7:-1::
Unix (MD5):
pw : le mot de passe (? bits).
salt : un block généré aléatoirement (48 bits).
MD5(plaintext) : un block que de 128 bits crypté.
 hash(pw) = MD5(pw+salt)
Le hash est stocké dans le fichier /etc/shadow sous la forme :
LOGIN:$ALGO$SALT(8 char)$HASH(22 char):etc
Exemple: Smith:$1$3YziHbd2$r83xxCi/Ki2uzGbd40sPzm:12801:0:99999
Avant, les hash était stocké dans le fichier etc/passwd avec un accès en lecture seul. Maintenant pour
plus de sécurité, ils sont stockés dans etc/shadow et ne peuvent être lu que par un administrateur (il
est quand même possible de le récupérer en bootant sur un autre OS avec un CD ou une clé USB).
Win 9x (LM Hash):
Windows utilise le Lan Manager Hash pour hasher ses mots de passe. Le mot de passe est d'abord
converti en majuscule puis divisé en deux blocks de 7 caractères. Les 7 bytes sont utilisé comme clé
pour crypter une constante avec l'algorithme DES. Le LM Hash n'utilise pas de salt.
- 30 -
Les mots de passe LM alpha numérique sont déchiffrable en quelque secondes (99.9%).
Les mots de passe LM alpha numérique avec 15 caractères spécieux sont déchiffrable en quelque
minute (96%).
Win NT/2000/XP/Vista/Seven (NT LM Hash):
Ces versions de Windows utilisent le NT Lan Manager Hash (NT Hash). Cette fois, le mot de passe n'est
pas séparé en deux, peut-être plus grand que 14 caractères et n'est pas converti en majuscule. Il
n'utilise toujours pas de salt.
Par défaut, LM Hash et NT Hash sont tous les deux conservés sur la machine (jusqu'à Windows XP
compris). Il est donc possible de récupérer le mot de passe NT Hash en récupérant le mot de passe LM
Hash. Il est possible de désactivé le LM Hash (désactivé par défaut à partir de Vista), dans le registre ou
en choisissant un mot de passe de plus de 14 caractères.
En dessous de W2K, XP et 2003, NT Hash et LN Hash sont stockés dans le Sécurity Account Manager ou
dans le Active Directory (ntds.dit). Le fichier est crypté mais par défaut la clé est récupérable et le
fichier de hash également avec les droits administrateur et un exploit ou en bootant sur un autre OS.
10.5 Conclusion :
Un mot de passe doit contenir au moins un élément de chaque point :
- Digit (0..9)
- Letter (a..Z)
- Ponctuation (?!)
- Caractère de contrôle (Ctrl+s, $^s$)
- Un caractère spécial dans les 7 premiers caractère (pourquoi 7 ?)
Basé sur une phrase connue (en prenant la première lettre de chaque mot).
Ne pas recyclé de mot de passe ou noter ses mots de passe.
Utiliser des niveaux de sécurité.
Modifier son mot de passe immédiatement si on pense avoir été victime d'une attaque.
Ne pas utiliser l'option « retenir mon mot de passe ».

\section{Time-memory Trade-off}
\subsection{Motivations}
Une fonction de hash est une fonction facile à calculer dans un sens, mais compliqué dans l'autre
(inverse).
DL problème (Discret logarithm) :
Avec p et g tell que ga mod p, il est difficile de retrouver a.
RSA problème :
Avec n et e tell que me mod n, il est difficile de retrouver m.
Exaustive search :
- Recherche en temps réel : recherche (N) stockage (0) pré-calcule (0)
- Recherche pré-calculé : recherche (0) stockage (N) pré-calcule (N)
\subsection{Hellman Tables}
Les table de Hellman est un principe de pré-calcul qui améliore considérablement la vitesse d'une
attaque en temps réelle. (la page wikipédia des rainbow table est assez bien faites pour comprendre le
principe donc n'hésitez pas : http://fr.wikipedia.org/wiki/Table\_arc-en-ciel)
Soit,
h: A $\to$B La fonction de hashage de notre mot de passe.
r: B $\to$ A Une fonction (arbitraire, rapide, surjective, déterministe) de réduction qui permet de
revenir à l'ensemble (A) de départ de notre fonction de hashage.
Un nombre M de chaine est généré avec comme une valeur arbitraire de départ dans A :
%Start1 = A1,1─h B1,1 ─r A1,2 ─ h  B1,2─ r... ─rA1,t-1─ h B1,t-1─r A1,t = End1
%Start2 = A2,1─h B2,1─r  A2,2─ h B2,2─r... ─rA2,t-1─ h B2,t-1─r A2,t = End2
...
%Startm = Am,1─h Bm,1─ r Am,2─ h Bm,2─ r ... ─ r Am,t-1─ h Bm,t-1─ rAm,t = Endm
Toutes les chaines doivent couvrir au final l'ensemble A. À la fin du pré-calcule d'une chaine,
on ne garde que la première valeur et la dernière valeur de la chaine. Les valeurs intermédiaire son
retrouvable avec la fonction de hashage et de réduction.
Il est possible d'obtenir des collisions entre différentes chaine, celle-ci vont alors avoir les même
valeurs jusqu'au bout de la chaine.
Recherche dans la table :
On considère une empreinte H engendrée à partir d'un mot de passe P. La première étape consiste à
prendre H, lui appliquer la dernière fonction de réduction utilisée dans la table, et regarder si ce mot
de passe apparaît dans la dernière colonne de la table. Si cette occurrence n'est pas trouvée alors on
peut déduire que l'empreinte ne se trouvait pas à la fin de la chaîne considérée. Il faut revenir un cran
- 32 -
en arrière. On reprend H, on lui applique l'avant-dernière fonction de réduction, on obtient un
nouveau mot de passe. On hache ce mot de passe, on applique la dernière fonction de réduction et on
regarde si le mot de passe apparaît dans la table.
Cette procédure itérative se continue jusqu'à ce que le mot de passe calculé en fin de chaîne
apparaisse dans la table (si rien n'est trouvé, l'attaque échoue).
Une fois le mot de passe découvert dans la dernière colonne, on récupère le mot de passe qui se
trouve dans la première colonne de la même ligne. On calcule à nouveau la chaîne tout en comparant
à chaque itération l'empreinte obtenue à partir du mot de passe courant avec l'empreinte H du mot de
passe inconnu P. S'il y a égalité, alors le mot de passe courant correspond à celui recherché et l'attaque
a réussi ; plus précisément, on a trouvé un mot de passe dont l'empreinte est la même que celle de P,
ce qui est suffisant.
Exemple :
À partir d'une empreinte ("re3xes"), on calcule la dernière réduction utilisée dans la table et on
regarde si le mot de passe apparaît dans la dernière colonne de la table (étape 1)
Si le test échoue ("rambo" n'apparaît pas dans la table), on passe au point 2 où l'on calcule une
chaîne avec les deux dernières réductions.
Si ce test échoue à nouveau, on recommence avec 3 réductions, 4 réductions, etc. jusqu'à
trouver une occurrence du mot de passe dans la table. Si aucune chaîne ne correspond alors
l'attaque échoue.
Si le test réussit (étape 3, ici "linux23" apparaît en fin de chaîne et également dans la table), on
récupère le mot de passe à l'origine de la chaîne qui a abouti à "linux23". Il s'agit ici de
"passwd".
On génère la chaîne (étape 4) et on compare à chaque itération l'empreinte avec l'empreinte
recherchée. Le test est concluant, on trouve ici l'empreinte "re3xes" dans la chaîne. Le mot de
passe courant ("culture") est celui qui a engendré la chaîne : l'attaque a réussi.
Pour les tables de Hellman, c'est le même principe que les table de Hellman à l'exception qu'il n'y a
pas une seul fonction de réduction mais plusieurs selon la position dans les colonnes.
Source : http://fr.wikipedia.org/wiki/Rainbow\_table
Les faux positifs (false alarm) :
Étant donné un hash C compris dans l'ensemble B, tell que Y1 = r(C), alors il est possible que la chaine
que nous reconstruisons lors de la recherche, converge sur la même valeur final Ej / Ys d'une chaine de
notre table sans pour autant en faire partie:
- 33 -
Dans ce cas, l'algorithme détecte une valeur final commune à notre table (droite qui va de Sj vers Ej) et
à la chaine (courbe qui va de Y1 vers Ys) que nous venons de reconstruire, il va alors falloir reconstruire
la chaine à partir du début (Sj) et au final se rendre compte que c'est un faux positif lorsqu'il arrive à
l'endroit où il devait trouver le hash recherché : C' ≠ C.

\subsection{Rainbow tables}
Les rainbow tables suivent le même principe que les tables de Hellman à la différence qu'elles n'ont
pas une seule fonction de réduction mais une variété en fonction de la longueur des chaines. Souvent
c'est un simple paramètre qui varie dans la fonction de réduction. Cela permet de les différencier selon
la position dans la chaine. Ainsi, dès à présent, si deux chaines rentrent en collision à des colonnes
différente, cela n'a pas d'effet sur la suite de la chaine, elles ne vont pas converger vers la même
valeur car leurs fonctions de réduction varient légèrement. Dans le cas où la collision se fait à la même
colonne, il est possible de le détecter (comment ?).
Une rainbow table parfaite est une table qui n'a aucune collision entre deux chaines à une colonne
donnée (qui ne se merge pas pour avoir au final, toutes les deux la même valeur de fin de chaine).
Quelque théorème:
-
 Avec t et un nombre N suffisamment grand, le nombre maximum de chaines normalement
requis pour une rainbow table parfaite (sans merge) est de :
m max(t) ≈ (2.N)/(t+1)
-
 Avec t et pour un problème de taille N, le pourcentage de réussite espéré d'une rainbow
table parfaite est de :
Pmax(t) ≈ 1- (1- (2/(t+1)) )t
Qui tend vers 1-e-2 ≈ 86% lorsque t est grand.
-
 Le dernier théorème sert à calculer « the average cryotanalysis time » avec N, m, l et t
donné :

\subsection{Conclusion}
TMTO n'est jamais meilleur qu'une attaque Brute force, mais il a du sens dans certain cas (attaque
répétée, temps limité pour l'attaque, etc).
Il faut bien analyser le problème avant de s'attaquer à celui-ci. 56-bit DES est par exemple impossible à
craquer : en brute force il faudrait 20 ans, avec TMTO 1 semaines, mais 8000 ans pour générer la table
(ainsi que 512Gb).
L'attaque consiste d'abord à pré calculer la table :
\begin{itemize}
  \item Éliminer les chaines qui se mergent.
  \item Paralléliser le calcul.
\end{itemize}
Puis vient l'attaque à proprement parler, la recherche du hash dans la table :
\begin{itemize}
  \item Paralléliser l'attaque.
  \item Pas besoin de beaucoup de mémoire.
\end{itemize}

\section{WEP}
12.1 Introduction Wifi :
Un point d'accès connecté via un câble au réseau. Plusieurs machines connectées via WIFI à ce point
d'accès.
Plusieurs manière : réseau ad hoc et mode infrastructure (cf cours de réseau). L'utilisation la plus
simple et la plus courante des réseaux ad-hoc est faite par les réseaux sans fil Wi-Fi en permettant une
mise en place rapide d'une connexion réseau entre deux ordinateurs.
Les stations et le point d'accès communique via des ondes. (Non c'est vrai ?)
Problème : n'importe qui peut capter/injecter des ondes avec une interface appropriée. (Record du
monde : transmission de 382 km)
War driving
Un plouc peut se balader dans les rues en voitures et construire un carte des réseaux disponibles (un
PC, un GPS et un bon software et le tour est joué).
Security needs (en générale, valable pas uniquement pour le WEP)
Vu que n'importe qui peut intervenir sur un réseau sans fil, on a besoin de s'assurer de trois choses :
- L'authentification
- L'intégrité des données
- La confidentialité des données
Authentification
Plusieurs possibilités :
- pas d'authentification : n'importe qui peut se connecter au réseau.
c'est de moins en moins utilisé. Les providers donnent des boites contenant des clés par
défaut. Il reste néanmoins certains endroits comme les cafés où on trouve des hots spots
gratuits. A ne pas confondre avec les trucs de type FON.
- Authentification via le SSID
chaque modem broadcast son SSID. Tous les clients a porté peuvent découvrir le modem. On
peut utiliser ce principe comme authentification : le client doit connaitre le SSID pour pouvoir
se connecter. Pas du tout sécurisé car n'importe qui peut eavesdropper le SSID quand un client
légitime se connecte.
en pratique, il suffit de sniffer le réseau avec kismet ou Airodump.
- MAC address authentification
le routeur contient une liste de MAC adresses autorisées et vérifie si celle du client en fait
partie. L'attaquant peut tjs sniffer une adresse MAC valable facilement. Il lui suffit de changer
la sienne pour pouvoir se connecter au réseau. 3 lignes de commande en Unix :
ifconfig INTERFACE down
- 36 -
-
ifconfig INTERFACE hw ether NEW\_MAC\_ADR
ifconfig INTERFACE up
Authentification cryptée
WEP ou WPA. C'est ce qui nous intéresse vraiment.
WEP : wired equivalent privacy
Le WEP fait partie du standard de l'IEEE 802.11 (standard de tout ce qui est réseau sans fil). Son but
est de rendre les réseaux LAN aussi sécurisés que les réseaux câblés.
Les grandes propriétés du WEP sont :
- Pas de key management
- Pas de protection contre les attaques replay
- Authentification : une clé pour plusieurs utilisateurs
- La confidentialité repose sur RC4 (stream cipher)
- Dispose d'un system de vérification d'intégrité : CRC-32
No key management
Toutes les stations sans fils et le routeur ont la même clé pré-partagée. Cette clé est utilisée pour
l'authentification et pour les encryptions.
La clé est encodée manuellement sur chaque machine. Trop souvent, la clé entrée par le fabricant et
n'est plus jamais modifiée.
Replay attacks
Vu que la clé n'est jamais changée, le méchant pirate peut attendre que le IV soit réutilisé et alors
renvoyer un message qui est déjà passé. L'IV est sur 24 bits (16 777 216). Grace à notre ami le
théorème des anniversaires, on sait qu'il y a 50% de chances qu'un IV se répète après 5000 paquets.
Plusieurs solutions peuvent être implémentées pour éviter ca:
- Nounce : challenge/réponse. A envoie un nonce (c.-à-d. un chiffre aléatoire fraichement
généré). B l'encrypte et l'envoie à A. A vérifie que c'est bon.
- Timestamp : A envoie son message accompagné d'un MAC (message authentification code : ça
sert à être sûr de l'intégrité du message. Prenez le cours de crypto si vous voulez en savoir
plus)
- Sequence number : A et B discute en incrémentant un compteur.
-
Intégrité
L'intégrité est assurée via CRC (contrôle de redondance cyclique). Cependant, CRC est designé pour
détecter les erreurs de transmissions (erreurs aléatoires) mais pas les erreurs intelligentes des pirates.
- 37 -
Schéma d'encryptions
La figure ci-dessous présente le schéma d'encryptions.
On entre un message m. Celui-ci est coupé en byte mi et encodé via un xor avec le keystream.
Le keystream est généré par RC4.
RC4 est un stream cipher orienté byte. Il est donc très efficient en software (10x plus rapide que DES)
et a l'avantage d'être simple. Il se passe en deux phases : l'initialisation et la génération de clé.
Initialisation & scrambling
RC4 comporte un registre principal: le registre d'état (N=256 bytes). Il est noté S dans la suite
La clé et le vecteur d'initialisation servent à initialiser un second registre le registre K.
Dans un premier temps, le tableau S est rempli des 0->N-1.
Ensuite, le tableau est mélangé avec le registre de K de la manière suivante.
Génération de clé
Vient ensuite la génération de clé: elle est faite de la manière suivante :
Maintenant que nous connaissons le fonctionnement du WEP, nous pouvons passer à
l'authentification cryptée.
- 38 -
Je crois que l'image en dit suffisamment long. C'est comme d'habitude un challenge/réponse.
12.2 Attaque contre le WEP :
1ere attaque : Exhaustive search
WEP: 40bits de clé + 24 bits de IV = 64 bits  264 bits c'est encore faisable avec un gros gros pc.
Il existe des versions de WEP avec 104 bits de clé +24 bit d'IV. Là c'est plus possible.
2eme attaque : CRC property
CRC a une sale propriété : CRC est linéaire par rapport au XOR c.-à-d. que
Ça craint vachement !
Je m'explique : un attaquant connait l'adresse ip contenue dans le message M. Il peut calculer un ∆m
tel que m xor ∆m = l'adresse ip du pirate. Il lui suffit ensuite de calculer CRC(∆m) et de xor le tout.
L'attaquant a réussi à modifier un message sans connaitre la clé K tout en gardant l'intégrité du
message.
3eme attaque : Keystream Reused
Comme dit précedement, le IV fait 24 bits=3bytes.
Vu que les gens changent jamais de clé, un keystream donnée (cad un IV) revient relativement
souvent.
En général l'IV est un compteur. Il est donc possible de forcer la réutilisation d'un IV en reboutant le
routeur (compteur réinitialisé). Il y a 50% de chances de revoir passer un IV dans 212 paquets (birthday
paradox).
Si on connait un message qui a été encrypté avec un certain IV, on peut retrouver un autre message
encrypté avec un même IV.
C= K xor M
C'=K xor M'  M' = C' xor C xor M
A nouveau pas besoin de la clé.
on peut facilement connaitre un message M( entête de paquets connus, etc)
4eme attaque : Cryptanalyse
C'est ici que les choses marantes commencent !
2 ans après la sortie de WEP, on commence à piger que y des IV qui font tout foirer. Ils permettent de
retrouver certains états internes du registre S dont je vous parlais précédemment et parfois même la
clé complète.
Pour réaliser cette attaque, on doit connaitre l'IV et les premiers bytes d'un message et de son chiffré.
(L'IV est envoyé en clair et le chiffré est sniffé)
Le message est encrypté avec la clé (IV||key) = (K0, K1, K2, K3, K4,..., K7) = (3,255, X, K3,..., K7)
- 39 -
Donc l'IV est (3, 255, X) et X est supposé connu.
On effectue les quatre premières étapes de l'initialisation :
On suppose pour un instant que on s'arrete ici pour l'initialisation.
Si on doit encoder un truc à ce stade, le keystream généré sera 6+X+K3 :
Comme le keystream est connu, on peut trouver K3 pour i=3
Cependant, on ne peut pas arrêter l'initialisation au 4eme stade.
Quelle est la probabilité que les éléments 0,1 et 3 du tableau ne soient pas modifiés dans le reste de
l'initialisation? Apparemment c'est
 .
Donc si on voit suffisamment de IV, on finira bien par en trouver un qui fera l'affaire.
Une fois que K3 est connu, on refait la même chose avec un autre IV :
Si IV = {4, 255, X}, on peut effectuer la même démarche. On trouve alors pour i=4,
On peut donc trouver toute la clé byte par byte en continuant.
En moyenne, il faut 4millions de IV pour retrouver une clé de 128 bits. Le nombre de IV nécessaire est
linéaire en fonction de la longueur de clé (-> très mauvais).
Further attacks
Il existe des attaques du même styles mais plus puissantes mais elles ne sont (heureusement) pas
détaillées dans le cours.

\section{WPA}
\subsection{WPA}
Le WPA ou « Wifi protected access » a pour but de remplacer le WEP. Comme vous avez pu le voir, le
WEP c'est de la merde. Ce remplacement est un patch assez urgent avant la publication de WPA2
(802.11i).
Depuis 2003, tous les appareils Wifi doivent implémenter le WPA.
Propriétés-
-
-
-
-
-
et amélioration par rapport au WEP
Un compteur est utilisé pour éviter les attaques par répétition.
L'IV est sur 48 bits
Chaque utilisateur est authentifié contrairement au WEP ou les machines seulement sont
identifiées.
Les clés sont dynamiquement rafraichies en utilisant TKIP.
AES est utilisé dans WPAE au lieu de RC4 dans WEP et WPA2
Il existe deux modes de fonctionnement du WPA :
o Personnal : on utilise alors des « pre-shared key » (PSK)
chaque appareil connecté utilise le même mot de passe de 256 bits.
Cette clé peut être saisie soit sous forme de chaîne de 64 chiffres hexadécimaux, ou
comme une phrase de passe de 8 à 63 caractères ASCII imprimables. Si les caractères
ASCII sont utilisés, la clé de 256 bits est calculée en appliquant la fonction de
dérivation de clé PBKDF2 au mot de passe, en utilisant le SSID comme le sel de 4096
itérations de HMAC-SHA1. L'authentification de se fait via EAP-MD5. (MD5 +
Extensible Authentication Protocol voir plus loin)
o Entreprise : on utilise l « IEEE802.1x Authentification Server »
chaque utilisateur doit s'identifier. On a besoin d'un serveur qui gère tous les comptes
des utilisateurs (genre Radius)
http://en.wikipedia.org/wiki/IEEE_802.1X
13.2 Architecture et protocoles :
On considère 3 acteurs :
- Le client (supplicant)
- Le point d'accès (authentificateur)
- Le serveur d'authentification
- 41 -
Il y a quatre phases importantes
- L'accord sur la politique de sécurité
- L'authentification
- La génération de clé et leur distribution
- La confidentialité et l'intégrité des données
Phase 1 : L'accord sur la politique de sécurité
Phase 2: l'authentification
Pour l'authentification, comme dit précédemment EAP est utilisé.
EAP n'est vraiment un protocole d'authentification en lui-même. Seul, il ne sert à rien. Il a été créé
pour transporter les messages selon un cadre bien défini. Il contient 4 sortes de message : request,
response, succes et failure. Les paquets request sont émis par l'authentificateur (Acces Point). Le
suppliant doit y répondre correctement afin de recevoir un paquet succes.
- 42 -
L'AP peut envoyer autant de paquets request qu'il a envie avant d'accorder ou non l'accès.
EAP est utilisé avec plusieurs outils :
- Legacy based methods : EAP-MD5
- Méthodes basées sur un certificat : EAP-TLS, EAP-TTLS, PEAP
- Méthodes basées sur des mots de passe : LEAP, SPEKE
- ...
EAP-MD5
Utilisé uniquement en mode Personnal WPA.
L'authentification se fait via le hash d'un challenge et du mot de passe. Cependant, MD5 est
absolument plus suffisant de nos jours. Cette méthode n'est PAS sécurisée.
Comme le dit Wikipédia: It offers minimal security; the MD5 hash function is vulnerable to dictionary
attacks, and does not support key generation, which makes it unsuitable for use with dynamic WEP, or
WPA/WPA2 enterprise.
EAP-TLS
EAP est utilisé avec le TLS. EAP se base sur les PKI (public key infrastructure -> certificat) pour sécuriser
les communications entre le serveur d'authentification (AS). L'authentification est donc très fortement
sécurisée mais nécessite la mise en place des PKI.
Le client et le serveur ont un certificat.
L'EAP-TLS est donc essentiellement utilisé en entreprise.
EAP-TTLS
C'est grosso modo la même chose sauf qu'ici il n'y a que l'AS qui a un certificat. On peut utiliser une
méthode d'authentification moins sécurise pour le client (CHAP ou PAP)
-
-
PAP : Password authentication protocol
Quand l'AS te demande d'envoyer ton pass, tu l'envoies.
CHAP: Challenge-Handshake Authentication Protocol
lorsque tu reçois un challenge, le prouveur encrypte le challenge avec sa clé et envoie le chiffré
au vérifieur
- 43 -
Phase 3 : key dérivation
4 way hands shake:
http://en.wikipedia.org/wiki/IEEE_802.11i-2004 -> c'est bien expliqué dans le cours y a que dale.
The Master Key key is designed to last the entire session and should be exposed as little as possible.
Therefore the four-way handshake is used to establish another key called the PTK (Pairwise Transient
Key). The PTK is generated by concatenating the following attributes: PMK, AP nonce (ANonce), STA
nonce (SNonce), AP MAC address, and STA MAC address. The product is then put through PBKDF2-
SHA1 as the cryptographic hash function.
The handshake also yields the GTK (Group Temporal Key), used to decrypt multicast and broadcast
traffic.
1.2.3.4.The AP sends a nonce-value to the STA (ANonce). The client now has all the attributes to
construct the PTK.
The STA sends its own nonce-value (SNonce) to the AP together with a MIC, including
authentication, which is really a Message Authentication and Integrity Code: (MAIC).
The AP sends the GTK and a sequence number together with another MIC. This sequence
number will be used in the next multicast or broadcast frame, so that the receiving STA can
perform basic replay detection.
The STA sends a confirmation to the AP.
- 44 -
As soon as the PTK is obtained it is divided into five separate keys:
PTK (Pairwise Transient Key – 64 bytes)
1. 16 bytes of EAPOL-Key Confirmation Key (KCK)– Used to compute MIC on WPA EAPOL Key
message
2. 16 bytes of EAPOL-Key Encryption Key (KEK) - AP uses this key to encrypt additional data sent
(in the 'Key Data' field) to the client (for example, the RSN IE or the GTK)
3. 16 bytes of Temporal Key (TK) – Used to encrypt/decrypt Unicast data packets
4. 8 bytes of Michael MIC Authenticator Tx Key – Used to compute MIC on unicast data packets
transmitted by the AP
5. 8 bytes of Michael MIC Authenticator Rx Key – Used to compute MIC on unicast data packets
transmitted by the station
Pour ceux qui veulent comparer avec le schéma du cours :
- 45 -
1.2.GEK (Group Encryption Key): Key for data encryption (used by CCMP for authentication and
encryption and by TKIP).
GIK (Group Integrity Key): Key for data authentication (used only by Michael with TKIP)
Group key Hands shake
The GTK used in the network may need to be updated due to the expiry of a preset timer. When a
device leaves the network, the GTK also needs to be updated. This is to prevent the device from
receiving any more multicast or broadcast messages from the AP.
To handle the updating, 802.11i defines a Group Key Handshake that consists of a two-way handshake:
1. The AP sends the new GTK to each STA in the network. The GTK is encrypted using the KEK
assigned to that STA, and protects the data from tampering, by use of a MIC.
2. The STA acknowledges the new GTK and replies to the AP.
En gros à partir de la clé principale (master key), la station et l'AS vont générer toute une série de clé.
Chaque clé est utilisée dans un but différent. Le schéma ci-dessous explique bien les relations entre ces
différents clés.
- 46 -
Phase 4 : data integrity
Deux algorithmes possibles peuvent être utilisés avec WPA :
- TKIP : RCA+ MIC
- CCMP : AES en mode compteur (WPA2 uniquement)
TKIP: key mixing scheme and encryption
MIC
Mic utilise l'algorithme de Michael a la place de CRC32. Michael est utilisé avec une clé. Pourquoi
Michael ? Simplement parce que c'était le plus puissant des MIC compatible avec les vieilles cartes.
A cause des faiblesses de Michael, le réseau est coupé pendant une minute si 2 frames ne passent pas
le check de Michael. Ensuite, des nouvelles clés doivent être régénérées.
- 47 -
13.3 Attaques :
Attaque sur PSK
Toutes les clés dérivent de la PSK :
PMK = PBKDF2(SSID,PSK) suivi d'un 4-way hands shake.
Sachant qu'on peut écouter ce hand shake, on peut tester toutes les clés possibles :
Procédure :
Pour chaque candidat de mot de passe, on calcule le PMK associé.
On calcule ensuite la PTK
Et finalement le MIC et on le compare avec celui qu'on a écouté.
Il existe des rainbows tables pré calculée pour les 1000 SSID les plus courants.
Attaque sur TKIP
-
 injection de paquets
-
 Décryptions de paquets de l'AP vers le client
 \subsection{Conclusion}
Glossaire:

\section{IPsec}
Protéger les données au network layer avec Virtual Private Networks (VPNs) construits avec IPSec
(Internet Protocol Security).
14.1 VPN :
Les VPNs permettent d'étendre un réseau privé à travers un réseau public.
Applications: Interconnexion de sites distants à travers Internet, accès au réseau de l'entreprise à
partir d'un ordinateur connecté à Internet comme si on était à l'entreprise.
Mécanisme :
  \item 2 sets d'adresses : IP public pour échanger des données sur Internet et des adresses privées
qui routent les paquets entre les hôtes sur le réseau privé.
  \item Encapsulation : le paquet est encapsulé dans un paquet IP pour son voyage sur Internet.
  \item Encryption : pour éviter le eavesdropping et les modifications de données sur le réseau
public.
Protocoles :
  \item PPTP (Point to Point Tunneling Protocol) : développé par Microsoft.
  \item L2TP (Layer 2 Tunneling Protocol) : développé par IETF, rassemble L2F de Cisco et PPTP de
Microsoft.
  \item IPSec (IP Security) : développé par IETF.
14.2 IPSec :
Le standard est ouvert et extensible. Des algorithmes publics sont utilisés pour la confidentialité,
l'authentification et l'intégrité.
14.2.1 Modes d'opération :
  \item Transport : les données sont encryptées et/ou authentifiées. La sécurité est faite de
manière end-to-end.
  \item Tunnel : tout le paquet est encapsulé dans un nouveau paquet. La sécurité peut être faite
par les routeurs intermédiaires.
Chaque routeur contient une Security Policy Database qui défini quel paquet doit être sécurisé
en fonction de sa destination et de sa source par exemple (e.g. : sécuriser le trafic telnet,
UDP,...)
14.2.2 Protocole d'échange de clés :
Internet Key Exchange (IKE) :
Protocole permettant l'installation d'un tunnel sécurisé entre partenaires.
Objectifs :
  \item Authentification du partenaire
  \item Echange de clé entre partenaires
  \item Négociation du paramètre
- 51 -
Phases :
1. Négocier un IKE SA (Internet Key Exchange Security Association) pour protéger les négociations
  \item Authentification : pre-shared secret (PSS), clés asymétriques, utiliser la clé publique
d'une certification authority (X.509)
  \item Echange de clés : clés générées avec Diffie-Hellman
  \item Négocier le paramètre
o Main mode : clé de session générée avec Diffie-Hellman (plein de paramètres) and
utilisée pour protéger le reste de la transaction (négociation du paramètre et
authentification).
 un eavesdropper ne sait pas dire l'identité des paires.
o Agressive mode : piggyback sur l'étape d'authentification pour l'échange de Diffie-
Hellman, les identités et le pwd hash sont envoyés en clair. MAIS ne doit échanger
que 3 paquets !
 attaquable (CKY cookie dans les headers des paquets IKE)
2.Définir les besoins du SA (Security Association) pour les flux ESP (Encapsulated Security
Payload) ou AH (Authetification Header)
Quick mode :
  \item Sans Perfect Forward Secrecy (PFS) :
o Les session keys sont rafraichies périodiquement (1h)
o Les session keys sont issues du même secret
o Voler ce secret compromet toutes les clés
  \item Avec PFS
o A chaque nouvelle session key, on fait un Diffie-Hellman (plus lent)
o Voler 1 clé ne compromet pas les clés précédentes
Security Association :
Avant de pouvoir échanger des paquets en toute sécurité, les ordinateurs doivent établir une
Security Association (SA) en utilisant IKE.
Pour chaque communication sécurisée, le SA mémorise les algorithmes, les clés, la période de
validité des clés, le numéro de séquence et l'identité du partenaire.
Il faut 1 SA pour chaque flux unidirectionnel : une connexion TCP demande un SA pour chaque
direction. (1 SA par destination par protocole, par port).
Les SAs sont identifiés par un Security Parameter Index (SPI) :
  \item La source indique le SPI sur tous les paquets qui sont envoyés (elle décide quels paquets
doivent être exécuté avec quel SA)
  \item La destination utilise le SPI pour trouver le SA qui décrit comment traiter ce paquet.
14.2.3 Protocole sécurisé :
Authentication Header (AH)
Pour vérifier l'authenticité et l'intégrité du paquet.
- 52 -
L'authentification est calculée sur base de :
  \item Données qui suivent le AH
  \item Le header (mis à 0 quand il faut calculer l'information d'authentification)
  \item Les champs importants dans l'IP header : source, destination, protocol, length, version,...
  \item Les champs exclus de l'IP header : type of service, flags, fragment offset, TTL, header
checksum.
L'algorithme utilisé pour générer les données d'authentification est négocié à la création du SA.
 utilise le HMAC : sécurité démontrable, peut être utilisé avec n'importe quelle hash function,
maintient les performances de la hash function originale :
HMAC(K,m) = H( (K xor opad) || H((K xor ipad) || m) )
HMAC-SHA1-96 signifie HMAC calculé avec H=SHA1 et la sortie tronquée à 96 bits.
Encapsulated Security Payload (ESP)
Pour l'encryption et l'authentification du paquet.
L'encryption est seulement faite sur l'encapsulated data et le trailer (à la fin du paquet). PAS
d'encryption sur les champs du header, ni sur les données d'authentification.
L'authentification (optionnelle) est faite sur l'ESP header et tout ce qui suit MAIS PAS sur l'IP
header (différence avec AH).
Exemple d'algorithme :
  \item Encryption : DES-CBC, NULL
  \item Authentification : HMAC-SHA-96, HMAC-MD5-96, NULL
NULL ne peut pas être utilisé dans le même SA pour l'encryption et l'authentification.
14.2.4 IPSec avec NAT
AH : que ça soit en mode transport ou tunnel, AH authentifie l'IP header et donc ne
permet pas au NAT de modifier l'IP source sans provoquer une perte d'intégrité.
ESP :
  \item Mode transport : en TCP ou UDP, les headers sont encryptés et donc le checksum ne peut
pas être mis à jour
  \item Mode tunnel : le checksum TCP ou UDP fait référence à l'IP header interne, qui n'est pas
modifié par le NAT.
 ESP en mode tunnel est le seul moyen de traverser un NAT !
- 53 -
Sécurité au transport layer.
15. SSL :
15.1 SSL
SSL = Secure Socket Layer à été développé par NetScape
15.1.1 Applications
  \item Créer un nouveau protocole à partir d'un protocole existant : HTTP -> HTTPS
o Désavantage : seuls les clients supportant TLS peuvent se connecter
o Avantage : sur que la communication est sécurisée
  \item Etendre un protocole pour négocier SSL/TLS : demander TLS avec STARTTLS
o Avantage : le client ne doit pas forcément supporter TLS pour utiliser le service
Exemples :
  \item HTTPS :
o L'utilisation de TLS n'est pas négociable
o Garanti la confidentialité et l'authenticité des données transmises
o Le server doit avoir un certificat
o Le client peut en avoir un (e.g. : eBanking)
  \item Mail
o ESMTP, POP3, IMAP transmettent par défaut le mot de passe en clair
o TLS protège le mot de passe et le contenu du mail
15.2 TLS
TLS = Transport Layer Security (version actuelle de SSL). Se situe entre l'application et le network layer.
- 54 -
15.2.1 Record Layer
Traitement des données
  \item Fragmentation
  \item Compression (optionnel)
  \item Authentification (MAC)
  \item Encryption
Transmets les fragments traités au transport layer
Une fois, transmis, les opérations inverses sont effectuées.
MAC :
MAC= hash ( MAC_key || Pad2 ||

hash(MAC_key || Pad1 || Seq_Nb || Length || Content))
- MAC_key: secret shared by client and server.
- Pad1: constant character 0x36 repeated 48 times (if MD5) or 40 times (if SHA1).
- Pad2: constant character 0x5c repeated the same number of times.
- Seq_Nb: sequence number of this message.
- Hash: Either HMAC-MD5 or HMAC-SHA1
- Length: Length in bytes of the compressed record.
- Content: Compressed record.
Encryption :
Faite sur des records compresses et authentifiés.
Soit avec du chiffrement par blocs:
o DES (40 bits ou 56 bits), 3DES, IDEA, RC2 (40 bits)
o AES (128 bits ou 256 bits) dans TLS v1.1 (256 bits pour les mails UCL)
Soit avec du chiffrement en flux:
o NULL, RC4 (40 bits ou 128 bits)
 le client doit refuser des clés de 40 bits si cette longueur est suggérée par le serveur.
15.2.2 Handshake protocol
Négociation de:
- protocole version (SSL 3.0, TLS 1.0, TLS 1.1).
- algorithmes:
o Key exchange (RSA, Diffie-Hellman).
o Encryption (DES, 3DES, IDEA, RC4, RC2, AES).
o MAC (HMAC-MD5, HMAC-SHA).
- Le client propose les algorithmes par ordre de préférence, le serveur choisi.
- Authentification optionnelle du partenaire en utilisant un certificat.
- Les messages ne sont pas cryptés.
- Le dernier message authentifie l'échange.
- 55 -

Client_Hello : utilisé par le client pour initialiser une connexion SSL (envoyé en clair sans
signature)
o Crypto : liste des algorithmes cryptographiques supportés (Authentication + key
exchange + cipher + hash).
Content:
-
-
-
-
-
-
Protocol Version.
32 bytes long random number.
Composed of two parts:
o 4 bytes Unix time (number of seconds since 01/01/1970)
o 28 bytes random number
Optional Session Identifier.
o Each SSL session has an identifier which can be used later to restart a
session.
List of supported Ciphers.

List of supported Compression Methods.
  \item Server_Hello : réponse du serveur (envoyé en clair sans signature)
Content:
-
-
-
-
-
Protocol version: highest version of the protocol supported by both client and
server.
Random number.
Optional Session Identifier, if it allows sessions to be resumed.
Cipher Suite: One of the cipher suites proposed by client.
Compression Method.
- 56 -

Server_Certificate : le serveur s'authentifie (un serveur peut avoir plusieurs certificats à partir
de différentes certification authorities). Le certificat peut aussi être envoyé par le client quand
l'identification du client est demandée par le serveur avec Certificate_Request.
Content:
-
A list of X.509 certificates:
o Server certificate.
o Certificates of certification authorities if any.


Server Hello_Done : indique que le serveur a fini la 1ere phase du handshake (envoyé non
crypté).
Client_Key_Exchange : utilisé par le client pour envoyé le PreMasterSecret (encrypté avec la
clé publique du serveur).
o Utilise des nonce de 32B et un masterSecret de 48B
Content:
-
Encrypted PreMasterSecret with the public key of the server.


Change_Cipher_Spec :
o Utilisé par le client et le serveur pour signaler qu'il vont utiliser une nouvelle clé.
o Durant le handshake, indique que le prochain message sera encrypté avec la clé
appropriée.
Handshake_Finished :
o Envoyé par le client et le serveur pour confirmer l'établissement d'une session SSL
sécurisée. (Sécurisée ssi le client et le serveur ont reçu ce message).
o Permet de détecter les man in the middle attack sur des messages Client_Hello et
Server_Hello (e.g.: proposer un chiffrement plus faible).
o Premier message encrypté dans chaque direction.
Content:
-
-
Keyed hash (MD5 or SHA-1) of all the handshake messages and the
MasterSecret.

\section{Kerberos}
1.1 Motivation :
Parfois plusieurs utilisateurs veulent avoir l'accès à plusieurs serveurs sur un même réseau. Une
solution pour gérer la sécurité de l'authentification est que chaque serveur retienne les mots de passe
de chaque utilisateur. Cette approche n'est cependant pas idéale :
  \item Elle n'est pas sure. Un seul serveur compromis peut compromettre tous les utilisateurs.
  \item Elle est inefficace. Les mots de passes doivent chaque fois être changés sur tous les serveurs.
  \item Elle n'est pas pratique. Les mots de passes doivent être entrés pour chaque requête.
Dès lors, il est plus raisonnable de centraliser les informations. Kerberos permet cela.
1.2 Kerberos :
Kerberos est un protocole d'authentification réseau qui repose sur un mécanisme de clés secrètes
l'utilisation de tickets. Il est basé sur de la cryptographie à clef symétrique. Une fois que
l'authentification est faire, on peut accéder au réseau sans devoir redonner son mot de passe ou nom
d'utilisateur.
1.3 Elements principaux :
Kerberos utilise tous les éléments suivants :
  \item Un client C. C'est la personne qui veut se connecter au réseau.
  \item Un serveur d'authentification AS 1. Ce serveur va fournir au client un premier ticket (TGT 2 ), au
client qui en fait la demande.
  \item Un serveur de ticket (TGS 3). Une fois qu'un client présente son ticket à ce serveur, ce dernier
peut lui donner un autre ticket pour accéder à certaines ressources.
  \item un serveur S dont le client veut avoir l'accès. Pour avoir l'accès à ce serveur, le client doit avoir
le ticket correspondant fourni par le TGS.
Au final, pour qu'un client ait accès au serveur S, il doit avoir son ticket ainsi qu'un authentifiant.
Le ticket contient les informations suivantes
  \item Ic, l'identité du client.
  \item v, la période de validité du ticket.
  \item Kc,s, la clef symétrique de session à utiliser entre le client et le serveur.
  \item D'autres trucs comme des flags, l'adresse IP, etc.
Ce ticket est encryptée avec la clef Ks du serveur. L'authentifiant est juste l'identité du client (Ic) suivi
d'un timestamp t. Regardons à présent tous les messages échangés pour obtenir ce fameux ticket. Le
schéma ci dessous reprend l'ensemble des échanges.
Analysons les séparément.
1. Authentication server.
2. Ticket-granting ticket.
3. Ticket granting server.
- 58 -
1.3.1 Echange entre le client et AS:
Le client doit être authentifié pour pouvoir accéder au TGS. Dès lors, il va en tout premier lieu faire une
demande d'authentification à AS. Ce dernier va lui renvoyer un TGT encrypté avec la clef du TGS ainsi
qu'une clef de session encryptée avec la clef de C. Plus particulièrement, on a les messages suivants :
1. Ic, ITGS,N. Le client envoie à AS son identité, celle du TGS dont il veut faire une requête, ainsi
qu'un nonce utilisé pour éviter les replay attaques.
2. {ITGS,N,KC,TGS}KC , {IC, v,KC,TGS}KTGS . Le premier bloc peut être décrypté par le client pour
qu'il puisse le KC,TGS nécessaire pour la suite du protocole. Le deuxième bloc ne peut être
décrypté que par le TGS.
1.3.2 Echange entre le client et TGS:
Une fois le TGT reçu, le client va envoyer une requête au TGS. Le TGS va alors utiliser la clef de session
pour vérifier le ticket. Il peut alors voir si le client est autorisé à accéder au serveur S.
Si oui, il va donner au client un ticket lui accordant ce service. Pratiquement, on a les messages
suivants
1. Is,N0, {IC, v,KC,TGS}KTGS , {IC, t}Kc,TGS . On a l'identité du service voulu, un nounce, le TGT
ainsi que d'autres données encryptées avec la clef KC,TGS entre le client et le TGS et qui
comprend l'identité du client et un timestamp.
2. {IS,N0,KC,S}KC,TGS , {IC, v,KC,S}Ks . Le TGS répond un message pouvant être décrypté par le
client et qui contient une clef de session avec le service S, et un autre message ne pouvant être
décrypté que par S.
1.3.3 Echange entre le client et S:
Finalement, le client va demander l'accès au serveur S à travers les messages suivants
1. {Ic, v,KC,S}KS , {IC, t}KC,S . Le client envoie son ticket de service ainsi qu'un autre message que
le serveur ne pourra décrypter que quand il aura obtenu le KC,S. Il pourra alors voir que les
deux Ic correspondent bien.
2. {t}KC,S . Le serveur lui renvoie simplement un timestamp que le client peut décrypter.
Le client peut maintenant accéder au serveur !
1.4 Commentaires :
Avec Kerberos, la responsabilité de stocker les tickets revient exclusivement au client. Par
ailleurs, une fois qu'un client a eu son authentification (il la garde environ 8 heures), il ne doit plus
contacter AS. Il peut accéder aux autres services même si ce dernier est down. Et il est finalement
impossible de révoquer le ticket d'un client qui a été authentifié.

\section{PGP}
Pretty Good Privacy (PGP) est un logiciel de chiffrement et de déchiffrement cryptographique.
Il garantit la confidentialité et l'authentification pour la communication des données. Il est le plus
souvent utilisé pour la signature de données, et sa gestion des clefs.
17.1 Encryption hybride :
PGP utilise pour encrypter les données à la fois des clefs asymétriques et symétriques, comme
le montre le schéma suivant.
Le système est donc hybride. Pour chiffrer un document, PGP utilise une clef symétrique aléatoire (clef
de session). Ensuite, PGP chiffre la clef de session avec la clef publique du destinataire.
Cela à d'une part l'avantage d'être rapide, au lieu de crypter l'entièreté du document avec la clef
publique (fort lent), on ne l'utilise que pour crypter la clef de session qui est nettement plus petite, et
d'autre part, le problème de distribution des clefs est réglé puisque la symétrique est encryptée. Pour
l'encryption symétrique beaucoup d'algorithmes peuvent être utilisés (TDES, IDEA, AED, etc.), pour
l'asymétrique on utilise généralement le RSA ou l'Elgamal qui se base sur le problème du logarithme
discret.
17.2 Signature de la clef publique :
Pour la signature, on utilise généralement l'algorithme DSA ou RSA. L'Elgamal n'est plus
recommandé du fait qu'une attaque a révélé que les clefs Elgemal n'étaient plus sûre lors qu'elles sont
utilisées pour l'encryption et pour la signature. De manière générale, il ne faut jamais utiliser la même
clef pour plusieurs choses.
17.3 Protection des clefs privées :
Les clefs privées sont encryptées sur le disque de l'utilisateur avec un algorithme se basant sur
une passphrase. Quand la clef est nécessaire, l'utilisateur utilise sa passphrase pour déchiffrer sa clef.
Cependant, pour préserver la sécurité, et respecter le principe du maillon le plus faible, la passphrase
doit être choisie intelligemment de sorte qu'elle soit aussi solide que les clefs utilisées pour la
signature ou l'encryption. Le tableau ci dessous reprend une correspondance de taille entre les clefs
symétriques et asymétriques pour avoir une même solidité.
- 60 -
17.4 Validité des clefs publiques :
Comment être sur qu'une clef qu'on utilise pour chiffrer un message est la clef correcte ? Une
manière sûre de vérifier cela, est de faire une rencontre IRL, vérifier l'identité de la personne, vérifier
le hash de la clef, et puis seulement signer la clef. En procédant comme cela, on peut être sûr que
la clef n'a pas été modifiée par la suite. On peut aussi utiliser des certificats.
17.5 Distribution des clefs publiques :
Il y a deux notions importances dans PGP.
1. La validité : Je sais à qui appartient cette clef.
2. La confiance : Je sais que cette personne ne signe pas des clefs n'importe comment.
Quand on signe une clef, on déclare sa validité. Sachant cela, on peut former un graphe dirigé
où les sommets correspondent aux différentes personnes et où les arêtes A ! B signifient que A a validé
la clef de B. On met aussi des poids sur les arêtes, 1 si A fait entièrement confiance à B, et 0.5 s'il lui
fait partiellement confiance. Dès lors, une personne peut considérer les clefs d'autres personnes
comme valide si d'une part, elle fait totalement confiance à une personne qui a vérifié l'adresses de
ces autres personnes (A fait confiance à B, et B a validé la clef de C : La clef de C est valide pour A) , ou
si d'autre part, elle fait moyennement confiance à plusieurs personnes qui ont vérifié la clef (A fait
moyennement confiance à B et C, mais B et C ont tout deux validé la clef de D : le clef de D est valide
pour A).
17.6 Révocation de clef :
Une manière naïve de procéder est de simplement enlever la clef du serveur. Cependant, il se
peut que la clef a été dupliquée sur d'autres serveurs, que des personnes l'aient téléchargée, etc.
On doit donc procéder autrement. On peut créer un certification de révocation de la clef quand on
génère cette dernière. On l'envoie alors sur le serveur PGP quand on veut révoquer la clef. Pour bien
faire, il faut aussi l'envoyer aux correspondants réguliers. Par ailleurs, lors de la génération d'une clef,
on met une limite de validité qui fait que la clef expire après cette date.

\end{document}
