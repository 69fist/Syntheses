\documentclass[en]{../../../eplnotes}

\hypertitle{security-INGI2347}{8}{INGI}{2347}
{Houtain Nicolas \and Gorby Nicolas Kabasele Ndonda}
{Professor}
$$$$

\section{Week 1}

\textsc{Security} = Goal Vs adversary threat
\begin{enumerate}
    \item \textbf{Policy} :
        \begin{itemize}
            \item confidentitiality
            \item integrity
            \item availabilty
        \end{itemize}

    \item \textbf{Threat model} : assumptions about adversary capability
        
        (Example: Be sure that adversary never can directly access on
        disk)

    \item \textbf{Mechanism} : Software/Hardware/System/\ldots enforce
        policy

    \item[Note :]  Policy and mechanism are independant (policy is the
        goal and mechanism how make there goal)

\end{enumerate}
The goal is met if there is no way in which the adversary w.r.t the threat model can violate the policy.
Security is hard because it's not always possible to think about all the way to break into a system.
It's a tradeoff betwenn the relevance of the file system and the information on the threat model.
\subsection{Policy}
Interaction between the policy of different computers systems can lead to vulnarability.
(TODO dessin gmail)

\subsection{Threat model}
Because the technology constantly evolves, the assumption made with threat model must be revised.
(ex:brute force attacks or phishing)

\subsection{Mechanism}
A bug in the sofware/hardware/System can violate the policy as they can lead to an attack.

\section{Week 2}

\subsection{Chapter 1 : what is security engineering}
cfr : http://www.cl.cam.ac.uk/~rja14/Papers/SEv2-c01.pdf

\paragraph{ }



\end{document}
