\documentclass[en]{../../../eplnotes}

\hypertitle{Distributed}{8}{INGI}{2345}
{Nicolas Houtain \and Gorby Nicolas Ndonda Kabasele}
{Peter Van Roy}
$$$$

Attention : This summary is actually based on course note


(Need slide to understand)

\section{Teaser}

\subsection{Parallel and distributed computing}
\begin{itemize}
    \item Parallel computing : many node, optimize performance, no
        failure
    \item[$\to$] Tightly coupled(low latency/delay and high performance)

    \item Distributed computing : many node in collaboration with
        \textcolor{red}{partial failure}
    \item[$\to$] Loosely coupled(high latency and low perfomance)
\end{itemize}

\subsection{Consensus}
Atomic broadcast $\equiv$ consensus (proof slide 13)

\begin{enumerate}
    It's possible to resolve consensus if we have atomic broadcast and vice-versa.
    \item broadcast $\to$ consensus : We take the first proposal as they have an order 
    \item consensus $\to$ broadcast : The subject of the consensus is the order to take.
\end{enumerate}

Paxos est ce qui est le plus utilisé pour les consensus (TODO)

\begin{itemize}
    \item Asynchronous : There is no bound on the time for a message to arrive and to be computed, it resolve consensus iff 0 node crashes
    \item Partially synchronous : It start asynchronous and then become synchronous(it get an upper bound, we know it will happen but we don't know when.
	  Consensus sit < $\frac{n}{2}$ crashes
    \item Synchronous : Bound known for delivering and computation of message.Consensus with n-1 crashes
\end{itemize}

\paragraph{Asynchronous vs Synchronous}

Bound is simulated with a expect bound to be in partially synchronous.

\subsection{Failure detector}
Bound exist but we don't know the exact value because this bound can
change with time (if RTT increase for example)

\section{Formal models of distributed system}




\begin{thebibliography}{1}
\bibitem{wikiadmheur} http://en.wikipedia.org/wiki/Admissible\_heuristic, {\em Wikipedia}
%\bibitem{Propagation} P.G. Fontolliet, {\em Traité d'Electricité}, Volume XVIII, Ecole polytechnique fédéral de Lausanne, pp 72-73
\end{thebibliography}

\end{document}
