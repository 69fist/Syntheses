\clearpage{}
\section{Explain the goals and principles of postmortem analysis. Describe
the Capability Maturity Model and its five levels. Compare to SPICE and ISO
9000.}

\subsection{Postmortem analysis}

\paragraph{Idea:} analyze what went wrong.

\paragraph{}
A postmortem analysis is a post-implementation assessment of all aspects of the
project (products, processes and resources). It is intended to determine whether
goals were met or not, and to identify areas of improvement for future projects.\newline

The analysis usually takes place shortly after a project is completed (in
practice, it can take place from just before delivery to 12 months afterwards).

\paragraph{Postmortem analysis process}

\begin{itemize}
    \item \textbf{Project survey}
        \subitem{} Ask no more than needed, without compromising confidentiality.
    \item \textbf{Objective information}
        \subitem{} Collect relevant data: cost (effort, LOC), schedule, quality.
    \item \textbf{Debriefing meeting}
        \subitem{} Allow team members to report problems.
    \item \textbf{Project history day}
        \subitem{} Identify the root causes of the key problems, review schedule predictability charts.
    \item \textbf{Publishing the results}
        \subitem{} Focus on lessons learned.
\end{itemize}

\subsection{Capability Maturity Model}
A model should examine the software process as a whole and characterize
the elements of a good process. The CMM has 5 levels of maturity, each with a
set of key process areas. Note that these are not 5 discrete rankings, but a
continuous scale.\newline

The assessment is based on:
\begin{itemize}
    \item A questionnaire
    \item Requests for evidence to verify the answers
\end{itemize}

\begin{figure}[!ht]
    \centering
    \begin{scriptsize}
        \begin{tikzpicture}[node distance=3cm, on grid, auto]
    \node[draw, rectangle] (l1) {Level 1: Initial};
    \node[draw, rectangle, above right =of l1] (l2) {Level 2: Repeatable};
    \node[draw, rectangle, above right =of l2] (l3) {Level 3: Defined};
    \node[draw, rectangle, above right =of l3] (l4) {Level 4: Managed};
    \node[draw, rectangle, above right =of l4] (l5) {Level 5: Optimizing};

    \path[->, line width=2pt] (l1) edge [bend left=45] node {Process discipline} (l2);
    \path[->, line width=2pt] (l1) edge [bend right=45, right] node {Process management} (l2);
    \path[->, line width=2pt] (l2) edge [bend left=45] node {Process definition} (l3);
    \path[->, line width=2pt] (l2) edge [bend right=45, right] node {Engineering management} (l3);
    \path[->, line width=2pt] (l3) edge [bend left=45] node {Process control} (l4);
    \path[->, line width=2pt] (l3) edge [bend right=45, right] node {Quantitative management} (l4);
    \path[->, line width=2pt] (l4) edge [bend left=45] node {Continuous process improvement} (l5);
    \path[->, line width=2pt] (l4) edge [bend right=45, right] node  {Change management} (l5);
\end{tikzpicture}

    \end{scriptsize}
    \caption{Capability maturity model}
\end{figure}

\paragraph{Level 1: Initial}
\begin{itemize}
    \item Development process is ad hoc or even chaotic
    \item Cannot describe the process
    \item No key process areas
\end{itemize}

\paragraph{Level 2: Repeatable}
\begin{itemize}
    \item Identified inputs and outputs, constraints (budget, schedule), resources
    \item Project management in place
    \item Measurements on project
    \item Key process areas: management activities
\end{itemize}

\paragraph{Level 3: Defined}
\begin{itemize}
    \item Management and engineering activities are documented, standardized and integrated
    \item Measurements on products
    \item Key process areas: organization
\end{itemize}

\paragraph{Level 4: Managed}
\begin{itemize}
    \item Quality is measured, tracked and managed
    \item Early projects feedback to later projects
    \item Key process areas: quantitative and quality management
\end{itemize}

\paragraph{Level 5: Optimizing}
\begin{itemize}
    \item Quantitative feedback is incorporated to produce continuous process improvement
    \item Key process areas: change management
\end{itemize}

\subsection{SPICE (ISO Standard 15504)}

This is an international standard for process assessment. SPICE stands for
Software Process Improvement and Capability dEtermination. It harmonizes and extends the existing process assessment methods (CMM and its
descendants).\newline

Six levels of capability for each process area:
\begin{itemize}
    \item[0.] \textbf{Not performed}
    \item[1.] \textbf{Performed informally}
    \item[2.] \textbf{Planned and tracked}
    \item[3.] \textbf{Well-defined}
    \item[4.] \textbf{Quantitatively controlled}
    \item[5.] \textbf{Continuously improved}
\end{itemize}

Whereas the CMM addresses organization, SPICE addresses processes.

\subsection{ISO 9000}

ISO 9000 is a series of standards that specify actions to be taken when any
system (i.e.\ not necessarily a software system) has quality goals and
constraints. In particular, ISO 9000 applies when a buyer requires a supplier
to demonstrate a given level of expertise in designing and building a product.
\newline

Among the ISO 9000 standards, ISO 9001 is the most applicable to the way we develop and
maintain software. It explains what a buyer must do to ensure that the supplier conforms to
design, development, production, installation and maintenance requirements.
\newline

Since ISO 9001 is quite general, there is ISO 9000--3 that provides guidelines for interpreting
ISO 9001 in a software context.
\newline

The ISO 9000 standards are used to regulate internal quality and to ensure the quality of
suppliers.
\newline

\paragraph{Summary:}
\begin{itemize}
    \item \textbf{ISO 9000:} quality management systems and principles.
    \item \textbf{ISO 9001:} requirements that organizations must fulfil.
    \item \textbf{ISO 9000--3:} guidelines for applying ISO 9001 to software.
\end{itemize}
