\documentclass[en]{../../../eplexercises}

\hypertitle{translators-INGI2132}{8}{INGI}{2132}
{Author1\and Author2\and Author3}
{Professor}


\section{Lexical Anal. IS-IS}

\section{Grammar}

\subsection{Give an example of ambiguity in a real programming language. How is it managed in practice?}
\todo{explain}
\begin{align}
E ::=& E+E & & E ::= TE'\\
E ::=& E\times E & & E' ::=  +TE'\\
E ::=& (E) & &E' ::= \epsilon\\
E ::=& id & & T ::= \times FT'\\
& & & T' ::= \epsilon\\
& & & F ::= id\\
& & & F ::= (E)
\end{align}
À gauche, elle est ambigüe (on peut avoir deux arbres) mais pas à droite ($LL(1)$)

\section{Parsing}

\section{Chapitre 4 : Type checking}

\subsection{Why a two phase analysis ? Describe what each phase does.}

L'analyse sémantique de programmes j-- (et Java) requière deux parcours de l'AST parce qu'un nom de classe ou le nom d'un membre peut être référencé avant qu'il ne soit décllaré dans le code source.

\subsubsection{pre-analyse}

La pré-analyse traverse l'AST moins profondément que l'analyse.
Cependant, elle va suffisamment loin pour :
\begin{itemize}
\item Les déclarations des noms des types importés.
\item Les déclaration des noms des classes définies par l'utilisateur.
\item Les déclarations de champs.
\item Les déclarations de méthodes (incluant leur signature et le type des paramètres).
\end{itemize}

Pour cette raison, \texttt{preAnalyze()} ne doit être défini que pour certains noeuds de l'AST : \texttt{JCompilationUnit, JClassDeclaration, JFieldDeclaration, JMethodDeclaration, JConstructorDeclaration}.

Donc, comme on peut le constater, la phase de pré-analyse ne parcourt pas le corps des méthodes.

\subsubsection{analyse}

Elle s'effectue logiquement après la pré-analyse, et parcourt l'entièreté de l'AST, jusqu'au feuilles.
Elle sert à :
\begin{itemize}
\item Réécrit les initialisations de champs et de variables locales comme des assignements.
\item Déclare les paramètres formels et les variables locales.
\item Alloue des stack frames pour les paramètres formels et les variables locales.
\item Résout le type des expressions et assure les règles de types du langage.
\item Reclassifier les noms ambigus.
\item Faire une partie de "tree surgery".
\end{itemize}

\subsection{When is the final modfifier of classes checked ?}

Dans la méthode \texttt{preAnalyze()} applicable aux noeuds de type \texttt{JClassDeclaration}.

Si une classe est "finale", elle ne peut pas être étendue. On vérifie donc pour chaque classe que le super-type n'est pas final.

\subsection{When is abstract method body checked ?}

Une méthode abstraite possède comme particularité qu'elle ne peut pas avoir de corps.
On vérifie cela dans la méthode \texttt{preAnalyze()} applicable sur les noeuds de type
\texttt{JMethodDeclaration}.

\subsection{Explain memory allocation ? How variable offset are computed ?}

Voir pages 143 à 152 du bouquin. \#JeFaisCommePoulpyPourReseau\\
Mémoire: this, parameters, block (nextOffset + la place en mémoire)

\subsection{Explain how variable shadowing is detected.}

Dans le context de la méthodes, on alloue les stack frames (voir question précédente) au fur et à mesure de l'avancement dans la méthode. Chacune de ses frames contient, entre autre, le nom de la variable. Ce qu'on fait, c'est que dans la méthode \texttt{analyze()} applicable sur les noeuds de type \texttt{JVariableDeclaration}, avant d'ajouter la variable au \texttt{LocalContext}, on fait un lookup dans ce context pour vérifier que le nom n'est pas déjà utilisé. Si c'est le cas, on reporte une erreur ; sinon, on ajoute la variable au context.

\subsection{Explain where type-checking is done ?}

Dans la méthode \texttt{analyze()} applicable sur les noeuds de type \texttt{JVariable}.\\
Surtout pour les check: if avec un boolean, cast correct

\subsection{What do we do for the cast ?}

Ca se fait dans la méthode \texttt{analyze()} pour les noeuds \texttt{JExpression}. On vérifie le type dans lequel on veut
caster l'expression. S'il est valide, alors on crée un \texttt{Converter} qui va être utilisé au moment de la génération de code pour générer le code requis pour le cast. S'il n'y a aucun \texttt{Converter} défini, alors on peut conclure que le cast est invalide.




\end{document}
