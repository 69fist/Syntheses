\documentclass[en]{../../../eplerrata}

\newcommand{\la}{\langle}
\newcommand{\ra}{\rangle}

\DeclareExpandableDocumentCommand{\erratumcorrected}{mmmmmm}{\erratum[It is how it was corrected in \cite[p.~#1, l.~$#2$]{nesterov2004introductory}.]{#3}{#4}{#5}{#6}}

\hypertitle{Nonlinear programming}{8}{INMA}{2460}
{Benoît Legat}
{Yurii Nesterov}

\section{Introductory Lectures on Convex Programming~\cite{nesterov1998introductory}}

\begin{errata}
  \erratum{5}{-4}{May be}{Maybe}
  \erratumcorrected{xvi}{-13}{6}{9}{been never}{never been}
  \erratum{7}{16}{cam}{can}
  \erratum{14}{13}{$\la x, x \ra$}{$\sqrt{\la x, x \ra}$ ou $\la x, x \ra^{\frac{1}{2}}$}
  \erratum{15}{-5}{$\lfloor L \frac{\sqrt{n}}{2\epsilon} \rfloor \lfloor$}{$\lfloor L \frac{\sqrt{n}}{2\epsilon} \rfloor$}
  \erratum{16}{-12}{$]\frac{L}{2\epsilon}[$}{$\lfloor \frac{L}{2\epsilon} \rfloor$}
  \erratumcorrected{12}{-1}{18}{13}{let we}{if we}
  \erratumcorrected{13}{-12}{18}{-8}{may be}{maybe}
  \erratum{21}{-12}{$\lim_{y_k \to x_k}$}{$\lim_{y_k \to \bar{x}}$}
  \erratum{22}{1}{$f'(\bar{x}$}{$f'(\bar{x})$}
  \erratum[We use Lemma~1.2.3 \cite[p.~39 l.~4]{nesterov1998introductory}]{38}{-16}{$f \in C_L^{2,2}$}{$f \in C_L^{1,1}$}
  \erratum[$f$ is not defined here, the objective is $f_0$.]{47}{-8}{$\lim_{k \to \infty} f(x_k)$}{$\lim_{k \to \infty} f_0(x_k)$}
  \erratum{47}{-1}
  {we have $\Phi(x_*)=0$. Therefore $x_* \in Q$ and $\Psi^* = f_0(x_*) + \Phi(x_*) = f_0(x_*) \geq f^*$.}
  {See the errata of~\cite{nesterov2004introductory}.}
  \erratum{69}{-2}{$\phi_{k+1}^* = (1-\alpha_k)\phi_k$}{$\phi_{k+1}^* = (1-\alpha_k)\phi_k^*$}
  \erratum{71}{6}{$x_{k+1}=y_k-h_kf'(x_k)$}{$x_{k+1}=y_k-h_kf'(y_k)$}
  \erratum{81}{10}{$x_Q=x_Q(\gamma,\bar{x}),g_Q=g_Q(\gamma,\bar{x})$}{$x_Q=x_Q(\bar{x};\gamma),g_Q=g_Q(\bar{x};\gamma)$}
  \erratumcorrected{89}{5}{83}{1}{$\{\phi)k(x)\}$}{$\{\phi_k(x)\}$}
  \erratumcorrected{118}{-8}{109}{-2}{$\phi(\mathcal{A}(x)$}{$\phi(\mathcal{A}(x))$}
  \erratumcorrected{121}{-1}{112}{-8}{Lipshitz}{Lipschitz}
  \erratumcorrected{127}{6}{117}{7}{cinvexity}{convexity}
  \erratum{119}{-11}{$\la g, x_0-x$}{$\la  g, x_0-x \ra$}
  \erratumcorrected{137}{-15}{126}{3}{staring}{starting}
  \erratumcorrected{140}{8}{128}{-16}{$\omega_f(v_f(\bar{x};x))$}{$\omega_f(\bar{x}; v_f(\bar{x};x))$}
  \erratumcorrected{140}{17}{128}{-7}{$\la g(x), y - x \ra$}{$\la g(x), y - \bar{x} \ra$}
  \erratumcorrected{137}{8}{129}{9}{$\omega_f(v_k^*)$}{$\omega_f(x^*; v_k^*)$}
  \erratum{130}{-2}{$f_k^*$}{$f_N^*$}
  \erratumcorrected{143}{-6}{131}{8}{3.2.3)}{(3.2.3)}
  \erratum{131}{-17}{$i=1 \ldots m$}{$j=1, \ldots, m$}
  \erratumcorrected{146}{4}{133}{13}{iur}{our}
  \erratum{133}{-6}{abou}{about}
  \erratumcorrected{146}{-13}{133}{-5}{repotrts}{reports}
  \erratumcorrected{146}{-4}{134}{3}{$x \in Q$}{$, x \in Q$}
  \erratumcorrected{148}{5}{135}{-7}{inceased}{increased}
  \erratum{136}{12}{THe}{The}
  \erratumcorrected{148}{-3}{136}{-9}{thatthe}{thatthe}
  \erratumcorrected{150}{11}{138}{5}{Insted}{Instead}
  \erratumcorrected{151}{14}{139}{4}{immedeately}{immediately}
  \erratumcorrected{152}{-6}{140}{9}{certer}{center}
  \erratumcorrected{154}{-1}{142}{5}{$H_k^{-1}(x-x_k),x-x_k$}{$H_k^{-1}(x-y_k),x-y_k$}
  \erratum{142}{11}{$B_0(x_0,R)$}{$B_2(x_0,R)$}
  \erratumcorrected{155}{-8}{143}{1}{feasble}{feasible}
  \erratumcorrected{156}{2}{143}{9}{dependedce}{dependence}
  \erratumcorrected{156}{17}{143}{-9}{volumetrictic}{volumetric}
\end{errata}

\section{Introductory Lectures on Convex Programming~\cite{nesterov2004introductory}}

\begin{errata}
  \erratum[We use Lemma~1.2.3 \cite[p.~38 l.~1]{nesterov2004introductory}]{38}{-16}{$f \in C_L^{2,2}$}{$f \in C_L^{1,1}$}
  \erratum[$f$ is not defined here, the objective is $f_0$.]{48}{9}{$\lim_{k \to \infty} f(x_k)$}{$\lim_{k \to \infty} f_0(x_k)$}
  \erratum[$f^*$ was used in \cite{nesterov1998introductory} and replaced by $f(x^*)$. This one was forgotten.]{48}{12}{$\Psi^* \leq f^*$}{$\Psi^* \leq f(x^*)$}
  \erratum[$\Psi^* = f_0(x_*) + \Phi(x_*) = f_0(x_*)$ is incorrect, which $t$ to use ? The limit ?
  The $\Phi(x_*) \lim t_k = 0 \cdot \infty$ so we cannot say much. Also the fact that $\lim t_k \to \infty$ is not the reason that $f_0(x_*) \leq f_0(x^*)$.]{48}{-13}
  {we have $\Phi(x_*)=0$ and $f_0(x_*) \leq f_0(x^*)$. Thus $x_* \in Q$ and $\Psi^* = f_0(x_*) + \Phi(x_*) = f_0(x_*) \geq f_0(x^*)$.}
  {we have $\Phi(x_*)=0$. Thus $x_* \in Q$ and $f_0(x_*) \geq f_0(x^*)$.
  Since $f_0(x_*) \leq \Psi^* \leq f_0(x^*)$, we can conclude that $f_0(x_*) = f_0(x^*)$.}
  \erratum{73}{-2}{$\phi_{k+1}^* = (1-\alpha_k)\phi_k$}{$\phi_{k+1}^* = (1-\alpha_k)\phi_k^*$}
  \erratum{75}{-11}{$x_{k+1}=y_k-h_kf'(x_k)$}{$x_{k+1}=y_k-h_kf'(y_k)$}
  \erratum{129}{-3}{$\la g, x_0-x$}{$\la  g, x_0-x \ra$}
  \erratum{143}{12}{$f_k^*$}{$f_N^*$}
  \erratum{144}{4}{$i=1 \ldots m$}{$j=1, \ldots, m$}
  \erratum{155}{8}{$B_0(x_0,R)$}{$B_2(x_0,R)$}
\end{errata}

\biblio

\end{document}
