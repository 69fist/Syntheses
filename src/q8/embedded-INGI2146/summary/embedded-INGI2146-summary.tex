\documentclass[en]{../../../eplsummary}

\hypertitle{embedded-INGI2146}{8}{INGI}{2146}
{Gorby Nicolas Kabasele Ndonda\and Author2\and Author3}
{Professor}

\section{Introduction}
\begin{description}
	\item[Mobile:] Portable devices with wireless communication,
	running stand-alone or client applications.
	\item[Embedded:] An embedded system is a computer system with
	a dedicated function within a larger mechanical or electrical system.
\end{description}
Embedded systems are everywhere in smartphones, cars,\ldots Typical
characteristics is the following:

\begin{itemize}
	\item Cheap
	\item Reduced power consumption
	\item Real-time
	\item Robust
\end{itemize}
To achieve these characteristic, there need to be a tight coupling between
the hardware, the OS and the application.

\paragraph{Real-Time Systems}
Real-Time Systems monitor and have an impact on the physical world via sensors
and actuators. Because of their nature, they have requirements/constraints on timing.
\begin{description}
	\item[Hard constraint] Example Electronic Engine
	\begin{itemize}
		\item Potentially severe consequences if reaction/result is produced after
		deadline.
		\item Results has no value (or even negative value) after deadline
	\end{itemize}
	\item[Firm constraint] Cruise Control
	\begin{itemize}
		\item Occasional miss tolerated, but degrades QoS
		\item No or little value after deadline
	\end{itemize}
	\item[Soft constraint]  User Interface
		\begin{itemize}
			\item Result still has value after deadline
		\end{itemize}
\end{description}

\paragraph{Real-Time Tasks}
\begin{description}
	\item[Periodic:] Must be executed with fixed interarrival time
	\item[Sporadic:] Have known minimum interarrival time
	\item[Aperiodic:] No known interarrival time
\end{description}

\section{Wireless Sensor Networks}
Wireless Sensor Networks are composed of node used for monitoring and they
communicates with each other and/or a remote server.
\subsection{Resource Constraint}
\begin{itemize}
	\item Power
	\begin{itemize}
		\item Batteries and Energy Harvesting
		\item Turn off idle devices
		\item Requires efficient OS and applications
		\item Reduce communication
		\item Duty cycle is the percentage of one period in which a signal or system
		active.
	\end{itemize}
	\item Bandwidth
	\item Memory: just a few kBytes
	\item CPU: a few MHz
	\item Data Transmission
	\begin{itemize}
		\item Multihop wireless Network to reach servers
		\item Self-Organizing Networks
	\end{itemize}
	\item Security: Not enough resources for sophisticated intrusion detection,
	encryption.
	\item Hardware: Must be robust enough to support harsh environment
\end{itemize}
IoT = network of physical objects equipped with electronics and network
connectivity that can be sensed and controlled remotely.

\section{Operating Systems}
\subsection{Hardware}
\begin{description}
	\item[Microcontroller Unit] It contains the
	\begin{itemize}
		\item CPU
		\item Memory (RAM and non-volatile)
		\item Interfaces to sensors and actuators
		\item Communication interfaces and interfaces to external memory
	\end{itemize}
	\item[JTAG/BSL] Allows to write code and data into flash memory or RAM
	of the MCU without using one of the communication interface
	\begin{itemize}
		\item Can be used to debbuging code on the MCU from a PC
		\item Can upload code/data to the memory without any OS or program
		running.
	\end{itemize}
	\item[Transceiver] Used to send data and receive data 
	\item[Interfaces and Sensors] Interfaces to sensor/Actuators
\end{description}

\subsection{Interrupts}
In order to make a software that react to the physical world two strategy:
\begin{description}
	\item[Polling] Read status of I/O every few milliseconds $\to$ waste of
	energy because CPU always busy
	\item[Interrupts] Signal to processor that current code execution should be interrupted
	because something important has happened
	\begin{itemize}
		\item CPU stops execution and jumps to a specific place in the code that is 
		responsible for handling the interrupt
		\item Exist in all CPUs
		\item Handle by OS and device drivers
		\item Interrupts are expensive becauce of context switching
	\end{itemize} 
\end{description}

\subsection{OS for embedded systems}
OS are not necessarly needed but they free the programmer from some concerns
\begin{itemize}
	\item Handle interrupts (events triggered when something happens)
	\item Manage memory
	\item Write your own process scheduler if you need multi-processing for
	background task
\end{itemize}
\paragraph{Note:}
Linux can be used for embedded systems but it requires some memory and is 
really suited for these kind of systems$\to$ preemptive scheduling leads to 
latency for user code as kernel code cannot be preempted.

\subsubsection{Real-Time OS}
OS specifically designed for real-time systems:
\begin{itemize}
	\item Much smaller than desktop/server OS
	\item Reduced set of functionality
	\item Fast context switch
	\item Guaranteed time-bounded response to interrupts
	\item Tries to avoid code that cannot be preempted
\end{itemize}

\subsubsection{OSs for WSN}
\begin{itemize}
	\item Thin abstraction layer to hardware(I/O, timers, \ldots)
	\item Limited multiprocessing
	\item Specific network protocol implementations
	\item Power Saving
\end{itemize}
\paragraph{TinyOS}
TinyOS applications consist of components that you can use in your own
application. Each component is provided with an interfaces defining the 
commands (function) and event available in the component.\\
\begin{itemize}
	\item The device sleeps most of the time and they wake up when
	an event is triggered.
	\item Event-handling function execute and then device goes back to sleep.
\end{itemize}
\subparagraph{Task}
Tasks are functions that canbe preempted by interrupts but not by other tasks.
\begin{itemize}
	\item Task scheduler is a FIFO queue
	\item Not possible to post the same task while it is in the queue
	\item Device sleeps when task queue is empty and no event to process
\end{itemize}
\subparagraph{Synchronous vs Asynchronous}
Hardware events can happen at any time and interrupt the current execution.
This can result in race conditions for variables shared between hardware and
task.
Compile differentiates between
\begin{itemize}
	\item Synchronous code: reachable only from tasks
	\item Asynchronous code: reachable from interrupts
\end{itemize}
Static analysis allows compiler to detect race conditions so programmer
are forced to explicitly use atomic section.
\subparagraph{Optimization}
\begin{itemize}
	\item TinyOS programmed in NesC (subset of C)$\to$ No pointers or 
	dynamic allocation.
	\item Static analysis is used for optimization $\to$ can reduce apps by 60\%
\end{itemize}

\paragraph{Contiki}
Differences with TinyOS
\begin{itemize}
	\item No special description language
	\item Cooperative process instead of task queue
	\item Core (kernal + libraries) is uploaded once
	\item Applications can be loaded at run-time
\end{itemize}
Application are composed of one or multiple processes. Proccesses are
cooperative, meaning that other processes can only run if current process
stops.
\begin{itemize}
	\item Processes communicate with each other by posting event
	\begin{itemize}
		\item Synchronous events: delivered immediately
		\item Asynchronous events: delivered later
	\end{itemize}
	\item Processes implemented as Proto-Threads
	\begin{itemize}
		\item Like threads but cheaper
		\item No separate stack per process $\to$ Local variables lose
		their value after a context switch
	\end{itemize}
	\item Process can be preempted by interrupts triggerd by hardware events.
\end{itemize}
\paragraph{RIOT}
RIOT application programmed in C or C++ and used Multi-threading 
(not event-driven). $\to$ Use priority based scheduling.

\subsubsection{Networking Stacks}
uIP : Ipv4 stack in 4-5 KBytes of code.
\begin{itemize}
	\item One single packet buffer $\to$ easier to manage
	\item Incoming packet must be processed immediately $\to$ Radio interface queue
	incoming packet while processing.
	\item Retransmission is left to the application to recreate lost packet segments
	\item Send segment one at the time, wait ACK before sending next segment
\end{itemize}

\section{TinyDB}
%%TODO
\section{Real-Time Scheduling}
\begin{description}
	\item[Job/processes/threads]: A task to be executed which compete for
	limited resource, the CPU
	\item[Scheduling:] Given N jobs, in which order should the CPU execute
	them
\end{description}
In general purposes OS( Linux,Windows, \ldots) scheduling has the following
goals:
\begin{description}
	\item[Fairness] All jobs should be finally executed (no starvation)
	\item[Optimize response time] Should be as short as possible on average
	\item[Optimize throughput] serve as many jobs as possible on average
	(Time to wait for CPU + Execution Time)
\end{description}
In real-time system it's a bit different, the most important thing is that
task meets their deadline
\begin{itemize}
	\item Don't care about fairness and throughput
	\item Focus on worst-case performance
\end{itemize}

\subsection{Scheduling in general purpose OS}
Based on the idea of Round Robin scheduling which is preemptive with time
slice:
\begin{enumerate}
	\item All jobs wait in a queue for the CPU
	\item Job at the head of the queue receives service for time quantum Q
	\item Then it is interrupted by the next waiting job and goes back to the
	end of the queue
	\item Repeated until job is finished
	\item If Q is small, illusion of parallelism but careful of the cost for
	context switching
\end{enumerate}
By default all jobs are equal: order in queue is order of arrival and same time
slice for all jobs. But there are some variation with different priority and 
different time slice. Priority can be assigned manually or automatically

\subsection{Schedulability}
\begin{itemize}
	\item Given a set of tasks with their periods, deadlines, etc.
	\item Given a scheduling algorithm
\end{itemize}
The task set is \textbf{schedulable} if all jobs meet their deadlines\\
Differents ways to know whether a task set is schedulable:
\begin{itemize}
	\item Measurement on real system : expensive!
	\item Simulation
	\item Mathematical analysis
\end{itemize}
We will focus on mathematical analysis.
\paragraph{Assumption}
\begin{itemize}
	\item Single CPU
	\item No priority inversion
	\item Zero overhead for context switching
	\item Fixed number $N$ of independent periodic tasks with
	period $T_i$ and deadline $D_i$ (relative to begin of period), $i$=1,\ldots, $N$
	\item Task $i$ has a worst-case execution time $C_i$
\end{itemize}

\subsubsection{Rate Monotonic Scheduling}
RM assingns a static priority to each task, the task with the shortest period
has highest priority. Preemption is possible.\\
Assuming $D_i=T_i$ for all tasks, it can be shown that RM is an optimal
preemptive static priority scheduling algorithm:
\begin{itemize}
	\item If a task is not schedulable with RM, no other preemptive static
	priority algorithm can schedule it.
\end{itemize}
If we assume $D_i=T_i$ for all tasks, a sufficient (but too pessimistic) test 
for schedulability with RM is:
$$\sum_{i=1}^N \frac{C_i}{T_i}\leq N(2^{\frac{1}{N}}-1)$$
RM is easy to implement with low overhead since task priorities are set at 
design time and does not change during run time but RM does not 
allow high CPU utilization because scheduling too rigid

\paragraph{Worst Case Execution Time}
To know worst case execution we can either do measurement or use static analysis

\subsubsection{Deadline Monotonic Scheduling}
RM does is not good for $D_i < T_i$, so in DM, task with shortest deadline
has highest priority. DM is an optimal preemptive static scheduling algorithm for $D_i<T_i$.
$$\sum_{i=1}^N \frac{C_i}{D_i}\leq N(2^{\frac{1}{N}}-1)$$

\paragraph{Response Time Analysis}
RTA provides a sufficient and necessary test for all fixed-priority preemptive 
scheduling algorithms (RM,DM,\ldots)\\
A task set is schedulable if worst-case response time $R_i \leq D_i$ for all tasks.
\begin{itemize}
	\item Response time $R_i$ = $C_i + I_i$ where $I_i$ is the
	worst case interference time = amount of time a task is delayed by execution of higher priority tasks
\end{itemize}
\begin{equation}
R_i = C_i + I_i$$ $$R_i = C_i + \sum_{j\in hp(i)} \lceil \frac{R_i}{T_j}\rceil C_j
\label{rta-eq}
\end{equation}
whete hp(i) is the set of tasks with higher priotity than task i 
$\to \lceil \frac{R_i}{T_j}\rceil$ is the number of preemptions of task i by task j.
~\eqref{rta-eq} is a recursive formula so it is solve by finding smallest fixed point:
$$R_i^0 = C_i$$
$$R_i^{m+1} = C_i + \sum_{j \in hp(i)} \lceil \frac{R_i^m}{T_j}\rceil C_j$$

\paragraph{Earliest Deadline First}
Optimal preemptive \textbf{dynamic} priority scheduling algorithm 
where task with earliest absolute deadline has the highest priority. The
absolute deadline is the time when the job is ready + $D_i$.

\subsection{Priority Inversion}
We have assumed that task are independent, in a real systems, tasks can be 
dependent because they access a shared resource in a critical section (CS).
Response time becomes:
%%TODO add imge slide 27 chap Scheduling
$$R_i = C_i + I_i + B_i$$ where $B_i$ is the time the task is 
blocked because a resource has been locked.
\paragraph{Note:} RTA a test only sufficient, not necessary with blocking.\\

A task with highest priority can be \textbf{unboundedly} blocked by tasks with
lower priority.

\subsubsection{Priority Inheritance Protocol}
To avoid this problem, a low-priority task inherits the priority of the 
blocked high-priority task.
\begin{itemize}
	\item When task i is blocked by a CS held by task k 
	and prio(i) > prio(k) $\to$ prio(k) := prio(i)
	\item When task k leaves the CS:
		\begin{itemize}
			\item If task k no longer blocks any tasks, it returns to its old priority
			\item If task k still blocks other tasks, it inherits their highest priority
		\end{itemize}
\end{itemize}

\subsubsection{Priority Ceiling Protocol}
Alternative to priority inheritance, a shared resource R can be accessed
only tasks $S_R = \{ t_1,\ldots,t_m\}$.
\begin{itemize}
	\item Assign a priority ceiling $C_R$ to that resource
	$$C_R = max_{t_i\in S}(prio(t_i))$$
	\item When a task locks that resource, its priority is immediately boosted $C_R$
\end{itemize}

\paragraph{Note:}
Priority inheritance and priority ceiling require a scheduler that can handle
dynamic priorities

\subsection{Aperiodic/sporadic tasks with hard deadlines}
\begin{itemize}
	\item Take the lowest inter-arrival time L of those tasks
	\item Treat them as a virtual periodic task with period L in 
	schedulability analysis
	\item Too pessimistic
\end{itemize}

\subsection{Aperiodic/sporadic tasks without hard deadlines}
\begin{itemize}
	\item Maintain the hard deadlines for the periodic tasks
	\item Try to reduce response time for the other tasks
\end{itemize}
Another solution is the Polling Server:
\begin{itemize}
	\item A periodic task (''server'') with period $T_S$ to serve 
	aperiodic/sporadic requests
	\item Incoming aperiodic/sporadic jobs are queued in a queue
	\item In one execution period, the server only runs up to $C_S$ time units
\end{itemize}
Performance can be controlled by the period and the priority of the server
tasks and by $C_S$. 

\begin{itemize}
	\item Advantage: Polling server can be treated like a periodic task with
	WCET $C_S$
	\item Disadvantage: If an aperiodic/sporadic job is not handled by the
	server in a time period, it has to wait for the next period $\to$ 
	response time increases
	\item Can be extended to multiple servers with different priorities,
	periods and $C_S$ for different classes/types jobs.
\end{itemize}
\end{document}
