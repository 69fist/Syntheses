\documentclass[en]{../../../eplsummary}

\newcommand{\E}{\mathbb{E}}
\DeclareMathOperator{\var}{\mathbb{V}ar}

\DeclareMathOperator{\expo}{Expo}
\DeclareMathOperator{\erlang}{Erlang}
\DeclareMathOperator{\po}{Po}

\renewcommand{\P}{\mathbf{P}}
\renewcommand{\r}{\mathbf{r}}
\renewcommand{\v}{\mathbf{v}}
\newcommand{\w}{\mathbf{w}}
\renewcommand{\k}{\mathbf{k}}
\newcommand{\1}{\mathbf{1}}

\usepackage{algorithm}
\usepackage{algorithmic}

% http://tex.stackexchange.com/questions/86883/cancelling-out-cells-in-tables
\usepackage{tikz}
\newcommand{\tikzmark}[1]{\tikz[remember picture,overlay, baseline=-0.5ex]\node (#1){};}
\newcommand{\connect}[3][3mm]{\tikz[remember picture,overlay]\draw[shorten <=-#1, shorten >=-#1](#2)--(#3);}

\hypertitle{Discrete stochastic models}{8}{3}{4}
{Benoît Legat}
{Philippe Chevalier}

\begin{center}
  \begin{tikzpicture}
    \node[left] at (-2,-1) {Poisson Processes};
    \node[left] at (-2, 1) {Renewal Processes};
    \draw[thick,->] (-2, 1) to (-1,.1);
    \draw[thick,->] (-2, -1) to (-1,-.1);
    \node[right] at (-1,0) {Markov Processes};
    \draw[thick,->] (.8, .8) to (.8,.2);
    \node[above] at (.8, .8) {Markov Chains};
    \draw[thick,->] (2.8, 0) to (3.4,0);
    \node[right] at (3.4, 0) {Queueing Systems};
    \draw[thick,->] (2.5, 1) to (3.4,1);
    \node[right] at (3.4, 1) {Markov Decision Processes (MDP)};
  \end{tikzpicture}
\end{center}

We can see that the Semi-Markov Processes are the generalization of all the process we will see.

\begin{center}
  \begin{tabular}{l|l|l}
    & No state & $X_n$ only dependent on $X_{n-1}$\\
    \hline
    No transition time & & Markov Chain\\
    \hline
    $T_n$ general & Renewal Process & Semi-Markov Process\\
    \hline
    $T_n \sim \expo(\lambda)$ & Poisson Process & Markov Process
  \end{tabular}
\end{center}

\section{Probability}
For more information for this chapter, read \cite[\S1.7]{gallager1995discrete} or \cite[\S1.5]{gallager2010discrete}.
%\[ \bar{F}_X(x) = 1 - F_X(x) \]

\begin{myineg}[Markov's inequality]
  Let $Y$ be a positive random variable.
  We have
  \[ \Pr[Y \geq y] \leq \frac{1}{y}\E[Y] \]
  \begin{proof}
    We can see that
    \begin{align*}
      E[Y]
      & = \int_0^\infty x f(x) \dif x\\
      & \geq \int_y^\infty x f(x) \dif x\\
      & \geq y \int_0^y f(x) \dif x\\
      & = y \Pr[Y \geq y].
    \end{align*}
  \end{proof}
\end{myineg}

If the variance is known, there is a useful corollary
\begin{myineg}[Chebyshev's inequality]
  Let $Y$ be a random variable.
  For an $\epsilon > 0$, we have
  \[ \Pr[|Y-\E[Y]| \geq \epsilon] \leq \frac{\sigma_Y^2}{\epsilon^2}. \]
  \begin{proof}
    Let $Z = (Y - \E[Y])^2$, we see that $Z$ is a positive random variable so we can use Markov's inequality with $z = \epsilon^2$:
    \begin{align*}
      \var[Y]
      & = E[Z]\\
      & \geq \epsilon^2 \Pr[Z \geq \epsilon^2]\\
      & = \epsilon^2 \Pr[(Y - \E[Y])^2 \geq \epsilon^2].
    \end{align*}
  \end{proof}
\end{myineg}

It is easy to prove the weak law of large numbers using Chebyshev's inequality.
\begin{mytheo}[Weak Law of Large Numbers]
  Let $X_1, \ldots, X_n$ we idependent and identically distributed random variables
  with finite mean $\bar{X}$ and finite variance $\sigma_X^2$.
  Denote their sum
  \[ S_n = X_1 + \cdots + X_n. \]
  For any $\epsilon > 0$,
  \[ \lim_{n \to \infty} \Pr\Big[ \Big|\frac{S_n}{n} - \bar{X}\Big| \geq \epsilon \Big] = 0. \]

  \begin{proof}
    Let's fix $n$ and consider the random variable $Y = S_n/n$.
    It is easy to see that $\E\{Y\} = \bar{X}$ and $\var\{Y\} = \sigma_X^2/n$~\cite[(1.41)]{gallager2010discrete}.
    Applying the Chebyshev's inequality we get
    \[ \Pr[|Y - \E[Y]| \geq \epsilon] \leq \frac{\sigma_X^2}{n\epsilon^2}. \]
    Since $\sigma_X^2 = 0$, we see directly that
    \[ \lim_{n \to \infty} \Pr[|Y - \E[Y]| \geq \epsilon] = 0. \]
  \end{proof}
\end{mytheo}

\begin{mytheo}[Central limit theorem (CLT)]
  Let $X_1, X_2, \ldots$ be iid random variables with
  finite mean $\bar{X}$ and finite variance $\sigma^2$.
  Then for every real number $z$,
  \[ \lim_{n \to \infty} \Pr\left\{ \frac{S_n-n\bar{X}}{\sigma\sqrt{n}} \sim \Phi(z) \right\} \]
  where $\Phi$ is the normal distribution function
  \[ \Phi(z) = \int_{-\infty}^z \frac{1}{\sqrt{2\pi}} \exp\Big(-\frac{y^2}{2}\Big) \dif y. \]
\end{mytheo}

\begin{center}
``The strong law requires considerable patience to understand, but it is a basic and essential result in understanding stochastic processes''\hfill\cite[p.~35]{gallager2010discrete}.
\end{center}

\begin{mytheo}[Strong Law of Large Numbers (SLLN)]
  For each integer $n \geq 1$.
  Let $X_1, X_2, \ldots$ be iid random variables with $\E[|X|] < \infty$.
  Then
  \[ \Pr\Big\{ \lim_{n \to \infty} \frac{S_n}{n} = \bar{X} \Big\} = 1. \]
\end{mytheo}

In short, the Weak Law of Large Numbers guarantees a convergence in probability and the Strong Law a convergence in distribution.

\section{Poisson Processes}
\cite[\S2]{gallager1995discrete,gallager2010discrete}

A Poisson Process is represented by an arrival rate $\lambda$.

The time between two consecutive arrivals are represented by the iid random variable $X_i$ (see \figuref{poisson}).
For a fixed $i$, $X_i \sim \expo(\lambda)$ is a continuous random variable for which
\begin{align*}
  f_X(t) & = \lambda \exp(-\lambda t), & x \geq 0,\\
  F_X(t) & = 1 - \exp(-\lambda t), & x \geq 0,\\
  \E[X]  & = \frac{1}{\lambda},\\
  \var[X]  & = \frac{1}{\lambda^2}.
\end{align*}
The key property of the exponential distribution is the lack of memory:
$\forall t,x > 0$,
\[ \Pr[X > t+x \mid X > t] = \Pr[X > x]. \]
It somewhat defies the intuition.
If there have been no arrival for a lot of time, it does not mean that it is more likely that there will be an arrival soon.

The exponential law shouldn't be misinterpreted, it really means that the chance of arrival is uniform in time.
However the uniform distribution doesn't fit here
since $X_1$ represents the time for one arrival so we kind of stop counting when someone has arrived.
Therefore of course $f_{X_1}(t_1) > f_{X_1}(t_2)$ if $t_1 < t_2$.
That doesn't mean that it is less likely that someone arrive at $t_2$ than $t_1$.
It just means that it is less likely that the \emph{first} arrival will be at $t_2$ than $t_1$.

Their sum, $S_n$ is the time from the start the the $n$th arrival.
For a fixed $n$, $S_n \expo \erlang_n(\lambda,n)$ is a continuous random variable for which
\begin{align*}
  f_{S_n}(t) & = \frac{\lambda^nt^{n-1}\exp(-\lambda t)}{(n-1)!}, & t \geq 0,\\
  F_{S_n}(t) & = \frac{\gamma(n,\lambda t)}{(n-1)!} = 1 - \sum_{k=0}^{n-1} \frac{1}{k!}\exp(-\lambda t) (\lambda t)^n, & t \geq 0\\
  \E[S_n] & = \frac{n}{\lambda},\\
  \var[S_n] & = \frac{n}{\lambda^2}.
\end{align*}

$N(t)$ is the number of arrival from the start to the time $t$.
For a fixed $t$, $N(t) \sim \po(\lambda t)$ is a discrete random variable for which
\begin{align*}
  \Pr[N(t) = n] & = \frac{(\lambda t)^n\exp(-\lambda t)}{n!}, & n \geq 0,\\
  F_{N(t)}(n)   & = \frac{\Gamma(n+1,\lambda t)}{n!} = \exp(-\lambda t) \sum_{i=0}^n \frac{(\lambda t)^i}{i!}, & n \geq 0\\
  \E[N(t)]      & = \lambda t,\\
  \var[N(t)]    & = \lambda t.
\end{align*}
The key property of $N$ is
\begin{description}
  \item[Property of stationary increment] The number of arrivals between $t$ and $t'$ is
    \[ N(t') - N(t) = N(t'-t) - N(t-t) = N(t'-t). \]
  \item[Property of independent increment] For any sequence $0 < t_1 < t_2 < t_2 < \cdots < t_k$,
    the random variables $N(t_1), N(t_2-t_1), \ldots, N(t_k-t_{k-2})$ are independent.
\end{description}
For small $t = \delta$, we can use the following equations to build approximations
\begin{align}
  \label{eq:inc1}
  \Pr[N(\delta) = 0]    & = 1 - \lambda \delta + o(\delta)\\
  \label{eq:inc2}
  \Pr[N(\delta) = 1]    & = \lambda \delta + o(\delta)\\
  \label{eq:inc3}
  \Pr[N(\delta) \geq 2] & = o(\delta)
\end{align}
where $o$ is such that $\lim_{x \downarrow 0} \frac{o(x)}{x} = 0$ and $o(0) = 0$.

It is important to stress the fact that on the contrary to what the intuition would say,
if there have been many arrival or few arrival preceding a time $t$, it won't affect the time before the next arrival after $t$.

\begin{figure}[!ht]
  \centering
  \begin{tikzpicture}
    \draw[thick,->] (0,0) to (11.5,0);
    \draw (0,.1) to (0,-.1);
    \draw[<->] (0,-1) to (1,-1);
    \node[below] at (.5,-1) {$X_1$};
    \draw (1,.1) to (1,-.1);
    \node[below] at (1,-.1) {$t_1$};
    \draw[<->] (1,-1) to (3.5,-1);
    \node[below] at (2.25,-1) {$X_2$};
    \draw (3.5,.1) to (3.5,-.1);
    \node[below] at (3.5,-.1) {$t_2$};
    \draw[<->] (3.5,-1) to (7,-1);
    \node[below] at (5.25,-1) {$X_3$};
    \draw (7,.1) to (7,-.1);
    \node[below] at (7,-.1) {$t_3$};
    \draw[<->] (7,-1) to (8,-1);
    \node[below] at (7.5,-1) {$X_4$};
    \draw (8,.1) to (8,-.1);
    \node[below] at (8,-.1) {$t_4$};
    \draw[<->] (8,-1) to (11,-1);
    \node[below] at (9.5,-1) {$X_5$};
    \draw (11,.1) to (11,-.1);
    \node[below] at (11,-.1) {$t_5$};

    \draw[<->] (0,-2) to (3.5,-2);
    \node[below] at (1.75,-2) {$S_2$};
    \draw[<->] (0,-2.5) to (8,-2.5);
    \node[below] at (4,-2.5) {$S_4$};

    \draw[dashed] (5,.3) to (5,0);
    \node[above] at (5,.3) {$N(t) = 2$};

    \draw[dashed] (10,.3) to (10,0);
    \node[above] at (10,.3) {$N(t) = 4$};
  \end{tikzpicture}
  \caption{Example of Poisson Process}
  \label{fig:poisson}
\end{figure}

A crucial question that the reader is totally right to ask now is:
how do we prove that ``something'' is a Poisson Process ?
Excellent question ! Let me answer this using the following theorem.
\begin{mytheo} % TODO define renewable process
  Consider a process counting arrivals.
  \begin{itemize}
    \item The process is a Poisson Process with arrival rate $\lambda$.
    \item The process is a renewable process (i.e. iid arrivals) for which the interarrival intervals have an exponential distribution function with parameter $\lambda$.
    \item The counting process has independent and stationary increment properties and for all $y \geq 0$, \( N(t) \sim \po(\lambda t) \).
    \item The counting process has independent and stationary increment properties and for all \eqref{eq:inc1}, \eqref{eq:inc2} and \eqref{eq:inc3} hold for all $t,\delta \geq 0$
  \end{itemize}
\end{mytheo}

\paragraph{Combining Poisson Processes}
If we combine two \emph{independent} poisson processes with arrival rate $\lambda_1$ and $\lambda_2$,
the combination has arrival rate $\lambda_1 + \lambda_2$.

\paragraph{Subdividing a Poisson Process}
If an arrival for a Poisson Process is either redirected to
the process 1 with probability $p$ or to
the process 2 with probability $1-p$, then
the process 1 and 2 are \emph{independent} poisson processes with respective arrival rate
$\lambda_1 = p\lambda$ and $\lambda_2 = (1-p)\lambda$.

\subsection{Symmetry and order}
If it is known that $N(T) = n$,
$S_1, \ldots, S_n$ behave like uniform random variables of support $[0,T]$.
Precisely, we can see the sampling process for $S_1, \ldots, S_n$
as picking $n$ uniform random variables $U_1, \ldots, U_n$ in $[0,T]$ then sorting them
and giving to $S_i$ the value of the $i$th one.

Since we sort them, the $n$ values $s_1, \ldots, s_n$ have probability 0 to happen if they are not in nondecreasing order (of course),
but if they are in increasing order, there are $n!$ different choice of $u_1, \ldots, u_n$ that give those values of $s_1, \ldots, s_n$.
Therefore, we have
\[ f_{S_1, \ldots, S_n}(s_1, \ldots, s_n \mid N(T) = n) = \frac{n!}{T^n}. \]
The $X_i$ are still independent an for $1 \leq i \leq n$,
\begin{align*}
  f_{X_i}(t \mid N(T) = n) & = n \Big(\frac{T-t}{T}\Big)^{n-1}, & t & \geq 0,\\
  F_{X_i}(t \mid N(T) = n) & = 1 - \Big(\frac{T-t}{T}\Big)^n,   & t & \geq 0,\\
  \E[X_i \mid N(T) = n]    & = \frac{t}{n+1}.
\end{align*}
For $S_i$ we have
\begin{align*}
  f_{S_i}(t \mid N(T) = n) & = \frac{t^{i-1}(T-t)^{n-i}n!}{T^n(n-i)!(i-1)!} & t & \geq 0.
\end{align*}

\subsection{Non-homogeneous Poisson Processes}
For now we have assumed that $\lambda(t)$ was constant in time, i.e. $\lambda(t) = \lambda$ $\forall t \geq 0$.

If it is not constant, we lose the property of stationary increment.
We could keep it if we consider a non-linear time scale.
Let's consider small increments $\dif t$.
The property of stationary increment said that $N(t + \dif t) - N(t) = N(\dif t)$.
For Non-homogeneous Poisson Processes, we have instead
\[ N\Big(t + \frac{\dif t}{\lambda(t)}\Big) - N(t) = N\Big(\frac{\dif t}{\lambda_0}\Big). \]
That is, we partition the time axis in increments $\dif t/\lambda(t)$.
We have
\begin{align*}
  \Pr\Big[N\Big(t+\frac{\delta}{\lambda(t)}\Big) - N(t) =    0\Big] & = 1 - \delta + o(\delta)\\
  \Pr\Big[N\Big(t+\frac{\delta}{\lambda(t)}\Big) - N(t) =    1\Big] & = \delta + o(\delta)\\
  \Pr\Big[N\Big(t+\frac{\delta}{\lambda(t)}\Big) - N(t) \geq 2\Big] & = o(\delta)
\end{align*}

From this idea, one can show that~\cite[Theorem~2.5]{gallager2010discrete}
\begin{align*}
  \Pr[N(t_2)-N(t_1) = n] & = \frac{[m(t_1,t_2)]^n \exp[-m(t_1,t_2)]}{n!}, & t_1,t_2 & \geq 0,\\
  m(t_1, t_2)            & = \int_{t_1}^{t_2} \lambda(\tau) \dif \tau.
\end{align*}
Note that $\Pr[N(t) = n]$ is easily obtained by taking $t_1 = 0$ and $t_2 = t$.

\begin{myexem}
  Let's consider an M/G/$\infty$ Queueing System.
  We want to analyse the number of people in the being serviced at a given time $\tau$.
  Let $N_\tau(t)$ be the number of people arrived between $0$ and $t$ and still being serviced at $\tau$.
  We have the independent increment property.
  Let's see the stationarity of the increment:
  \[ N_\tau(t+\dif t) - N_\tau(t) = (1 - F_G(\tau - t)) N(\dif t). \]
  That means that $\lambda_\tau(t) = (1-F_G(\tau-t)) \lambda$.
  We have
  \begin{align}
    \notag
    m(0,\tau) & = \lambda \int_0^\tau 1 - F_G(\tau - t) \dif t\\
    \notag
              & = \lambda \tau - \lambda \int_0^\tau \int_0^{\tau-t} f_G(t') \dif t' \dif t\\
    \notag
              & = \lambda \tau - \lambda \int_0^\tau f_G(t') \int_0^{\tau-t'} \dif t \dif t'\\
    \notag
              & = \lambda \tau - \lambda \int_0^\tau f_G(t') (\tau-t') \dif t'\\
    \label{eq:mginf}
              & = \lambda \int_0^\tau f_G(t') t' \dif t'
               %& = \lambda \int_0^\infty f_G(t') \int_{t_1}^{\min(t_2,t')}  \dif t \dif t'\\
  \end{align}
  so
  \begin{align*}
    \lim_{\tau \to \infty} m(0,\tau) & = \lambda\E[G].
  \end{align*}
  Therefore
  \[ \lim_{\tau\to\infty} \Pr[N_{\tau}(\tau) = n] = \frac{[\lambda\E[G]]^n \exp[-\lambda\E[g]]}{n!} \]
  hence the number of people being serviced tends to (or is in steady state) a Poisson distribution $\po(\lambda\E[G])$.
  In other word, $N_\tau$ tends to be an homogeneous Poisson Process with arrival rate $\lambda$ that has started at time $\tau - \E[G]$.

  It is interesting to compare \figuref{mginf} with \eqref{eq:mginf}.
  For $\tau < \inf$, we $m(0,\tau) < \lambda \E[G]$ because the
  part of the integral form $\tau$ to $\infty$ is missing.

  \begin{figure}[!ht]
    \begin{subfigure}{0.5\textwidth}
      \begin{tikzpicture}[scale=1.3]
        \draw[thick, domain=0:4] plot
        (\x, {1/2});
        \draw[thick, domain=0:4] plot
        (\x, {exp((\x)-4)});
        \draw[dashed,->] (0,0) to (4.5,0);
        \draw[dashed,->] (0,0) to (0,1.5);
        \draw (-0.1,1/2) to (0.1,1/2);
        \node[left] at (-0.1,1/2) {$\lambda$};
        \draw (-0.1,1) to (0.1,1);
        \node[left] at (-0.1,1) {$1$};
        \draw (4,.1) to (4,-.1);
        \node[below] at (4,-.1) {$\tau$};
        \node[above] at (4,1) {$1 - F_G(\tau-t)$};
      \end{tikzpicture}
      \caption{Plot of $\lambda$ and the probability to be still serviced at $\tau$.
      $\lambda(t)$ is the product of the two curves.}
      \label{fig:mginf}
    \end{subfigure}
    \begin{subfigure}{0.5\textwidth}
      \begin{tikzpicture}[scale=1.3]
        \draw[thick, domain=0:4] plot
        (\x, {1/2});
        \draw[thick, domain=0:3] plot
        (\x, {0});
        \draw[thick] (3,0) to (3,1);
        \draw[thick, domain=3:4] plot
        (\x, {1});
        \draw[dashed,->] (0,0) to (4.5,0);
        \draw[dashed,->] (0,0) to (0,1.5);
        \draw (-0.1,1/2) to (0.1,1/2);
        \node[left] at (-0.1,1/2) {$\lambda$};
        \draw (-0.1,1) to (0.1,1);
        \node[left] at (-0.1,1) {$1$};
        \draw (4,.1) to (4,-.1);
        \node[below] at (4,-.1) {$\tau$};
        \draw (3,.1) to (3,-.1);
        \node[below] at (3,-.1) {$\tau-\E[G]$};
      \end{tikzpicture}
      \caption{$N_\tau(\tau)$ tends to a homogeneous Poisson Process
      of arrival rate $\lambda$ having started at $\tau-\E[G]$.}
    \end{subfigure}
  \end{figure}
\end{myexem}

\section{Renewal Processes}
\cite[\S3]{gallager1995discrete,gallager2010discrete}
A Renewal Process is a counting process for which the times between 2 arrivals are iid random variables.

We define $X_i, S_n, N(t)$ the same way as for Poisson Process except that here $X_i$ is not an exponential.
After an arrival, it is like $t = 0$ since the $X_i$ are iid: i.e.
for $t \geq 0$, $N(S_n+t) - N(S_n)$ has the same distribution than $N(t)$.
However, it is not the same of the property of stationary increment of Poisson process where we can take any $t,t'$.
Here we look at the increment starting at $S_n$ because we cannot start at the middle of an arrival because since the $X_i$
are not exponential they are not memoryless.

The mean $\bar{X} = \E[X_i]$ of $X_i$ is supposed to be finite but the variance $\sigma$ can be infinite.

We have a useful property linking $S_n$ and $N(t)$: $\forall n,t$,
\begin{align}
  \label{eq:snnt}
  S_n \leq t & \Leftrightarrow N(t) \geq n & S_n \geq t & \Leftrightarrow N(t) < n.
\end{align}
Note that since Poisson Processes is a particular case of Renewal Processes, it is also true for Poisson Processes.

We can show that
\[ \lim_{t \to \infty} N(t) = \infty \]
with probability 1 and
\[ \lim_{t \to \infty} \E[N(t)] = \infty. \]
Note that for the second identity,
since $\E[N(t)]$ is deterministic we do not need to specify the type of convergence (e.g. with probability 1, ...).

Using this identity and the String Law of Large Numbers (SLLN), we can show that
\[ \lim_{t \to \infty} \frac{N(t)}{t} = \frac{1}{\bar{X}} \]
with probability 1.

\begin{mydef}[Decision rule]
  A \emph{decision rule} is a sequence $\{I_n\}_{n=1}^\infty$ that is the indicator function of a set $\{\, 1 \leq n \leq N \,\}$ where $N$ is a random function.
\end{mydef}

\begin{mydef}[Stopping time]
  A \emph{stopping time} for a sequence of rv $\{X_i\}_{i=1}^\infty$ is a positive integer rw $N$ such that
  $\forall n$, $I_n$ is independent of $X_i$ for $i \geq n$.
\end{mydef}

Note that for renewal processes, $N(t)$ is not a stopping time.
Indeed, using \eqref{eq:snnt},
we can see that $I_n = [N(t) \geq n] = [S_n \leq t]$ where $\P$ is 1 if $P$ is true and 0 otherwise.

$N(t)+1$ however, is a stopping time,
since that means that $I_n = [N(t)+1 \geq n] = [N(t) \geq n-1] = [S_{n-1} \leq t]$
and $S_{n-1}$ is independent of $X_i$ for $i \geq n$.

\begin{mytheo}[Wald's equality]
  If $N$ is a stopping time of a sequence of rv $\{X_i\}_{i=1}^\infty$ then
  \[ \E[S_N] = \bar{X}\E[N]. \]
\end{mytheo}

Note that now that the distribution of $X_i$ is not necessarily an exponential,
it may be discrete.
The distribution can even be arithmetic,
i.e. $\exists d > 0$ such that $X_i$ can only be a multiple of $d$: $\Pr[X_i/d \in \mathbb{N}] = 1$.
We call $d$ the distribution step.

\begin{mytheo}[Blackwell]
  If the distribution of $X_i$ is arithmetic with a step $d$, then $\forall n \geq 1$,
  \[ \lim_{t \to \infty} [\E[N(t + nd)] - \E[N(t)]] = \frac{nd}{\E[X]}, \]
  otherwise, $\forall \delta \geq 0$,
  \[ \lim_{t \to \infty} [\E[N(t + \delta)] - \E[N(t)]] = \frac{\delta}{\E[X]}. \]
\end{mytheo}
We could be tempted to say that the derivative of $\E[N(t)]$ with $t$ is $1/\E[X]$ but for that we would need to prove that this is still true for $\delta \to 0$.

With a non-arithmetic distribution with $P[X_i = 0] = 0$, we have
\begin{align*}
  \lim_{t \to \infty} \Pr[N(t+\delta) - N(t) =    0] & = 1 - \delta/\bar{X} + o(\delta)\\
  \lim_{t \to \infty} \Pr[N(t+\delta) - N(t) =    1] & = \delta/\bar{X} + o(\delta)\\
  \lim_{t \to \infty} \Pr[N(t+\delta) - N(t) \geq 2] & = o(\delta)
\end{align*}
so asymptotically, we have stationnary increment.

Do we have independent increment asymptotically ?
No.
However if we merge many independent renewal processes we will get closer and closer
to a poisson process with independent increment.

\[ \lim_{t \to \infty} \frac{1}{t} \int_0^t Y(\tau) \dif \tau = \frac{1}{2}\bar{X} + \frac{1}{2}\frac{\sigma^2}{\bar{X}} > \frac{1}{2}\bar{X} \]
with probability 1.

\subsection{Renewal-Reward Processes}
A \emph{reward function} is a function $R(t)$ that only depends on the the particular inter-renewal interval containing $t$.
Let $X(t)$ be the duration of this interval, i.e. $X(t) = X_{N(t)+1} = S_{N(t)+1} - S_{N(t)}$.
We denote the \emph{age} $Z(t) = t - S_{N(t)}$ as the time we have already waited for the current renewal
and the \emph{residual life} $Y(t) = X(t) - Z(t) = S_{N(t)+1} - t$.

Since $R(t)$ only depends on the particular inter-renewal interval containing $t$,
it only depends on $X(t),Y(t),Z(t)$ and since $Y(t) = X(t) - Z(t)$,
it only depends on $X(t)$ and $Z(t)$.
Denote $\mathcal{R}$ the function such that $R(t) = \mathcal{R}(Z(t), X(t))$.

Let $R(n)$ be the reward accumulated at the $n$th interval.
As $X_n$ are iid, the $R_n$ are iid and for any $n$,
\[ \E[R_n] = \int_{x = 0}^\infty \int_{z=0}^x \mathbb{R}(z,x) \dif z f_X(x) \dif x. \]

We can show that if $\bar{X} < \infty$, then
\begin{equation}
  \label{eq:rew}
  \lim_{t \to \infty} \frac{1}{t} \int_0^t R(\tau) \dif \tau = \frac{\E[R_n]}{\bar{X}}
\end{equation}
with probability 1.
An easy way to show it, which is also useful to use directly is to see that
\[ \lim_{t \to \infty} \frac{1}{t} \int_0^t R(\tau) \dif \tau = \lim_{t \to \infty} \frac{N(t)}{t} \cdot \frac{1}{N(t)} \int_0^t R(\tau) \dif \tau = \frac{1}{\bar{X}} \cdot \E[R_n]. \]

Let $X_t$ be the asymptotic distribution of $X(t)$, i.e. $X_t = \lim_{t \to \infty} X(t)$.
We can see that $f_{X_t}(x) = \frac{xf_X(x)}{\bar{X}}$ so if $X$ has a arithmetic distribution with a step $d$,
\[ \lim_{n \to \infty} \E[R(nd)] = \frac{\E[R_n]}{\bar{X}}, \]
otherwise,
\[ \lim_{t \to \infty} \E[R(t)] = \frac{\E[R_n]}{\bar{X}}. \]

As examples, with $R(t) = X(t)$, we have $\mathbb{R}(z,x) = x$ so $\E[R_n] = \E[X^2]$ and
\[ \lim_{t \to \infty} \frac{1}{t} \int_0^t X(\tau) \dif \tau = \lim_{t \to \infty} \E[X(t)] = \frac{\E[X^2]}{\bar{X}} \geq \bar{X} \]
with probability 1.

With $R(t) = Z(t)$, we have $\mathbb{R}(z,x) = z$ so $\E[R_n] = \E[X^2]/2$ and
\[ \lim_{t \to \infty} \frac{1}{t} \int_0^t Z(\tau) \dif \tau = \lim_{t \to \infty} \E[Z(t)] = \frac{\E[X^2]}{2\bar{X}} \geq \frac{\bar{X}}{2} \]
with probability 1.
For $R(t) = Y(t)$, the result is the same than with $Z$ since $\E[Z_n] = \E[X_n]/2$ and $\E[Y_n] = \E[X_n] - \E[Y_n]$.

It is not surprising that we have the inequalities ${} \geq \bar{X}$ and ${} \geq \bar{X}/2$ since it is more likely that
when choosing a random $t$ we fall into an inter-renewal interval with large duration rather than one with a small duration.

\subsubsection{Little's Law}
Let us analyse a queuing system with
$A(t)$ denoting the number of arrivals between 0 and $t$ and
$D(t)$ denoting the number of departures between 0 and $t$.
The rv $L(t) = A(t) - D(t)$ is the number of custumers in the system at $t$
and the iid rv $W_i$ of mean $\bar{W}$ are the time spent in the system by the customer $i$.

We would like to compute the averate waiting time but $L$ depends on the all the customer of the queue,
not just the last one.
However, if we look at idle time, i.e. time when no one is in the system,
and the busy period between the idle times, we can see that $L(t)$ only depends on the busy period at $t$.
Therefore, if we consider the counting process for which $N(t)$ is the number of busy period already finished,
$L(t)$ is a reward function for this process.

We can apply \eqref{eq:rew} (actually we reuse the prove to make it clear),
\[ \bar{L} = \lim_{t \to \infty} \frac{1}{t} \int_0^t L(\tau) \dif \tau = \lim_{t \to \infty} \frac{A(t)}{t} \cdot \frac{\sum_{i=1}^{A(t)} W_i}{A(t)} = \frac{1}{\bar{X}} \cdot \bar{W}. \]
Here $\E[R_n] = \E[L_n] = \bar{W}$.

For a Poisson Process of arrival rate $\bar{X} = 1/\lambda$, we have $\bar{L} = \lambda \bar{W}$, so
\[ \bar{L} = \lim_{t \to \infty} \E[L(t)] = \lim_{t \to \infty} \frac{1}{t} \int_0^t L(\tau) \dif \tau = \lambda \bar{W}. \]

\paragraph{M/G/1 queue}
Note that in $W_i$ we have the waiting time in the queue and the service time.
Let us consider a M/G/1 queue, the reward function $L_q$ be the number of customers in the queue at time $t$ and the rv $Z_i$ of mean $\bar{Z}$ be the service time.
We also have the identity
\[ \bar{L}_q = \lim_{t \to \infty} \E[L_q(t)] = \lim_{t \to \infty} \frac{1}{t} \int_0^t L_q(\tau) \dif \tau = \lambda \bar{W}_q \]
Since $\bar{W} = \bar{W}_q + \bar{Z}$ and $\bar{L} = \bar{L}_q + \rho$
where $\rho$ is the expected asymptotic number of customers being served,
we have
\[ \rho = \lambda \bar{Z}. \]

\section{Markov Chains}
\cite[\S4--5]{gallager1995discrete,gallager2010discrete}, \cite[Appendix~A]{puterman2014markov}
A Markov Chain is a stochastic process with fixed time interval, $\{X_n\}_{n=0}^\infty$
where each random variable $X_n$ only depends on $X_{n-1}$ and not on $X_{n-2},X_{n-3},\ldots$.
We note
\[ P_{ij} = \Pr[X_n = j \mid X_{n-1} = i]. \]
$X_n$ is the state of the Markov chain.
The set of possible state is countable.
If $P_{ij}(n)$ does not depend on $n$ the chain is stationary or time homogeneous,
we only consider this case.

Let $P_{ij}^k = \Pr[X_n = j \mid X_{n-k} = i]$.
If we consider the matrix $[P^k]$ such that its entry at line $i$ column $j$ is $P_{ij}^k$ we can see that
$\P^k = \P^k$,
it is like the relation with the number of paths in Graph Theory.
It is called the Chapman-Kolmogorov Equation.
Note that if the number of state if infinite, the matrix $P$ has size $\infty \times \infty$.

\paragraph{Classes}
We say that state $j$ is \emph{accessible} from state $i$ (denoted $i \to j$) if $P_{ij}^k > 0$ for some $k \geq 0$.
State $j$ \emph{communicates} with state $i$ (denoted $i \leftrightarrow j$) if $i \to j$ and $j \to i$.
A set of states $S$ is
%\begin{itemize}
%  \item \emph{closed} if no state outside $S$ is accessible from any state in $S$,
%  \item \emph{irreducible} if no proper subset of $S$ is closed,
%  \item
    \emph{a class} if no state outside $S$ communicates with any state in $S$ and all pairs of states in $S$ communicate.
%\end{itemize}
%If Chain contains a single class with all the states, it is called irreducible.
%If there is only one closed irreducible set, the chain is called unichain,
%otherwise it is called multichain.

If we consider the induced graph of the chain,
i.e. the graph where the nodes are the state and there is an edge between $i$ and $j$ if $i \to j$,
the classes are the strongly connected components.

If there is only one class the chain is said to be \emph{irreducible},
note that this class can be transient, null recurrent or recurrent.

We will now see two class properties: the type and the period.
We define these two properties for a state and
we can say that the property of the class is simply the property of its states because of the following theorem.
\begin{mytheo}
  In a class, all states are of the same type and same period.
\end{mytheo}

\subsection{Type of a state}
Let $f_{ij}(n)$ be the probability that the \emph{first} visit to $j$ is at time $n$ when starting at $i$ at time 0.
Note that we have $f_{ij}(1) = P_{ij}$ and the recurrence relation
\[ f_{ij}(n) = \sum_{k \neq j} P_{ik}f_{kj}(n-1). \]
We also define $F_{ij}(n)$ as the probability that $j$ is visited at time $n$ \emph{or before} when starting at $i$ at time 0.
We have clearly
\[ F_{ij}(n) = \sum_{m=1}^n f_{ij}(m). \]
Let the random variable $T_{ij}$ be the time for the first visit to $j$ when starting at $i$.
Its mean is
\[ \bar{T}_{ij} = 1 + \sum_{n=1}^\infty 1 - F_{ij}(n). \]

Note that if $F_{jj}(\infty) < 1$, that means that there is a probability to never return at $j$ when starting at $j$.
In that case $\bar{T}_{jj} = \infty$, we say that $j$ is a \emph{transient} state.
If $T_{jj} = \infty$ we say that $j$ is recurrent. See Table~\ref{tab:state}.

\begin{table}[!ht]
  \centering
  \begin{tabular}{c|c|c}
    & $F_{jj}(\infty) < 1$ & $F_{jj}(\infty) = 1$\\
    \hline
    $\bar{T}_{jj} < \infty$ & \tikzmark{b} & positive recurrent\\
    \hline
    $\bar{T}_{jj} = \infty$ & transient & null recurrent
    \connect[0mm]{b.north west}{b.south east}
    \connect[0mm]{b.north east}{b.south west}
  \end{tabular}
  \caption{Types of states}
  \label{tab:state}
\end{table}

\subsection{Period of a state}
The period of a given state $j$ is the greatest common divider of
all the numbers $n$ such that $P_{jj}^n > 0$.
If the period is 1, we say that the state is aperiodic.

If the period of a class is $d$,
there exists a partition $S_1, \ldots, S_d$ of the states of the class
such that for $1 \leq n \leq d-1$, all the transitions from a state of $S_n$ are to a state $S_{n+1}$
and all the transition from a state of $S_d$ are to a state of $S_1$.

\subsection{Ergodic class}
A class recurrent and aperiodic is called \emph{ergodic}.
\begin{mytheo}
  If the states $i$ and $j$ are in the same class
  and their class is ergodic then
  \[ \lim_{n \to \infty} P_{ij}^n = \frac{1}{\bar{T}_{jj}}. \]
\end{mytheo}

\subsection{Perron vector}
The Perron vector $\pi$ represents the limiting probability distribution.
If it exists, independently on where we start, after an infinite number of transition,
we have a probability $\pi_j$ to be at the state $j$.
We represent $\pi$ as a line vector.
\begin{mytheo}
  For an irreducible Markov Chain,
  the system
  \begin{align*}
    \pi & = \pi \P\\
    \pi\mathbf{1} & = 1\\
    \pi & \geq 0.
  \end{align*}
  has a unique solution iff all the states (which have the same type since it is irreducible)
  are positive recurrent.
  If the solution exists, satifies $\pi_j = 1/\bar{T}_{jj}$ for all state $j$.
\end{mytheo}

If we take a random vector $x \geq 0$ such that $x \mathbf{1} = 0$ and compute
$\lim_{n \to \infty} x \P^n$, do we get $\pi$ ?
This actually only works if the chain is aperiodic.
For example, for
\[
  \P =
  \begin{pmatrix}
    0 & 1\\
    1 & 0
  \end{pmatrix},
\]
$\pi = \begin{pmatrix}1/2 & 1/2\end{pmatrix}$
but if $x \neq \pi$,
we will have $x\P^{2n} = x$ and $x\P^{2n+1} = x\P \neq x$.

We can see that $\P^{n} = \P \cdot \P^{n-1}$ so the lines are the lines
or $\P$ for which we apply $n-1$ times $\P$.
Therefore
\begin{mytheo}
  For an ergodic Markov Chain for which the unique class is positive recurrent,
  \[ \lim_{n \to \infty} \P^n = \mathbf{1} \pi. \]
\end{mytheo}

\subsection{Finite Markov Chain}
We say that a Markov Chain is finite if it has a finite number of states.
Note that the result we have derived involving the transition matrix $\P$ are more relevant for Finite Markov Chain.

For Finite Markov Chain,
\begin{itemize}
  \item null recurrent states do not exist,
    all recurrent state are positive recurrent and
  \item there is at least one class that is recurrent.
\end{itemize}

Therefore in the case of an irreducible Finite Markov Chain,
the unique class is positive recurrent.

If the Finite Markov Chain is not irreducible,
there may be transient states but there
also may be several recurrent classes.

If there is only one recurrent class,
the chain is called unichain.
If this recurrent class is ergodic, we say that the chain is ergodic recurrent.

We have seen the condition for the existence of the Perron vector for irreducible Markov Chain.
If the Markov Chain is finite and unichain,
the Perron vector also exists.
For every transient state $j$ we have $\pi_j = 0$ and the part of $\pi$ in the recurrent class
is the same if we remove the transient states.

\subsection{Markov Decision Processes (MDP)}
\cite[\S4.5]{gallager1995discrete,gallager2010discrete},
\cite{puterman2014markov}

\subsection{Markov Chain with Rewards}
Suppose we assign to each state a reward $r_i$.
If the reward is rather assigned to each transition from $i$ to $j$,
just take $r_i = \sum_{j} r_{ij} P_{ij}$, the expected reward at state $i$.

Clearly, the expected reward for an infinite time is just infinite (in most cases).
Let us look at the expected reward over $n$ transitions starting at $i$.
We define $v_i(n)$ as the expected reward of the $n$ states $i, \ldots$ visited plus
$v_j(0)$ where $j$ is the $(n+1)$th state.
The value of the vector $\mathbf{v}(0)$ doesn't really matter since we will look at the value of $v_i(n)$ for large $n$.

We know that the distribution will tend to be $\pi$ even if we start at $i$ for ergodic Markov Chains.
Let
\( g = \pi \r, \)
we can expect that for large $p$, $\v(n) = \w + ng\1 + \beta\1$, where $w_i$ is the relative advantage of starting at $i$.
Since it is relative we will typically set $w_1 = 0$ (otherwise, $\beta$ would be useless).
We cannot hope to get rid of $\beta$ since $\v(0)$ is arbitrary and its contribution is not related to the starting state $i$
for large $n$.

We can see that $\w$ should satify the relation
\[ \w + g\1 + \r + \P\w. \]
Since $\P$ is row stochastic, we can see that $\w$ is solution iff $\w + \alpha\1$ for all $\alpha \in \R$.
Therefore, it is wrong to thing that according to this equation $g$ depends on $\w$ and for different values of
$\w$ we have different values of $g$.
What we rather have is that for values of $g$ different than $\pi \r$ we have no solution for $\w$ and if
we have a solution for $\w$ we have an infinite number of solution.
If we add a condition such as $w_1 = 0$, we can hope to get a unique solution.

Actually, if the Markov Chain is irreducible then with the conditions such as $w_1 = 0$ or $\1^T\w = 0$,
the solution exists and is unique.
Furthermore, if it is irreducible, we have
\[ \lim_{n \to \infty} \v(n) - ng\1 = \w + \beta\1 \]
with $\beta = \pi(\v(0) + \w)$.

Again, we see that since $\pi\1 = 1$, the right hand side does not change if we replace $\w$ by $\w+\alpha\1$.
That means that this equation holds whether we add $w_1 = 0$, $w_2 = 0$ or $\1^T\w$ to find the value of $\w$.

\subsection{Markov Decision Processes}
Suppose that we want to find the decisions that maximize $g$ or $\v(n)$.
The decisions can affects the matrix $\P$ and the rewards vector $\r$.
At each state $i$ we can take any decision of the set of decisions $K_i$ and the decision taken at state $i$ does not affect the decision $K_j$ for $j \neq i$.
Denote the line $i$ of the transition matrix if we take the decision $k_i$ at $i$ as $P_i^{(k_i)}$
and the reward at $i$ as $r_i^{(k_i)}$.

\subsection{Dynamic Policy: Dynamic Programming Algorithm}
Suppose we want to maximize all the entries of $\v(n)$ at each $n$
and that we can change the policy at each step.
Normally it is not well defined to maximize different scalar at the same time but since we can
change the policy we can do it so it is well defined.

We can see that we can compute the optimal decision $k_i^*(n)$ if we know the optimal decisions $\k^*(n-1), \ldots, \k^*(1)$.
Actually, we just need the corresponding vector $\v^*(n-1)$.
We have
\[ k^*_i(n) \in \argmax_{k \in K_i} r_i^{(k)} + \P_i^{(k)} \v^*(n-1) \]
Once we have $k_i^*(n)$ we can get $v_i^*(n)$.
In short, we can compute $\v^*(n)$ if we have $\v^*(n-1)$.

\paragraph{Principle of Dynamic Programming}
Consider the graph that have the nodes $0, 1, \ldots$
and an edge from $n$ to $n-1$ for all $n = 1, 2, \ldots$.
At node $n$, the edges represents what we need to compute $\v^*(n)$.
When this directed graph is acyclic that means that there is an order of the node such that we can compute every node in that order with problem of requirement.
Think about the course and their requirement,
if there is no cyclic requirement there is an order of the course such that the student can follow all the course in that order.
This is the principle of dynamic programming.
Computing the nodes in this order is called the bottom up approach,
it is used when the order is easy to compute.
Otherwise we just use the top down approach that consists in recursively compute every node that is a requirement that is not yet computed remembering the result of previous computation.
It is very usefull to think of Dynamic Programming as a method to be used when the computation requirements form a Directed Acyclic Graph (DAG)~\cite[\S15.3]{cormen2009algorithm} and Greedy Programming as a method to be used when the feasible solutions have a matroid structure~\cite[\S16.4]{cormen2009algorithm}.

This results in the Algorithm~\ref{algo:dynalgo}.

\begin{algorithm}
  \caption{Dynamic Programming algorithm for a given $\v(0)$ using the bottom up approach.}
  \label{algo:dynalgo}
  \begin{algorithmic}
    \FOR{$n = 1, 2, \ldots$}
      \FOR{each state $i$}
        \STATE $v_i^*(n) \leftarrow \max_{k \in K_i} r_i^{(k)} + \P_i^{(k)} \v^*(n-1)$
      \ENDFOR
    \ENDFOR
  \end{algorithmic}
\end{algorithm}

\subsection{Stationnary policy: Policy iteration}
Let us now suppose that we want to maximize $g$ and that we must apply the same policy $k_i$ for all $n$.

Does it change something to impose that we cannot change the policy ?
It depends on $\v(0)$.
If $\v(0) = \w^{(\k^*)}$ where $w^{(\k^*)}$ is the relative advantage for $P^{(\k^*)}$ and $r^{(\k^*)}$,
then even if we allow to change the policy, the optimal policy will be the same for each $n$, i.e. $\k^*(n) = \k^*$.

\begin{mytheo}
  If the Markov Chain is irreducible then $\k^*$ is a stationnary optimal policy iff
  for all policy $\k$,
  \[ \r^{(\k^*)} + \P^{(\k^*)} \w^{(\k^*)} \geq \r^{k} + \P^{k} \w^{(\k^*)}. \]
\end{mytheo}

This optimality condition allows us to write Algorithm~\ref{algo:policyiteration}.
The number of iteration is finite since $g^{\k^n}$ increases at each iteration.
\begin{algorithm}
  \caption{Policy Iteration algorithm (Howard~\cite{howard1960dynamic}).}
  \label{algo:policyiteration}
  \begin{algorithmic}
    \STATE Choose an artitrary starting policy $\k^0$
    \STATE $n \leftarrow 1$
    \WHILE{$\k^n \neq \k^{n-1}$}
      \STATE Compute $\w^{(\k^{n-1})}$
      \FOR{each state $i$}
        \STATE $k_i^n \leftarrow \argmax_{k \in K_i} r_i^{(k)} + \P_i^{(k)} \w^{(\k^{n-1})}$
      \ENDFOR
      \STATE $n \leftarrow n + 1$
    \ENDWHILE
  \end{algorithmic}
\end{algorithm}

If the Markov Chain is irreducible for all policy
and ergodic for an optimal stationnary policy $\k^*$ then
\[ \lim_{n \to \infty} \v^*(n) - ng^{(\k^*)}\1 = \w^{(\k^*)} + (\beta'-\pi^{(\k^*)}\w^{(\k^*)})\1. \]
The relative gain $\w$ of all optimal stationnary policy is the same.
Also we have
\[ \min_i v_i^*(n) - v_i^*(n-1) \leq g^{(\k^*)} \leq \max_i v_i^*(n) - v_i^*(n-1) \]

That gives Algorithm~\ref{algo:valueiteration}.
At the end of the algorithm we have $g^{\k^*} = \min_i v_i^*(n) - v_i^*(n-1) = \max_i v_i^*(n) - v_i^*(n-1)$.
\begin{algorithm}
  \caption{Value Iteration algorithm for a given $\v(0)$.}
  \label{algo:valueiteration}
  \begin{algorithmic}
    \STATE Choose an arbitrary starting value $\v(0)$
    \STATE $n \leftarrow 1$
    \WHILE{$\min_i v_i^*(n) - v_i^*(n-1) < \max_i v_i^*(n) - v_i^*(n-1)$}
      \FOR{each state $i$}
        \STATE $v_i^*(n) \leftarrow \max_{k \in K_i} r_i^{(k)} + \P_i^{(k)} \v^*(n-1)$
      \ENDFOR
      \STATE $n \leftarrow n + 1$
    \ENDWHILE
  \end{algorithmic}
\end{algorithm}


\section{Markov Processes}
\cite[\S6]{gallager1995discrete,gallager2010discrete}

\section{Queueing Systems}

\nocite{*}
\biblio[alpha]

\end{document}
