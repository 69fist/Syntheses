\documentclass[en]{../../../eplsummary}

\newcommand{\E}{\mathbb{E}}
\DeclareMathOperator{\var}{\mathbb{V}ar}

\DeclareMathOperator{\expo}{Expo}
\DeclareMathOperator{\erlang}{Erlang}
\DeclareMathOperator{\po}{Po}

\hypertitle{Discrete stochastic models}{8}{3}{4}
{Benoît Legat}
{Philippe Chevalier}

\begin{center}
  \begin{tikzpicture}
    \node[left] at (-2,-1) {Poisson Processes};
    \node[left] at (-2, 1) {Renewal Processes};
    \draw[thick,->] (-2, 1) to (-1,.1);
    \draw[thick,->] (-2, -1) to (-1,-.1);
    \node[right] at (-1,0) {Markov Processes};
    \draw[thick,->] (.8, .8) to (.8,.2);
    \node[above] at (.8, .8) {Markov Chains};
    \draw[thick,->] (2.8, 0) to (3.4,0);
    \node[right] at (3.4, 0) {Queueing Systems};
    \draw[thick,->] (2.5, 1) to (3.4,1);
    \node[right] at (3.4, 1) {Markov Decision Processes (MDP)};
  \end{tikzpicture}
\end{center}

\section{Probability}
For more information for this chapter, read \cite[\S1.7]{gallager1995discrete} or \cite[\S1.5]{gallager2010discrete}.
%\[ \bar{F}_X(x) = 1 - F_X(x) \]

\begin{myineg}[Markov's inequality]
  Let $Y$ be a positive random variable.
  We have
  \[ \Pr[Y \geq y] \leq \frac{1}{y}\E[Y] \]
  \begin{proof}
    We can see that
    \begin{align*}
      E[Y]
      & = \int_0^\infty x f(x) \dif x\\
      & \geq \int_y^\infty x f(x) \dif x\\
      & \geq y \int_0^y f(x) \dif x\\
      & = y \Pr[Y \geq y].
    \end{align*}
  \end{proof}
\end{myineg}

If the variance is known, there is a useful corollary
\begin{myineg}[Chebyshev's inequality]
  Let $Y$ be a random variable.
  For an $\epsilon > 0$, we have
  \[ \Pr[|Y-\E[Y]| \geq \epsilon] \leq \frac{\sigma_Y^2}{\epsilon^2}. \]
  \begin{proof}
    Let $Z = (Y - \E[Y])^2$, we see that $Z$ is a positive random variable so we can use Markov's inequality with $z = \epsilon^2$:
    \begin{align*}
      \var[Y]
      & = E[Z]\\
      & \geq \epsilon^2 \Pr[Z \geq \epsilon^2]\\
      & = \epsilon^2 \Pr[(Y - \E[Y])^2 \geq \epsilon^2].
    \end{align*}
  \end{proof}
\end{myineg}

It is easy to prove the weak law of large numbers using Chebyshev's inequality.
\begin{mytheo}[Weak Law of Large Numbers]
  Let $X_1, \ldots, X_n$ we idependent and identically distributed random variables
  with finite mean $\bar{X}$ and finite variance $\sigma_X^2$.
  Denote their sum
  \[ S_n = X_1 + \cdots + X_n. \]
  For any $\epsilon > 0$,
  \[ \lim_{n \to \infty} \Pr\Big[ \Big|\frac{S_n}{n} - \bar{X}\Big| \geq \epsilon \Big] = 0. \]

  \begin{proof}
    Let's fix $n$ and consider the random variable $Y = S_n/n$.
    It is easy to see that $\E\{Y\} = \bar{X}$ and $\var\{Y\} = \sigma_X^2/n$~\cite[(1.41)]{gallager2010discrete}.
    Applying the Chebyshev's inequality we get
    \[ \Pr[|Y - \E[Y]| \geq \epsilon] \leq \frac{\sigma_X^2}{n\epsilon^2}. \]
    Since $\sigma_X^2 = 0$, we see directly that
    \[ \lim_{n \to \infty} \Pr[|Y - \E[Y]| \geq \epsilon] = 0. \]
  \end{proof}
\end{mytheo}

\begin{mytheo}[Central limit theorem (CLT)]
  Let $X_1, X_2, \ldots$ be iid random variables with
  finite mean $\bar{X}$ and finite variance $\sigma^2$.
  Then for every real number $z$,
  \[ \lim_{n \to \infty} \Pr\left\{ \frac{S_n-n\bar{X}}{\sigma\sqrt{n}} \sim \Phi(z) \right\} \]
  where $\Phi$ is the normal distribution function
  \[ \Phi(z) = \int_{-\infty}^z \frac{1}{\sqrt{2\pi}} \exp\Big(-\frac{y^2}{2}\Big) \dif y. \]
\end{mytheo}

\begin{center}
``The strong law requires considerable patience to understand, but it is a basic and essential result in understanding stochastic processes''\hfill\cite[p.~35]{gallager2010discrete}.
\end{center}

\begin{mytheo}[Strong Law of Large Numbers (SLLN)]
  For each integer $n \geq 1$.
  Let $X_1, X_2, \ldots$ be iid random variables with $\E[|X|] < \infty$.
  Then
  \[ \Pr\Big\{ \lim_{n \to \infty} \frac{S_n}{n} = \bar{X} \Big\} = 1. \]
\end{mytheo}

In short, the Weak Law of Large Numbers guarantees a convergence in probability and the Strong Law a convergence in distribution.

\section{Poisson Processes}
\cite[\S2]{gallager1995discrete,gallager2010discrete}

A Poisson Process is represented by an arrival rate $\lambda$.

The time between two consecutive arrivals are represented by the iid random variable $X_i$ (see \figuref{poisson}).
For a fixed $i$, $X_i \sim \expo(\lambda)$ is a continuous random variable for which
\begin{align*}
  f_X(t) & = \lambda \exp(-\lambda t), & x \geq 0,\\
  F_X(t) & = 1 - \exp(-\lambda t), & x \geq 0,\\
  \E[X]  & = \frac{1}{\lambda},\\
  \var[X]  & = \frac{1}{\lambda^2}.
\end{align*}
The key property of the exponential distribution is the lack of memory:
$\forall t,x > 0$,
\[ \Pr[X > t+x \mid X > t] = \Pr[X > x]. \]
It somewhat defies the intuition.
If there have been no arrival for a lot of time, it does not mean that it is more likely that there will be an arrival soon.

The exponential law shouldn't be misinterpreted, it really means that the chance of arrival is uniform in time.
However the uniform distribution doesn't fit here
since $X_1$ represents the time for one arrival so we kind of stop counting when someone has arrived.
Therefore of course $f_{X_1}(t_1) > f_{X_1}(t_2)$ if $t_1 < t_2$.
That doesn't mean that it is less likely that someone arrive at $t_2$ than $t_1$.
It just means that it is less likely that the \emph{first} arrival will be at $t_2$ than $t_1$.

Their sum, $S_n$ is the time from the start the the $n$th arrival.
For a fixed $n$, $S_n \expo \erlang_n(\lambda,n)$ is a continuous random variable for which
\begin{align*}
  f_{S_n}(t) & = \frac{\lambda^nt^{n-1}\exp(-\lambda t)}{(n-1)!}, & t \geq 0,\\
  F_{S_n}(t) & = \frac{\gamma(n,\lambda t)}{(n-1)!} = 1 - \sum_{k=0}^{n-1} \frac{1}{k!}\exp(-\lambda t) (\lambda t)^n, & t \geq 0\\
  \E[S_n] & = \frac{n}{\lambda},\\
  \var[S_n] & = \frac{n}{\lambda^2}.
\end{align*}

$N(t)$ is the number of arrival from the start to the time $t$.
For a fixed $t$, $N(t) \sim \po(\lambda t)$ is a discrete random variable for which
\begin{align*}
  \Pr[N(t) = n] & = \frac{(\lambda t)^n\exp(-\lambda t)}{n!}, & n \geq 0,\\
  F_{N(t)}(n)   & = \frac{\Gamma(n+1,\lambda t)}{n!} = \exp(-\lambda t) \sum_{i=0}^n \frac{(\lambda t)^i}{i!}, & n \geq 0\\
  \E[N(t)]      & = \lambda t,\\
  \var[N(t)]    & = \lambda t.
\end{align*}
The key property of $N$ is
\begin{description}
  \item[Property of stationary increment] The number of arrivals between $t$ and $t'$ is
    \[ N(t') - N(t) = N(t'-t) - N(t-t) = N(t'-t). \]
  \item[Property of independent increment] For any sequence $0 < t_1 < t_2 < t_2 < \cdots < t_k$,
    the random variables $N(t_1), N(t_2-t_1), \ldots, N(t_k-t_{k-2})$ are independent.
\end{description}
For small $t = \delta$, we can use the following equations to build approximations
\begin{align}
  \label{eq:inc1}
  \Pr[N(\delta) = 0]    & = 1 - \lambda \delta + o(\delta)\\
  \label{eq:inc2}
  \Pr[N(\delta) = 1]    & = \lambda \delta + o(\delta)\\
  \label{eq:inc3}
  \Pr[N(\delta) \geq 2] & = o(\delta)
\end{align}
where $o$ is such that $\lim_{x \downarrow 0} \frac{o(x)}{x} = 0$ and $o(0) = 0$.

One can check his understanding of $S_n$ and $N(t)$ by verifying that $\forall n,t$,
\( S_n \leq t \Leftrightarrow N(t) \geq n. \)

It is important to stress the fact that on the contrary to what the intuition would say,
if there have been many arrival or few arrival preceding a time $t$, it won't affect the time before the next arrival after $t$.

\begin{figure}[!ht]
  \centering
  \begin{tikzpicture}
    \draw[thick,->] (0,0) to (11.5,0);
    \draw (0,.1) to (0,-.1);
    \draw[<->] (0,-1) to (1,-1);
    \node[below] at (.5,-1) {$X_1$};
    \draw (1,.1) to (1,-.1);
    \node[below] at (1,-.1) {$t_1$};
    \draw[<->] (1,-1) to (3.5,-1);
    \node[below] at (2.25,-1) {$X_2$};
    \draw (3.5,.1) to (3.5,-.1);
    \node[below] at (3.5,-.1) {$t_2$};
    \draw[<->] (3.5,-1) to (7,-1);
    \node[below] at (5.25,-1) {$X_3$};
    \draw (7,.1) to (7,-.1);
    \node[below] at (7,-.1) {$t_3$};
    \draw[<->] (7,-1) to (8,-1);
    \node[below] at (7.5,-1) {$X_4$};
    \draw (8,.1) to (8,-.1);
    \node[below] at (8,-.1) {$t_4$};
    \draw[<->] (8,-1) to (11,-1);
    \node[below] at (9.5,-1) {$X_5$};
    \draw (11,.1) to (11,-.1);
    \node[below] at (11,-.1) {$t_5$};

    \draw[<->] (0,-2) to (3.5,-2);
    \node[below] at (1.75,-2) {$S_2$};
    \draw[<->] (0,-2.5) to (8,-2.5);
    \node[below] at (4,-2.5) {$S_4$};

    \draw[dashed] (5,.3) to (5,0);
    \node[above] at (5,.3) {$N(t) = 2$};

    \draw[dashed] (10,.3) to (10,0);
    \node[above] at (10,.3) {$N(t) = 4$};
  \end{tikzpicture}
  \caption{Example of Poisson Process}
  \label{fig:poisson}
\end{figure}

A crucial question that the reader is totally right to ask now is:
how do we prove that ``something'' is a Poisson Process ?
Excellent question ! Let me answer this using the following theorem.
\begin{mytheo} % TODO define renewable process
  Consider a process counting arrivals.
  \begin{itemize}
    \item The process is a Poisson Process with arrival rate $\lambda$.
    \item The process is a renewable process (i.e. iid arrivals) for which the interarrival intervals have an exponential distribution function with parameter $\lambda$.
    \item The counting process has independent and stationary increment properties and for all $y \geq 0$, \( N(t) \sim \po(\lambda t) \).
    \item The counting process has independent and stationary increment properties and for all \eqref{eq:inc1}, \eqref{eq:inc2} and \eqref{eq:inc3} hold for all $t,\delta \geq 0$
  \end{itemize}
\end{mytheo}

\paragraph{Combining Poisson Processes}
If we combine two \emph{independent} poisson processes with arrival rate $\lambda_1$ and $\lambda_2$,
the combination has arrival rate $\lambda_1 + \lambda_2$.

\paragraph{Subdividing a Poisson Process}
If an arrival for a Poisson Process is either redirected to
the process 1 with probability $p$ or to
the process 2 with probability $1-p$, then
the process 1 and 2 are \emph{independent} poisson processes with respective arrival rate
$\lambda_1 = p\lambda$ and $\lambda_2 = (1-p)\lambda$.

\subsection{Symmetry and order}
If it is known that $N(T) = n$,
$S_1, \ldots, S_n$ behave like uniform random variables of support $[0,T]$.
Precisely, we can see the sampling process for $S_1, \ldots, S_n$
as picking $n$ uniform random variables $U_1, \ldots, U_n$ in $[0,T]$ then sorting them
and giving to $S_i$ the value of the $i$th one.

Since we sort them, the $n$ values $s_1, \ldots, s_n$ have probability 0 to happen if they are not in nondecreasing order (of course),
but if they are in increasing order, there are $n!$ different choice of $u_1, \ldots, u_n$ that give those values of $s_1, \ldots, s_n$.
Therefore, we have
\[ f_{S_1, \ldots, S_n}(s_1, \ldots, s_n \mid N(T) = n) = \frac{n!}{T^n}. \]
The $X_i$ are still independent an for $1 \leq i \leq n$,
\begin{align*}
  f_{X_i}(t \mid N(T) = n) & = n \Big(\frac{T-t}{T}\Big)^{n-1}, & t & \geq 0,\\
  F_{X_i}(t \mid N(T) = n) & = 1 - \Big(\frac{T-t}{T}\Big)^n,   & t & \geq 0,\\
  \E[X_i \mid N(T) = n]    & = \frac{t}{n+1}.
\end{align*}
For $S_i$ we have
\begin{align*}
  f_{S_i}(t \mid N(T) = n) & = \frac{t^{i-1}(T-t)^{n-i}n!}{T^n(n-i)!(i-1)!} & t & \geq 0.
\end{align*}

\subsection{Non-homogeneous Poisson Processes}
For now we have assumed that $\lambda(t)$ was constant in time, i.e. $\lambda(t) = \lambda$ $\forall t \geq 0$.

If it is not constant, we lose the property of stationary increment.
We could keep it if we consider a non-linear time scale.
Let's consider small increments $\dif t$.
The property of stationary increment said that $N(t + \dif t) - N(t) = N(\dif t)$.
For Non-homogeneous Poisson Processes, we have instead
\[ N\Big(t + \frac{\dif t}{\lambda(t)}\Big) - N(t) = N\Big(\frac{\dif t}{\lambda_0}\Big). \]
That is, we partition the time axis in increments $\dif t/\lambda(t)$.
We have
\begin{align*}
  \Pr\Big[N\Big(t+\frac{\delta}{\lambda(t)}\Big) - N(t) =    0\Big] & = 1 - \delta + o(\delta)\\
  \Pr\Big[N\Big(t+\frac{\delta}{\lambda(t)}\Big) - N(t) =    1\Big] & = \delta + o(\delta)\\
  \Pr\Big[N\Big(t+\frac{\delta}{\lambda(t)}\Big) - N(t) \geq 2\Big] & = o(\delta)
\end{align*}

From this idea, one can show that~\cite[Theorem~2.5]{gallager2010discrete}
\begin{align*}
  \Pr[N(t_2)-N(t_1) = n] & = \frac{[m(t_1,t_2)]^n \exp[-m(t_1,t_2)]}{n!}, & t_1,t_2 & \geq 0,\\
  m(t_1, t_2)            & = \int_{t_1}^{t_2} \lambda(\tau) \dif \tau.
\end{align*}
Note that $\Pr[N(t) = n]$ is easily obtained by taking $t_1 = 0$ and $t_2 = t$.

\begin{myexem}
  Let's consider an M/G/$\infty$ Queueing System.
  We want to analyse the number of people in the being serviced at a given time $\tau$.
  Let $N_\tau(t)$ be the number of people arrived between $0$ and $t$ and still being serviced at $\tau$.
  We have the independent increment property.
  Let's see the stationarity of the increment:
  \[ N_\tau(t+\dif t) - N_\tau(t) = (1 - F_G(\tau - t)) N(\dif t). \]
  That means that $\lambda_\tau(t) = (1-F_G(\tau-t)) \lambda$.
  We have
  \begin{align}
    \notag
    m(0,\tau) & = \lambda \int_0^\tau 1 - F_G(\tau - t) \dif t\\
    \notag
              & = \lambda \tau - \lambda \int_0^\tau \int_0^{\tau-t} f_G(t') \dif t' \dif t\\
    \notag
              & = \lambda \tau - \lambda \int_0^\tau f_G(t') \int_0^{\tau-t'} \dif t \dif t'\\
    \notag
              & = \lambda \tau - \lambda \int_0^\tau f_G(t') (\tau-t') \dif t'\\
    \label{eq:mginf}
              & = \lambda \int_0^\tau f_G(t') t' \dif t'
               %& = \lambda \int_0^\infty f_G(t') \int_{t_1}^{\min(t_2,t')}  \dif t \dif t'\\
  \end{align}
  so
  \begin{align*}
    \lim_{\tau \to \infty} m(0,\tau) & = \lambda\E[G].
  \end{align*}
  Therefore
  \[ \lim_{\tau\to\infty} \Pr[N_{\tau}(\tau) = n] = \frac{[\lambda\E[G]]^n \exp[-\lambda\E[g]]}{n!} \]
  hence the number of people being serviced tends to (or is in steady state) a Poisson distribution $\po(\lambda\E[G])$.
  In other word, $N_\tau$ tends to be an homogeneous Poisson Process with arrival rate $\lambda$ that has started at time $\tau - \E[G]$.

  It is interesting to compare \figuref{mginf} with \eqref{eq:mginf}.
  For $\tau < \inf$, we $m(0,\tau) < \lambda \E[G]$ because the
  part of the integral form $\tau$ to $\infty$ is missing.

  \begin{figure}[!ht]
    \begin{subfigure}{0.5\textwidth}
      \begin{tikzpicture}[scale=1.3]
        \draw[thick, domain=0:4] plot
        (\x, {1/2});
        \draw[thick, domain=0:4] plot
        (\x, {exp((\x)-4)});
        \draw[dashed,->] (0,0) to (4.5,0);
        \draw[dashed,->] (0,0) to (0,1.5);
        \draw (-0.1,1/2) to (0.1,1/2);
        \node[left] at (-0.1,1/2) {$\lambda$};
        \draw (-0.1,1) to (0.1,1);
        \node[left] at (-0.1,1) {$1$};
        \draw (4,.1) to (4,-.1);
        \node[below] at (4,-.1) {$\tau$};
        \node[above] at (4,1) {$1 - F_G(\tau-t)$};
      \end{tikzpicture}
      \caption{Plot of $\lambda$ and the probability to be still serviced at $\tau$.
      $\lambda(t)$ is the product of the two curves.}
      \label{fig:mginf}
    \end{subfigure}
    \begin{subfigure}{0.5\textwidth}
      \begin{tikzpicture}[scale=1.3]
        \draw[thick, domain=0:4] plot
        (\x, {1/2});
        \draw[thick, domain=0:3] plot
        (\x, {0});
        \draw[thick] (3,0) to (3,1);
        \draw[thick, domain=3:4] plot
        (\x, {1});
        \draw[dashed,->] (0,0) to (4.5,0);
        \draw[dashed,->] (0,0) to (0,1.5);
        \draw (-0.1,1/2) to (0.1,1/2);
        \node[left] at (-0.1,1/2) {$\lambda$};
        \draw (-0.1,1) to (0.1,1);
        \node[left] at (-0.1,1) {$1$};
        \draw (4,.1) to (4,-.1);
        \node[below] at (4,-.1) {$\tau$};
        \draw (3,.1) to (3,-.1);
        \node[below] at (3,-.1) {$\tau-\E[G]$};
      \end{tikzpicture}
      \caption{$N_\tau(\tau)$ tends to a homogeneous Poisson Process
      of arrival rate $\lambda$ having started at $\tau-\E[G]$.}
    \end{subfigure}
  \end{figure}
\end{myexem}

\section{Renewal Processes}
\cite[\S3]{gallager1995discrete,gallager2010discrete}

\section{Markov Chains}
\cite[\S4--5]{gallager1995discrete,gallager2010discrete}

\section{Markov Decision Processes (MDP)}
\cite{puterman2014markov}

\section{Markov Processes}
\cite[\S6]{gallager1995discrete,gallager2010discrete}

\section{Queueing Systems}

\nocite{*}
\biblio[alpha]

\end{document}
