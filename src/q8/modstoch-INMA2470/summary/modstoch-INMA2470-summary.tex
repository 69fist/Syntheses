\documentclass[en]{../../../eplsummary}

\newcommand{\E}{\mathbb{E}}
\DeclareMathOperator{\var}{\mathbb{V}ar}

\DeclareMathOperator{\expo}{Expo}
\DeclareMathOperator{\erlang}{Erlang}
\DeclareMathOperator{\po}{Po}
\DeclareMathOperator{\diag}{diag}

\renewcommand{\P}{\mathbf{P}}
\renewcommand{\r}{\mathbf{r}}
\renewcommand{\v}{\mathbf{v}}
\newcommand{\w}{\mathbf{w}}
\renewcommand{\k}{\mathbf{k}}
\newcommand{\1}{\mathbf{1}}
\newcommand{\Q}{\mathbf{Q}}
\newcommand{\p}{\mathbf{p}}

\usepackage{bm}
\newcommand{\bnu}{\bm{\nu}}
\newcommand{\bpi}{\bm{\pi}}

\usepackage{algorithm}
\usepackage{algorithmic}

% http://tex.stackexchange.com/questions/86883/cancelling-out-cells-in-tables
\usepackage{tikz}
\usepackage{tkz-graph}
\newcommand{\tikzmark}[1]{\tikz[remember picture,overlay, baseline=-0.5ex]\node (#1){};}
\newcommand{\connect}[3][3mm]{\tikz[remember picture,overlay]\draw[shorten <=-#1, shorten >=-#1](#2)--(#3);}

\hypertitle{Discrete stochastic models}{8}{INMA}{2470}
{Benoît Legat}
{Philippe Chevalier}

\begin{center}
  \begin{tikzpicture}
    \node[left] at (-2,-1) {Poisson Processes};
    \node[left] at (-2, 1) {Renewal Processes};
    \draw[thick,->] (-2, 1) to (-1,.1);
    \draw[thick,->] (-2, -1) to (-1,-.1);
    \node[right] at (-1,0) {Markov Processes};
    \draw[thick,->] (.8, .8) to (.8,.2);
    \node[above] at (.8, .8) {Markov Chains};
    \draw[thick,->] (2.8, 0) to (3.4,0);
    \node[right] at (3.4, 0) {Queueing Systems};
    \draw[thick,->] (2.5, 1) to (3.4,1);
    \node[right] at (3.4, 1) {Markov Decision Processes (MDP)};
  \end{tikzpicture}
\end{center}

We can see that the Semi-Markov Processes are the generalization of all the process we will see.

\begin{center}
  \begin{tabular}{l|l|l}
    & No state & $X_n$ only dependent on $X_{n-1}$\\
    \hline
    No transition time & & Markov Chain\\
    \hline
    $U_n$ general & Renewal Process & Semi-Markov Process\\
    \hline
    $U_n \sim \expo(\lambda)$ & Poisson Process & Markov Process
  \end{tabular}
\end{center}

\section{Probability}
For more information for this chapter, read \cite[\S1.7]{gallager1995discrete} or \cite[\S1.5]{gallager2010discrete}.
%\[ \bar{F}_X(x) = 1 - F_X(x) \]

\begin{myineg}[Markov's inequality]
  Let $Y$ be a positive random variable.
  We have
  \[ \Pr[Y \geq y] \leq \frac{1}{y}\E[Y] \]
  \begin{proof}
    We can see that
    \begin{align*}
      E[Y]
      & = \int_0^\infty x f(x) \dif x\\
      & \geq \int_y^\infty x f(x) \dif x\\
      & \geq y \int_y^\infty f(x) \dif x\\
      & = y \Pr[Y \geq y].
    \end{align*}
  \end{proof}
\end{myineg}

If the variance is known, there is a useful corollary
\begin{myineg}[Chebyshev's inequality]
  Let $Y$ be a random variable.
  For an $\epsilon > 0$, we have
  \[ \Pr[|Y-\E[Y]| \geq \epsilon] \leq \frac{\sigma_Y^2}{\epsilon^2}. \]
  \begin{proof}
    Let $Z = (Y - \E[Y])^2$, we see that $Z$ is a positive random variable so we can use Markov's inequality with $z = \epsilon^2$:
    \begin{align*}
      \var[Y]
      & = E[Z]\\
      & \geq \epsilon^2 \Pr[Z \geq \epsilon^2]\\
      & = \epsilon^2 \Pr[(Y - \E[Y])^2 \geq \epsilon^2].
    \end{align*}
  \end{proof}
\end{myineg}

It is easy to prove the weak law of large numbers using Chebyshev's inequality.
\begin{mytheo}[Weak Law of Large Numbers]
  Let $X_1, \ldots, X_n$ we idependent and identically distributed random variables
  with finite mean $\bar{X}$ and finite variance $\sigma_X^2$.
  Denote their sum
  \[ S_n = X_1 + \cdots + X_n. \]
  For any $\epsilon > 0$,
  \[ \lim_{n \to \infty} \Pr\Big[ \Big|\frac{S_n}{n} - \bar{X}\Big| \geq \epsilon \Big] = 0. \]

  \begin{proof}
    Let's fix $n$ and consider the random variable $Y = S_n/n$.
    It is easy to see that $\E\{Y\} = \bar{X}$ and $\var\{Y\} = \sigma_X^2/n$~\cite[(1.41)]{gallager2010discrete}.
    Applying the Chebyshev's inequality we get
    \[ \Pr[|Y - \E[Y]| \geq \epsilon] \leq \frac{\sigma_X^2}{n\epsilon^2}. \]
    Since $\sigma_X^2 = 0$, we see directly that
    \[ \lim_{n \to \infty} \Pr[|Y - \E[Y]| \geq \epsilon] = 0. \]
  \end{proof}
\end{mytheo}

\begin{mytheo}[Central limit theorem (CLT)]
  Let $X_1, X_2, \ldots$ be iid random variables with
  finite mean $\bar{X}$ and finite variance $\sigma^2$.
  Then for every real number $z$,
  \[ \lim_{n \to \infty} \Pr\left\{ \frac{S_n-n\bar{X}}{\sigma\sqrt{n}} \sim \Phi(z) \right\} \]
  where $\Phi$ is the normal distribution function
  \[ \Phi(z) = \int_{-\infty}^z \frac{1}{\sqrt{2\pi}} \exp\Big(-\frac{y^2}{2}\Big) \dif y. \]
\end{mytheo}

\begin{center}
``The strong law requires considerable patience to understand, but it is a basic and essential result in understanding stochastic processes''\hfill\cite[p.~35]{gallager2010discrete}.
\end{center}

\begin{mytheo}[Strong Law of Large Numbers (SLLN)]
  For each integer $n \geq 1$.
  Let $X_1, X_2, \ldots$ be iid random variables with $\E[|X|] < \infty$.
  Then
  \[ \Pr\Big\{ \lim_{n \to \infty} \frac{S_n}{n} = \bar{X} \Big\} = 1. \]
\end{mytheo}

In short, the Weak Law of Large Numbers guarantees a convergence in probability and the Strong Law a convergence in distribution.

\section{Poisson Processes}
\cite[\S2]{gallager1995discrete,gallager2010discrete}

A Poisson Process is characterized by an arrival rate $\lambda$.

The time between the $(i-1)$th arrival and the $i$th arrival is represented by the iid random variable $U_i$ (see \figuref{poisson}).
For a fixed $i$, $U_i \sim \expo(\lambda)$ is a continuous random variable for which
\begin{align*}
  f_U(t) & = \lambda \exp(-\lambda t), & t \geq 0,\\
  F_U(t) & = 1 - \exp(-\lambda t), & t \geq 0,\\
  \E[U]  & = \frac{1}{\lambda},\\
  \var[U]  & = \frac{1}{\lambda^2}.
\end{align*}
The key property of the exponential distribution is the lack of memory:
$\forall t,u > 0$,
\[ \Pr[U > t+u \mid U > t] = \Pr[U > u]. \]
It somewhat defies the intuition.
If there have been no arrival for a lot of time, it does not mean that it is more likely that there will be an arrival soon.

The exponential law shouldn't be misinterpreted, it really means that the chance of arrival is uniform in time.
However the uniform distribution doesn't fit here
since $U_1$ represents the time for one arrival so we kind of stop counting when someone has arrived.
Therefore of course $f_{U_1}(t_1) > f_{U_1}(t_2)$ if $t_1 < t_2$.
That doesn't mean that it is less likely that someone arrive at $t_2$ than $t_1$.
It just means that it is less likely that the \emph{first} arrival will be at $t_2$ than $t_1$.

Their sum, $S_n$ is the time from the start the the $n$th arrival.
For a fixed $n$, $S_n \sim \erlang_n(\lambda,n)$ is a continuous random variable for which
\begin{align*}
  f_{S_n}(t) & = \frac{\lambda^nt^{n-1}\exp(-\lambda t)}{(n-1)!}, & t \geq 0,\\
  F_{S_n}(t) & = \frac{\gamma(n,\lambda t)}{(n-1)!} = 1 - \sum_{k=0}^{n-1} \frac{1}{k!}\exp(-\lambda t) (\lambda t)^n, & t \geq 0\\
  \E[S_n] & = \frac{n}{\lambda},\\
  \var[S_n] & = \frac{n}{\lambda^2}.
\end{align*}

$N(t)$ is the number of arrival from the start to the time $t$.
For a fixed $t$, $N(t) \sim \po(\lambda t)$ is a discrete random variable for which
\begin{align*}
  \Pr[N(t) = n] & = \frac{(\lambda t)^n\exp(-\lambda t)}{n!}, & n \geq 0,\\
  F_{N(t)}(n)   & = \frac{\Gamma(n+1,\lambda t)}{n!} = \exp(-\lambda t) \sum_{i=0}^n \frac{(\lambda t)^i}{i!}, & n \geq 0\\
  \E[N(t)]      & = \lambda t,\\
  \var[N(t)]    & = \lambda t.
\end{align*}
The key property of $N$ is
\begin{description}
  \item[Property of stationary increment] The number of arrivals between $t$ and $t'$ is
    \[ N(t') - N(t) = N(t'-t) - N(t-t) = N(t'-t). \]
  \item[Property of independent increment] For any sequence $0 < t_1 < t_2 < t_2 < \cdots < t_k$,
    the random variables $N(t_1), N(t_2-t_1), \ldots, N(t_k-t_{k-2})$ are independent.
\end{description}
For small $t = \delta$, we can use the following equations to build approximations
\begin{align}
  \label{eq:inc1}
  \Pr[N(\delta) = 0]    & = 1 - \lambda \delta + o(\delta)\\
  \label{eq:inc2}
  \Pr[N(\delta) = 1]    & = \lambda \delta + o(\delta)\\
  \label{eq:inc3}
  \Pr[N(\delta) \geq 2] & = o(\delta)
\end{align}
where $o$ is such that $\lim_{x \downarrow 0} \frac{o(x)}{x} = 0$ and $o(0) = 0$.

It is important to stress the fact that on the contrary to what the intuition would say,
if there have been many arrival or few arrival preceding a time $t$, it won't affect the time before the next arrival after $t$.

\begin{figure}[!ht]
  \centering
  \begin{tikzpicture}
    \draw[thick,->] (0,0) to (11.5,0);
    \draw (0,.1) to (0,-.1);
    \draw[<->] (0,-1) to (1,-1);
    \node[below] at (.5,-1) {$U_1$};
    \draw (1,.1) to (1,-.1);
    \node[below] at (1,-.1) {$t_1$};
    \draw[<->] (1,-1) to (3.5,-1);
    \node[below] at (2.25,-1) {$U_2$};
    \draw (3.5,.1) to (3.5,-.1);
    \node[below] at (3.5,-.1) {$t_2$};
    \draw[<->] (3.5,-1) to (7,-1);
    \node[below] at (5.25,-1) {$U_3$};
    \draw (7,.1) to (7,-.1);
    \node[below] at (7,-.1) {$t_3$};
    \draw[<->] (7,-1) to (8,-1);
    \node[below] at (7.5,-1) {$U_4$};
    \draw (8,.1) to (8,-.1);
    \node[below] at (8,-.1) {$t_4$};
    \draw[<->] (8,-1) to (11,-1);
    \node[below] at (9.5,-1) {$U_5$};
    \draw (11,.1) to (11,-.1);
    \node[below] at (11,-.1) {$t_5$};

    \draw[<->] (0,-2) to (3.5,-2);
    \node[below] at (1.75,-2) {$S_2$};
    \draw[<->] (0,-2.5) to (8,-2.5);
    \node[below] at (4,-2.5) {$S_4$};

    \draw[dashed] (5,.3) to (5,0);
    \node[above] at (5,.3) {$N(t) = 2$};

    \draw[dashed] (10,.3) to (10,0);
    \node[above] at (10,.3) {$N(t) = 4$};
  \end{tikzpicture}
  \caption{Example of Poisson Process}
  \label{fig:poisson}
\end{figure}

A crucial question that the reader may ask now is:
how do we prove that ``something'' is a Poisson Process ?
Excellent question ! Let me answer this using the following theorem.
\begin{mytheo} % TODO define renewable process
  Consider a process counting arrivals.
  \begin{itemize}
    \item The process is a Poisson Process with arrival rate $\lambda$.
    \item The process is a renewable process (i.e. iid arrivals) for which the interarrival intervals have an exponential distribution function with parameter $\lambda$.
    \item The counting process has independent and stationary increment properties and for all $y \geq 0$, \( N(t) \sim \po(\lambda t) \).
    \item The counting process has independent and stationary increment properties and for all \eqref{eq:inc1}, \eqref{eq:inc2} and \eqref{eq:inc3} hold for all $t,\delta \geq 0$
  \end{itemize}
\end{mytheo}

\paragraph{Combining Poisson Processes}
If we combine two \emph{independent} poisson processes with arrival rate $\lambda_1$ and $\lambda_2$,
the combination has arrival rate $\lambda_1 + \lambda_2$.

\paragraph{Subdividing a Poisson Process}
If an arrival for a Poisson Process is either redirected to
the process 1 with probability $p$ or to
the process 2 with probability $1-p$, then
the processes 1 and 2 are \emph{independent} poisson processes with respective arrival rate
$\lambda_1 = p\lambda$ and $\lambda_2 = (1-p)\lambda$.

\subsection{Symmetry and order}
If it is known that $N(T) = n$,
$S_1, \ldots, S_n$ behave like uniform random variables of support $[0,T]$.
Precisely, we can see the sampling process for $S_1, \ldots, S_n$
as picking $n$ uniform random variables $U_1, \ldots, U_n$ in $[0,T]$ then sorting them
and giving to $S_i$ the value of the $i$th one.

Since we sort them, the $n$ values $s_1, \ldots, s_n$ have probability 0 to happen if they are not in nondecreasing order (of course),
but if they are in increasing order, there are $n!$ different choice of $u_1, \ldots, u_n$ that give those values of $s_1, \ldots, s_n$.
Therefore, we have
\[ f_{S_1, \ldots, S_n}(s_1, \ldots, s_n \mid N(T) = n) = \frac{n!}{T^n}. \]
The $U_i$ are still independent an for $1 \leq i \leq n$,
\begin{align*}
  f_{U_i}(t \mid N(T) = n) & = n \Big(\frac{T-t}{T}\Big)^{n-1}, & t & \geq 0,\\
  F_{U_i}(t \mid N(T) = n) & = 1 - \Big(\frac{T-t}{T}\Big)^n,   & t & \geq 0,\\
  \E[U_i \mid N(T) = n]    & = \frac{t}{n+1}.
\end{align*}
For $S_i$ we have
\begin{align*}
  f_{S_i}(t \mid N(T) = n) & = \frac{t^{i-1}(T-t)^{n-i}n!}{T^n(n-i)!(i-1)!} & t & \geq 0.
\end{align*}

\subsection{Non-homogeneous Poisson Processes}
For now we have assumed that $\lambda(t)$ was constant in time, i.e. $\lambda(t) = \lambda$ $\forall t \geq 0$.

If it is not constant, we lose the property of stationary increment.
We could keep it if we consider a non-linear time scale.
Let's consider small increments $\dif t$.
The property of stationary increment said that $N(t + \dif t) - N(t) = N(\dif t)$.
For Non-homogeneous Poisson Processes, we have instead
\[ N\Big(t + \frac{\dif t}{\lambda(t)}\Big) - N(t) = N\Big(\frac{\dif t}{\lambda_0}\Big). \]
That is, we partition the time axis in increments $\dif t/\lambda(t)$.
We have
\begin{align*}
  \Pr\Big[N\Big(t+\frac{\delta}{\lambda(t)}\Big) - N(t) =    0\Big] & = 1 - \delta + o(\delta)\\
  \Pr\Big[N\Big(t+\frac{\delta}{\lambda(t)}\Big) - N(t) =    1\Big] & = \delta + o(\delta)\\
  \Pr\Big[N\Big(t+\frac{\delta}{\lambda(t)}\Big) - N(t) \geq 2\Big] & = o(\delta)
\end{align*}

From this idea, one can show that~\cite[Theorem~2.5]{gallager2010discrete}
\begin{align*}
  \Pr[N(t_2)-N(t_1) = n] & = \frac{[m(t_1,t_2)]^n \exp[-m(t_1,t_2)]}{n!}, & t_1,t_2 & \geq 0,\\
  m(t_1, t_2)            & = \int_{t_1}^{t_2} \lambda(\tau) \dif \tau.
\end{align*}
Note that $\Pr[N(t) = n]$ is easily obtained by taking $t_1 = 0$ and $t_2 = t$.

\begin{myexem}
  Let's consider an M/G/$\infty$ Queueing System.
  Let the rv $Z$ represent the service distribution.
  We want to analyse the number of people in the being serviced at a given time $\tau$.
  Let $N_\tau(t)$ be the number of people arrived between $0$ and $t$ and still being serviced at $\tau$.
  We have the independent increment property.
  Let's see the stationarity of the increment:
  \[ N_\tau(t+\dif t) - N_\tau(t) = (1 - F_Z(\tau - t)) N(\dif t). \]
  That means that $\lambda_\tau(t) = (1-F_Z(\tau-t)) \lambda$.
  We have
  \begin{align}
    \notag
    m(0,\tau) & = \lambda \int_0^\tau 1 - F_Z(\tau - t) \dif t\\
    \notag
              & = \lambda \tau - \lambda \int_0^\tau \int_0^{\tau-t} f_Z(t') \dif t' \dif t\\
    \notag
              & = \lambda \tau - \lambda \int_0^\tau f_Z(t') \int_0^{\tau-t'} \dif t \dif t'\\
    \notag
              & = \lambda \tau - \lambda \int_0^\tau f_Z(t') (\tau-t') \dif t'\\
    \label{eq:mginf}
              & = \lambda \int_0^\tau f_Z(t') t' \dif t'
               %& = \lambda \int_0^\infty f_Z(t') \int_{t_1}^{\min(t_2,t')}  \dif t \dif t'\\
  \end{align}
  so
  \begin{align*}
    \lim_{\tau \to \infty} m(0,\tau) & = \lambda\E[Z].
  \end{align*}
  Therefore
  \[ \lim_{\tau\to\infty} \Pr[N_{\tau}(\tau) = n] = \frac{[\lambda\E[Z]]^n \exp[-\lambda\E[Z]]}{n!} \]
  hence the number of people being serviced tends to (or is in steady state) a Poisson distribution $\po(\lambda\E[Z])$.
  In other word, $N_\tau$ tends to be an homogeneous Poisson Process with arrival rate $\lambda$ that has started at time $\tau - \E[Z]$.

  It is interesting to compare \figuref{mginf} with \eqref{eq:mginf}.
  For $\tau < \inf$, we $m(0,\tau) < \lambda \E[Z]$ because the
  part of the integral form $\tau$ to $\infty$ is missing.

  \begin{figure}[!ht]
    \begin{subfigure}{0.5\textwidth}
      \begin{tikzpicture}[scale=1.3]
        \draw[thick, domain=0:4] plot
        (\x, {1/2});
        \draw[thick, domain=0:4] plot
        (\x, {exp((\x)-4)});
        \draw[dashed,->] (0,0) to (4.5,0);
        \draw[dashed,->] (0,0) to (0,1.5);
        \draw (-0.1,1/2) to (0.1,1/2);
        \node[left] at (-0.1,1/2) {$\lambda$};
        \draw (-0.1,1) to (0.1,1);
        \node[left] at (-0.1,1) {$1$};
        \draw (4,.1) to (4,-.1);
        \node[below] at (4,-.1) {$\tau$};
        \node[above] at (4,1) {$1 - F_Z(\tau-t)$};
      \end{tikzpicture}
      \caption{Plot of $\lambda$ and the probability to be still serviced at $\tau$.
      $\lambda(t)$ is the product of the two curves.}
      \label{fig:mginf}
    \end{subfigure}
    \begin{subfigure}{0.5\textwidth}
      \begin{tikzpicture}[scale=1.3]
        \draw[thick, domain=0:4] plot
        (\x, {1/2});
        \draw[thick, domain=0:3] plot
        (\x, {0});
        \draw[thick] (3,0) to (3,1);
        \draw[thick, domain=3:4] plot
        (\x, {1});
        \draw[dashed,->] (0,0) to (4.5,0);
        \draw[dashed,->] (0,0) to (0,1.5);
        \draw (-0.1,1/2) to (0.1,1/2);
        \node[left] at (-0.1,1/2) {$\lambda$};
        \draw (-0.1,1) to (0.1,1);
        \node[left] at (-0.1,1) {$1$};
        \draw (4,.1) to (4,-.1);
        \node[below] at (4,-.1) {$\tau$};
        \draw (3,.1) to (3,-.1);
        \node[below] at (3,-.1) {$\tau-\E[Z]$};
      \end{tikzpicture}
      \caption{$N_\tau(\tau)$ tends to a homogeneous Poisson Process
      of arrival rate $\lambda$ having started at $\tau-\E[Z]$.}
    \end{subfigure}
  \end{figure}
\end{myexem}

% TODO arithmetic, Bernoulli

\section{Renewal Processes}
\cite[\S3]{gallager1995discrete,gallager2010discrete}
A Renewal Process is a counting process for which the times between 2 arrivals are iid random variables.

We define $U_i, S_n, N(t)$ the same way as for Poisson Process except that here $U_i$ is exponentially distributed.
After an arrival, it is like $t = 0$ since the $U_i$ are iid: i.e.
for $t \geq 0$, $N(S_n+t) - N(S_n)$ has the same distribution than $N(t)$.
However, it is not the same of the property of stationary increment of Poisson process where we can take any $t,t'$.
Here we look at the increment starting at $S_n$ because we cannot start at the middle of an arrival because since the $U_i$
are not exponential they are not memoryless.

The mean $\bar{U} = \E[U_i]$ of $U_i$ is supposed to be finite but the variance $\sigma$ can be infinite.

We have a useful property linking $S_n$ and $N(t)$: $\forall n,t$,
\begin{align}
  \label{eq:snnt}
  S_n \leq t & \Leftrightarrow N(t) \geq n & S_n \geq t & \Leftrightarrow N(t) < n.
\end{align}
Note that since Poisson Processes is a particular case of Renewal Processes, it is also true for Poisson Processes.

We can show that
\[ \lim_{t \to \infty} N(t) = \infty \]
with probability 1 and
\[ \lim_{t \to \infty} \E[N(t)] = \infty. \]
Note that for the second identity,
since $\E[N(t)]$ is deterministic we do not need to specify the type of convergence (e.g. with probability 1, ...).

Using this identity and the Strong Law of Large Numbers (SLLN), we can show that
\begin{equation}
  \label{eq:Nt}
  \lim_{t \to \infty} \frac{N(t)}{t} = \frac{1}{\bar{U}}
\end{equation}
with probability 1.

\begin{mydef}[Decision rule]
  A \emph{decision rule} is a sequence $\{I_n\}_{n=1}^\infty$ that is the indicator function of a set $\{\, 1 \leq n \leq N \,\}$ where $N$ is a random function.
\end{mydef}

\begin{mydef}[Stopping time]
  A \emph{stopping time} for a sequence of rv $\{U_i\}_{i=1}^\infty$ is a positive integer rw $N$ such that
  $\forall n$, $I_n$ is independent of $U_i$ for $i \geq n$.
\end{mydef}

Note that for renewal processes, $N(t)$ is not a stopping time.
Indeed, using \eqref{eq:snnt},
we can see that $I_n = [N(t) \geq n] = [S_n \leq t]$ where $\P$ is 1 if $P$ is true and 0 otherwise.

$N(t)+1$ however, is a stopping time,
since that means that $I_n = [N(t)+1 \geq n] = [N(t) \geq n-1] = [S_{n-1} \leq t]$
and $S_{n-1}$ is independent of $U_i$ for $i \geq n$.

\begin{mytheo}[Wald's equality]
  If $N$ is a stopping time of a sequence of rv $\{U_i\}_{i=1}^\infty$ then
  \[ \E[S_N] = \bar{U}\E[N]. \]
\end{mytheo}

Note that now that the distribution of $U_i$ is not necessarily an exponential,
it may be discrete.
The distribution can even be arithmetic,
i.e. $\exists d > 0$ such that $U_i$ can only be a multiple of $d$: $\Pr[U_i/d \in \mathbb{N}] = 1$.
We call $d$ the distribution step.

\begin{mytheo}[Blackwell]
  If the distribution of $U_i$ is arithmetic with a step $d$, then $\forall n \geq 1$,
  \[ \lim_{t \to \infty} [\E[N(t + nd)] - \E[N(t)]] = \frac{nd}{\E[U]}, \]
  otherwise, $\forall \delta \geq 0$,
  \[ \lim_{t \to \infty} [\E[N(t + \delta)] - \E[N(t)]] = \frac{\delta}{\E[U]}. \]
\end{mytheo}
We could be tempted to say that the derivative of $\E[N(t)]$ with $t$ is $1/\E[U]$ but for that we would need to prove that this is still true for $\delta \to 0$.

With a non-arithmetic distribution with $P[U_i = 0] = 0$, we have
\begin{align*}
  \lim_{t \to \infty} \Pr[N(t+\delta) - N(t) =    0] & = 1 - \delta/\bar{U} + o(\delta)\\
  \lim_{t \to \infty} \Pr[N(t+\delta) - N(t) =    1] & = \delta/\bar{U} + o(\delta)\\
  \lim_{t \to \infty} \Pr[N(t+\delta) - N(t) \geq 2] & = o(\delta)
\end{align*}
so asymptotically, we have stationnary increment.

Do we have independent increment asymptotically ?
No.
However if we merge many independent renewal processes we will get closer and closer
to a poisson process with independent increment.

\[ \lim_{t \to \infty} \frac{1}{t} \int_0^t Y(\tau) \dif \tau = \frac{1}{2}\bar{U} + \frac{1}{2}\frac{\sigma^2}{\bar{U}} > \frac{1}{2}\bar{U} \]
with probability 1.

\subsection{Renewal-Reward Processes}
A \emph{reward function} is a function $R(t)$ that only depends on the the particular inter-renewal interval containing $t$.
Let $U(t)$ be the duration of this interval, i.e. $U(t) = U_{N(t)+1} = S_{N(t)+1} - S_{N(t)}$.
We denote the \emph{age} $Z(t) = t - S_{N(t)}$ as the time we have already waited for the current renewal
and the \emph{residual life} $Y(t) = U(t) - Z(t) = S_{N(t)+1} - t$.

Since $R(t)$ only depends on the particular inter-renewal interval containing $t$,
it only depends on $U(t),Y(t),Z(t)$ and since $Y(t) = U(t) - Z(t)$,
it only depends on $U(t)$ and $Z(t)$.
Denote $\mathcal{R}$ the function such that $R(t) = \mathcal{R}(Z(t), U(t))$.

Let $R(n)$ be the reward accumulated at the $n$th interval.
As $U_n$ are iid, the $R_n$ are iid and for any $n$,
\[ \E[R_n] = \int_{x = 0}^\infty \int_{z=0}^x \mathbb{R}(z,x) \dif z f_U(x) \dif x. \]

We can show that if $\bar{U} < \infty$, then
\begin{equation}
  \label{eq:rew}
  \lim_{t \to \infty} \frac{1}{t} \int_0^t R(\tau) \dif \tau = \frac{\E[R_n]}{\bar{U}}
\end{equation}
with probability 1.
An easy way to show it, which is also useful to use directly is to see that
\[ \lim_{t \to \infty} \frac{1}{t} \int_0^t R(\tau) \dif \tau = \lim_{t \to \infty} \frac{N(t)}{t} \cdot \frac{1}{N(t)} \int_0^t R(\tau) \dif \tau = \frac{1}{\bar{U}} \cdot \E[R_n]. \]

Let $U_t$ be the asymptotic distribution of $U(t)$, i.e. $U_t = \lim_{t \to \infty} U(t)$.
We can see that $f_{U_t}(u) = \frac{uf_U(x)}{\bar{U}}$ so if $U$ has a arithmetic distribution with a step $d$,
\[ \lim_{n \to \infty} \E[R(nd)] = \frac{\E[R_n]}{\bar{U}}, \]
otherwise,
\[ \lim_{t \to \infty} \E[R(t)] = \frac{\E[R_n]}{\bar{U}}. \]

As examples, with $R(t) = U(t)$, we have $\mathbb{R}(z,x) = x$ so $\E[R_n] = \E[U^2]$ and
\[ \lim_{t \to \infty} \frac{1}{t} \int_0^t U(\tau) \dif \tau = \lim_{t \to \infty} \E[U(t)] = \frac{\E[U^2]}{\bar{U}} \geq \bar{U} \]
with probability 1.

With $R(t) = Z(t)$, we have $\mathbb{R}(z,x) = z$ so $\E[R_n] = \E[U^2]/2$ and
\[ \lim_{t \to \infty} \frac{1}{t} \int_0^t Z(\tau) \dif \tau = \lim_{t \to \infty} \E[Z(t)] = \frac{\E[U^2]}{2\bar{U}} \geq \frac{\bar{U}}{2} \]
with probability 1.
For $R(t) = Y(t)$, the result is the same than with $Z$ since $\E[Z_n] = \E[U_n]/2$ and $\E[Y_n] = \E[U_n] - \E[Y_n]$.

It is not surprising that we have the inequalities ${} \geq \bar{U}$ and ${} \geq \bar{U}/2$ since it is more likely that
when choosing a random $t$ we fall into an inter-renewal interval with large duration rather than one with a small duration.

For an application of Renewal-Reward Process, jump to the Section~\ref{sec:littleslaw}.

\section{Markov Chains}
\cite[\S4--5]{gallager1995discrete,gallager2010discrete}, \cite[Appendix~A]{puterman2014markov}
A Markov Chain is a stochastic process with fixed time interval
and a sequence of random variable $\{X_n\}_{n=0}^\infty$
where each random variable $X_n$ only depends on $X_{n-1}$ and not on $X_{n-2},X_{n-3},\ldots$.
We note
\[ P_{ij} = \Pr[X_n = j \mid X_{n-1} = i]. \]
$X_n$ is the state of the Markov chain.
The set of possible state is countable.
If $P_{ij}(n)$ does not depend on $n$ the chain is stationary or time homogeneous,
we only consider this case.

Let $P_{ij}^k = \Pr[X_n = j \mid X_{n-k} = i]$.
If we consider the matrix $\P^k$ such that its entry at line $i$ column $j$ is $P_{ij}^k$ we can see that
$\P^k = (\P)^k$,
it is like the relation with the number of paths in Graph Theory.
It is called the Chapman-Kolmogorov Equation.
Note that if the number of state if infinite, the matrix $\P$ has size $\infty \times \infty$ so
that the result we will derive involving the transition matrix $\P$ are more relevant for Finite Markov Chain
even if as we have just seen $\P^k$ is not a matrix multiplication, it has as much sens whether the number of states is finite or not.

\paragraph{Classes}
We say that state $j$ is \emph{accessible} from state $i$ (denoted $i \to j$) if $P_{ij}^k > 0$ for some $k \geq 0$.
State $j$ \emph{communicates} with state $i$ (denoted $i \leftrightarrow j$) if $i \to j$ and $j \to i$.
A set of states $S$ is
%\begin{itemize}
%  \item \emph{closed} if no state outside $S$ is accessible from any state in $S$,
%  \item \emph{irreducible} if no proper subset of $S$ is closed,
%  \item
    \emph{a class} if no state outside $S$ communicates with any state in $S$ and all pairs of states in $S$ communicate.
%\end{itemize}
%If Chain contains a single class with all the states, it is called irreducible.
%If there is only one closed irreducible set, the chain is called unichain,
%otherwise it is called multichain.

If we consider the induced graph of the chain,
i.e. the graph where the nodes are the state and there is an edge between $i$ and $j$ if $i \to j$,
the classes are the strongly connected components.

If there is only one class the chain is said to be \emph{irreducible},
note that this class can be transient, null recurrent or recurrent.

We will now see two class properties: the type and the period.
We define these two properties for a state and
we can say that the property of the class is simply the property of its states because of the following theorem.
\begin{mytheo}
  In a class, all states are of the same type and same period.
\end{mytheo}

\subsection{Type of a state}
Let $f_{ij}(n)$ be the probability that the \emph{first} visit to $j$ is at time $n$ when starting at $i$ at time 0.
Note that we have $f_{ij}(1) = P_{ij}$ and the recurrence relation
\[ f_{ij}(n) = \sum_{k \neq j} P_{ik}f_{kj}(n-1). \]
We also define $F_{ij}(n)$ as the probability that $j$ is visited at time $n$ \emph{or before} when starting at $i$ at time 0.
We have clearly
\[ F_{ij}(n) = \sum_{m=1}^n f_{ij}(m). \]
Let the random variable $T_{ij}$ be the time for the first visit to $j$ when starting at $i$.
Its mean is
\[ \bar{T}_{ij} = 1 + \sum_{n=1}^\infty 1 - F_{ij}(n). \]

Note that if $F_{jj}(\infty) < 1$, that means that there is a probability to never return at $j$ when starting at $j$.
In that case $\bar{T}_{jj} = \infty$, we say that $j$ is a \emph{transient} state.
If $T_{jj} = \infty$ we say that $j$ is recurrent. See Table~\ref{tab:state}.

\begin{table}[!ht]
  \centering
  \begin{tabular}{c|c|c}
    & $F_{jj}(\infty) < 1$ & $F_{jj}(\infty) = 1$\\
    \hline
    $\bar{T}_{jj} < \infty$ & \tikzmark{b} & positive recurrent\\
    \hline
    $\bar{T}_{jj} = \infty$ & transient & null recurrent
    \connect[0mm]{b.north west}{b.south east}
    \connect[0mm]{b.north east}{b.south west}
  \end{tabular}
  \caption{Types of states}
  \label{tab:state}
\end{table}

\subsection{Period of a state}
The period of a given state $j$ is the greatest common divider of
all the numbers $n$ such that $P_{jj}^n > 0$.
If the period is 1, we say that the state is aperiodic.

If the period of a class is $d$,
there exists a partition $\mathcal{X}_1, \ldots, \mathcal{X}_d$ of the states of the class
such that for $1 \leq n \leq d-1$, all the transitions from a state of $\mathcal{X}_n$ are to a state $\mathcal{X}_{n+1}$
and all the transition from a state of $\mathcal{X}_d$ are to a state of $\mathcal{X}_1$.

\subsection{Ergodic class}
A class recurrent and aperiodic is called \emph{ergodic}.
\begin{mytheo}
  If the states $i$ and $j$ are in the same class
  and their class is ergodic then
  \[ \lim_{n \to \infty} P_{ij}^n = \frac{1}{\bar{T}_{jj}}. \]
\end{mytheo}

\subsection{Steady-state distribution}
The steady-state distribution $\bpi$ represents the limiting probability distribution.
If it exists, independently on where we start, after an infinite number of transition,
we have a probability $\pi_j$ to be at the state $j$.
We represent $\bpi$ as a line vector.
\begin{mytheo}
  For an irreducible Markov Chain,
  the system
  \begin{align}
    \label{eq:piP}
    \bpi & = \bpi \P\\
    \notag
    \bpi\mathbf{1} & = 1\\
    \notag
    \bpi & \geq 0.
  \end{align}
  has a unique solution iff all the states (which have the same type since it is irreducible)
  are positive recurrent.
  If the solution exists, satifies $\pi_j = 1/\bar{T}_{jj}$ for all state $j$.
\end{mytheo}

If we take a random vector $x \geq 0$ such that $x \mathbf{1} = 0$ and compute
$\lim_{n \to \infty} x \P^n$, do we get $\bpi$ ?
This actually only works if the chain is aperiodic.
For example, for
\[
  \P =
  \begin{pmatrix}
    0 & 1\\
    1 & 0
  \end{pmatrix},
\]
$\bpi = \begin{pmatrix}1/2 & 1/2\end{pmatrix}$
but if $x \neq \bpi$,
we will have $x\P^{2n} = x$ and $x\P^{2n+1} = x\P \neq x$.

We can see that $\P^{n} = \P \cdot \P^{n-1}$ so the lines are the lines
or $\P$ for which we apply $n-1$ times $\P$.
Therefore
\begin{mytheo}
  For an ergodic Markov Chain for which the unique class is positive recurrent,
  \[ \lim_{n \to \infty} \P^n = \mathbf{1} \bpi. \]
\end{mytheo}

\subsection{Reversible Markov chain}
\cite[\S5.3]{gallager1995discrete,gallager2010discrete}

Consider at backward chain: at each transition,
we are at a state $X_n$ we wonder what were the state $X_{n-1}$ we were coming from.
Not surprisingly, the probability distribution of the states we were at $n-1$ does not
depend on $X_{n+1}, X_{n+2}, \ldots$, i.e. the backward chain is also a Markov Chain.
The way to compute the transition probabilities of the backward chain it is using the Bayes rule
\[ P_{ij}^*(n) = \Pr[X_{n-1} = j | X_n = i] = \frac{P_{ji}\Pr[X_{n-1} = j]}{\Pr[X_n = i]}. \]
We can see that the backward chain is not homogeneous.
For example, consider the chain starting at the state 0, with $1/2$ probability
of doing $+1$ and $1/2$ of doing $-1$ at each transition.
$\Pr[X_1 = 1] = 1/2 = \Pr[X_1 = -1]$ and $\Pr[X_1 = x] = 0$ for $x \notin \{-1,1\}$.
$\Pr[X_0 = 0 | X_1 = 1] = 1$ but $\Pr[X_2 = 0 | X_3 = 1] = P_{01} \cdot \Pr[X_2=0]/\Pr[X_3=1] = (1/2) \cdot (1/2) / (3/8) = 2/3$.

In steady-state, $\Pr[X_{n-1} = j] = \pi_j$ and $\Pr[X_n = i] = \pi_i$ so the backward chain is homogeneous and we have
\[ P_{ij}^* = P_{ji} \frac{\pi_j}{\pi_i} \]
or equivalently, $P_{ij}^*\pi_i = P_{ji} \pi_j$.
We say that a Markov chain is reversible if in steady-state,
$P_{ij}^* = P_{ij}$ (or equivalently $P_{ij}\pi_i = P_{ji} \pi_j$),
i.e. in steady-state there is no distinction between backward and forward order.
If you were to observe a reversible Markov Chain in steady-state, you would not be able to tell whether it is running
backward or forward.

We can see that for periodic Markov Chain with period $d > 2$,
it is not reversible,
Recall that we can do a partition of the states $\mathcal{X}_1, \ldots, \mathcal{X}_d$.
Suppose the current state belongs tof $\mathcal{X}_2$,
if the next state belongs to $\mathcal{X}_3$,
it is going forward and if it belongs to $\mathcal{X}_1$,
it is going backward.

The reversibility is very useful because if we feel that a chain is reversible,
we can find the transition probabilities using the following theorem.
\begin{mytheo}
  \label{theo:reversible}
  Consider an irreducible Markov Chain.
  If the vector $\bpi > 0$ is such that
  \begin{equation}
    \label{eq:reversible}
    P_{ij}\pi_i = P_{ji} \pi_j, \quad \forall i,j
  \end{equation}
  and $\bpi\1 = 1$
  then $\bpi$ is the steady-state distribution
  and the chain is reversible.
\end{mytheo}
Note that if the period is 2 the there is no steady-state,
or there is a steady-state for even time and one for odd times.
The value of $\bpi$ found by this theorem is the average of both.

We have also seen that if a chain is reversible, then \eqref{eq:reversible} holds.
That can be used to prove that a chain is not reversible.
We can see that if $P_{ij} > 0$ and $P_{ji} = 0$, we must have $\pi_i = 0$.
If $i$ it is not transient, that is impossible so the chain cannot be reversible.
It can also be shown that if a chain is reversible,
for every cycle, $i_1, \ldots, i_n$ we must have $P_{i_1i_2} \cdots P_{i_{n-1}i_n} = P_{i_ni_{n-1}} \cdots P_{i_2i_1}$,
that can also be used to prove that a chain is not reversible.

What can we do if a chain is not reversible to find the vector $\bpi$ ?
We can still use the backward chain, it won't be indistinguishable from the forward chain but it doesn't matter.
\begin{mytheo}
  \label{theo:backward}
  Consider an irreducible Markov Chain.
  if the vector $\bpi > 0$ is such that
  \begin{equation}
    \label{eq:backward}
    P_{ij}^*\pi_i = P_{ji} \pi_j, \quad \forall i,j
  \end{equation}
  and $\bpi\1 = 1$
  then $\bpi$ is the steady-state distribution
  and $\P^*$ is the transition matrix of the backward chain in steady-state.
\end{mytheo}
To use this Theorem, the $P_{ij}*$ need to be guessed.
Remember when trying to guess it that it is the probability that $X_{n-1} = i$
if $X_{n} = j$ in \emph{steady-state}.
It is not something valid for all $n$.
\begin{center}
  ``One might think that guessing is somehow unscientific,
  but in fact, the art of educated guessing and intuitive reasoning is at the heart of all good scientific work.''
  \cite[p.~273]{gallager2010discrete}
\end{center}

\subsubsection{Birth-Death Markov chains}
Consider the Markov Chain represented by the \figuref{birthdeath}.
The experienced reader have already guessed that this chain is reversible.
We will, at the same time, prove it and compute the steady-state distribution using Theorem~\ref{theo:reversible}.

\begin{figure}[!ht]
  \centering
  \begin{tikzpicture}
    \SetGraphUnit{3}
    \SetUpEdge[style={->}, labelstyle={draw}]
    \Vertex{0}
    \EA(0){1} \EA(1){2} \EA(2){3} \EA[empty=true](3){4}
    \node[right] at (12,0) {$\cdots$};
    \tikzset{EdgeStyle/.append style = {bend left}}
    \Edge[label=$p_0$](0)(1)
    \Edge[label=$p_1$](1)(2)
    \Edge[label=$p_2$](2)(3)
    \Edge[label=$p_3$](3)(4)
    \Edge[label=$q_1$](1)(0)
    \Edge[label=$q_2$](2)(1)
    \Edge[label=$q_3$](3)(2)
    \Loop[dist=50,dir=EA,label=$1-p_0$,labelstyle={fill=white,draw}](0)
    \Loop[dist=50,dir=EA,label=$1-p_1-q_1$,labelstyle={fill=white,draw}](1)
    \Loop[dist=50,dir=EA,label=$1-p_2-q_2$,labelstyle={fill=white,draw}](2)
    \Loop[dist=50,dir=EA,label=$1-p_3-q_3$,labelstyle={fill=white,draw}](3)
  \end{tikzpicture}
  \caption{Birth-Death Markov Chain}
  \label{fig:birthdeath}
\end{figure}

If we apply it between $i$ and $i+1$, we get $p_i\pi_i = q_{i+1}\pi_{i+1}$.
Let $\rho_i = p_i / q_{i+1}$, we have $\pi_{i+1} = \pi_i \rho_i$.
More generally, we have $\pi_n = \pi_0 \prod_{i=0}^{n-1} \rho_i$.
Since $\bpi\1 = 1$, we have
\[ \pi_0 = \frac{1}{1 + \sum_{n=1}^\infty \sum_{i=0}^{n-1} \rho_i}. \]
\begin{itemize}
  \item If $\sum_{n=1}^\infty \sum_{i=0}^{n-1} \rho_i < \infty$ (e.g. $\exists \epsilon > 0$ and $I$ such that $\rho_i \leq 1 - \epsilon$ for all $i > I$)
    then all the states are positive recurrent: $\bpi > 0$.
  \item Otherwise, no state is positive recurrent: $\bpi = 0$.
\end{itemize}

Suppose now that $p_i$ and $q_i$ are constant.
\begin{itemize}
  \item If $\rho < 1$, all the states are positive recurrent and $\pi_i = \rho^i(1-\rho)$.
  \item Otherwise, they are either null-recurrent or transient.
    If $p + q = 1$ and $\rho > 1$ they are all transient.
\end{itemize}

\section{Finite Markov Chains}
We say that a Markov Chain is finite if it has a finite number of states.

For Finite Markov Chain,
\begin{itemize}
  \item null recurrent states do not exist,
    all recurrent state are positive recurrent and
  \item there is at least one class that is recurrent.
\end{itemize}

Therefore in the case of an irreducible Finite Markov Chain,
the unique class is positive recurrent.

If the Finite Markov Chain is not irreducible,
there may be transient states but there
also may be several recurrent classes.

If there is only one recurrent class,
the chain is called unichain.
If this recurrent class is ergodic, we say that the chain is ergodic recurrent.

We have seen the condition for the existence of the steady-state distribution for irreducible Markov Chain.
If the Markov Chain is finite and unichain,
the steady-state distribution vector also exists.
For every transient state $j$ we have $\pi_j = 0$ and the part of $\bpi$ in the recurrent class
is the same if we remove the transient states.

\subsection{Markov Chain with Rewards}
\cite[\S4.5]{gallager1995discrete,gallager2010discrete},

Suppose we assign to each state a reward $r_i$.
If the reward is rather assigned to each transition from $i$ to $j$,
just take $r_i = \sum_{j} r_{ij} P_{ij}$, the expected reward at state $i$.

Clearly, the expected reward for an infinite time is just infinite (in most cases).
Let us look at the expected reward over $n$ transitions starting at $i$.
We define $v_i(n)$ as the expected reward of the $n$ states $i, \ldots$ visited plus
$v_j(0)$ where $j$ is the $(n+1)$th state.
The value of the vector $\mathbf{v}(0)$ doesn't really matter since we will look at the value of $v_i(n)$ for large $n$.

We know that the distribution will tend to be $\bpi$ even if we start at $i$ for ergodic Markov Chains.
Let
\( g = \bpi \r, \)
we can expect that for large $n$, $\v(n) = \w + ng\1 + \beta\1$, where $w_i$ is the relative advantage of starting at $i$.
Since it is relative we will typically set $w_1 = 0$ (otherwise, $\beta$ would be useless).
We cannot hope to get rid of $\beta$ since $\v(0)$ is arbitrary and its contribution is not related to the starting state $i$
for large $n$.

We can see that $\w$ should satify the relation
\[ \w + g\1 = \r + \P\w. \]
Since $\P$ is row stochastic, we can see that if $\w$ is solution then $\w + \alpha\1$ is also solution for all $\alpha \in \R$.
Therefore, it is wrong to think that according to this equation $g$ depends on $\w$ and for different values of
$\w$ we have different values of $g$.
What we rather have is that for values of $g$ different than $\bpi \r$ we have no solution for $\w$ and if
we have a solution for $\w$ we have an infinite number of solution.
If we add a condition such as $w_1 = 0$, we can hope to get a unique solution.

Actually, if the Markov Chain is irreducible then with the conditions such as $w_1 = 0$ or $\1^T\w = 0$,
the solution exists and is unique.
Furthermore, if it is irreducible, we have
\[ \lim_{n \to \infty} \v(n) - ng\1 = \w + \beta\1 \]
with $\beta = \bpi(\v(0) + \w)$.

Again, we see that since $\bpi\1 = \1$, the right hand side does not change if we replace $\w$ by $\w+\alpha\1$.
That means that this equation holds whether we add $w_1 = 0$, $w_2 = 0$ or $\1^T\w$ to find the value of $\w$.

\subsection{Markov Decision Processes (MDP)}
\cite[\S4.6]{gallager1995discrete,gallager2010discrete},
\cite{puterman2014markov}

Suppose that we want to find the decisions that maximize $g$ or $\v(n)$.
The decisions can affects the matrix $\P$ and the rewards vector $\r$.
At each state $i$ we can take any decision of the set of decisions $K_i$ and the decision taken at state $i$ does not affect the decision $K_j$ for $j \neq i$.
Denote the line $i$ of the transition matrix if we take the decision $k_i$ at $i$ as $P_i^{(k_i)}$
and the reward at $i$ as $r_i^{(k_i)}$.

\subsection{Dynamic Policy: Dynamic Programming Algorithm}
Suppose we want to maximize all the entries of $\v(n)$ at each $n$
and that we can change the policy at each step.
Normally it is not well defined to maximize different scalar at the same time but since we can
change the policy we can do it so it is well defined.

We can see that we can compute the optimal decision $k_i^*(n)$ if we know the optimal decisions $\k^*(n-1), \ldots, \k^*(1)$.
Actually, we just need the corresponding vector $\v^*(n-1)$.
We have
\[ k^*_i(n) \in \argmax_{k \in K_i} r_i^{(k)} + \P_i^{(k)} \v^*(n-1) \]
Once we have $k_i^*(n)$ we can get $v_i^*(n)$.
In short, we can compute $\v^*(n)$ if we have $\v^*(n-1)$.

\paragraph{Principle of Dynamic Programming}
Consider the directed graph that have the nodes $0, 1, \ldots$
and an edge from $n$ to $n-1$ for all $n = 1, 2, \ldots$.
At node $n$, the edges represents the requirements to compute $\v^*(n)$.
When this directed graph is acyclic that means that there is an order of the node such that we can compute every node in that order without any problem of requirement.
Think about the courses and their requirements,
if there is no cyclic requirement there is an order of the course such that the student can follow all the course in that order.
This is the principle of dynamic programming.
Computing the nodes in this order is called the bottom up approach,
it is used when the order is easy to compute.
Otherwise we just use the top down approach that consists in recursively compute every node that is a requirement that is not yet computed remembering the result of previous computation.
It is very useful to think of Dynamic Programming as a method to be used when the computation requirements form a Directed Acyclic Graph (DAG)~\cite[\S15.3]{cormen2009algorithm} and Greedy Programming as a method to be used when the feasible solutions have a matroid structure~\cite[\S16.4]{cormen2009algorithm}.

This results in the Algorithm~\ref{algo:dynalgo}.

\begin{algorithm}[!ht]
  \caption{Dynamic Programming algorithm for a given $\v(0)$ using the bottom up approach.}
  \label{algo:dynalgo}
  \begin{algorithmic}
    \FOR{$n = 1, 2, \ldots$}
      \FOR{each state $i$}
        \STATE $v_i^*(n) \leftarrow \max_{k \in K_i} r_i^{(k)} + \P_i^{(k)} \v^*(n-1)$
      \ENDFOR
    \ENDFOR
  \end{algorithmic}
\end{algorithm}

\subsection{Stationnary policy: Policy iteration}
Let us now suppose that we want to maximize $g$ and that we must apply the same policy $k_i$ for all $n$.

Does it change something to impose that we cannot change the policy ?
It depends on $\v(0)$.
If $\v(0) = \w^{(\k^*)}$ where $w^{(\k^*)}$ is the relative advantage for $P^{(\k^*)}$ and $r^{(\k^*)}$,
then even if we allow to change the policy, the optimal policy will be the same for each $n$, i.e. $\k^*(n) = \k^*$.

\begin{mytheo}
  If the Markov Chain is irreducible then $\k^*$ is a stationnary optimal policy iff
  for all policy $\k$,
  \[ \r^{(\k^*)} + \P^{(\k^*)} \w^{(\k^*)} \geq \r^{k} + \P^{k} \w^{(\k^*)}. \]
\end{mytheo}

This optimality condition allows us to write Algorithm~\ref{algo:policyiteration}.
The number of iteration is finite since $g^{\k^n}$ increases at each iteration.
\begin{algorithm}[!ht]
  \caption{Policy Iteration algorithm (Howard~\cite{howard1960dynamic}).}
  \label{algo:policyiteration}
  \begin{algorithmic}
    \STATE Choose an artitrary starting policy $\k^0$
    \STATE $n \leftarrow 1$
    \WHILE{$\k^n \neq \k^{n-1}$}
      \STATE Compute $\w^{(\k^{n-1})}$
      \FOR{each state $i$}
        \STATE $k_i^n \leftarrow \argmax_{k \in K_i} r_i^{(k)} + \P_i^{(k)} \w^{(\k^{n-1})}$
      \ENDFOR
      \STATE $n \leftarrow n + 1$
    \ENDWHILE
  \end{algorithmic}
\end{algorithm}

If the Markov Chain is irreducible for all policy
and ergodic for an optimal stationnary policy $\k^*$ then
\[ \lim_{n \to \infty} \v^*(n) - ng^{(\k^*)}\1 = \w^{(\k^*)} + (\beta'-\bpi^{(\k^*)}\w^{(\k^*)})\1. \]
The relative gain $\w$ of all optimal stationnary policy is the same.
Also we have
\[ \min_i v_i^*(n) - v_i^*(n-1) \leq g^{(\k^*)} \leq \max_i v_i^*(n) - v_i^*(n-1) \]

That gives Algorithm~\ref{algo:valueiteration}.
At the end of the algorithm we have $g^{\k^*} = \min_i v_i^*(n) - v_i^*(n-1) = \max_i v_i^*(n) - v_i^*(n-1)$.
\begin{algorithm}[!ht]
  \caption{Value Iteration algorithm for a given $\v(0)$.}
  \label{algo:valueiteration}
  \begin{algorithmic}
    \STATE Choose an arbitrary starting value $\v(0)$
    \STATE $n \leftarrow 1$
    \WHILE{$\min_i v_i^*(n) - v_i^*(n-1) < \max_i v_i^*(n) - v_i^*(n-1)$}
      \FOR{each state $i$}
        \STATE $v_i^*(n) \leftarrow \max_{k \in K_i} r_i^{(k)} + \P_i^{(k)} \v^*(n-1)$
      \ENDFOR
      \STATE $n \leftarrow n + 1$
    \ENDWHILE
  \end{algorithmic}
\end{algorithm}

\section{Semi-Markov Processes}
\cite[\S6.8]{gallager1995discrete,gallager2010discrete}
A Semi-Markov Process is the generalization of all the process we have seen.
It can either be seen as a renewal process with states where the distribution of $U_n$ depends on $X_{n-1}$ and $X_n$
or as a Markov Process for which we add a time $U_{ij}(n)$ between the transition between the state $i$ and $j$.
The Markov Process is somewhat composed of an \emph{embedded Markov Chain} and an \emph{embedded renewal process} for which
the transition rate depends on the current state of the embedded Markov Chain.

We say that a Markov Process is irreducible if the embedded Markov Chain is irreducible.

At a time $t$, we are in a transition between $X_{n-1}$ and $X_n$ for some $n$.
The distribution of the time $U(n)$ that this distribution will take depends on the value of $X_{n-1}$ and $X_n$.
Since we are not sure of the value of $X_n$ we do not know its distribution in advance.
For each pair of state $i,j$ we have a distribution of mean $\bar{U}(i,j)$.
Since we know the probability to have a transition to $j$ when at $i$ we can be more precise,
at a state $i$ we have a distribution of mean $\bar{U}(i)$ for the transition time where
\[ \bar{U}(i) = \sum_j P_{ij} \bar{U}(i,j). \]

Since $\bar{U}(i)$ may differ from one state to another,
$\pi_i > \pi_j$ does not mean that is is more likely to be in state $j$ than state $i$.
If the embedded Markov Chain is irreducible and positive recurrent with steady-state distribution $\bpi$,
then the steady-state probability to be at state $i$ is
\begin{equation}
  \label{eq:pisemi}
  p_i = \frac{\pi_i \bar{U}(i)}{\bar{U}_\pi} = \frac{\bar{U}(i)}{\bar{U}_{ii}}
\end{equation}
where
\[ \bar{U}_\pi = \sum_j \pi_j \bar{U}(j) \]
is the expected transition time \emph{in steady-state}.
The expected return time is $\bar{U}_{ii}$ which is like $\bar{T}_{ii}$ but with the transition time taken into account.
We have $\pi_i = 1/\bar{T}_{ii}$, from the above equation, we see that
\begin{equation*}
  \bar{U}_{ii} = \bar{T}_{ii} \bar{U}_\pi.
\end{equation*}
Using \eqref{eq:Nt} ($\bar{U}_\pi$ is only valid in steady state but since we take $t \to \infty$, we turn a blind eye to it)
we get
\[ \lim_{t \to \infty} \frac{N(t)}{t} = \frac{1}{\bar{U}_\pi}. \]
with probability 1.

Consider the counting process $N_{jj}(t)$,
we can see that it is a renewal process with transition time with mean $\bar{U}_{jj}$.
For the counting process $N_{ij}(t)$ it is almost the same,
the only difference is the first transition time that hs mean $\bar{U}_{ij}$.
A renewal process with just a different first transition time is a delayed renew process.
We can still use \eqref{eq:Nt} for such process so we have
\[ \lim_{t \to \infty} \frac{N_{ij}(t)}{t} = \frac{1}{\bar{U}_{jj}}. \]
with probability 1.

One last identity:
The probability that we are in the transition from $i$ to $j$ in steady-state is
\[ p_i \frac{P_{ij}\bar{U}(i,j)}{\bar{U}(i)}. \]

\section{Markov Processes}
\cite[\S6]{gallager1995discrete,gallager2010discrete}
A Markov Process is a Semi-Markov Process such that $P_{ii} = 0$ for all state $i$ (i.e. no ``self-loop''),
and $U_n$ only depends on $X_{n-1}$ and is exponential of rate $\nu_{X_{n-1}}$.
We define the transition rates to from $i$ to $j$ as $Q_{ij} = \nu_i P_{ij}$ (i.e. $\Q = (\diag{\bnu})\P$),
we see that $\sum_j Q_{ij} = \nu_i$, in other words, $(\Q - \diag(\bnu)) \1 = 0$.

Here $\bar{U}(i,j) = \bar{U}(i) = 1/\nu_i$ so \eqref{eq:pisemi} becomes
\begin{equation}
  \label{eq:ppinu}
  p_i = \pi_i / (\nu_i \bar{U}_\pi)
\end{equation}
and if $\bar{U}_\pi < \infty$ which is the case in virtually all applications,
$\pi_i = p_i \nu_i \bar{U}_\pi$.
Note that this gives an expression of $\bpi$ in terms of $\p$ because using $\bpi\1 = 1$ we see that $\bar{U}_\pi = \sum_i p_i \nu_i \eqdef \bar{U}_p$:
\begin{equation}
  \label{eq:pipnu}
  \bpi = \p\diag{\bnu} / \bar{U}_p
\end{equation}
Using this in \eqref{eq:piP} then cancelling $\bar{U}_p$ and using $\Q = (\diag{\bnu})\P$ gives \eqref{eq:pnuQ}.
\begin{mytheo}
  Let $\p$ be the solution of the following system
  \begin{align}
    \label{eq:pnuQ}
    \p \diag{\bnu} & = \p\Q\\
    \p \1 & = 1.
  \end{align}
  If $\bar{U}_p < \infty$ then the solution is unique, $\p > 0$, the embedded Markov Chain is positive recurrent
  and $\bpi$ is given by \eqref{eq:pipnu}.

  If the embedded Markov Chain is positive recurrent and $\bar{U}_\pi < \infty$, then the system has a unique solution
  given by \eqref{eq:ppinu}.
\end{mytheo}

Let $p_{ij}(t) = \Pr[X(t)=j|X(0)=i]$,
we have the \emph{Chapman-Komogorov differential equation}:
\[ \fdif{\p(t)}{t} = (Q - \diag(\nu))\p(t) = \p(t)(Q - \diag(\nu)). \]
We have seen that $Q - \diag(\nu)$ has an eigenvalue 0 with right eigenvector $\1$ and left eigenvector $\p$ (see \eqref{eq:pnuQ}).
It turns out that the other eigenvalues have a negative real part.
Let $\Lambda,V$ be such that $\Q = V\Lambda V^{-1}$, we have
\( \p(t) = V\exp(t\Lambda)V^{-1} \).
As $t \to \infty$, for all the negative eigenvalues, $\exp(t\lambda_i) \to 0$ and for the eigenvalue 0, $\exp(t0) = 1$ so we have
\[ \lim_{t \to \infty} p(t) = \1\p. \]
That means that in steady state the probability to be in $j$ does not depend on the starting state,
i.e. $\lim_{t \to \infty} p_{ij}(t) = p_j$.

\subsection{Uniformization}
Note that since we have enforced that $P_{ii} = 0$, we also have $Q_{ii} = 0$.
Suppose that we increase $Q_{ii}$ and let $Q_{ij}$ constant for $j \neq i$, what happens ?
The probability $P_{ij}$ decreases for $j \neq i$ since $P_{ii}$ increases and $\P$ must stay row-stochastic
but since we want to keep $Q_{ij}$ constant, $\nu_i$ must increase.
We have an higher rate $\nu_i$ but a part of this rate is just a self-loop so the outgoing rate to $j \neq i$ doesn't change.
Since $\Q = \diag(\nu)$, and $Q_{ij}$ is constant for $j \neq i$ we see that if we increase $Q_{ii}$ from 0 to $Q_{ii}$, $\nu_i$ increases by $Q_{ii}$.
We see that in \eqref{eq:pnuQ}, the columns $j$ is increased by $p_jQ_{jj}$ on both side so the solution $\p$ is not affected.
However, $\pi_j$ increases since we have added a chance to self-loop insead of leaving state $j$.
If we set $Q_{ii}$ at each node such that $\nu_i$ is a constant $\nu^*$, we will have $\bpi = \p$.
Since $\nu_i$ is the same at each state, the embedded Poisson Process won't need to care about the states anymore,
it will just behave like a Poisson of arrival rate $\nu^*$.

It is important to see that the uniformized process is related to the original process.
We have seen that $\p$ is unchanged but $X(t)$ is also unchanged.
We just add dummy self-loop events, quantities sensible to the events like $N(t)$ or $\nu_i$ increases but
$X(t)$ is unaffected by those dummy events since we stay at the same state.
This uniformized process is very useful to extend the Markov Decision theory to Markov Processes.

\subsection{Reversibility}
Remember the backward chain for Markov Chains.
Here we consider a similar backward process.
The tricky part is how we consider the start and end time at state $i$.
If the forward chain is at state $i$ from $t_1$ to $t_2$,
the backward chain is at state $i$ from $t_2$ to $t_1$ (not $t_3$ to $t_2$ or $t_1$ to $t_0$).
With that choice, the backward process is a Markov Process the same rate $\nu_i$ at state $i$
and probability $P_{ij}^*$ such that $P_{ij}^* \pi_i = P_{ji} \pi_j$.
A similiar way to define the backward process is through $Q_{ij}^*$:
$Q_{ij}^* p_i = Q_{ji} p_j$.

We define the process to be reversible if $Q_{ij} = Q_{ij}^*$.
\begin{mytheo}
  Consider an irreducible Markov Process.
  If the steady states probability $\p > 0$ exists and $\bar{U}_p < \infty$ then
  the Markov Process is reversible iff the embedded Markov Chain is reversible
\end{mytheo}

We have a similar Theorem than Theorem~\ref{theo:reversible}
to prove reversibility and find the steady-state distribution at the same time.
\begin{mytheo}
  \label{theo:reversibleprocess}
  Consider an irreducible Markov Process.
  If the vector $\p \geq 0$ is such that
  \begin{equation}
    \label{eq:reversibleprocess}
    Q_{ij} p_i = Q_{ji} p_j, \quad \forall i,j,
  \end{equation}
  $\p\1 = 1$ and $\bar{U}_p < \infty$,
  then $\p$ is the steady-state distribution,
  $\p > 0$,
  the process is reversible and
  the embedded chain is positive recurrent.
\end{mytheo}

If the Process is not reversible,
we have a Theorem similar to Theorem~\ref{theo:backward}
to find the steady-state distribution and prove
that our guess on the backward process is right.
\begin{mytheo}
  Consider an irreducible Markov Process.
  If the vector $\p > 0$ and the nonnegative matrix $\Q^*$ are such that
  \begin{align*}
    \Q^*\1 & = \Q\1,\\
    Q_{ij}^*p_i & = Q_{ji} p_j, \quad \forall i,j,
  \end{align*}
  $\bpi\1 = 1$ and $\bar{U}_p < \infty$,
  then $\bpi$ is the steady-state distribution,
  $\p > 0$,
  $\P^*$ is the transition matrix of the backward process and
  the embedded chain is positive recurrent.
\end{mytheo}

\subsubsection{Birth-Death Markov Process}
The Birth-Death Markov Process is represented by the \figuref{birthdeathprocess}.
We have $Q_{i,i+1} = \lambda_i$ and $Q_{i,i-1} = \mu_i$.
To see this as a Markov Process, we can set $\nu_i = \lambda_i + \mu_i$, $P_{i,i+1} = \lambda_i/\nu_i$, $P_{i,i-1} = \mu_i/\nu_i$.
However we do not really care, we will rather work with $\Q$ and $\p$.

\begin{figure}[!ht]
  \centering
  \begin{tikzpicture}
    \SetGraphUnit{3}
    \SetUpEdge[style={->}, labelstyle={draw}]
    \Vertex{0}
    \EA(0){1} \EA(1){2} \EA(2){3} \EA[empty=true](3){4}
    \node[right] at (12,0) {$\cdots$};
    \tikzset{EdgeStyle/.append style = {bend left}}
    \Edge[label=$\lambda_0$](0)(1)
    \Edge[label=$\lambda_1$](1)(2)
    \Edge[label=$\lambda_2$](2)(3)
    \Edge[label=$\lambda_3$](3)(4)
    \Edge[label=$\mu_1$](1)(0)
    \Edge[label=$\mu_2$](2)(1)
    \Edge[label=$\mu_3$](3)(2)
  \end{tikzpicture}
  \caption{Birth-Death Markov Process}
  \label{fig:birthdeathprocess}
\end{figure}

We have following result useful to prove reversibility for general Birth-Death Markov Process.
\begin{mytheo}
  For a Birth-Death Markov Process, if there is a solution $\p$ of \eqref{eq:reversibleprocess}
  such that $\p\1 = 1$ and $\bar{U}_p < \infty$, then the process is reversible and the embedded Markov Chain is positive recurrent and reversible.
\end{mytheo}

If we define $\rho_i$ as $\lambda/\mu_{i+1}$,
we have exactly the same relations than with Birth-Death Markov Chain with $\p$ instead of $\bpi$.

We use Theorem~\ref{theo:reversibleprocess}, to get $p_{i+1} = \rho_i p_i$.
Since $\p\1 = 1$, we have
\[ \p_0 = \frac{1}{1 + \sum_{n=1}^\infty \sum_{i=0}^{n-1} \rho_i}. \]

Suppose now that $p_i$ and $q_i$ are constant and $\rho < 1$,
we have $p_i = \rho^i(1-\rho)$.
We can see that in steady state, $\Pr[X(t) \geq n] = \rho^n$ and
\begin{equation}
  \label{eq:barL}
  \bar{L} = \lim_{t \to \infty} \E[X(t)] = \sum_{n=1}^\infty np_n = \sum_{n=1}^\infty \Pr[X(t) \geq n] = \frac{\rho}{1-\rho}.
\end{equation}
where $\bar{L}$ is the average queue length for M/M/1 queue with arrival rate $\lambda$ and exponential service time of rate $\mu$.
Note that the 2 differente series are different ways to compute the same thing, the equality between them is not necessarily trivial.

We end this section with the following Theorem.
Note that the 2 first properties are also valid for Birth-Death Markov Chain.
\begin{mytheo}[Burke]
  Given an M/M/$s$ queueing system in steady state with $\lambda < s\mu$,
  \begin{enumerate}
    \item the departure process is Poisson with rate $\lambda$,
    \item the state $X(t)$ at any time $t$ is independent of departures prior to $t$, and
    \item for First Come First Served (FCFS) service,
      given that a customer departs at time $t$,
      the arrival time of that customer is independent of the departures prior to $t$.
  \end{enumerate}
\end{mytheo}
Not that $\lambda < \mu$ implies that $\rho < 1$ so the Process is reversible.
For the backward process (which is the same that the forward process in steady state since it is reversible),
the third property is
``given that a customer arrives at time $t$,
the departure time of that customer is independent of the arrivals after $t$.''
This is a direct consequence of the FCFS service.

\section{Queueing Systems}
We represent a Queueing System by ``A/B/$s$''
where
\begin{itemize}
  \item A is the type of distribution of the arrival time interval,
  \item B is the time of distribution of the service time,
  \item $s$ is the number of servers.
\end{itemize}
There is a single queue and as soon as one server is not busy and the queue is not empty,
the first customer of the queue is assigned to the server and the service time begins.

Different values of A and B are
\begin{itemize}
  \item M which represents an exponential distribution,
  \item D which represents an deterministic duration,
  \item E which represents an Erlang distribution,
  \item G which represents a General distribution.
\end{itemize}

This is a Semi-Markov Process for which the states are the number of customer in the system.
The Birth-Death Markov Process is a good example, it is an M/M/1 Queueing System.

Denote
the number of arrivals between 0 and $t$ as $A(t)$ and
the number of departures between 0 and $t$ as $D(t)$.
The rv $L(t) = A(t) - D(t)$ is the number of customers in the system at $t$
and the iid rv $W_i$ are the time spent in the system by the customer $i$.

Note that in $W_i$ we have the waiting time in the queue and the service time.
Let $(W_q)_i$ be time spent in the queue by the customer $i$, $L_q$ be the number of customers in the queue at time $t$ and the rv $Z_i$ of mean $\bar{Z}$ be the service time.

In \emph{steady state},
we are interested by
\begin{itemize}
  \item the expected number of customer in the system $\bar{L} = \lim_{t \to \infty} \E[X(t)]$,
  \item the expected time spent in the system $\bar{W}$,
  \item the expected number of customer in the queue $\bar{L}_q$ and
  \item the expected time spent in the queue $\bar{W}_q$.
\end{itemize}

Those 4 values are linked together by the Little's Law and the identity
\begin{equation}
  \label{eq:wwq}
  \bar{W} = \bar{W}_q + \bar{Z}.
\end{equation}
Once we have one we get the 3 other ones easily.
Typically, we compute $\bar{L}$ similarly as with \eqref{eq:barL}.

\subsection{Little's Law}
\label{sec:littleslaw}
We would like to compute the averate waiting time but $L$ depends on the all the customer of the queue,
not just the last one.
However, if we look at idle time, i.e. time when no one is in the system,
and the busy period between the idle times, we can see that $L(t)$ only depends on the busy period at $t$.
Therefore, if we consider the counting process for which $N(t)$ is the number of busy period already finished,
$L(t)$ is a reward function for this process.

We can apply \eqref{eq:rew} (actually we reuse the proof to make it clear),
\[ \bar{L} = \lim_{t \to \infty} \frac{1}{t} \int_0^t L(\tau) \dif \tau = \lim_{t \to \infty} \frac{A(t)}{t} \cdot \frac{\sum_{i=1}^{A(t)} W_i}{A(t)} = \frac{1}{\bar{U}} \cdot \bar{W} \]
with probability 1.
Here $\E[R_n] = \E[L_n] = \bar{W}$.

Similarly if we consider the reward function $L_q(t)$ we get
\[ \bar{L}_q = \lim_{t \to \infty} \frac{1}{t} \int_0^t L_q(\tau) \dif \tau = \frac{1}{\bar{U}} \bar{W}_q \]
with probability 1.

Note that using Little's Law, \eqref{eq:wwq} becomes
\[ \bar{L} = \bar{L}_q + s\rho. \]
where $\rho = \frac{\bar{Z}}{s\bar{U}}$, $s\rho$ is the expected number of customer being served in steady-state.

\subsubsection{M/G/$s$ queue}
For Poisson arrival, we can rid of this ``with probability 1''.
Let $\lambda$ such that $\bar{U} = 1/\lambda$, we have
\begin{align*}
  \bar{L}   & = \lim_{t \to \infty} \E[L(t)] = \lim_{t \to \infty} \frac{1}{t} \int_0^t L(\tau) \dif \tau = \lambda \bar{W}\\
  \bar{L}_q & = \lim_{t \to \infty} \E[L_q(t)] = \lim_{t \to \infty} \frac{1}{t} \int_0^t L_q(\tau) \dif \tau = \lambda \bar{W}_q.
\end{align*}
where the equality with the integral is with probability 1 but if we remove it we get a deterministic equality.

\subsubsection{M/M/$s$ queue}
For M/M/$s$ queue, let $\mu$ such that $\bar{Z} = 1/\mu$, we get $\rho = \frac{\lambda}{s\mu}$,
our notation is consistent with Birth-Death Markov Processes.
For $\rho < 1$, the steady-state probabilities are
\[
  p_n =
  \begin{cases}
    \frac{n^n}{n!}\rho^np_0 & n \leq s\\
    \frac{s^s}{s!}\rho^np_0 & n \geq s
  \end{cases}
\]
with
\[ p_0 = \left(\sum_{n=0}^{s-1} \frac{s^n}{n!}\rho^n + \frac{s^s}{s!} \frac{\rho^s}{1-\rho}\right)^{-1}. \]
and the average number of customers in the queue is
\[ \bar{L}_q = p_0\frac{s^s}{s!}\frac{\rho^{s+1}}{(1-\rho)^2} \]

\subsubsection{M/G/1 queue}
For M/G/1 queue, we have the Pollaczek–Khinchine formula
\[ \bar{W}_q = \frac{\lambda(\bar{Z}^2 + \sigma_Z^2)}{2(1-\lambda\bar{Z})}. \]

\subsection{Fluid Approximation}
Suppose now that $\lambda(t)$, $\mu(t)$ and even $s(t)$ are not constant.

For M/G/$s$, the steady-state approximation is valid if
\[ \left|\fdif{\rho(t)}{t}\right| \ll s(t)\mu(t)(1-\rho(t))^3. \]
Therefore, if $\rho(t)$ does not varies to much and $\rho(t) < 1$, it is ok.

If $\rho(t) \leq 1$ and $1 - \rho(t) \ll 1$, the approximation is difficult,
we use diffusion approximations.
If the process stays in this zone for a short time, we just ignore the problem
but if it stays in the zone for a long time, we take the average of $\rho$
and use it in a steady-state approximation.

If $\rho(t) > 1$,
we assume that the variance of the rv $U$ and $Z$ is small compared to the variability of $\rho(t)$
and we use a deterministic model.
This is called a fluid approximation.

There is 3 regimes:
\begin{description}
  \item[No Queue]
    If $L_q(t) = 0$ and $\rho(t) < 1$ then the queue stays empty $\fdif{L_q(t)}{t} = 0$
    and the effective service rate is $\lambda(t)$.
  \item[Queue growth]
    If $\rho(t) \geq 1$ then the queues grows $\fdif{L_q(t)}{t} = \lambda(t) - s(t)\mu(t) \geq 0$
    and the effective service rate is $s(t)\mu(t)$.
  \item[Queue decline]
    If $L_q(t) > 0$ and $\rho(t) < 1$ then the queue shrinks $\fdif{L_q(t)}{t} = \lambda(t) - s(t)\mu(t) < 0$
    and the effective service rate is $s(t)\mu(t)$.
\end{description}

Note that if $s(t)\mu(t)$ is constant and $\lambda(t)$ increases then decreases,
the maximum queue length is in the Queue decline regime.
For a customer, it is interested to arrive before the peak of $\lambda(t)$.

If we want to choose $s(t)\mu(t)$ to minimize $\alpha s(t)\mu(t) + \beta \int \bar{W_q}(t) \dif t$,
we should take $s(t)\mu(t)$ such that the time the queue is non-empty is $\sqrt{\frac{2\alpha}{\beta}}$.

\nocite{*}
\biblio[alpha]

\end{document}
