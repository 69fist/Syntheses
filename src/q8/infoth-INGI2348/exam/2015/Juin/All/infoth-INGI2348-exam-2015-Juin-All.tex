\documentclass[en]{../../../../../../eplexam}

\hypertitle{Information theory and coding}{8}{INGI}{2348}{2010}{Juin}
{Beno\^it Legat}
{Benoît Macq, Jérôme Louveaux Jérôme and Pereira Olivier}

\paragraph{Discussion link}
\url{TODO}

\section{BM-1}
Explain why the cryptographic coding is done after the compression.
\begin{solution}
  For two reasons.
  \begin{itemize}
    \item First because the cryptographic protocols are stronger when the message space is distributed uniformly
    \item and second because the output of the cryptographic does not contains any redundance so we cannot compress anything.
  \end{itemize}
\end{solution}

\section{BM-2}
Encode the sentence
\begin{center}
  miss ping is missing
\end{center}
\nosolution

\section{BM-3}
Give the rate distortion for a variable $X$ with PDF $f_X(x) = 2 - 2x$, $0 \leq x \leq 1$.
\nosolution

\section{JL-1}
A transmission is performed using the Hamming code $\mathbb{H}_3$ of redundancy 3.
\begin{enumerate}
  \item Determine a parity and generating matrix for the code.
    What is the code rate?

    \textbf{Hint}: You can always use an equivalent code by reordering the columns of the parity matrix
    if it makes computations simpler.
  \item Characterize the correcting capability of the code.
    What is the covering radius of the code?
  \item In the following case when is it sure that the error is going to be corrected,
    when is it sure it will \emph{not} be corrected,
    when is it unsure (i.e. it depends on the particular codeword transmitted and error positions):
    \begin{itemize}
      \item 1 bit erro in the block
      \item 2 bit errors in the block
      \item 3 bit errors in the block
    \end{itemize}
  \item Establish the list of coset leaders for this code.
  \item Assume that, after transmission on a binary symmetric channel,
    the received vector is given by
    \[ \underline{y} =
      \begin{pmatrix}
        1 & 0 & 0 & 0 & 1 & 0 & 0
      \end{pmatrix}
    \]
    for the code you have defined in question 1.
    Perform the syndrome decoding for the received vector $\underline{y}$.
\end{enumerate}
\nosolution

\section{JL-2}
We consider a memoryless channel taking 3 possible inputs,
which are the nonzero elements of the Galois field $\mathbb{F}_4$ of size 4.
The primitive element of $\mathbb{F}_4$ is denoted by $\alpha$ and satisfies $\alpha^2 = \alpha + 1$.
The channel can be used at the rate $10^6$ symbols per second.
For each input $X$, the channel output is charaterized as
\[
  Y =
  \begin{cases}
                 X & \text{with probability } q = 0.89\\
    \alpha \cdot X & \text{with probability } q = 0.11
  \end{cases}
\]
\begin{enumerate}
  \item Write down the channel transition probability... TODO
\end{enumerate}
\nosolution

\section{OP-1}
TODO
\nosolution

\end{document}
