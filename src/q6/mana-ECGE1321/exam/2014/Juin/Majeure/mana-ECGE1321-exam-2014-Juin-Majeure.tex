\documentclass[fr]{../../../../../../eplexam}

\hypertitle{Management humain}{6}{ECGE}{1321}{2014}{Juin}
{Florian Thuin}
{Nathalie Delobbe}

\section*{Exemples de questions ouvertes sur 8 points}

\begin{enumerate}
\item Comme illustré dans le documentaire « Call-center, les nouveaux prolétaires », les travailleurs de ce secteur souffrent souvent de conditions de travail difficiles et certains finissent par être médicalement inaptes au travail en raison d’un épuisement professionnel. En mobilisant tous les différents niveaux d’intelligibilité d’Ardoino et en illustrant votre réponse à l’aide du documentaire, analysez les différents facteurs susceptibles d’expliquer les phénomènes d’épuisement observés dans ce secteur.

\item Dans le secteur des call-centers, le système technique impose des contraintes fortes aux travailleurs. En illustrant votre propos par le reportage « Call-centers : les nouveaux prolétaires », analysez quelques conséquences possibles de ce systèmes technique sur le système social. En référence aux principes de l’approche socio-technique, identifiez aussi quelques principes d’action possibles pour mettre en place une articulation plus harmonieuse entre système technique et système social dans ce secteur. Identifiez également quelques difficultés et obstacles auxquels cette approche risque d’être confrontée.

\item Les façons de concevoir, d’expliquer et de gérer les problèmes de motivation au travail ont très fortement évoluées au fil des théories managériales. En vous appuyant sur les 4 rôles de la fonction RH identifiés par D. Ulrich, analysez comment la conception de la fonction « ressources humaines » qui prévalait à un moment donné était en lien avec l’approche de la motivation au travail.

\item Les phénomènes de leadership dans les organisations ont fait l’objet d’analyses et de théories très diversifiées. En mobilisant tous les différents niveaux d’intelligibilité d’Ardoino, analysez les facteurs évoqués pour expliquer l’émergence et/ou la qualité du leadership par les auteurs ayant traité de ce sujet.

\item Les rôles de « Champion des employés » et de « Partenaire stratégique » sont situés aux antipodes l’un de l’autre dans le modèle d’Ulrich. Expliquez en quoi ces deux rôles de la fonction RH sont fondamentalement divergents et identifiez au moins deux critiques qui peuvent être adressées à chacun de ces rôles. Identifiez également quelques priorités d’action typiques de chacun de ces rôles et susceptibles d’être concrètement mises en œuvre dans le contexte du groupe Walibi.

\item Le film « Ressources humaines » de Laurent Cantet présente l’histoire d’un jeune stagiaire en GRH, révolté après la découverte d’un plan de réduction d’effectif, et de son père, résigné face à la situation. En mobilisant tous les différents niveaux d’intelligibilité d’Ardoino, analysez les facteurs susceptibles d’expliquer la révolte du fils et la résignation du père dans cette même situation.

\item Dans le film « Ressources humaines » de Laurent Cantet, plusieurs conceptions de la fonction « ressources humaines » co-existent dans les pratiques, discours et intentions des acteurs. En vous appuyant sur les 4 rôles de la fonction RH identifiés par D. Ulrich, analysez les conceptions présentes dans le film, en les situant dans leur contexte et en discutant les articulations, complémentarités ou décalages entre ces conceptions.
\end{enumerate}

\section*{Exemples de questions ouvertes sur 6 points}

\begin{enumerate}
\item Selon le documentaire « Call-center, les nouveaux prolétaires », quelle est la conception de la fonction « ressources humaines » qui prévaut dans ce secteur d’activité ? Argumentez soigneusement votre réponse en vous appuyant sur le modèle des 4 rôles d’Ulrich et illustrez-là sur base du documentaire.

\item Les « trente glorieuses » (1944-1974) ont été une période particulièrement propice au développement des instances et des acquis de la concertation sociale en Belgique. Expliquez les grandes orientations et les principales étapes de ce développement. Quelles sont les instances qui en ont résulté et quels ont été les principaux acquis de cette période ?

\item Depuis l’apologie de la culture organisationnelle comme facteur de performance des entreprises par Peters et Waterman, les scientifiques se sont attachés à étudier s’il existait un lien entre la culture d’une organisation et sa performance. Quels sont les principaux résultats de leurs travaux ? En particulier, quels ont été les récents apports du projet GLOBE à cette ligne de recherche ?

\item Le style de leadership contribue à la fois à façonner la culture ambiante dans une organisation et est lui-même influencé par les schémas de pensée et d’action ainsi que par les valeurs jugées acceptables dans une organisation. En vous appuyant sur le modèle de Quinn et sur les théories du leadership, analysez les relations existant entre style de leadership et culture organisationnelle. 

\item Quels sont les facteurs susceptibles de maximiser la performance d’une équipe ? Ces facteurs sont-ils aussi de nature à accroître sa capacité d’innovation ? Argumentez soigneusement votre réponse en vous appuyant sur les théories présentées dans le cadre de ce cours.

\item Le leadership est un concept particulièrement populaire en management mais est-il un concept indispensable? Quels sont les apports mais aussi les limites de l’exercice du leadership ? Argumentez soigneusement votre réponse en vous appuyant sur les théories présentées dans le cadre de ce cours.

\item Dans un call-center, semblable à celui présenté dans le reportage « Call-centers : les nouveaux prolétaires » ; un directeur souhaite améliorer la motivation et la qualité du service offert par son personnel en humanisant le travail. En vous inspirant du modèle d’enrichissement du travail de Hackman et Oldham, quelles pistes d’action concrètes pouvez-vous suggérer à ce directeur ?

\item Le modèle développé par Porter et Lawler (1968) a pour ambition d’articuler les principales théories antérieures de la motivation au travail. Expliquez ce modèle, éventuellement à l’aide d’un schéma au verso de cette feuille, et montrez quelles sont les théories antérieures qu’il synthétise.
\end{enumerate}

\section*{Exemples de questions sur les travaux pratiques}

\begin{enumerate}
\item Citez tout d’abord les différents niveaux d’intelligibilité du modèle d’Ardoino. Ensuite, analyser les problèmes survenus dans l’entreprise MOULTS (vue en séance au travers du cas « Qui a tué la vente ? ») à partir des cinq niveaux d’Ardoino.

\item Expliquez brièvement les différents rôles de la fonction RH (Ulrich). Selon l'article analysé au TP (Karl Bosmans, de l'hôpital d'Hasselt), quel rôle de la fonction RH est mis en avant pour décrire la fonction RH au sein de cet hôpital? Appuyez votre argumentation de divers exemples.

\item A l’aide du modèle des 4 rôles de la fonction RH suggéré par D.Ulrich, décrivez la fonction RH qui prévaut au sein des Cliniques Universitaires St-Luc. Illustrez votre réponse à l’aide de divers exemples issus de l’interview de Mme Christine Thiran, DRH de cet hôpital.

\item Analysez la culture d’entreprise de North American Tool and Die Company (NATD) (vue en séance au travers du cas « Le château de l’affection ») selon le modèle des valeurs concurrentes de Quinn. Quel autre modèle peut être utilisé pour analyser la culture de cette entreprise? Expliquez.

\item Analysez la culture d’entreprise de North American Tool and Die Company (NATD) (vue en séance au travers du cas « Le château de l’affection ») selon la théorie de l’oignon de Schein. Définissez brièvement chaque niveau et appuyez votre argumentation de divers exemples.

\item Faire l'analyse du cas "le château de l'affection" à partir des différentes dimensions de la théorie d'Hofstede. Quel autre modèle peut être utilisé pour analyser la culture de cette entreprise? Expliquez.

\item Présentez et détaillez les différentes étapes de la vie d'un groupe selon Tuckman. Appliquez cette théorie à la dynamique de votre groupe de travail et expliquez les différentes étapes de vie que vous avez connues dans votre propre sous-groupe tout au long de ce quadrimestre ? Expliquez également pourquoi certaines étapes n'ont éventuellement pas eu lieu. Argumentez et illustrez vos propos
\end{enumerate}

\end{document}
