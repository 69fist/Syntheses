\documentclass[fr]{../../../../../../eplexam}

\hypertitle{Management humain}{6}{ECGE}{1321}{2015}{Juin}
{Florian Thuin}
{Nathalie Delobbe}

\section*{Exemples de questions ouvertes sur 8 points}

\begin{enumerate}
    \item Sur base d'un texte annexé (histoire d'un mec qui s'est fait
        virer parce qu'il gérait mal un département d'une société
        composé d'ingénieurs, de techniciens et d'assistants qui se
        mettaient sur la gueule et qui communiquaient mal avec
        l'extérieur, abusait des heures supp' et le mec laissait tout
        faire et il était en retard sur ses documents administratifs et
        dépassait les budgets), faire une analyse de son style de
        leadership et utiliser Ardoino pour expliquer les problèmes,
        illustrer avec des éléments du texte. Expliquer ce qu'est une
        entreprise libérée et quelles sont les conditions de son succès.
    \item Sur base d'un texte annexé (parc d'attraction où il y a des
        problèmes d'absentéisme, pas de motivation, beaucoup de jobs
        étudiants). Sur base de 2 thoéries de la motivation, expliquer
        pourquoi ils ne sont pas motivés, comment les motiver.
    \item Sur base d'un texte année (parc d'attraction, voir question
        précédente) dire quel rôle du modèle d'Ulrich devrait être
        utilisé en priorité si on décidait d'engager un nouveau DRH.
        Donner 2-3 mises en pratiques concrètes que le DRH pourrait
        proposer. Donner 3 obstacles auxquels le DRH pourrait faire
        face. Dire en quoi c'est un challenge pour le DRH (en fonction
        du rôle qu'on a choisi) de devoir faire attention aux différents
        types de contrats.
\end{enumerate}

\section*{Exemples de questions ouvertes sur 6 points}

\begin{enumerate}
    \item Définir les étapes du modèle de vie d'un groupe de Tuckman,
        expliquer en quoi elle consiste et quels sont les enjeux et
        risques de ces étapes. Illustrez à l'aide d'exemples au choix.
    \item Pendant la période de l'administration du personnel et de la
        gestion stratégique des RH, quelles sont les pratiques héritées
        en terme de rémunération ?
    \item Faire le modèle de Sathé. Donner 3 exemples d'applications
        concrètes. Dire si une charte de valeurs est intéressante ou pas
        pour une entreprise.
\end{enumerate}

\section*{Exemples de questions sur les travaux pratiques}

\begin{enumerate}
    \item Sur base du modèle de Schein, définir chaque partie du modèle
        et l'appliquer pour analyser la culture d'Ikea.
    \item Avec les étapes de la vie d'un groupe de Tuckman, expliquez
        les étapes et montrer un exemple via le travail de groupe ou les
        séances de TP.
    \item Faire le modèle d'Ardoino par rapport à notre expéreince du
        travail de groupe.
\end{enumerate}

\end{document}
