\documentclass[a4paper, 11pt, onecolumn]{article}

\usepackage[french]{babel}
\usepackage[utf8]{inputenc}
\usepackage[margin=2cm]{geometry}
\usepackage{framed}
\usepackage{color}

\title{Analyse de données quantitative\\Question 7}
\date{}

\begin{document}

\maketitle

\textit{Quelles sont les caractéristiques importantes du clustering? Expliquez en détail l’algorithme de clustering hiérarchique de Ward et explicitez/dérivez la formule calculant la perte d’inertie inter-classes lorsqu’on regroupe deux nuages de points. Décrivez les autres types de distances qui peuvent être utilisées en clustering hiérarchique.}

\begin{enumerate}
\item \textbf{Contexte}
  On possede un \textbf{ensemble de données} avec 
  
  \begin{enumerate}
    \item {une \textbf{variable dépendante} dont on aimerait pouvoir prévoir l'apparition}
    \item {et des \textbf{variables explicative} qui nous serviront à expliquer (prévoir) la variable dépendante.}
  \end{enumerate}
  
  Cet ensemble est donc \textbf{déscriptif} \textit{puisqu'il décrit une situation observée}, et l'on aimerait qu'il devienne un ensemble\textbf{ predictif} \textit{pour prédire la realisation de situations.}

On va essayer de répartir la variable dépendante dans différentes classes.

  \begin{description}
\begin{framed}
    \item[Une classe] : Groupement d'observation apparaissant sous des
      conditions semblables.
\end{framed}
  \end{description}

\item \textbf{Clustering}

  \begin{description}
    \item[Clustering] : Volonté de regrouper les données en groupes bien séparé (\textit{Clusters}). $\rightarrow$ Trouver des ensembles de données bien séparé est souvent rare
  \end{description}

Il existe deux types de regroupement : 
\begin{itemize}
  \item celles basé sur la proximité (\textit{càd qui effectue
les calculs sur la matrice de distante et/ou similarité})
\item celles basé sur les variables des données (\textit{càd qui effectue les calculs sur les
  données})
  \end{itemize}

  \begin{description}
\begin{framed}
\item[Notion de proximité] : Elle permet le regroupement sur base de similaritées/différences
  entre les différentes données. \\
  \underline{Note:} On utilise une matrice de proximité pour définir les proximités.
\end{framed}
  \end{description}


Il y a 4 etapes dans le clustering:
\begin{enumerate}
\item \textbf{Le "nettoyage" (pre-processing) des données}
Effacer les données redondantes et correlé et effacer les outliers.
Effectuer un PCA (numérique) ou une MCA (categorielle), pour réduire l'espace des données
\item \textbf{Le choix des proximités}
On crée la matrice de distance et de similarité (Dans les cours il définis une 6-7 methodes de calcul des distances et des similarité)
\item \textbf{La validation du clustering} On regarde si il y a vraiment des groupes bien séparé
\item \textbf{L'interpretation du clustering}
\end{enumerate}



\item \textbf{Hierarchy clustering}

  \begin{description}
    \item [Hierrachy clustering] : C'est un algorithme de groupement hiérarchique. Il en existe de deux types
\begin{enumerate}
  \item \textbf{Bottom-up}: On agglomere des données (la plus part des méthodes sont agglomérative)
  \item \textbf{Top-down}: On divise des données en deux groupes
\end{enumerate}
\end{description}

\paragraph{Algorithme agglomératif en 4 étapes:}
\begin{enumerate}
\item Nous possèdons $N$ individus à groupe et nous calculons une matrice de distance (ou de similarité) entre les individus (les individus sont en fait des groupes).
\item On trouve les deux éléments les plus proche et on les merge en un nouvel élément (on possède désormais $n-1$ éléments).
\item On recalcule la matrice
\item On répète (b) et (c) jusqu'a l'agglomération totale
\end{enumerate}
$\rightarrow$ A la fin de ces différentes étapes nous obtenons un \textbf{dendrogramme} 
(\textit{un arbre en gros}) avec différents groupes de taille différentes. 

\paragraph{Différents moyens d'agglomération des groupes (différents calculs de distance)}
\begin{itemize}
\item \textbf{Single linkage}: Distance entre deux groupes est la distance la plus petite
$$d(G_1,G_2) = min(d(x^{(1)}, x^{(2)}))$$
\item \textbf{Complete linkage}: Distance entre deux groupes est la distance la plus grande
$$d(G_1,G_2) = max(d(x^{(1)}, x^{(2)}))$$
\item \textbf{Average linkage}: Moyenne des distance entre les deux groupes
$$d(G_1,G_2) = \frac{1}{|G_1||G_2|}\sum_{x^{(1)} \in G_1\\x^{(2)} \in G_2} d(x^{(1)}, x^{(2)})$$
\end{itemize}

\item \textbf{Ward's criterion}

\begin{framed}
  \textbf{Notion d'inertie} indispensable à la justification de cette méthode. Il existe plusieurs
  inertie :
Soit un groupe $G_k (k: 1\rightarrow K)$
\begin{itemize}
  \item L'inertie dans un groupe (\textcolor{red}{intra-class}): $I_k = \sum_{i \in G_k}d^2(x_i,g_k)$
  \item L'inertie de \textcolor{red}{tout} les groupes: $I_W = \sum_{k=1}^{K} I_k$
  \item L'inertie entre les groupes (\textcolor{red}{inter-class}): $I_B = \sum_{k=1}^{K}|G_k|d^2(g_k,g)$
  \item L'inertie \textcolor{red}{totale}: $I_{tot} = I_W + I_B$
\end{itemize}
\end{framed}

Le critère de Ward calcul l'augmentation d'inertie dans un groupe. Chaque individu au départ est un groupe avec une inertie de 0. Quand on agglomère deux éléments on augmente $I_W$ et l'on réduit $I_B$ ($I_{tot}$ reste constant).

Lorsque l'on passe d'un groupe $P_{k+1}$ à un groupe $P_{k}$ en agglomérant deux groupes on va grouper $A$ et $B$ (avec $A,B \in P_{k+1}$) tel que $|I_W(P_k)-I_W(P_{k+1})|$ (ou $\Delta I_W$) soit minimum. On a aussi 
$$|I_W(P_k)-I_W(P_{k+1})| = \frac{|A||B|}{|A|+|B|}d^2(g_A, g_B)$$
Il reste à prouver que
$$\Delta I_B = \frac{|A||B|}{|A|+|B|}d^2(g_A, g_B)$$

\begin{framed}
Parlons de l'inertie à partir d'un point $z$. Soient $g$ le centre de gravité du nuage de points $x_i$ possédant chacun un poids $n(i)$, et $n_x$ le nombre de $x_i$ différents. Nous avons
\begin{eqnarray}
\sum_{i=1}^{n_x} n(i) ||x_i-z||^2 &=& \sum_{i=1}^{n_x} n(i) ||x_i-g+g-z||^2\\
&=&  \sum_{i=1}^{n_x} n(i) \left[||x_i-g||^2 + ||g-z||^2 - 2(g-z)^T(x_i-g)\right]
\end{eqnarray}
Comme $g$ est le centre de gravité des points $x_i$, nous aurons $\sum_{i=1}^{n_x} (x_i-g) = 0$ (en effet $g=\frac{1}{n_x}\sum_{i=1}^{n_x} x_i$)
\begin{eqnarray}
\sum_{i=1}^{n_x} n(i) ||x_i-z||^2 &=& \sum_{i=1}^{n_x} n(i) \left[||x_i-g||^2 + ||g-z||^2\right]\\
&=& \left[\sum_{i=1}^{n_x} n(i) ||x_i-g||^2\right] + N ||g-z||^2
\end{eqnarray}
Avec $N = \sum_{i=1}^{n_x} n(i)$. Ce calcul nous donne \textit{l'inertie par rapport au centre de gravité global} plus \textit{l'inertie par rapport à z}.
\end{framed}

Pour continuer notre démonstration nous allons supposer que les centres de gravité $g_A$ et $g_B$ (des groupes $A$ et $B$) ont été groupé. Le centre de gravité commun est
$$g_{AB}=\frac{n_Ag_A+n_Bg_B}{n_A+n_B}$$
\begin{framed}
Nous allons calculer l'inertie entre les groupes avant ($k+1$ groupes) le rassemblement et apres ($k$ groupes).
\begin{eqnarray}
I_B(k)&=&(n_A+n_B)||g_{AB}-g||^2+...\\
I_B(k+1)&=&n_A||g_A-g||^2+n_B||g_B-g||^2+...
\end{eqnarray}
Nous savons par la formule du dessus que 
\begin{eqnarray}
\sum_{i=1}^{n_x} n(i) ||x_i-z||^2 &=& \left[\sum_{i=1}^{n_x} n(i) ||x_i-g||^2\right] + N ||g-z||^2\\
n_A||g_A-g||^2+n_B||g_B-g||^2 &=& \left[n_A||g_A-g_{AB}||^2+n_B||g_B-g_{AB}||^2\right]\\
&& + N ||g_{AB}-g||^2\\
&=&  \left[n_A||g_A-g_{AB}||^2+n_B||g_B-g_{AB}||^2\right]\\
&& + (n_A+n_B) ||g_{AB}-g||^2\\
\end{eqnarray}
Ce qui ramené dans la formule de l'intertie nous donne
\begin{eqnarray}
I_B(k+1)&=&n_A||g_A-g_{AB}||^2+n_B||g_B-g_{AB}||^2+ (n_A+n_B) ||g_{AB}-g||^2+...
\end{eqnarray}
\end{framed}

Nous allons terminer notre démonstration en calculant sur base de ces formules la variation de l'intertie entre les groupes

\begin{framed}
\begin{eqnarray}
\Delta I_B &=& I_B(k+1)-I_B(k)\\
&=& n_A||g_A-g_{AB}||^2+n_B||g_B-g_{AB}||^2\\
&=& n_A||g_A-\frac{n_Ag_A+n_Bg_B}{n_A+n_B}||^2+n_B||g_B-\frac{n_Ag_A+n_Bg_B}{n_A+n_B}||^2\\
&=& n_A||\frac{n_Bg_A+n_Bg_B}{n_A+n_B}||^2+n_B||\frac{n_Ag_B+n_Ag_A}{n_A+n_B}||^2\\
&=& \frac{n_An_B}{n_A+n_B}||g_A-g_B||^2\\
&=& \frac{n_An_B}{n_A+n_B}dist^2(g_A,g_B)
\end{eqnarray}
\end{framed}

\end{enumerate}

\end{document}
