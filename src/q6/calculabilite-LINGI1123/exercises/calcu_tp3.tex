\documentclass[a4paper,onecolumn,11pt]{article}

\usepackage[margin=2cm]{geometry}
\usepackage[french]{babel}
\usepackage[utf8]{inputenc}

\author{}
\title{Calculabilité\\Travaux Pratique 3}
\date{}

\begin{document}

\maketitle
\section*{Exercice 1}
\begin{itemize}
\item[a)]

$S_1 = \{n | P_n(0) \; halt\}$

Supposons que $S_1$ est décidable.

Construisons un programme q.

\item[b)]

$S_2 = \{n | P_n(k) = k\}$

On pose que $S_2$ est décidable.

Construisons un programme q, ce programme execute $P_n(k)$ et puis retourne le parametre x.
$$P_q(x) \equiv P_n(k) \; return \; x;$$

Ensuite on peut ecrire

$$P_H(n,k) \equiv return \; P_{S_2}(q)$$

Si x est renvoyé, le programme q appartient à $S_2$ et sinon $P_q(k)$ a bouclé et donc $P_n(k)$.

On a donc trouvé un moyen qui prouve que H est décidable ce qui est une contradiction, il faut donc rejetter l'hypothese de base.

\item[c)]

$S_3 = \{(n,m) | \phi_n = \phi_m\}$

Peut on déterminer la décidabilité de $S_3$, c'est à dire déterminer si $(n,m) \in S_3$. On va supposer que $S_3$ est décidable et utiliser cela pour montrer l'inverse.

Construisons deux programmes.

$$P_{q_1} \equiv \perp$$
$$P_{q_2} \equiv P_n(k)$$

Si on sait déterminer $(q_1, q_2) \in S_3$, on sait dire que $\phi_{q_1} = \phi_{q_2}$.

$P_H(n,k) \equiv return \; 1 - P_{S_3}(q_1, q_2)$

Si $P_{S_3}$ retourne 1, alors $\phi_{q_1} = \phi_{q_2} = \phi_n(k) = \perp$ on sait que le programme $P_n(k)$ ne s'arrete pas et $P_H(n,k)$ retourne 0.

Si $P_{S_3}$ retourne 0, alors $\phi_{q_1} \neq \phi_{q_2} = \phi_n(k) \neq \perp$ on sait que le programme $P_n(k)$ ne s'arrete pas et $P_H(n,k)$ retourne 1.

On sait que H n'est pas décidable, il y a donc une contradiction, li faut donc rejeter l'hypothèse de base.

\item[d)]

$S_4 = \{n | \phi_n$ is a non-total function$\}$

On pose que $S_4$ est décidable.

Construisons le programme q,

$$P_q(x) \equiv P_n(k);$$

Nous avons

$$P_H(n,k) \equiv 1-P_{S_4};$$

Si le programme $P_H(n,k)$ retourne 1, alors $\forall k \; P_n(k) = \; halt$

Si le programme $P_H(n,k)$ retourne 0, alors $\exists k \; P_n(k) = \perp$

\item[e)] Sensiblement le même exercice que le 1.c
\end{itemize}

\end{document}