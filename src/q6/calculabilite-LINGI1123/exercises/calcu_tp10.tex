\documentclass[11pt]{article}

\usepackage{framed}
\usepackage[utf8]{inputenc}
\usepackage[margin=2cm]{geometry}
\usepackage{listings}

\title{Calculabilité\\Travaux pratique 10}
\date{}

\begin{document}

\maketitle

\section{Exercice 1}

En supposant qu'on est un algorithme pour SAT en temps polynomial. Il faut trouver un algorithme qui trouve les valeurs qui rendent la formule vraie.

\begin{framed}
$Assign\{\}$

$\textbf{if}\ !SAT(f)\ \textbf{return}\ null$

$\textbf{else} \forall x \in Vars(f)$

$\{$

$\textbf{if}\ SAT(F|_{Assign\cup \{x, true\}})$

$assign = assign \cup \{(x,true)\}$

$\textbf{else}\ assign = assign \cup \{(x,true)\}$

$\}$

$\textbf{return}\ assign$
\end{framed}

\section{Exercice 2}

\end{document}