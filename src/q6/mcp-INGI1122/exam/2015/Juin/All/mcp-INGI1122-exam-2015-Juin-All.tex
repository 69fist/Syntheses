\documentclass[en]{../../../../../../eplexam}

\hypertitle{mcp-INGI1122}{6}{INGI}{1122}{2015}{Juin}
{Author1\and Author2\and Author3}
{Professor}

\section*{Enoncé}

Soit un tableau de taille $n$ et $i$, $j$ deux indices tels que $0 \le i
< j < n$, et tels que le signe de l'élément i est différent du signe de
l'élément j. Alors, il existe $k$ tel que $i \le k < j$ et tel que
l'élément en position $k$ et $k+1$ sont de signes opposés. Vous devez
trouver un des indices k grâce à une recherche dichotomique.

\section{}

Démontrez que $i \le \left( i+(j-i)/2 \right)$ avec $/$ qui
correspond à la division entière.

\section{}

Donnez les conventions de représentations.

\section{}

Donnez les spécifications de votre méthode.

\section{}

Construisez l'algorithme avec la méthode vue au cours (ITER, INIT, H et
CLOT).

\section{}

Donnez le code Java correspondant à votre algorithme.

\section{}

Preuve classique de votre algorithme avec Hoare etc

\section{}

Trouvez le variant

\end{document}
