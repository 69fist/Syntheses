\documentclass[fr]{../../../eplsummary}

\usepackage{amsfonts}
\usepackage{pict2e}
\usepackage{framed}
\usepackage{float}

\hypertitle{Méthodes de conception de programmes}{6}{INGI}{1122}
{Nicolas Houtain}
{José Vander Meulen}

\section{Raisonnement par récurrence et/ou par induction}

Raisonement par récurrence:
\begin{enumerate}
\item La proposition $P_0$ est vraie
\item quel que soit l'entier n, la proposition $P_{n+1}$ est vraie, chaque fois que la proposition $P_n$ est vraie
\end{enumerate}

\section{Spécifications}

Les spécifications permettent de définir la notion de \textbf{programmes correct}. Un programme est utile parce qu'il permet de fournir une information à son utilisateur et que cet information lui permet de faire des choix.

La \textbf{spécification} va permettre à l'utilisateur d'interpreter le résultat de l'exécution d'un programme.

Différentes étapes
\begin{enumerate}
\item \textbf{Enoncer le problème}: ce que fait le programme
\item \textbf{Définir le problème}: donner un certain nombre de définitions de concepts et de notions. Specifier $\neq$ définir le problème, car le problème pré-existe à toute définition et les définitions ne suffisent pas pour ocmprendre le problème en profondeur.
\item \textbf{Poser le problème}: Le problème indépendament du formalisme ou du language choisis
\item \textbf{Théorie du problème}: Véritable théorie mathématique. Il faut définir les propriétées utilisées.
\item \textbf{Ecrire la spécification}: c'est un texte clair, concis et précis qui doit permettre à n'importe qui de comprendre le problème.
\item \textbf{Décomposition en sous-problème}
\end{enumerate}

\subsection{Spécification de programmes}

La spécification d'une méthode permet de faire complètement abstraction du code de la méthode. Il faut que ce soit un raisonnement purement logique, tout doit être justifié. On spécifie l'effet d'un appel de la méthode. La spécification comprends une hypothèse sur \textbf{tous les contextes possibles}. Il y a donc des aspects statiques (accessiilités et disponibilité des packages) et dynamiques (conditions sur l'environnement de l'appel et sur les objets existant).

\subsection{Convetions de représentation}

On indique comment représenter informatiquement les données et les résultats du programme.

\subsection{Bonne spécification}

Une bonne spécification:
\begin{itemize}
\item Compréhensible
\item Complète
\item Généralité et cas limites: On élimine des cas limites si ca simplifie la spécification et l'implémentation
\item Specification $\neq$ algorithme
\end{itemize}

\section{Décomposition en sous-problèmes}

\subsection{Spécification}

Il existe trois mise en formes pour les spécifictions, voici la version formelle:
\begin{itemize}
\item En-tête
\item Précondition
\item Postcondition
\item Résultat
\end{itemize}
Le but de ces différentes version est la \textbf{compréhensibilité}, il faut donc eviter de compliquer trop.

\subsection{Théorie du problème}

Pour définir la théorie du problème il va d'abord falloir identifier des \textbf{propriétés}.

On peut ensuite définir un \textbf{Théorème} qui explique le problème et en quoi un resultat satisfait le problème.

Après quoi il faudra prouver le dit théorème \textit{a posteriori}.

\subsection{Découpe en sous-problèmes}

Nous allons ensuite découper le problème en sous problème et bien évidemment spécifier ceux-ci. Les spécifications doivent être ecrites de sorte que:
\begin{enumerate}
\item Il doit être possible d'écrire un code correct pour chaque sous-problème, \textbf{indépendamment} du code de la méthode principale et de celui des autres sous-problèmes
\item il doit être possible d'écrire le code de la méthode principale en ne tenant compte que de la spécification du des sous-problèmes
\end{enumerate}

On peut coder les sous-problèmes comme des fragments de code ou comme une méthode.

\section{Construction de fragments itératifs de programmes}

\subsection{Invariant de boucle}

\textbf{Une boucle} est un fragment de programme dont l'exécution comporte (au moins dans certains cas) la ré-exécution, un nombre fixe ou variable de fois, d'un fragment de programme plus petit.

Une boucle se décompose en trois fragments de programmes: \textsc{INIT}, \textsc{ITER},  \textsc{CLOT} et \textsc{H}.
\begin{itemize}
	\item \textsc{INIT}: instructions d'initialisations
	\item \textsc{ITER}: corps de boucle
	\item \textsc{CLOT}: instructions de clôture
	\item \textsc{H}: la condition d'arrêt
\end{itemize}

Deux problèmes peuvent empêcher l'exécution d'une boucle de se terminer normalement:
\begin{enumerate}
	\item \textsc{H} ne devient jamais vrai
	\item Erreur d'éxecution (ex: ArrayOutOfBound)
\end{enumerate}
\textbf{L'invariant de boucle} est une condition sur les valeurs des variables de l'environnement. A chaque execution du fragment itératif l'invariant doit être vrai.

\subsection{Notation de Hoare-Manna}

Notation de Hoare-Manna:
$$\{P\}\ S\ \{Q\}$$
\begin{itemize}
	\item $S$:Fragment de programme don les variables sont définies par un environnement $e$
	\item $Q\ P$: Conditions au sujet des variables de $e$
\end{itemize}
Cette notation est tout simplement une abréviation pour l'affirmation suivante: \textit{Si la condition P est vraie pour les variables de l'environnement e, alors l'exécution de S, pour ces valeurs initiales, ne provoque pas d'erreurs; de plus, elle se termine et elle le fait avec des valeurs des variables qui vérifient la condition Q}.

On peut décomposer la construction de la boucle en quatre points:
\begin{enumerate}
	\item $\{Pre\}\ INIT\ \{INV\}$
	\item $\{INV\ \&\&\ !H\}\ ITER\ \{INV\}$
	\item $\{INV\ \&\&\ H\}\ CLOR\ \{Post\}$
	\item L'éxecution de la boucle se termine pour tout environnement qui respecte $Pre$
\end{enumerate}

\section{Preuves de programmes}

Les preuves de programmes ne sont ni plus ni moins qu'en une reformulation \textit{a posteriori} des raisonnements "constructifs" présentés dans le chapitre précédent.

\begin{itemize}
\item \textbf{Trouver l'invariant}: Il faut construire l'énoncé mathématique (noté $INV$)
\item \textbf{Preuve de $\{Pre\}\ INIT\ \{INV\}$}
\item \textbf{Preuve de $\{INV\ \&\&\ !H\}\ ITER\ \{INV\}$}
\item \textbf{Preuve de $\{INV\ \&\&\ H\}\ CLOR\ \{Post\}$}
\item \textbf{Preuve de terminaison}

\end{itemize}

\end{document}
