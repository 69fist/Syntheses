\documentclass[a4paper,onecolumn,12pt]{IEEETran}

\usepackage[french]{babel}
\usepackage[utf8]{inputenc}
\usepackage{amsfonts}
\usepackage{amsmath}
\usepackage{amssymb}
\usepackage{listings}
\usepackage{pict2e}

\title{Calculabilité\\Théorème du point-fixe\\Démonstration}

\begin{document}

\maketitle

\begin{enumerate}
	\item $h(u,v) = 
			\left\lbrace
			\begin{array}{lcl}
				\phi_{\phi_u(u)}(v) \; si \; \phi_u(u) \neq \perp \\
				\perp \; sinon
			\end{array}\right.$, $h$ est calculable.
	\item $h(u,v) = \phi_{S(u)}(v)$, $S$ est totale calculable (prop. S)
	\item $g(u)=f(S(u))$, $g$ est total calculable
\end{enumerate}

$\exists k$ tel que $\phi_k(u)=g(u)=f(S(u))$ (k calcule une fonction totale donc il ne boucle jamais).

\begin{eqnarray}
	h(k',v) &=& \phi_{S(k')}(v)\\
	h(k',v) &=& \phi_{\phi_{k'}(k')}(v) \\
	&=& \phi_{f(S(k'))}(v)
\end{eqnarray}

La ligne 1 utilise la proposition 2, la ligne 2 utilise la proposition 1 et la dernière ligne utilise la $3^e$ proposition. On voit que la ligne 1 est égale à la ligne 3 ce qui finit la démonstration.

\end{document}