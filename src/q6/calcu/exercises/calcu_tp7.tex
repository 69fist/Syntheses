\documentclass[a4paper,11pt,onecolumn]{article}

\usepackage[french]{babel}
\usepackage[utf8]{inputenc}
\usepackage{pict2e}
\usepackage{amssymb}
\usepackage{xcolor}
\usepackage{framed}
\usepackage[margin=2cm]{geometry}
\usepackage{float}

\definecolor{green}{RGB}{0, 125, 0}

\title{Calculabilité\\Travaux Pratique 7}

\begin{document}

\maketitle

\section{Notions}

Base fonction:
\begin{itemize}
	\item $a() = a \;\;\;\;\;\;\; \mathbb{N}^0 \rightarrow \mathbb{N}$ (const)
	\item $S(n) = n+1 \;\;\;\;\;\;\; \mathbb{N} \rightarrow \mathbb{N}$ (succ)
	\item $p_i^k(x_1, ..., x_i, ..., x_k) = x_i \;\;\;\;\;\;\; \mathbb{N}^k \rightarrow \mathbb{N}$ (projection)
\end{itemize}

Composition: $f(\bar{x}) = g(h_1(\bar{x}), ..., h_m(\bar{x}))$

Primitive recursion:
\begin{itemize}
	\item $f(\bar{x}, 0) = g(\bar{x})$
	\item $f(\bar{x}, n+1) = h(\bar{x}, n, f(\bar{x}, n))$
\end{itemize}


\section{Exercices}
\begin{enumerate}

\item \textit{Ecris la fonction suivante comme une fonctin recursive primitive: $f(n) = \sum_{i=0}^{n}i^2$}

\begin{itemize}
	\item[\textbf{Cas de base}] $$f(0) = g(x)$$
	\item[\textbf{Pas recursif}] $$f(n+1) = h(n,f(n))$$
\end{itemize}

Il ne reste plus qu'a définir les fonctions $f$ et $g$

\begin{itemize}
	\item[] \textit{Fonction constante}: $g(x) = 0$
	\item[] \textit{Composition}: $h(n,a) = add(h_1(n,a),h_2(n,a))$
	\item[] \textit{Projection}: $h_1(n,a) = p_2^2(n,a)=a$
	\item[] \textit{Composition}: $h_2(n,a) = mult(h_1'(n,a), h_2(n,a))$
	\item[] \textit{Composition}: $h_1'(n,a) = S(h"(n,a))$
	\item[] \textit{Prokection}: $h"(n,a)=p_1^2(n,a) = n$
	\item[] $h_2' = h_1'$
\end{itemize}

Et en version beaucoup moins formelle
\begin{eqnarray}
	f(0) &=& 0 \\
	f(n+1) &=& h(n,f(n)) \\
	&=& g(h_1(n), h_2(f(n))) \\
	&=& add(mult(succ(n), succ(n)), f(n))
\end{eqnarray}
Avec à la ligne (3) $h_1(n) = mult(S(n), S(n))$ et $h_2 = p_1^1$

\item \textit{Dans les expressions suivantes, lesquelles sont des expressions lambda?}

Les trois types d'expressions lambda sont:
\begin{enumerate}
	\item $x$ (variable)
	\item $\lambda x B$ (fonction-variable-expression)
	\item $F A$ (fonction-fonction)
\end{enumerate}

\begin{itemize}
	\item[(a)] \textsc{true} $x$
	\item[(b)] \textsc{true} $\lambda xy$
	\item[(c)] \textsc{false} $(\lambda xx)$ Il manque $A$ dans $F A$
	\item[(d)] \textsc{false} $\lambda \lambda xxy$
	\item[(e)] \textsc{true} $((\lambda xxy)y)$
	\item[(f)] \textsc{false} $\lambda x(\lambda xy)$
	\item[(g)] \textsc{false} $((\lambda xy)((x \lambda zx) \lambda xz))$
\end{itemize}

\item \textit{Le lambda calcul peut être utilisé pour des opérations booléenne. $true$ est représenté par $\lambda a \lambda ba$, et $false$ est représenté par $\lambda a \lambda bb$. L'operation $And(X,Y)$ est représenté par $((XY)\lambda a \lambda bb)$. L'operation $Not(X)$ est représenté par $\lambda a \lambda b ((Xb)a)$}.
\begin{itemize}
	\item[(a)] \textit{Ecrivez l'expression lambda pour $And(true, true)$, $And(true, false)$, $Not(true)$ et réduisez les.}
	\begin{eqnarray}
		And(true, true) &=& ((\color{red}\lambda a \color{blue}\lambda ba \color{green}\lambda a \lambda ba \color{black}) \lambda a \lambda bb)\\
		&=& (\color{red}\lambda c \color{blue}\lambda a \lambda ba \color{green}\lambda a \lambda bb\color{black})\\
		&=& \lambda a \lambda ba \\
		&=& true
	\end{eqnarray}
	A la ligne (6) on a renommé $b$ par $c$. On remplace les occurences de \color{red}rouge \color{black} dans \color{blue}bleu \color{black} par \color{green}vert\color{black}.
	\begin{eqnarray}
		And(true, false) &=&  ((\color{red}\lambda a \color{blue}\lambda ba \color{green}\lambda a \lambda bb \color{black}) \lambda a \lambda bb)\\
		&=& (\color{red}\lambda c \color{blue}\lambda a \lambda bb \color{green}\lambda a \lambda bb\color{black})\\
		&=& \lambda a \lambda bb \\
		&=& false
	\end{eqnarray}
	\begin{eqnarray}
		Not(true) &=& \lambda a \lambda b ((\color{red}\lambda a \color{blue}\lambda ba \color{green}b\color{black}) a)\\
		&=& \lambda a \lambda b (\color{red}\lambda b\color{blue}b\color{green}a\color{black})\\
		&=& \lambda a \lambda ba\\
		&=& true\\
		&=& \lambda a \lambda b (\color{red}\lambda c\color{blue}b\color{green}a\color{black})\\
		&=& \lambda a \lambda bb \\
		&=& false
	\end{eqnarray}
	Cet exemple nous montre l'importance du renomage pour avoir le bon resultat, de plus on commence par executer l'interieur de la parenthese.
	\item[(b)]\textit{A partir de votre expérience, essayez d'exprimer la fonction lambda correspondant à $Or(X,Y)$}.
	\begin{eqnarray*}
		True &=& ((\lambda a \lambda ba X)Y)\\
		&=& (\lambda c XY) \\
		&=& X
	\end{eqnarray*}	
	\begin{eqnarray*}
		False &=& ((\lambda a \lambda bb X)Y)\\
		&=& (\lambda cc X) \\
		&=& Y
	\end{eqnarray*}
	\begin{eqnarray*}
		OR(X,Y) &=& ((X true) Y) \\
		&=& ((X \lambda a \lambda ba)Y)
	\end{eqnarray*}
	\textbf{XOR pour l'examen?}
\end{itemize}
	\item \textit{Vrai ou faux?}
	\begin{itemize}
		\item[(a)] \textit{La fonction $halt$ est calculable pour la récursion primitive.}
		
\textsc{true}, seulement les fonctions totale (pas de boucle infini)

		\item[(b)] \textit{La fonction $interpret$ peut être représenté par une fonction primitive recursivepour la fonction primitive recursive}

\textsc{false}, Hoare-Allyson

		\item[(c)] \textit{Réduire une expression lambda menne toujours à réduire cet expression lambda}

\textsc{false}, voir cours $(\lambda x (xx) \lambda x (xx))$

		\item[(d)] \textit{Si réduire deux séquences d'expressions lambda conduit à une forme réduire, elles sont équivalentes}

\textsc{true}, voir cours Théorème Church-Rosser
	
		\item[(e)] \textit{L'algorithme de réduction d'une expression lambda peut être représenté par une expression lambda}

\textsc{true}, modèle complet de la calculabilité a un interpret
	\end{itemize}
\end{enumerate}

\end{document}