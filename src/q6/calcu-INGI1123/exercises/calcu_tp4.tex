\documentclass[a4paper,onecolumn,11pt]{article}

\usepackage[french]{babel}
\usepackage[margin=2cm]{geometry}
\usepackage[utf8]{inputenc}
\usepackage{amsfonts}
\usepackage{amsmath}
\usepackage{amssymb}
\usepackage{listings}
\usepackage{pict2e}

\title{Calculabilité\\Travaux Pratique 4}
\date{}

\begin{document}

\maketitle

\textsc{Rappel:} Rice
$$ A \subseteq \mathbb{N}\ \textbf{si}\ A = \emptyset\ \textbf{ou}\ A = \mathbb{N}\ \textbf{ou}\ A\ non\ recursif\ \textbf{alors}\  \forall i \in A, j \in \bar{A} : \varphi_i \neq \varphi_j$$

\section*{Exercice 1}

\begin{itemize}
	\item[(a)] Oui dépend de $f$
	
	Soit $f$ n'est pas calculable $\Rightarrow A$ est récursif $(A = \emptyset)$
	
	Soit $f$ est calculable
	
	$\Rightarrow A \neq 0, A \neq \mathbb{N}$ $(\exists$ des programme qui ne calcule pas $f)$
	
	$\forall i \in A, \forall j \in \bar{A}$, on a $\varphi_i \neq \varphi_j$ (par définition $A$ est l'ensemble des programme qui calcule $f$).
	
	Donc par le théorème on a que $A$ n'est pas récursif.
	
	\item[(b)] Soit $A$ l'ensemble des programmes qui retourne 0 comme résultat pour une valeur $x_0$. On sait que $A \neq \emptyset$, car il y a au moins un programme dans A (ex: $P_Y(x) \equiv$ return 0). Et l'on sait que  $A \neq \mathbb{N}$ car il y a au moins un programme dans $\overline{A}$ (ex: $P_n(x) \equiv$ return 42)
	$$A \neq \emptyset$$
	$$A \neq \mathbb{N}$$
	Par définition de l'ensemble A, on sait que tout les programmes de $A$, il existe une valeur $x_0$ tel que $\varphi_i(x_0) = 0$. On sait aussi que dans $\overline{A}$ l'ensemble des programme vont retourner des valeurs différentes de 0.
	$$\forall i \in A : \exists x_0 : \varphi_i(x_0) = 0$$
	$$\forall j \in \bar{A} : \forall x : \varphi_j(x) \neq 0$$
	
	Par le théorème $A$ n'est pas récursif.
	
	\item[(c)] $A \neq \emptyset$ (ex: $P_Y(0) \equiv$ return 0)
	
	$A \neq \mathbb{N}$ (ex: $P_n(0) \equiv$ 1001 instructions + return 0)
	
	$\forall i \in A : \exists \varphi_i(0) = 0$
	
	$\forall j \in \bar{A} : \exists \varphi_j(0) = 0$
	
	$P_A(n) \equiv$
	
	\textsc{for} in n $[0;1000]$ executeOneInstruction(n,0);
	 
	\textsc{if}(terminated(n)) return 1;
	 
	\textsc{else} return 0;
	
	Comme on peut écrire un programme qui calcule si un programme appartient à $A$ on peut dire que $A$ est récursif.
	
	\item[(d)] $A \neq \emptyset$ ex: $P_i(x) \equiv$ return 42;
	
	$A \neq \mathbb{N}$ ex: $P_j(x) \equiv$ while true;
	
	$\forall i \in A \; \exists x_1, x_2 : x_1 \neq x_2 \land \phi_i(x_1) \neq \perp \; \phi_j(x_2) \neq \perp$
	
	$\forall j \in \bar{A} \; \nexists x_1, x_2 : x_1 \neq x_2 \land \phi_i(x_1) \neq \perp \; \phi_j(x_2) \neq \perp$
	
	$\forall j \in \bar{A} \; \forall x_1, x_2 : x_1 \neq x_2 \land \phi_j(x_1) = \perp \; \phi_i(x_2) = \perp$
	
	$\forall i \in A, j \in \bar{A}: \; \exists x_1, x_2: x_1 \neq x_2 \land [\phi_i(x_1) \neq \phi_j(x_1) \lor \phi_i(x_2) \neq phi_j(x_2)]$
	
	$\forall i \in A, j \in \bar{A} \; \phi_i \neq \phi_j \Rightarrow$ A n'est pas récursif.
	
\end{itemize}

\section*{Exercice 4}

\textbf{L'exercice suivant n'a pas été fait en séance et il se peut dès lors que la résolution soit incomplète, ou incorrect}

Soit $A \subseteq N$, d'après le théorème de Rice, si $A$ est récursif et $A \neq \emptyset$ et $A\neq \mathbb{N}$ alors nous avons $\exists i \in \overline{A}$ et $\exists j \in A$ tel que $\varphi_j = \varphi_i$

Supposons que  $\forall i \in \overline{A}$ et $\forall j \in A$ nous avons $\varphi_j \neq \varphi_i$:

D'après le théorème du points fixe, soient $x \geq 0$ et $f$ une fonction totale calculable, $\exists k$ tel que $P_k(x) = P_{f(k)}(x)$.

Posons donc la fonction totale calculable $f =
\left\{
\begin{array}{ll}
i & si\ x \in A\\
j & si\ x \in \overline{A}
\end{array}
\right.
$ avec $i \in \overline{A}$ et $j \in A$ (nous pouvons dire que cette fonction est calculable car $A$ est récursif).

Il existe donc $k$ tel que $P_k(x) = P_{f(k)}(x)$, dès lors deux cas se présentent
\subsection*{$k \in A$}

Si $k \in A$ alors on a $P_k(x) = P_{f(k)}(x) = P_i(x)$, dès lors $\varphi_k = \varphi_i$ et c'est une contradiction avec notre supposition car $k \in A$ et $i\in \overline{A}$

\subsection*{$k \in \overline{A}$}

Si $k \in \overline{A}$ alors on a $P_k(x) = P_{f(k)}(x) = P_j(x)$, dès lors $\varphi_k = \varphi_j$ et c'est une contradiction avec notre supposition car $k \in \overline{A}$ et $j\in A$

Nous devons rejeter l'hypothese de base

\section*{Exercice 5}

\begin{itemize}
	\item[a)] Faux, propriété sur les fonctions calculé ne sont pas décidable (on peut le montrer avec une réduction avec HALT).
	\item[b)] Faux (même justification)
	\item[c)] Vrai, cas particulier du théorème S-m-n
\end{itemize}

\end{document}