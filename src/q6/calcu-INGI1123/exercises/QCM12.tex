\begin{longtable}{p{13cm}|l}
    Si A est dans $DTIME(n^{2})$, alors A est dans $DSPACE(n^{2})$ & VRAI\\
    %    \textit{Justification:} & \\
    \hline
    Si A est dans $NTIME(n^{2})$, alors $A$ est dans $DTIME(n^{2})$& FAUX \\
    %    \textit{Justification:} & \\
    \hline
    Si $A \in NTIME(f)$ alors $A\in DTIME(f)$ & FAUX \\
    %    \textit{Justification:} & \\
    \hline
    Si $A \in NTIME(n^{2})$ alors $A \in NP$ & VRAI \\
    %    \textit{Justification:} & \\
    \hline
    S'il existe un algorithme Java de complexité temporelle $O(n^{2})$ décidant l'ensemble A, alors il existe une machine de Turing de complexité temporelle $O(n^{2})$ décidant l'ensemble A.& FAUX \\
    \textit{Justification:} Pas de transfert de complexité entre formalismes. On peut seulement dire que la MT va résoudre le problème en un temps polynomial. On ne peut pas assurer $O(n^{2})$ & \\
    \hline
    Le choix d'un modèle de calculabilité n'influence pas la classe $DTIME(n^{2})$ & VRAI \\
    %    \textit{Justification:} & \\
    \hline
    Si $A \leq_{p} B$ alors $A \leq_{a} B$. & VRAI \\
    \textit{Justification:} Car la réduction polynomiale, en plus d’avoir une réduction algorithmique, on a aussi une information sur la complexité. & \\
    \hline
    La classe P est strictement inclue dans la classe EXPTIME. & VRAI \\
    \textit{Justification:} Aucun problème polynomial n’est plus dur qu’un problème exponentiel. & \\
    \hline
    Tout problème calculable est au moins dans exptime. & FAUX \\
    %    \textit{Justification:} & \\
    \hline
    Un problème NP-complet peut toujours être décidé par un programme non déterministe de complexité polynomiale. & VRAI \\
    \textit{Justification:} Définition de NP & \\
    \hline
    Pour déterminer si un problème A est NP-complet, il suffit de déterminer que A est polynomialement réductible un problème NP complet connu ( EG sat).& FAUX \\
    \textit{Justification:} Il faut être dans NP, on ne dit pas qu’il est dans NP. De plus, il faut être plus difficile qu’un problème qui est NP-complet. & \\
    \hline
    Le problème du voyageur de commerce est NP-complet. & VRAI \\
    %    \textit{Justification:} & \\
    \hline

\end{longtable}