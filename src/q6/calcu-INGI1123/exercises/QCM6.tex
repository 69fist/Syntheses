\begin{longtable}{p{13cm}|l}
    La propriété S-m-n affirme que tout numéro de programme calculable peut être transformé en un numéro équivalent, mais avec moins de paramètres & FAUX \\
    %    \textit{Justification:} & \\
     \hline
    
    Les propriétés S-m-n et S sont équivalentes & FAUX \\
    %    \textit{Justification:} & \\
     \hline
    
    Tous les langages de programmation satisfont la propriété S-m-n & VRAI \\
    %    \textit{Justification:} & \\
     \hline
     
     Le théorème du point fixe est une conséquence du théorème de Rice & FAUX\\
     \textit{Justification:} L'inverse est vrai. & \\
     \hline
     Si deux programmes P1 et P2 calculent la même fonction, alors il existe un transformateur $f$ de programmes ($f$ fonction totale calculable), tel que f(P1)=P2. & VRAI \\
     \textit{Justification:} Rien à voir avec le théorème du point fixe. La fonction constance P2 peut servir de $f$. & \\
     \hline
     
     Si $f$ est un transformateur de programmes ($f$ fonction totale calculable), alors il existe deux programmes P1 et P2 tels que f(P1)=P2 ainsi que P1 et P2 calculent la même fonction. & VRAI \\
     \textit{Justification:} Point fixe & \\
     \hline
     
     Si $f$ est un transformateur de programmes ($f$ fonction totale calculable), alors il existe deux programmes P1 et P2 tels que f(P1)=P2 ainsi que P1 et P2 calculent la même fonction totale & FAUX \\
     %    \textit{Justification:} & \\
     \hline
     
     Le théorème du point fixe permet de démontrer que la fonction HALT est non calculable & VRAI \\
     \textit{Justification:} Il peut tout démontrer : K est non récursif, HALT non-récursif & \\
     \hline
     
     L'ensemble des nombres réels calculables est énumérable & VRAI \\
     \textit{Justification:} Autant que de programmes & \\
    
%     Le théorème du point fixe est une conséquence du théorème de Rice & FAUX \\
%     \textit{Justification:} L'inverse est vrai & \\ \hline
%     Si deux programmes $P1$ et $P2$ calculent la même fonction, alors il existe un transformateur $f$ de programmes ($f$ fonction totale calculable), tel que $f(P1)=P2$. & \\
%     \textit{Justification:} Rien à voir avec le théorème du point fixe. La fonction constante P2 peut servir de f. & \\ \hline
%     Si f est un transformateur de programmes ($f$ fonction totale calculable), alors il existe deux programmes $P1$ et $P2$ tels que $f(P1) = P2$ ainsi que $P1$ et $P2$ calculent la même fonction. & VRAI \\
%     %    \textit{Justification:} & \\
%      \hline
%     Si f est un transformateur de programmes ($f$ fonction totale calculable), alors il existe deux programmes $P1$ et $P2$ tels que $f(P1) = P2$ ainsi que $P1$ et $P2$ calculent la même fonction totale. & FAUX \\    %    \textit{Justification:} & \\
%      \hline
%     Le théorème du point fixe permet de démontrer que la fonction HALT est non calculable & VRAI \\
%     \textit{Justification:} Il peut tout démontrer : $K$ est non récursif, HALT est non récursif & \\
%      \hline
%     L'ensemble des nombres réels calculables est énumérable & VRAI \\
%     %    \textit{Justification:} & \\
\end{longtable}