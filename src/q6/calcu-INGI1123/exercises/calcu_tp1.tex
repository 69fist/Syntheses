\documentclass[a4paper,onecolumn,11pt]{article}

\usepackage[french]{babel}
\usepackage[utf8]{inputenc}
\usepackage[margin=2cm]{geometry}
\usepackage{amsfonts}
\usepackage{listings}
\usepackage{pict2e}
\usepackage{framed}
\usepackage[margin=2cm]{geometry}
\usepackage{float}

\title{Calculabilité\\Travaux Pratique 1}
\date{}

\begin{document}

\maketitle

\section*{Exercice 1}

$A_i$ un ensemble dénombrable

\begin{itemize}
\item[a)] \textit{Montrer que $A_1*A_2$ est dénombrable}

\begin{picture}(120,120)
	\setlength{\unitlength}{20px}
	\put(1,0){\line(0,1){6}}
	\put(0,5){\line(1,0){6}}
	\put(0,2.5){$A_2$}
	\put(3,5.5){$A_1$}
	\put(1.1,0.2){$A_{\infty 1}$}
	\put(1.3,1.1){$...$}
	\put(1.1,2.2){$A_{31}$}
	\put(1.1,3.3){$A_{21}$}
	\put(1.1,4.4){$A_{11}$}
	\put(2.2,3.3){$A_{22}$}
	\put(2.2,4.4){$A_{12}$}
	\put(3.3,4.4){$A_{13}$}
	\put(4.4,4.4){$...$}
	\put(5.0,4.4){$A_{1\infty}$}
	\put(5.0,0.2){$A_{\infty\infty}$}
	\put(2.5,0.4){$..............$}
	\put(5.5,4.0){$.$}
	\put(5.5,3.75){$.$}
	\put(5.5,3.50){$.$}
	\put(5.5,3.25){$.$}
	\put(5.5,3.0){$.$}
	\put(5.5,2.75){$.$}
	\put(5.5,2.50){$.$}
	\put(5.5,2.25){$.$}
	\put(5.5,2.00){$.$}
	\put(5.5,1.75){$.$}
	\put(5.5,1.50){$.$}
	\put(5.5,1.25){$.$}
	\put(5.5,1.00){$.$}
	\put(5.5,0.75){$.$}
\end{picture}

La solution est de suivre le chemin $A_{11}$, $A_{12}$, $A_{21}$, $A_{31}$, $A_{22}$,etc. En suivant ainsi les diagonales on peut lier tout les éléments entre eux. $A_1*A_2$ est donc énumérable. 

\item[b)]\textit{Montrer que $\cup_{i=0}^{\infty}A_i$ est dénombrable}

Le raisonnement est le même que pour l'exercice précédent.

\end{itemize}


\section*{Exercice 2}

Dans les ensembles suivants, lesquels sont dénombrables, lesquels ne le sont pas? Démontrez vos réponses.

\begin{itemize}
\item[a)]\textit{$\{x \in \mathbb{N}|x$ is prime$\}$}

Cet ensemble est énumerable car $\mathbb{N}$ est énumérable.

\item[b)]\textit{$\mathbb{Q}$}

Cet ensemble est enumerable car on peut représenter ces éléments comme des couples d'entier ($\frac{n}{d}$).

\item[c)]\textit{$[0,1] \cap \mathbb{J}$ (i.e. $\mathbb{J}$ représente les nombres irrationnels entre 0 et 1).}

Commençons par poser l'hypothèse selon laquelle $\mathbb{J}$ est énumérable. On sait que $\mathbb{R} = \mathbb{Q} \cup \mathbb{J}$. Or on sait que $\mathbb{Q}$ est énumérable et par l'hypothèse posé au début de cette démonstration, on suppose que $\mathbb{J}$ est énumérable. En conclusion on aurait que $\mathbb{R}$ est énumérable, ce qui est absurde. On doit donc rejeter l'hypothèse de base.

\item[d)]\textit{$X^*$ avec $X$ un ensemble énumérable ($X^*$ est l'ensemble de tout les sous-ensembles finis de $X$).}

Réecrivons dans un premier temps l'ensemble $X*=\cup^{\infty}_{i=0}X^i$ tel que $X^i = \{$sous ensemble de taille $i$ de $X\}$. Démontrons maintenant que l'ensemble est énumérable par induction.
\begin{enumerate}
	\item \textsc{Cas de base:} $X^1 = X$ (énumérable)
	\item \textsc{Hypothèse:} $X^i$ est énumérable
	\item \textsc{Pas inductif:} $X^{i+1} = X^i*X$ est énumérable
\end{enumerate}
En conclusion $X^*$ est énumérable.

\item[e)]\textit{\textbf{L'ensemble de toute les fonctions des $\mathbb{N}$ dans $\{0,1\}$}}

\begin{eqnarray*}
	f_p(x) &=& 1 \; si \; f_x(x)=0\\
	&=& 0 \; sinon \\
	f(p) &=& ?
\end{eqnarray*}

On va utiliser la diagonalisation de Cantor pour montrer que ca n'est pas énumérable.

\textit{(Voir scan des notes pour le schéma, j'en ai marre de faire le dessin).}

\end{itemize}

\section*{Exercice 3}

Ecris un algorithme en pseudocode qui liste ces ensembles

\begin{itemize}
	\item[a)]\textit{$\{x\in\mathbb{N}|x$ is prime$\}$}
	
	\begin{lstlisting}[frame=single]
print 0;
for i in [i;->]
  print i;
  print -i;
	\end{lstlisting}	
	
	\item[b)]\textit{$\{a,b,c\}^*$ (tout les mots formés avec l'alphabet $\{a,b,c\}$}
	
	\begin{lstlisting}[frame=single]
def gen(size, prefix)
  if size == 0 : print prefix;
  else
    gen(size-1, prefix+"a");
    gen(size-1, prefix+"b");
    gen(size-1, prefix+"c");
for i in [0;->]
  gen(i,"");
	\end{lstlisting}	
\end{itemize}

\section*{Exercice 5}
Vrai ou faux?
\begin{itemize}
	\item[a)]\textit{L'ensemble des fonctions de $\mathbb{N}$ dans $\mathbb{N}$ on la même cardinalité que $\mathbb{N}$.}

	Faux, voir 2.e.	
	
	\item[b)]\textit{L'ensemble des sous-ensembles fini de $\mathbb{N}$ est énumérable.}

	Vrai, voir 2.d.		
	
	\item[c)]\textit{$X$ est enumérable si $X^*$ est énumérable.}
	
	Vrai, voir 2.d.		
	
	\item[d)]\textit{L'ensemble des séquences finies d'un alphabet énumérable infini est énumérable.}

	Vrai, voir 2.d.		
		
	\item[e)]\textit{L'ensemble des séquences infinies d'un alphabet finis est énumérable.}
	
	Faux
\end{itemize}

\end{document}