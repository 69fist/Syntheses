\documentclass[a4paper, 12pt]{article}

\usepackage[francais]{babel} % Document en français
\usepackage[utf8]{inputenc} % Document au format UTF8
\usepackage[T1]{fontenc} % Suppression d'un warning pour la langue francaise
\usepackage[left=2.2cm, right=2.2cm, top=2.5cm, bottom=2.5cm]{geometry} % Mise en page
\usepackage{amsmath}

\author{Florian \bsc{Thuin}}
\title{Problem Set 11 : NP-completeness}
\date{11 mai 2015}

\begin{document}
\maketitle

\tableofcontents

\section*{Rappel}

A $\leq_{a}$ B c'est la réduction algorithmique. Lorsqu'on dit qu'un
problème A peut se réduire algorithmiquement à un problème alors s'il
existe un programme pour résoudre B, il existe un programme au moins
aussi performant pour résoudre A. On s'en fout de comment il est fait,
on s'intéresse uniquement à la calculabilité d'un problème (décidable ou
indécidable). B recursive $\Rightarrow$ A recursive.

A $\leq_{f}$ B c'est la réduction fonctionnelle. On traduit le problème
A en un problème B. On utilise une seule fois l'algorithme de B pour
résoudre A (et seulement à la fin). C'est plus contraignant que la
réduction algorithmique mais ça permet de tirer des conclusions de
calculabilité et de \dots ?

A $\leq_{p}$ B c'est la réduction polynomiale. C'est une réduction
fonctionnelle mais l'algorithme qui permet de résoudre A en le
transformant en instance de B doit être en temps polynomial.

\section{Exercice 1}

\begin{tabular}{ll}
    $\exists P_{SAT}(F) : \phi_{SAT} (F)$ = & $1$ if $\exists x_{1}, x_{2},\dots, x_{F} : F \mid_{x_{1}, x_{2},\dots, x_{\mid F\mid}}$ = TRUE \\
    & 0 otherwise
\end{tabular}

$P_{SAT} \in P$ 

\begin{tabular}{ll}
    P(F) tripleegal & Assign = \{\} \\
    & if not SAT(F) return NULL\\
    & else $\forall x \in \forall ans(F) :$\\
    & \textbf{if} $SAT(F \mid_{Assign \cup \{(x, true)\})} == 1$ \\
    & Assign = Assign $\cup$ \{(x,true)\} \\
    & \textbf{else} Assign = Assign $\cup$ \{(x,false)\}
\end{tabular}

Il y a exactement $n = \mid Vars(F)\mid$ appels à SAT()
$SAT \in P \Rightarrow P_{F} \in P$ return Assign

\section{Exercice 2}

HC $\in$ NP-C

$\Rightarrow$ HC $\in$ NP ; $\forall$ A $\in$ NP : A $\leq_p$ HC

P $\subseteq$ NP

Rappel : P $\neq$ NP $\Rightarrow$ P $\cap$ NP = $\emptyset$ est FAUX.

\paragraph{(a)}

Si $P \neq NP$, VRAI (si on souhaite un algorithme déterministe).

Si $P = NP$, FAUX. Puisque par définition H $\in$ NP-C, si NP est
équivalent à P alors il existe un algorithme qui résoud le problème en
un temps polynomial.

\paragraph{(b)}
Si $P \neq NP$, FAUX.

Si $P = NP$, VRAI. Il ne reste plus qu'à le trouver.

\paragraph{(c)}
Si $P \neq NP$, VRAI.

Si $P = NP$, FAUX.
\paragraph{(d)}
Si $P \neq NP$, VRAI.

Si $P = NP$, VRAI.
\paragraph{(e)}

Si $P \neq NP$, FAUX : pas tous (mais au moins 1).

Si $P = NP$, FAUX.


\section{Exercice 3}

We know HC $\in$ NP-C. Prove that TS $\in$ NP-C :

\begin{enumerate}
    \item Prove TS $\in$ NP (in the slides)
    \item Prove $\forall A \in NP : A \leq_{p} TS$ knowing that $\forall
        A \in NP : A \leq_{p} HC $
        \subitem $\Rightarrow$ Prove $HC \leq_{p} TS$
\end{enumerate}

La première partie est démontrée dans les slides.

$\forall A \in NP : A \leq_{p} HC \leq_{p} TS$ 

On doit montrer $\exists f$ total calculable :

G $\in$ HC $\Leftrightarrow f(G) \in TS$

G $\in$ HC $\rightarrow) G' = f(G) \rightarrow G' \in? TS$ 

\begin{tabular}{ll}
    f(G) = & \dots\\
    & \dots \\
    & \dots \\
\end{tabular}

$\exists P_{f}$ polynomial time : $\rho_{P_{f}} = f$

TS(X) : $\exists$ cycle $c$ : cost(c) $\leq$ x. On prend un graphe et on
met chacune des arêtes à un poids de 1. On a un graphe G=(V,E). Pour être
sûr qu'il n'y a pas de cycle, la somme doit être égale au nombre
d'arêtes : $TS(\mid V\mid)$

$G \in HC \Leftrightarrow G' \in TS(X)$

$f(G) = G'$

$f(G)$ = mettre un poids de 1 à toutes les arêtes de G.

$G=(V,E)$

$P_{HC}(G)$ tripleegal $P_{TS}(f(G), n)$ where $n = \mid V\mid$ 

\section{Exercice 4}

X $\leq_{p}$ Y

La NP-complétude dit : 

\begin{itemize}
    \item Y $\in$ NP-C : $\forall$ A $\in$ NP : A $\leq_{p}$ Y
    \item Y $\in$ NP
\end{itemize}

\paragraph{(a)}

FAUX

X $\in$ NP-C $\Rightarrow$ Y $\in$ NP-C est faux. 
X $\in$ NP-C $\Rightarrow$ Y $\in$ NP-Difficile est vrai. 

\paragraph{(b)}

FAUX

Y $\in$ NP-C $\and$ X $\in$ NP $\Rightarrow$ X $\in$ NP-C est faux.

\paragraph{(c)}

VRAI

X $\in$ NP-C $\and$ Y $\in$ NP $\Rightarrow$ Y $\in$ NP-C est vrai.

\paragraph{(d)}

FAUX
X $\in$ P $\Rightarrow$ Y $\in$ P est faux

\paragraph{(e)}

Y $\in$ P $\Rightarrow$ X $\in$ P

\end{document}
