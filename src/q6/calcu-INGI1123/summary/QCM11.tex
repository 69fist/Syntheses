\begin{tabular}{p{13cm}|l}
    Si la complexité temporelle d'un algorithme est $O(n^{2})$, alors elle est aussi $O(n^{3})$ & VRAI \\
    %    \textit{Justification:} & \\
    \hline
    Si la complexité spatiale d'un algorithme est $O(n^{3})$, alors elle ne peut pas être $O(n^{2})$  & FAUX \\
    %    \textit{Justification:} & \\
    \hline
    Un algorithme de complexité $O(n^{2})$ est toujours plus rapide qu'un algorithme de complexité $O(n^{3})$ & FAUX \\
    %    \textit{Justification:} & \\
    \hline
    Un problème qui peut être résolu par un algorithme (de complexité) polynomial est pratiquement faisable. & VRAI \\
    %    \textit{Justification:} & \\
    \hline
    Un problème qui peut être résolu par un algorithme (de complexité) exponentiel est pratique infaisable (intrinsèquement complexe) & VRAI \\
    %    \textit{Justification:} & \\
    \hline
    Si un problème est intrinsèquement complexe en MT alors il est aussi intrinsèquement complexe pour le langage Java.& VRAI\\
    %    \textit{Justification:} & \\
    \hline
    Un ensemble est a-réductible (algorithmiquement réductible) à son complément.& VRAI\\
    %    \textit{Justification:} & \\
    \hline
    Tout ensemble récursivement énumérable est a-réductible à HALT. & VRAI\\
    %    \textit{Justification:} & \\
    \hline
    Un ensemble est f-réductible (fonctionnellement réductible) à son complément. & FAUX \\
    %    \textit{Justification:} & \\
    \hline
    Si A peut être décidé par un algorithme polynomial et si B est f-réductible à A, alors B peut être décidé par un algorithme polynomial. & FAUX \\
    \textit{Justification:} Parce que ce que la f-réductibilité veut dire : on peut calculer B à partir de A et on a une information sur la dernière instruction, qui est polynomiale. On ne connait pas le reste du code (f (b)). &\\
\end{tabular}