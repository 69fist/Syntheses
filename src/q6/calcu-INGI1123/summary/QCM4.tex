\begin{longtable}{p{13cm}|l}
    L'ensemble HALT est récursivment énumérable & VRAI \\
    %    \textit{Justification:} & \\
    \hline
    Une fonction calculable peut être calculée par une infinité de fonctions & VRAI \\
    %    \textit{Justification:} & \\
    \hline
    Le complément de l'ensemble K est récursivment énumérable & FAUX \\
    %    \textit{Justification:} & \\
    \hline
    Il existe des ensembles non récursivment énumérables & VRAI \\
    \textit{Justification:} Le complément de K & \\
    \hline
    Tout ensemble est récursivement énumérable ou co-récursivement énumérable & FAUX \\
    %    \textit{Justification:} & \\
    \hline
    Soit la fonction $f(i) = 1$ si $\varphi_{i}(i) \neq \bot$, $f(i) = 0$ sinon. La fonction $f$ est calculable. & FAUX \\
    \textit{Justification:} Sinon, on pourrait décider HALT. & \\
    \hline
    Soit la fonction $f(i) = 1$ si $\varphi_{i}(i) \neq \bot$, $f(i) = 0$ sinon. Le domaine de $f$ est récursif. & VRAI \\
    \textit{Justification:} Nous travaillons avec un entier en input et en output, donc la fonction est $\mathbb{N} \rightarrow \mathbb{N}$ est ce domaine est récursif. & \\
    \hline
    Soit la fonction $f(i) = 1$ si $\varphi_{i}(i) \neq \bot$, $f(i) = 0$ sinon. L'image de $f$ est un ensemble récursif. & VRAI \\
    \textit{Justification:} $\{0,1\}$ ne contient que deux entiers et est donc récursif &  \\
    \hline
    $\varphi_{i}(i)$ dénote la fonction numéro $i$ (selon l'énumération choisie pour les fonctions) & FAUX \\
    \textit{Justification:} $i$ est un numéro de programme & \\
    \hline
    Il existe un langage de programmation (non-trivial) dans lequel la fonction HALT est calculable & VRAI \\
    %    \textit{Justification:} & \\
    \hline
    Il existe un langage de programmation (non-trivial) dans lequel on peut programmer la fonction HALT ainsi que l'interpréteur de ce langage. & FAUX \\
    %    \textit{Justification:} & \\
    \hline
    Si un langage de programmation (non-trivial) permet de programmer son interpréteur, alors la fonction HALT est calculable dans ce langage & FAUX \\
    %    \textit{Justification:} & \\
    \hline
    Il n'existe pas de langage de programmation (non-trivial) dans lequel toutes les fonctions calculées sont totales & FAUX  \\
    \textit{Justification:} La version mini-Java en est la preuve. & \\
    \hline
    Il n'existe pas de langage de programmation ne permettant de calculer que toutes les fonctions totales calculables & VRAI\\
    \textit{Justification:} Théorème de Hoare-Allison & \\
    \hline
    Il existe une fonction totale calculable qui n'est l'extension d'aucune fonction partielle calculable & FAUX \\
    \textit{Justification:} C'est très facile de créer une fonction partielle à partir d'une fonction totale (il suffit de remplacer le retour d'une valeur par $\bot$. La fonction totale serait alors extension de la fonction partielle. & \\
    \hline
    Il existe une fonction partielle calculable telle qu'aucune fonction totale calculable n'est une extension de cette fonction partielle & VRAI \\
    \textit{Justification:} HALT est une fonction partielle calculable telle qu'aucune fonction totale calculable existe pour la représenter. & \\

\end{longtable}