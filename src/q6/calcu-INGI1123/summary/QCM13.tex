\begin{tabular}{p{13cm}|l}
     Un problème NP-complet est intrinsèquement complexe. & ? \\
    \textit{Justification:} ON EN SAIT RIEN (intrinsèquement complexe = qui n'a pas d'algorithme polynomial) VRAI SI P=/=NP et FAUX si P=NP. & \\
    \hline
    Le choix d’un modèle de calculabilité n’influence pas les classes P et NP. & VRAI \\
    \textit{Justification:} ça n'influence pas les classes (indépendante des langages). & \\
    \hline
    Le problème de la programmation linéaire est NP-complet. & FAUX \\
    \textit{Justification:} la programmation linéaire est dans P (ensemble d'inéquations avec une fonction objective qu'on veut optimiser). & \\
    \hline
    Un problème de décision dans P , alors le problème consistant à calculer une solution est également dans P. & VRAI \\
    \textit{Justification:} Pourquoi la complexité ferait des choses qui ne servent à rien ? Si on se focalise sur un problème de décision, c'est pcq c'est plus facile à traiter d'un point de vue théorique mais dans la pratique ça ne change rien! Donner la réponse si on sait qu'elle existe ça ne change rien. & \\
    \hline
    Un problème intrinsèquement complexe est dans EXPTIME.  & FAUX \\
    \textit{Justification:} y'a aussi les problèmes hyperpolinomiaux.& \\
    \hline
    Un problème NP-complet peut être résolu par un algorithme non déterministe de complexité temporelle polynomiale. & VRAI. \\
    \textit{Justification:} La question qui met dehors à un examen oral. & \\
    \hline
    Un problème NP-complet peut être résolu par un algorithme non déterministe de complexité spaciale polynomiale. & VRAI \\
    \textit{Justification:}  la complexité spatiale est toujours bornée par la complexité temporelle. & \\
    \hline
    Si SAT $\in$ P, alors P $\subseteq$ NP.& VRAI \\
    \textit{Justification:} P est dans NP de toute façon mais même P=NP. & \\
    \hline
    Si $SAT \in P$, alors P = NP. & VRAI \\
    %\textit{Justification:} & \\
    \hline
    Si SAT $\leq_{a}$ A et $A \in P$, alors SAT $\in$ P.& FAUX \\
    \textit{Justification:} on a besoin de la réduction polynomiale et pas la réduction algorithmique (cette dernière sert uniquement à la calculabilité, alors que la réduction polynomiale dit combien de fois on utilise l'algorithme !) & \\
    \hline

\end{tabular}
