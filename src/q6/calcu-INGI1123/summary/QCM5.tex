\begin{longtable}{p{13cm}|l}
   % Si $A$ est un sous-ensemble (strict et non vide) récursif de programmes Java, alors toute fonction calculable peut être calculée par un programme Java qui n'est pas dans $A$. & FAUX \\
    %\textit{Justification:} Il existe une fonction et pas \og{} toute \fg{} & \\
    L'ensemble des programmes Java calculant une fonction f telle que $f(10)=10$ est un ensemble récursif. & FAUX \\
    \textit{Justification:} Si ça concerne le comportement de la fonction, Rice nous dit que c'est impossible & \\ \hline
    L'ensemble des programmes Java calculant une fonction $f$ telle que $f(10) = 10$ est un ensemble récursivement énumérable. & VRAI \\
    \textit{Justification:} C'est vrai mais ce n'est pas un théorème, si le résultat=10 alors return 1; mais ça n'a aucun lien avec Rice. & \\ \hline
    Toute propriété relative aux programmes est non calculable. & FAUX \\
    \textit{Justification:} Certaines propriétés concernant le programme (comme sa longueur ou sa syntaxe) sont calculables. & \\ \hline
    Si $A$ est un sous-ensemble (strict et non-vide) récursif de programmes Java, alors toute fonction calculée par un programme de A est aussi calculée par un programme du complément de $A$. & FAUX \\
    \textit{Justification:} Tous les programmes de + de 10 lignes n'ont pas un équivalent de moins de 10 lignes. Il y a une différence entre $\forall$ et $\exists$. & \\
    %La propriété S-m-n affirme que tout numéro de programme calculable peut être transformé en un numéro équivalent, mais avec moins de paramètres. & FAUX \\
    %    \textit{Justification:} & \\
    %Les propriétés S-m-n et S sont équivalentes & FAUX \\
    %    \textit{Justification:} & \\
    %Tous les langages de programmation satisfont la propriété S-m-n & VRAI \\
    %    \textit{Justification:} & \\
\end{longtable}