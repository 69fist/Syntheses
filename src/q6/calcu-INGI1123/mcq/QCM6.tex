\begin{mcqs}
  \mcq{La propriété S-m-n affirme que tout numéro de programme calculable peut être transformé en un numéro équivalent, mais avec moins de paramètres}{0}
  {Voir QCM5.}
  \mcq{Les propriétés S-m-n et S sont équivalentes}{0}
  {S-m-n $\Rightarrow$ S mais l'inverse n'est pas vrai.}
  \mcq{Tous les langages de programmation satisfont la propriété S-m-n}{1}
  {Voir QCM5.}
  \mcq{Le théorème du point fixe est une conséquence du théorème de Rice}{0}
  {L'inverse est vrai.}
  \mcq{Si deux programmes P1 et P2 calculent la même fonction, alors il existe un transformateur $f$ de programmes ($f$ fonction totale calculable), tel que $f$(P1)=P2.}{1}
  {Rien à voir avec le théorème du point fixe. La fonction constante P2 peut servir de $f$.}
  \mcq{Si $f$ est un transformateur de programmes ($f$ fonction totale calculable), alors il existe deux programmes P1 et P2 tels que $f$(P1)=P2 ainsi que P1 et P2 calculent la même fonction.}{1}
  {Point fixe}
  \mcq{Si $f$ est un transformateur de programmes ($f$ fonction totale calculable), alors il existe deux programmes P1 et P2 tels que $f$(P1)=P2 ainsi que P1 et P2 calculent la même fonction totale}{0}
  {Si $f$ remplace l'input par un programme qui fait juste une boucle, son output est toujours le même programme.
  Ce programme output tout le temps $\bot$ et ne peut donc pas être total.}
  \mcq{Le théorème du point fixe permet de démontrer que la fonction HALT est non calculable}{1}
  {Il peut tout démontrer : $K$ est non récursif, HALT est non récursif}
  \mcq{L'ensemble des nombres réels calculables est énumérable}{1}
  {Autant que de programmes}
\end{mcqs}
