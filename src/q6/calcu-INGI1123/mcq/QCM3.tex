\begin{longtable}{p{13cm}|l} 
    Tout langage est récursif & FAUX \\
    \textit{Justification:} Un langage est un ensemble de mots finis. Tous les ensembles d'entiers ne sont pas récursifs, car il y a autant de sous-ensembles de $\mathbb{N}$ qu'il y a d'éléments dans $\mathbb{R}$. Ils ne sont donc pas tous récursifs. & \\
    \hline
    Un ensemble énumérable est récursif & FAUX \\
    \textit{Justification:} $\mathbb{K}$ est énumérable mais n'est pas récursif. \\
    \hline
    Un sous-ensemble infini d'un ensemble récursif est récursif & FAUX \\
    \textit{Justification:} K est un sous ensemble de $\mathbb{N}$. $\mathbb{N}$ est récursif mais K n'est pas récursif. \\
    \hline
    Un ensemble fini est récursif & VRAI \\
    \textit{Justification:} Si l'ensemble est fini, on peut créer un programme de taille finie qui gère l'entièreté des cas et donc dire s'il est ou non dans l'ensemble ($\Rightarrow$ récursif) & \\
    \hline
    Le complément d'un ensemble récursif est récursif & VRAI \\
%    \textit{Justification:} & \\
    \hline
    L'ensemble des rationnels est récursivement énumérable & VRAI \\
%    \textit{Justification:} & \\
    \hline
    Un sous-ensemble infini d'un ensemble récursivement énumérable est récursivement énumérable & FAUX \\
    \textit{Justification:} Exemple : le complément de K qui est bien un sous-ensemble de $\mathbb{N}$ & \\
    \hline
    Un sous-ensemble fini d'un ensemble énumérable est récursivement énumérable & VRAI \\
%    \textit{Justification:} & \\
    \hline
    L'union d'une infinité énumérable d'ensembles récursivement énumérable est récursivement énumérable & FAUX \\
    \textit{Justification:} L'union d'une infinité énumérable d'ensemble énumérables aurait été VRAI. \\
    \hline
    Le complément d'un ensemble récursivement énumérable est récursivement énumérable & FAUX \\
    \textit{Justification:} Exemple K \\
    \hline
    Une fonction dont la table est infinie est non calculable & FAUX \\
%    \textit{Justification:} & \\
    \hline
    Un algorithme donné ne calcule qu'une et une seule fonction & VRAI \\
%    \textit{Justification:} & \\
    \hline
    Il existe des ensembles non récursivement énumérables & VRAI \\
%    \textit{Justification:} & \\
    \hline
    Une fonction calculable peut être calculée par une infinité de programmes & VRAI \\
%    \textit{Justification:} & \\
    \hline
    L'ensemble HALT est récursivement énumérable & VRAI \\
    %    \textit{Justification:} & \\
    \hline
    Il existe des ensembles récursifs qui ne sont pas récursivement énumérables & FAUX \\
    \textit{Justification:} Etre récursif est plus "dur" que d'être récursivement énumérable \\
    \hline
    Tout ensemble de paires d'entiers est récursif & FAUX \\
    \textit{Justification:} Par exemple l'ensemble $\{(n,x) \mid \mathit{halt}(n,x)=1\}$ n'est pas récursif & \\
    \hline
    L'ensemble des sous-ensembles récursivement énumérables de $\mathbb{N}$ est énumérable & VRAI \\
    \textit{Justification:} Pour chaque sous-ensemble, il existe un programme Java qui l'énumère et l'ensemble des programmes Java est énumérable. & \\
    \hline
    Un sous-ensemble d'un ensemble récursivement énumérable est récursivement énumérable & FAUX \\
    \textit{Justification:} $\overline{K} \subseteq \mathbb{N}$. $\mathbb{N}$ est récursivement énumérable, $\overline{K}$ n'est pas récursivement énumérable. & \\
    \hline
    Il existe des ensembles récursifs qui ne sont pas énumérables & FAUX \\
    \textit{Justification:} Tout ensemble récursif est énumérable \\
    \hline
    Si le domaine d'une fonction est fini, alors cette fonction est calculable & VRAI \\
    \textit{Justification:} Les fonctions constantes sont toujours calculables & \\
    \hline
    Si le domaine d'une fonction est infini, alors cette fonction est non-calculable. & FAUX \\
    \textit{Justification:} La fonction "+1" est infinie mais elle est calculable \\
    \hline
    Une fonction constante est toujours calculable & VRAI \\
    \textit{Justification:} Par exemple "print 1, print 2, print 3, ..." \\
    \hline
    Un programme Java calcule une infinité de fonctions & FAUX \\
    %    \textit{Justification:} & \\

\end{longtable}
