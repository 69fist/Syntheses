\begin{mcqs}
  \mcq{Tout langage est récursif}{0}
  {Un langage est un ensemble de mots finis. Tous les ensembles d'entiers ne sont pas récursifs, car il y a autant de sous-ensembles de $\mathbb{N}$ qu'il y a d'éléments dans $\mathbb{R}$. Ils ne sont donc pas tous récursifs.}
  \mcq{Un ensemble énumérable est récursif}{0}
  {$\mathbb{K}$ est énumérable mais n'est pas récursif.}
  \mcq{Un sous-ensemble infini d'un ensemble récursif est récursif}{0}
  {$K$ est un sous ensemble de $\mathbb{N}$. $\mathbb{N}$ est récursif mais $K$ n'est pas récursif.}
  \mcq{Un ensemble fini est récursif}{1}
  {Si l'ensemble est fini, on peut créer un programme de taille finie qui gère l'entièreté des cas et donc dire s'il est ou non dans l'ensemble ($\Rightarrow$ récursif)}
  \mcq{Le complément d'un ensemble récursif est récursif}{1}
  {Il suffit d'appeler le programme de l'ensemble de départ et d'inverser la réponse.}
  \mcq{L'ensemble des rationnels est récursivement énumérable}{1}
  {On sait énumérer les rationelles avec un programme (tableau puis zigzag).}
  \mcq{Un sous-ensemble infini d'un ensemble récursivement énumérable est récursivement énumérable}{0}
  {Exemple : le complément de $K$ qui est bien un sous-ensemble infini de $\mathbb{N}$}
  \mcq{Un sous-ensemble fini d'un ensemble énumérable est récursivement énumérable}{1}
  {Un ensemble fini est récursif donc récursivement énumérable.}
  \mcq{L'union d'une infinité énumérable d'ensembles récursivement énumérable est récursivement énumérable}{1}
  {Supposons qu'on veulent savoir si $x$ est dans l'union.
   Soient $A_1, A_2 \ldots$ ces ensembles et $f_1, f_2, \ldots$ leur fonctions d'énumération totales calculables.
   Faisons un tableau où sur la ligne $i$ on met $f_i(0), f_i(1), f_i(2), \ldots$.
   Parcourons le tableau en zigzag pour et à chaque fois qu'on arrive à la case $i,j$, on demande à $f_i$ le prochain élément
   de son énumération $f_i(j)$.
   Si c'est $f_i(j) = x$ alors $x$ est dans l'union et on renvoit 1.}
  \mcq{Le complément d'un ensemble récursivement énumérable est récursivement énumérable}{0}
  {Exemple: $K$}
  \mcq{Une fonction dont la table est infinie est non calculable}{0}
  {Exemple: $f(x) = x^2$}
  \mcq{Un algorithme donné ne calcule qu'une et une seule fonction}{1}
  {Un programme output toujours la même output pour la même input et ça détermine une unique fonction.}
  \mcq{Il existe des ensembles non récursivement énumérables}{1}
  {Exemple: $\bar{K}$}
  \mcq{Une fonction calculable peut être calculée par une infinité de programmes}{1}
  {Il suffit de rajouter des lignes inutiles.}
  \mcq{L'ensemble HALT est récursivement énumérable}{1}
  {Avec l'input $n,x$ on exécute $n$ sur $x$ avec l'intepréteur universelle.
  S'il termine on renvoit $1$ sinon on termine pas mais c'est pas grâve car on veut pas être récursif.}
  \mcq{Il existe des ensembles récursifs qui ne sont pas récursivement énumérables}{0}
  {Etre récursif est plus ``dur'' que d'être récursivement énumérable}
  \mcq{Tout ensemble de paires d'entiers est récursif}{0}
  {Par exemple l'ensemble $\{(n,x) \mid \mathit{halt}(n,x)=1\}$ n'est pas récursif}
  \mcq{L'ensemble des sous-ensembles récursivement énumérables de $\mathbb{N}$ est énumérable}{1}
  {Pour chaque sous-ensemble, il existe au moins un programme Java qui l'énumère et l'ensemble des programmes Java est énumérable.}
  \mcq{Un sous-ensemble d'un ensemble récursivement énumérable est récursivement énumérable}{0}
  {$\overline{K} \subseteq \mathbb{N}$. $\mathbb{N}$ est récursivement énumérable, $\overline{K}$ n'est pas récursivement énumérable.}
  \mcq{Il existe des ensembles récursifs qui ne sont pas énumérables}{0}
  {Tout ensemble récursif est énumérable. En informatique, on ne gère que les input qui peuvent être définie de manière finie
  donc on ne gère pas les ensemble non-énumérables.}
  \mcq{Si le domaine d'une fonction est fini, alors cette fonction est calculable}{1}
  {On peut hardcoder toute les image dans le programme et détermine laquelle on prend avec un ``case'' sur l'input.}
  \mcq{Si le domaine d'une fonction est infini, alors cette fonction est non-calculable.}{0}
  {La fonction constante ``$+1$'' a un domaine infini mais elle est calculable}
  \mcq{Une fonction constante est toujours calculable}{1}
  {Par exemple ``print 1, print 2, print 3, ...''}
  \mcq{Un programme Java calcule une infinité de fonctions}{0}
  {Il en calcule une seule.}
\end{mcqs}
