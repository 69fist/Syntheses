\begin{mcqs}
  \mcq{Une Machine de Turing est un modèle abstrait ne pouvant pas être exécuté.}{0}
  {}
  \mcq{Une Machine de Turing dont le ruban serait fini à gauche ne serait pas un modèle complet de la calculabilité}{0}
  {Si les cases sont numérotées $[..., -2, -1, 0, 1, 2 ...]$ on peut les réaranger comme suit: $[0, 1, -1, 2, -2, ...]$}
  \mcq{Une Machine de Turing dont les seuls mouvements de la tête de lecture serait à droite ne serait pas un modèle complet de la calculabilité.}{1}
  {On ne saurait pas réécrire, on aura donc pas de mémoire}
  \mcq{Une Machine de Turing ne calcule que des fonctions totales.}{0}
  {Il est possible de boucler}
  \mcq{Soit A un ensemble récursivement énumérable mais non récursif. Une Machine de Turing avec A comme oracle est plus puissante qu'une machine de Turing sans oracle.}{1}
  {}
  \mcq{Une machine de Turing universelle nécessite l'utilisation d'au moins trois rubans}{0}
  {L'ajout de rubans est seulement une question de facilité ; il est toujours possible de revenir à un seul ruban.}
  \mcq{Toute fonction T-calculable est effectivement calculable.}{1}
  {}
  \mcq{Une machine de Turing non déterministe permet de calculer plus de fonctions}{0}
  {C'est un modèle abstrait, il a la même puissance parce que on doit simuler l'exécution non déterministe par un programme déterministe}
  \mcq{Soit une machin de Turing T qui reçoit en entrée une représentation d'une machin de Turing et qui fournit (toujours) comme résultat une représentation d'une machine de Turing. Il existe deux machines de Turing T1 et T2 tel que
    \begin{enumerate}
      \item l'exécution de T sur la représentation de T2 donne pour résultat T2 ;
      \item T1 et T2 calculent la même fonction
    \end{enumerate}
  }{1}
  {Point fixe}
\end{mcqs}
