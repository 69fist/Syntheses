\begin{mcqs}
  \mcq{Si une fonction est calculable par une Machine de Turing, alors cette fonction est calculable par un programme Java}{1}{}
  \mcq{Le modèle de calcul BLOOP possède les propriétés SD et SA.}{1}{}
  \mcq{Soit $D$ un nouveau modèle de calculabilité. Si toute fonction calculable est calculable dans D et si toute fonction calculable dans D est effectivement calculable, alors D est un modèle complet de la calculabilité.}{0}
  {Il manque l'aspect concret de réalisation.}
  \mcq{Un formalisme D de calculabilité possède la propriété U (description universelle) lorsque l'interpréteur de D est calculable.}{0}
  {%
    % Pour satisfaire U, il faut que l'interpréteur soit D-calculable
  }
  \mcq{Les propriétés SA, CD et S sont suffisantes pour que D soit un modèle complet de la calculabilité.}{1}
  {$SA \Rightarrow SD$, $SA \land CD \Rightarrow U$ et $CD \land S \Rightarrow CA$.}
  \mcq{La calculabilité est une discipline née après l'apparition des ordinateurs.}{0}
  {}
  \mcq{Les machines pensent-elles ?}{2}
  {La question est ouverte.}
\end{mcqs}
