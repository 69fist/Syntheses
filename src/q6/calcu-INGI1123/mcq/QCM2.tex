\begin{longtable}{p{13cm}|l}
  L'ensemble des rationnels est énumérable & VRAI \\
  \textit{Justification :} On peut les lister par diagonale \\ \hline
  Un sous-ensemble infini d'un ensemble énumérable est énumérable & VRAI \\ \hline
  Tout ensemble infini de chaînes finies de caractères est énumérable & VRAI \\ \hline
  Tout ensemble infini de chaînes infinies de caractères est énumérable & FAUX \\
  \textit{Justification :} Elements entre [0,1] n'est pas énumérable \\ \hline
  L'ensemble des fonctions de $\mathbb{N}$ vers \{0,1\} est non énumérable & FAUX. \\
  \textit{Justification :} Pour chaque fonction de \{0,1\} vers $\mathbb{N}$ correspond à une fonction. Comme les S.E de \{0,1\} sont non-énumérables, c'est non-énumérable  \\ \hline
  L'ensemble des fonctions de \{0,1\} vers $\mathbb{N}$ est énumérable & VRAI \\ \hline
  Tout langage (alphabet fini) est énumérable & FAUX \\ 
  \textit{Justification :} Comme les chaines peuven-être infinies. Par contre en informatique les chaines sont finies donc VRAI dans ce cas. \\\hline
  Toute fonction bijective est injective & VRAI \\ 
  \textit{Justification :} Injectif: Ensemble d'arrivée n'est pas la cible de deux éléments de l'ensemble de départ; Bijectif: tous élément est cible de 1 et 1 seul.\\ \hline
  Une fonction dont la table est infinie ne peut être décrite de manière finie & FAUX \\
  \textit{Justification :} $f:\mathbb{R}\rightarrow\mathbb{R} : f(x)=x^{2}$ & \\ \hline
  Toute fonction totale est surjective & FAUX \\ \hline
  Toute extension d'une fonction surjective est surjective & VRAI \\\hline
  Tout ensemble non-énumérable peut être mis en bijection avec l'ensemble des réels & FAUX \\ \hline
  L'ensemble des fonctions de $\mathbb{N}$ dans $\mathbb{N}$ est calculable & FAUX \\ \hline
  L'ensemble des programmes Java est calculable & VRAI \\ \hline
  L'énumérabilité des programmes 	Java et la non énumérabilité des fonctions de $\mathbb{N}$ vers $\mathbb{N}$ est une preuve de l'existence de fonctions non calculables & VRAI \\
  \textit{Justification :} On ne sait pas faire assez de programmes pour calculer toutes les fonctions. \\ \hline
  L'ensemble des fonctions calculables est énumérable & VRAI \\
  Nombre de fonctions calculables = Nombre de programmes. \\ \hline
  L'ensemble des fonctions non-calculables est énumérable & FAUX \\
\end{longtable}
