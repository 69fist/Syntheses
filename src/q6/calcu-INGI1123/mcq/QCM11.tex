\begin{mcqs}
  \mcq{Si la complexité temporelle d'un algorithme est $O(n^{2})$, alors elle est aussi $O(n^{3})$}{1}
  {}
  \mcq{Si la complexité spatiale d'un algorithme est $O(n^{3})$, alors elle ne peut pas être $O(n^{2})$}{0}
  {$O(n^{3})$ est une borne supérieure mais ça peut être moins!}
  \mcq{Un algorithme de complexité $O(n^{2})$ est toujours plus rapide qu'un algorithme de complexité $O(n^{3})$}{0}
  {}
  \mcq{Un problème qui peut être résolu par un algorithme (de complexité) polynomial est pratiquement faisable.}{1}
  {Pratiquement faisable = Polynomial}
  \mcq{Un problème qui peut être résolu par un algorithme (de complexité) exponentiel est pratiquement infaisable}{0}
  {Un problème est intrinsèquement complexe, un algorithme est pratiquement infaisable}
  \mcq{Si un problème est intrinsèquement complexe en MT alors il est aussi intrinsèquement complexe pour le langage Java.}{1}
  {}
  \mcq{Un ensemble est a-réductible (algorithmiquement réductible) à son complément.}{1}
  {}
  \mcq{Tout ensemble récursivement énumérable est a-réductible à HALT.}{1}
  {Halt est le ``plus difficile''}
  \mcq{Un ensemble est $f$-réductible (fonctionnellement réductible) à son complément.}{0}
  {}
  \mcq{Si $A$ peut être décidé par un algorithme polynomial et si $B$ est $f$-réductible à $A$, alors $B$ peut être décidé par un algorithme polynomial.}{0}
  {Parce que ce que la $f$-réductibilité veut dire : on peut calculer $B$ à partir de $A$ et on a une information sur la dernière instruction, qui est polynomiale. On ne connait pas le reste du code ($f$ (b)).}
\end{mcqs}
