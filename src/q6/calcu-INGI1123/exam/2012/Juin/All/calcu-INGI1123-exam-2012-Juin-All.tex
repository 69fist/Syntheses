\documentclass[en]{../../../../../../eplexam}

\hypertitle{Calculabilité et Complexité}{6}{INGI}{1123}{2012}{Juin}
{Florian Thuin}
{Yves Deville}

\section{Discussion link}
\url{http://www.forum-epl.be/viewtopic.php?t=10806}

\section{Vrai/Faux}

Chacun de ces QCM reprend la matière d'un chapitre du cours. Les QCM
sont par 5 et valent 2 points.

\begin{itemize}
  \item 3 bonnes réponses sur 5 vous donnent 0.5 point.
  \item 4 bonnes réponses sur 5 vous donnent 1 point.
  \item 5 bonnes réponses sur 5 vous donnent 2 points.
\end{itemize}

\begin{tabular}{p{13cm}|l}
Toute fonction dont le domaine est fini est calculable & \\ \hline
Toute fonction dont le domaine est infini est non calculable & \\ \hline
Toute fonction constante est calculable & \\ \hline
Il existe des ensembles récursifs, mais non énumérables & \\ \hline
L'ensemble des sous-ensembles récursivement énumérables de N est récursif & \\ \hline
Un sous-ensemble infini d'un ensemble récursivement énumérable est récursivement énumérable & \\ \hline
\end{tabular}
\bigskip

$f: \mathbb{N} \Rightarrow \mathbb{N}$ définie par: $f(i) = 1$ si $\varphi_i(i) = \perp; 0$ sinon.

\begin{tabular}{p{13cm}|l}
Cette fonction est calculable & \\ \hline
Le domaine de f est récursif & \\ \hline
L'image de f est NP-complet & \\ \hline
L'image de f est énumérable & \\ \hline
$\varphi_{i}$ est la fonction de numéro i (suivant l'ordre de l'énumération des fonctions) & \\ \hline
\end{tabular}
\bigskip

Soit la machine de Turing M suivante, avec $\Sigma = \{1, +\}$, $q_{0}$ = état initial

\begin{tabular}{p{13cm}|l}
M est une machine de Turing déterministe & \\ \hline
M possède 2 états & \\ \hline
Le résultat de l'exécution de M avec 111+11 comme input est 11111 & \\ \hline
Le résultat de l'exécution de M avec 11+1+1 comme input est 1111 & \\ \hline
Une machine de Turing universelle nécessite au moins 3 rubans & \\ \hline
\end{tabular}
\bigskip

\begin{tabular}{p{13cm}|l}
Soit D un nouveau modèle de calculabilité. Si toute fonction calculable est calculable dans D et si toute fonction calculable dans D est calculable, alors D est un modèle complet de la calculabilité. & \\ \hline
Un formalisme D de calculabilité possède la propriété U lorsque l'interpréteur de D est calculable. & \\ \hline
BLOOP est SD et SA. & \\ \hline
Le lambda calcul est un modèle de calculabilité complet & \\ \hline
\end{tabular}
\bigskip

\begin{tabular}{p{13cm}|l}
Si un langage de programmation (non trivial) permet de programmer son interpréteur, alors la fonction halt est calculable dans ce langage. & \\ \hline
BLOOP permet de programmer son interpréteur. & \\ \hline
Les grammaires sensibles au contexte permettent de définir tous les langages récursifs. & \\ \hline
Un sous-ensemble d'un langage régulier est un langage régulier. & \\ \hline
\end{tabular}
\bigskip

\begin{tabular}{p{13cm}|l}
L'ensemble des programme Java calculant une fonction f telle que f(10) = 10 est un ensemble récursivement énumérable. & \\ \hline
Si f est un transformateur de programmes( f fonction totale calculable), alors il existe deux programmes P1 et P2 tels que f(P1)=P2 ainsi que P1 et P2 calculent la même fonction. & \\ \hline
Un algorithme donné calcule une infinité de fonctions. & \\ \hline
\end{tabular}
\bigskip

\begin{tabular}{p{13cm}|l}
NP $\subseteq$ EXPTIME & \\ \hline
Un problème qui est pratiquement infaisable est dans EXP & \\ \hline
Soit A dans DTIME($n^2$), alors A est pratiquement infaisable. & \\ \hline
Soit A dans NTIME($n^2$) alors A est dans NSPACE($n^2$) & \\ \hline
S'il existe un programme Java calculant une fonction en O($n^3$), alors il existe un programme en MT déterministe qui calcule une fonction identique en O($n^3$).
\end{tabular}
\bigskip

\begin{tabular}{p{13cm}|l}
Si A $\leq_f$ B alors A $\leq_a$ B & \\ \hline
Si A $\leq_a$ B, alors si B est non récursif alors A est non récursif & \\ \hline
Si A $\leq_a$ B, alors si B est récursivement énumérable alors A est récursivement énumérable & \\ \hline
Si A $\leq_p$ B, B $\in$ P, alors A $\in$ P & \\ \hline
Si A $\leq_p$  B, B $\in$ NP, alors A $\in$ NP & \\ \hline
\end{tabular}
\bigskip

\begin{tabular}{p{13cm}|l}
Si SAT $\notin$ P, alors P $\not=$ NP & \\ \hline
Si SAT $\in$ P, alors P = NP & \\ \hline
Si SAT $\leq_a$ A, A $\in$ P, alors SAT $\in$ P & \\ \hline
\end{tabular}

\section{Questions ouvertes}

\begin{enumerate}
	\item Soit $A \subseteq \mathbb{N}$. Montrer que si A est fini, A est récursif.
	\item Enoncer le théorème de Rice (sans démonstration) et expliquer son utilité et ses conséquences pour la calculabilité.
	\item Définir NP-Complet, énoncer le théorème de Cook précisément et expliquer aussi précisément que possible les étapes de la démonstration.
\end{enumerate}

\end{document}
