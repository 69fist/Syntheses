\documentclass[a4paper,onecolumn,11pt]{article}

\usepackage[margin=2cm]{geometry}
\usepackage[french]{babel}
\usepackage[utf8]{inputenc}
\usepackage{amsfonts}

\author{}
\title{Calculabilité\\Travaux Pratique 3}
\date{}

\begin{document}

\maketitle

Ne sont repris dans ce fichier que les définitions marqué comme telle, certains termes etant définis dans le cours mais n'etant pas marqué spécifiquement comme définition.

\section{Introduction}
\textit{Pas de définitions}
\section{Concepts}
\begin{description}
	\item[Ensemble énumérable] Un ensemble est énumérable ou dénombrable si il est fini ou si il a le même cardinal que $\mathbb{N}$
\end{description}
\section{Calculabilité: résultats fondamentaux}
\begin{description}
	\item[Fonction calculable] Une fonction f est calculable si il existe un algorithme qui, recevant comme donnée n'importe quels nombre naturels $x_1, ..., x_n$ fourni comme résultat $f(x)$ si celui-ci existe.
	\item[]...\textit{pas fini}
\end{description}
\section{}
\section{}
\section{Introduction à la complexité}
\begin{description}
    \item[Complexité] Mesure de l'efficacité d'un programme pour un type de ressources, de deux types: spatiale (espace mémoire) et temporelle (temps d'exécution)
	\item[Complexité d'un problème] complexité de l'algorithme le plus efficace résolvant le problème
	\item[Notation "grand O"] $O(g(n))$ le nombre d'étapes exécutées par le programme pour une donnée de taille n n'est pas plus élevé que $cg(n)$, pour des données de tailles n suffisamment grande 
	\item[Problème intrinsèquement complexe] un problème est intrinsèquement complexe s'il n'existe pas d'algorithme de complexité polynomiale résolvant le problème
\end{description}
\section{Classes de complexité}
\begin{description}
    \item[Ensemble algorithmiquement réductible] Un ensemble $A$ est algorithmiquement réductible à un ensemble $B$ ($A \leq_a B$) si en supposant $B$ récursif, A est récursif
	\item[Classe A-complet] Un problème E est A-complet par rapport à une relation de réduction $\leq$ si
	\begin{enumerate}
		\item $E \in A$
		\item $\forall B \in A : B \leq E$
	\end{enumerate}
	\item[Classe A-difficile] Un problème E est A-difficile par rapport à une relation de réduction $\leq$ si
	\begin{enumerate}
		\item $\forall B \in A : B \leq E$
	\end{enumerate}
	\item[Ensemble fonctionellement réductible] Un ensemble $A$ est fonctionnellement réductible à un ensemble $B$ ($A \leq_f B$) si il existe une fonction totale calculable $f$ telle que $a \in A \Leftrightarrow f(a) \in B$
	\item[Ensemble polynomialement réductible] Un ensemble $A$ est polynomialement réductible à un ensemble $B$ ($A \leq_p B$) si il existe une fonction totale calculable $f$ de complexité temporelle polynomiale telle que $a \in A \Leftrightarrow f(a) \in B$
\end{description}

\end{document} 