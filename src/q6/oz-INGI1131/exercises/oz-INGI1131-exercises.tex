\documentclass[fr]{../../../eplexercises}

\usepackage{../../../eplcode}

\lstset{language={Oz},morekeywords={for,do,lazy}}

\newcommand{\lpar}{$\langle$}
\newcommand{\rpar}{$\rangle$}

\hypertitle{Programming Languages Concepts}{6}{INGI}{1131}
{Benoît Legat\and François Robinet\and Florian Thuin}
{Peter Van Roy}

\section{Meeting Oz and its basic data structures}
    
  \subsection{Follow the tutorial}
    1. Follow the tutorial at
    
      \url{http://www.mozart-oz.org/documentation/tutorial/node2.html}
      
      Do only chapter 2. Do not continue with chapter 3!
      
  \subsection{Oz Browser}
      
    \begin{lstlisting}
     {Browse 'Hello Nurse'}
    \end{lstlisting}
    
    Par défaut, Oz Browser affichera 'Hello Nurse' puisqu'il considère ça comme un atome 

    \begin{lstlisting}
     {Browse "Hello Nurse"}
    \end{lstlisting}

    Par défaut, Oz Browser affichera les valeurs ASCII des caractères de la chaîne. Le comportement sera modifié et affichera les lettres \og{} lisibles \fg{} dans le cas où on configure l'affichage des Strings (Options $\Rightarrow$ Representation $\Rightarrow$ Strings)

  \subsection{Browse a list or a record}
3. The browser only allows you to browse a value, but there is no limitation on that.
What if you want to browse X, Y and Z at the same time in the following case?
    \begin{lstlisting}
    local
      X=x Y=y Z=z
    in
      {Browse ..........}
    end
    \end{lstlisting}
    
    \subsubsection{Using a list}
    \begin{lstlisting}
    local
      X=x Y=y Z=z
    in
      {Browse [X Y Z]}
    end
    \end{lstlisting}

    \subsubsection{Using a record}
    
        \begin{lstlisting}
    local
      X=x Y=y Z=z
    in
      {Browse a(X Y Z)}
    end
    \end{lstlisting}

  \subsection{Oz Compiler and the errors}

    \begin{lstlisting}
	local
	  F1 % = fun {$ X} X*X end
	  F2 % = fun {$ X} X+X end
	in
	  {Browse ({F1 3} - {F2 3}) + 4.0}
	end
    \end{lstlisting}
    
    Oz Compiler vous affiche le message suivant :
    
    \emph{application suspends forever}
    
    \emph{Hint: undetermined variable called as procedure}
    
    \bigskip
    
    Ce résultat est logique puisque le \% s'utilise pour faire un commentaire. Les commentaires impliquent qu'on assigne aucune valeur à $F1$ et $F2$, dès lors l'appel de $F1$ et $F2$ comme fonctions ne pourra jamais être calculé le programme se bloque. Le compilateur vérifie que les variables seront liées lors de l'exécution avant de permettre au programme de s'exécuter (pour éviter les problèmes - simples - d'arrêt d'exécution)
    
    \bigskip
    
    Considérons maintenant le code sans les commentaires :
    
    \begin{lstlisting}
	local
	  F1 = fun {$ X} X*X end
	  F2 = fun {$ X} X+X end
	in
	  {Browse ({F1 3} - {F2 3}) + 4.0}
	end
    \end{lstlisting}

    Oz Compiler n'affiche aucun avertissement puisque toutes les variables sont maintenant liées. Cependant, à l'exécution, Oz Emulator affichera une erreur de type (\og{} type error \fg{}) impliquée par le fait que Oz n'accepte pas l'addition d'entiers et de flottants. Les fonctions $F1$ et $F2$ retournent toutes les deux des nombres entiers lorsqu'elles sont appelées avec des nombres entiers en arguments, mais l'addition $+ 4.0$ qui est un flottant n'est pas permise.
    
  \subsection{Calculate the roots of a function}
    Oz est un très bon calculateur. Pour le montrer, on souhaite calculer les racines d'une fonction donnée :
    
    \[
     f(X) = X^{2} + 5X - 150
    \]
    
    \[
     \textrm{roots}(f(X)) = \frac{-b \pm \sqrt{b^{2} - 4ac}}{2a}
    \]
    
    \subsubsection{Avec des variables locales}
    
    \begin{lstlisting}
      declare
      A=1.0
      B=5.0
      C=~150.0
      X0=(~B+{Sqrt B*B-4.0*A*C})/2.0*A
      X1=(~B-{Sqrt B*B-4.0*A*C})/2.0*A

      {Browse X0}
      {Browse X1}
    \end{lstlisting}
    
    \subsubsection{Sans variables locales}

    \begin{lstlisting}
      {Browse (~5.0+{Sqrt 5.0*5.0-4.0*1.0*~150.0})/2.0*1.0}
      {Browse (~5.0-{Sqrt 5.0*5.0-4.0*1.0*~150.0})/2.0*1.0}
    \end{lstlisting}

  \subsection{Variable et identifiant - La portée}
  
  \begin{lstlisting}
    local X in
      X = a
      X = b
    end
  \end{lstlisting}

  Ce code est un code qui ne fonctionne pas. L'erreur est simple mais est la base de Oz : on est dans un langage qui ne permet que le \og{} single-assignment \fg{}, autrement dit on ne peut pas assigner deux valeurs différentes à la même variable.
  
  Une variable est une zone mémoire. Un identifiant d'une variable est une lettre majuscule dans le code qui pointe vers la zone mémoire où la variable est stockée. Le même identifiant ne pouvant pas pointer vers deux zones mémoires différentes, ce code est faux.
  
  \begin{lstlisting}
    local X in
      X = a
      local X in
	X = b
      end
    end
  \end{lstlisting}

  Ce code fonctionne. Cela peut paraitre à première vue anormal puisque le code précédent ne fonctionnait pas. La grande différence ici est la portée des identificateurs (\og{} scope \fg{}). Chaque identificateur pointe vers des variables au sein d'un \emph{environnement}, un environnement fait le lien entre les identifiants et les variables qui leur sont liées.
  
  Chaque 'local ... end' définissant un environnement contextuel différent, les identificateurs peuvent se référer à des valeurs différentes puisque malgré que ça soit la même lettre, ce sont deux identifiants dans des environnements différents, donc les deux identificateurs sont différents.
  
  \begin{lstlisting}
    local
      Y =1
      X =1
    in
      local
	Y=2
      in
	{Browse [X Y]}
      end
      {Browse [X Y]}
    end
  \end{lstlisting}
  
  Le code ci-dessus affiche dans un premier temps [1 2] et ensuite [1 1]. Cela s'explique également par la portée des identificateurs. Le premier local définit un environnement contextuel dans lequel Y=1 X=1. Le second local définit un environnement dans lequel Y=1.
  
  Dès lors, lorsque Browse cherche à afficher les variables liées aux identifiants, il recherche les identifcateurs qui ont la portée la plus proche, autrement dit qui ont été définis dans l'environnement contextuel le plus proche.

  \subsection{Lists and records}
  La liste [1 2 3] contient 4 éléments : 1, 2, 3, nil sous la forme 1|2|3|nil
  
  La liste [[1 2 3]] contient 2 éléments : la liste 1|2|3|nil et nil sous la forme (1|2|3|nil)|nil
  
  Pour afficher le second élément de la liste dans la liste [[1 2 3]] il faut faire :
  
  \begin{lstlisting}
    L = [[1 2 3]]
    {Browse L.1.2.1}
  \end{lstlisting}
  
  Avec la fonction Nth implémentée dans Oz, on peut faire la même chose :
  
  \begin{lstlisting}
    L=[[1 2 3]]
    {Browse {Nth {Nth L 1} 2}}
  \end{lstlisting}
  
  \bigskip
  
  Feeding the following code
  
  \begin{lstlisting}
   {Browse '#'(a:5 b:2 3 4) == '#'(1:3 b:2 a:5 2:4)}
  \end{lstlisting}
  
  retourne true. Cela s'explique par la construction des structures Records et du sucre syntaxique. Pour mieux le comprendre, il faut se référer à la sémantique du langage.
  
  En Oz, la sémantique du langage pour un record est :
  
  \lpar record\rpar, \lpar pattern\rpar ::==\lpar literal\rpar | \lpar literal\rpar(\lpar feature\rpar$_{1}$:$x_{1}$ ... \lpar feature\rpar$_{n}$:$x_{n}$)

  Si les features ne sont pas explicitement spécifiées, à la compilation Oz leur assignera des numéros selon l'ordre dans lequel ils sont présents dans le record.

  \bigskip
  
  Supposons que l'on souhaite accéder à $d$ dans le record suivant :
  
  \begin{lstlisting}
    R='#'(a [b '#'(c d) e] f)
  \end{lstlisting}
  
  Pour le faire simplement, on peut écrire explicitement (autrement dit, en langage noyau) la structure de R :
  
  \begin{lstlisting}
    R='#'(1:a 2:'|'(1:b 2:'#'(1:c 2:d) 3:e) 3:f)
  \end{lstlisting}


  
f Suppose you have that  . How do you access d through
R?
3(a) Example
(b) Exercise
Figure 1: Graphs
  \subsection{Browsing cyclic records}
8. The Browser offers special support for browsing cyclic records. You can find details in
section 2.7 of http://www.mozart-oz.org/documentation/browser/index.html.
a In the following example, X is defined as a cyclic record. Feed this piece of code
and observe what happens:
declare
X = a(1 X)
{Browse X}
Now select the option ‘Minimal Graph’ in Options/Representation and feed the
code again. How do you explain what you are observing?
b Let’s map graphs to records as follows; each node is a record. The label of the
record is the name of the node. The features of the records are the names of the
outgoing edges. The value of the record at a particular feature is the destination
of the corresponding edge. Under this definition, we can say that the graph in
Figure 1a is mapped to the record R defined below:
declare
X = e(g h)
R = a(b(d X) c(X f))
Given this mapping, consider the graph in Figure 1b. How would you define the
corresponding record in Oz?
  \subsection{Browse that code, bro !}
9. In each of the following cases, and before feeding the code, say what is going to
be shown in the Browser and justify your answer. Run the code to confirm your
understanding.
a local X Y in
X = 1|2|Y
X = Y
{Browse X.2.2.1}
end
b local X Y Z in
X = 1|X
Y = X|Z
Z = 2|3|4|nil
{Browse Y.1.2.1}
end
c local X Y Z in
X = a(b X)
Y = c(X Z)
Z = d(e f g h)
{Browse Y.1.2.1}
end
5

\skipape
\skipape

\clearpage
\section{}
\subsection{}
\begin{solution}
  \lstinputlisting{APE4/q1.oz}
\end{solution}
\subsection{}
\begin{solution}
  \lstinputlisting{APE4/q2.oz}
\end{solution}
\subsection{}
\nosolution
\subsection{}
\begin{solution}
  \lstinputlisting{APE4/q4.oz}
\end{solution}

\clearpage
\section{}
\subsection{}
\begin{solution}
  \lstinputlisting{APE5/q1.oz}
\end{solution}

\skipape

\clearpage
\section{}
\subsection{}
\begin{solution}
  \lstinputlisting{APE7/q1.oz}
\end{solution}
\subsection{}
\begin{solution}
  \lstinputlisting{APE7/q2.oz}
\end{solution}
\subsection{}
\begin{solution}
  \lstinputlisting{APE7/q3.oz}
\end{solution}
\subsection{}
\nosolution
\subsection{}
\nosolution
\subsection{}
\begin{solution}
  \lstinputlisting{APE7/q6.oz}
\end{solution}

\clearpage
\section{}
\subsection{}
\subsection{}
\subsection{}
\begin{solution}
  \lstinputlisting{APE8/Lab8-1-2-3.oz}
\end{solution}
\subsection{}
\subsection{}
\begin{solution}
  \lstinputlisting{APE8/Lab8-4-5.oz}
\end{solution}
\subsection{}
\begin{solution}
  \lstinputlisting{APE8/Lab8-6.oz}
\end{solution}
\subsection{}
\begin{solution}
  \lstinputlisting{APE8/Lab8-7.oz}
\end{solution}
\subsection{}
\begin{solution}
  \lstinputlisting{APE8/Lab8-8.oz}
\end{solution}
\subsection{}
\begin{solution}
  \lstinputlisting{APE8/Lab8-9.oz}
\end{solution}

\clearpage
\section{}
\subsection{}
\begin{solution}
  \lstinputlisting{APE9/Lab9-1.oz}
\end{solution}
\subsection{}
\begin{solution}
  \lstinputlisting{APE9/Lab9-2-asynchronous.oz}
  \lstinputlisting{APE9/Lab9-2-synchronous.oz}
\end{solution}
\subsection{}
\begin{solution}
  \lstinputlisting{APE9/Lab9-3.oz}
\end{solution}
\subsection{}
\begin{solution}
  \lstinputlisting{APE9/Lab9-4.oz}
\end{solution}
\subsection{}
\begin{solution}
  \lstinputlisting{APE9/Lab9-5.oz}
\end{solution}
\subsection{}
\begin{solution}
  \lstinputlisting{APE9/Lab9-6.oz}
\end{solution}

\skipape

\clearpage
\section{}
\subsection{}
\nosolution
\subsection{}
\begin{solution}
  \lstinputlisting{APE11/q2.oz}
\end{solution}
\subsection{}
\begin{solution}
  \lstinputlisting{APE11/q3.oz}
\end{solution}
\subsection{}
\begin{solution}
  \lstinputlisting{APE11/q4.oz}
\end{solution}
\subsection{}
\begin{solution}
  \lstinputlisting{APE11/q5.oz}
\end{solution}
\subsection{}
\begin{solution}
  \lstinputlisting{APE11/q6a.oz}
  \lstinputlisting{APE11/q6bc.oz}
\end{solution}

\skipape

\clearpage
\section{}
\subsection{}
\begin{solution}
  \lstinputlisting{APE13/q1.oz}
\end{solution}
\subsection{}
\begin{solution}
  \lstinputlisting{APE13/q2.oz}
\end{solution}
\subsection{}
\subsection{}
\begin{solution}
  \lstinputlisting{APE13/q34.oz}
\end{solution}
\subsection{}
\begin{solution}
  \lstinputlisting{APE13/q5.oz}
\end{solution}

\clearpage
\section{}
\subsection{}
\begin{solution}
  \lstinputlisting{APE14/q1.oz}
\end{solution}

\end{document}
