\documentclass[a4paper, 11pt]{article}

\usepackage[utf8x]{inputenc}
\usepackage[french]{babel}
\usepackage{float}
\usepackage[margin=2cm]{geometry}
\usepackage{framed}
\usepackage{graphicx}
\usepackage{hyperref}
\usepackage{fancyhdr}
\usepackage{pdfpages}
\usepackage{palatino}
\usepackage{enumitem}

\usepackage[bottom]{footmisc}
\usepackage{desclist}
\usepackage{tocloft}

\fancyhead[L]{LELEC1930 - Introduction aux télécommunications}
\fancyhead[R]{Réponse aux questions}
\pagestyle{fancy}

\title{LELEC1930\\Introduction aux télécommunications\\Réponse aux questions}
\date{}

\begin{document}

\maketitle

\begin{enumerate}

% Question 1
\item \textit{Qu'est-ce que la pupinisation ? Pourquoi n'est-elle plus utilisée au-jourd'hui ?}

On exprime l'impédance caractéristique d'une ligne comme étant 
$$Z_c = \sqrt{\frac{R + j\omega L}{G + j\omega C}}$$
Et l'on exprime l'exposant de propagation $\gamma$ avec $\alpha$ l'atténuation et $\beta$ le déphasage comme étant
$$\gamma = \sqrt{(R + j\omega L)(G + j\omega C)} = \alpha + j\beta$$

Dans une ligne on suppose G comme étant nul, si dans notre ligne nous avons la résistance $R$ qui est nettement plus grande que l'inductance $L$, alors nous avons:
$$\gamma = \sqrt{\frac{\omega RC}{2}} + j\sqrt{\frac{\omega RC}{2}}$$
$$Z_c = \sqrt{\frac{R}{j\omega C}}$$
Nous obtenons donc que l'atténuation et le déphasage sont directement proportionnel à la fréquence du signal et l'impédance caractéristique de la ligne qui est inversément proportionnel à celle-ci (car $\omega = 2\pi f$).

Si par contre dans notre ligne nous avons la résistance $R$ qui est nettement plus petite que l'inductance $L$, nous obtenons le résultat suivant:
$$\gamma = \frac{R}{2}\sqrt{\frac{C}{L}} + j\omega \sqrt{LC}$$
$$Z_c = \sqrt{\frac{L}{C}}$$
Ce qui nous donne une impédance et une attenuation indépendante du signal et un déphasage qui dépend de la fréquence. 

Dès lors l'opération de pupinisation consiste à augmenter artificiellement l'inductance de la ligne en ajoutant le long de celle-ci des inductances, afin de ne pas se trouver dans le premier cas. Cette opération n'est plus effectué de nos jours, car cela provoque de l'atténuation sur les hautes fréquences, fréquences étant utilisé dans des choses tel que l'ADSL

% Question 2
\item \textit{Comment caractérise-t'on la directivitée d'une antenne ? Donnez un exemple d'application où l'on veut une antenne directive et un exemple d'application où on ne veut pas d'antenne directive.}

La directivité d'une antenne est comparable à sa précision. Il est caractérisé par un angle d'ouverture faible. 

Typiquement un émetteur wifi par exemple essaye d'avoir un angle d'ouverture large pour permettre une diffusion large au utilisateur. (En effet, les utilisateurs du wifi peuvent se trouver n'importe où).

A contrario, on peut avoir envie d'une bonne directivité pour une antenne tv (type Yagi) comme on en trouve sur les toits. En effet, l'emetteur ne va pas bouger du jour au lendemain, et si l'on se focalise dans la bonne direction on permet de réduire le bruit du signal.

% Question 3
\item \textit{Décrivez un type de modulation analogique au choix.}

Soit la porteuse $v(t) = A_c cos(\omega_c t + \varphi_c)$, pour avoir une modulation d'amplitude nous allons ajouter le signal \textit{modulant} $m(t)$ multiplié par un \textit{indice de modulation} $k$ à ce signal.
$$s(t) = [1+km(t)] A_c cos(\omega_c t + \varphi_c)$$
Il s'agit donc d'une modulation d'amplitude (AM), ou l'amplitude va varier en fonction du signal modulant. Il faut bien faire attention dans le choix de l'indice de modulation, pour eviter la surmodulation. En effet un indice trop élevé pourrait causer un renversement de phase (\textit{faire un schéma}) ce qui pourrait causer de l'ambiguïté lors de la démodulation.

Pour démoduler ce type de modulation on va multiplier le signal modulé par la porteuse (elle doit être disponible à l'émission et à la réception), nous aurons ainsi
\begin{eqnarray*}
A_c s(t)cos(\omega_c t + \varphi_c)*cos(\omega_c t + \varphi_c) &=&A_c s(t)cos^2(\omega_c t + \varphi_c)\\
&=& A_c s(t) \frac{1+ cos(2\omega_c t + 2\varphi_c)}{2}\\
&=& \frac{A_cs(t)}{2} + \frac{A_cs(t)cos(2\omega_c t + 2\varphi_c)}{2}
\end{eqnarray*}
Un filtrage passe bas nous laissera le signal démodulé.

La modulation va aussier créer deux bandes de même taille à gauche et à droite de la porteuse, on voit ca en convertissant le signal du temporel au fréquentiel
$$f(t) cos(\omega_0 t) \Leftrightarrow \frac{F(\omega - \omega_0) + F(\omega + \omega_0)}{2}$$
Dès lors il va falloir choisir combien de bande conserver (en effet il y a doublon dans l'information). Dès lors plusieurs modes existent:
\begin{itemize}
\item DSB (doubleside band): envoi des deux bandes
\item LSB (lowerside band): on envoi les deux bandes intérieures
\item USB (upperside band): on envoi les deux bandes extérieure
\item QAM (bande lattérale unique): on transforme le signal $m(t)$ en deux signaux $s_1(t)$ et $s_2(t)$ et l'on envoi $s(t) = s_1(t) cos(\omega_c t) + s_2(t) sin(\omega_c t)$. Ainsi une démodulation avec $cos(\omega_c t)$ nous fournira $s_1(t)$ et une démodulation avec $sin(\omega_c t)$ nous fournira $s_2(t)$. 
\item VSB (vestigialside band): il s'agit d'un envoi en bande lattéral résiduel, quand une fréquence est trop près de 0 Hz on ne peut pas envoyer une bande lattérale unique, on va alors atténuer au maximum l'un des cotés.
\end{itemize}

La modulation d'amplitude à un niveau de bruit constant assez élevé. 

% Question 4
\item \textit{Décrivez le rôle des différents blocs du shéma de modulation difitale ci dessous.}


% Question 5
\item \textit{Quel est le principe du CDMA ? Donnez deux exemples d'options alternatives pour implémenter la division de l'accès pour les différents utilisateurs.}

Le CDMA (code division multiple access) ne divise pas la bande fréquence, chaque personne qui communique sur le canal va recevoir un identifiant (un code) qui va lui permettre de communiquer sur la fréquence. L'avantage par rapport à un TDMA ou FDMA est que si le nombre d'utilisateur est restreint, ils vont pouvoir profiter de la totalité de la bande passante.

Le TDMA, va diviser la bande passante en un ensemble de timeslot que les différents utilisateurs vont pouvoir utiliser, ce qui limite l'usage de chaque utilisateur à un certains temps.

Le FDMA va permettre de séparer le canal en un ensemble de canauxs de fréquence indépendant. Il faut faire attention dans ce type de multiplexage que les canaux soit bien séparé

% Question 6
\item \textit{Qu'est-ce que le mécanisme ARQ ? Pourquoi est-il parfois utilisé à la
place des codes correcteurs d'erreurs et quand}

Le mécanisme ARQ est utilisé quand on désire une plus grande fiabilité dans les communications. Il s'agit d'un système de transmission de donnée appartenant à la couche Liaison (niveau 2) du modèle OSI. Il existe plusieurs modes à ce systèm de transmission.
\begin{itemize}
\item \textbf{Stop-and-wait} A chaque paquet de donné envoyé l'emetteur attend une confirmation du récepteur. Si il y a une erreur, l'émetteur renvoi le paquet. 
\item \textbf{Go-back-n} L'émetteur envoi ses paquets sans attendre la réponse de l'utilisateur, si il tombe sur un message d'erreur il identifie le paquets manquant et redémarre ses envois à partir de l'endroit ou il y a eu une erreur. Des que le récepteur voit une érreur il arrete de récupérer des paquets.
\item \textbf{Selective-repeat} Lorsqu'une erreur intervient le récepteur préviens l'émetteur en lui disant quel paquet à foiré, mais il continue à récuperer les paquets. A un moment l'emetteur va renvoyer le paquets en question.
\end{itemize}

Le mécanisme ARQ est utilisé parfois à la place des codes correcteurs d'erreurs tous simplement parce que dans certains cas utiliser un code correcteur d'erreurs est simplement trop couteux. En effet, supposons que l'on veut envoyer $10^6$ bits par
blocs de 1000bits. Si on a une ligne T1 avec un taux d'erreur (BER) de $10^{-6}$ (sur 1 millions de bit, il y en un seul fautif)
et un ligne T2 avec un taux d'erreur de $10^{-4}$. \\
Si on utilise un code(1010,1000) pour corriger les erreurs simples. On a besoin de 10000 bits de contrôle (car on a 1000 bloc )
cela pour T1 et T2.

Par contre si utilise un code(1001,1000) pour détecter les erreurs (par exemple la parité):
\begin{itemize}
\item Dans T1,  pour les 1000 blocs, on aura envoyé les 1000 bits de contrôle + la répétition des 1001 bits du blocs qui contient le bit éronné, on a donc ajouter 2001 bits (raisonable).
\item Dans T2, on aura aussi envoyé 1000 bits de contrôle mais en plus la répétiion des 100*10001 bits des blocs érronés (le 100 viens du fait que sur les $10^6$ bits il y en a 100 fautifs). On aura donc ajouter en tout avec notre code détecteurs d'erreurs plus de 100 000 bits.
\end{itemize}


% Question 7
\item \textit{Quel est l'intérêt de la DCT dans le codage JPEG ? La DCT constitue-t'elle une opération avec ou sans pertes ?}

La DCT est un codage sans perte qui va transformer une matrice de 8x8 pixels contenant de la luminance en une matrice de 8x8 coéfficient DCT. La matrice DCT contient la représentation en 2 dimensions et de facon discrète d'une cosinus, où les points bas du cosinus représente du blanc et les points haut du noir (et entre ces valeurs des tonalité de gris). Les plus basses fréquence sont contenue en haut à gauche et les plus haut en bas à droite.

Dans le cadre JPEG, la matrice de coefficient représente la participation de ces différents valeurs DCT au bloc de 8x8 pixels. Par multiplication par une matrice quantification on va essayer d'annuler les blocs contenant les fréquences les plus élevé (celles etant les moins visible à l'oeil human). Si l'on ne peut pas se débarasser d'un pixel, on peut se débarasser de certains coéfficient, ceux en bas à droite. Une mesure de la distortion de l'image permet d'ajuster le nombre de coefficient à supprimer. Tout les coefficients annulés ne seront donc pas repris dans la nouvelle image. La compression JPEG est avec perte (pas le DCT). 

% Question 8
\item \textit{Qu'est-ce que l'alignement temporel dans le système GSM ? Pourquoi est-il seulement nécessaire dans la liaison montante ?}

L'alignement temporel (ou alignement dynamique) est nécessaire du fait que les GSM peuvent se déplacer dans l'espace (s'éloigner ou se rapprocher de la tour BTS). Si tel est le cas il est possible que le délai du GSM de la personne en question jusqu'a la tour cause un décalage hors de son timeslot, et cela peut donc causer des interférences avec un autre timeslot. Il est donc important d'adapter ma réception dans le système GSM. Cela n'a d'interet que dans la liaison montante car le GSM ne recoit que le signal de la tour. 

% Question 9
\item \textit{Qu'est-ce qu'un algorithme cryptographique à clef publique ? Dans quel
cadre est-il généralement utilisé ? Quels sont ses désavantages ?}

Un algorithme à clef publique (ou asymétrique) utilise une clef publique pour crypter les données et une clef privé permettant le décryptage (faire schéma). Ils sont utilisé par exemple dans le cadre des transactions bancaire. Les désavantage de tel algorithme réside dans la certification des clefs publiques, il faut s'assurer que n'importe qui ne récupère pas la clef publique.

\end{enumerate}




\section*{Formules utiles}

\begin{itemize}
\item[$P_{T_i}$] Puissance émetteur isotrope équivalente
\item[$P_{R_i}$] Puissance récepteur isotrope équivalente
\item[$P_T$] Puissance émetteur
\item[$G_T$] Gain émetteur
\item[$P_R$] Puissance récepteur
\item[$G_R$] Gain Récepteur
\item[$L_c$] Pertes dans le cable
\item[$L_{FS}$] Perte dans l'air
\item[$D$] portée (durant le cours c'est $r$, le rayon isotrope
\item[$A$] Air effective de l'antenne
\end{itemize}

$$P_{T_i} = \frac{P_TG_T}{L_c}$$
$$L_{FS} = \left(\frac{4\pi D}{\lambda}\right)^2$$
Dans la formule suivante, si il s'agit d'une parabole on doit multiplier les pertes au dénominateur par $\theta$ (les pertes causé par la pluie, la moisisure, les gaz, ...).
$$P_R = P_{T_i}G_R \left(\frac{\lambda}{4\pi D}\right)^2=\frac{P_{T_i}G_R}{L_{FS}}$$
$$P_R = \frac{A}{4\pi D^2}P_TG_T$$
$$P_R = P_{R_i}G_R$$

\end{document}
