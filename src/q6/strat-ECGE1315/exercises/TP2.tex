\subsection{Rappel théorique}
	\subsubsection{Contexte}
	\subsubsection{La chaîne de valeur}
	\subsubsection{La capacité stratégique}
		\begin{table}[!ht]
			\begin{center}
				\begin{tabular}{p{4cm}|p{5.5cm}|p{5.5cm}}
				& Ressources & Compétences \\ \hline
				Capacité nécessaire pour intervenir sur un marché (capacité seuil) &
					Ressources requises
						\begin{itemize}
							\item Tangibles
							\item Intangibles
						\end{itemize}
						&
						Compétences nécessaires
				\\ \hline
				Capacité nécessaire pour obtenir un avantage concurrentiel
				&
				Ressources uniques
						\begin{itemize}
							\item Tangibles
							\item Intangibles
						\end{itemize}
						&
						Compétences fondamentales
				\\
				\end{tabular}
			\end{center}
		\end{table}

		Les ressources sont :
		\begin{itemize}
			\item physiques (équipements, bâtiments,...)
			\item financières (avoirs, dettes, revenus, obligations)
			\item humaines (quantité, qualité, savoir, savoir-faire)
			\item intellectuelles (brevets, marques, base de données,...)
		\end{itemize}

		Les compétences correspondent à l'utilisation des ressources (coopération, capacité d'innovation, relations avec les clients et les fournisseurs, expérience,...).
	\subsubsection{L'avantage concurrentiel}
		\enquote{Un avantage concurrentiel réside dans la valorisation
		de ressources et de compétences qui confèrent à l'organisation
	une position dominante dans son groupe stratégique.}
		
		\bigskip
		
		\begin{tabular}{l|p{10cm}}
			\textbf{V}aleur & Les capacités génèrent-elles une valeur pour les clients ?\\
			\hline
			\textbf{R}areté & Les capacités ne sont-elles détenues que par un nombre limité de concurrents ?\\
			\hline
			\textbf{I}nimitabilité & Les concurrents éprouvent-ils des difficultés à imiter les capacités ?\\
			\hline
			\textbf{N}on-substituabilité & Le risque de substitution est-il faible ?\\
		\end{tabular}
	\subsubsection{SWOT}

\subsection{Application pratique : Benetton}
	\subsubsection{Historique}
		\begin{description}
			\item[1960] : Création de la marque par Luciano Benetton et sa famille
				\subitem \enquote{Entreprise industrielle de mode}
			\item[1963] : Ouverture du premier magasin Benetton
			\item[1966] : Inauguration de la première usine Benetton
			\item[1969] : Ouverture du premier magasin à Paris
				\subitem Magasin \enquote{vitrine} et couleurs personnalisées
			\item[1981] : Création des chaussures Tip-Top et ligne pour enfant 0-12 ans
			\item[1982] : Rencontre de Luciano Benetton avec le photographe Oliviero Toscani
			\item[1985] : L'enseigne achète l'écurie de Formule 1 de Toleman
			\item[1985] : La famille Benetton s'installe au Luxembourg
				\subitem Campagnes de publicités \enquote{buzz}
			\item[1988] : Création des lignes montres et cosmétiques ainsi que du parfum Colors
			\item[1990] : La marque est présente dans 120 pays
			\item[1994] : Oliviero et Lucianno créent la Fabrica
			\item[1995] : Benetton F1 obtient le titre de champion du monde
			\item[2000] : 5500 magasins vendent la marque Benetton dans le monde entier
			\item[2003] : Le père Luciano cède l'entreprise à son jeune manager Silvano Cassano
		\end{description}
	\subsubsection{La chaîne de valeur}
	\subsubsection{La capacité stratégique}
		\begin{table}[!ht]
			\begin{center}
				\begin{tabular}{p{4cm}|p{5.5cm}|p{5.5cm}}
				& Ressources & Compétences \\ \hline
				Capacité seuil &
						\begin{itemize}
							\item Personnel ;
							\item Bureaux et entrepôts ;
							\item Système d'information ;
							\item Base de clients
						\end{itemize}
						&
						Compétences
						
						\begin{itemize}
							\item managériales ;
							\item commerciales ;
							\item informatiques
						\end{itemize}
				\\ \hline
				Capacité nécessaire pour obtenir un avantage concurrentiel
				&
						\begin{itemize}
							\item 90\% sous-traitants travaillent exclusivement pour Benetton
							\item Boutiques standardisées : 12 modèles d'agencement
							\item Système d'appel d'offre
							\item Valeur de la marque, notoriété internationale
						\end{itemize}
						&
						Innovation technologique : teinture en plongée
						
						Contrôles de qualité (échantillon, stockage et expédition)
						
						Gestion de l'information :
						\begin{itemize}
							\item Collecte : évolutions de la mode (agents) ;
							\item Traitement : réseau informatique, boutique/centre et usines/sous-traitants
						\end{itemize}
				\\
				\end{tabular}
			\end{center}
		\end{table}
	\subsubsection{L'avantage concurrentiel}
		Benetton a construit son AC sur trois innovations :

		\begin{enumerate}
			\item Innovation organisationnelle (structure en réseau)
			\item Innovation commerciale (boutiques en libre-service)
			\item Innovation technologique (teinture en plongée)
		\end{enumerate}

		Développer un avantage concurrentiel durable :

		\begin{tabular}{l|p{10cm}}
			\textbf{V}aleur & Gamme des couleurs\\
			\hline
			\textbf{R}areté & Cohésion sociale, structure familiale\\
			\hline
			\textbf{I}nimitabilité & Notoriété, copie du système logistique\\
			\hline
			\textbf{N}on-substituabilité & Nouvelles enseignes\\
		\end{tabular}

		Quid de la pérennité ? Ce qui est unique aujourd'hui risque de ne plus l'être demain !
	\subsubsection{SWOT}
		\begin{table}[!ht]
			\begin{center}
				\begin{tabular}{|>{\centering\arraybackslash}P{7.5cm}||>{\centering\arraybackslash}P{7.5cm}|}
				 \hline
				   \textbf{S}trengths (Forces) &  \textbf{W}eaknesses (Faiblesses) \\ \hline
				  \begin{itemize} 
					\item Externalisation maximale
					\item Marketing fort, partenariats
					\item Relations de confiance intimes
					\item Réseau logistique perfectionné
				  \end{itemize}
				  & 
				  \begin{itemize} 
					\item Benetton s'est enfermé dans son modèle gagnant
					\item Diminution de l'innovation
					\item Valeur de la marque battue par les campagnes trop violentes
					\item Système logistique obsolète (Zara)
				  \end{itemize}
				  \\  \hline
				   \textbf{O}pportunities (Opportunités) &  \textbf{T}hreats (Menaces) \\ \hline
				  \begin{itemize}
					\item Pénétration de marché
					\item Développement de nouveaux segments/marchés
					\item Nouvelles technologies
				  \end{itemize}
				  &
				  \begin{itemize}
				   \item Changement de génération (tissu social menacé)
				   \item \enquote{Fast Fashion}
				   \item Internet
				   \item Augmentation de la concurrence
				  \end{itemize}
				  \\ \hline
				  
				\end{tabular}
				\caption{Illustration de SWOT sur l'entreprise Benetton}
			\end{center}
		\end{table}
	\subsubsection{Plan de relance}
	\begin{itemize}
		\item Changement d'approche organisationnelle
			\subitem Rachat des principaux fournisseurs
			\subitem Ouverture d'usines implantées dans des pays à bas coûts de main d'oeuvre
			\subitem Remplace les petits boutiques indépendantes/mégastores possédés en propre
		\item Externalisation maximale
			\subitem Intégration verticale
	\end{itemize}
