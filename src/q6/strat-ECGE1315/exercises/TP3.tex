\subsection{Rappel théorique}
	\subsubsection{La segmentation}
	\begin{table}[!ht]
		\begin{center}
			\begin{tabular}{p{7.5cm}|p{7.5cm}}
			Segmentation marketing & Segmentation stratégique \\
			\begin{itemize}
				\item Concerne un secteur d'activité de l'entreprise
	 			\item Vise à diviser les acheteurs en groupes caractéristés par les mêmes besoins, les mêmes habitudes, les mêmes comportements d'achat.
	 			\item Permet d'adapter les produits aux consommateurs, de sélectionner les cibles privilégiées, de définir le marketing-mix.
	 			\item Provoque des changements à court et moyen terme
			\end{itemize}
			&
			\begin{itemize}
	 			\item Concerne les activités de l'entreprise dans son ensemble.
	 			\item Vise à diviser ces activités en groupes homogènes qui relèvent de la même technologie, des mêmes marchés, des mêmes concurrents.
	 			\item Permet de révéler les opportunités de création ou d'acquisition de nouvelles activités et les nécessités de développement ou d'abandon d'activités actuelles
	 			\item Provoque des changements à moyen et long terme
			\end{itemize}
			\\
			\end{tabular}
		\end{center}
	\end{table}
	\subsubsection{Identification des DAS}
	Un \textbf{domaine d'activité stratégique} est une sous-partie de l'organisation à laquelle il est possible d'allouer ou retirer des ressources de manière indépendante et qui correspond à une combinaison spécifique de facteurs clés de succès. \newline

	Un \textbf{domaine d'activité stratégique} est une partie de l'organisation pour laquelle il existe un marché extérieur de biens ou de services qui est \underline{différent} d'un autre domaine d'activité stratégique.

% 	\begin{tabular}{l|l|l}
% 		& Même DAS & DAS différents \\
% 		\hline
% 		Facteurs clés de succès & Même combinaison & Combinaisons différentes \\
% 		\hline
% 		\textbf{Critères externes} & & \\
% 		Clientèle & Mêmes clients & Clients différents \\
% 		Marché pertinent & Même marché & Marchés différents \\
% 		Distribution & Même réseau & Réseaux différents \\
% 		Concurrence & Mêmes concurrents & Concurrents différents \\
% 		\hline
% 		\textbf{Critères internes} & & \\
% 		Technologies & Identiques & Différentes \\
% 		Compétences & Identiques & Différentes \\
% 		Synergies & Fortes & Faibles \\
% 		Structure de coûts & Coûts partagés prépondérants & Coûts spécifiques prépondérants \\
% 		\hline
% 		\textbf{Chaîne de valeur} & Une seule chaîne de valeur & Plusieurs chîanes de valeur \\
% 		\hline
% 	\end{tabular}
\subsection{Application pratique : Club Med}

