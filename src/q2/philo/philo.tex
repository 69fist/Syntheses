\documentclass[11pt,a4paper]{article} % french

\usepackage[utf8]{inputenc}
\usepackage[frenchb]{babel}
\usepackage[T1]{fontenc}
\usepackage{lmodern}
\usepackage[pdfauthor={Benoît Legat},
pdftitle={Synthèse de Philosophie Q2 - LFSAB1802},
pdfsubject={Philosophie}]{hyperref}

\title{Synthèse Philosophie Q2 - LFSAB1802}
\author{Benoît Legat \and Julien Vaes}

\begin{document}
\maketitle

\part{Le monde gréco-romain}

Remarque : ces notes reprennent l'essentiel des développements doctrinaux dont il a été
question durant les deux premières séances du cours magistral ; elles ne se substituent
pas à vos notes personnelles, mais leur servent de support.

Une distinction s'impose entre être philosophe et parler de philosophie. Car on peut, hélas ! être professeur de philosophie sans être philosophe, de même que l'on peut parler
sans agir : le véritable philosophe se reconnaît moins à son langage qu'à sa vie qui révèle
(ou du moins s'efforce de révéler, puisqu'à défaut d'excellence, une intention qui vise à la
droiture sans y parvenir encore n'est pas méprisable) le sérieux de sa démarche.

Les penseurs antiques ont été particulièrement attentifs à cette distinction, et des satiristes
comme Lucien de Samosate (IIe siècle ap. J.-C.), grand pourfendeur de bouffons en tous
genres, ont très justement ridiculisé les soi-disant philosophes qui, en dépit de leurs
grands airs et de leurs beau discours, se comportaient de façon indigne. La valeur d'un
homme ne se mesure pas aux paroles qu'il profère, mais se laisse entrevoir dans ce qu'il
fait de sa vie, jour après jour. En ce sens, il n'est pas anodin que l'éminent biographe et
moraliste que fut Plutarque de Chéronée (Ier-IIe siècles ap. J.-C.), dans son œuvre
consacrée aux grandes figures de l'Antiquité gréco-romaine, mette l'accent moins sur les
grands exploits que sur les traits quotidiens permettant de voir qui ils étaient vraiment en
dehors des situations exceptionnelles, quand leur personnalité se donnait à voir avec la
plus grande netteté.

Parmi les philosophes qui ont également insisté sur ce point, nous pouvons encore citer le
nom de Cicéron (Ier siècle av. J.-C.), pour qui il est indispensable d'unir la théorie à la
pratique : la théorie doit s'épanouir dans la pratique, sous peine de n'être qu'un discours
creux, indigne de l'attention d'une personne sérieuse. Lui-même s'est d'ailleurs efforcé de
conjuguer l'excellence intellectuelle à la gestion des affaires en refusant d'isoler la pensée
de l'action. La constantia est à ce prix : la constantia, qui est ici moins la ``constance''
que la ``cohérence'' d'une démarche dont il serait absurde de vouloir séparer les deux
versants, théorique et pratique. Quoique distincts, on se gardera donc de les séparer : de
même que les doigts d'une main se distinguent les uns des autres, mais sont réunis dans
l'unité de la main qui interdit leur pure et simple séparation.

C'est que la philosophie s'intéresse aux conditions de la vie bonne, et non de la seule pensée droite. Au fond, une seule question a de l'importance : qu'est-ce qu'une vie réussie ?
C'est là très exactement le cœur du questionnement philosophique, qui se penche donc,
tout naturellement, sur l'identité de l'homme. En effet, la vie qu'il s'agit de réussir est
celle d'un être humain ; et il faut donc se demander ce qui fait l'homme, ce qu'il en est de
sa nature propre.

Quand les philosophes antiques parlent d'une vie réussie comme d'une vie vertueuse, il
ne faut pas se méprendre : la vertu n'est pas une affaire austère et rébarbative. Étymologiquement, ``vertu'' vient de virtus, terme latin qui désigne la qualité de ce qui fait de
l'homme (le mâle, vir) un homme digne de ce nom. Plus profondément, le terme grec
pour la vertu est aretê, qui veut dire exactement ``excellence''. Un homme vertueux est
donc, au sens premier, un homme excellent, un être humain qui s'est accompli comme
être humain.

L'homme qui, comme de juste, recherche le bonheur, cherche donc à être vertueux en ce
sens-là : à être excellemment l'être humain qu'il est. D'où, une fois encore, la nécessité de
répondre à cette question : qu'est-ce que l'homme ? qui rejoint l'injonction fameuse que
les Grecs inscrivirent sur le temple d'Apollon delphique : ``Connais toi toi-même'', sache
ce que c'est que d'être authentiquement un être humain. Pour connaître les conditions de
l'excellence d'une chose, il est en effet essentiel de pouvoir dire ce qu'est cette chose, car
qui me dira que doit pouvoir faire, par exemple, un bon marteau, si je ne sais pas ce qu'est
exactement un marteau, si la nature de ce que doit être un marteau m'échappe ?

Nombre de penseurs antiques, à la question ``qu'est-ce que l'homme ?'', vont répondre
que l'homme est homme par son âme. Évitons pourtant un grave malentendu ici : l'âme
s'oppose certes au corps, mais pas nécessairement comme l'immatériel au matériel. De
fait (au risque de la caricature), on peut dire que la très grande majorité des philosophes,
jusqu'au IIe siècle ap. J.-C., sont matérialistes. Matérialistes, parce que, pour eux, l'âme
est aussi matérielle que le corps, même si elle se présente sur un mode plus subtil. Une
comparaison sommaire mais efficace consisterait à dire que l'âme et le corps sont comme
des états différents d'une même substance physique : l'eau peut être solide, mais elle peut
aussi être gazeuse, et pourtant, c'est toujours d'eau qu'il s'agit ; de même, la matière peut
être dense, épaisse et visible à l'œil nu, ou ``raréfiée'', subtile et invisible pour l'œil, mais
pas le moins du monde immatérielle pour cela.

C'est un point auquel on doit être attentif quand on parle de philosophie, et, de manière
générale, quand on aborde une question d'un point de vue diachronique ou que l'on
s'efforce de restituer le contenu d'une pensée dans une autre langue : derrière un même
mot peuvent se cacher des idées, des nuances, des ``colorations'' bien distinctes. Pour
prendre un exemple, lorsque le philosophe et poète épicurien latin Lucrèce (Ier siècle av.
J.-C.) parle d'âme répandue dans tout le corps et d'esprit situé en un lieu précis de ce
corps, on peut être tenté, au premier abord, de se dire qu'on est là devant une conception
``périmée'' de l'être humain. Pourtant, ce que Lucrèce appelle ``âme'' correspond assez
bien à ce que nous nommons ``système nerveux central'' ; et ce qui est pour lui
``l'esprit'' renvoie, mutatis mutandis, aux processus chimiques que nous découvrons dans
le cerveau. Ainsi, derrière un vocabulaire qui paraît suranné, se cache parfois quelque
chose qui, traduit en termes plus ``actuels'', conserve toute sa pertinence.

D'où vient alors l'opposition entre corps-matière et âme immatérielle ? De Platon, entre
autres et en particulier : pour lui, l'âme relève d'un ordre de réalité irréductible à la matière. Il ne s'agit donc pas d'une matière plus subtile, mais de quelque chose de
radicalement différent. Cette idée, qu'il existe quelque chose de ``supérieur'' à la matière,
de totalement autre qu'elle, a été défendue par Aristote, disciple de Platon, qui, bien qu'il
ait largement infléchi la doctrine de son maître (car leurs vues respectives s'opposent sur
bien des points), lui a néanmoins été fidèle sur ce point en particulier : la matière n'épuise
pas le sens du réel, dont il faut pouvoir rendre compte en recourant à une catégorie de
réalité plus profonde, l'immatériel ou suprasensible. Mais ces deux penseurs, il faut y
insister, font longtemps figure d'exceptions sur cette question bien particulière.

L'homme est donc d'abord et avant tout son âme, selon les philosophes antiques. Et pour
la plupart d'entre eux pendant plusieurs siècles, cette âme, bien que spécifiquement distincte du corps avec lequel elle a partie liée durant cette vie, n'en est pas moins matérielle.
Notons encore que le nom latin de l'âme est animus (ou anima, mais ceci nous entraînerait trop loin, retenons seulement que les deux termes permettent une série de nuances),
c'est-à-dire ``souffle'' vital, exactement : principe d'animation qui fait de l'être qui en est
doté un être animé ou, cela revient au même, vivant.

Une fois encore, qu'il n'y ait pas de malentendu : un être vivant, précisément parce qu'il
est vivant, possède une âme. L'arbre, le chat et l'être humain ont tous une âme. Car
``âme'' ne veut pas dire autre chose que cela : principe qui fait de l'être qu'il anime un
être vivant. Maintenant, une âme n'est pas une autre âme ; et ce sont leurs fonctions qui
distinguent les âmes. Sans dogmatiser ici sur des frontières sans doute moins étanches que
ne le pensaient les philosophes antiques, on peut discerner trois grands ``types'' d'âme :
l'âme végétative, la sensitive et l'intellective ou intellectuelle.

Une âme végétative est celle d'un végétal ; elle n'est autre que le principe qui rend raison
du fait que ce végétal est bien un être vivant et que, à ce titre, il possède les fonctions ``de
base'' du vivant : nutrition, croissance et reproduction. Si nous considérons ensuite
l'animal, nous voyons qu'il possède d'autres fonctions (en plus des fonctions de base susdites, bien entendu), qui relèvent de sa sensibilité et qu'un arbre ne possède pas : vue,
ouïe, etc. Enfin, à ces fonctions s'ajoute, chez l'homme, une capacité d'abstraction, de
penser non plus seulement des réalités concrètes individuelles, mais se s'élever à des
concepts, d'appréhender des notions universelles : on parle d'une fonction intellective ou
intellectuelle.

Si nous revenons à présent à la question ``qu'est-ce que l'homme ?'', nous pouvons, selon l'optique des philosophes antiques, répondre en disant qu'il s'agit d'un vivant dont la
spécificité est le caractère intellectuel, rationnel de son âme.
Le propre de l'homme, c'est-à-dire ce qui le constitue comme un vivant nettement distinct de tous les autres êtres animés, c'est que l'âme qui le fait homme possède cette capacité abstractive, intellective, qui
permet de le définir comme animal (animé) rationnel.

S'ensuit une conséquence capitale pour la philosophie : si l'homme est ainsi qualifié, la
vertu, l'excellence donc, consiste pour lui à se perfectionner selon cette nature rationnelle
qu'il possède en propre. Autrement dit, la vertu est raison. Une vie humainement réussie
est une vie conforme à ce que dicte la raison, qui fait de nous ce que nous sommes en vérité : des êtres humains, distincts des autres vivants.

Chaque être a donc une excellence propre, une vertu propre, qui consiste à mener une vie
en accord avec sa nature. Pour l'homme, il s'agit de mener une vie ``en conformité'' avec
cette nature rationnelle qui est la sienne. Ce point de vue a été développé avec une force
toute particulière par les philosophes stoïciens ; pour eux, non seulement la vertu est raison, mais la réciproque est vraie aussi : la raison est vertu.
\begin{center}*\end{center}
Ce nom de ``Stoïciens'' leur vient de ce que leurs premiers maîtres, à Athènes, enseignaient à l'ombre d'un portique public, c'est-à-dire d'une galerie couverte, stoa en grec.
Fondée à la fin du IVe siècle av. J.-C. par Zénon de Citium, cette école philosophique
s'est particulièrement bien acclimatée parmi les élites dirigeantes du monde romain dès le
IIe siècle av. J.-C., lorsque la cité de Romulus a largement étendu son emprise sur
l'ensemble du bassin méditerranéen. Les plus fameux représentants de l'école à l'époque
impériale sont trois personnages remarquables : Sénèque, précepteur impérial et sénateur
romain (Ier siècle ap. J.-C.), Épictète, ancien esclave et enseignant affranchi (Ier-IIe siècles ap. J.-C.), et Marc Aurèle, qui présida personnellement aux destinées du monde
romain, puisqu'il régna comme empereur pendant près de vingt ans (IIe siècle ap. J.-C.).

Les Stoïciens insistent sur la distinction entre l'âme et le corps, tout en conservant le parti
pris matérialiste dont il a déjà été question. Le ``bon fonctionnement'' de l'âme s'appelle
vertu quand elle suit les préceptes de la raison et se conforme à la rationalité totale qui
gouverne l'univers. Car l'univers, dans son ensemble, est rationnel à leurs yeux, et tout
s'y déploie d'après une logique totale, providentielle et immanente ; et l'homme est une
partie de ce tout rationnel, qui doit s'efforcer de correspondre à la marche générale de
l'ensemble, sous peine d'être entraîné malgré lui par le cours inéluctable du destin :
comme le dit Sénèque, ``les destins guident ceux qui les suivent, et traînent derrière eux
ceux qui se rebiffent''.

Pourtant, ce bon ordre des choses est gravement perturbé par les passions. Littéralement,
étymologiquement, les passions, pathê en grec, sont des choses que l'on subit, dont on
n'est pas le maître, sur lesquelles la raison n'exerce pas ou pas assez son contrôle : un
accès de panique, un désordre amoureux, etc. Ce sont très exactement des maladies mentales, selon la traduction latine que propose Cicéron, perturbationes animi : quand la
raison ne fonctionne pas bien, c'est l'humanité de l'homme qui est en péril. En effet, si ce
qui fait que l'homme est homme est la raison, alors, quand celle-ci est dominée par des
pulsions, des passions dont les mouvements lui échappent, l'homme déchoit en quelque
sorte de son humanité.

Là encore, il est essentiel de ne pas se méprendre : à une époque où nous valorisons (sottement) la passion, ce discours peut paraître rébarbatif. Pourtant, il s'insère dans la suite
logique de ce qui précède : la passion n'est pas quelque chose de beau et d'exaltant, mais
elle est ce qui nous rend étrangers à notre propre humanité, elle est un esclavage qui nous
arrache à notre véritable condition, un viol de notre intégrité d'êtres humains. La passion
signifie la défaite de ce qui fait de nous des êtres libres, puisqu'elle nous rend esclaves de
ce sur quoi nous n'avons pas de prise. La passion, c'est l'échec de l'humanité, un raté de
l'agir moral, le signe d'une déficience mentale que l'on doit tout mettre en œuvre pour
éradiquer. Car il ne saurait être question d'atténuer simplement les passions, pour les
Stoïciens, qui, sur ce point, tiennent un discours sans concession : la passion doit être éliminée purement et simplement.

Le sage, idéal stoïcien de l'homme absolument vertueux, c'est-à-dire pleinement réalisé
comme être humain, est-il donc semblable à une machine dénuée de sentiments ? Nullement, en dépit des apparences et de ce que suggère un de ces faux amis que l'on se
gardera de traduire littéralement : l'apatheia des philosophes du Portique n'est pas
l'apathie, une léthargie qui rendrait l'homme semblable à une statue. Au contraire, cette
``absence de passions'' qu'ils préconisent va de pair avec des émotions destinées à les
remplacer, des émotions raisonnables, et donc vraiment humaines, qui ne sont telles que
parce qu'elles sont placées sous le contrôle de notre raison. L'exultation doit céder la
place à une joie tranquille, la peur panique à une précaution bien conduite, la passion
amoureuse à une affection sincère et ainsi de suite.

Puis donc que la vertu est raison, c'est que le vice est déraison et folie. Les Stoïciens,
comme d'autres penseurs antiques, sont en ce sens des intellectualistes. Comprenons toutefois que cela ne signifie pas que l'homme idéal est un raisonneur enfermé dans sa tour
d'ivoire ! Cela veut seulement dire qu'être vertueux, pour un être humain, consiste à vivre
au niveau de l'intelligence comme expression de la raison. Le mal est donc de l'ordre du
dysfonctionnement de la raison, de l'ignorance. En disant cela, les Stoïciens reprennent
une fameuse affirmation de Socrate (Ve siècle av. J.-C.), le maître de Platon : nul n'est
délibérément mauvais.

Le mal est ignorance, car on ne peut pas connaître ce qui est bon et ne pas le vouloir.
Deux remarques de toute première importance doivent être faites ici. La première, c'est
que cette affirmation n'ôte pas la responsabilité du mal à celui qui le commet :
l'ignorance peut être coupable. L'autre, c'est que les anciens philosophes et nous ne
concevons pas la volonté de manière identique. En effet, nous sommes habitués à penser
qu'on peut fort bien savoir quelque chose sans agir en conséquence, parce que la volonté
``fonctionne'' indépendamment de l'intelligence. Or cette idée d'une volonté indépendante de l'intelligence n'est pas née tout d'un coup, mais a été progressivement élaborée à
la fin de l'Antiquité. Pour dire les choses simplement, un philosophe antique considère
que la volonté suit nécessairement l'intelligence.

Comment expliquer alors que l'on sait certaines choses, et que pourtant la volonté,
comme on dit, ne suit pas ? C'est ce qu'on appelle, en grec, la question de l'akrasia. En
réalité, disent les Stoïciens, qui sont évidemment bien conscients de cette tension et
connaissent les affres d'une conscience que nous dirions ``déchirée'' entre ce qu'elle sait
devoir faire et ce qu'elle veut faire, c'est qu'on prétend savoir, mais qu'en fait, on ne sait
pas vraiment, au sens où l'on n'a pas vraiment intériorisé ce savoir, on ne l'a pas vraiment fait sien. Ce savoir que l'on dit avoir est donc superficiel, une sorte d'épiphénomène
mental ou seulement langagier, de même qu'on peut répéter des syllabes sans en mesurer
la portée ou sans en comprendre le sens ; pas étonnant, dans ces conditions, que l'agir,
que la volonté ne suive pas : en réalité, ce que nous sommes tentés d'appeler ``faiblesse
de la volonté'' n'est pas autre chose qu'une faiblesse... de l'intelligence.

Il faut ici se rappeler le point d'où nous sommes partis : l'action doit être le prolongement
du savoir. Cela se comprend à présent beaucoup mieux, puisque, on vient de le voir, le
bon fonctionnement de la raison, qui fait de nous des êtres humains, consiste à dissiper
l'ignorance, et donc le mal, pour connaître le vrai. Si cette connaissance est véritable et
qu'on la fait vraiment sienne, la volonté sera entraînée par cette intelligence saine, et
l'action suivra la connaissance, de sorte que l'on agira comme on pense.

Or, remarquons-le bien, ce qui vient d'être dit concerne l'être humain en tant que tel,
c'est-à-dire indépendamment de sa race, de son rang social et de son sexe. Cela nous paraît aller de soi aujourd'hui, mais cette idée d'universalité, qui ne s'est imposée largement
qu'à une date très récente (et encore, pas partout) est assez révolutionnaire à l'époque où
les Stoïciens la défendent. Bien qu'ils ne rejettent pas l'esclavage – Épictète lui-même,
qui a été esclave, ne s'oppose pas à cette institution qui, dans l'Antiquité, s'impose
comme une évidence socioculturelle –, les philosophes du Portique affirment quelque
chose de beaucoup plus fondamental : l'esclavage juridique, institutionnalisé, ne concerne
que les corps ; les âmes, elles, sont naturellement libres, et le seul esclavage qui peut les
atteindre est celui de la déraison, des passions. Le véritable esclave n'est donc pas celui
dont une chaîne entrave le corps, mais celui dont l'âme malade est le jouet de ses propres
passions. Partant, il y a des empereurs qui sont esclaves de leur folie et de leurs vices,
mais il y a des gens dont on vend le corps au marché, mais dont l'âme est libre.

On pourrait croire qu'il s'agit là de beaux mots pour échapper à une condition difficile à
assumer au quotidien, mais on voit clairement, d'après ce qui précède, qu'il s'agit de bien
plus que cela, puisque ce n'est pas par son corps, mais par son âme que l'homme est
homme. Le Stoïcisme déclare en conséquence que l'homme, qui possède en lui-même les
ressources pour se libérer de tout esclavage véritable en rétablissant les droits de la raison
sur sa vie, suit très exactement le cours de la rationalité cosmique elle-même : entre le
sage, même si c'est une femme barbare que l'on vend sur un marché (une triple tare selon
les habitudes de pensée qui ont généralement cours dans le monde antique – on peut le
regretter, mais c'est ainsi), et Jupiter, le roi des dieux et la rationalité de l'univers lui-même, il n'y a fondamentalement aucune différence.

L'éthique stoïcienne, exigeante certes mais nullement inhumaine – que du contraire ! –, se
présente ainsi comme une dévotion totale à la raison qui réside en tout être humain, et par
laquelle celui-ci peut communier à l'ordre d'un monde dans lequel il a sa place. Les désagréments, les peines et les souffrances, si pénibles soient-ils, ne sont pas des maux,
puisqu'ils n'enchaînent pas ce par quoi nous sommes des hommes, libres et invulnérables
à défaut, peut-être, d'être immortels : notre âme. Être libre, vertueux, et (cela revient au
même) heureux, tout cela est en notre pouvoir et à portée de chacun de nous, pour peu que
nous prenions fermement la décision de l'être.

\part{Le monde chinois ancien}

Remarque à propos de la translittération du chinois : plusieurs systèmes existent, adaptés
aux langues des érudits qui les ont établis.
Vers le milieu XXe siècle, le gouvernement
chinois a défini le système de transcription pinyin (littéralement ``épeler les sons''), qui a
depuis été largement adopté par la communauté internationale, et que je suivrai donc ici ;
voilà pourquoi il sera question, par exemple, de dao et non de tao, qui est la transcription
dite de l'École Française d'Extrême-Orient.\\

La notion de ``philosophie'' étant liée, pour partie, à la civilisation grecque au sein de
laquelle elle née, et à la pensée Occidentale, héritière de l'hellénisme, il est délicat
d'employer ce terme sans autre précision pour évoquer les réalités du monde chinois.
Non
que la philosophie soit absente de la Chine, bien entendu, mais tout simplement parce que
la délimitation stricte d'un champ du savoir relevant de la philosophie comme nous la
comprenons (et encore, pas de façon univoque) est étrangère à l'histoire culturelle de
l'Empire du milieu : le terme par lequel on désigne la philosophie en chinois a été forgé
au XIXe siècle, et est, du point de vue de la signification, un calque du nom grec : zhexue,
``étude de la sagesse''.

Jusque là, ceux que nous appelons philosophes se considéraient plus exactement comme
des pédagogues, des lettrés (ru), dépositaires de la culture (wen).
Confucius, qui vécut au
tournant du Ve siècle avant notre ère, était un de ces lettrés, désireux de revivifier les
valeurs du temps passé.
Pour faire bref, à une époque où le pouvoir central des Zhou ne
jouissait plus que d'une autorité théorique, où la réalité du pouvoir était entre les mains de
grands vassaux et d'une multitude de principautés rivales qui ne cessaient de se faire la
guerre, Confucius souhaitait que l'on réinvestît les vertus des anciens sages et rois de
jadis, qui avaient su apporter au monde chinois sa prospérité et sa paix.

Confucius n'avait donc pas d'ambition proprement novatrice.
Et pourtant : en travaillant à
la restauration des valeurs de l'ancienne aristocratie féodale, il les transforma de
l'intérieur : le modèle du junzi, de ``la personne de qualité'' (comme nous disions
autrefois un ``gentilhomme'') devenait accessible à tout un chacun, indépendamment de son
rang social : une aristocratie du mérite devait prendre le pas sur celle du sang.
Pour devenir
cet homme de bien, la voie royale est l'étude (xue), qu'il faut entendre au sens large
d'``apprentissage'', car il ne s'agit pas tant d'être penché sur ses livres que de développer
sa personnalité morale.
Voilà pourquoi, selon le mot de Confucius, tout le monde a
quelque chose à nous apprendre : les bons nous apprennent à imiter leurs vertus ; les
méchants, à ne pas vouloir leur ressembler.

Les livres, pour autant, ne sont pas exclus – que du contraire : la tradition confucéenne
accordera une importance capitale à des écrits ``classiques'' (relevant de genre littéraires
variés, un peu comme on trouve, dans la Bible, des écrits poétiques, des livres historiques,
recueils gnomiques, histoires édifiantes, etc.), qui illustrent cette voie (dao, le terme
appartient à la langue courante et n'est pas l'apanage d'une tradition philosophique en
particulier) des anciens rois.
Parmi ces derniers, mentionnons au moins la personnalité de
Shun, figure légendaire de la fin du IIIe millénaire av. J.-C., qui est exemplaire à plusieurs
égards.
Le père de Shun s'était remarié après la mort de son épouse ; sa nouvelle femme
comme le fils de celle-ci haïssaient le jeune homme, et le vieux père en était venu à
partager, lui aussi, ces sentiments hostiles.
Malgré d'incessantes vexations et même tentatives
d'assassinat (!), Shun continua de témoigner à son indigne famille une piété filiale (xiao)
exemplaire, qui le distingua tant et si bien qu'il finit par être désigné pour monter sur le
trône.
Finalement, sa vertu triompha des sentiments contraires des gens de sa famille, qui
revinrent de leurs égarements.

Cette histoire édifiante met en avant deux éléments de première importance.
D'abord, la
piété filiale est la vertu chinoise traditionnelle par excellence, à un degré qui nous stupéfie
(car notre tradition privilégie, en dernière analyse, la justice ; et, si nous honorons la piété
filiale, elle ne constitue pas un absolu).
Ensuite, la vertu (de) telle que la conçoivent les
Chinois relève certes de la catégorie d'excellence, mais surtout, il faut la rapprocher du
``charisme'', d'une sorte d'aura magique qui exerce son influence autour d'elle à la manière
dont un parfum diffuse sa fragrance : Shun n'agit pas directement sur son entourage
pour le transformer, mais se contente de développer la vertu qui doit être la sienne en tant
que fils de la famille ; c'est par elle-même, ensuite, que cette vertu triomphe des mauvais
penchants de l'entourage de Shun, comme la bonne odeur, si elle est suffisamment
``forte'', transforme l'atmosphère du lieu où elle est répandue.

Confucius, dans la droite ligne de cette tradition, estime que le premier devoir d'un bon
gouvernant consiste à ``rectifier les noms'' (zhengming), c'est-à-dire à veiller à ce que
chacun agisse conformément au statut qui le désigne et lui assigne sa place dans la société.
Pour peu que chacun s'occupe d'être excellemment celui qu'il prétend être, l'harmonie
se rétablira d'elle-même.
Le souverain, ses ministres, les pères, leurs fils, tous doivent
connaître la place qui est la leur, sans chercher à obtenir celle qui ne leur revient pas ; s'ils
apprennent à être en vérité ceux qu'ils devraient être – un véritable souverain est un souverain
qui veille au bien de son peuple ; un véritable serviteur ne cherche pas à évincer
son supérieur, mais l'épaule de toutes ses forces ; etc. –, le monde connaîtra la paix, et les
barbares eux-mêmes, à ce spectacle, voudront se convertir aux bienfaits de la civilisation.

Parmi les vertus qui font l'homme de bien et qui s'apprennent au contact des hommes et
de l'exemple des anciens sages, distinguons-en deux en particulier : le sens des rites (li) et
l'humanité (ren).
Celle-ci, d'abord, est la vertu suprême : le sinogramme qui la désigne
est formé de la clé de l'homme (qui se dit également ren) et du nombre 2, de sorte que,
selon ce que suggère immédiatement la lecture du terme qui la désigne, la vertu
d'humanité est celle qui se donne carrière quand un homme de bien interagit avec ses
semblables ; et de là cette définition (parmi plusieurs autres) qu'en donne Confucius lui-même :
``L'humanité consiste à aimer les gens'' (ren ai ren).
Naturellement, cette bienveillance
de l'homme de qualité dans ses relations avec les autres hommes ne se déploie
pas de manière anarchique.
Ici interviennent les ``rites'' (li), les convenances et conventions
héritées de la tradition.
Or ces usages, qui concernent aussi bien le protocole
sophistiqué de la cour que la manière de s'habiller ou de se tenir à table au jour le jour,
valent par eux-mêmes : les convenances ne sont pas des normes arbitraires, mais possèdent
une valeur objective et sacrée (même si le terme ne convient pas parfaitement,
puisqu'il suppose une distinction sacré/profane typiquement occidentale), de même que
les préceptes quotidiens qui structurent la vie des Juifs orthodoxes et qu'on peut lire, en
particulier, dans le Lévitique et le Deutéronome, sont sacrés à leurs yeux et régissent les
moindres faits et gestes de leur quotidien.
Ces conventions rituelles, qui concernent tous
les aspects de la vie, ne forment un carcan étouffant que pour l'observateur extérieur, car
ils doivent, chez celui qui apprend à les faire siens, devenir une seconde nature, celle de
l'homme qui se comporte comme un être humain digne de ce nom au regard de la tradition qu'il assume et représente.

Le message de Confucius est ainsi une pédagogie de l'excellence individuelle au service
de la communauté humaine, au sein de laquelle chaque homme a un rôle à jouer à la place
qui lui revient.
Mencius (IVe siècle av. J.-C. ; on dit Mencius ou Mengzi, comme Confucius
ou Kongzi, selon que l'on privilégie la forme latinisée sous laquelle ces deux
penseurs sont connus en Occident depuis les XVIIe et XVIIIe siècle, ou bien la transcription
pinyin) est le plus fameux des continuateurs de l'œuvre de moralisation entreprise par
Confucius.
Sa lecture de l'éthique confucéenne se fonde sur un pari : Mencius veut croire
en la bonté de l'homme.
Non que l'homme soit bon dès le départ, mais il possède en lui
les germes (duan, entre autres termes) de la bonté, des jeunes pousses qui, moyennant des
soins appropriés, deviendront des vertus au sens propre.
Nous ne sommes pas une terre
vierge, mais nous sommes d'entrée de jeu orientés vers la vertu, vers la moralité.
Et nous
parviendrons à ce vers quoi nous porte notre nature, mais seulement à condition
d'entretenir avec soin ces germes, pour les développer sans les négliger ni forcer leur
croissance.
Il est constant, en effet, que celui qui arrose trop ses plantes les noie ; et, selon
le célèbre apologue mencien, le fermier qui, pour accélérer sa récolte ``aida'' les jeunes
pousses à grandir en les tirant impatiemment vers le haut vit périr ses espoirs en même
temps que les germes déracinés...

Ce pari sur la bonté de l'homme s'appuie sur des faits d'expérience qui laissent entrevoir
l'existence, en nous, des premiers commencements de la vertu : supposez que, à
l'occasion d'une promenade, seul, vous aperceviez un enfant risquant de tomber dans un
puits.
Avant même de juger de ce que vous ferez (ou ne ferez pas...) pour l'aider, vous
éprouvez comme un pincement au cœur, qui montre que la compassion n'est pas absente
de votre âme.
Permettrez-vous à ce germe de compassion de devenir progressivement la
vertu d'humanité, ou étoufferez-vous ce premier mouvement de votre âme ? Dans un cas
comme dans l'autre, vous avez ressenti le serrement, qui prouve que vous êtes ``fait'',
naturellement, pour la vertu et la bonté dont cette émotion est un prodrome.
De même,
vous découvrirez que vous êtes fait pour l'équité, le rite ou la convenance sociale, et la
sagesse.
Tout homme, hélas ! ne deviendra pas vertueux (même une terre fertile ne rapporte
rien si un cultivateur incapable ou malveillant la maltraite, si les conditions
extérieures sont déplorables, etc.), mais tout homme est par nature tourné vers ces vertus :
la sainteté est dans l'ordre des choses, alors que la perversité est contre-nature.

Un important corollaire de ce qui précède, c'est que tous les caractères qui signalent une
personne vertueuse, en ce compris, donc, les convenances rituelles et tout ce qui est de
l'ordre du construit dans les rapports humains, tout cela suppose qu'il n'y a pas de rupture
entre la nature et la culture.
La culture, ce n'est rien d'autre que le développement de la nature bien comprise.
D'autres voix se font entendre pourtant, parmi les penseurs chinois,
qui se méfient du construit culturel, et dénoncent le dévoiement ``civilisé'' de la nature :
ce sont notamment les Daoïstes.
\begin{center}*\end{center}
Il est délicat de parler de ``daoïsme'', car l'emploi d'un vocable unique suggère une
continuité d'école, une communauté de valeurs, etc.
Or, le ``daoïsme'' connaît un grand
nombre de variantes, selon les époques et les auteurs considérés : entre l'alchimiste
préoccupé par le ``raffinement du cinabre'' et la confection d'une drogue d'immortalité, le
stratège se réclamant de Laozi, le mystique et philosophe du langage Zhuangzi, les
prêtres-exorcistes de l'Unité orthodoxe, les disparités sont souvent plus patentes que les
similitudes.
Conservons donc l'étiquette ``daoïste'' en ayant à l'esprit l'extrême diversité
de ce qu'elle recouvre.

Le texte le plus fameux du ``courant'' est le fameux Daodejing, le Classique de la voie et
de la vertu, attribué à Laozi.
Or ce texte n'est probablement pas né du jour au lendemain,
mais résulte plutôt d'une tradition qui a fini par se fixer, peut-être dans le courant du IVe
siècle av. J.-C., durant la période dite des ``Royaumes combattants'' (Ve-IIIe siècles av.
J.-C., jusqu'à l'unification sous l'égide de Qin Shi Huangdi, le ``premier auguste empereur
de Qin'').
L'histoire de Laozi, sage fonctionnaire dégoûté par la conduite des
puissants, qui aurait dicté son œuvre avant de partir définitivement vers l'Occident, relève
de la légende.
Mais une légende dont on redira l'importance en évoquant, plus tard,
l'arrivée du bouddhisme en Chine à partir du Ier siècle de notre ère, sous les Han postérieurs.

Ce court texte en prose rythmée, à la manière d'une mystérieuse incantation qui n'est pas
sans évoquer les ruminations divinatoires des chamans, est probablement le plus traduit
des textes chinois dans le monde occidental.
La cause de son succès ? Son extrême diffusion
dans le monde chinois, assurément, mais aussi son caractère ``crypté'', qui le prête à
une foule d'interprétations : s'agit-il d'un manuel de gouvernance, d'un poème mystique,
d'un traité de stratégie, de métaphysique, d'hygiène, de...
? Un peu de tout cela à la fois,
sans doute, avec cependant deux éléments très clairs : un refus de la course aux honneurs
et de l'activisme prôné par les disciples méritocrates de Mozi (dont nous ne parlerons pas
ici), d'une part, et un rejet du formalisme confucéen, de l'autre.

Laozi recommande de se rendre semblable à l'eau, qui paraît faible, puisqu'elle est
contrainte d'adopter la forme que lui impose ce à quoi elle vient se heurter et qui lui dicte
la configuration à adopter, mais qui, en définitive, l'emporte sur les montagnes elles-mêmes
, puisqu'elle est capable de les creuser.
Cette idée de la force du faible conduit à
tirer profit des ressources, de l'énergie à disposition dans l'univers, plutôt que d'opposer
des intentions personnelles limitées aux choses, et de disperser en vain son énergie propre.
C'est la doctrine du ``non-agir'' (wuwei), que l'on se gardera bien de confondre avec
l'apathie : le non-agir n'est pas l'inertie passive, mais le refus du dirigisme, de la
contrainte intentionnelle et de l'affirmation du moi.
Or rien n'est plus efficace que cette
détermination à accompagner le cours des choses sans chercher à s'affirmer égoïstement.

L'un des plus remarquables penseurs daoïstes, qui vécut au IVe siècle av.
J.-C., est
Zhuangzi.
Parmi les thèmes récurrents de son œuvre, nous retiendrons en priorité son désintérêt pour les affaires politiques et sa critique du langage.
Tel l'oiseau mythique ne se
nourrissant que d'aliments choisis et dédaignant le rat mort dont se gave un oiseau aux
goûts moins relevés, le sage se moque des fonctions publiques, inutiles et fastidieuses.
Voire extrêmement dangereuses, puisque les trésors que l'on obtient en se mettant au service des puissants sont comme les perles que garde le dragon, et dont on ne se saisit
jamais sans courir un grand risque pour sa vie.
Du reste, la vie à la cour est bien étrangère
à la simplicité naturelle qui convient mieux à l'homme, qui, à l'instar de la tortue, fera
bien de préférer traîner sa queue dans la boue plutôt que de savoir combien sa carapace
sera précieuse, après sa mort, dans le trésor de quelque potentat étranger.

Au fond, quand nous admirons ce qui menace l'équilibre de notre existence, et notre vie
elle-même, nous nous trompons de perspective, nous nous abusons nous-mêmes en ne
comprenant pas la vraie nature de nos besoins.
Ceci relève de l'étude du langage à laquelle se livre Zhuangzi, et qui, pour l'essentiel, revient à affirmer tout ensemble que le
langage est certes un outil précieux (que ce philosophe maniait d'ailleurs en virtuose),
mais que nous avons une fâcheuse tendance à nous laisser éblouir par l'éclat des mots.

Car, au fond, le danger du langage, c'est qu'en disant le monde, il le fige dans des catégories toutes faites, il le découpe, introduisant l'artificiel où il n'a pas sa place.
Plus grave
encore, alors que notre langage reflète notre propre rapport au monde, nous sommes tentés d'absolutiser des catégories relatives, en parlant au nom de l'universel quand nous
n'avons pourtant qu'un point de vue limité sur lui.
À cet égard, nous ressemblons à la
grenouille qui passe sa vie au fond d'un puits, et croit que le monde à sa portée dans ce
puits est le tout de l'univers, et qu'elle connaît le ciel, alors que, si elle sortait seulement
du puits, elle verrait qu'il est infiniment plus vaste que ce qu'elle en avait vu jusque là.

Et quand bien même notre perception serait juste, elle ne le serait que relativement à
nous : dormir dans un endroit humide n'est pas une bonne idée...
indiscutablement, mais
seulement pour un homme ; si le poisson pouvait parler, il dirait le contraire et serait bien
à plaindre s'il devait dormir au sec.
Le bien, l'utile, le beau : ces catégories ne valent que
pour nous, et non dans l'absolu : quand une belle demoiselle se présente, l'homme la regarde tout autrement que le tigre, et le poisson plonge se cacher...
Personne n'a tort dans
cette histoire, sinon celui qui affirmerait que, seul, il a raison.
Bien malin, en effet, celui
qui pourrait dire quelle doit être la référence, la norme d'après laquelle il serait permis de
parler au nom de l'absolu.
Faute de pouvoir adopter un autre point de vue que le nôtre,
voilà que nous raisonnons comme s'il n'en existait pas d'autre ; et si nous pouvions changer, nous ririons sûrement de notre ancienne sottise, comme cette femme qui se désolait
en apprenant qu'elle devait quitter la maison de ses parents pour se marier.
Une fois mariée, pourtant, elle mena la vie de château ; et cette situation aussi agréable qu'inattendue
lui fit prendre conscience de ce que son ancienne frayeur reposait seulement sur
l'ignorance de ce qu'allait être sa condition future.

Dans ces conditions, le mieux que nous puissions faire est d'étendre les limites de notre
perspective, pour essayer de penser par-delà les barrières que nous dressons initialement
en y enfermant notre prétendu savoir.
Nous devons ressembler à cet homme qui acheta un
jour un baume pour soigner les engelures.
L'inventeur avait limité ce baume à son propre
usage, pour se garder de la morsure du froid quand il exerçait son métier de laveur de
soie.
Il crut faire une bonne affaire en vendant la recette, mais l'acquéreur destina le
baume à un usage plus considérable, en soignant les marins de la flotte de son maître, qui
put ainsi mettre en déroute l'armée adverse.
Cette application ingénieuse lui valut de
grandes récompenses, qu'il dut en définitive à sa largeur de vues, puisqu'il avait su adapter à d'autres usages le bien qu'il avait acquis.

C'est là ce que nous devons faire avec le langage : non pas le rejeter en raison des dangers
qu'il recèle, mais tirer parti de ses ressources sans se laisser abuser par lui.
Car les choses ne sont pas toujours ce qu'elles paraissent, ou du moins ne le sont-elles pas assurément,
selon l'apologue célèbre du rêve du papillon : Zhuangzi avait rêvé qu'il était un papillon,
puis il s'éveilla et se rendit compte que cela avait été un rêve...
à moins qu'il ne fût réellement un papillon rêvant seulement qu'il était Zhuangzi ?

Le langage crée un univers artificiel, que l'on doit dominer pour qu'il ne nous domine pas.
Mais cette maîtrise dont parle Zhuangzi n'est pas une mainmise par laquelle l'homme
affirmerait son contrôle sur les choses.
Rejoignant Laozi, il pense que l'homme sage ressemble à de l'eau, et que, loin de chercher à affirmer sa spécificité, ses préférences, ses
intentions, il doit s'oublier en quelque sorte, pour atteindre au ziran, littéralement le ``de
soi-même ainsi'', le spontané.
On comprend que cette spontanéité que vise le philosophe
n'a strictement rien de commun avec le ``cri du cœur'' romantique, puisqu'il s'agit en fait
de s'effacer comme individu pour prendre part à la marche du grand tout sans vouloir
l'insérer de force dans des catégories limitées.
Cette spontanéité est donc celle d'un acquis
devenu une seconde nature, comme l'est l'habileté prodigieuse du grand pianiste dont les
doigts glissent tout naturellement sur les touches de l'instrument.

L'homme accompli est celui qui, par une discipline personnelle que Zhuangzi appelle le
``jeûne du cœur'' (xinzhai), a su devenir un miroir où viennent se refléter les choses, suivant un métaphore qui connaîtra un succès considérable, chez les Bouddhistes aussi bien
que chez les néo-Confucianistes à partir des Song, au XIe siècle.

\part{Bouddhisme et néo-Confucianisme}

Lorsque nous parlons du Bouddhisme, de quoi parlons-nous ? d'une religion, d'une sagesse ou d'une philosophie ? Une fois encore, nous éprouvons ici les limites de
catégories trop étanches ; et l'on peut d'ailleurs poser la même question à propos du
Christianisme, que certains de ses premiers disciples, aux IIe et IIIe siècles en particulier, présentaient volontiers comme une philo-sophie, puisqu'il s'agit, en vérité, d'aimer
la Sagesse éternelle et incarnée, et que le Christ est le Logos, c'est-à-dire la Raison
même de toutes choses.
Retenons donc ici l'idée, très importante, que, pour être commodes et dotées d'une indiscutable pertinence, les catégories qui nous permettent
d'établir des taxonomies plus ou moins élaborées n'épousent jamais parfaitement les
contours plus subtils des réalités qu'elles s'efforcent de décrire.\\

Le Bouddhisme propose une vision du monde, plus exactement un ensemble de visions
du monde dont les premiers linéaments remontent à la prédication du Bouddha historique, un contemporain de Confucius (fin VIe-début Ve siècles av. J.-C.) qui vécut et
enseigna du côté de ce qui est aujourd'hui la frontière népalaise et dans le Nord de
l'Inde actuelle.

Le choc provoqué par la découverte de ce qu'on s'était efforcé de dissimuler à sa jeunesse dorée – vieillesse, maladie, mort – décida celui qui se nommait alors Siddhârtha
Gautama à mener une vie d'ascète itinérant.
S'étant aperçu que les macérations et autres pénitences auxquelles se livraient les ermites, à force d'excès, finissaient par n'être
plus perçues comme des moyens, mais comme des fins en soi, il se concentra tout entier
sur son objectif, ce qui lui permit d'atteindre l'éveil (sk. bodhi, qui signifie exactement
``connaissance, révélation'' de celui qui s'est éveillé au savoir véritable, d'où le nom de
buddha qu'on lui donne – au passage : dans la suite, la mention ``sk.'' désigne un terme
sanskrit, la langue de l'Inde classique ; et ``p.'', un terme pâli, la langue sacrée du Theravâda, dont nous reparlerons ci-après).
L'enseignement qu'il dispensa alors est
concentré dans ce qu'on appelle le sermon de Bénarès (aujourd'hui VârâNasî, dans
l'Uttar Pradesh, un état du Nord de l'Inde), où il révéla ses quatre ``nobles vérités'' :
\begin{enumerate}
	\item Tout est souffrance (p. dukkha), c'est-à-dire que toute chose comporte nécessairement
		une part d'insatisfaction, engendre un sentiment de malaise, d'incomplétude, de trop ou
		de trop peu : on s'attache à des gens qui nous sont finalement arrachés, on vieillit et on
		meurt, on est contraint de se rapprocher de ce que l'on aimerait éviter, etc.
	\item Cette souffrance ne vient pas de nulle part, mais est la conséquence imputable à une
		cause, que le Bouddha identifie comme la ``soif'' (p. taNhâ), c'est-à-dire le désir
		d'existence, qui nous pousse à investir nos actes d'intentions qui en ``alourdissent'' la
		charge karmique.
		Le karma (sk.), qui signifie simplement ``acte'', désigne plus exac		tement, pourrait-on dire, le code génétique de l'acte, c'est-à-dire l'intention qui le
		constitue comme acte.
De fait, poser un acte, c'est y injecter un objectif qui le dote
		d'une sorte d'épaisseur et lui confère une certaine densité.
Cela a des conséquences, qui
		font que l'acte entraîne des effets, qui seront à leur tour suivis d'effets subséquents, et
		ainsi de suite, nous entraînant dans la roue du saMsâra (sk.), de la transmigration.
		Comprenons bien que cette transmigration qui nous fait renaître (plutôt que nous réin		carner) n'est pas une chance, mais le signe le plus tangible de cette catastrophe qui
		consiste à ne pouvoir échapper à l'enchaînement des vies successives.
	\item Pourtant, cette situation n'est pas une fatalité : de même qu'un feu cesse de brûler
		quand on ne l'alimente plus et qu'un effet se dissipe lorsqu'on supprime la cause qui le
		produit, il est possible de supprimer la souffrance en s'attaquant à la soif qui l'entraîne,
		de manière à atteindre le nirvâNa (sk.), c'est-à-dire l'extinction.
La nature exacte de
		celle-ci est difficile à cerner, s'agissant d'un état par définition étranger à tout ce qui est
		doté d'une charge karmique, un état d'où est absent toute souffrance, puisque tout désir
		en a été éliminé : le nirvâNa est analogue à ce qui se produit lorsqu'une flamme s'éteint.
	\item La quatrième et dernière ``noble vérité'' consiste alors à décrire les moyens à mettre
		en œuvre pour atteindre à cette délivrance ; c'est ce que l'on appelle l'``octuple sen		tier''.
Il s'agit de huit éléments de connaissance, de discipline morale et mentale qui
		doivent nous permettre d'ouvrir les yeux sur la réalité du monde, de manière à com		prendre que tout ce à quoi nous nous attachons sottement est illusoire (sk. mâyâ) et
		cause de souffrance, et, partant, de nous en détacher.
\end{enumerate}
La plus tenace des illusions dont nous devons nous défaire est celle du moi : le Bouddhisme originel insiste sur le fait que ce que j'appelle ``moi'' ne possède pas de réalité
substantielle, dans la mesure où il y a certes des états de conscience, des concepts, des
perceptions, des sensations et des formes physiques qui se succèdent (on les appelle les
``cinq agrégats''), mais rien de tout cela ne peut être ``relié'' à un socle stable, à un
support permanent qui subsisterait ``sous'' les déterminations fluentes.
On peut illustrer
cela en considérant un enfant qui grandit et devient un vieillard : nous avons
l'impression qu'il s'agit d'une même substance durable, affectée de diverses altérations
en un sens périphériques, qui n'entament sont intégrité substantielle ; nous pensons,
pour le dire autrement, que c'est la même et unique personne qui est d'abord un enfant,
ensuite un adulte, et enfin un vieillard.
Pourtant, à tout prendre, on sait aujourd'hui que
pas une seule des cellules qui formaient cet enfant n'est encore présente chez le vieillard : toutes sont mortes à un moment ou à un autre, ont été remplacées, et leurs
remplaçantes ont été remplacées à leur tour, si bien que ce qui nous apparaît comme une
entité stable n'a pas la consistance que nous lui prêtons : la vérité est qu'il n'y a qu'une
universelle fluence.
Il n'y donc a pas de ``je'' ; et ce ``moi'' dont je parle n'est que la
réunion supposée d'états de consciences, d'actes, de concepts, de formes corporelles qui
se succèdent de manière stroboscopique, si l'on peut dire.
En ce sens, le monde est impermanent, et possède un caractère illusoire, non parce qu'il n'existerait pas, mais parce
qu'il est vide, c'est-à-dire privé de toute stabilité.

On comprend par là toute la différence qu'il y a entre le Bouddhisme et son succédané
qui pare de ce nom une forme de psychologie positive où il n'est question que de ``réalisation de soi'' : dans le Bouddhisme, notre principal problème est précisément ce soi
dont il faut apprendre à se défaire ! À cette fin, il nous faut considérer le flux incessant
qui emporte tout et dont on doit se dégager, allégeant ainsi progressivement le fardeau
karmique qui nous est échu, en délaissant la soif d'existence et les illusions attenantes
pour atteindre à la complète délivrance.

C'est là une démarche extrêmement exigeante, et le plus sûr, aux yeux du Bouddha historique, est d'embrasser la voie monastique, de manière à s'efforcer d'éteindre
progressivement le désir à travers nos transmigrations successives.
Finalement, après
des éons d'efforts, on devient un arhat (sk.), définitivement engagé sur la voie qui
conduit à l'extinction : parvenu au terme de son existence, l'arhat ne renaît plus et est
délivré du cycle des transmigrations.

Le Bouddhisme, cependant, se diversifie comme s'est diversifié le Christianisme.
Deux
branches principales vont ``récupérer'' l'enseignement du Bouddha et essaimer en dehors de l'Inde.
Il est remarquable, en effet, de constater qu'après une période où il est en
faveur dans le sous-continent indien (en particulier sous l'empire Maurya, au IIIe siècle
av. J.-C.), le Bouddhisme reflue et finit par être progressivement évincé, par le Brahmanisme d'abord, et l'Hindouisme ensuite, si bien qu'il s'épanouit finalement loin des
terres qui l'on vu naître.
Il y a d'abord la ``Voie des anciens'' (p. theravâda), la plus
proche, du point de vue doctrinal, du Bouddhisme originel, qui fleurit dans l'île de
Ceylan (Sri Lanka) et se diffuse, à partir de là, dans toute l'Asie du Sud-Est ; c'est la
forme du Bouddhisme qui, associée à un héritage indien très manifeste, se retrouve donc
en Birmanie, en Thaïlande, au Laos, au Cambodge, dans le Sud du Vietnam, en Indonésie et en Malaisie.

La Bouddhisme Theravâda est contesté par ceux qui le dénigrent comme un ``Petit Véhicule'' (Hînayâna) en développant de leur côté un ``Grand Véhicule'' (Mahâyâna).
Celui-ci se déplace vers le Nord-Ouest, vers l'Asie centrale ; de là, il se transporte le
long de la route de la soie, jusqu'en Chine, où il s'introduit dès le Ier siècle ap. J.-C.
De
la Chine, où le Mahâyâna va prospérer, il se diffuse dans l'ensemble du monde sinisé,
c'est-à-dire dans le Nord de l'actuel Vietnam, en Corée et au Japon.
Un des traits caractéristiques qui distinguent le Bouddhisme du Grand Véhicule est qu'il accorde moins
d'importance au salut ``individuel'' (avec la figure de l'arhat), au profit d'un salut
conçu sur le mode de l'entraide grâce au secours apporté par des ``Héros pour l'éveil''
(sk. bodhisattva), qui retardent délibérément leur entrée dans le nirvâNa afin de porter
secours à ceux qui peinent sur la voie de l'éveil.
Un bodhisattva est ainsi une quasidivinité bienveillante et secourable qui nous facilite la tâche en contribuant, par sa vertu
propre, à alléger notre fardeau karmique, pour peu qu'on se place sous sa protection.
\begin{center}*\end{center}
Si le Bouddhisme a prospéré en Chine, c'est au départ et dans une certaine mesure sur
base d'un malentendu.
En effet, le Bouddhisme partait avec de sérieux handicaps dans
sa conquête spirituelle du monde chinois : en favorisant la vie monastique, et donc la
retraite hors du monde ainsi que le célibat, il s'opposait directement à des valeurs cardinales de la Chine, à savoir le service rendu au Fils du Ciel et le devoir de piété filiale
(xiao).
Pas de pire crime en effet, dans la société chinoise traditionnelle, que de refuser
de donner une descendance à ses parents, car cela a pour effet de rompre la continuité
des rites familiaux et des devoirs liés au culte des ancêtres.

La méprise consista en ceci : les textes bouddhiques, d'abord peu nombreux, furent traduits en chinois par des érudits qui choisirent de recourir à la terminologie daoïste,
donnant aux lettrés de l'Empire du milieu l'illusion que ces textes bouddhiques étaient
apparentés à ceux de Laozi, et ne relevaient donc pas fondamentalement d'une culture
étrangère.
Or cela cadrait fort bien avec la légende de Laozi quittant la Chine à la fin de
sa vie, allant son chemin vers l'Occident : les textes bouddhiques n'arrivaient-ils pas en
Chine par la route de la soie, c'est-à-dire en provenance de l'Occident ? Cette méprise
initiale favorisa l'implantation du Bouddhisme en Chine à une époque où les intellectuels appréciaient particulièrement l'arrivée sur le marché d'écrits à discuter dans le
cadre de leurs réunions lettrées, en particulier après la chute des Han au début du IIIe
siècle ap. J.-C.

Puis, le Bouddhisme, en se diffusant plus largement, apparut progressivement comme
un élément réellement exogène, nettement distinct du Daoïsme.
Mais cela n'advint que
tardivement (Ve-VIe siècles), lorsque les textes bouddhiques avaient déjà solidement
intégré le paysage intellectuel du monde chinois.
La diffusion de plus en plus large de
cette religion étrangère, le succès des penseurs (auprès des lettrés) et des mages bouddhistes (auprès du peuple et des cours semi-barbares qui s'étaient multipliées dans le
Nord de la Chine à cette époque), permirent au Bouddhisme de prospérer, si bien que
des écoles de pensée indiennes purent s'établir en Chine.

Ainsi, par exemple, l'école Mâdhyamika (sk.) de Nâgârjuna (IIe-IIIe siècles ap. J.-C. ?),
dont la doctrine hautement spéculative développait une ``Voie du milieu'' – ce que signifie le nom de Mâdhyamika – rejetant le mode de pensée binaire.
Au risque de la
caricature et par souci de brièveté, disons que, pour Nâgârjuna, caractériser les choses
comme simplement vraies ou fausses relève d'une vision simpliste du monde.
Pour faire droit à la doctrine bouddhiste du caractère universellement illusoire/vide des choses tout
en respectant la réalité de notre expérience commune, qui nous donne à voir un monde
non dénué de stabilité, nous devons distinguer deux plans de réalité, deux points de vue
sur le réel : il est vrai de dire que le monde possède une certaine densité, mais c'est une
vérité relative, car cela n'est vrai que du point de vue de l'observateur qui constate pour
lui-même cette consistance dont il dote le monde ; maintenant, à considérer les choses
du point de vue de la vérité absolue, il faut dire que le monde est impermanent et que,
sous l'illusion substantielle, il n'y a que du vide.
Ainsi donc, une même affirmation peut
être à la fois vraie et fausse : vraie du point de vue relatif, et fausse du point de vue absolu ; et inversement.
On comprend par là que le vide dont il est question, et qui
constitue, du point de vue absolu, la nature réelle de toute chose, n'est pas un pur néant :
le néant n'est rien, alors que le vide désigne d'abord la commune impermanence,
l'absence de densité, de stabilité substantielle d'un monde tout entier marqué par le flux
qui l'emporte, sans que rien n'y demeure constant.

Pour Nâgârjuna, donc, ce que nous percevons du monde depuis le point de vue relatif
qui est habituellement le nôtre est comme le voile qui couvre la réalité et en dessine,
pour ainsi dire, la périphérie.
Adopter le point de vue de l'absolu revient alors à ôter ce
voile, pour découvrir ce qui se cache derrière lui : le vide.
Au fond, nous sommes des
rêveurs qui doivent apprendre à s'éveiller du monde onirique où ils sont plongés ; le
rêve (l'univers du relatif) se dissipe alors aux yeux de qui accède au monde tel qu'il est
(l'absolu).
Ce rêve était bel et bien une illusion, non pas au sens où il n'existait pas – car
c'était bel et bien un rêve ! mais sa consistance se dissipe en regard de la réalité telle
qu'elle est en elle-même.

Pourtant, l'implantation d'un Bouddhisme affirmant son indianité à travers les puissantes métaphysiques des maîtres comme Nâgârjuna, Asanga ou Vasubandhu va se heurter
aux préférences intellectuelles des Chinois.
En forçant le trait, on peut dire en effet que
les Indiens, comme les Grecs, ont une sensibilité plus grande à la philosophie spéculative et à la métaphysique que les Chinois ; ces derniers, à l'instar des Romains, se
sentent plus concernés par l'efficacité ou la mise en œuvre concrète dans la pratique.
Partant, après son arrivée discrète sous le manteau du Daoïsme et une phase de développement subséquent ``à visage découvert'' et durant lequel s'affirme son indianité, le
Bouddhisme va connaître une nouvelle phase d'expansion dans le monde chinois en
s'acculturant, en s'acclimatant à la sensibilité proprement chinoise pour l'efficacité
concrète.
Ce qui va donc se développer en Chine, à partir des VIe et VIIe siècles de notre ère, c'est donc un Bouddhisme revisité ``à la chinoise'', où l'affirmation forte de la
non-substantialité du moi perd une part de sa radicalité, où l'éveil ne doit pas être une
affaire si difficile à atteindre qu'il faut des siècles pour y parvenir, etc.
Une fois encore,
on peut, en caricaturant, dire que le Chinois veut tout, et tout de suite.

Sans nous appesantir ici sur la diversité des écoles bouddhiques chinoises liées au
Grand Véhicule, retenons en particulier le rôle joué par le Bouddhisme Jingtu, c'est-àdire de la ``Terre pure'' : le bouddha Amitâbha (l'un des principaux bouddhas auxquels
se réfère le Mahâyâna, à côté de Siddhârtha Gautama) a préparé pour les fidèles un paradis dit ``de la Terre pure'', où transmigreront leurs âmes pour y œuvrer à leur salut
dans des conditions plus favorables et plus douces que celles qu'ils connaissent dans la
vie présente.
Pour bénéficier de cette transmigration, il suffit de répéter une formule de
confiance exprimant la foi que l'on place dans la compassion de ce Bouddha.
Mieux : le
fidèle bénéficie également du secours direct d'un bodhisattva en particulier, Avalokiteshvara, ``le Seigneur-qui-voit'', et que les Chinois appellent Guanyin.
Ce bienveillant
``Héros pour l'éveil'' joue le rôle d'une divinité secourable, protégeant ceux qui se
confient en lui pour les guider vers le paradis de la Terre pure.
Il est extrêmement intéressant de constater que, sous les Song (Xe-XIIIe siècles), ce bodhisattva, initialement
représenté sous des traits masculins, a fini par prendre une apparence féminine, devenant ainsi une sorte de mère céleste, dont le rôle n'est pas sans évoquer celui que joue la
très sainte Vierge Marie dans l'économie de la doctrine chrétienne.

L'aspect proprement religieux se manifeste ici plus nettement que l'aspect philosophique, qui était, en revanche, plus nettement appuyé dans le Bouddhisme spéculatif de
Nâgârjuna, par exemple.
Une autre forme de Bouddhisme très marquée par des composantes religieuses, et même plus ou moins superstitieuses, est constituée par un courant
partiellement indépendant du Grand Véhicule, et qu'on appelle le Véhicule de Diamant
(Vajrayâna) : il s'agit du Bouddhisme tibétain, qui s'est développé parallèlement au
Tantrisme indien, avec lequel il est apparenté, avant de se diffuser auprès des Mongols,
puis dans le monde chinois.
Ce Bouddhisme haut en couleurs (assez connu des Occidentaux grâce à la popularité dont bénéficie le Dalaï-lama, chef de la principale école de
cette forme de Bouddhisme), intègre bien des éléments chamaniques dont l'efficacité,
comme la récitation des fameux mantra (sk.), tient au fait même de leur énonciation.

Pour en revenir plus spécifiquement au Bouddhisme tel qu'il se développe dans
l'univers culturel de la Chine, une interprétation volontiers iconoclaste de la doctrine du
Grand Véhicule va connaître un succès prodigieux : le chan, ``méditation'', qui traduit
le sanskrit dhyâna et auquel nous nous référons habituellement sous son nom japonais
de zen (plusieurs termes liés au Bouddhisme nous sont ainsi familiers sous leur forme
japonaise ; cela tient au fait que ce sont des représentants japonais de ces doctrines qui
les ont fait connaître à un large public occidental, en particulier durant la première moitié du XXe siècle).
Une fois encore, sacrifions à la brièveté : le chan radicalise la
doctrine bouddhique relative à l'éveil.
Au fond, de quoi s'agit-il ? De changer son point
de vue sur la réalité, pour découvrir que nous n'avons pas à chercher la vérité ailleurs,
puisqu'elle se trouve toujours déjà là, sous nos yeux.
Le tout est d'apprendre à ouvrir les
yeux pour la voir telle qu'elle est, et non telle que nous sommes habitués à la voir, avec,
s'il est permis de s'exprimer ainsi, les ``lunettes'' de l'illusion.

Il ``suffit'' donc de changer de point de vue.
Les Bouddhistes vont s'interroger sur les
modalités de ce changement, en se demandant s'il advient progressivement (ainsi pensent ceux que l'on appelle ``grad(u)alistes'', pour qui l'éveil se conquiert par degrés, de
même qu'un miroir reflète de plus en plus nettement la réalité qui l'entoure au fur et à
mesure que l'on fait disparaître les taches à sa surface) ou soudainement.
Cette dernière
position est celle des ``subitistes'', qui assurent que l'on parvient à l'éveil d'un seul
coup : le ``miroir'' n'est pas sale, et les seules taches sont celles que l'on imagine.

C'est ce deuxième point de vue qui va s'imposer largement dans le Bouddhisme chan ;
en conséquence, les représentants de ce mouvements rejettent la nécessité absolue de
pratiques ascétiques bien spécifiques, puisque l'éveil peut advenir sans qu'il soit nécessaire de s'entourer d'expédients propres à atteindre graduellement à une prise de
conscience qui consiste, en vérité, à changer tout d'un coup son regard sur le monde.
Pour le dire autrement, la vie monastique n'est pas nécessairement un facteur propre à
nous conduire à l'éveil : celui-ci peut s'atteindre dans une vie extrêmement banale,
puisque la vérité est toujours déjà présente partout et en tout lieu, et qu'on l'atteint non
par une révélation transcendante, mais dès lors qu'on s'y est rendu disponible.

Mais comment se rend-on ainsi ``disponible'' à ce point de vue correct sur le réel ?
Toutes sortes de méthodes sont préconisées, de la plus rude à la plus douce.
Parmi les
plus frappantes (au sens propre...), on peut mentionner celles dont usait Linji, au IXe
siècle.
Ce qui nous empêche de voir les choses telles qu'elles sont, ce sont nos habitudes
mentales, les illusions que nous entretenons et dont nous ne parvenons pas à nous défaire, parce que nous les avons profondément intégrées et qu'elles régissent notre mode
de fonctionnement : le seul fait de chercher à s'en défaire relève d'une démarche intentionnelle fondée sur ces évidences acquises qui nous détournent d'un regard simple et
franc sur le monde.
La solution consiste donc à faire perdre pied, à provoquer une manière de commotion mentale pour que, durant un instant, on soit coupé du ``sol'' de ses
représentations habituelles.
Grâce à cet effet de surprise, on se retrouve momentanément
``déconnecté'' d'avec son regard ordinaire sur les choses ; et ce moment peut être
l'occasion d'un nouveau regard porté sur le monde.
Tout est bon, dans ces conditions,
pour provoquer un état de choc qui soit une passerelle vers l'éveil.
Le maître pose une
question...
et surprend le disciple qui réfléchit à une réponse judicieuse en le frappant
inopinément d'un coup de bâton ; le maître formule un discours aussi abominablement
choquant que possible pour déstabiliser son auditeur ; etc.

Tous, naturellement, ne sont pas également partisans d'une telle ``thérapie de choc'' ;
d'autres maîtres chan préconisent en particulier la ``méditation assise'' (ch. zuochan, j.
zazen) : en apprenant à adopter une posture physique et psychique appropriée, nous
pouvons faire le calme en nous pour n'être plus submergés par nos réflexes de fonctionnement habituels.
Pour ``alimenter'' cette méditation, il peut être bon de méditer sur
des réflexions ou questions absurdes (les ``cas juridiques'', ch. gong'an, j. kôan : entre
autres, songer au bruit que fait une main applaudissant seule) et qui, à force d'être tournées et retournées dans tous les sens, épuisent nos habitudes mentales et les conduisent
au point de rupture, qui est alors l'occasion de voir le monde différemment.
\begin{center}*\end{center}
La prospérité du Bouddhisme institutionnel va être durement mise à mal par une persécution de grande envergure déclenchée par le pouvoir central chinois vers le milieu du
IXe siècle.
Sous couvert de raisons idéologiques, l'entreprise vise en fait à briser la
puissance temporelle inquiétante des monastères et à accaparer les richesses considérables accumulées par les moines.
Bien que de courte durée, la persécution affaiblit
considérablement l'église bouddhique (le chan, en raison de son lien plus lâche avec
l'institution, est moins affecté) qui, tout en conservant une visibilité et une importance
indiscutables dans l'univers chinois pendant les siècles suivant, perd sa préséance intellectuelle au bénéfice d'un Confucianisme renaissant.

Pourtant, ce retour en force de la pensée de Confucius n'est pas une restitution à
l'identique de l'antique philosophie de l'époque des Royaumes combattants.
En réalité,
le néo-Confucianisme, qui prend son essor au XIe siècle, revendique l'héritage de
Confucius tout en intégrant (plus ou moins consciemment) des éléments bouddhistes et
daoïstes, si bien que Confucius eût été très étonné de voir que l'on professait sous son
nom une doctrine empruntant largement à la métaphysique du Bouddhisme et à la cosmologie spéculative des Daoïstes.
Des emprunts altérés, cela va sans dire, mais qui ne
sont pas concevables indépendamment de l'influence exercée par les doctrines qui les
ont suggérés.
Prenons un exemple : l'idée que toutes choses partagent une commune
nature dans un jeu de renvois aux dimensions cosmiques est clairement tributaire du
Bouddhisme ; la différence est cependant que cette commune nature-du-Bouddha
(``bouddhéité'', en chinois fo-xing) assimilée au vide devient, chez certains néo-Confucianistes, l'omniprésence du principe structurant li, expression d'une singularité
ontologique qui dit l'ordre (métaphysique et moral) des réalités constituant un monde
qui n'a rien d'une illusion impermanente.
De façon plus directement parlante, on peut
aussi évoquer les techniques de ``méditation assise'', largement reprises par les néo-Confucianistes.

Divers penseurs de premiers plan contribuent à ce renouveau confucéen, qui fait ensuite
l'objet d'une synthèse géniale dans l'œuvre de Zhu Xi (1130-1200).
Celui-ci fonde le
progrès moral sur une maîtrise érudite du vaste corpus littéraire dont les Entretiens de
Confucius et le livre de Mencius forment, avec deux traités tirés du Mémoire sur le rites,
le centre de gravité.
L'application dans l'étude doit ainsi soutenir la pénétration intellectuelle des principes moraux qui sont aussi la norme même de l'univers, et conduire le
lettré à la perfection de la sagesse au service de la communauté humaine.

Si Zhu Xi a fait l'objet de vives critiques de son vivant, sa pensée finit par s'imposer,
tant et si bien que sa manière d'envisager la formation de l'élite lettrée devient la norme,
presque sans interruption, depuis le début du XIVe siècle jusqu'à la suppression du système des examens, au début du XXe siècle.
L'exercice d'une fonction publique, dans le
monde chinois, passe en effet par un recrutement fondé sur des concours, qui, instaurés
dès les premiers temps de l'Empire, avaient été perfectionnés sous les Tang, au VIe siècle.
Sous les Yuan, au XIV siècle, l'interprétation des Classiques par Zhu Xi devient la
norme de l'orthodoxie confucéenne pour plus d'un demi millénaire.

Mais qui dit concours et examens dit aussi académisme et, à plus ou moins long terme,
sclérose d'une scolastique finissant par tourner à vide.
Pour beaucoup de lettrés, l'accès
à une fonction publique devient alors une fin en soi, le formalisme et le bachotage font
perdre de vue le but véritable de l'éducation lettrée : de fait, l'étude recommandée par
Confucius (et Zhu Xi, s'il est bien compris) n'est pas l'érudition, l'habileté dialectique
ou la virtuosité verbale, mais l'apprentissage de la vie morale, le progrès vers la sagesse.
Ce sera en particulier le rôle de Wang Yang-ming (1472-1529) de le rappeler en critiquant le dévoiement des intellectuels oublieux de l'aspect le plus fondamental de
l'enseignement confucianiste.
Pédagogue, poète, stratège, homme politique et philosophe de génie, Wang développe sa pensée suivant deux axes majeurs qui tiennent en
deux expressions chinoises dont nous allons dire un mot : liangzhi et zhixing heyi.

Liangzhi signifie littéralement ``connaissance du bien'' ; Wang Yang-ming désigne par
là la connaissance morale innée, la conscience morale que possède, selon lui, tout un
chacun, et ce, de façon parfaite et achevée dès l'origine.
Concrètement, tout homme que
je rencontre est un Confucius, un sage.
Naturellement, Wang n'est pas naïf : tout le
monde est très loin de se comporter comme un sage.
Ce qu'il veut dire, c'est que, bien
que nous soyons tous des sages au fond de notre cœur, nous ne le savons pas (une maison peut être alimentée en électricité, il y fait cependant sombre aussi longtemps qu'on
n'y allume pas la lumière) ; aussi le mal dont nous sommes responsables relève-t-il non
pas de notre nature profonde, mais de l'égoïsme qui vient s'interposer entre ce que nous
sommes en vérité et la manière dont nous agissons.
Pour prendre une métaphore parlante, on peut dire que, chaque jour, le soleil brille dans le ciel ; si nous ne le voyons pas,
c'est à cause des nuages qui nous empêchent de le voir.
De façon analogue, notre conscience morale, qui est aussi celle de tout homme, brille dans le ciel de notre xin (cœuresprit, siège de l'activité consciente et de la personnalité morale d'un individu), auquel
elle s'identifie ; si nous agissons mal, c'est que nous laissons ce ciel intérieur se charger
de nuages qui empêchent le bien foncier de s'exprimer.

On voit donc qu'il s'agit d'un pari extrêmement optimiste sur la nature humaine, qui
n'exclut pas, dans le même temps, un constat réaliste de notre misère morale ordinaire.
Cet optimisme foncier se situe explicitement dans la lignée de Mencius ; mais on comprend que la doctrine mencienne a été ``revisitée'' par le Bouddhisme, puisque Mencius
parlait seulement de jeunes pousses, de commencements et de promesses de vertu, et
non pas de vertus d'emblée parfaites, alors que Wang affirme que toutes les vertus se
trouvent toujours déjà à notre disposition, au fond de nous, pour peu que nous dissipions
(ce que ne va pas sans efforts !) les désirs égoïstes qui font obstacle à leur exercice sans
pour cela affecter aucunement leur nature intrinsèque.
Au modèle de développement des
vertus décrit par Mencius se substitue donc un modèle de recouvrement de la vertu sise
dans le cœur-esprit de chaque homme.

À cette conception du sens moral inné s'ajoute la doctrine exprimée par la formule
zhixing heyi, ``unité de la connaissance et de l'action''.
Très simplement, il s'agit de
dire que connaître et agir ne sont que les deux facettes d'une seule et même réalité :
conceptuellement, la connaissance se distingue de l'action (la première est le point de
départ de la seconde ; et la seconde, l'aboutissement de la première), mais elles n'en
sont pas moins liées comme l'avers et le revers d'une pièce de monnaie, l'ubac et l'adret
d'une montagne, deux éléments distincts et cependant indissociables.
Celui qui connaît
véritablement agit en conséquence ; car prétendre connaître sans que cette connaissance
s'accompagne pourtant de l'action appropriée est le signe d'une connaissance superficielle, d'une connaissance indigne de ce nom.
La preuve que vous connaissez la piété
filiale, ce n'est pas de pouvoir réciter le Classique de la piété filiale, mais de mettre
cette vertu en œuvre.
[Et pour les cinéphiles, dans Matrix, quand Néo dit à Morpheus
qu'il connaît le kung-fu, celui-ci lui demande de le prouver, non pas en récitant des règles, mais en se mesurant à lui.]

On retrouve là un élément dont il avait été question à propos de la philosophie antique
occidentale, et notamment chez les Stoïciens, dans la droite ligne de la doctrine socratique selon laquelle ``nul n'est méchant délibérément''.
Inopérante, la connaissance
révèle qu'elle est seulement superficielle.

Wang Yang-ming est ainsi convaincu que le sens véritable de l'étude consiste à recouvrer ce sens moral inné en dissipant les désirs égoïstes, qui obscurcissent notre
connaissance du bien et court-circuitent notre capacité à agir convenablement.
La sincérité (cheng) n'est pas autre chose, et l'attention respectueuse à l'égard de notre liangzhi
doit nous faire comprendre que toute activité de notre vie est un lieu propre au déploiement de ses ressources.
Un jour, comme un disciple du maître se plaignait de n'avoir
pas le temps de se retirer dans son cabinet d'étude pour se plonger dans les textes classiques et en tirer du profit spirituel, Wang lui expliqua que de tels progrès n'étaient pas
tributaires du temps libre que nous pouvions consacrer aux livres, puisque tout nous doit
nous être bon pour apprendre à en finir avec notre égoïsme : sommes-nous tenus de présider un procès ? Apprenons à ne pas nous irriter, ni juger de façon partisane ; et de
même dans toutes les circonstances de la vie.
Une métaphore illustre cela à merveille :
celui qui se préoccupe trop d'érudition, de composition élégante et de maîtrise théorique
ressemble à un homme qui, voulant se faire cuire du riz, se soucie à ce point de surveiller son feu qu'il en oublie de mettre de l'eau et du riz dans sa casserole.
Au bout d'un
moment, sa casserole vide se fend sous l'action de la chaleur, et devient inutilisable ;
l'infortuné a ainsi dissipé en vain ses efforts, et – c'est le cas de le dire –, il reste sur sa
faim.
Dans le même ordre d'idées, on pourrait dire que celui qui étudie un manuel de
cuisine n'est pas rassasié pour autant ; le manuel n'a de sens que si l'on se met à cuisiner pour de bon.
Et le meilleur cuisinier n'est pas nécessairement celui qui est le plus
expert quand on l'interroge sur les détails de l'ouvrage.
Ainsi en va-t-il de l'étude (xue),
qui est un apprentissage de la vie morale, et dont l'essentiel revient à mettre un terme
aux mouvements égoïstes, de façon à permettre à notre sens moral de répandre sa lumière dans notre vie et dans celle de tous les hommes.

\end{document}
