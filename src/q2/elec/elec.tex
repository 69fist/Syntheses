\documentclass[11pt,a4paper]{article}

\usepackage{booktabs}
\usepackage{tikz}
\usepackage{pgfplots}

\usetikzlibrary{arrows,positioning,fit,shapes,calc}

\usepackage[utf8]{inputenc}
\usepackage[frenchb]{babel}
\usepackage[T1]{fontenc}
\usepackage{lmodern}
\usepackage{amsfonts}
\usepackage{amsmath}
\usepackage[squaren, Gray]{SIunits}
\usepackage{numprint}
\usepackage{esint}
\usepackage{esvect}
\usepackage{circuitikz}
\usepackage[pdfauthor={Benoît Legat, Nicolas Cognaux et Lucas Nyssens},
pdftitle={Synthèse d'Électronique Q2 - LFSAB1202},
pdfsubject={Électronique}]{hyperref}
\usepackage{appendix}

\newtheorem{defin}{Definition}[section]
\newtheorem{nota}[defin]{Notation}
\newtheorem{prop}[defin]{Propriete}

\newcommand{\B}{\vec B}
\newcommand{\E}{\vec E}
%\newcommand{\EMF}{\varepsilon}
\newcommand{\EMF}{\mathcal{E}}

\DeclareMathOperator{\newdiv}{div}
\DeclareMathOperator{\diff}{d}

%%\newcommand{\dif}{d} % Uncomment this and comment the next line to get a classic differential
\newcommand{\dif}{\diff\!}

%%Tres Draft je sais... %%
\setlength{\parindent}{0em} % Vire les alinéas
\setlength{\parskip}{0.6em} % Agrandit l'espace entre les paragraphs

\title{Synthèse d'Électronique Q2 - LFSAB1202}
\author{Nicolas Cognaux \and Benoît Legat \and Lucas Nyssens}

\begin{document}
\maketitle

\part{Électromagnétisme}

\section{Champ magnétique}
%Un champ magnétique $\B$ est un champ créé par une déplacement de charge (selon la loi de Biot-Savart, voir Section~\ref{sec:bs})
qui provoque une force sur les charges en mouvement (c'est la force de Lorentz, voir Section~\ref{sec:lorentz}).

Dans un aimant, les lignes de champ vont du Nord au Sud.
{\bf Il n'y a jamais de monopôle magnétique !}

Le pôle Nord d'un aimant pointe vers le pôle Nord géographique qui est le pôle Sud magnétique.

$B$ est l'induction magnétique et $H$ le champ magnétique.
On a la relation
\[ B = \mu_0\mu_rH \]
où $\mu_0$ est la perméabilité du vide, elle vaut $\unit{4\pi \cdot 10^{-7}}{\tesla\meter\per\ampere}$.
À chaque fois qu'on utilisera $\mu_0$ sans $\mu_r$, on considérera qu'on est dans le vide.
Si la formule doit être utilisée dans un matériaux avec un certain $\mu_r$,
il suffit de remplacer $\mu_0$ par $\mu_0\mu_r$.
Attention néanmoins, si on est dans le cas d'un matériau ferromagnétique qui se magnétise (i.e. $H$ varie), $\mu_r$ n'est pas constant !
Plus d'info à la Section~\ref{sec:ferro}.

\section{Flux}
Le flux électrique d'un champ magnétique $E$ à travers une surface orientée $S$ vaut
\[ \Phi_E = \int \E \dif \vec S \]
Le flux magnétique d'un champ magnétique $B$ à travers une surface orientée $S$ vaut
\[ \Phi_B = \int \B \dif \vec S \]

\section{Force de Lorentz}
\label{sec:lorentz}
Un courant est un déplacement de charges.
Si ce courant rencontre un champ magnétique $\B$ et un champ magnétique $\E$, il y a création d'une force:
$$ \vec F = q \E + q \vec v \times \B $$
On l'appelle la force de Lorentz.

Dans le cas d'un courant parcourant un champ, la formule devient:
$$ \dif \vec F = I\dif \vec l \times \B $$

\paragraph{Remarques}
\begin{itemize}
	\item Par définition du produit vectoriel, $\vec v \perp \vec v \times \B$, donc $W = \vec v \cdot q\vec v \times \B = 0$.
		C'est à dire que le champ magnétique ne peut que changer la \textbf{direction} d'une particule chargée,
		\textbf{pas} la \textbf{norme} de sa \textbf{vitesse}.
\end{itemize}

\section{Force et couple sur une boucle}
Considérons une boucle de surface $S$ parcourue par un courant $I$ et un champ magnétique constant $\B$.
Soit $\vec S$, le vecteur surface ayant comme norme $S$, normal à la surface de la boucle et dont le sens est déterminé par le sens de $I$ et la règle de la main droite.
Soit
\footnote{$\vec \mu$ est aussi parfois noté $\vec m$.}
$\vec \mu = I \cdot \vec S$
, le moment dipolaire magnétique.
On peut calculer le torque exercé par le champ magnétique sur la boucle ainsi
\[ \vec \tau = \vec \mu \times \B \]
On a aussi une énergie potentielle valant
\[ U = -\vec \mu . \B = -\mu B \cos \phi \]
On voit ici que $\vec \mu$ et $\B$ tendent à s'alligner.
\begin{itemize}
	\item Si $\phi = 0$, $U = - \mu B$, c'est un équilibre stable, le niveau le plus bas d'énergie potentielle.
	\item Si $\phi = \pi$, $U = \mu B$, c'est un équilibre instable, le niveau le plus haut d'énergie potentielle.
		$\tau = 0$ mais il suffit d'un fifrelin pour que ça ne soit plus le cas.
\end{itemize}

La boucle garder sa forme pour annuler les forces latérales.
Si elle est déformable, elle tendra vers une forme circulaire.

\subsection{Application: moteur DC}
On peut donc utiliser la force de Lorentz pour créer un couple.
On utilise ça les moteur électrique.
Un cadre est connecté à une borne + et une borne - par des balais.
Il tourne et quand il arrive à $\phi = 0$, il a encore un peu de vitesse, donc il dépasse $\phi = 0$.
$\tau$ veut le ramener vers $\phi = 0$ car c'est le point de plus basse énergie potentielle.
Seulement, comme, il s'est retourné, la partie du cadre connectée à la borne + est maintenant connecté à la borne - et vice versa.
C'est à dire que le courant s'inverse, le point de plus basse énergie devient donc $\phi = \pi$.
$\tau$ veut donc continuer dans le même sens mais arrivé en $\phi = \pi$, il a encore un peu de vitesse, il dépasse donc $\phi = \pi$ et ainsi de suite.

\paragraph{Approximation utilisée}
Dans cette section, nous avons négligé le champ magnétique $\B$ généré par le courant $I$ passant dans la boucle.

\section{Loi de Gauss}
L'intégrale du champ sur une surface fermée entourant un volume
\[ \oiint \B \cdot \dif \vec S = 0 = \Phi \]
C'est dû au fait qu'un monopôle magnétique isolé n'existe pas.
Plus d'explication à l'Annexe~\ref{ann:gauss}.
\paragraph{Conséquences}
\begin{itemize}
	\item Les lignes de champs se bouclent sur elles-mêmes;
	\item Conservation du flux magnétique;
	\item Le flux magnétique revient du pôle Sud au pôle Nord en traversant l'aimant.
\end{itemize}

\section{Loi de Bio-Savart}
\label{sec:bs}
Un courant crée un champ magnétique.
Ce champ est la somme vectorielle de tous les champs générés par toutes les charges de ce courant.
Dans le \textbf{vide}, il vaut
\[ d\B = \frac{\mu_0}{4\pi}\frac{I\dif \vec l\times \hat r}{r^2} \]
Il est important de noter que $||\hat r|| = 1$.
Il existe une autre version de la formule où on remplace $\hat r$ par $\frac{\vec r}{r}$ et donc, au dénominateur, $r^2$ devient $r^3$, ce qui revient au même.

Le champs à une distance $x$ sur l'axe d'une spire de rayon $R$ traversée par un courant $I$ peut être calculé par Biot-Savart et vaut
\[ \B = \frac{\mu_0}{2}\cdot\frac{IR^2}{(R^2 + x^2)^{\frac{3}{2}}} \]
Pour déterminer le champ au centre d'une spire, on utilise Biot-Savart.
%% NC: Alors ici la démo j'ai pas réellement compris son truc donc je vais pas me risquer à expliquer. Il y a un angle phi qui aparait sans raison... Si quelqu'un pouvait clarifier. ;)
\paragraph{Conséquences}
\begin{itemize}
	\item Le champ magnétique le long d'une bobine longue (pas infinie) n'est pas constant.
	\item Les bobines de Helmholtz sont des bobines espacées d'une distance $d = R$ où $R$ est le rayon des deux bobines.
		Cette configuration permet de générer un champ magnétique presque constant sur toute la longueur séparant les bobines comme vous pouvez le constatez par la Figure~\ref{fig:deqr}.
\end{itemize}
\begin{figure}
	\begin{center}
		\begin{tikzpicture}[scale=0.7]
			\begin{axis}[xlabel=$x$, ylabel=$B$]
				\addplot {1/(1+x^2)+1/(1+(1-x)^2))};
				\addlegendentry{$d = R$}
				\addplot {1/(1+x^2)+1/(1+(2-x)^2))};
				\addlegendentry{$d = 2R$}
				\addlegendentry{une fonction}
			\end{axis}
		\end{tikzpicture}
	\end{center}
	\caption{$B$ dans l'axe de deux bobines espacée d'une distance $d$ où $R=1$}
	\label{fig:deqr}
\end{figure}

\section{Loi d'Ampère}
La loi d'Ampère dit que tout fil parcouru par un courant est toujours entouré d'un champ magnétique.
Dans tout contour fermé,
$$\oint \B \dif \vec l = \mu_0 I$$

Pour déterminer le sens dans lequel faire l'intégrale, il faut utiliser la règle de la main droite.
Où le pouce est le sens du courant et les doigts dans la direction de $\dif\vec l$.

On peut aussi utiliser
\[ \oint \vec H \cdot \dif \vec l = I \]
Où $\B = \mu_0 \vec H$ dans le vide.

\paragraph{Remarque}
Pour que la formule marche dans tous les cas, il faut introduire 3 types de courants:
$I_\mathrm{C}$, $I_\mathrm{D}$ et $I_\mathrm{M}$.
On définit
\[ I = I_\mathrm{C} + I_\mathrm{D} + I_\mathrm{M} \]
Le premier est le courant électrique que nous connaissons qui est dû à un déplacement de charge.
Le second vaut
\[ I_\mathrm{D} = \varepsilon_0\frac{\dif \Phi_E}{\dif t} \]

Seulement, dans la plupart des cas, considérer uniquement le premier suffit.
Mais par exemple, si on prend une boucle autour d'une capacité,
il faut que $\mu_0I$ soit le même quand on passe par le courant que quand on passe entre les deux plaques donc il faut un courant entre les plaques.
C'est $I_\mathrm{D}$ qui entre les deux plaques n'est pas nul qui établit l'égalité.

Par contre, il est très rare qu'on ne puisse pas négliger $I_\mathrm{M}$.
On a donc la loi d'Ampère généralisée par Maxwell
\[ \oint \B \dif \vec l = \mu_0 \left(I_\mathrm{C} + \varepsilon_0 \frac{\dif \Phi_E}{\dif t}\right) \]

\subsection{Applications de la loi d'Ampère}
\subsubsection{Champ dans un solénoïde long}
\label{sec:bbl}
\paragraph{Postulat}
\begin{itemize}
	\item $B_\mathrm{int}$ est constant et parallèle à l'axe de la bobine;
	\item $B_\mathrm{ext}$ est nul.
\end{itemize}
Par Ampère,
\begin{eqnarray*}
	B &=& \mu_0nI\\
	H &=& nI
\end{eqnarray*}
où $n$ est le nombre de spires par unité de longueur.
Pour la première équation, la bobine doit être dans le vide.
La deuxième équation gère les cas où la bobine n'est pas dans le vide.

\subsubsection{Champ dans un solénoïde toroïdal}
\label{sec:bst}
\paragraph{Postulat}
\begin{itemize}
	\item $B_\mathrm{int}$ est constant et tangeant au tore;
	\item $B_\mathrm{ext}$ est nul.
\end{itemize}
Par Ampère,
\begin{eqnarray*}
	B &=& \mu_0\frac{NI}{2\pi r}\\
	H &=& \frac{NI}{2\pi r}
\end{eqnarray*}
où $N$ est le nombre de spires et $r$ le rayon moyen du tore.
Pour la première équation, la bobine doit être dans le vide.
La deuxième équation gère les cas où la bobine n'est pas dans le vide.

\subsubsection{Champ généré par un long fil}
\paragraph{Postulat}
\begin{itemize}
	\item $B$ est constant pour une même distance du file et perpendiculaire au fil;
\end{itemize}
Par Ampère,
\[ \B = \frac{\mu_0I}{2\pi r} \]
où $r$ est la distance par rapport au fil.

Si on a deux fils parallèles à une distance $r$ l'un de l'autre, on a donc
\[ \dif F = \mu_0\frac{I_1I_2}{2\pi r}\dif l \]
où $F$ les attirent l'un à l'autre si le courant est dans le même sens et les repoussent s'il est dans le sens contraire.

\section{Inductance}
Dans l'inductance, chaque spire produit un champ magnétique et donc un flux.
Ce flux est intercepté par chaque spire et produit donc un courant lors de la variation de ce flux (selon la loi de Lenz-Faraday, voir Section \ref{sec:faraday}).
On définit l'inductance comme suit
\[ L = N\frac{\Phi_B(I)}{I} \]
où $N$ est le nombre de spires et $\Phi_B(I)$ le flux dans une spire.
$\Phi_B(I)$ est proportionnel à $I$ donc $L$ ne dépend pas de $I$.

\subsection{Applications}
\subsubsection{Inductance dans un solénoïde long}
Comme vu à la Section~\ref{sec:bbl},
\[ B = \mu_0nI \]
On a donc
\begin{align*}
	L &= N\frac{\Phi_B(I)}{I}\\
	&= \mu_0 nNS\\
	&= \mu_0 \frac{N^2}{l}S
\end{align*}
où $S$ est la surface d'une tranche du solénoïde et $l$ est sa longueur.

\subsubsection{Inductance dans un solénoïde toroïdal}
Comme vu à la Section~\ref{sec:bst},
\[ B = \mu_0\frac{NI}{2\pi r} \]
On a donc
\begin{align*}
	L &= N\frac{\Phi_B(I)}{I}\\
	&= \mu_0 \frac{N^2}{2\pi r}S\\
	&= \mu_0 \frac{N^2}{2r}a^2
\end{align*}
où $S$ est la surface d'une tranche du solénoïde, $r$ moyen le rayon du tore et $a$ tel que $S = \pi a^2$.

\section{Loi de Lenz-Faraday}
\label{sec:faraday}
En faisant varier le champ dans une boucle (et donc le flux), on crée un courant.
\[ \EMF = \oint \E \dif \vec l = -\frac{d\Phi_B}{dt} \]
\paragraph{Conséquences}
\begin{itemize}
	\item Le courant créé s'oppose à la règle de la main droite.
		C'est à dire que $\EMF$ s'oppose au changement de flux;
	\item Il ne faut pas de conducteur pour transmettre un courant.
		Un simple contour fermé interceptant le champ suffit
		(Important pour tout ce qui concerne la radio).
\end{itemize}

\subsection{Applications}
\subsubsection{Système moteur}
\begin{figure}[!ht]
	\begin{center}
		\begin{circuitikz}[american voltages]
			\fill (4.95,-0.1) rectangle (5.05,3.1);
			\fill [green!50!black, opacity=0.3] (0,0) rectangle (4.95,3);
			\draw (5,0) -- (8,0) to[open, o-o] (8,3) -| (0,2)
			to[battery, l=$V$] (0,1) -- (0,0);
			\draw (0,0) to [short, i=$\color{red}{I-I'}$] (5,0);
			\node[green!50!black] at (2.5,1.5) {$\Phi_B$};
			\draw[blue!80!black] (5.05,2.25) edge[style=-stealth] (6,2.25);
			\node[blue!80!black] at (5.5,2.55) {$\vec F_b$};
			\draw (5.05,1.5) edge[style=-stealth] (5.8,1.5);
			\node at (5.4,1.8) {$\vec v$};
			\draw (5.05,0.75) edge[style=-stealth] (6.2,0.75);
			\node at (5.6,1.05) {$\vec F$};
			\draw[<->] (7,0) -- (7,3);
			\node at (6.8,1.5) {$l$};
			\fill[green!50!black] (1,0.8) circle (0.05);
			\draw[green!50!black] (1,0.8) circle (0.2);
			\node[green!50!black] at (1.4,0.85) {$\B$};
			\fill[green!50!black] (1,2.2) circle (0.05);
			\draw[green!50!black] (1,2.2) circle (0.2);
			\node[green!50!black] at (1.4,2.25) {$\B$};
			\fill[green!50!black] (4,0.8) circle (0.05);
			\draw[green!50!black] (4,0.8) circle (0.2);
			\node[green!50!black] at (4.4,0.85) {$\B$};
			\fill[green!50!black] (4,2.2) circle (0.05);
			\draw[green!50!black] (4,2.2) circle (0.2);
			\node[green!50!black] at (4.4,2.25) {$\B$};
		\end{circuitikz}
	\end{center}
	\caption{Système moteur/générateur}
	\label{fig:sysmg}
\end{figure}
Le système moteur correspond à la Figure~\ref{fig:sysmg} avec $\vec F = 0$.
$V$ fournit un courant $I$ qui induit une force $F_b$.
$V$ fournit de l'énergie.
$F_b$ donne à la barre une vitesse $v$ telle que
\[ \frac{\dif v}{\dif t} = \frac{F_B}{m} = \frac{IlB}{m} \]
qui va induire une force ``contre-électromotrice'',
ici, un champs électrique $E_m$.
Par Lenz-Faraday, il va induire une tension, donc un courant $I'$ qui s'oppose au courant fourni par la batterie $I$.

\subsubsection{Système générateur}
Si on applique une force $F$ telle que $F > F_b$, alors $I' > I$ et la batterie va recevoir du courant,
donc se recharger (accumuler de l'énergie).

Le réversibilité totale dit que tout moteur peut devenir un générateur et vice-versa.
En effet, un courant peut créer un $\B$ et les $\B$ créent des courants.

\section{Les courants de Foucault}
Les courants de Foucault apparaissent lorsqu'on fait varier le flux magnétique sur une surface.
Il y a une infinité de lignes conductrices et donc une infinité de courants créés qui s'opposent au champ magnétique.
Il y a donc une attraction dûe à la force de Lorentz.
Ce phénomène est utilisé pour le freinage par exemple.
\paragraph{Conséquences}
\begin{itemize}
	\item Il y a une lévitation magnétique. Elle est utilisée dans les trains.
	\item Il n'y a pas de $\B$ dans les supraconducteurs.
\end{itemize}

\section{Tranformateur}
Dans un transformateur, une bobine crée un flux, ce flux se déplace dans le noyau de ferrite et est intercepté par une deuxième bobine.
Le transformateur ne fonctionne qu'en alternatif car le flux doit être variable.
On utilise des lamelles comme noyau de ferrite pour éviter les courants de foucault et donc les pertes d'énergie dans les transformateurs.

\section{Densité d'énergie des champs}
Une densité d'énergie est une énergie par volume.
Les unités sont donc $[\joule\per\meter\cubed]$.
Si on a un corps de volume $V$ et de densité d'énergie $E$, on a
\[ W = V \cdot E \]
\subsection{Énergie associée à un champ électrique}
Énergie dans une capacité:
\[ U_E = \frac{CV^2}{2} \]
Densité d'énergie électrique ($\varepsilon_r$ constant):
\[ E_E = \int \varepsilon_0 \varepsilon_r E \dif E = \varepsilon_0 \varepsilon_r \int E \dif E = \frac{\varepsilon E^2}{2} \]
\subsection{Énergie associée à un champ magnétique}
Énergie dans une inductance:
\[ U_M = \frac{LI^2}{2} \]
Densité d'énergie magnétique ($\mu_r$ constant):
\[ E_M = \int \mu_0 \mu_r H \dif H = \mu_0 \mu_r \int H \dif H = \frac{\mu H^2}{2} = \frac{B^2}{2\mu} \]
Si $\mu_r$ non constant
\[ E_M = \int_0^{B_\mathrm{max}} H \cdot \dif B \]

\subsection{Application}
Pour déterminer l'énergie nécessaire pour aimanter un aimant à son champ magnétique rémanant $B_r$,
on calcule l'aire entre l'axe $B$ et la courbe d'hystérésis pour la magnétisation de l'aimant de 0 au champ magnétique à saturation $B_s$ et on y soustrait l'aire entre l'axe $B$ et la courbe d'hystérésis pour la magnétisation de l'aimant de $B_s$ à $B_r$.
\[ \int_{B = 0}^{B = B_s} H \dif B + \int_{B = B_s}^{B = B_r} H \dif B \]
Remarquez la présence du plus $+$ parce que le deuxième terme est déjà négatif et qu'on considérait l'aire positive dans la phrase précédente.

\section{Force portante magnétique}
La force de portance d'un électroaimant est l'ensemble des densités énergétiques du système (conservation d'énergie):
$$W = \sum E_M \times V \Rightarrow W = \frac{B^2}{2\mu_\mathrm{fer}}V_\mathrm{fer} + \frac{B^2}{2\mu_\mathrm{air}}V_\mathrm{air}$$
Dans un électroaimant, l'entrefer diminue et donc son volume aussi.
La force d'un électroaimant avec deux entrefers est donc:
\[ F_x = \frac{B^2S}{\mu_0} \]

Dans un électroaimant avec entre-fer, Le flux reste constant y compris dans l'entre-fer.
Tout le champ se retrouve donc dans l'entre-fer.
Cela explique la démagnétisation des aimants permanents ouverts.

\part{Magnétisme dans la matière}
Soit $H$, le champ magnétique ou magnétisant.
C'est le champ extérieur, ce n'est pas encore le champ présent dans le matériau.
$B$ est le champ magnétique ou induction magnétique.
C'est le champ qui est ressenti dans le matériau.
Il est induit par un certain champ magnétique $H$.

Pour passer de $H$ à $B$, il faut savoir qu'un matériau, sous un champ magnétique $H$, se magnétise.
On note la magnétisation $M$.
On sait maintenant écrire
\[ B = \mu_0 (H + M) \]

On définit aussi $\mu_r = 1 + \chi$ tels que
\[ B = \mu_0\mu_r H = \mu_0 (H + \chi H) \]
On appelle $\chi$ la susceptibilité magnétique d'un matériau.

Il y a 3 types de magnétisme: diamagnétisme, paramagnétisme et ferromagnétisme.

$\mu_r$ et $\chi$ sont constants pour les matériaux diamagnétiques et paramagnétiques mais pour les ferro ou ferrimagnétiques, $B$ n'est pas proportionnel à $H$ donc $\mu_r$ et $\chi$ ne sont \textbf{pas constants} !

Dans le vide, $M = 0$ et $\chi = 0$ donc $\mu_r = 1$ et $B = \mu_0 H$.

On note parfois $B_0 = \mu_0 H$, l'induction magnétique qu'il y aurait dans le vide pour un certain $H$.
\section{Diamagnétisme}
Presque tous les matériaux sont diamagnétiques jusqu'à un certain point (i.e. un certain $H$), même les matériaux paramagnétiques et ferromagnétiques.
C'est parce qu'un moment magnétique peut être induit dans la plupart des atomes quand ils sont placés dans un champ magnétique.
Ce moment magnétique induit est antiparallèle au champ magnétique externe.
La somme des ces faibles moments magnétiques donne au matériau un très faible champ magnétique $M$ antiparallèle au champ magnétique externe.
Ce champ disparait quand le champ magnétique externe est disparait.

\section{Paramagnétisme}
Quand un matériau paramagnétique est placé dans un champ magnétique, le champ aide les moments magnétiques des atomes à s'alligner.
Ça produit un champ magnétique $M$ dans le matériau qui est parallèle au champ appliqué.
Ce champ disparait quand le champ magnétique externe est disparait.

\section{Ferromagnétisme}
\label{sec:ferro}
Dans un ferromagnétique, il y a formation de domaines magnétiques.
Dans ces domaines, les moments dipolaires sont tous alignés dans le même sens.
Lorsque $H$ augmente, les domaines tendent à se fusionner pour finalement n'en former plus qu'un qui est parallèle à $H$.
Dans ce cas, $M$ est à saturation, on le note $M_s$.

Pour un matériau ferromagnetique, on a $H \ll M$ donc on peut approximer
\[ B \approx \mu_0 M \]
C'est pourquoi, dans beaucoup de courbes d'hystérésis, on met $B$ en ordonnée alors que ça devrait être $M$.
C'est parce qu'on néglige le terme $\mu_0 H$ de $B = \mu_0 H + \mu_0 M$.


Les ferromagnétiques durs ont une courbe d'hystérésis plus large et nécessitent donc plus d'énergie pour etre magnétisés/démagnétisés que les ferromagnétiques doux.

Si on place une ferrite à l'intérieur d'une bobine, l'entièreté du $\B$ se retrouvera dans la ferrite, le $\B_{ext}$ est considéré comme nul.

\section{Superconductivité}
Certains matériaux sont des superconducteur.
Ils ont une température critique $T_c$ qui varie d'un matériau à l'autre.
Pour chaque température $T < T_c$, il existe un champ magnétique critique $H_c(T)$ tel que
pour tout $H < H_c(T)$, le matériau soumis à un champ magnétique externe $H$ à une température $T$ est en phase superconductrice
(voir Figure~\ref{fig:tbc}).

$H_c(T)$ est décroissant.
À une certaine température $T > T_c$, quelle que soit le champ magnétique $H$, le matériau sera en phase normale (voir Figure~\ref{fig:tbc}).

Si un matériau est en phase superconductrice, $B = 0$ à l'intérieur du matériau, c'est à dire que $M = -H$.
On appelle ça l'\emph{effet Meissner}.

Ça a aussi pour effet de rapprocher les lignes de champ magnétique extérieures plus proche les unes des autres lorsqu'elle sont proche du matériau.
C'est à dire que $B$ augmente.

En phase normale, $B \approx \mu_0 H$ à l'intérieur du matériau.

\begin{figure}
	\begin{center}
		\begin{tikzpicture}[scale=0.7]
			\begin{axis}[xmin=0,xmax=5,ymin=0,ymax=50, xlabel=$T(\kelvin)$, ylabel=$B_c(\milli\tesla)$]
				%\xlabel{$T$}
				\addplot[smooth, color=blue, domain=0:4.1] {41.2*sqrt(1 - (x/4.1)^2)};
				\addlegendentry{$B_c(T)$}
			\end{axis}
			\node at (2.5, 2) {\begin{minipage}{2cm}Phase superconductrice\end{minipage}};
			\node at (5.5, 4) {\begin{minipage}{1.5cm}Phase normale\end{minipage}};
			\draw[style=-stealth] (6, 0.5) to (5.64, 0);
			\node at (6.2, 0.7) {$T_s$};
		\end{tikzpicture}
	\end{center}
	\caption{Relation entre $T$ et $B_c$ pour le mercure pur}
	\label{fig:tbc}
\end{figure}

\part{Résumé}
Lois de Gauss
\begin{align*}
	\oiint \E \cdot \dif \vec S &= \frac{1}{\varepsilon_0\varepsilon_r}\iiint \rho \dif V = \frac{Q_\mathrm{encl}}{\varepsilon_0\varepsilon_r}\\
	\oiint \B \cdot \dif \vec S &= 0
\end{align*}
Loi de Lenz-Faraday
\[ \oint \E \cdot \dif \vec l = \EMF = - \frac{\dif \int \B \dif \vec S}{\dif t} = - \frac{\dif \Phi_B}{\dif t} \]
Loi d'Ampère
\[ \oint \B \cdot \dif \vec l = \mu_0\mu_r \int \vec J \cdot \dif \vec S = \mu_0\mu_r \left(I_\mathrm{C} + \varepsilon_0 \frac{\dif \Phi_E}{\dif t}\right) \]

\appendix
\part{Annexes}
L'annexe est consituée de raisonnements intéressants faisant des liens avec différentes matières.
Leur contenu ne fait pas partie de la matière.
\section{Explication de la loi de Gauss}
\label{ann:gauss}
Par le théorème de la divergence, on a
\[ \Phi = \oiint \B \cdot \dif \vec S = \iiint \newdiv \B \cdot \dif V \]
Seulement, on a $\newdiv \B = 0$ qui est dû au fait qu'il n'existe pas de monopôle magnétique.
D'où $\Phi = 0$.

Pour $\E$, on a $\newdiv \E = \frac{\rho}{\varepsilon_0}$.
Où $\rho$ est la densité de charge par unité de volume.
D'où $\Phi = \frac{Q_\mathrm{int}}{\varepsilon_0}$.

\end{document}
