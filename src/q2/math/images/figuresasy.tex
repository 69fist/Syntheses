\documentclass{article}
\usepackage[pdftex]{graphicx}
\usepackage{asymptote}
\usepackage[french]{babel}
\usepackage[latin1]{inputenc}
\begin{document}


%% 1) A
\begin{asy}
// Variables
import patterns;
unitsize(1inch);
pair A=(-1,-0.5),B=(1,-0.5),C=(1,0.5),D=(-1,0.5),E=(1.3,-0.25),F=(1.3,0.25);
path rect=A--B--C--D--cycle;
// Le rectangle
add("hachure",hatch(1.2mm,NE,.3mm+black));
fill(scale(0.95)*rect,pattern("hachure"));
// La bordure
draw(A+(-0.01,-0.01)--B+(0.01,-0.01));
draw(D+(-0.01,+0.01)--C+(0.01,0.01));
draw(A+(-0.01,-0.01)--D+(-0.01,+0.01),dashed);
draw(B+(0.01,-0.01)--C+(0.01,0.01),dashed);
// Points isol�s
dot(E);
dot(F);
\end{asy}

%% 2) int(A)
\begin{asy}
// Variables
import patterns;
unitsize(1inch);
pair A=(-1,-0.5),B=(1,-0.5),C=(1,0.5),D=(-1,0.5),E=(1.3,-0.25),F=(1.3,0.25);
path rect=A--B--C--D--cycle;
// Le rectangle
add("hachure",hatch(1.2mm,NE,.3mm+black));
fill(scale(0.95)*rect,pattern("hachure"));
// La bordure
draw(A+(-0.01,-0.01)--B+(0.01,-0.01),dashed);
draw(D+(-0.01,+0.01)--C+(0.01,0.01),dashed);
draw(A+(-0.01,-0.01)--D+(-0.01,+0.01),dashed);
draw(B+(0.01,-0.01)--C+(0.01,0.01),dashed);
\end{asy}


%% 3) fermeture(A)
\begin{asy}
// Variables
import patterns;
unitsize(1inch);
pair A=(-1,-0.5),B=(1,-0.5),C=(1,0.5),D=(-1,0.5),E=(1.3,-0.25),F=(1.3,0.25);
path rect=A--B--C--D--cycle;
// Le rectangle
add("hachure",hatch(1.2mm,NE,.3mm+black));
fill(scale(0.95)*rect,pattern("hachure"));
// La bordure
draw(A+(-0.01,-0.01)--B+(0.01,-0.01));
draw(D+(-0.01,+0.01)--C+(0.01,0.01));
draw(A+(-0.01,-0.01)--D+(-0.01,+0.01));
draw(B+(0.01,-0.01)--C+(0.01,0.01));
// Points isol�s
dot(E);
dot(F);
\end{asy}


%% 4) fronti�re(A)
\begin{asy}
// Variables
import patterns;
unitsize(1inch);
pair A=(-1,-0.5),B=(1,-0.5),C=(1,0.5),D=(-1,0.5),E=(1.3,-0.25),F=(1.3,0.25);
path rect=A--B--C--D--cycle;
// Le rectangle
// vide
// La bordure
draw(A+(-0.01,-0.01)--B+(0.01,-0.01));
draw(D+(-0.01,+0.01)--C+(0.01,0.01));
draw(A+(-0.01,-0.01)--D+(-0.01,+0.01));
draw(B+(0.01,-0.01)--C+(0.01,0.01));
// Points isol�s
dot(E);
dot(F);
\end{asy}


%% 5) limite(A)
\begin{asy}
// Variables
import patterns;
unitsize(1inch);
pair A=(-1,-0.5),B=(1,-0.5),C=(1,0.5),D=(-1,0.5),E=(1.3,-0.25),F=(1.3,0.25);
path rect=A--B--C--D--cycle;
// Le rectangle
add("hachure",hatch(1.2mm,NE,.3mm+black));
fill(scale(0.95)*rect,pattern("hachure"));
// La bordure
draw(A+(-0.01,-0.01)--B+(0.01,-0.01));
draw(D+(-0.01,+0.01)--C+(0.01,0.01));
draw(A+(-0.01,-0.01)--D+(-0.01,+0.01));
draw(B+(0.01,-0.01)--C+(0.01,0.01));
\end{asy}


%% 6) point isol�(A)
\begin{asy}
// Variables
import patterns;
unitsize(1inch);
pair A=(-1,-0.5),B=(1,-0.5),C=(1,0.5),D=(-1,0.5),E=(1.3,-0.25),F=(1.3,0.25);
// Points isol�s
dot(E);
dot(F);
draw((0.7,0)--(1.9,0),invisible);
draw((1.3,-0.5)--(1.3,0.5),invisible);
\end{asy}

%% 7) Coo sph�riques

\end{document}