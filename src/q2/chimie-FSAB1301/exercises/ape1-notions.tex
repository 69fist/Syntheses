
\section{Notions de base}

\subsection{}
La distance parcourue lors d'un marathon est de \SI{42,195}{\kilo\meter}. Convertissez cette distance en miles et yards, sachant que $\SI{1}{mil} = \SI{1.609}{\kilo\meter} = \SI{1760}{yd}$.

\begin{solution}
    $ \SI{42.195}{\kilo\meter} = \frac{\SI{42.195}{\kilo\meter}}{\SI{1.609}{\kilo\meter\per mil}} = \SI{26.22}{mil} = \SI{26.22}{mil} \times \SI{1760}{yd \per mil} = \SI{46155}{yd} $
\end{solution}

\subsection{}
Le poids lancé en athlétisme par les hommes était à l'origine un boulet de marine anglaise de 16 livres. Déterminez sa masse réglementaire (minimale) avec une précision de \SI{10}{\gram}, sachant que $\SI{1}{lb} = \SI{453.6}{\gram}$.

\begin{solution}
    $ \SI{16}{lb} = \SI{16}{lb} \times \SI{453.6}{\gram\per lb} = \SI{7.26}{\kilo\gram} $
\end{solution}

\subsection{}
Quelle vitesse exprimée en km/h a dû atteindre en fin d'élan Sergeï Bubka (\SI{80}{\kilo\gram}) lorsqu'il a établi le record du monde de saut à la perche de \SI{6.14}{\meter} ? Considérez que sa technique de saut est parfaite et que son centre de gravité corporel se trouve à \SI{1}{\meter} du sol.

\begin{solution}
    On suppose que toute son énergie cinétique est convertie en énergie potentielle de gravité :
    \begin{align*}
        \frac{mv^2}{2} &= mgh \\
        v &= \sqrt{2gh} = \sqrt{2 \times 9.81 \times (6.14 - 1)} = \SI{10.04}{\meter\per\second} = \SI{36.2}{\kilo\meter\per\hour}
    \end{align*}
\end{solution}

\subsection{}
UneSInstallation photovoltaïque a produit durant une journée d'été (\SI{16}{\hour} \SI{38}{\minute} de clarté) \SI{29.5}{\kilo\watt\hour}. Quelle puissance électrique moyenne a-t-elle développée ? Sur base de cette puissance moyenne, quelle quantité d'énergie exprimée en kcal a-t-elle pu délivrer en \SI{1}{\hour}, sachant que $ \SI{1}{kcal} = \SI{4184}{\joule} $ ?

\begin{solution}
    $ P_\mathrm{av} = \frac{\SI{29.5}{\kilo\watt\hour}}{\SI{16}{\hour} \SI{38}{\minute}} = \frac{\SI{106.2e6}{\joule}}{\SI{59880}{\second}} = \SI{1774}{\watt} $ \\
    En \SI{1}{\hour}, l'installation délivre $ \SI{1774}{\joule} = \SI{.424}{kcal} $.
\end{solution}

\subsection{}
Déterminez le nombre de moles présentes dans \SI{75}{\gram} de sulfate d'aluminium \ce{Al2(SO4)3}.

\begin{solution}
    $ n = \frac{m}{M} = \frac{75}{2 \times 27 + 3 \times (32 + 4 \times 16)} = \SI{.219}{\mole} $
\end{solution}

\subsection{}
Calculez en grammes la masse d'une molécule de glucose \ce{C6H12O6}.

\begin{solution}
    La masse molaire du glucose est de \SI{180}{\gram\per\mole} ; une mole pèse donc \SI{180}{\gram}. On sait qu'une mole contient \num{6.022e23} molécules ; donc $ m = \frac{\num{180}}{\num{6.022e23}} = \SI{299}{\yocto\gram} $.
\end{solution}

\subsection{}
L'analyse d'un échantillon de carbure de calcium pur donne le résultat suivant : \SI{62.5}{\%} en masse de calcium et \SI{37.5}{\%} en masse de carbone. Déterminez la formule chimique du carbure de calcium.

\begin{solution}
    Dans \SI{100}{\gram} de carbure de calcium il y a \SI{62.5}{\gram} de calcium et \SI{37.5}{\gram} de carbone, autrement dit resp. \SI{1.56}{\mole} et \SI{3.13}{\mole}. On remarque qu'il y a deux fois plus de carbone que de calcium ; la formule est donc \ce{CaC2}.
\end{solution}

\subsection{}
Un échantillon de \SI{28.1}{\gram} de \ce{Co} réagit avec du chlore en excès. On obtient \SI{61.9}{\gram} de produit. Déterminez la formule chimique du produit obtenu.

\begin{solution}
    \begin{tabular}{l|lllll}
        & \ce{2 $x$ Co} & \ce{+} & \ce{$y$ Cl2} & \ce{->} & \ce{2Co_$x$ Cl_$2y$} \\
        \hline\hline
        $m_i$ & \SI{28.1}{\gram} & & & & \SI{0}{\gram} \\
        \hline
        $n_i$ & \SI{.476}{\mole} & & & & \SI{0}{\mole} \\
        $ \Delta n $ & \SI{-.476}{\mole} & & & & \SI{+.476}{\mole} \\
        $ n_f $ & \SI{0}{\mole} & & & & $\frac{61.9}{59x + 71.0y}$ \\
        \hline
        $ m_f $ & \SI{0}{\gram} & & & & $\SI{61.9}{\gram}$
    \end{tabular} \\
    On a l'équation
    \begin{align*}
        \num{0.476} &= \frac{\num{61.9}}{59x + 71.0y} \\
        59x + 71.2y &= 130
    \end{align*}
    La solution la plus simple avec $x$ et $y$ entiers est $x = 1$ et $y = 1$. La formule du produit est donc \ce{CoCl2}.
\end{solution}

\subsection{}
Un échantillon de \SI{1.43}{\gram} d'un métal M réagit avec un excès d'oxygène. On obtient \SI{2.04}{\gram} d'un oxyde \ce{M2O3}. Calculez la masse atomique du métal et identifiez-le.

\begin{solution}
    On pose $x$ comme étant la masse atomique de M. \\
    \begin{tabular}[]{l|lllll}
        & \ce{4M} & \ce{+} & \ce{3O2} & \ce{->} & \ce{2M2O3} \\
        \hline\hline
        $m_i$ (\si{\gram}) & 1.43 & & & & 0 \\
        \hline
        $n_i$ (\si{\mole}) & $\frac{1.43}{x}$ & & & & 0 \\
        $\Delta n$ (\si{\mole}) & $-\frac{1.43}{x}$ & & & & $+\frac{1.43}{2x}$ \\
        $n_f$ (\si{\mole}) & 0 & & & & $\frac{1.43}{2x}$ \\
        \hline
        $m_f$ (\si{\gram}) & 0 & & & & $\frac{1.43}{2x} (2x + 48)$
    \end{tabular} \\
    On a donc l'équation 
    \begin{align*}
        \frac{1.43}{x} (x+24) &= 2.04 \\
        x &= \SI{56.26}{\gram\per\mole} 
    \end{align*}
    Cette masse atomique correspond à celle du fer \ce{Fe}.
\end{solution}

\subsection{}
Déterminez la masse de nitrate d'aluminium qu'il faut utiliser pour préparer \SI{250}{\milli\liter} de solution \SI{.1}{M}.

\begin{solution}
    La formule de nitrate d'aluminium est \ce{Al(NO3)3}.
    \[ m = nM = cVM = 0.1 \times 0.250 \times 213 = \gram{5.33} \]
\end{solution}

\subsection{}
Le chlore peut être obtenu au laboratoire en faisant réagir le dioxyde de manganèse \ce{MnO2} avec de l'acide chlorhydrique, \ce{HCl}. Outre le chlore, \ce{Cl2}, on obtient du chlorure de manganèse, \ce{MnCl2}, et de l'eau, \ce{H2O}. Écrivez et pondérez la réaction et déterminez la quantité de chlore obtenue à partir de \gram{100} de dioxyde de manganèse.

\begin{solution}
    \begin{tabular}[]{cccc}
        \ce{MnO2} & \ce{+ 4HCl ->} & \ce{Cl2} & \ce{ + MnCl2 + 2H2O} \\
        \hline
        \gram{100} & & & \\
        \mole{-1.15} & & \mole{1.15} & \\
        & & \gram{81.6} & 
    \end{tabular}\\
    On obtient donc \gram{81.6} de chlore.
\end{solution}

\subsection{}
Le coke est une forme impure de carbone utilisé pour la production industrielle des métaux à partir de leurs oxydes. Si le coke contient \SI{95}{\percent} de carbone en masse, quelle masse de coke faudra-t-il pour réagir complètement avec \SI{1}{\tonne} d'oxyde de fer (III) ? Le carbone se transforme en \ce{CO} et \ce{CO2} dans le rapport molaire 2/3 - 1/3.

\begin{solution}
    En connaissant le rapport molaire, on sait qu'il y aura deux fois plus de \ce{CO} que de \ce{CO2}. On note $x$ la masse de coke qui réagit. \\
    \begin{tabular}[]{cccc}
        \ce{9C} & \ce{+} & \ce{4Fe2O3} & \ce{-> 6CO + 3CO2 + \dots} \\
        \hline
        $0.95x$ & & \gram{e6} & \\
        $0.079x$ & & \mole{6250} & \\
    \end{tabular} \\
    En pondérant on a $4 \times 0.079 x = 9 \times 6250$, ce qui nous donne $ x = \SI{177.6}{\kilo\gram}$.
\end{solution}

\subsection{}
L'hydroxyde de sodium, \ce{NaOH}, réagit avec l'acide sulfurique, \ce{H2SO4}, pour donner du sulfate de sodium, \ce{Na2SO4}. Si on ajoute \gram{60} de \ce{NaOH} à \gram{20} de \ce{H2SO4}, combien de grammes de \ce{Na2SO4} seront produits ?

\begin{solution}
    L'acide réagit en excès avec la soude ; on aura donc un reste de \ce{NaOH}. \\
    \begin{tabular}[]{cccccc}
        \ce{2NaOH} & \ce{+} & \ce{H2SO4} & \ce{->} & \ce{Na2SO4} & \ce{+ 2H2O} \\
        \hline
        \gram{60} & & \gram{20} & & & \\
        \mole{1.5} & & \mole{0.2} & & & \\
        \mole{-0.4} & & \mole{-0.2} & & \mole{0.2} & \\
        & & & & \gram{28.4} & \\
    \end{tabular}\\
    On obtient donc \gram{28.4} de sulfate de sodium.
\end{solution}

\subsection{}
Combien de \ce{Co} métallique peut être produit par réaction à chaud de \gram{500} d'oxyde de \ce{Co(II)} avec \gram{10} d'hydrogène ?

\begin{solution}
    L'oxyde de cobalt réagit en excès avec l'hydrogène. \\
    \begin{tabular}[]{cccccc}
        \ce{CoO} & \ce{+} & \ce{H2} & \ce{->} & \ce{Co} & \ce{+ H2O} \\
        \hline
        \gram{500} & & \gram{10} & & & \\
        \mole{6.67} & & \mole{5} & & & \\
        \mole{-5} & & \mole{-5} & & \mole{5} & \\
        & & & & \gram{295} & \\
    \end{tabular} \\
    On obtient donc \gram{295} de cobalt.
\end{solution}

\subsection{}
En rangeant le stock chimique du laboratoire, un chercheur trouve une bouteille avec la mention incomplète ``soude diluée à \ldots''. Afin de déterminer la concentration en hydroxyde de sodium, \ce{NaOH}, il prépare une solution d'acide chlorhydrique, \ce{HCl} de concentration \SI{0.01}{\mole\per\liter} et constate qu'il lui faut \SI{25}{\milli\liter} de cette solution pour réagir complètement (neutraliser) la soude contenue dans \SI{100}{\milli\liter} prélevés dans la bouteille. Écrivez l'équation de la réaction de l'acide chlorhydrique avec l'hydroxyde de sodium. Quelle est la concentration en \ce{NaOH} de la solution dans la bouteille ?

\begin{solution}
    On note $x$ la concentration en soude. \\
    \begin{tabular}[]{cccc}
        \ce{NaOH} & \ce{+} & \ce{HCl} & \ce{-> NaCl + H2O} \\
        \hline
        \liter{.1} & & \liter{.025} & \\
        $0.1x$ & & \mole{.00025} & \\
    \end{tabular} \\
    On a l'équation $0.1x = 0.00025$, donc $x = \mpl{.0025}$.
\end{solution}

\subsection{}
Le vinaigre peut être considéré comme une solution aqueuse d'acide acétique, un acide organique de formule \ce{CH3COOH}. \SI{29}{\milli\liter} d'une solution \mpl{.25} de soude, \ce{NaOH}, sont nécessaires pour réagir complètement avec \SI{100}{\milli\liter} d'une solution diluée 10 fois de vinaigre pur. Quel est le degré du vinaigre, c'est-à-dire la masse d'acide acétique contenue dans \gram{100} de vinaigre ? Quelle quantité (masse) de \ce{NaOH} le chercheur a-t-il utilisé pour préparer \SI{100}{\milli\liter} de solution de soude ?

\begin{solution}
    On note $x$ la concentration en acide acétique du vinaigre. Après dilution, la concentration ne sera plus que de $0.1x$. \\
    \begin{tabular}[]{cccc}
        \ce{CH3COOH} & \ce{+} & \ce{NaOH} & \ce{-> CH2COONa + H2O} \\
        \hline
        \liter{.1} & & \liter{.029} & \\
        $0.01x$ & & \mole{0.00725} & \\
    \end{tabular} \\
    On a $ 0.01x = 0.00725 \Rightarrow x = \mpl{.725} $. Dans \gram{100} de vinaigre, on a donc \mole{.0725} d'acide acétique, ce qui nous donne une masse de \gram{4.35}. Le vinaigre est de degré \SI{4.35}{\degree}.

\end{solution}

\subsection{}
On dispose d'une solution concentrée d'acide sulfurique, \ce{H2SO4}. Afin d'en déterminer la concentration, on en prélève \milliliter{10} que l'on dilue 100 fois avec de l'eau distillée. On prépare une solution de soude avec \gram{5} de \ce{NaOH} et \milliliter{250} d'eau distillée. \milliliter{72} de cette solution de soude sont nécessaires pour neutraliser \milliliter{100} de la solution diluée d'acide. Écrivez l'équation de la réaction d'acide sulfurique avec l'hydroxyde de sodium. Quelle est la concentration molaire de la solution concentrée d'acide sulfurique ?

\begin{solution}
    On note $x$ la concentration de l'acide sulfurique. Après dilution, sa concentration ne sera plus que de $0.01x$. \\
    Pour la soude, sachant qu'on en a \gram{5} pour \milliliter{250}, sa concentration est de \mpl{.5}. \\
    \begin{tabular}[]{cccc}
        \ce{H2SO4} & \ce{+} & \ce{2NaOH} & \ce{-> Na2SO4 + 2H2O} \\
        \hline
        \liter{.1} & & \liter{.072} & \\
        $0.001x$ & & \mole{.036} &
    \end{tabular} \\
    En pondérant correctement, on a l'équation $0.036 = 2\times 0.001x \Rightarrow x = \mpl{18} $.
\end{solution}

\subsection{}
Le manganèse peut être obtenu en deux étapes à partir de la rhodochrosite, un minéral rose-rouge composé de carbonate de manganèse (\ce{MnCO3}). Le carbonate de manganèse est d'abord chauffé, à l'abri de l'air, et se décompose en monoxyde de manganèse(II) et en dioxyde de carbone. Le manganèse pur est ensuite obtenu à partir du monoxyde de manganèse par aluminothermie : il est porté à haute tempréature en présence d'aluminium (réducteur) pour former du \ce{Mn} et de l'alumine (\ce{Al2O3}). Le rendement de l'étape de décomposition thermique est de 90\% et celui de l'étape d'aluminothermie de 80\%.
\begin{enumerate}[label=\alph*]
    \item Écrivez l'équation chimique de la décomposition thermique du carbonate de manganèse.
    \item Écrivez l'équation chimique de l'aluminothermie du monoxyde de manganèse(II).
    \item Quelle quantité de \ce{MnCO3} est nécessaire pour produire \gram{100} de \ce{Mn} ?
    \item Quelle quantité de dioxyde de carbone est alors dégagée par le procédé ?
\end{enumerate}

\begin{solution}
    On note $x$ la quantité en \si{\mole} de \ce{MnCO3}. \\
    \begin{tabular}[]{ccccc}
        \ce{MnCO3} & \ce{->} & \ce{MnO} & \ce{+} & \ce{CO2} \\
        \hline
        $x$ & & $0.9x$ & & $0.9x$ 
    \end{tabular} \\
    \begin{tabular}[]{cccc}
        \ce{3MnO} & \ce{+ 2Al^3+ ->} & \ce{3Mn^2+} & \ce{+ Al2O3} \\
        \hline
        $0.9x$ & & $0.8\times 0.9x$ & 
    \end{tabular} \\
    On souhaite produire \gram{100} de \ce{Mn}, c'est-à-dire \mole{1.82}. On a alors l'équation
    \begin{align*}
        0.8 \times 0.9 x &= 1.82 \\
        x &= \mole{2.53} = \gram{290}
    \end{align*}
    On obtient donc en \ce{CO2} $ 0.9x = \mole{2.28} = \gram{100} $.
\end{solution}

