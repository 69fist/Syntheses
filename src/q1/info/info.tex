\documentclass[11pt,a4paper]{article}

% French
\usepackage[utf8x]{inputenc}
\usepackage[frenchb]{babel}
\usepackage[T1]{fontenc}
\usepackage{lmodern}
\usepackage{ifthen}

% Color
% cfr http://en.wikibooks.org/wiki/LaTeX/Colors
\usepackage{color}
\usepackage[usenames,dvipsnames,svgnames,table]{xcolor}
\definecolor{dkgreen}{rgb}{0.25,0.7,0.35}
\definecolor{dkred}{rgb}{0.7,0,0}

% Floats and referencing
\newcommand{\sectionref}[1]{section~\ref{sec:#1}}
\newcommand{\annexeref}[1]{annexe~\ref{ann:#1}}
\newcommand{\figuref}[1]{figure~\ref{fig:#1}}
\newcommand{\tabref}[1]{table~\ref{tab:#1}}
\usepackage{xparse}
\NewDocumentEnvironment{myfig}{mm}
{\begin{figure}[!ht]\centering}
{\caption{#2}\label{fig:#1}\end{figure}}

% Listing
\usepackage{listings}
\lstset{
  numbers=left,
  numberstyle=\tiny\color{gray},
  basicstyle=\rm\small\ttfamily,
  keywordstyle=\bfseries\color{dkred},
  frame=single,
  commentstyle=\color{gray}=small,
  stringstyle=\color{dkgreen},
  %backgroundcolor=\color{gray!10},
  %tabsize=2,
  rulecolor=\color{black!30},
  %title=\lstname,
  breaklines=true,
  framextopmargin=2pt,
  framexbottommargin=2pt,
  extendedchars=true,
  inputencoding=utf8x
}

\newcommand{\matlab}{\textsc{Matlab}}
\newcommand{\octave}{\textsc{GNU/Octave}}
\newcommand{\qtoctave}{\textsc{QtOctave}}
\newcommand{\oz}{\textsc{Oz}}
\newcommand{\java}{\textsc{Java}}
\newcommand{\clang}{\textsc{C}}
\newcommand{\keyword}{mot clef}

% Math symbols
\usepackage{amsmath}
\usepackage{amssymb}
\usepackage{amsthm}
\DeclareMathOperator*{\argmin}{arg\,min}
\DeclareMathOperator*{\argmax}{arg\,max}

% Sets
\newcommand{\Z}{\mathbb{Z}}
\newcommand{\R}{\mathbb{R}}
\newcommand{\Rn}{\R^n}
\newcommand{\Rnn}{\R^{n \times n}}
\newcommand{\C}{\mathbb{C}}
\newcommand{\K}{\mathbb{K}}
\newcommand{\Kn}{\K^n}
\newcommand{\Knn}{\K^{n \times n}}

% Chemistry
\newcommand{\std}{\ensuremath{^{\circ}}}
\newcommand\ph{\ensuremath{\mathrm{pH}}}

% Theorem and definitions
\theoremstyle{definition}
\newtheorem{mydef}{Définition}
\newtheorem{mynota}[mydef]{Notation}
\newtheorem{myprop}[mydef]{Propriétés}
\newtheorem{myrem}[mydef]{Remarque}
\newtheorem{myform}[mydef]{Formules}
\newtheorem{mycorr}[mydef]{Corrolaire}
\newtheorem{mytheo}[mydef]{Théorème}
\newtheorem{mylem}[mydef]{Lemme}
\newtheorem{myexem}[mydef]{Exemple}
\newtheorem{myineg}[mydef]{Inégalité}

% Unit vectors
\usepackage{esint}
\usepackage{esvect}
\newcommand{\kmath}{k}
\newcommand{\xunit}{\hat{\imath}}
\newcommand{\yunit}{\hat{\jmath}}
\newcommand{\zunit}{\hat{\kmath}}

% rot & div & grad & lap
\DeclareMathOperator{\newdiv}{div}
\newcommand{\divn}[1]{\nabla \cdot #1}
\newcommand{\rotn}[1]{\nabla \times #1}
\newcommand{\grad}[1]{\nabla #1}
\newcommand{\gradn}[1]{\nabla #1}
\newcommand{\lap}[1]{\nabla^2 #1}


% Elec
\newcommand{\B}{\vec B}
\newcommand{\E}{\vec E}
\newcommand{\EMF}{\mathcal{E}}
\newcommand{\perm}{\varepsilon} % permittivity

\newcommand{\bigoh}{\mathcal{O}}
\newcommand\eqdef{\triangleq}

\DeclareMathOperator{\newdiff}{d} % use \dif instead
\newcommand{\dif}{\newdiff\!}
\newcommand{\fpart}[2]{\frac{\partial #1}{\partial #2}}
\newcommand{\ffpart}[2]{\frac{\partial^2 #1}{\partial #2^2}}
\newcommand{\fdpart}[3]{\frac{\partial^2 #1}{\partial #2\partial #3}}
\newcommand{\fdif}[2]{\frac{\dif #1}{\dif #2}}
\newcommand{\ffdif}[2]{\frac{\dif^2 #1}{\dif #2^2}}
\newcommand{\constant}{\ensuremath{\mathrm{cst}}}

% Numbers and units
\usepackage[squaren, Gray]{SIunits}
\usepackage{sistyle}
\usepackage[autolanguage]{numprint}
%\usepackage{numprint}
\newcommand\si[2]{\numprint[#2]{#1}}
\newcommand\np[1]{\numprint{#1}}

\newcommand\strong[1]{\textbf{#1}}
\newcommand{\annexe}{\part{Annexes}\appendix}

% Bibliography
\newcommand{\biblio}{\bibliographystyle{plain}\bibliography{biblio}}

\usepackage{fullpage}
% le `[e ]' rend le premier argument (#1) optionnel
% avec comme valeur par défaut `e `
\newcommand{\hypertitle}[7][e ]{
\usepackage{hyperref}
{\renewcommand{\and}{\unskip, }
\hypersetup{pdfauthor={#6},
            pdftitle={Synth\`ese d#1#2 Q#3 - L#4#5},
            pdfsubject={#2}}
}

\title{Synth\`ese d#1#2 Q#3 - L#4#5}
\author{#6}

\begin{document}

\ifthenelse{\isundefined{\skiptitlepage}}{
\begin{titlepage}
\maketitle

 \paragraph{Informations importantes}
   Ce document est grandement inspiré de l'excellent cours
   donné par #7 à l'EPL (École Polytechnique de Louvain),
   faculté de l'UCL (Université Catholique de Louvain).
   Il est écrit par les auteurs susnommés avec l'aide de tous
   les autres étudiants, la vôtre est donc la bienvenue.
   Il y a toujours moyen de l'améliorer, surtout si le cours
   change car la synthèse doit alors être modifiée en conséquence.
   On peut retrouver le code source à l'adresse suivante
   \begin{center}
     \url{https://github.com/Gp2mv3/Syntheses}.
   \end{center}
   On y trouve aussi le contenu du \texttt{README} qui contient de plus
   amples informations, vous êtes invité à le lire.

   Il y est indiqué que les questions, signalements d'erreurs,
   suggestions d'améliorations ou quelque discussion que ce soit
   relative au projet
   sont à spécifier de préférence à l'adresse suivante
   \begin{center}
     \url{https://github.com/Gp2mv3/Syntheses/issues}.
   \end{center}
   Ça permet à tout le monde de les voir, les commenter et agir
   en conséquence.
   Vous êtes d'ailleurs invité à participer aux discussions.

   Vous trouverez aussi des informations dans le wiki
   \begin{center}
     \url{https://github.com/Gp2mv3/Syntheses/wiki}.
   \end{center}
   comme le statut des synthèses pour chaque cours
   \begin{center}
     \url{https://github.com/Gp2mv3/Syntheses/wiki/Status}.
   \end{center}
   vous pouvez d'ailleurs remarquer qu'il en manque encore beaucoup,
   votre aide est la bienvenue.

   Pour contribuer au bug tracker et au wiki, il vous suffira de
   créer un compte sur Github.
   Pour interagir avec le code des synthèses,
   il vous faudra installer \LaTeX.
   Pour interagir directement avec le code sur Github,
   vous devez utiliser \texttt{git}.
   Si cela pose problème,
   nous sommes évidemment ouverts à des contributeurs envoyant leurs
   changements par mail ou n'importe quel autre moyen.
\end{titlepage}
}{}

\ifthenelse{\isundefined{\skiptableofcontents}}{
\tableofcontents
}{}
}


% cfr http://en.wikibooks.org/wiki/LaTeX/Colors
\usepackage{color}
\usepackage[usenames,dvipsnames,svgnames,table]{xcolor}

\usepackage{appendix}

\usepackage{listings}
\usepackage{numprint}

\hypertitle{Informatique}{1}{1401}{Benoît Legat}{Benoît Legat}

\definecolor{dkgreen}{rgb}{0.25,0.7,0.35}
\definecolor{dkred}{rgb}{0.7,0,0}
\lstset{language={Java},numbers=left,numberstyle=\tiny\color{gray},
basicstyle=\rm\footnotesize,keywordstyle=\bfseries\color{dkred},frame=single,
commentstyle=\color{gray}=small, stringstyle=\color{dkgreen}}

\newcommand{\java}{\emph{Java}}

\part{Instructions, expressions et variables}

\section{Instructions}
Un programme est composé principalement d'instructions qu'il exécute l'une à la
suite de l'autre.

Chaque instruction se termine par un \verb|;|, on les écrit tout le temps
une par ligne pour aider à la lecture du programme mais le compilateur
pourrait comprendre le programme s'ils n'étaient que sur une seule ligne.

\subsection{Blocs d'instruction}
Certaines structures de \java{} (comme le \verb|if|) demandent un bloc.
Un bloc, c'est simplement un regroupement d'instruction.
On le délimite avec des accolades
(\verb|{| et \verb|}|).
\begin{lstlisting}
{
  <instruction_1>;
  <instruction_2>;
  ...
  <instruction_n>;
}
\end{lstlisting}

\paragraph{Portée des variables}
Les variables déclarées dans un bloc ne sont pas accessibles à l'extérieur
du bloc.

\section{Expressions}

Un composé fondamental de ces instructions est l'expression.
Ces expressions sont composée d'opérations entre d'autres expressions.
Une valeur ou une variable sont tous les deux des expressions.
Chaque expression est évaluée lors de l'exécution du programme.

\section{Variables}
Dans un programme, on utilise des variables comme en mathématiques.
Par contre, en informatique, les variables peuvent changer de valeur.

Chaque variable a un type et en \java{}, on ne peut pas changer son type.

Pour créer une variable, il faut lui donner un type et ce type ne changera
plus par la suite.
\subsection{Types}
Il y a deux sortes de type en \java{}, les types primitifs et les classes.

Il y a 8 types primitifs mais les plus importants sont les
\begin{itemize}
  \item \verb|int|, ce sont des nombres entiers, ils valent par exemple
    \verb|0| ou \verb|-8| ou encore \verb|42|;
  \item \verb|double|, ce sont les nombres à virgule, il valent par exemple
    \verb|3.14| ou \verb|1.41|;
  \item \verb|char|, ce sont les charactères, il valent par exemple \verb|'a'|
    ou \verb|'A'| ou \verb|'0'| ou même un retour à la ligne \verb|'\n'|;
  \item \verb|boolean|, ils valent \verb|true| ou \verb|false|;
\end{itemize}
Les autres sont comme les \verb|int| ou les \verb|double| mais utilisent plus
ou moins de mémoire. Cependant, à l'époque actuelle, la mémoire n'est plus
si limitée et ils sont moins utilisés.

Par contre il est important de remarquer ces deux types qu'ils ne savent
pas contenir des nombres d'une taille arbitraire, ils ont chacun leur limite
\begin{itemize}
  \item Un \verb|int| sait seulement contenir un entier appartenant à
    $[-2^{31}; 2^{31}[$. On retient que
    \[ 2^{31} = 2 \cdot \left(2^{10}\right)^3 = 2 \cdot (1024)^3
    \approx 2 \cdot (1000)^3 = \numprint{2e9} \]
  \item Un \verb|double| retient un nombre à virgule flottante, c'est à dire
    qu'il le retient en écriture scientifique.
    Il retient ses décimales et la puissance de 10.
    Il peut donc être très grand mais n'est pas si précis que ça.

    Vous pouvez vous en apercevoir en vérifiant que l'expression
    \begin{lstlisting}
0.1 + 0.2 == 0.3
    \end{lstlisting}
    est évalué à \verb|false|.
    C'est du au fait que les décimales sont écrite en binaire dans la mémoire
    et que ces nombres ne sont pas si rond en binaire.

    En règle générale, on ne test jamais l'égalité entre deux \verb|double|
    mais plutôt la valeur absolue de leur différence.
\end{itemize}

\subsection{Manipulation de variables}
On peut utiliser les variables de 3 manières différents, on peut les créer,
leur assigner une valeur et la lire dans une expression.
Dans les exemples de cette sous-section, on considèrera qu'on utilise des
\verb|int| mais c'est évidemment le même principe pour les autres types
primitifs.

\subsubsection{Création d'une variable}
Pour créer une variable, il faut lui choisir un nom qui n'est pas encore
utilisé, il ne peut pas commencer par un chiffre mais peut contenir chiffres,
lettres et underscores (\verb|_|).

Supposons qu'on crée un \verb|int| qu'on appelle \verb|foobar|.
\begin{lstlisting}
int foobar;
\end{lstlisting}
Lors de leur création, les variables sont initialisées à \verb|0| ce qui pour
les \verb|boolean| signifie \verb|false|.

On peut aussi donner une valeur lors de l'initialisation de la variable
comme dans l'exemple suivant
\begin{lstlisting}
int foobar = 1;
\end{lstlisting}

\subsubsection{Assignation}
On peut modifier la valeur d'une variable à tout moment du programme en lui
assignant la valeur d'une expression. Il faut bien évidemment que la valeur
de l'expression soit du même type. La syntaxe est la suivante
\begin{lstlisting}
foobar = <expression>;
\end{lstlisting}

Il est important de remarquer que le \verb|=| de l'assignation n'est pas le
même qu'en mathématique, on peut par exemple faire
\begin{lstlisting}
foobar = foobar + 1;
\end{lstlisting}

Dans ce cas ci, il est important d'insister sur l'ordre à l'exécution.
On évalue d'abord le membre de droite qui n'est rien d'autre qu'une expression
puis on assigne cette valeur dans le membre de droite.

\subsubsection{Casting}
Supposons que \verb|a| soit de type \verb|<type_a>| et \verb|b| soit de type
\verb|<type_b>| et qu'on veuille assigner à \verb|a| la valeur de \verb|b|.

Si \verb|<type_a>| et \verb|<type_b>| ne sont pas les mêmes, on a deux
possibilités.
\begin{itemize}
  \item Soit on fait un \emph{cast explicite}, c'est à dire qu'on
    spécifie à \java{} qu'on veut transformer la valeur de \verb|b| pour
    qu'elle ait le type \verb|<type_a>|. La syntaxe est la suivante
    \begin{lstlisting}
(<type_a>) b
    \end{lstlisting}
    Évidemment, ça ne marche pas pour n'importe quel type.
    On peut caster n'importe quel type primitif dans n'importe quel type
    primitif mais on ne peut pas caster un type primitif dans un objet ni
    le contraire (sauf entre les types primitifs et leur classe wrapper comme
    \verb|int| et \verb|Integer|).
    Par contre, on peut caster un objet dans un autre si et seulement si,
    l'un hérite de l'autre ou le contraire.
  \item Soit on fait un \emph{cast implicite}, comme son nom l'indique,
    on ne dit rien à \java{} mais il effectue tout de même un cast.
    Entre type primitifs, en règle générale,
    \java{} fait un cast quand il ne perd pas de précision.
    Les casts se faisant implicitement sont donc
    \begin{itemize}
      \item \verb|char| à \verb|int|;
      \item \verb|int| à \verb|double|;
      \item \verb|int| à \verb|Integer| et vice versa;
      \item \verb|char| à \verb|Character| et vice versa;
      \item \verb|double| à \verb|Double| et vice versa;
      \item \verb|boolean| à \verb|Boolean| et vice versa;
      \item \verb|<class_1>| à \verb|<class_2>| si \verb|<class_1>|
        hérite de \verb|<class_2>|.
    \end{itemize}
    Sans parler des \verb|byte|, \verb|short| et \verb|float|.
    De plus, il y a une relation de transitivité entre les casts
    implicites, c'est à dire que si on sait caster implicitement
    \verb|<type_1>| dans \verb|<type_2>| et
    \verb|<type_2>| dans \verb|<type_3>|, on sait caster implicitement
    \verb|<type_1>| dans \verb|<type_3>|.
\end{itemize}

\paragraph{Casting et héritage}
\label{sec:cast_class}
On a vu qu'on pouvait caster des objet dans une classe parente.
Mais quand on fait cela, on ne sait évidemment plus appeler ses méthodes
spécifiques que la classe parente n'a pas.
Grâce au keyword \verb|instanceof|, on peut néanmoins recaster l'objet
dans la classe de départ en vérifiant que c'est bien une instance de cette
classe.
Par exemple, on peut implémenter la méthode \verb|equals| comme suit
\begin{lstlisting}
public class Foo {
  int data;
  public Foo (int data) {
    this.data = data;
  }
  public boolean equals (Object o) {
    if (o instanceof Foo) {
      Foo foo = (Foo) o;
      return this.data == foo.data;
    }
    return false;
  }
}
\end{lstlisting}
Ainsi, ça permet de donner comme argument à \verb|equals| un objet de
n'importe quel classe car toute classe hérite de \verb|Object|.

\subsubsection{Les constantes}
En informatique, on aime bien se donner des règles supplémentaires à respecter
qui sont vérifiées par le compilateur pour diminuer le plus possible les
chances que notre programme compile mais ne fonctionne pas.

Pour cela, il est possible de spécifier que des variables sont constantes.
Dès qu'on voudra modifier leur valeur, le code ne compilera pas.

Pour cela, à la création de la variable, on ajoute le keyword \verb|final|
de la manière suivante.
\begin{lstlisting}
final <type> <name>[ = <expression>];
\end{lstlisting}
Si on ne donne pas \verb| = <expression>|, on peut encore lui faire une
assignation.
Ça permet d'initialiser les variables dans le constructeur en fonction
de ses paramètres.

\paragraph{Assignation de la même valeur}
En fait, on le droit d'assigner une valeur à une constante autant de
fois qu'on veut, tant que cette valeur est la même que celle de la constante
ou que la contante n'a pas encore été initialisée.

\paragraph{Convention}
Par convention, on écrit tout le temps les constantes en majuscule.

\subsubsection{Les raccourcis}
Certaines assignations sont très courrantes, elles possèdent donc un raccourcis.
\begin{lstlisting}
foo = foo + bar;
\end{lstlisting}
peut s'écrire
\begin{lstlisting}
foo += bar;
\end{lstlisting}
Il y a des raccourcis similaire pour les opérateurs
\verb|-|, \verb|*| et \verb|/|.

Encore plus raccourcis, au lieu d'écrire
\begin{lstlisting}
foo += 1;
\end{lstlisting}
on peut écrire
\begin{lstlisting}
foo++;
\end{lstlisting}
Il y a un raccourci pour l'opérateur \verb|-|.

\section{Les opérateurs}
Les expressions seraient bien pauvres si on ne pouvait pas les manipuler
entre elles.
Il existe les opérateurs suivants dont le résultat est du même type
que les deux variables sur lesquels ils sont appliqués
\begin{center}
  \begin{tabular}{|c|l|}
    \hline
    \verb|a + b| & Vaut la somme de \verb|a| et \verb|b|\\
    \hline
    \verb|a - b| & Vaut la soustraction de \verb|a| par \verb|b|\\
    \hline
    \verb|a * b| & Vaut la multiplication de \verb|a| et \verb|b|\\
    \hline
    \verb|a / b| & Vaut la division de \verb|a| par \verb|b|\\
    \hline
    \verb|a % b| & Vaut le reste de la division euclidienne
    de \verb|a| par \verb|b|\\
    \hline
  \end{tabular}
\end{center}
Il existe aussi des opérateurs dont le résultat est un \verb|boolean|
\begin{center}
  \begin{tabular}{|c|l|}
    \hline
    \verb|a == b| & Vaut \verb|true| si \verb|a| est égal à \verb|b|\\
    \hline
    \verb|a != b| & Vaut \verb|true| si \verb|a| est différent de \verb|b|\\
    \hline
    \verb|a < b| & Vaut \verb|true| si \verb|a| est plus petit
    strictement que \verb|b|\\
    \hline
    \verb|a <= b| & Vaut \verb|true| si \verb|a| est plus petit
    ou égal à \verb|b|\\
    \hline
    \verb|a > b| & Vaut \verb|true| si \verb|a| est plus grand
    strictement que \verb|b|\\
    \hline
    \verb|a >= b| & Vaut \verb|true| si \verb|a| est plus grand
    ou égal à \verb|b|\\
    \hline
  \end{tabular}
\end{center}

Tous ces opérateurs sont des opérateurs binaires car ils recquièrent deux
valeurs.
Il existe aussi un opérateur unaire et un opérateur ternaire
\begin{center}
  \begin{tabular}{|c|l|}
    \hline
    \verb|!a| & Vaut \verb|true| si \verb|a| vaut \verb|false| et
    vaut \verb|false| sinon\\
    \hline
    \verb|a ? b : c| & Vaut \verb|b| si \verb|a| vaut \verb|true| et
    vaut \verb|c| sinon\\
    \hline
  \end{tabular}
\end{center}

\part{Structures de contrôle}
Quand on éxecute un programme, les instructions se font exécuter l'une après
l'autre, dans l'ordre dans lequel elles ont été écrites.

Seulement, il existe des moyens de passez outre cet ordre.

\paragraph{Les conditions}
Une condition est toute expression s'évaluant en \verb|boolean|.

\section{Bloc, instruction ou structure de contrôle ?}
\label{sec:bloci}
Ce qui est magique avec les stuctures de contrôle, c'est qu'elles permettent
de manipuler, non plus des variables, mais carrément des instructions !
Elles permettent de déterminer si certaines instructions vont être exécutées
en fonction de la valeur d'une condition.

Comme on veut parfois appliquer cela sur plusieurs instructions,
on dit que ces structures conditionnelles s'appliquent sur
des blocs d'instructions.
Seulement, on peut aussi donner une seule instruction où même directement
une structure de contrôle sans les mettre dans un bloc (sans accolade).

En fait, on si on ne leur donne pas un bloc mais directement des instructions,
ces structures prennent la première instruction.

Attention néanmoins car la pratique de donner une instruction au lieu d'un
bloc à ces structures, bien qu'acceptée par \java{}, n'enchante pas
tous les programmeurs.
En effet, elle ajoute souvent de l'ambiguïté et rappelle donc pour beaucoup
du temps perdu à débugger.

Pour illustrer cela, remarquez que, contrairement aux apparences,
\begin{lstlisting}
if (a_number % another_number == 0)
  amount_of_divisors++;
  is_prime = false;
\end{lstlisting}
est équivalent à
\begin{lstlisting}
if (a_number % another_number == 0) {
  amount_of_divisors++;
}
is_prime = false;
\end{lstlisting}
et non à
\begin{lstlisting}
if (a_number % another_number == 0) {
  amount_of_divisors++;
  is_prime = false;
}
\end{lstlisting}

\section{Les structures conditionnelles}
Il est possible d'exécuter un bloc d'instruction seulement si une condition
est vraie.
\begin{lstlisting}
if (<condition>) <bloc>
\end{lstlisting}
Le bloc sera exécuté seulement si la condition vaut \verb|true|.

Il est possible de demander d'exécuter un autre bloc si la condition est fausse.
\begin{lstlisting}
if (<condition>) <bloc_1>
else <bloc_2>
\end{lstlisting}
Si la condition vaut \verb|true|, le premier bloc sera exécuté, sinon, le
second bloc sera exécuté.

\subsection{Quid des else if ?}
Rappelons-nous (voir Section~\ref{sec:bloci}) qu'à la place d'un bloc,
on peut donner une structure de contrôle.
On peut donc dire que
\begin{lstlisting}
if (<condition_1>) <bloc_1>
else if (<condition_2>) <bloc_2>
else <bloc_3>
\end{lstlisting}
est équivalent à
\begin{lstlisting}
if (<condition_1>) <bloc_1>
else {
  if (<condition_2>) <bloc_2>
  else <bloc_3>
}
\end{lstlisting}

Les \verb|else if| ne sont donc pas vraiment un nouveau concept lorsqu'on
connait les \verb|else| et les \verb|if|.

\section{Les boucles}
Une boucle exécute un bloc \emph{tant que} la condition est vraie.
Avant chaque exécution du bloc, il vérifie si la condition vaut \verb|true|.
Si c'est le cas, le bloc est exécuté.
Sinon, le programme poursuit son court.

La syntaxe est fort semblable au \verb|if|.
\begin{lstlisting}
while (<condition>) <bloc>
\end{lstlisting}

\subsection{Les boucles for}
Lorsqu'on veut itérer une opération, souvent, on a un itérateur qu'on
initialise avant la boucle et qu'on incrémente après chaque boucle.
\begin{lstlisting}
<init>;
while (condition) {
  ...
  <inc>;
}
\end{lstlisting}

La boucle \verb|for| permet d'écrire ça de façon plus condensée
\begin{lstlisting}
for (<init>; <condition>; <inc>) {
  ...
}
\end{lstlisting}

\begin{itemize}
  \item \verb|<init>| peut être nimporte quelle instruction mais si elle crée
    des variables, ces dernières seront locales au \verb|for|.
  \item \verb|<inc>| peut être n'importe quelle instruction, ça peut même
    contenir plusieurs instruction séparées par des virgules mais cette
    pratique n'est pas appréciée par tout le monde.
\end{itemize}
Le fait que les variables déclarées dans \verb|<init>| est assez pratique car
souvent, on y déclare des variables dont on aura pas besoin après le \verb|for|.
C'est souvent des variables qui nous permettent de parcourir un tableau,
c'est pourquoi on les appelle des itérateurs.

\part{Les tableaux}
Le tableau de \java{} est une structure de donnée assez intéressante.
Elle permet de stocker un nombre fixe de variables de même type
ou héritant d'une même classe (Voir Section~\ref{sec:cast_class}).

\section{Création de tableaux}
Une variable ne contient jamais un tableau en lui même,
tout comme les objects, elle contient une référence vers un tableau.
À sa création, elle vaut \verb|null|.

Pour créer un tableau, un utilise le keyword \verb|new|.
Il crée un nouveau tableau et renvoit sa référence.

\subsection{Création de tableau à une dimension}
On utilise la syntaxe suivante
\footnote{Ici, les \\ devant les [, ] servent à dire que ce n'est
pas de crochets pour dire que c'est optionnel mais vraiment des
crochets qu'il faut mettre.}
\begin{lstlisting}
<type>\[\] <name> = new <type>\[<length>\];
\end{lstlisting}

Par exemple, pour faire un tableau qu'on nomme \verb|tab| de 42 \verb|int|,
on écrit
\begin{lstlisting}
int[] tab = new int[42];
\end{lstlisting}

\subsection{Création de tableau à plusieurs dimension}
On fait un tableau à plus d'une dimension en faisant des tableaux de tableau.
Par exemple, si dans chaque case d'un tableau à 2 éléments, on met un
tableau de \verb|int| à 4 éléments, on a une matrice $2 \times 4$.
Ça s'écrit
\begin{lstlisting}
int[][] matrix = new int[2][4];
\end{lstlisting}

\section{Manipulation de tableau}
Lorsqu'on a créé un tableau, on peut changer ses valeurs (sauf si c'est un
tableau de constante bien entendu) ou juste y accéder.

Il faut cependant savoir qu'en informatique, on commence à compter à partir
de 0, donc la \verb|n|\ieme{} case est à l'indice \verb|n-1|.

Pour obtenir la 21\ieme{} case du tableau \verb|tab|, il faut donc faire
\begin{lstlisting}
tab[21]
\end{lstlisting}

Pour assigner à la 2\ieme{} colonne de la 1\iere{} ligne de \verb|matrix|
la valeur 42, il faut écrire
\begin{lstlisting}
matrix[0][1] = 42;
\end{lstlisting}

\subsection{Taille de tableau}
La taille d'un tableau est son nombre d'éléments.
Elle est fixe.
Pour obtenir la taille du tableau \verb|tab|, on fait
\begin{lstlisting}
tab.length
\end{lstlisting}
et ça vaudrait 2.

Si on voulait le nombre de colonnes d'une matrice, il suffit de demander
le nombre d'élément de n'importe quelle ligne (s'il y en a au moins une !).
\begin{lstlisting}
matrix[0].length
\end{lstlisting}
Mais dans la plupart des cas,
on peut juste redemander la longueur pour chaque ligne.

\subsection{Exemples}
Pour illustrer cela, voici comment faire une méthode qui fait la somme des
éléments d'un tableau de \verb|int| à deux dimension.
\begin{lstlisting}
public static int sumMatrix (int[][] matrix) {
  sum = 0;
  for (int i = 0; i < matrix.length; i++) {
    for (int j = 0; j < matrix[i].length; j++) {
      sum += matrix[i][j];
    }
  }
  return sum;
}
\end{lstlisting}

Et voici une méthode qui fait la copie d'un tableau.
En effet, quand on fait une assignation d'un tableau avec la valeur d'un autre
tableau, on fait une copie de la référence mais le tableau reste le même !
\begin{lstlisting}
public static int arrayCopy (int[] array) {
  int[] copy = new int[array.length];
  for (int i = 0; i < array.length; i++) {
    copy[i] = array[i];
  }
  return copy;
}
\end{lstlisting}

\part{Les classes}
\section{La portée}
Les variables et les méthodes ont une certaine portée, c'est à dire
qu'on ne peut pas y accéder depuis n'importe où.

À chaque fois qu'on défini une méthode ou une variable de classe ou
d'instance, on doit lui attribuer une portée.

Il existe 3 portées différentes
\begin{itemize}
  \item \verb|public|: peut être accédé depuis n'importe quelle
    classe et n'importe quel object.
  \item \verb|protected|: peut être accédé depuis toute classe ou object
    qui hérite de la classe en question.
  \item \verb|private|: peut être accédé uniquement depuis la classe
    en question et ses instances.
\end{itemize}

De plus, on ne peut pas accéder à une variable ou méthode d'instance
depuis une méthode de classe.

Pour spécifier la portée d'une variable ou d'une méthode, on rajoute
keywork \verb|public|, \verb|protected| ou \verb|private| en fonction
de la portée au début de sa définition.

En anglais, portée se dit \emph{scope}.

\section{Variables et méthodes}
Dans une classe, on peut définir des variables et des méthodes.
Ces derniers peuvent être soit statiques, soit non-statiques.
\begin{itemize}
  \item Statiques: ce sont les méthodes et variables de classe.
    Elles sont dans la classe uniquement pour des raisons de portée.
    Les variables statiques ne sont créées qu'une seule fois même si
    42 instances de la classe sont créées.
    C'est pourquoi les constantes sont souvent statiques.
  \item Non-statiques: ce sont les méthodes et variables d'instance.
    De nouvelles variables est créée à chaque création d'instance de
    la classe.
    Les variables d'instance référencées dans les méthodes d'instance
    sont celles spécifique de l'object en question.
\end{itemize}

Pour spécifier si une variable ou une méthode est statique,
on ajoute le keyword \verb|static|
à la suite du keyword pour la portée.
Si le keyword n'est pas présent, elle est non-statique.

\subsection{Méthodes}
Une méthode est comme une fonction mathématique, on lui donne des
argument et elle renvoit une valeur en fonction de ces arguments.

\subsubsection{Définition de méthode}
Pour définir une méthode, on doit spécifier le type de la valeur
qu'elle renvoit ainsi que le type de chacun des paramètres et le nom
avec lesquels on les référencera à l'intérieur de la méthode.
La syntaxe est la suivante
\begin{lstlisting}
<scope> <return_type> <name> (<type_1> <param_1>, ...) bloc
\end{lstlisting}
Cette ligne (sans le \verb|bloc|) est appelée la \emph{signature} de la méthode.
\begin{itemize}
  \item \verb|<scope>| est la portée de la méthode. Il peut être
    suivit par d'autres keywords.
  \item \verb|<return_type>| est le type de la valeur retournée par
    la méthode.
    Si la méthode ne retourne rien, \verb|<return_type>| vaut \verb|void|.
  \item \verb|<name>| est le nom de la méthode.
  \item \verb|<type_i>| est le type du \verb|i|\ieme{} paramètre et
    \verb|<param_i>| est son nom.

    \verb|<type_i> <param_i>| est appelé le \emph{paramètre formel}.
  \item \verb|bloc| est un bloc d'instruction doté d'un keyword supplémentaire:
    \verb|return|. Sa syntaxe est la suivante
    \begin{lstlisting}
    return <expression>;
    \end{lstlisting}
    \verb|return| termine la méthode immédiatement
    (même s'il restait des instructions à exécuter) et
    renvoie \verb|<expression>| comme valeur de retour de la méthode.
    \begin{itemize}
      \item
        Si \verb|<return_type>| vaut \verb|void|, on peut quand même
        arrêter l'exécution de la méthode avec l'instruction \verb|return;|
      \item
        Si \verb|<return_type>| ne vaut pas void,
        le compilateur doit pouvoir s'assurer qu'il pourra toujours
        trouver quoi retourner sinon, le code ne compilera pas.
        Par exemple, si, un \verb|return| est dans une structure de contrôle
        et pas après, il doit alors y en avoir un dans chaque embranchement
        et le \verb|else| doit être présent.
        Par exemple,
        \begin{lstlisting}
public static double div (double a, double b) {
  if (b != 0) {
    return a / b;
  }
}
        \end{lstlisting}
        ne compilera pas.
    \end{itemize}
\end{itemize}

\subsubsection{Appel de méthode}
Pour appeler une méthode, il faut utiliser la syntaxe suivante
\footnote{[ et ] sont là pour indiquer que c'est optionnel.}
\begin{lstlisting}
[<locator>.]<name>(<arg_1>, ...)
\end{lstlisting}
On peut remarquer que l'appel d'une méthode est une \emph{expression}
dont le type est le type de retour de la méthode en question.
Si elle retourne \verb|void|, on ne peut pas demander ce que revoit cette
expression car elle ne renvoit rien.
\begin{itemize}
  \item \verb|<locator>| dépend du cas dans lequel on est
    \begin{itemize}
      \item Si on veut accéder à une méthode de classe extérieure à la classe
        dans laquelle on est, il vaut le nom de cette classe.
        Si on est dans la classe en question, \verb|<locator>.| est optionnel.
      \item Si on veut accéder à une méthode d'instance, il vaut le nom
        de la variable référençant cet objet.
        Si on est dans l'objet en question, \verb|<locator>.| est optionnel ou
        on utilise le keyword \verb|this| pour obtenir la référence vers
        l'objet.
    \end{itemize}
  \item \verb|name| est le nom de la méthode.
  \item \verb|<arg_i>| est une \emph{expression} qui donne la valeur
    du \verb|i|\ieme{} argument de la méthode.
    Ça équivaut à faire une assignation
    \begin{lstlisting}
    <type_i> <param_i> = <arg_i>;
    \end{lstlisting}
    à la différence que le type auquel est évalué de l'expression
    \verb|<arg_i>| sert à déterminer quelle fonction est appelée.

    \verb|<arg_i>| est appelé le \emph{paramètre effectif}.
    \paragraph{Attention} Comme c'est une assignation,
    si \verb|<arg_i>| est une variable, \verb|<param_i>| sera une copie de
    cette variable !
    Dès lors, changer \verb|<param_i>| ne change pas la valeur de
    \verb|<arg_i>|.
    Attention néanmoins car si \verb|<arg_i>| est la référence vers
    un object, c'est la référence qui est copiée, pas l'objet !
\end{itemize}
\java{} va localiser la fonction à appeler en grâce au \verb|<locator>|,
au \verb|<name>| et au type des \verb|<arg_i>| (dans l'ordre dans lequels ils
sont !).

\subsection{Variables}
La syntaxe pour utiliser les variables dans une classe est la même
que celle des méthodes sauf qu'on ne donne pas d'arguments.
On la définit comme suit
\begin{lstlisting}
<scope> <return_type> <name>[ = <expression>];
\end{lstlisting}
où l'initialisation avec \verb|<expression>| est optionnelle.
Elle s'utilise dans une expression comme suit
\begin{lstlisting}
[<locator>.]<name>
\end{lstlisting}
Et on lui fait une assignation comme suit
\begin{lstlisting}
[<locator>.]<name> = <expression>;
\end{lstlisting}

\java{} distingue les variables des méthodes au fait qu'on ne mette pas
de parenthèses.
Par exemple,
\begin{lstlisting}
x.length
\end{lstlisting}
c'est la variable \verb|length| de l'objet \verb|x| et
\begin{lstlisting}
x.length()
\end{lstlisting}
c'est la méthode \verb|length| qui ne prend pas d'argument de l'objet \verb|x|.

%TODO:Objet vs instance:
% on ne peut pas dire objet d'une classe mais on peut dire instance d'une classe

\annexe
\section{Écrire, compiler et exécuter}
Nous allons différencier la programmation par IDE et en console.
Les remarques sur l'écriture, la programmation et la compilation
sont dans la partie console mais elles valent aussi pour la
programmation à travers un IDE.

\subsection{IDE}
Un \textbf{IDE} (\textbf{I}ntegrated \textbf{D}evelopment \textbf{E}nvironment)
fourni un éditeur de texte et des boutons pour compiler et exécuter.
Tout ça avec une interface graphique.

Pour \java{}, on peut citer notamment
\begin{itemize}
  \item Eclipse;
  \item Netbeans;
  \item BlueJ.
\end{itemize}
Ils sont tous les trois libres donc gratuits.

\subsection{Console}
Programmer en console signifie utiliser
l'éditeur de texte et le compilateur séparément.
On appelle ça programmer à la console car on appelle souvent le compilateur
et on exécute le programme à travers l'invite de commande.

Supposons qu'on crée le projet \verb|Foobar| contenant la classe
\verb|Foo| et la classe \verb|Bar|,
la méthode \verb|main| étant dans la classe \verb|Foo|.

Une structure classique serait de placer \verb|Foo.java| et \verb|Bar.java|
dans un dossier \verb|src/| et placer un fichier \verb|README|
à la racine du projet expliquant brièvement ce que fait le programme.

\subsubsection{Se déplacer dans la console}
Dans une console, on se situe toujours dans un dossier.
Lorqu'on ouvre une console, on se trouve dans le dossier personnel.
Pour savoir dans lequel on se trouve,
exécutez la commande suivante
(\textbf{P}rint \textbf{W}orking \textbf{D}irectory):
\begin{lstlisting}[language={sh}]
$ pwd
/home/jean
\end{lstlisting}

Pour voir les fichiers dans le dossier courant, exécutez la commande suivante:
%TODO: LiSt ou List Segments ?
\begin{lstlisting}[language={sh}, morekeywords={ls}]
$ ls
Desktop                    Music                Templates
Documents                  Pictures             Videos
Downloads                  Public
\end{lstlisting}

Pour se déplacer, il faut utiliser la commande
\verb|cd| (\textbf{C}hange \textbf{D}irectory).
\begin{lstlisting}[language={sh}, morekeywords={xx}]
$ cd Documents
$ pwd
/home/jean/Documents
$ cd ..
$ pwd
/home/jean
\end{lstlisting}

\subsubsection{Écrire}
Pour écrire \verb|Foo.java|, \verb|Bar.java| et \verb|README|,
il nous faut un éditeur de texte.

\begin{itemize}
  \item Sur Linux, les plus simples sont {\em Gedit} et {\em Kate};
  \item Sur Mac OS, {\em TextMate} est un excellent choix;
  \item Sur Windows, je conseille {\em Notepad++}.
\end{itemize}

Il y en a plein d'autre évidemment. Comment ne pas citer {\em Emacs} et
{\em Vim} qui, bien qu'étant très anciens, sont considérés par beaucoup
pour être les plus complets et les plus puissants.
Attention néanmoins, la {\em learning curve} de ces derniers est assez raide.

Si vous n'avez pas d'interface graphique et que vous ne voulez pas utiliser
un éditeur de texte aussi compliqué que {\em Emacs} ou {\em Vim}, considérez
{\em nano} ou {\em pico}.

Pour lancer un éditeur de texte depuis la console,
positionnez-vous dans le même dossier que le fichier à ouvrir
et exécutez la commande suivante:
\begin{lstlisting}[language={sh}, morekeywords={vim}]
$ vim Foo.java
\end{lstlisting}
Pour un éditeur de texte graphique,
il vaut mieux l'appeler ajouter un \verb|&| pour qu'il s'exécute en
background et que l'invite de commande ne se bloque pas:

\begin{lstlisting}[language={sh}, morekeywords={gedit}]
$ gedit Foo.java &
\end{lstlisting}

\subsubsection{Compiler}
Il faut compiler chaque classe en fichier \verb|.class|.
Pour cela, la commande à utiliser est \verb|javac|.
Avec un terminal ouvert et positionné dans le fichier \verb|src/|,
exécutez la commande suivante:
\begin{lstlisting}[language={sh}, morekeywords={javac}]
$ javac Foo.java Bar.java
\end{lstlisting}
Notez que l'ordre entre les fichiers \verb|.java| n'a aucune importance.

\subsubsection{Exécuter}
À nouveau, ouvrez un terminal et positionnez le dans le fichier \verb|src/|.
Il faut dire à \java{} dans quelle classe se trouve la méthode \verb|main|.
Dans notre cas, c'est dans la classe \verb|Foo|.
Dès lors, exécutez la commande suivante:
\begin{lstlisting}[language={sh}, morekeywords={java}]
$ java Foo
\end{lstlisting}
Notez qu'on écrit pas d'extension à \verb|Foo|, on écrit ni \verb|Foo.java|
ni \verb|Foo.class|.

\end{document}
