\documentclass[fr]{../../../eplsummary}

\usepackage{array}
\usepackage{fancybox}
\usepackage{float}
\usepackage{colortbl}
\usepackage{makecell}
\usepackage{graphicx}
\usepackage{titlesec}
\usepackage{qtree}
\usepackage{tensor}
\usepackage{circuitikz}

\usepackage{../../../eplelec}
\usepackage[SIunits]{../../../eplunits}
%\usetikzlibrary{babel}

\hypertitle[']{\'Electricit\'e}{1}{FSAB}{1201}
{Nicolas Cognaux\and Guillaume Fran\c{c}ois\and Beno\^it Legat}
{Jean-Didier Legat}

\part{Électrostatique}
\section{Champ électrique et charge électrique}
Deux charges de même signe se repoussent.
Deux charges de signes opposés s'attirent.

\subsection{Loi de Coulomb}
L'intensité de cette force vaut
\[ |F| = \frac{1}{4\pi\perm_0}\frac{|q_1q_2|}{r^2} \]
avec $q_1$ et $q_2$ les charges respectives, $r$ la distance qui les sépare et $\perm_0$ la permittivité du vide
\[ \perm_0 \approx \si{8.85e-12}{C^2/Nm^2} \]

On utilise aussi parfois
\[ k \eqdef \frac{1}{4\pi\perm_0} \approx
\si{9e9}{\newton\squaren{\meter}\per\squaren{\coulomb}} \]

\subsection{Champ électrique}
Le champ électrique est la force électrostatique par unité de charge
\[ \vec{E} = \frac{\vec{F}}{q} \]
Le champ électrique $\vec{E}$ causé par une charge $q$ située à une distance $r$ peut donc être réécrit comme suit
\[ \vec{E} = \frac{1}{4\pi\perm_0}\frac{q}{r^2}\hat{r} \]
$\hat{r}$ est le vecteur unitaire pointant radialement vers l'extérieur.
L'amplitude de $\vec{E}$ en un point peut être vue comme la force que subirait une charge de $\SI{+1}{C}$ en ce point. \\\
\begin{tabular}{ll}
  Champ d'un point de charge & \(\frac{1}{4\pi\perm_0}\frac{q}{r^2}\)\\
  Champ à  l'extérieur d'une sphère &
  \(\frac{1}{4\pi\perm_0}\frac{Q}{r^2}\)\\
  Champ à  l'intérieur d'une sphère &
  \(\frac{1}{4\pi\perm_0}\frac{Qr}{R^3}\)\\
  Champ d'une plaque infinie & \(\frac{\sigma}{2\perm_0}\)\\
  Champ entre deux plaques de charges opposées & \(\frac{\sigma}{\perm_0}\)
\end{tabular}

\subsection{Lignes de champ électrique}
Les lignes de champ électrique partent
de la charge positive et vont vers la charge négative.
\begin{itemize}
  \item Elles ne s'intersectent jamais;
  \item Leur nombre donne une idée de l'intensité du champ électrique;
  \item Le champ électrique est tangent aux lignes de champ électrique.
\end{itemize}

\section{Dipôles}
Un dipôle électrique est une paire de charges électriques de même amplitude $q$ mais de signes opposés, séparés par une distance $d$. Le moment $\vec{p}$ du dipôle, orienté de la charge négative à la charge positive, a une amplitude $p=qd$. Dans un champ électrique $\vec{E}$, le dipôle subit un moment de force
\[ \vec{\tau} = \vec{p} \times \vec{E} \]
L'énergie potentielle vaut
\[ U = -\vec{p} \cdot \vec{E} \]

\section{Flux électrique et loi de Gauss}
Le flux électrique mesure la quantité de champ électrique passant à travers une surface.
\[ \Phi_E = \int \vec{E} \cdot \dif\vec{A} =
\int E_{\perp} \dif A = \int E\cos\phi \, \dif A \]

\subsection{Loi de Gauss}
Si ce flux $\Phi_E$ est calculé sur une surface fermé,
alors il vaut la somme des charges à l'intérieur
de cette surface fermée divisée par $\perm_0$
\[ \oint \vec{E} \cdot \dif\vec{A} = \frac{Q_\mathrm{encl}}{\perm_0} \]

\subsection{Application}
Pour les conducteurs, les charges sont à la surface,
soit $\sigma$ la quantité de charge par unité de surface, on a par Gauss
\[ \oint E_\perp \dif A = \frac{\sigma A}{\perm_0} \]
Si la forme du conducteur est suffisamment symétrique
pour que $E_\perp$ soit constant, on a donc
\[ E_{\perp} = \frac{\sigma}{\perm_0} \]

\section{Energie potentielle électrique}
Le travail que doit effectuer une charge pour se rendre de $a$ à $b$ est égal à la différence de potentiel électrique entre ces deux points
\[ W_{a\rightarrow b} = U_a - U_b = -\Delta U = \int_{a}^{b} \vec{F} \cdot \dif \vec{l} \]
Dans le cas d'une charge $q_0$ se déplaçant sous l'effet d'une charge $Q$, on a
\[ W_{a\rightarrow b} 
    = \int_{a}^{b} \vec{F} \cdot \dif \vec{l}
    = \int_{r_a}^{r_b} F \dif r
    = \int_{r_a}^{r_b}\frac{1}{4\pi\perm_0}\frac{Qq_0}{r^2} \dif r
    = \frac{Qq_0}{4\pi\perm_0}\left(\frac{1}{r_a} - \frac{1}{r_b}\right) \]

Ainsi, l'énergie potentielle que possède une charge $q_0$ en présence d'une autre charge $q$ vaut, à une constance près,
\[ U = \frac{1}{4\pi\perm_0} \frac{qq_0}{r} \]
De manière générale, l'énergie potentielle en présence d'un ensemble de charges vaut, à une constante près,
\[ U = \frac{q_0}{4\pi\perm_0} \sum_i \frac{q_i}{r_i} \]

\section{Potentiel électrique}
Le potentiel électrique est l'énergie potentielle par unité de charge
\[ V \eqdef \frac{U}{q_0} \]
La différence de potentiel entre deux points représente le travail que doit effectuer une charge de \SI{+1}{C} pour se mouvoir entre ces points.

\begin{tabular}{ll}
    Potentiel dû à une charge ponctuelle & $\frac{1}{4\pi\perm_0} \frac{q}{r}$ \\
    Potentiel dû à un ensemble de charges & $\frac{1}{4\pi\perm_0} \sum_i \frac{q_i}{r_i}$ \\
    Potentiel dû à une distribution de charge & $\frac{1}{4\pi\perm_0} \int\frac{\dif q}{r}$
\end{tabular}

On peut également calculer une différence de potentiel en intégrant le champ électrique
\[ V_{ab} = V_a - V_b = \int_{a}^{b}\vec{E}\cdot\dif\vec{l}
= \int_{a}^{b}E\cos{\phi} \dif l \]
Inversement, le champ électrique en un point est le gradient du potentiel électrique en ce point
\[ \vec{E} = - \grad V \]

\section{Capacités et diélectriques}
Une capacité est une paire de conducteurs séparés par un isolant. Lorsqu'elle est chargée, chaque conducteur possède une charge opposée de même amplitude $Q$, proportionnelle à la différence de potentiel entre les conducteurs
\[ Q = CV \qquad C = \perm_0\frac{A}{d} \]
Capacité équivalente
\begin{description}
    \item[en série] $\frac{1}{C_\text{éq}} = \sum_i \frac{1}{C_i}$
    \item[en parallèle] $C_\text{éq} = \sum_i C_i$
\end{description}

L'énergie stockée dans une capacité vaut
\[ U = \frac{1}{2}CV^2 = \frac{Q^2}{2C} = \frac{1}{2}QV \]
La densité énergétique (énergie par unité de volume) vaut
\[ u = \frac{1}{2} \perm_0 E^2 \]

L'insertion entre les conducteurs d'un matériau diélectrique augmente la capacité d'un facteur $K$, appelé constante diélectrique du matériau
\[ \perm = K \perm_0 \qquad C = KC_0 = \perm \frac{A}{d} \qquad V = \frac{1}{K}V_0 \]
\[ u = \frac{1}{2} \perm E^2 \]

Loi de Gauss dans un diélectrique ($Q_\text{encl-free}$ n'inclut que les charges libres)
\[
    \oint K\vec{E} \cdot \dif\vec{A} = \frac{Q\text{encl-free}}{\perm_0}
\]

\part{Courant continu}
%      ___
%  ___(- +)___
% |    ---    |
% |    ELEC   |
% |_____||____|
\section{Courant électrique (DC)}
\[ I = \fdif{q}{t}\qquad{\qquad{\qquad}}\rho =
\frac{E}{J}\qquad{\qquad{\qquad}}q(t) = \int_{t_0}^{t}i(t) \dif t + q{t_0} \]

\section{Résistance et résistivité}
\[ R = \frac{\rho{L}}{A} \qquad{\qquad{\qquad}}V = IR \]

\section{Force électromotrice et puissance}
\[ P = VI = I^2R \] 
\[ V_{ab} = \EMF - IR_{interne}\qquad{\qquad{\qquad}}w
= \int_{t_1}^{t_2}p(t) \dif t \]

\section{Lois de Kirchoff's}
\[ \sum{I} = 0  \qquad{\qquad{\qquad}}\sum{V} = 0 \]

\section{Capacités et inductances}
\begin{table}[H]
  \begin{center}
    \begin{tabular}{|m{6.5cm}|m{6.5cm}|}
      \hline
      \multicolumn{1}{|c|}{\textbf{Capacité}} &
      \multicolumn{1}{c|}{\textbf{Inductance}}\\
      \hline
      \[ I(t) = C\fdif{V}{t} \] & \[ V(t) = L\fdif{I}{t} \]\\
      \[ \tau = RC \] & \[ \tau = \frac{L}{R} \]\\
      \[ U = \frac{CV^2}{2} = \frac{Q^2}{2C} \] & \[ U = \frac{LI^2}{2} \]\\
      \hline
      \multicolumn{1}{|c|}{\textbf{Circuit RC}} &
      \multicolumn{1}{c|}{\textbf{Circuit RL}}\\
      \hline
      \[ \fdif{V_{0-}}{t} = 0 \Rightarrow I_{0-} = 0 \]
      & \[ \fdif{I_{0-}}{t} = 0 \Rightarrow V_{0-} = 0 \]\\
      \[ U_{0-} = U_{0+} \Rightarrow V_{0-} = V_{0+} \]
      & \[ U_{0-} = U_{0+} \Rightarrow I_{0-} = I_{0+} \]\\
      \[ \fdif{V_{\infty}}{t} = 0 \Rightarrow I_{\infty} = 0 \]
      & \[ \fdif{I_{\infty}}{t} = 0 \Rightarrow V_{\infty} = 0 \]\\
      \[ V_{t>0}(t) = A + Be^{-t/\tau} \]
      & \[ I_{t>0}(t) = A + Be^{-t/\tau} \]\\
      \[ V_{0+} = A + B \] & \[ I_{0+} = A + B \]\\
      \[ V_{\infty} = A \] & \[ I_{\infty} = A \]\\
      \hline
    \end{tabular}
  \end{center}
\end{table}

\section{Circuit R-L-C}
\[ \frac{Q^2}{2C} + \frac{LI^2}{2} = \frac{Q^{2}_{max}}{2C} \]
\[ I = \pm \sqrt{\frac{Q^{2}_{max} - q^2}{LC}} \]
\[ w = \sqrt{\frac{1}{LC}}\qquad{\qquad{\qquad}}w' = \sqrt{\frac{1}{LC}
- \frac{R^2}{4C^{2}}} \]

\part{Courant alternatif}
\begin{mynota}
  Quand le courant (resp. la tension) n'est pas constant, on l'écrit avec
  une majuscule, i.e. $i$ (resp. $v$).
  Lorsque c'est une sinusoïde, on écrit le maximum de la sinusoïde
  avec une majuscule, e.g. $i = I \cos(\omega t)$.
\end{mynota}
\begin{mydef}[Courants et tensions alternatives (AC)]
Dire qu'un courant ou une tension est alternative, c'est dire
qu'elle n'est pas constante et dépend du temps.
Si la phase de $i$ est nulle,
\[ i = I\cos(\omega{t})
\qquad{\qquad{\qquad}}
v = V\cos{(\omega{t} + \phi_2)} \]
%\[ X(t) = A + Be^{-\frac{t}{\tau}} \] % FIXME ???
\end{mydef}

\section{Valeurs efficaces}
\[ I_\mathrm{rms} = \frac{I}{\sqrt{2}}\qquad{\qquad{\qquad}}V_\mathrm{rms} =
\frac{V}{\sqrt{2}}\qquad{\qquad{\qquad}}P_\mathrm{rms}
= V_\mathrm{rms}I_\mathrm{rms}\cos{\theta} \]

\section{Capacités et inductances}
\begin{table}[H]
  \begin{center}
    \begin{tabular}{m{6.5cm}|m{6.5cm}}
      \textbf{Capacité} & \textbf{Inductance}\\
      \hline
      \[ V_C = \frac{I_{0}\sin(\omega{t})}{\omega{C}}\]&\[V_{L} =
      -L\omega{}I_{0}\sin(\omega{t}) \]\\
      \[ X_C = \frac{1}{\omega{C}}\]&\[X_L = L\omega \]
    \end{tabular}
  \end{center}
\end{table}

\section{Circuit R-L-C}
\[Z = \sqrt{R^2 + (X_L - X_C)^2} =
\sqrt{R^2 + \left(\omega{L} - \frac{1}{\omega{C}}\right)^2}\]
\[\tan{\phi} = \cfrac{\omega{L} -\cfrac{1}{\omega{C}}}{R}\]

\section{Transformateur}
% http://forum.mathematex.net/latex-f6/tikz-probleme-de-library-sans-doute-t15733.html
\shorthandoff{:!}
\begin{center}
  \begin{circuitikz}[american voltages] \draw
    (0,0) node[transformer] (T) {}
    (T.B1) -| (3.5, 0) to[R, v=$V_2$, i = $I_2$] (3.5,-2) |- (T.B2)
    (T.A1) -| (-3.5, 0) to[sV, v=$V_1$, i = $I_1$] (-3.5,-2) |- (T.A2)
    (-1,-1) node{$N_1$}
    (1,-1) node{$N_2$}
    ;
  \end{circuitikz}
\end{center}
\shorthandon{:!}

Le transformateur respecte deux équations
\[\frac{V_1}{V_2} = \frac{N_1}{N_2}\qquad{\qquad{\qquad}}V_1I_1 = V_2I_2\]

Par la conservation du flux (voir la matière du cours LFSAB1202), on a
\[ \frac{V_1}{V_2} = \frac{N_1}{N_2} \]

Si on considère que le transformateur ne perd pas d'énergie, on a
\[ V_1I_1 = V_2I_2 \]

Attention au sens des courants, si on les défini dans l'autre sens,
les signes peuvent changer donc ce sont des égalités au signe près.

% TODO insert http://www.forum-epl.be/viewtopic.php?p=108384#108384

\end{document}
