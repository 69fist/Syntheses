\documentclass[11pt,a4paper]{article}

% French
\usepackage[utf8x]{inputenc}
\usepackage[frenchb]{babel}
\usepackage[T1]{fontenc}
\usepackage{lmodern}
\usepackage{ifthen}

% Color
% cfr http://en.wikibooks.org/wiki/LaTeX/Colors
\usepackage{color}
\usepackage[usenames,dvipsnames,svgnames,table]{xcolor}
\definecolor{dkgreen}{rgb}{0.25,0.7,0.35}
\definecolor{dkred}{rgb}{0.7,0,0}

% Floats and referencing
\newcommand{\sectionref}[1]{section~\ref{sec:#1}}
\newcommand{\annexeref}[1]{annexe~\ref{ann:#1}}
\newcommand{\figuref}[1]{figure~\ref{fig:#1}}
\newcommand{\tabref}[1]{table~\ref{tab:#1}}
\usepackage{xparse}
\NewDocumentEnvironment{myfig}{mm}
{\begin{figure}[!ht]\centering}
{\caption{#2}\label{fig:#1}\end{figure}}

% Listing
\usepackage{listings}
\lstset{
  numbers=left,
  numberstyle=\tiny\color{gray},
  basicstyle=\rm\small\ttfamily,
  keywordstyle=\bfseries\color{dkred},
  frame=single,
  commentstyle=\color{gray}=small,
  stringstyle=\color{dkgreen},
  %backgroundcolor=\color{gray!10},
  %tabsize=2,
  rulecolor=\color{black!30},
  %title=\lstname,
  breaklines=true,
  framextopmargin=2pt,
  framexbottommargin=2pt,
  extendedchars=true,
  inputencoding=utf8x
}

\newcommand{\matlab}{\textsc{Matlab}}
\newcommand{\octave}{\textsc{GNU/Octave}}
\newcommand{\qtoctave}{\textsc{QtOctave}}
\newcommand{\oz}{\textsc{Oz}}
\newcommand{\java}{\textsc{Java}}
\newcommand{\clang}{\textsc{C}}
\newcommand{\keyword}{mot clef}

% Math symbols
\usepackage{amsmath}
\usepackage{amssymb}
\usepackage{amsthm}
\DeclareMathOperator*{\argmin}{arg\,min}
\DeclareMathOperator*{\argmax}{arg\,max}

% Sets
\newcommand{\Z}{\mathbb{Z}}
\newcommand{\R}{\mathbb{R}}
\newcommand{\Rn}{\R^n}
\newcommand{\Rnn}{\R^{n \times n}}
\newcommand{\C}{\mathbb{C}}
\newcommand{\K}{\mathbb{K}}
\newcommand{\Kn}{\K^n}
\newcommand{\Knn}{\K^{n \times n}}

% Chemistry
\newcommand{\std}{\ensuremath{^{\circ}}}
\newcommand\ph{\ensuremath{\mathrm{pH}}}

% Theorem and definitions
\theoremstyle{definition}
\newtheorem{mydef}{Définition}
\newtheorem{mynota}[mydef]{Notation}
\newtheorem{myprop}[mydef]{Propriétés}
\newtheorem{myrem}[mydef]{Remarque}
\newtheorem{myform}[mydef]{Formules}
\newtheorem{mycorr}[mydef]{Corrolaire}
\newtheorem{mytheo}[mydef]{Théorème}
\newtheorem{mylem}[mydef]{Lemme}
\newtheorem{myexem}[mydef]{Exemple}
\newtheorem{myineg}[mydef]{Inégalité}

% Unit vectors
\usepackage{esint}
\usepackage{esvect}
\newcommand{\kmath}{k}
\newcommand{\xunit}{\hat{\imath}}
\newcommand{\yunit}{\hat{\jmath}}
\newcommand{\zunit}{\hat{\kmath}}

% rot & div & grad & lap
\DeclareMathOperator{\newdiv}{div}
\newcommand{\divn}[1]{\nabla \cdot #1}
\newcommand{\rotn}[1]{\nabla \times #1}
\newcommand{\grad}[1]{\nabla #1}
\newcommand{\gradn}[1]{\nabla #1}
\newcommand{\lap}[1]{\nabla^2 #1}


% Elec
\newcommand{\B}{\vec B}
\newcommand{\E}{\vec E}
\newcommand{\EMF}{\mathcal{E}}
\newcommand{\perm}{\varepsilon} % permittivity

\newcommand{\bigoh}{\mathcal{O}}
\newcommand\eqdef{\triangleq}

\DeclareMathOperator{\newdiff}{d} % use \dif instead
\newcommand{\dif}{\newdiff\!}
\newcommand{\fpart}[2]{\frac{\partial #1}{\partial #2}}
\newcommand{\ffpart}[2]{\frac{\partial^2 #1}{\partial #2^2}}
\newcommand{\fdpart}[3]{\frac{\partial^2 #1}{\partial #2\partial #3}}
\newcommand{\fdif}[2]{\frac{\dif #1}{\dif #2}}
\newcommand{\ffdif}[2]{\frac{\dif^2 #1}{\dif #2^2}}
\newcommand{\constant}{\ensuremath{\mathrm{cst}}}

% Numbers and units
\usepackage[squaren, Gray]{SIunits}
\usepackage{sistyle}
\usepackage[autolanguage]{numprint}
%\usepackage{numprint}
\newcommand\si[2]{\numprint[#2]{#1}}
\newcommand\np[1]{\numprint{#1}}

\newcommand\strong[1]{\textbf{#1}}
\newcommand{\annexe}{\part{Annexes}\appendix}

% Bibliography
\newcommand{\biblio}{\bibliographystyle{plain}\bibliography{biblio}}

\usepackage{fullpage}
% le `[e ]' rend le premier argument (#1) optionnel
% avec comme valeur par défaut `e `
\newcommand{\hypertitle}[7][e ]{
\usepackage{hyperref}
{\renewcommand{\and}{\unskip, }
\hypersetup{pdfauthor={#6},
            pdftitle={Synth\`ese d#1#2 Q#3 - L#4#5},
            pdfsubject={#2}}
}

\title{Synth\`ese d#1#2 Q#3 - L#4#5}
\author{#6}

\begin{document}

\ifthenelse{\isundefined{\skiptitlepage}}{
\begin{titlepage}
\maketitle

 \paragraph{Informations importantes}
   Ce document est grandement inspiré de l'excellent cours
   donné par #7 à l'EPL (École Polytechnique de Louvain),
   faculté de l'UCL (Université Catholique de Louvain).
   Il est écrit par les auteurs susnommés avec l'aide de tous
   les autres étudiants, la vôtre est donc la bienvenue.
   Il y a toujours moyen de l'améliorer, surtout si le cours
   change car la synthèse doit alors être modifiée en conséquence.
   On peut retrouver le code source à l'adresse suivante
   \begin{center}
     \url{https://github.com/Gp2mv3/Syntheses}.
   \end{center}
   On y trouve aussi le contenu du \texttt{README} qui contient de plus
   amples informations, vous êtes invité à le lire.

   Il y est indiqué que les questions, signalements d'erreurs,
   suggestions d'améliorations ou quelque discussion que ce soit
   relative au projet
   sont à spécifier de préférence à l'adresse suivante
   \begin{center}
     \url{https://github.com/Gp2mv3/Syntheses/issues}.
   \end{center}
   Ça permet à tout le monde de les voir, les commenter et agir
   en conséquence.
   Vous êtes d'ailleurs invité à participer aux discussions.

   Vous trouverez aussi des informations dans le wiki
   \begin{center}
     \url{https://github.com/Gp2mv3/Syntheses/wiki}.
   \end{center}
   comme le statut des synthèses pour chaque cours
   \begin{center}
     \url{https://github.com/Gp2mv3/Syntheses/wiki/Status}.
   \end{center}
   vous pouvez d'ailleurs remarquer qu'il en manque encore beaucoup,
   votre aide est la bienvenue.

   Pour contribuer au bug tracker et au wiki, il vous suffira de
   créer un compte sur Github.
   Pour interagir avec le code des synthèses,
   il vous faudra installer \LaTeX.
   Pour interagir directement avec le code sur Github,
   vous devez utiliser \texttt{git}.
   Si cela pose problème,
   nous sommes évidemment ouverts à des contributeurs envoyant leurs
   changements par mail ou n'importe quel autre moyen.
\end{titlepage}
}{}

\ifthenelse{\isundefined{\skiptableofcontents}}{
\tableofcontents
}{}
}


\usepackage{array}
\usepackage{fancybox}
\usepackage{float}
\usepackage{colortbl}
\usepackage{makecell}
\usepackage{graphicx}
\usepackage{titlesec}
\usepackage{qtree}
\usepackage{tensor}
\usepackage{circuitikz}

\hypertitle{Électricité}{1}{1201}{Nicolas Cognaux, Guillaume François et Benoît Legat}{Nicolas Cognaux \and Guillaume François \and Benoît Legat}

\part{Électrostatique}
\section{Champ électrique et charge électrique}
Deux charges avec le même signe se repoussent.
Deux charges avec un signe opposés s'attirent.

\subsection{Loi de Coulomb}
L'intensité de cette force vaut
\[ |F| = \frac{1}{4\pi\varepsilon_0}\frac{|q_1q_2|}{r^2} \]
On utilise aussi parfois
\[ k \eqdef \frac{1}{4\pi\varepsilon_0} \approx
\si{9e9}{\newton\squaren{\meter}\per\squaren{\coulomb}} \]
\subsection{Electric Field and Electric Forces}
Le champ électrique est la force électrique par unité de charge
\[ \vec{E} = \frac{\vec{F}}{q} \]
$\vec{E}$ peut donc être réécrit comme suit
\[ \vec{E} = \frac{1}{4\pi\varepsilon_0}\frac{q}{r^2}\hat{r} \]
\begin{tabular}{ll}
	Champ d'un point de charge & \(\frac{1}{4\pi\varepsilon_0}\frac{q}{r^2}\)\\
	Champ à  l'extérieur d'une sphère & \(\frac{1}{4\pi\varepsilon_0}\frac{Q}{r^2}\)\\
	Champ à  l'intérieur d'une sphère & \(\frac{1}{4\pi\varepsilon_0}\frac{Qr}{R^3}\)\\
	Champ d'une plaque infinie & \(\frac{\sigma}{2\varepsilon_0}\)\\
	Champ entre deux plaques & \(\frac{\sigma}{\varepsilon_0}\)
\end{tabular}
\subsection{Lignes de champ électrique}
Les lignes de champ électrique partent de la charge positive et vont vers la charge négative.
\begin{itemize}
	\item Elles ne s'intersectent jamais;
	\item Leur nombre donne une idée de l'intensité du champ électrique;
	\item Le champ électrique est tangeant aux lignes de champ électrique.
\end{itemize}

\section{Dipôles}
Si deux charges opposées sont liés, elles peuvent former un dipôle électrique.
Il y a rotation donc il y a un couple.

L'intensité du moment de force vaut
\[ \tau = pE\sin{\theta} \]
\[ \vec{\tau} = \vec{P} \times \vec{E} \]

L'énergie potentielle d'un dipôle électrique est
\[ U = -\vec{p} \cdot \vec{E} \]

\section{Flux électrique}
Le flux électrique est la quantité de charges électriques passant à travers une surface.
\[ \Phi_E = \int \vec{E} \cdot d\vec{A} = \int E_{\perp} dA = \int E\cos\phi dA \]
\subsection{Loi de Gauss}
Si ce flux $\Phi_E$ est calculé sur une surface fermé, alors il vaut la somme des charges à l'intérieur de cette surface fermée divisée par $\varepsilon_0$
\[ \oint \vec{E} \cdot d\vec{A} = \frac{Q_\mathrm{encl}}{\varepsilon_0} \]
\subsection{Application}
Pour les conducteurs, les charges sont à la surface, soit $\sigma$ la quantité de charge par unité de surface, on a par Gauss
\[ \oint E_\perp dA = \frac{\sigma A}{\varepsilon_0} \]
Si la forme du conducteur est suffisament symétrique pour que $E_\perp$ soit constant, on a donc
\[ E_{\perp} = \frac{\sigma}{\varepsilon_0} \]

\section{Energie électrique}
\[ W_{a\rightarrow b} = U_a - U_b = -\Delta{U} = \int_{a}^{b}\vec{F} d\vec{r} \]
\[ \int_{a}^{b}F dr = \int_{a}^{b}\frac{1}{4\pi\varepsilon_0}\frac{Qq_0}{r^2} dr = \frac{Qq_0}{4\pi\varepsilon_0}\left(\frac{1}{a} - \frac{1}{b}\right) \]

\section{Potentiel électrique}
\[ V \eqdef \frac{U}{q_0} = \frac{Q}{4\pi\varepsilon_0{r}} \]
\[ V_{a \rightarrow b} = V_a - V_b = \int_{a}^{b}\vec{E}.d\vec{l} = \int_{a}^{b}E\cos{\phi} dl \]

\section{Capacités et diélectriques}
\[ C = \frac{Q}{V_{ab}} = \frac{\varepsilon_0A}{d}\qquad{\qquad{\qquad}} U = \frac{CV^2}{2} = \frac{Q^2}{2C} = \frac{QV}{2} \]
\[ K = \frac{C}{C_0}  \qquad{\qquad{\qquad}}   K = \frac{V_0}{V}  \qquad{\qquad{\qquad}}    K\varepsilon_0 = \varepsilon \]
\[ u = \frac{1}{2}K\varepsilon_0E^2 = \frac{\varepsilon{E^2}}{2} \qquad{\qquad{\qquad}} \oint{K}\vec{E}.d\vec{A} = \frac{Q_{encl}}{\varepsilon_0} \]

\part{Courant continu}
%      ___
%  ___(- +)___
% |    ---    |
% |    ELEC   |
% |_____||____|
\section{Courant électrique (DC)}
\[I = \frac{dq}{dt}\qquad{\qquad{\qquad}}\rho = \frac{E}{J}\qquad{\qquad{\qquad}}q(t) = \int_{t_0}^{t}i(t) dt + q{t_0}\]

\section{Résistance et résistivité}
\[R = \frac{\rho{L}}{A} \qquad{\qquad{\qquad}}V = IR\]

\section{Force électromotrice et puissance}
\[P = VI = I^2R\] 
\[ V_{ab} = \varepsilon - IR_{interne}\qquad{\qquad{\qquad}}w = \int_{t_1}^{t_2}p(t) dt\]

\section{Lois de Kirchoff's}
\[\sum{I} = 0  \qquad{\qquad{\qquad}}\sum{V} = 0\]

\section{Capacités et inductances}

\begin{table}[H]
	\begin{center}
		\begin{tabular}{m{6.5cm}|m{6.5cm}}
			\textbf{Capacité} & \textbf{Inductance}\\
			\hline
			\[I(t) = \frac {CdV}{dt}\] & \[V(t) = \frac{LdI}{dt}\]\\
			\[\tau = RC\] & \[\tau = \frac{L}{R}\]\\
			\[U = \frac{CV^2}{2} = \frac{Q^2}{2C}\] & \[U = \frac{LI^2}{2}\]
		\end{tabular}
	\end{center}
\end{table}

\section{Circuit R-L-C}
\[ \frac{Q^2}{2C} + \frac{LI^2}{2} = \frac{Q^{2}_{max}}{2C} \]
\[ I = \pm \sqrt{\frac{Q^{2}_{max} - q^2}{LC}} \]
\[ w = \sqrt{\frac{1}{LC}}\qquad{\qquad{\qquad}}w' = \sqrt{\frac{1}{LC} - \frac{R^2}{4C^{2}}} \]
\part{Courant alternatif}
\section{Courant électrique (AC)}
\[i = I\cos{\omega{t}}\qquad{\qquad{\qquad}}v = V\cos{(\omega{t} + \phi)}\]
\[X(t) = A + Be^{-\frac{t}{\tau}}\]
\section{Valeurs efficaces}
\[I_{eff} = \frac{I}{\sqrt{2}}\qquad{\qquad{\qquad}}V_{eff} = \frac{V}{\sqrt{2}}\qquad{\qquad{\qquad}}P_{eff} = V_{eff}I_{eff}\cos{\theta}\]
\section{Capacités et inductances}
\begin{table}[H]
	\begin{center}
		\begin{tabular}{m{6.5cm}|m{6.5cm}}
			\textbf{Capacité} & \textbf{Inductance}\\
			\hline
			\[V_C = \frac{I_{0}\sin{\omega{t}}}{\omega{C}}\]&\[V_{L} = -L\omega{}I_{0}\sin{\omega{t}}\]\\
			\[X_C = \frac{1}{\omega{C}}\]&\[X_L = L\omega\]
		\end{tabular}
	\end{center}
\end{table}

\section{Circuit R-L-C}
\[Z = \sqrt{R^2 + (X_L - X_C)^2} = \sqrt{R^2 + \left(\omega{L} - \frac{1}{\omega{C}}\right)^2}\]
\[\tan{\phi} = \cfrac{\omega{L} -\cfrac{1}{\omega{C}}}{R}\]

\section{Transformateur}
\begin{center}
\begin{circuitikz}[american voltages] \draw
	(0,0) node[transformer] (T) {}
	(T.B1) -| (3.5, 0) to[R, v=$V_2$, i = $I_2$] (3.5,-2) |- (T.B2)
	(T.A1) -| (-3.5, 0) to[sV, v=$V_1$, i = $I_1$] (-3.5,-2) |- (T.A2)
	(-1,-1) node{$N_1$}
	(1,-1) node{$N_2$}
	;
\end{circuitikz}
\end{center}

Le transformateur respecte deux équations
\[\frac{V_1}{V_2} = \frac{N_1}{N_2}\qquad{\qquad{\qquad}}V_1I_1 = V_2I_2\]

Par la conservation du flux (voir la matière du cours LFSAB1202), on a
\[ \frac{V_1}{V_2} = \frac{N_1}{N_2} \]

Si on considère que le transformateur ne perd pas d'énergie, on a
\[ V_1I_1 = V_2I_2 \]

Attention au sens des courants, si on les défini dans l'autre sens, les signes peuvent changer donc ce sont des égalités au signe près.

\end{document}
