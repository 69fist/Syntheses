\section{Fondements}

\subsection{Ensembles}
\begin{itemize}
    \item [(a)] Faire une représentation graphique de $A \subset B$, $A \cup B$ et $A \cap B$.
    \item [(b)] Réécrire la définition d'un sous-ensemble en utilisant les quantificateurs.
    \item [(d)] Soient $A$, $B$, $C$ trois ensembles. Montrer graphiquement, et démontrer que $A \cap (B \cup C) = (A\cap B)\cup(A\cap C)$.
    \item [(h)] Soient $A = \{x\in\nat, x \leq 12 \}$ et $B = \{x\in\nat, x \leq 14 \}$. Donner $A \cup B$ et $A \cap B$.
    \item [(i)] Soit $A = \{1, 2, 3 \}$. Donner $P(A)$.
    \item [(j)] Soient $A$ et $B$ des ensembles.
        \begin{enumerate}[label=(\roman*)]
            \item Démontrer que $P(A) \cup P(B) \subset P(A \cup B)$.
            \item Donner également un exemple où on n'a pas nécessairement $P(A \cup B) = P(A) \cup P(B)$.
            \item Est-il vrai que $P(A \cap B) = P(A) \cap P(B)$ ?
        \end{enumerate}
\end{itemize}

\begin{solution}

\begin{itemize}
    \item [(a)] 
        \begin{tikzpicture}[framed]
            \draw 
            (0, 0) circle  (1.5cm)
            ++(1.5, 0) node[right] {$B$}
            (-.75, 0) circle (.5cm)
            ++(.5, 0) node[right] {$A$}
            ;
        \end{tikzpicture}
        \begin{venndiagram2sets}
            \fillA
            \fillB
        \end{venndiagram2sets}
        \begin{venndiagram2sets}
            \fillACapB
        \end{venndiagram2sets}
    \item [(b)] 
        $A \subset B \Leftrightarrow \forall a \in A : a \in B$ 
    \item [(d)]
        $A \cap (B \cup C) = \{ x | \text{$x\in A$, et $x\in B$ ou $x \in C$} \} = \{ x | \text{$x\in A$ et $x\in B$, ou $x\in A$ et $x\in C$} \}$
        \begin{center}
            \begin{venndiagram3sets}
                \fillACapB
                \fillACapC
            \end{venndiagram3sets}
        \end{center}

    \item [(h)] 
        $A \cup B = B$, et $A \cap B = A$.
    \item [(i)] $P(A) = \{ \emptyset, \{a\}, \{b\}, \{c\}, \{a, b\}, \{b, c\}, \{a, c\}, \{a, b, c\} \}$.
    \item [(j)] 
        \begin{enumerate}[label=(\roman*)]
            \item $P(A) \cup P(B)$ contient des ensembles exclusivement composés d'éléments de $A$ ou de $B$. $P(A \cup B)$ contient \emph{tous} les sous-ensembles qu'on peut former avec les éléments de $A$ et $B$ ; le second contient forcément le premier.
            \item $A = \{ 1 \}$ et $B = \{ 2 \}$. De manière générale, lorsque $A \Delta B$ n'est pas vide.
            \item Non, prendre l'exemple précédent comme contre-exemple.
        \end{enumerate}
\end{itemize}

\end{solution}

\subsection{Fonctions}
Soient $f : A \rightarrow B$ et $g : B \rightarrow C$ deux applications. Démontrer l'implication suivante : si $ g \circ f$ est une application injective, alors $f$ est aussi injective. 

\begin{solution}
    On démontre la contraposée. Supposons que $f$ soit non injective, c'est-à-dire qu'il existe $x_1, x_2$ tels que $f(x_1) = f(x_2)$ et $x_1 \neq x_2$. Par conséquent, $g \circ f (x_1) = g \circ f(x_2)$, or $x_1 \neq x_2$ ; $g \circ f$ n'est donc pas injective.
\end{solution}

\subsection{Quantificateurs}
La définition de la continuité d'une fonction en un point s'énonce ainsi : \emph{On dit qu'une fonction $f$ est continue en un point $x$ appartenant à $I$, dans $I$, si $f$ a une limite en $x$ dans $I$, c'est-à-dire s'il existe un réel $a$ tel que la condition suivante soit vérifiée :\\
pour tout réel $\epsilon$ positif, il existe un réel $\alpha$ positif tel que l'implication suivante soir vraie : si $y$ est dans $I$ et $|y-x| < \alpha$ alors $|f(y) - a| < \epsilon$.}\\
    Réécrire cette définition en utilisant les quantificateurs.

\begin{solution}
    $f$ est continue en $x$ $\Leftrightarrow$
    \[ \exists a, \forall \epsilon > 0, \exists \alpha > 0, \forall y \in I : |y-x| < \alpha \Rightarrow |f(y) - a| < \epsilon \]
\end{solution}


\subsection{Compositeur et preneur d'image}
Soient les ensembles $A \eqdef \{ 1, 2, 3, 4, 5 \}$ et $B \eqdef \{ a, b, c, d \}$. Considérons les fonctions $f:A\rightarrow B$ et $g:B\rightarrow A$ définies par
\[ f \eqdef \begin{pmatrix}
        1 & 2 & 3 & 4 & 5 \\
        b & a & c & b & a
    \end{pmatrix}
    \qquad g \eqdef \begin{pmatrix}
        a & b & c & d \\
        3 & 4 & 5 & 1
    \end{pmatrix}
\]

\begin{enumerate}[label=(\roman*)]
    \item Calculer les fonctions composées $f \circ g$ et $g \circ f$.
    \item Déterminer l'ensemble image $f(\{2, 4, 5\})$ et les ensembles images réciproques $f^{-1}(\{a\})$ et $g^{-1}(\{1, 2, 4\})$.
\end{enumerate}

\begin{solution}
    \begin{enumerate}[label=(\roman*)]
        \item $f \circ g  = 
            \begin{pmatrix}
                a & b & c & d \\
                c & b & a & b
            \end{pmatrix}$, 
            $g \circ f =
            \begin{pmatrix}
                1 & 2 & 3 & 4 & 5 \\
                4 & 3 & 5 & 4 & 3
            \end{pmatrix}$
        \item $f(\{2,4,5\}) = \{a,b\}$, 
            $f^{-1}(\{a\}) = \{2,5\}$, 
            $g^{-1}(\{1, 2, 4\}) = \{b, d\}$
    \end{enumerate}
\end{solution}

\subsection{Coïncidence inévitable lors d'une réunion amicale}
Montrer que dans une réunion où chaque personne a été invitée par un ami il y a au moins deux personnes qui ont le même nombre d'amis présents. On aura compris que tout qui a invité quelqu'un participe lui-même à la réunion.

\begin{solution}
    Soit la fonction $f : A \rightarrow B$ qui donne le nombre d'amis d'une personne. Il s'agit de démontrer que $f$ n'est pas injective (car alors il existe $p_1 \neq p_2$ tels que $f(p_1) = f(p_2)$). Soit $n$ le nombre de participants. On a $|A| = n$ ; toute personne a un certain nombre d'amis. On sait aussi que pour tout $b \in B$, $1 \leq b \leq n-1$ : en effet, toute personne a au moins un ami (celui qui l'a invité) et au plus $n-1$ amis (on n'est pas l'ami de soi-même). Donc $|B| = n-1 < |A|$. En vertu du principe des tiroirs, la fonction est injective.
\end{solution}

\subsection{Omniprésence du principe d'induction}
En raisonnant par induction, démontrer que pour tout naturel $n$, le naturel $2^{2n}-1$ est divisible par $3$.

\begin{solution}
    C'est vrai pour $n = 0$ ($0$ est divisible par $3$). Supposons que ce soit vrai pour tout $k \geq 0$, et développons le cas $k+1$ :
    \[ 2^{2(k+1)} - 1 = 4 \cdot 2^{2k} - 1 = 3\cdot 2^{2k} + (2^{2k} - 1) \]
    Le premier terme est évidemment divisible par $3$, et le second l'est par hypothèse. Donc la proposition est vraie pour tout $n \geq 0$.
\end{solution}

