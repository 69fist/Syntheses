\documentclass[fr]{../../../eplexercises}

\usepackage{siunitx}
\usepackage{enumitem}
\usepackage{venndiagram}
\DeclareMathOperator{\Ker}{Ker}
\newcommand{\nat}{\mathbb{N}}
\usetikzlibrary{backgrounds}

%\begin{document}

\hypertitle{Mathématique}{1}{FSAB}{1101}
{Mattéo Couplet}
{Kouider Ben-Naoum, Olivier Pereira et Vincent Wertz}

\section{Équations différentielles}

\subsection{Mise en œuvre 2}
Résoudre l'équation différentielle suivante
\[ \fdif{y}{x} = \frac{y}{2x} \]

\begin{solution}
    C'est une équation différentielle à variables séparables, on peut donc la réécrire comme suit
    \[ \frac{\dif y}{y} = \frac{\dif x}{2x} \]
    On intègre membre à membre
    \begin{align*}
        \int \frac{1}{y} \dif y &= \int \frac{1}{2x} \dif x \\
        \ln y &= \frac{1}{2} \ln x + C' \\
        y(x) &= C \sqrt{x}
    \end{align*}
\end{solution}

\subsection{Mise en œuvre 6}
Résoudre l'équation différentielle suivante
\[ \fdif{y}{x} + \frac{2y}{x} = \frac{1}{x^2} \]

\begin{solution}
    C'est une équation différentielle linéaire non homogène ; on procède par la méthode du facteur d'intégration
    \[ \mu(x) = \int \frac{2}{x} \dif x = \ln x^2 \]
    On multiplie membre à membre par $e^{\mu(x)} = x^2$ :
    \begin{align*}
        x^2 \fdif{y}{x} + 2xy &= 1 \\
        \left( x^2 y \right) &= 1 \\
        x^2 y &= x + C \\
        y(x) &= \frac{1}{x} + \frac{C}{x^2}
    \end{align*}
\end{solution}

\subsection{Mise en œuvre 9}
Soit $y_h$ une solution de l'équation homogène $y' + a(x)y = 0$ et $y_p$ une solution particulière de l'équation non homogène $y' + a(x)y = b(x)$. Montrer que $y = cy_h + y_p$ est aussi solution de l'équation non homogène pour toute valeur réelle de $c$.

\begin{solution}
    \paragraph{Méthode classique}
    On remplace la solution dans l'équation non homogène
    \begin{align*}
    (c y_h + y_p)' + a(x) (c y_h + y_p) &= b(x) \\
    c (y_h' + a(x) y_h) + (y_p' + a(x) y_p) &= b(x)
    \end{align*}
    Ce qui est vrai en vertu de l'énoncé.
    \paragraph{Méthode de l'algèbre linéaire}
    Soit $C^k(I)$ l'espace vectoriel réel formé des fonctions continues et dérivables $k$ fois sur $I$. On s'intéresse à l'application 
    \[ L: C^1(I) \rightarrow C(I) : y \rightarrow L(y) \eqdef y' + ay \]
    On vérifie aisément que cette application est linéaire, c'est-à-dire que $L(\alpha_1 y_1 + \alpha_2 y_2) = \alpha_1 L(y_1) + \alpha_2 L(y_2)$ pour tout $y1, y2 \in C^1(I)$. Traduisons les données de l'énoncé dans le langage de l'algèbre linéaire. On a $L(y_h) = L(c y_h) = 0$, donc $\{c y_h\} = \Ker L$, et $y_p$ est une solution de l'équation $L(y) = b$. Or, une propriété des applications linéaires nous dit que l'ensemble des solutions de l'équation $L(x) = b$ est égal à la somme d'une solution particulère et du noyau de $L$, ce qui clot la démonstration.
\end{solution}

\subsection{Réflexion 2}
Soit l'équation
\[ y' + ay = 0 \]
où $a$ est un nombre réel quelconque. On sait que $y = Ce^{-ax}$ est la solution générale. Montrer dans ce cas simple, que toutes les solutions s'écrivent uniquement sous cette forme-là.

\begin{solution}
    On démontre par contradiction. Supposons que la solution s'écrive autrement, c'est-à-dire que $C$ est fonction de $x$. On a donc une nouvelle solution
    \[ y = C(x) e^{-ax} \]
    Si on l'injecte dans l'équation, on trouve
    \begin{align*}
    C'(x) e^{-ax} - a C(x) e^{-ax} + a C(x) e^{-ax} &= 0 \\
    C'(x) &= 0
    \end{align*}
    $C$ est donc bien constant, ce qui contredit l'hypothèse. 
\end{solution}

\subsection{Réflexion 3}
\begin{enumerate}
    \item Soit $P(x)$ et $Q(x)$ deux fonctions continues sur l'intervalle $[a, b]$. Utilisez le théorème fondamental pour montrer que toute fonction $y$ qui satisfait l'équation
        \[ v(x) y = \int v(x) Q(x) \dif x + C \]
        avec $v(x) = e^{\int P(x) \dif x}$ est une solution de l'équation différentielle linéaire du premier ordre
        \[ \fdif{y}{x} + P(x) y = Q(x) \]
    \item Si $C = y_0v(x_0) - \int_{x_0}^x v(t) Q(t) \dif t$, montrer que toute solution $y$ du point ci-dessus satisfait la condition initiale $y(x_0) = y_0$.
\end{enumerate}

\begin{solution}
    \begin{enumerate}
        \item Par le théorème fondamental, on déduit que $v(x) y$ est une primitive de $v(x) Q(x)$, ou encore
            \begin{align*}
            \left( v(x) y \right) ' &= v(x) Q(x) \\
            P(x) v(x) y + v(x) y' &= v(x) Q(x) \\
            y' + P(x) y &= Q(x) \qquad \text{une exponentielle est $>0$}
            \end{align*}
            $y$ est donc bien solution de l'équation.
        \item 
            %TODO compléter la solution, j'ai pas trouvé :'(
    \end{enumerate}
\end{solution}

\subsection{Modélisation 1}
Find the amount in a savings account after one year if the initial balance in the account was \SI{1000}{\$}, if the interest is paid continuously into the account at a nominal rate of \SI{10}{\percent} per annum, compounded continuously, and if the account is being continuously depleted (by taxes, say) at a rate of $y^2/\num{1000000}$ dollars per year, where $y=y(t)$ is the balance in the account after $t$ years. How large can the account grow? How long will it take the account to grow to half this balance?

\begin{solution}
    Si l'unité de $t$ est l'année, on traduit les données de l'énoncé dans l'équation différentielle suivante 
    \[ \fdif{y}{t} = \num{e-1}y - \num{e-6} y^2 \]
    La condition initiale est $y(0) = \num{e3}$. C'est une équation différentielle à variables séparables, qu'on peut donc réécrire comme suit
    \[ \frac{\dif y}{\num{e-1} y - \num{e-6} y^2} = \dif t \]
    On intègre membre à membre
    \[ \int \frac{1}{\num{e-1} y - \num{e-6} y^2} \dif y = \int 1 \dif t \]
    L'intégrale de gauche se résout en somme des fractions simples ; on ne détaillera pas
    \begin{align*}
        10 \ln \left( \frac{y}{\num{e5} - y} \right) &= t + C' \\
        \frac{y}{\num{e5} - y} &= C''e^{t/10} \\
        y(t) &= \frac{\num{e5}}{Ce^{-t/10} + 1}
    \end{align*}
    On injecte la condition initiale afin de trouver $C$
    \begin{align*}
        y(0) &= \num{e3} \\
        \frac{\num{e5}}{C+1} &= \num{e3} \\
        C &= 99
    \end{align*}
        
    L'expression finale du montant en fonction du temps est donc
    \[ y(t) = \frac{\num{e5}}{99e^{-t/10} + 1} \]

    Voyons à quoi ressemble le compte bancaire après une longue période de temps 
    \[ \lim_{t \to \infty} y(t) = \SI{100000}{\$} \]
    On nous demande enfin en combien de temps atteint-on la moitié de ce montant ; il s'agit donc de résoudre
    \begin{align*}
        y(T) &= \num{5e4} \\
        T &\approx \SI{46}{\text{ans}}
    \end{align*}



\end{solution}



\section{Fondements}

\subsection{Ensembles}
\begin{itemize}
    \item [(a)] Faire une représentation graphique de $A \subset B$, $A \cup B$ et $A \cap B$.
    \item [(b)] Réécrire la définition d'un sous-ensemble en utilisant les quantificateurs.
    \item [(d)] Soient $A$, $B$, $C$ trois ensembles. Montrer graphiquement, et démontrer que $A \cap (B \cup C) = (A\cap B)\cup(A\cap C)$.
    \item [(h)] Soient $A = \{x\in\nat, x \leq 12 \}$ et $B = \{x\in\nat, x \leq 14 \}$. Donner $A \cup B$ et $A \cap B$.
    \item [(i)] Soit $A = \{1, 2, 3 \}$. Donner $P(A)$.
    \item [(j)] Soient $A$ et $B$ des ensembles.
        \begin{enumerate}[label=(\roman*)]
            \item Démontrer que $P(A) \cup P(B) \subset P(A \cup B)$.
            \item Donner également un exemple où on n'a pas nécessairement $P(A \cup B) = P(A) \cup P(B)$.
            \item Est-il vrai que $P(A \cap B) = P(A) \cap P(B)$ ?
        \end{enumerate}
\end{itemize}

\begin{solution}

\begin{itemize}
    \item [(a)] 
        \begin{tikzpicture}[framed]
            \draw 
            (0, 0) circle  (1.5cm)
            ++(1.5, 0) node[right] {$B$}
            (-.75, 0) circle (.5cm)
            ++(.5, 0) node[right] {$A$}
            ;
        \end{tikzpicture}
        \begin{venndiagram2sets}
            \fillA
            \fillB
        \end{venndiagram2sets}
        \begin{venndiagram2sets}
            \fillACapB
        \end{venndiagram2sets}
    \item [(b)] 
        $A \subset B \Leftrightarrow \forall a \in A : a \in B$ 
    \item [(d)]
        $A \cap (B \cup C) = \{ x | \text{$x\in A$, et $x\in B$ ou $x \in C$} \} = \{ x | \text{$x\in A$ et $x\in B$, ou $x\in A$ et $x\in C$} \}$
        \begin{center}
            \begin{venndiagram3sets}
                \fillACapB
                \fillACapC
            \end{venndiagram3sets}
        \end{center}

    \item [(h)] 
        $A \cup B = B$, et $A \cap B = A$.
    \item [(i)] $P(A) = \{ \emptyset, \{a\}, \{b\}, \{c\}, \{a, b\}, \{b, c\}, \{a, c\}, \{a, b, c\} \}$.
    \item [(j)] 
        \begin{enumerate}[label=(\roman*)]
            \item $P(A) \cup P(B)$ contient des ensembles exclusivement composés d'éléments de $A$ ou de $B$. $P(A \cup B)$ contient \emph{tous} les sous-ensembles qu'on peut former avec les éléments de $A$ et $B$ ; le second contient forcément le premier.
            \item $A = \{ 1 \}$ et $B = \{ 2 \}$. De manière générale, lorsque $A \Delta B$ n'est pas vide.
            \item Non, prendre l'exemple précédent comme contre-exemple.
        \end{enumerate}
\end{itemize}

\end{solution}

\subsection{Fonctions}
Soient $f : A \rightarrow B$ et $g : B \rightarrow C$ deux applications. Démontrer l'implication suivante : si $ g \circ f$ est une application injective, alors $f$ est aussi injective. 

\begin{solution}
    On démontre la contraposée. Supposons que $f$ soit non injective, c'est-à-dire qu'il existe $x_1, x_2$ tels que $f(x_1) = f(x_2)$ et $x_1 \neq x_2$. Par conséquent, $g \circ f (x_1) = g \circ f(x_2)$, or $x_1 \neq x_2$ ; $g \circ f$ n'est donc pas injective.
\end{solution}

\subsection{Quantificateurs}
La définition de la continuité d'une fonction en un point s'énonce ainsi : \emph{On dit qu'une fonction $f$ est continue en un point $x$ appartenant à $I$, dans $I$, si $f$ a une limite en $x$ dans $I$, c'est-à-dire s'il existe un réel $a$ tel que la condition suivante soit vérifiée :\\
pour tout réel $\epsilon$ positif, il existe un réel $\alpha$ positif tel que l'implication suivante soir vraie : si $y$ est dans $I$ et $|y-x| < \alpha$ alors $|f(y) - a| < \epsilon$.}\\
    Réécrire cette définition en utilisant les quantificateurs.

\begin{solution}
    $f$ est continue en $x$ $\Leftrightarrow$
    \[ \exists a, \forall \epsilon > 0, \exists \alpha > 0, \forall y \in I : |y-x| < \alpha \Rightarrow |f(y) - a| < \epsilon \]
\end{solution}


\subsection{Compositeur et preneur d'image}
Soient les ensembles $A \eqdef \{ 1, 2, 3, 4, 5 \}$ et $B \eqdef \{ a, b, c, d \}$. Considérons les fonctions $f:A\rightarrow B$ et $g:B\rightarrow A$ définies par
\[ f \eqdef \begin{pmatrix}
        1 & 2 & 3 & 4 & 5 \\
        b & a & c & b & a
    \end{pmatrix}
    \qquad g \eqdef \begin{pmatrix}
        a & b & c & d \\
        3 & 4 & 5 & 1
    \end{pmatrix}
\]

\begin{enumerate}[label=(\roman*)]
    \item Calculer les fonctions composées $f \circ g$ et $g \circ f$.
    \item Déterminer l'ensemble image $f(\{2, 4, 5\})$ et les ensembles images réciproques $f^{-1}(\{a\})$ et $g^{-1}(\{1, 2, 4\})$.
\end{enumerate}

\begin{solution}
    \begin{enumerate}[label=(\roman*)]
        \item $f \circ g  = 
            \begin{pmatrix}
                a & b & c & d \\
                c & b & a & b
            \end{pmatrix}$, 
            $g \circ f =
            \begin{pmatrix}
                1 & 2 & 3 & 4 & 5 \\
                4 & 3 & 5 & 4 & 3
            \end{pmatrix}$
        \item $f(\{2,4,5\}) = \{a,b\}$, 
            $f^{-1}(\{a\}) = \{2,5\}$, 
            $g^{-1}(\{1, 2, 4\}) = \{b, d\}$
    \end{enumerate}
\end{solution}

\subsection{Coïncidence inévitable lors d'une réunion amicale}
Montrer que dans une réunion où chaque personne a été invitée par un ami il y a au moins deux personnes qui ont le même nombre d'amis présents. On aura compris que tout qui a invité quelqu'un participe lui-même à la réunion.

\begin{solution}
    Soit la fonction $f : A \rightarrow B$ qui donne le nombre d'amis d'une personne. Il s'agit de démontrer que $f$ n'est pas injective (car alors il existe $p_1 \neq p_2$ tels que $f(p_1) = f(p_2)$). Soit $n$ le nombre de participants. On a $|A| = n$ ; toute personne a un certain nombre d'amis. On sait aussi que pour tout $b \in B$, $1 \leq b \leq n-1$ : en effet, toute personne a au moins un ami (celui qui l'a invité) et au plus $n-1$ amis (on n'est pas l'ami de soi-même). Donc $|B| = n-1 < |A|$. En vertu du principe des tiroirs, la fonction est injective.
\end{solution}

\subsection{Omniprésence du principe d'induction}
En raisonnant par induction, démontrer que pour tout naturel $n$, le naturel $2^{2n}-1$ est divisible par $3$.

\begin{solution}
    C'est vrai pour $n = 0$ ($0$ est divisible par $3$). Supposons que ce soit vrai pour tout $k \geq 0$, et développons le cas $k+1$ :
    \[ 2^{2(k+1)} - 1 = 4 \cdot 2^{2k} - 1 = 3\cdot 2^{2k} + (2^{2k} - 1) \]
    Le premier terme est évidemment divisible par $3$, et le second l'est par hypothèse. Donc la proposition est vraie pour tout $n \geq 0$.
\end{solution}

\end{document}
