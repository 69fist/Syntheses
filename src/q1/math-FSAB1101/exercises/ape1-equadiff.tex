\section{Équations différentielles}

\subsection{Mise en œuvre 2}
Résoudre l'équation différentielle suivante
\[ \fdif{y}{x} = \frac{y}{2x} \]

\begin{solution}
    C'est une équation différentielle à variables séparables, on peut donc la réécrire comme suit
    \[ \frac{\dif y}{y} = \frac{\dif x}{2x} \]
    On intègre membre à membre
    \begin{align*}
        \int \frac{1}{y} \dif y &= \int \frac{1}{2x} \dif x \\
        \ln y &= \frac{1}{2} \ln x + C' \\
        y(x) &= C \sqrt{x}
    \end{align*}
\end{solution}

\subsection{Mise en œuvre 6}
Résoudre l'équation différentielle suivante
\[ \fdif{y}{x} + \frac{2y}{x} = \frac{1}{x^2} \]

\begin{solution}
    C'est une équation différentielle linéaire non homogène ; on procède par la méthode du facteur d'intégration
    \[ \mu(x) = \int \frac{2}{x} \dif x = \ln x^2 \]
    On multiplie membre à membre par $e^{\mu(x)} = x^2$ :
    \begin{align*}
        x^2 \fdif{y}{x} + 2xy &= 1 \\
        \left( x^2 y \right) &= 1 \\
        x^2 y &= x + C \\
        y(x) &= \frac{1}{x} + \frac{C}{x^2}
    \end{align*}
\end{solution}

\subsection{Mise en œuvre 9}
Soit $y_h$ une solution de l'équation homogène $y' + a(x)y = 0$ et $y_p$ une solution particulière de l'équation non homogène $y' + a(x)y = b(x)$. Montrer que $y = cy_h + y_p$ est aussi solution de l'équation non homogène pour toute valeur réelle de $c$.

\begin{solution}
    \paragraph{Méthode classique}
    On remplace la solution dans l'équation non homogène
    \begin{align*}
    (c y_h + y_p)' + a(x) (c y_h + y_p) &= b(x) \\
    c (y_h' + a(x) y_h) + (y_p' + a(x) y_p) &= b(x)
    \end{align*}
    Ce qui est vrai en vertu de l'énoncé.
    \paragraph{Méthode de l'algèbre linéaire}
    Soit $C^k(I)$ l'espace vectoriel réel formé des fonctions continues et dérivables $k$ fois sur $I$. On s'intéresse à l'application 
    \[ L: C^1(I) \rightarrow C(I) : y \rightarrow L(y) \eqdef y' + ay \]
    On vérifie aisément que cette application est linéaire, c'est-à-dire que $L(\alpha_1 y_1 + \alpha_2 y_2) = \alpha_1 L(y_1) + \alpha_2 L(y_2)$ pour tout $y1, y2 \in C^1(I)$. Traduisons les données de l'énoncé dans le langage de l'algèbre linéaire. On a $L(y_h) = L(c y_h) = 0$, donc $\{c y_h\} = \Ker L$, et $y_p$ est une solution de l'équation $L(y) = b$. Or, une propriété des applications linéaires nous dit que l'ensemble des solutions de l'équation $L(x) = b$ est égal à la somme d'une solution particulère et du noyau de $L$, ce qui clot la démonstration.
\end{solution}

\subsection{Réflexion 2}
Soit l'équation
\[ y' + ay = 0 \]
où $a$ est un nombre réel quelconque. On sait que $y = Ce^{-ax}$ est la solution générale. Montrer dans ce cas simple, que toutes les solutions s'écrivent uniquement sous cette forme-là.

\begin{solution}
    On démontre par contradiction. Supposons que la solution s'écrive autrement, c'est-à-dire que $C$ est fonction de $x$. On a donc une nouvelle solution
    \[ y = C(x) e^{-ax} \]
    Si on l'injecte dans l'équation, on trouve
    \begin{align*}
    C'(x) e^{-ax} - a C(x) e^{-ax} + a C(x) e^{-ax} &= 0 \\
    C'(x) &= 0
    \end{align*}
    $C$ est donc bien constant, ce qui contredit l'hypothèse. 
\end{solution}

\subsection{Réflexion 3}
\begin{enumerate}
    \item Soit $P(x)$ et $Q(x)$ deux fonctions continues sur l'intervalle $[a, b]$. Utilisez le théorème fondamental pour montrer que toute fonction $y$ qui satisfait l'équation
        \[ v(x) y = \int v(x) Q(x) \dif x + C \]
        avec $v(x) = e^{\int P(x) \dif x}$ est une solution de l'équation différentielle linéaire du premier ordre
        \[ \fdif{y}{x} + P(x) y = Q(x) \]
    \item Si $C = y_0v(x_0) - \int_{x_0}^x v(t) Q(t) \dif t$, montrer que toute solution $y$ du point ci-dessus satisfait la condition initiale $y(x_0) = y_0$.
\end{enumerate}

\begin{solution}
    \begin{enumerate}
        \item Par le théorème fondamental, on déduit que $v(x) y$ est une primitive de $v(x) Q(x)$, ou encore
            \begin{align*}
            \left( v(x) y \right) ' &= v(x) Q(x) \\
            P(x) v(x) y + v(x) y' &= v(x) Q(x) \\
            y' + P(x) y &= Q(x) \qquad \text{une exponentielle est $>0$}
            \end{align*}
            $y$ est donc bien solution de l'équation.
        \item 
            %TODO compléter la solution, j'ai pas trouvé :'(
    \end{enumerate}
\end{solution}

\subsection{Modélisation 1}
Find the amount in a savings account after one year if the initial balance in the account was \SI{1000}{\$}, if the interest is paid continuously into the account at a nominal rate of \SI{10}{\percent} per annum, compounded continuously, and if the account is being continuously depleted (by taxes, say) at a rate of $y^2/\num{1000000}$ dollars per year, where $y=y(t)$ is the balance in the account after $t$ years. How large can the account grow? How long will it take the account to grow to half this balance?

\begin{solution}
    Si l'unité de $t$ est l'année, on traduit les données de l'énoncé dans l'équation différentielle suivante 
    \[ \fdif{y}{t} = \num{e-1}y - \num{e-6} y^2 \]
    La condition initiale est $y(0) = \num{e3}$. C'est une équation différentielle à variables séparables, qu'on peut donc réécrire comme suit
    \[ \frac{\dif y}{\num{e-1} y - \num{e-6} y^2} = \dif t \]
    On intègre membre à membre
    \[ \int \frac{1}{\num{e-1} y - \num{e-6} y^2} \dif y = \int 1 \dif t \]
    L'intégrale de gauche se résout en somme des fractions simples ; on ne détaillera pas
    \begin{align*}
        10 \ln \left( \frac{y}{\num{e5} - y} \right) &= t + C' \\
        \frac{y}{\num{e5} - y} &= C''e^{t/10} \\
        y(t) &= \frac{\num{e5}}{Ce^{-t/10} + 1}
    \end{align*}
    On injecte la condition initiale afin de trouver $C$
    \begin{align*}
        y(0) &= \num{e3} \\
        \frac{\num{e5}}{C+1} &= \num{e3} \\
        C &= 99
    \end{align*}
        
    L'expression finale du montant en fonction du temps est donc
    \[ y(t) = \frac{\num{e5}}{99e^{-t/10} + 1} \]

    Voyons à quoi ressemble le compte bancaire après une longue période de temps 
    \[ \lim_{t \to \infty} y(t) = \SI{100000}{\$} \]
    On nous demande enfin en combien de temps atteint-on la moitié de ce montant ; il s'agit donc de résoudre
    \begin{align*}
        y(T) &= \num{5e4} \\
        T &\approx \SI{46}{\text{ans}}
    \end{align*}



\end{solution}



