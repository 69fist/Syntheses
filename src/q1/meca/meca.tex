\documentclass[11pt,a4paper]{article}

% French
\usepackage[utf8x]{inputenc}
\usepackage[frenchb]{babel}
\usepackage[T1]{fontenc}
\usepackage{lmodern}
\usepackage{ifthen}

% Color
% cfr http://en.wikibooks.org/wiki/LaTeX/Colors
\usepackage{color}
\usepackage[usenames,dvipsnames,svgnames,table]{xcolor}
\definecolor{dkgreen}{rgb}{0.25,0.7,0.35}
\definecolor{dkred}{rgb}{0.7,0,0}

% Floats and referencing
\newcommand{\sectionref}[1]{section~\ref{sec:#1}}
\newcommand{\annexeref}[1]{annexe~\ref{ann:#1}}
\newcommand{\figuref}[1]{figure~\ref{fig:#1}}
\newcommand{\tabref}[1]{table~\ref{tab:#1}}
\usepackage{xparse}
\NewDocumentEnvironment{myfig}{mm}
{\begin{figure}[!ht]\centering}
{\caption{#2}\label{fig:#1}\end{figure}}

% Listing
\usepackage{listings}
\lstset{
  numbers=left,
  numberstyle=\tiny\color{gray},
  basicstyle=\rm\small\ttfamily,
  keywordstyle=\bfseries\color{dkred},
  frame=single,
  commentstyle=\color{gray}=small,
  stringstyle=\color{dkgreen},
  %backgroundcolor=\color{gray!10},
  %tabsize=2,
  rulecolor=\color{black!30},
  %title=\lstname,
  breaklines=true,
  framextopmargin=2pt,
  framexbottommargin=2pt,
  extendedchars=true,
  inputencoding=utf8x
}

\newcommand{\matlab}{\textsc{Matlab}}
\newcommand{\octave}{\textsc{GNU/Octave}}
\newcommand{\qtoctave}{\textsc{QtOctave}}
\newcommand{\oz}{\textsc{Oz}}
\newcommand{\java}{\textsc{Java}}
\newcommand{\clang}{\textsc{C}}
\newcommand{\keyword}{mot clef}

% Math symbols
\usepackage{amsmath}
\usepackage{amssymb}
\usepackage{amsthm}
\DeclareMathOperator*{\argmin}{arg\,min}
\DeclareMathOperator*{\argmax}{arg\,max}

% Sets
\newcommand{\Z}{\mathbb{Z}}
\newcommand{\R}{\mathbb{R}}
\newcommand{\Rn}{\R^n}
\newcommand{\Rnn}{\R^{n \times n}}
\newcommand{\C}{\mathbb{C}}
\newcommand{\K}{\mathbb{K}}
\newcommand{\Kn}{\K^n}
\newcommand{\Knn}{\K^{n \times n}}

% Chemistry
\newcommand{\std}{\ensuremath{^{\circ}}}
\newcommand\ph{\ensuremath{\mathrm{pH}}}

% Theorem and definitions
\theoremstyle{definition}
\newtheorem{mydef}{Définition}
\newtheorem{mynota}[mydef]{Notation}
\newtheorem{myprop}[mydef]{Propriétés}
\newtheorem{myrem}[mydef]{Remarque}
\newtheorem{myform}[mydef]{Formules}
\newtheorem{mycorr}[mydef]{Corrolaire}
\newtheorem{mytheo}[mydef]{Théorème}
\newtheorem{mylem}[mydef]{Lemme}
\newtheorem{myexem}[mydef]{Exemple}
\newtheorem{myineg}[mydef]{Inégalité}

% Unit vectors
\usepackage{esint}
\usepackage{esvect}
\newcommand{\kmath}{k}
\newcommand{\xunit}{\hat{\imath}}
\newcommand{\yunit}{\hat{\jmath}}
\newcommand{\zunit}{\hat{\kmath}}

% rot & div & grad & lap
\DeclareMathOperator{\newdiv}{div}
\newcommand{\divn}[1]{\nabla \cdot #1}
\newcommand{\rotn}[1]{\nabla \times #1}
\newcommand{\grad}[1]{\nabla #1}
\newcommand{\gradn}[1]{\nabla #1}
\newcommand{\lap}[1]{\nabla^2 #1}


% Elec
\newcommand{\B}{\vec B}
\newcommand{\E}{\vec E}
\newcommand{\EMF}{\mathcal{E}}
\newcommand{\perm}{\varepsilon} % permittivity

\newcommand{\bigoh}{\mathcal{O}}
\newcommand\eqdef{\triangleq}

\DeclareMathOperator{\newdiff}{d} % use \dif instead
\newcommand{\dif}{\newdiff\!}
\newcommand{\fpart}[2]{\frac{\partial #1}{\partial #2}}
\newcommand{\ffpart}[2]{\frac{\partial^2 #1}{\partial #2^2}}
\newcommand{\fdpart}[3]{\frac{\partial^2 #1}{\partial #2\partial #3}}
\newcommand{\fdif}[2]{\frac{\dif #1}{\dif #2}}
\newcommand{\ffdif}[2]{\frac{\dif^2 #1}{\dif #2^2}}
\newcommand{\constant}{\ensuremath{\mathrm{cst}}}

% Numbers and units
\usepackage[squaren, Gray]{SIunits}
\usepackage{sistyle}
\usepackage[autolanguage]{numprint}
%\usepackage{numprint}
\newcommand\si[2]{\numprint[#2]{#1}}
\newcommand\np[1]{\numprint{#1}}

\newcommand\strong[1]{\textbf{#1}}
\newcommand{\annexe}{\part{Annexes}\appendix}

% Bibliography
\newcommand{\biblio}{\bibliographystyle{plain}\bibliography{biblio}}

\usepackage{fullpage}
% le `[e ]' rend le premier argument (#1) optionnel
% avec comme valeur par défaut `e `
\newcommand{\hypertitle}[7][e ]{
\usepackage{hyperref}
{\renewcommand{\and}{\unskip, }
\hypersetup{pdfauthor={#6},
            pdftitle={Synth\`ese d#1#2 Q#3 - L#4#5},
            pdfsubject={#2}}
}

\title{Synth\`ese d#1#2 Q#3 - L#4#5}
\author{#6}

\begin{document}

\ifthenelse{\isundefined{\skiptitlepage}}{
\begin{titlepage}
\maketitle

 \paragraph{Informations importantes}
   Ce document est grandement inspiré de l'excellent cours
   donné par #7 à l'EPL (École Polytechnique de Louvain),
   faculté de l'UCL (Université Catholique de Louvain).
   Il est écrit par les auteurs susnommés avec l'aide de tous
   les autres étudiants, la vôtre est donc la bienvenue.
   Il y a toujours moyen de l'améliorer, surtout si le cours
   change car la synthèse doit alors être modifiée en conséquence.
   On peut retrouver le code source à l'adresse suivante
   \begin{center}
     \url{https://github.com/Gp2mv3/Syntheses}.
   \end{center}
   On y trouve aussi le contenu du \texttt{README} qui contient de plus
   amples informations, vous êtes invité à le lire.

   Il y est indiqué que les questions, signalements d'erreurs,
   suggestions d'améliorations ou quelque discussion que ce soit
   relative au projet
   sont à spécifier de préférence à l'adresse suivante
   \begin{center}
     \url{https://github.com/Gp2mv3/Syntheses/issues}.
   \end{center}
   Ça permet à tout le monde de les voir, les commenter et agir
   en conséquence.
   Vous êtes d'ailleurs invité à participer aux discussions.

   Vous trouverez aussi des informations dans le wiki
   \begin{center}
     \url{https://github.com/Gp2mv3/Syntheses/wiki}.
   \end{center}
   comme le statut des synthèses pour chaque cours
   \begin{center}
     \url{https://github.com/Gp2mv3/Syntheses/wiki/Status}.
   \end{center}
   vous pouvez d'ailleurs remarquer qu'il en manque encore beaucoup,
   votre aide est la bienvenue.

   Pour contribuer au bug tracker et au wiki, il vous suffira de
   créer un compte sur Github.
   Pour interagir avec le code des synthèses,
   il vous faudra installer \LaTeX.
   Pour interagir directement avec le code sur Github,
   vous devez utiliser \texttt{git}.
   Si cela pose problème,
   nous sommes évidemment ouverts à des contributeurs envoyant leurs
   changements par mail ou n'importe quel autre moyen.
\end{titlepage}
}{}

\ifthenelse{\isundefined{\skiptableofcontents}}{
\tableofcontents
}{}
}


\usepackage{tikz}

\newcommand{\hati}{\hat{\imath}}
\newcommand{\hatj}{\hat{\jmath}}
\newcommand{\hatk}{\hat{k}}

\hypertitle{Mécanique}{1}{FSAB}{1201}
{Nicolas Cognaux et Benoît Legat}{Nicolas Cognaux \and Benoît Legat}
%    _
%   /A\
%  |   |
%  |   C
%  E   |
%      M

\part{Vecteurs}
% +----------+
% | Vecteurs |
% +----------+
\section{Décomposition des vecteurs}
\[ A = \sqrt{{A_x}^2 + {A_y}^2} \]
\[ \theta = \arctan{\frac{A_y}{A_x}} \]

\section{Produit scalaire}
\[ \vec{A} \cdot \vec{B} = AB\cos{\theta} = |A||B|\cos{\theta} \]
\[ \vec{A} \cdot \vec{B} = A_x B_x + A_y B_y + A_z B_z \]

\section{Produit vectoriel}
\[ C = A \times B = AB\sin{\theta} \]
\begin{align*}
  \vec{C} & = \left(
  \begin{pmatrix}
    A_y&A_z \\
    B_y&B_z
  \end{pmatrix}, -
  \begin{pmatrix}
    A_b&A_z \\
    B_x&B_z
  \end{pmatrix},
  \begin{pmatrix}
    A_x&A_y \\
    B_x&B_y
  \end{pmatrix} \right)\\
  & = (A_yB_z - A_zB_y, A_zB_x - A_xB_z, A_xB_y - A_yB_x)
\end{align*}
On peut grâce à la règle de la main droite retrouver le produit des composantes
\begin{align*}
  \hati \times \hatj & = \hatk\\
  \hatj \times \hatk & = \hati\\
  \hatk \times \hati & = \hatj
\end{align*}
\[ A \times B =
  \begin{vmatrix}
    \hati&\hatj&\hatk\\
    A_x&A_y&A_z\\
    B_x&B_y&B_z
  \end{vmatrix}
\]

\part{Lois de Newton}
% +----------------+
% | Lois de Newton |
% +----------------+
\section{Loi d'inertie}
Si la somme des forces agissant sur un corps est nulle, alors il ne subit aucune
accélération et se déplacement à vitesse constante.
\[ \sum \vec{F} = 0 \]

\section{Loi du mouvement}
Soit un corps de masse $m$ : l'accélération subie par ce corps est
proportionnelle à la résultante des forces qu'il subit,
et inversement proportionnelle à sa masse m.
\[ \sum \vec{F} = m \vec{a} \]

\section{Principe d'action-réaction}
Tout corps A exerçant une force sur un corps B
subit une force d'intensité égale,
de même direction mais de sens opposé, exercée par ce corps B.
\[ \vec{F}_{A\text{ sur }B} = -\vec{F}_{B\text{ sur }A} \]

\part{Équilibre}
% +-----------+
% | Équilibre |
% +-----------+
Un corps rigide est dit en équilibre s'il ne bouge ni ne tourne sur lui-même.
\section{Conditions}
Pour qu'il soit en équilibre, il faut que deux conditions soient respectées
\begin{itemize}
  \item La somme des forces est nulle.
    \[ \sum \vec{F} = 0 \]
  \item La somme des moments de force en un point A est nulle.
    \[ \sum \vec{\tau_A} = 0 \]
    avec
    $\tau_A = \vec{AP} \times \vec{F}$
    où $P$ est le point où $F$ s'applique.

    Si la première condition est respectée,
    cette somme ne dépend pas du point $A$ choisis.
\end{itemize}

\part{Mouvements}
% +------------+
% | Mouvements |
% +------------+
\section{Mouvement rectiligne (MRU/MRUA)}
\[ v_x = \lim_{\Delta t \to 0} \frac{\Delta x}{\Delta t} = \frac{dx}{dt} \]
\[ a_x = \lim_{\Delta t \to 0} \frac{\Delta v_x}{\Delta t} = \frac{dv_x}{dt} \]

\[ v_x = v_{0x} + a_{x}t \]
\[ x = x_0 + v_{0x} t + \frac{a t^2}{2} \]

\section{Trajectoire d'un projectile}

\begin{figure}
  \begin{center}
    \begin{tikzpicture}[domain=1:4]
      \draw[->] (-0.2,0) -- (4.8,0) node[right] {$x$};
      \draw[->] (0,-0.2) -- (0,4.8) node[above] {$y$};
      \draw plot[id=sin] function{-x*x + 5.875*x - 4.375} node[right] {};
      \draw plot[id=vec] coordinates{(1,0.5)(2,4.375)}
      node[right] {$v_0$};
      \draw[fill=green!40] (1,0.5) -- (1,0.5)++(75:0.75)
      arc (75:0:0.75)-- (1,0.5);
      \draw (1,0.5)++(37.5:0.5) node {$\alpha_0$};
      \draw[dashed] (1,0.5) -- (1,0);
      \draw (1,0) -- (1,-0.1) node[below] {$x_0$};
      \draw[dashed] (1,0.5) -- (0,0.5);
      \draw (0,0.5) -- (-0.1,0.5) node[left] {$y_0$};
    \end{tikzpicture}
  \end{center}
  \caption{Trajectoire d'un projectile}
  \label{fig:proj}
\end{figure}
En utilisant les variables $v_0$ et $\alpha_0$
de la Figure~\ref{fig:proj}, on peut écrire
\begin{align}\label{eq:projvx}
  v_x & = v_0 \cos{ \alpha_0 }\\
  v_y & = v_0 \sin{ \alpha_0 } - gt \label{eq:projvy}
\end{align}
\[ x = \int v_x dt \stackrel{\eqref{eq:projvx}}{=}
x_0 + v_0 \cos{ \alpha_0 } t \]
\[ y = \int v_y dt \stackrel{\eqref{eq:projvy}}{=}
y_0 + v_0 \sin{ \alpha_0 } t - \frac{1}{2} g t^2 \]

\section{Mouvement circulaire uniforme (MCU)}
\[ a_{rad} = \frac{v^2}{R} \]
\[ T = \frac{2\pi R}{v} \]
\[ a_{rad} = \frac{4\pi^2 R}{T^2} \]
\[ a_{tan} = \frac{ d|\vec{v}| }{ dt } \]
\[ F_{net} = ma_{rad} = m\frac{v^2}{R} \]

\section{Courbe}
\paragraph{Si la courbe est plate}
\[ m\frac{v^2}{R} \leq \mu_smg \]
D'où
\[ v_{max} = \sqrt{\mu_s g R} \]
\paragraph{Si la courbe est incliné d'un angle $\beta$}
On utilise $n\cos\beta = mg$ et non comme certain auraient pu penser
$n = mg\cos\beta$ pour permettre à la voiture de suivre sa trajectoire,
voir \cite[p.~157]{young}.
On a donc aussi $n\sin\beta = ma_{rad}$.
\begin{align*}
  \tan{\beta} & = \frac{a_{rad}}{g}\\
  \tan{\beta} & = \frac{v^2}{gR}
\end{align*}

\section{Vitesse relative}
À une dimension
\[ v_{P/A-x} = v_{P/B-x} + v_{B/A-x} \]
À deux dimensions
\[ \vec{v}_{P/A} = \vec{v}_{P/B} + \vec{v}_{B/A} \]

\part{Résistance des fluides et vitesse terminale}
% +---------------------------------------------+
% | Résistance des fluides et vitesse terminale |
% +---------------------------------------------+
\section{À petite vitesse}
\[ F_f = kv \]
\[ v_t = \frac{mg}{k} \]
\section{À grande vitesse}
\[ F_f = Dv^2 \]
\[ v_t = \sqrt{ \frac{mg}{D} } \]

\part{Énergie potentielle et cinétique}
% +----------------------------------+
% | Énergie potentielle et cinétique |
% +----------------------------------+
\section{Travail}
\[ W = \int_{x_1}^{x_2} \vec{F} d\vec{l} \]
\section{Energie}
\[ K = \frac{mv^2}{2} \]
\[ U_{\text{potentielle gravitationnelle}} = mgy \]
$K$ et $U$ sont définis à une constante près.

\section{Forces conservatrices}
Lorsqu'aucune force extérieure agit sur le corps,
le travail est conservatif (il n'y a pas de perte d'énergie).
\[ E = K_1 + U_1 = K_2 + U_2 = \mathrm{cste} \]

\section{Forces non-conservatrices}
Lorsque des forces extérieurs agissent sur le corps
(comme le frottement de l'air),
le travail est non-conservatif et l'équation précédente devient
\[ K_1 + U_1 + W_\mathrm{ext} = K_2 + U_2 \]
où $W_\mathrm{ext}$ vaut
\[ W_\mathrm{ext} = (K_2 - K_1) + (U_2 - U_1) = F(x_2 - x_1) \]

\section{Energie potentielle élastique}
\[ F_{\text{ressort}} = kx \]
\[ U_{\text{élastique}} = \frac{kx^2}{2} \]
$k$ dépend de l'élasticité du ressort.

\part{Gravitation}
% +-------------+
% | Gravitation |
% +-------------+
\section{Force d'attraction d'un corps}
\[ F_g = \frac{Gm_1m_2}{r^2} \]
\[ G = \numprint[\newton\squaren{\meter}\per\squaren{\kilogram}]
{6.67e-11} \]
Le poids d'un corps est la somme des forces gravitationnelles exercées
sur celui-ci par tous les autres corps de l'univers.
\[ W_\mathrm{grav} = \int_{r_1}^{r_2}F_r dr \]
\[ U = -\frac{Gm_Em}{r_e} \]

\section{Satellite en orbite circulaire}
\[ \frac{GMm}{r^2} = \frac{mv^2}{r} \Leftrightarrow v_\mathrm{orbitale}
= \sqrt{\frac{GM}{r}} \]
\[ T = \frac{2\pi r}{v} = 2\pi r \sqrt{\frac{r}{GM}}
= \frac{2\pi r^{3/2}}{\sqrt{GM}} \]

\section{Vitesse de libération}
\[ E_\mathrm{totale} =
  E_\mathrm{potentielle} + E_{\text{cinétique}} \Leftrightarrow 0
= -\frac{GMm}{r} + \frac{mv^2}{2} \]
\[ \frac{GMm}{r} = \frac{mv^2}{2} \Leftrightarrow v_{\text{libération}}
\geq \sqrt{\frac{2GM}{r}} \]
\[ v_{\text{libération}} \geq \sqrt{2}v_\mathrm{orbitale} \]

\part{Quantité de mouvement et impulsion}
% +------------------------------------+
% | Quantité de mouvement et impulsion |
% +------------------------------------+
\section{Définition par la deuxième loi de Newton}
Soit $\vec{p}$ la quantité de mouvement d'un corps
\[ \vec{p} = m\vec{v} \]
\[ \sum \vec{F} = m\vec{a} = m\frac{d\vec{v}}{dt} = \frac{d\vec{p}}{dt} \]
Soit l'impulsion $\vec{J}$ définit comme suit
\[ \vec{J} = \int_{t_1}^{t_2} \sum\vec{F} dt =
  \int_{t_1}^{t_2}\frac{d\vec{p}}{dt} dt =
\int_{\vec{p_1}}^{\vec{p_2}} d\vec{p} = \vec{p_2} - \vec{p_1} \]

\section{Quantité de mouvement et énergie cinétique}
\begin{itemize}
  \item L'énergie cinétique correspond au travail total effectué sur un corps
    pour accélérer celui-ci de l'état d'équilibre à sa vitesse actuelle;
  \item Le momentum équivaut à l'impulsion pour accélérer
    un corps de l'état d'équilibre à sa vitesse présente.
\end{itemize}

\section{Conservation de la quantité de mouvement}
Si la somme des forces extérieurs est nulle,
alors la quantité de mouvement totale du système est contante.
\[ \vec{P} = \vec{p_1} + \vec{p_2} + ... \]
{\bf Attention ! Avec deux quantités de mouvement de directions différentes,
il faut utiliser l'addition vectorielle !}

\part{Collisions}
% +------------+
% | Collisions |
% +------------+
Pour chaque collision,
la quantité de mouvement totale initiale et finale sont égale.
\section{Types de collisions}
\begin{itemize}
  \item [Élastique]
    Les forces intervenant dans la collision sont conservatrices
    et l'énergie totale du système reste donc
    la même avant et après la collision.
  \item [Inélastique]
    L'énergie totale du système est moindre après la collision.
  \item [Totalement inélastique]
    Après la collision,
    les deux corps \emph{collent} ensemble pour ne former qu'un.
\end{itemize}

\section{Collisions complètement inélastiques}
Par la conservation de la quantité de mouvement
\[ m_A\vec{v_{A1}} + m_B\vec{v_{B1}} = (m_A + m_B)\vec{v_2} \]
D'où
\[ \vec{v_{2}} = \frac{m_A\vec{v_{A1}} + m_B\vec{v_{B1}}}{m_A + m_B} \]

\section{Collisions élastiques}
Considérons que nous sommes dans un problème à une dimension.
Par la conservation de la quantité de mouvement
\begin{align*}
  m_1\vec{u_1} + m_2\vec{u_2} & = m_1\vec{v_1} +  m_2\vec{v_{2}}\\
  \frac{1}{2}m_1u_{1}^2 + \frac{1}{2}m_2u_{2}^2
  & = \frac{1}{2}m_1v_{1}^2 + \frac{1}{2}m_2v_{2}^2\\
\end{align*}
D'où
\begin{align*}
  v_1 & = \frac{u_1(m_1 - m_2) + 2m_2u_2}{m_1 + m_2}\\
  v_2 & = \frac{u_2(m_2 - m_1) + 2m_1u_1}{m_1 + m_2}
\end{align*}
La résolution algébrique nous donne aussi la solution triviale
$v_1 = u_1$ et $v_2 = u_2$ qui est à rejeter car ça suppose que
les corps soient passés l'un à travers l'autre sans provoquer de collision.

\part{Mouvement périodique}
% +----------------------+
% | Mouvement périodique |
% +----------------------+
La position $x$ est en mouvement périodique
lorsqu'il existe une constant $k$ tel que
\[ F = kx \]
ça peut aussi marcher avec un angle
à la place de $x$ comme pour le cas du pendule.

\section{Formules}
\[ x = A\cos{(\omega{t} + \phi)} \]
\begin{itemize}
  \item \emph{$A$}: l'amplitude du mouvement;
  \item \emph{$\omega$}: La vitesse angulaire donnée par
    \[ \omega = \frac{V}{R}; \]
  \item \emph{$\phi$}: Le déphasage du mouvement.
\end{itemize}
\section{Fréquence, période et vitesse angulaire}
\[ T = \frac{1}{f} = \frac{2\pi}{\omega} \]
\[ \omega = 2\pi{f} = \frac{2\pi}{T} \]

\section{Oscillation d'un ressort}
Soit
\[ F = ma = m\frac{d^2x}{dt} = -kx \]
En résolvant l'équation différentielle on trouve
\[ x(t) = A\sin{(\omega{t})} \]
On peut ensuite trouver $\omega$ en injectant la solution dans l'équation
\begin{align*}
  m(A\sin{(\omega{t})})'' & = - kA\sin{(\omega{t})}\\
  -mA\omega^2\sin{(\omega{t})} & = - kA\sin{(\omega{t})}\\
  m\omega^2 & = k
\end{align*}
On trouve alors
\[ \omega = \sqrt{\frac{k}{m}} \]

\section{Pendule simple}
Soit
\[ F = ma = m\frac{d^2x}{dt} = m\frac{Ld^2\theta}{dt} = -mg\sin{(\theta )} \]
Pour des angles de faible amplitude on va pouvoir considérer
\[ \sin(\theta) \approx \theta \]
En effet, il est bien connu que
$\lim_{\theta \to 0} \frac{\sin\theta}{\theta} = 1$.
On obtient donc
\[ \frac{Ld^2\theta}{dt} = -g\theta \]
En résolvant l'équation différentielle on trouve
\[ \theta{(t)} = \theta_0\sin{(\omega{t})} \]
On peut ensuite trouver $\omega$ en injectant la solution dans l'équation
\begin{align*}
  L(\theta_0\sin{(\omega{t})})'' & = -g\theta{(t)}\\
  -L\theta_0\omega^2\sin{(\omega{t})} & = -g\theta_0\sin{(\omega{t})}\\
  L\omega^2 & = g
\end{align*}
On trouve alors
\[ \omega = \sqrt{\frac{g}{L}} \]
Similitude avec le ressort
\[ F_{\theta} = -mg\sin\theta \approx -mg\theta = -mg\frac{x}{L} \]
\[ \omega =\sqrt{\frac{k}{m}} =\sqrt{\frac{mg/L}{m}} = \sqrt{\frac{g }{L}} \]

\section{Energie dans un mouvement harmonique}
\[ E = \frac{1}{2}mv^2_x + \frac{1}{2}kx^2 = \frac{1}{2}kA^2 = \mathrm{cste} \]

\begin{thebibliography}{9}
  \bibitem{young}
    Young and Freedman,
    \emph{University Physics},
    13th Edition
\end{thebibliography}

\end{document}
