\documentclass[fr]{../../../../../../eplexam}

\usepackage{../../../../../../eplcode}

\hypertitle{Informatique}{1}{FSAB}{1401}{2012}{Juin}
{Mattéo Couplet}
{Olivier Bonaventure et Charles Pecheur}[
    \paragraph{Remarque de l'auteur}
    Ce document ne contient pas l'énoncé détaillé de l'examen ; il peut cependant être retrouvé à l'adresse suivante
    \begin{center}
        \url{https://drive.google.com/a/student.uclouvain.be}
    \end{center}
]

%\begin{document}

\lstset{language={Java}}
\let\code\lstinline % gros flemmard

\section{}
Écrivez la spécification, la signature et le corps du constructeur de la classe \code{Song} qui permet de représenter une chanson utilisée par le juke-box. Chaque chanson est caractérisée par un nom d'auteur (de type \code{String}), un titre (de type \code{String}) et une durée en secondes (de type \code{int}).

\begin{solution}
    \lstinputlisting{src/q1.java} 
\end{solution}

\newpage
\section{}
Dans la classe \code{Catalog}, écrivez le corps du constructeur dont la spécification est reprise ci-dessous :
\lstinputlisting{src/q2_spec.java}

\begin{solution}
    On crée un \code{BufferedReader} qui va lire le contenu du fichier. Après lecture de la première ligne qui contient le nombre de chansons, on instancie le tableau \code{contents}. On parcoure ensuite le fichier ligne par ligne en remplissant le tableau. Tout cela est englobé dans un \code{try catch} afin d'attraper une éventuelle \code{IOException}. Ne pas oublier de fermer le \code{BufferedReader} à la fin !
    \lstinputlisting{src/q2.java}
\end{solution}

\newpage
\section{}
La classe \code{PlayList} est une structure chainée permettant de stocker la liste des chansons qui doivent être jouées par le juke-box. Complétez le diagramme d'objets représentant la \code{Playlist plist} après l'exécution du programme suivant :
\lstinputlisting{src/q3_spec.java}
Représentez sur votre dessin la \code{PlayList plist} et ses différents nœuds (les trois instances de \code{Song} sont déjà dessinées) et les références entre ces objets.

\begin{solution}
    Cette question est à la recherche d'un contributeur !
    %TODO faire un joli diagramme :-)
\end{solution}

\newpage
\section{}
Dans la classe \code{PlayList}, écrivez le corps de la méthode \code{addBefore} dont la spécification est la suivante.

\lstinputlisting{src/q4_spec.java}

\begin{solution}
    On cherche le premier nœud tel que son \code{.next()} contient \code{songin} (en traitant le \code{head} comme un cas particulier).
    \lstinputlisting{src/q4.java}
\end{solution}

\newpage
\section{}
Dans la classe \code{JukeBox}, écrivez complètement la classe privée \code{JukeBoxListener} qui implémente l'interface \code{PlayerListener}, selon la spécification suivante. Inutile de recopier cette spécification.
\lstinputlisting{src/q5_spec.java}

\begin{solution}
    On définit un constructeur vide qui nous sera utile par la suite. Tant qu'une chanson est en cours, on ne fait rien. Puis, si la playlist n'est pas vide, on la \code{dequeue} et on place la nouvelle chanson dans le \code{player}.
    \lstinputlisting{src/q5.java}
\end{solution}

\newpage
\section{}
Dans la classe \code{JukeBox}, écrivez le corps du constructeur dont la spécification est reprise ci-dessous. N'oubliez pas d'associer au lecteur le \code{listener} défini à la question précédente.
\lstinputlisting{src/q6_spec.java}

\begin{solution}
    \lstinputlisting{src/q6.java}
\end{solution}

\newpage
\section{}
Dans la classe \code{Catalog}, écrivez le corps de la méthode \code{NAuthors} dont la spécification est la suivante.
\lstinputlisting{src/q7_spec.java}

\begin{solution}
    Il s'agit de compter le nombre de chansons ayant des auteurs distincts ; pour cela, on stocke les auteurs déjà visités dans une \code{LinkedList}. On parcoure le tableau \code{contents}, et pour chaque chanson on parcoure la \code{LinkedList} pour voir si l'auteur est déjà présent ; sinon, on l'ajoute à la fin de la liste. Ainsi, le nombre d'auteurs distincts égale la taille de la liste.
    \lstinputlisting{src/q7.java}
\end{solution}

\end{document}
