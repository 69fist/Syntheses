\documentclass[fr]{../../../../../eplexam}

\usepackage{../../math-exam}

\hypertitle{Math\'ematique}{3}{FSAB}{1103}{2013}{Juin}
{Dekimpe Rémi\and De Wolf Christophe\and Legat Beno\^it}
{Jean-François Remacle et Grégoire Winckelmans}

\paragraph{Resource utile}
\url{http://www.forum-epl.be/viewtopic.php?t=10154}

\section*{Question 1}
On considère le modèle de trafic routier LWR
avec la condition initiale suivante.
Soit un feu rouge situé en $x = 0$.
Pour les $-d \leq x < 0$,
la densité de trafic est maximale et les voitures sont,
de ce fait, à l'arrêt.
Pour les $x > 0$ et $x < -d$,
il n'y a pas de voitures.
En $t = 0$, le feu passe au vert.
\begin{enumerate}
  \item Transformer en une équation du transport de $\rho(x,t)$.
  \item Dessiner les caractéristiques de $\rho(x,t)$.
  \item Donner $x(t)$ la trajectoire d'une voiture située en $-d$
    (la fin de la file quand le feu passe au vert).
  \item Au bout de combien de temps la dernière voiture passe le feu ?
\end{enumerate}

\solution{
}

\section*{Question 2}
On considère l'EDP de Laplace pour $u(r,\theta)$:
$$\nabla ^2u =
\ffpart{u}{r} +\frac{1}{r} \fpart{u}{r} + \frac{1}{r^2}\ffpart{u}{\theta} = 0$$
dans un cercle de rayon $a$ avec une coupure en $\theta = 0$.
Sur le pourtour du cercle,
on impose que $u(a, \theta) = g(\theta)$.
Sur la coupure en $\theta = 0$,
on impose que $u(r, 0) = 0$ d'un côté et
$\fpart{u}{\theta}(r, 2\pi) = \frac{U_0}{2\pi}$ de l'autre.
\begin{enumerate}
  \item Faites un dessin du domaine dans le plan physique $(x,y)$ et dans le
    système $(r,\theta)$ et
    indiquez les conditions aux limites sur chaque bord.
  \item Obtenez par la méthode de séparation des variables,
    la solution sous forme d'un développement en série.
    Écrivez clairement les intégrales à effectuer
    pour obtenir les coefficients du développement.
\end{enumerate}

\solution{
  Comme il y a une coupure, notre solution ne doit pas spécialement être périodique.

  Appliquons la méthode de séparation de variable sur $u$ en posans
  $u(r, \theta) = R(r)\Theta(\theta)$.
  Il ne faut donc pas spécialement séparer les conditions non-homogènes,
  il faut faire les deux en même temps sinon,
  on aura la solution triviale $u = 0$.

  L'équation de Laplace devient
  $$r^2\frac{R''}{R} + r\frac{R'}{R} = -\frac{\Theta''}{\Theta}.$$
  On sait que le terme de gauche ne dépend pas de $\theta$ et celui de droite
  pas de $r$, comme ils sont égaux, ils sont constants.
  \begin{itemize}
    \item Si ces termes sont nuls, $\exists A, B \in \mathbb{R}$ tels que
      $\Theta(\theta) = A\theta + B$.
      On a $\Theta(0) = 0$, ce qui donne $B = 0$.

      En essayant une solution $R(r) = r^a$, on trouve $a = 0$.
      On a donc juste une constante.
      Au final, on a $u_0(r,\theta) = u_0(\theta) = C\theta$.

      Avec ce mode 0, on va s'occuper de la condition non-homogène en $\theta$
      pour que les autres modes soient homogènes en $\theta$.
      Du pure dévouement.
      On a donc $C = \frac{U_0}{2\pi}$.
    \item S'ils valent une constante négative,
      soit $k > 0$ tel qu'elle valle $-k^2$,
      $\exists D, E \in \mathbb{R}$ tels que
      $\Theta(\theta) = D\exp(k\theta) + E\exp(-k\theta)$.
      Comme le mode 0 s'occupe de la condition non-homogène en $\theta$,
      il reste plus que 2 conditions homogène en $\theta$ ce qui implique que $D = E = 0$.
    \item S'ils valent une constante positive,
      soit $k > 0$ tel qu'elle valle $k^2$,
      $\exists F, G \in \mathbb{R}$ tels que
      $\Theta(\theta) = F\cos(k\theta) + G\sin(k\theta)$.
      %si $k$ est rationnel. $k$ est donc rationnel mais comme la période doit
      %être $2\pi$, $k$ est même entier.
      %Notons alors $k_n = n$ où $n \in \mathbb{N}^*$.

      Comme $u(r, 0) = 0$ et que $R$ ne vaut
      pas 0 pour tout $x \in [0, a]$ sinon on a la solution triviale $u = 0$,
      on a $\Theta(0) = 0$ donc $F = 0$ et $\Theta(\theta) = G\sin(k\theta)$.
      La deuxième condition homogène en $\theta$ impose que $kG\cos(k2\pi) = 0$
      d'où $k_n = \frac{2n+1}{4}$.

      L'équation pour $R$ nous invite à essayer des solutions du genre
      $r^a$ où $a \in \mathbb{N}$.
      On trouve alors $r^{k_n}$ et $r^{-k_n}$.
      La solution générale est donc $H_nr^{k_n} + I_nr^{-k_n}$.
      Comme $k_n > 0$ et que notre solution doit rester finie même en $r = 0$,
      on doit avoir $I_n = 0$.

      Du coup, notre solution générale en $u$ est
      \[ u_g(r, \theta) = \sum_{n=1}^\infty J_n\sin(k_n\theta)r^n. \]
  \end{itemize}
  On peut maintenant attaquer la dernière condition non-homogène.
  \begin{align*}
    u_g(a,\theta) + u_0(a,\theta) & = g(\theta)\\
    J_n\sin(n\theta)a^n & = g(\theta) - \frac{U_0}{2\pi}\theta\\
    J_m\int_0^{2\pi}\sin(k_n\theta)\sin(k_m\theta) a^n d\theta & = \int_0^{2\pi}(g(\theta) - \frac{U_0}{2\pi}\theta)\sin(k_m\theta) \dif \theta\\
    J_m \pi a^m & = \int_0^{2\pi}(g(\theta) - \frac{U_0}{2\pi}\theta)\sin(k_m\theta) \dif \theta\\
    J_m & = \frac{1}{\pi a^m}\int_0^{2\pi}(g(\theta) - \frac{U_0}{2\pi}\theta)\sin(k_m\theta) \dif \theta.
  \end{align*}
}

Notre solution finale est donc \[u(r,\theta) = u_0(\theta) + u_g(r,\theta) \]

\section*{Question 3}
Idem que la question 3 de Janvier 2012.

\section*{Question 4}
Calculez l'intégrale suivante
$$\int_0^\infty\frac{\cos(x)}{x^2+a^2}\dif x$$
où $a$ est un réel strictement positif par la méthode des résidus.

\solution{
  On va considérer pour cela l'intégrale
  \[
    \oint_C f(z) dz =\oint_C \frac{e^{iz}}{z^2+a^2} dz
  \]
  que l'on va intégrer sur un demi-cercle C dans la moitié supérieur du plan complexe, dont on va faire tendre le rayon R vers l'infini.
  \paragraph{} Le dénominateur de $f(z)$ possède 2 racines: $ia$ et $-ia$. La racine $ia$ se trouve à l'intérieur du contour ($a<R$ car on fait tendre R vers l'infini) et c'est donc un pôle d'ordre 1. La méthode des résidus nous donne donc
  \[
    \oint f(z) dz=2\pi i \cdot \res(f,ia)=2\pi i\cdot g(ia) = \frac{\pi}{ae^{a}}
  \]
  où
  \[
    g(z)=(z-ia)\cdot f(z)=\frac{e^{iz}}{z+ia}
  \]

  \paragraph{} On peut décomposer l'intégrale sur le contour en une somme d'intégrales pour faire apparaître l'intégrale que nous devons calculer
  \[
    \oint_C f(z) dz=\int_0^{R} f(x)dx + \int_0^{\pi} f(Re^{i\theta})Re^{i\theta}id\theta+ \int_{-R}^0 f(x)dx
  \]
  \paragraph{} On peut annuler la deuxième intégrale grâce au deuxième Lemme de Jordan lorsque R tend vers l'infini. En effet,
  \[
    f(z)=\frac{1}{z^2+a^2} e^{iz}=h(z)e^{iz}
  \]
  Or
  \[
    \lim\limits_{z \to \infty} h(z)=\lim\limits_{z \to \infty} \frac{1}{z^2+a^2}=0
  \]
  Donc
  \[
    \int_0^{\pi} f(Re^{i\theta})Re^{i\theta}id\theta=0
  \]
  \paragraph{} Il nous reste les 2 intégrales le long de l'axe réel.
  \[
    \int_0^{\infty} \frac{e^{ix}}{x^2+a^2} dx + \int_{-\infty}^0 \frac{e^{ix}}{x^2+a^2}dx
  \]
  \[
    =\int_0^{\infty} \frac{e^{ix}}{x^2+a^2} dx+ \int_0^{\infty}\frac{e^{-ix}}{x^2+a^2} dx
  \]
  \[
    =\int_0^{\infty} \frac{e^{ix}+e^{-ix}}{x^2+a^2} dx
  \]
  \[
    =2\int_0^{\infty} \frac{\cos(x)}{x^2+a^2} dx
  \]
  Ce qui nous donne bien l'intégrale voulue.
  \paragraph{} On obtient donc
  \[
    \frac{1}{2}\int_0^{\infty} \frac{\cos(x)}{x^2+a^2} dx=\frac{1}{2}\oint_C \frac{e^{iz}}{z^2+a^2} dz=\frac{\pi}{2ae^{a}}
  \]
}

\end{document}
