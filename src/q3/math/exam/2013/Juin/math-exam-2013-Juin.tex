\documentclass[fr]{../../../../../eplexam}

\hypertitle{Math\'ematique}{3}{FSAB}{1103}{2013}{Juin}
{De Wolf Christophe\and Legat Beno\^it}
{Jean-François Remacle et Grégoire Winckelmans}

\paragraph{Resource utile}
\url{http://www.forum-epl.be/viewtopic.php?t=10154}

\section*{Question 1}
On considère le modèle de trafic routier LWR
avec la condition initiale suivante.
Soit un feu rouge situé en $x = 0$.
Pour les $-d \leq x < 0$,
la densité de trafic est maximale et les voitures sont,
de ce fait, à l'arrêt.
Pour les $x > 0$ et $x < -d$,
il n'y a pas de voitures.
En $t = 0$, le feu passe au vert.
\begin{enumerate}
  \item Transformer en une équation du transport de $\rho(x,t)$.
  \item Dessiner les caractéristiques de $\rho(x,t)$.
  \item Donner $x(t)$ la trajectoire d'une voiture située en $-d$
    (la fin de la file quand le feu passe au vert).
  \item Au bout de combien de temps la dernière voiture passe le feu ?
\end{enumerate}

\solution{
}

\section*{Question 2}
On considère l'EDP de Laplace pour $u(r,\theta)$:
$$\nabla ^2u =
\ffpart{u}{r} +\frac{1}{r} \fpart{u}{r} + \frac{1}{r^2}\ffpart{u}{\theta} = 0$$
dans un cercle de rayon $a$ avec une coupure en $\theta = 0$.
% FIXME est-ce que ça a du sens de mettre coupure ici ?
Sur le pourtour du cercle,
on impose que $u(a, \theta) = g(\theta)$.
Sur la coupure en $\theta = 0$,
on impose que $u(r, 0) = 0$ d'un côté et
$\fpart{u}{\theta}(r, 2\pi) = \frac{u_0}{2\pi}$ de l'autre.
\begin{enumerate}
  \item Faites un dessin du domaine dans le plan physique $(x,y)$ et dans le
    système $(r,\theta)$ et
    indiquez les conditions aux limites sur chaque bord.
  \item Obtenez par la méthode de séparation des variables,
    la solution sous forme d'un développement en série.
    Écrivez clairement les intégrales à effectuer
    pour obtenir les coefficients du développement.
\end{enumerate}

\solution{
  Appliquons la méthode de séparation de variable sur $u$ en posans
  $u(r, \theta) = R(r)\Theta(\theta)$.
  On a en fait 3 conditions sur $\Theta$
  car il faut aussi qu'il soit périodique.
  Il ne faut donc pas spécialement séparer les conditions non-homogènes,
  il faut faire les deux en même temps sinon,
  on aura la solution triviale $u = 0$.

  L'équation de Laplace devient
  $$r^2\frac{R''}{R} + r\frac{R'}{R} = -\frac{\Theta''}{\Theta}.$$
  On sait que le terme de gauche ne dépend pas de $\theta$ et celui de droite
  pas de $r$, comme ils sont égaux, ils sont constants.
  \begin{itemize}
    \item Si ces termes sont nuls, $\exists A, B \in \mathbb{R}$ tels que
      $\Theta(\theta) = A\theta + B$.
      Cela montre que $\Theta$ n'est pas périodique, ce qui ne convient pas.
    \item S'ils valent une constante négative,
      soit $k > 0$ tel qu'elle valle $-k^2$,
      $\exists A, B \in \mathbb{R}$ tels que
      $\Theta(\theta) = A\exp(k\theta) + B\exp(-k\theta)$ qui n'est à nouveau
      pas périodique.
    \item S'ils valent une constante positive,
      soit $k > 0$ tel qu'elle valle $-k^2$,
      $\exists A, B \in \mathbb{R}$ tels que
      $\Theta(\theta) = A\cos(k\theta) + B\sin(k\theta)$ qui est périodique
      si $k$ est rationnel. $k$ est donc rationnel mais comme la période doit
      être $2\pi$, $k$ est même entier.
      Notons alors $k_n = n$ où $n \in \mathbb{N}^*$.

      Comme $u(r, 0) = 0$ et que $R$ ne vaut
      pas 0 pour tout $x \in [0, a]$ sinon on a la solution triviale $u = 0$,
      on a $\Theta(0) = 0$ donc $A = 0$ et $\Theta(\theta) = B\sin(k\theta)$.

      L'équation pour $R$ nous invite à essayer des solutions du genre
      $r^a$ où $a \in \mathbb{N}$.
      On trouve alors $r^{k_n}$ et $r^{-k_n}$.
      La solution générale est donc $C_nr^{k_n} + D_nr^{-k_n}$.
      Comme $k_n > 0$ et que notre solution doit rester finie même en $r = 0$,
      on doit avoir $D_n = 0$.

      Du coup, notre solution générale en $u$ est
      \[ u_n(r, \theta) = \sum_{n=1}^\infty B_n\sin(n\theta)C_nr^n \]
      Après, pour que $\fpart{u}{\theta}(r, 2\pi)$ soit constant, il faut que
      $n = 1$ mais si $n = 1$, on aura jamais $g(\theta)$...
  \end{itemize}
  Bon ça marche pas...
  Oublions la périodicité, ça doit être pour ça qu'il y a une coupure...
  Le problème c'est que si on fait une contrainte à la fois,
  celle qui dit que $R$ est constant fait de toute façon tout foirer parce
  qu'on sait alors ni avoir $g(\theta)$ sur les bords, ni $0$ car ça donnerait
  la solution nulle...
}

\section*{Question 3}
Idem que la question 3 de Janvier 2012.

\section*{Question 4}
Calculez l'intégrale suivante
$$\int_0^\infty\frac{\cos(x)}{x^2+a^2}\dif x$$
où $a$ est un réel strictement positif par la méthode des résidus.

\solution{
}

\end{document}
