\documentclass[fr]{../../../../../../eplexam}

\usepackage{../../../math-exam}
\usepackage{../../../../../../eplunits}

\newcommand{\X}[1]{\ifthenelse{\equal{#1}{true}}{X}{}}

\newcommand{\vraioufaux}[1]{
  \begin{center}
    \begin{tabular}{p{0.7\textwidth}|c|c|}
      $+1$ point si bonne réponse, $-0.5$ point si mauvaise réponse,
      0 point si abstention & Vrai & Faux\\
      \hline
      (a) Le problème de condition initiale, avec $u(s, 0) = f(s)$ donné,
      est toujours bien posé:&\X{#1}&\\
      Supposons, pour la suite, le problème bien posé. Alors,&&\\
      (b) on ne peut trouver la solution que pour $t \geq 0$:&\X{#1}&\\
      (c) $I(t)=\int_{-\infty}^{\infty} u(x,t)\dif x$ est
      conservée au cours du temps:&\X{#1}&\\
      (d) $I(t)=\int_{-\infty}^{\infty} u^2(x,t)/2\dif x$ diminue
      toujours au cours du temps:&\X{#1}&\\
    \end{tabular}
  \end{center}
}


\hypertitle{Math\'ematique}{3}{FSAB}{1103}{2011}{Janvier}
{Mathieu Descamps\and Legat Beno\^it}
{Jean-François Remacle et Grégoire Winckelmans}

\paragraph{Resource utile}
\url{http://www.forum-epl.be/viewtopic.php?t=10112}


\section{}
On considère l'EDP suivante pour $u(x,t)$:
\[ \fpart{}{x}(cu) + \fpart{u}{t} = \fpart{}{x}(\alpha \fpart{u}{x}) \]
avec $\alpha = \alpha(x) \geq 0$ borné, et avec $c = c(x)$ borné.
On cherche la solution pour $- \infty < x < \infty$ et pour
$t \geq 0$. La condition initiale est $u(s,0) = u_0\frac{s}{L}$ avec $L$ une constante de longueur et $u_0$ une constante ayant les mêmes unités que $u$.


\begin{enumerate}
  \item Physiquement, que sont $\alpha(x)$ et $c(x)$ et quelles sont leurs dimensions ?
  \item On considère un problème en milieu infini ($-\infty < x < \infty$), avec $u \to 0$ pour $x \to \pm\infty$,
    et avec $\alpha(x) > 0$.
    \vraioufaux{false}
  \item On considère un problème en milieu semi-infini ($0 \leq x < \infty$), pour $t \geq 0$, et dans le cas avec $\alpha = 0$.
    On a que $c(x) = c_0 \exp(-x/L)$, avec $c_0$ et $L$ constants.
    La condition initiale pour $s \geq 0$ est $u(s,0) = u_0 \frac{s}{B}\left(1 - \frac{s}{B}\right)$, avec $u_0$ et $B$ constants.
    La condition limite est $u(0,t) = 0$.
    \begin{enumerate}
      \item Prouver que $I(t) = \int_{-\infty}^\infty u(x,t) \dif x$ est ici conservé au cours du temps.
      \item Obtenez le réseau des caractéristiques: équation et esquisse dans le plan
        (propre, avec des axes adimensionnels et chiffrés!).
      \item Déterminer graphiquement la région du plan où la solution $u(x,t)$ est nulle.
      \item Obtenez la solution $u(x,t)$ dans la région du plan où celle-ci n'est pas nulle.
    \end{enumerate}
\end{enumerate}

\begin{solution}
  \vraioufaux{true}

  Voir l'interro Octobre 2014 car elle est fort semblable.
\end{solution}


\section{}
On considère l'EDP de Laplace pour $u(r,\theta)$ :
$$\nabla ^2u = \frac{\partial^2u}{\partial r^2} +\frac{1}{r} \frac{\partial u}{\partial r} + \frac{1}{r^2}\frac{\partial^2u}{\partial \theta^2} = 0  $$ dans un cercle de rayon $a$. Sur le pourtour du cercle, on impose que $u(a,\theta) = g(\theta)$ continue et dérivable.
\begin{enumerate}
  \item Faites un dessin du domaine dans le plan physique $(x,y)$ et dans le système $(r,\theta)$ et indiquez les conditions aux limites sur chaque bord.
  \item Obtenez par la méthode de séparation des variables, la solution $u(r,\theta)$ pour $0 \leq r \leq a$ sous forme d'un développement en série. Écrivez clairement les intégrales à effectuer pour obtenir les coefficients du développement.
  \item Idem pour $v(r,\theta)$ pour $a \leq r$ obéissant aux mêmes conditions que $u$.
  \item Montrer que ces deux solutions $u$ et $v$ se raccordent de manière continue et dérivable.
\end{enumerate}

\begin{solution}
\end{solution}


\section{}
Montrer que P=NP.

\begin{solution}
\end{solution}

\section{}
What is the answer to Life, the Universe, and Everything ?

\begin{solution}
  42
\end{solution}

\end{document}
