\documentclass[fr]{../../../../../eplexam}

\usepackage{../../math-exam}

\hypertitle{Math\'ematique}{3}{FSAB}{1103}{2012}{Ao\^ut}
{Leboutte Pierre\and Legat Beno\^it}
{Jean-François Remacle et Grégoire Winckelmans}

\paragraph{Resource utile}
\url{http://www.forum-epl.be/viewtopic.php?t=11680}

\section*{Question 1}
En utilisant l'analyse complexe, calculez l'intégrale définie
\[ \int_0^{2\pi} \frac{1}{1+3\cos^2\theta} \dif\theta. \]

\solution{
  On obtient $\int_C f(z) \dif z$ où
  \[ f(z) = \frac{-4iz \dif z}{3z^4 + 10z^2 + 3} \]
  Le dénominateur a 4 racines:
  $\pm i/\sqrt{3}$ et $\pm i\sqrt{3}$.
  Les deux premières sont dans le cercle centré en 0 de rayon 1 autour duquel on fait l'intégrale donc l'intégrale vaut, par le théorème des résidus,
  \[ 2\pi i (\res(f(z-i/\sqrt{3})+\res(f+i\sqrt{3})) \]
  Les deux résidus sont égaux et valent
  \[ \frac{-4/3i(i/\sqrt{3})}{2i/\sqrt{3}(-1/3+3)} = -i/4 \]
  L'intégrale vaut donc
  \[ 2\pi i\left(-\frac{i}{4} - \frac{i}{4}\right) = \pi. \]
}

\section*{Question 2}
Soit l'EDP
\[ \fpart{u}{t} = \alpha\ffpart{u}{x} \]
avec $\alpha > 0$. Le domaine est borné: $0 \leq x \leq L$.
La condition initiale est $u(x, 0) = \mu_0 (1 - \frac{x}{L})$
avec $\mu_0 > 0$ constant.
Pour $t > 0$, on impose que $\fpart{u}{x}(0,t) = -\frac{\mu_0}{L}$
et que $u(L,t) = 0$.
\begin{enumerate}
  \item De quel type d'EDP s'agit-il ? Quelles sont les unités
    de $\alpha$ et qu'est-ce physiquement ?
    Mathématiquement, qu'est-ce qu'un temps court pour cette EDP ?
  \item
    La solution est de la forme  $u(x,y) = R(x,t) + \Theta(x,t)$ avec $R(x,t)$ la solution en régime et $\Theta(x,t)$ la solution transitoire . Obtenez d'abord $R(x,y)$ et ensuite $\Theta(x,t)$, par la méthode de séparation des variables..

    Écrivez clairement les intégrales à effectuer pour obtenir les coefficients du développement en série, ne les effectuez pas )
  \item Esquissez (proprement) l'évolution ici attendue du profil de
    $\frac{u}{\mu_0}$ en fonction de $\frac{x}{L}$ en quelques
    temps différents: temps ``court'', ``moyen'' et ``long''.
\end{enumerate}

\solution{
  \begin{enumerate}
    \item C'est une EDP Parabolique,
      $\alpha [\frac{m^2}{s}]$ est le coefficient de diffusivité.
      Un temps court est un instant peu après la mise en place du système.
      C'est à dire que l'élément commence a se diffuser dans le domaine.

    \item
      En posant $u = XT$, on a
      \[ \frac{1}{\alpha}\frac{T'}{T} = \frac{X''}{X}. \]
      Le membre de gauche ne dépend pas de $x$ et celui de droite ne dépend
      pas de $t$, comme ils sont égaux, ils sont donc constants.
      Cette constante peut valoir 0, $k^2$ ou $-k^2$ ce qui donne
      3 solutions différentes qu'on doit additionner pour espérer obtenir une
      solution générale.
      \begin{itemize}
        \item Pour $0$, tu as
          $u_1(x, t) = Ax + B$ ($X = \tilde{A}x + \tilde{B}, T = \tilde{C}$);
        \item pour $k^2$,
          $u_2(x, t) = e^{\alpha k^2t} (C\cosh(kx) + D\sinh(kx))$;
        \item pour $-k^2$,
          $u_3(x, t) = e^{-\alpha k^2t} (E\cos(kx) + F\sin(kx))$.
      \end{itemize}
      La solution générale est donc
      $u = u_1 + u_2 + u_3$.

      En $t \to \infty$, $u_2$ sera toute seule à tendre vers l'infini
      donc personne sera là pour la compenser.
      Seulement, nous, on veut pas d'infini comme $u$ est une grandeur physique.
      Il faut donc l'abattre avec $C = D = 0$.
      On nomme alors la $u_3$, $\Theta(x,t)$ et $u_1$, $R(x)$.
      En l'infini toujours, $\lim_{t\to\infty}\Theta(x,t) = 0$.
      $R(x)$ est donc tout seul pour gérer les conditions limites,
      on trouve alors $A = -\frac{\mu_0}{L}$ et $B = \mu_0$.
      On a donc $R(x) = \mu_0(1 - \frac{x}{L})$.
      On remarque que $R(x) = u(x,0)$.
      En $t = 0$, on a donc
      $\Theta(x, 0) = u(x, 0) - R(x) = 0$.
      D'où
      $E\cos(kx) + F\sin(kx) = 0$ $\forall x \in [0,L]$
      c'est à dire $E = F = 0$.
      On arrive donc à
      $u(x, t) = R(x) = \mu_0(1 - \frac{x}{L})$.
  \end{enumerate}
}

\section*{Question 3}
Soit l'EDP
\[ \fpart{u}{x} + \fpart{u}{y} = 1. \]
On cherche la solution $u(x,y)$ dans tout le plan.
On donne $u = 0$ sur $\Gamma$ défini par $x + y = 0$.
\begin{enumerate}
  \item Quelles sont les dimensions de $u$ ?
  \item Obtenez l'expression paramétrique
    de $\Gamma$ et observez ensuite le réseau
    des caractéristiques,
    équations et dessin (propre, axe chiffrés !).
  \item Obtenez explicitement $u(x,y)$.
\end{enumerate}

\solution{
  \begin{enumerate}
    \item $u$ est en mètre.
    \item On a
      \[\Gamma = \left\{
          \begin{matrix}
            x(s) = s\\ y(s) = -s
        \end{matrix}\right. \]

      Le long des caractéristiques, on a $\dif x = \dif y$ d'où
      \begin{align*}
        \int_{x(s)}^x \dif x' & = \int_{y(s)}^y \dif y'\\
        x-s & = y+s\\
        s & = \frac{x-y}{2}
      \end{align*}

    \item
      On a $\dif u = \dif x$ donc
      \[ \int_{u(x(s),y(s))}^{u}\dif u' = \int_{x(s)}^{x} \dif x'. \]
      Ce qui donne
      \[ u(x,y) - u(s,-s) = x - s  = x - \frac{x-y}{2} = \frac{x+y}{2}. \]

      On a finalemement
      $u(x,y) = \frac{x+y}{2} + u(s,-s)$
      où $u(s,-s) = 0$ d'où
      \[ u(x,y) = \frac{x+y}{2}. \]
  \end{enumerate}
}

\section*{Question 4}
On définit la fonction
\[ w = \arcsin z =
  \int_0^z \frac{\dif \tilde{z}}{\sqrt{1-\tilde{z}^2}} =
  \frac{1}{i} \log\left(iz + \sqrt{1-z^2}\right)
\]
\begin{enumerate}
  \item Obtenez les points de branchement de $w$
    (\textbf{Aide}: $iz + \sqrt{1 - z^2}$ ne s'annule jamais).
    Proposez aussi un choix de coupures.
    Pour la suite, on utilise la coupure principale.
  \item Vérifiez que la définition de $w$ est bien telle que $\sin w = z$.
  \item Obtenez l'expression de $w$ pour $z$ purement imaginaire
    ($z = iy$ avec $-\infty < y < \infty$).
    À partir de ça, obtenez aussi l'expression de $\arcsinh y$.
  \item Obtenez le développement en série de $w$ autour de $z_0 = 0$.
    (\textbf{Aide}: obtenez d'abord le développement de
    $\fdif{w}{z}$ puis intégrez).
    Quel est le rayon de convergence de la série ?
\end{enumerate}
\rappelscomplexes

\solution{
  \begin{enumerate}
    \item
      Il n'y a pas de points de branchements pour le premier $\log$ mais il y en a pour le $\log$ correspondant à la racine.
      Les deux valeurs qui annulent la racine sont $-1$ et $1$.
      On peut définir comme coupure $\frac{\pi}{2}\leq \theta_{-1} < \frac{5\pi}{2}$ et $-\frac{\pi}{2}\leq \theta_{1} < \frac{3\pi}{2}$ par exemple.
    \item
      On a
      \begin{align*}
        \sin(w) & = \frac{\exp(iw) - \exp(-iw)}{2i}\\
                & = \frac{\exp(2iw) - 1}{2i\exp(iw)}\\
                & = \frac{(iz + \sqrt{1-z^2})^2 - 1}{2i(iz+\sqrt{1-z^2})}\\
                & = \frac{-z^2 + 1 - z^2 + 2iz\sqrt{1-z^2} - 1}{2i(iz+\sqrt{1-z^2})}\\
                & = \frac{-2z^2 + 2iz\sqrt{1-z^2}}{2i(iz+\sqrt{1-z^2})}\\
                & = \frac{iz^2 + z\sqrt{1-z^2}}{iz+\sqrt{1-z^2}}\\
                & = z.
      \end{align*}
    \item
      On a
      $\sin(it) = \frac{\exp(iit) - \exp(-iit)}{2i}
      = -\frac{\exp(t) - \exp(-t)}{2i} = -\frac{\sinh(t)}{i} = i\sinh(t).$
      Dès lors, en posant $t = \arcsinh(x)$, on a
      $\sin(i\arcsinh(x)) = ix$
      d'où
      $\arcsinh(x) = -i\arcsin(ix).$
  \end{enumerate}
}

\end{document}
