\documentclass[fr]{../../../eplsummary}

\usepackage{esint}
\usepackage{esvect}

\usepackage{ucs}
\usepackage[version=3]{mhchem}
\usepackage{pgfplots}

\usepackage[SIunits]{../../../eplunits}
\usepackage{../../../eplchem}

\newcommand{\patm}{p_{\mathrm{atm}}}
\newcommand{\penv}{p_{\mathrm{env}}}
\newcommand{\psys}{p_{\mathrm{sys}}}

\hypertitle{Chimie}{3}{FSAB}{1302}
{Nicolas Cognaux\and Jean Herman\and Beno\^it Legat}
{Christian Bailly, Francis Delannay et Hervé Jeanmart}

\part{Gaz parfaits}
% +--------------+
% | Gaz parfaits |
% +--------------+
\section{Les gaz parfaits}
\subsection{Notion d'équation d'état pour un corps pur monophasique}
Un corps monophasique signifie constitué d'une seule phase,
c'est-à-dire que l'état de la matière est le même partout
(liquide, solide ou gazeux) et que la composition est la même partout (toujours le cas pour un corps pur mais pas nécessairement pour un mélange).

\begin{itemize}
  \item
    La pression $p$ exprimé en Pascal $\pascal = \newton\per\squaren{\meter}$;
  \item
    La température $\theta$, $T$ exprimé en $\celsius$ ou $\kelvin$.
    $T(\kelvin) = \theta(\celsius) + \si{273.15}{\kelvin}$;
  \item
    Le volume molaire (resp. massique)
    $V_m = \frac{V}{n}$ (resp. $v = \frac{V}{m}$)
    exprimé en $\meter\cubed\per\mole$ (resp. $\meter\cubed\per\kilogram$).
\end{itemize}

Ce sont des variables thermodynamiques intensives,
c'est-à-dire définies en chaque point du corps, peu importe la quantité de matière. À l'opposé, une variable extensive est proportionnelle à la quantité de matière et prend une seule valeur pour tout le corps (e.g. volume, masse, nb moles, énergie interne\dots).

On sait qu'il existe toujours une relation d'état,
c'est-à-dire une relation
qui lie les variables intensives d'un système
\[ F(p, T, V_m) = 0 \]
Cette relation n'est valable qu'à l'équilibre thermodynamique du système, c'est-à-dire si ses grandeurs intensives sont suffisamment constantes
dans le temps et homogènes dans l'espace.

\subsection{Boyle, Charles et Avogadro}
\subsubsection{Loi de Boyle (1662)}
Pour une quantité donnée de gaz (inerte), à température constante
\begin{equation} \label{eq:loi_boyle}
pV = \constant \triangleq B
\end{equation}
où on a défini la constante de Boyle $B$.
\subsubsection{Loi de Charles (1787)}
Pour une quantité donnée de gaz (inerte), à pression constante
\[ V(\theta) = V(\si{0}{\celsius})(1 + \alpha \cdot \theta) \]
$\alpha$ est le coefficient d'expansion thermique isobare:
\[ \alpha \eqdef
\frac{1}{V}\left(\fpart{V}{\theta}\right)_p \]

À pression suffisamment faible,
$\alpha$ est constant et indépendant de la température et du gaz (noté $\alpha_0$)
\[ \left(\frac{V(\theta)}{\alpha_0 V(0\celsius)}\right)_{p\rightarrow 0} =
\alpha_0^{-1}+\theta(\celsius) \eqdef T(\kelvin) \]
et $\alpha_0^{-1}$ vaut \si{273.15}{\celsius}.

On observe que le coefficient d'expansion thermique devient identique
pour tous les gaz lorsqu'on l'extrapole à pression nulle.
Ce qui est l'indice majeur de l'existence
d'une échelle absolue des températures.
Le volume d'un gaz parfait est proportionnel à la température absolue et
s'annule à \si{0}{\kelvin}.
C'est évidemment une extrapolation, il est impossible en pratique d'atteindre
\si{0}{\kelvin} et tous les gaz sont liquéfiés ou
gelés avant d'arriver à cette température.

Enfin, (toujours quand $p \rightarrow 0$) cette loi nous donne
\begin{equation} \label{eq:charles}
\dfrac{V(\theta)}{T(\kelvin)} = \alpha_0 V(0\celsius) = \constant \triangleq C
\end{equation}
où on a défini la constante de Charles $C$.

\subsubsection{Combinaison des lois de Boyle et Charles}
En utilisant les lois de Boyle et Charles (équations \eqref{eq:loi_boyle} et \eqref{eq:charles}) tout en considérant le cas général où la température, la pression et la quantité de gaz (masse de gaz $m$) peuvent varier ($B$ et $C$ sont donc fonctions des variables susnommées), on a
\[ \eqref{eq:loi_boyle} \Rightarrow pV = B(m,T) \]
\[ \eqref{eq:charles} \Rightarrow \frac{V}{T} = C(m,p) \]
\[ \Rightarrow \frac{B(m,T)}{T} = C(m,p)p = A(m) \]
\[ \Rightarrow \frac{pV}{T} = A(m) = A_0m \]
La dernière égalité vient du fait qu'expérimentalement on observe
une proportionnalité à la quantité de matière.

\subsubsection{Hypothèse d'Avogadro (1811)}
``Des volumes identiques de gaz à la même pression et
température contiennent le même nombre de molécules''.

$\frac{pV}{T}$ est donc proportionnel au nombre de molécules
\[ \frac{pV}{T} = Nk_B = nR \]
$N$ est le nombre de molécules,
$n$ le nombre de moles,
$N_A = \frac{N}{n}$ le nombre d'Avogadro\footnote{$N_A$ est donc le nombre de molécules par mole.} valant $\si{6.02214e23}{}$,
$R = \si{8.3145}{\joule\per\mole\kelvin}$ la constante des gaz parfaits et
$k_B = \si{1.3807e-23}{\joule\per\kelvin}$ la constante de Boltzmann.

\subsection{L'équation d'état des gaz parfaits}
\[ pV = Nk_BT = nRT \]
\[ \frac{pV_m}{T} = R \]
Ne pas oublier que l'idéalité n'est valable qu'à faible pression et
loin du point de liquéfaction.
La deuxième équation est là pour rappeler que l'équation d'état relie bien
les variables intensives et non extensives comme la première équation pourrait
le faire croire.

Des erreurs d'unités sont fréquemment commises en utilisant cette célèbre équation. C'est pourquoi il est utile de rappeler que si on utilise la valeur de $R$ fournie ci-dessus, cela sous-entend que l'on travaille dans le système d'unités SI. $p$ est donc en $\pascal$, $V$ en $\meter\cubed$, $n$ en $\mole$ bien sûr, $T$ en $\kelvin$ et l'unité de $R$ vous est déjà fournie ci-dessus (l'annexe \ref{sec:const} contient les facteurs de conversion entre les différentes unités de pression).

\subsection{La loi de Dalton}
La loi de Dalton dit que pour un mélange de gaz, la pression totale
excercée par le mélange est la somme des pressions partielles exercées
par chaque constituant.
\[ p_\mathrm{totale} = \sum_i^n p_i \]
Cette loi n'est valable que pour les gaz parfaits.
En effet, ce n'est qu'en considérant que les molécules n'ont pas d'influence
les unes sur les autres que leur effet peut être considéré comme additif.
\paragraph{Exemple}
Si on a un mélange composé de deux gaz parfaits 1 et 2
dans un volume $V$ à température $T$,
la pression exercée par le mélange vaut, par la loi de Dalton,
\[ p = \frac{n_1 R T}{V} + \frac{n_2 R T}{V} = \frac{(n_1 + n_2) R T}{V} \]

Ce qui s'explique par le fait que dans les hypothèses de base
de la théorie des gaz parfaits, on suppose la grandeur des molécules
négligeable par rapport à la distance qui les séparent.
Les molécules du gaz 1 ont donc la même influence sur la pression que
les molécules du gaz 2.

\section{La théorie de Bernoulli}
\subsection{Le billard moléculaire parfait}
\subsubsection{Hypothèses de base}
\begin{itemize}
  \item Un gaz est composé de très nombreuses molécules approximées
    par des sphères de diamètre $d$ et de masse $m$;
  \item Le diamètre des sphères est petit par rapport
    à la distance moyenne ($\lambda$) qui les sépare, qui elle même est petite
    par rapport à la dimension de l'échantillon;
  \item Les molécules se meuvent indépendamment des autres (sauf au moment des collisons),
    c'est à dire qu'on néglige leur potentiel d'interaction car $d \ll \lambda$;
  \item La mécanique classique est d'application et le système est conservatif
    c'est à dire que l'énergie cinétique est constante;
  \item Les parois n'absorbent aucune énergie lors des collisions.
\end{itemize}
Les limites de validité de la théorie sont
les limites de validité des hypothèses.

L'idée que la taille des molécules est petite par rapport à leur interdistance
est centrale pour la notion même de gaz parfait.
Si les molécules se rapprochent trop,
leur volume n'est plus négligeable et leurs interactions non plus.
On passe alors aux gaz réels puis aux liquides, ...
D'un point de vue moléculaire,
un gaz parfait est donc un gaz pour lequel les molécules sont
trop éloignées les unes des autres pour avoir des interactions attractives ou
répulsives.

Il y a trois conséquences essentielles de ces hypothèses
\begin{itemize}
  \item L'énergie totale des molécules est purement
    l'énergie cinétique de translation;
  \item Lorsque deux molécules entrent en collision,
    l'énergie cinétique et la quantité de mouvement totales sont
    conservées mais redistribuées au hasard entre les deux molécules;
  \item Lors d'une collision avec la paroi de l'enceinte,
    la molécule subit une réflexion spéculaire avec conservation
    de l'énergie cinétique.
\end{itemize}
La conservation de l'énergie cinétique des molécules après collision
n'est vraie qu'en l'absence de réaction chimique.

\subsection{Prédire la pression: description simplifiée}
Prenons un cube d'une longueur $L$ de coté,
contenant une molécule de vitesse $\vec{c} = (c_x, c_y, c_z)$.
Après collision avec une paroi (plan $yz$),
la vitesse devient $\vec{c_+} = (-c_x, c_y, c_z)$.
Connaissant la vitesse et la distance entre les deux parois opposées,
on peut déterminer le temps qui sépare deux collisions sur la paroi:
\[ \Delta t = \frac{2L}{|c_x|} \]

Le changement de quantité de mouvement du à une collision avec
la paroi est de $\Delta \vec{p} = m \cdot \Delta \vec{c}
= m (\vec{c_+} - \vec{c}) = m(-2c_x, 0, 0)$.
On peut donc calculer la force exercée sur la paroi
\[ F_x = \left\|\frac{\dif \vec{p_x}}{\dif t}\right\| \approx
\frac{\|\Delta \vec{p_x}\|}{\Delta t} = \frac{mc_x^2}{L}. \]

La pression exercée par \strong{une molécule} sur la parois est
la force exercée dessus par unité de surface
\[ p = \frac{F_x}{L^2} = \frac{m{c_x}^2}{L^3} = \frac{m{c_x}^2}{V} \]

Pour $N$ molécules identiques, on peut faire la somme ce qui donne
\[ p = \frac{m}{V} \sum_{i=1}^{N}{({c_x}^2)_i} \]

À cause des collisions entre les molécules,
toutes les vitesses ne sont pas identiques,
on doit donc passer par la vitesse moyenne, on définit donc
\[ N \overline{{c_x}^2} \eqdef \sum_{i=1}^{N}{({c_x}^2)_i} \]

On peut écrire
\begin{align*}
  c^2 & = {c_x}^2 + {c_y}^2 + {c_z}^2\\
  \overline{c^2} & =
  \overline{{c_x}^2} + \overline{{c_y}^2} + \overline{{c_z}^2}
\end{align*}

Par le fait que le système est isotrope donc
toutes les composantes de vitesses sont égales, on a donc
\[  \overline{{c_x}^2} = \frac{1}{3}\overline{c^2} \]

On a dès lors,
\[ pV = \frac{1}{3}Nm\overline{c^2} \]

%Dans le calcul, on n'a pas tenu compte des collisions des molécules
%entre elles mais ce n'est pas grave,
%car les collisions ne font que redistribuer l'énergie cinétique
%entre les molécules et ne changent pas l'énergie cinétique moyenne.
%FIXME: L'énergie cinétique d'une molécule ne change pas avec une collision
%Réponse : "Lorsque deux molécules entrent en collision, l'énergie cinétique
%et la quantité de mouvement totales sont conservées mais redistribuées au
%hasard entre les deux molécules." (voir Hypothèses de base dans la section
% Théorie de Bernoulli).

\subsection{Lien entre la température et l'énergie cinétique des molécules}
Si on dénote l'énergie cinétique d'une molécule par $k = \frac{mc^2}{2}$ alors
\[ pV = \frac 23 N\bar k \]
Comme $pV = Nk_BT$ par l'équation d'état des gaz parfaits, on obtient
\[ \bar k = \frac 32 k_B T \]
On peut en déduire que le température est
le reflet macroscopique de l'énergie d'agitation des molécules.

\subsection{Lien entre l'énergie cinétique et l'énergie interne}
La théorie cinétique des gaz aboutit à la conclusion que
le contenu énergétique interne d'un volume de gaz parfait (au repos),
qu'on appelle l'énergie interne $U$ est constitué exclusivement
par l'énergie cinétique des molécules.
\[ U = N\bar k = \frac 32 Nk_BT = \frac 32nRT \]
et l'énergie interne molaire vaut
\[ U_m = N_A\bar k = \frac 32 RT \]
L'énergie interne $U$ est le contenu énergétique d'un système fermé,
quand on a soustrait l'énergie cinétique et potentielle macroscopique,
c'est à dire quand le système est au repos dans le référentiel du laboratoire.
En toute généralité, $U$ contient une part d'énergie cinétique et
une part d'énergie potentielle liée aux interactions attractives et
répulsives des molécules mais pour un gaz parfait, il n'y a pas d'interaction.

L'équation n'est valable que pour les gaz monoatomiques,
c'est à dire ceux pour lesquels l'énergie cinétique se résume à
l'énergie de translation.

\section{``Preuves'' expérimentales}
\subsection{Mouvement Brownien (1827)}
Le mouvement erratique (= aléatoire) de minuscules particules à la surface d'un liquide,
sous l'impulsion du bombardement par les molécules de liquide.

\subsection{Loi des pressions partielles pour les mélanges de gaz parfaits}
``La pression totale d'un mélange de gaz parfaits est
la somme des pressions partielles des constituants'' John Dalton (1766-1844).

Cette loi découle naturellement de la théorie cinétique des gaz car
l'énergie cinétique des molécules de gaz ne dépend pas de leur nature mais
uniquement de la température absolue.

\subsection{Loi d'effusion de Graham (1846)}
Soit une enceinte remplie de gaz, percée d'un trou minuscule et
en contact avec le vide.
Si le trou est suffisamment petit,
le gaz quitte l'enceinte par ``effusion'',
c'est à dire sans collisions au niveau de la sortie.
Dans ce cas, le débit molaire de particules ($J$ en \mole\per\second)
hors de l'enceinte doit être proportionnel à la vitesse des particules.

Si on a deux types de molécules dans l'enceinte,
l'unicité de la température impose que
l'énergie cinétique des deux types soit identique
\[ \frac 32 k_BT = \bar k_1 = \bar k_2 = \frac{m_1\bar c_1^2}{2} =
\frac{m_2\bar c_2^2}{2} \]
Si on a le même nombre de molécules de chaque type
\[ \frac{J_1}{J_2} = \left(\frac{\bar c_1}{\bar c_2}\right) =
\left(\frac{\bar c_1^2}{\bar c_2^2}\right)^{1/2} =
\left(\frac{m_2}{m_1}\right)^{1/2} =
\left(\frac{M_2}{M_1}\right)^{1/2} \]
Le résultat de Graham a été perçu comme
une preuve de la validité de la théorie cinétique.

\subsection{Capacité calorifique d'un gaz à volume constant}
\[ C_V \eqdef \left(\fpart{Q}{T} \right)_V \]
$C_V$ est la chaleur spécifique,
c'est le rapport entre la quantité de chaleur réversible fournie au gaz et
le changement de température observé lorsque le volume est maintenu constant.
La notion de transformation réversible concerne les changements suffisamment
lents pour qu'on puisse à tout moment considérer le système à l'équilibre.

L'énergie interne d'un système est noté $U$, est une fonction d'état,
elle prend une valeur précise pour chaque état du système.
Pour un gaz parfait c'est dont un triplet ($T$,$p$,$V_m$),
dont on sait que deux valeurs déterminent automatiquement la troisième.
Par contre $Q$ et $W$ ne sont pas des fonction d'états,
pour un état précis on peut avoir plusieurs combinaisons de $Q$ et $W$ car
\[ \dif U = \delta Q + \delta W \]
À volume constant, il n'y a pas de travail exercé par le système et
$(\dif U)_V = (\delta Q)_V$ ce qui entraine deux autres égalités
\begin{align*}
  C_V & \eqdef \left(\fpart{Q}{T}\right)_V =
  \left(\fpart{U}{T}\right)_V\\
  C_{V,m} & \eqdef \left(\fpart{Q_m}{T}\right)_V =
  \left(\fpart{U_m}{T}\right)_V
\end{align*}
En reprenant l'énergie interne molaire d'un gaz parfait:
$U_m = \frac 32 RT$, on obtient
\[ C_{V,m} = \left(\fpart{U_m}{T}\right)_V =
\left(\fpart{\frac 32 RT}{T}\right)_V = \frac 32R \]
C'est une prédiction basée sur les hypothèses initiales,
elle est bien vérifiée pour les gaz monoatomiques mais pas pour les autres.

\section{Limites de validité}
\subsection{Capacité calorifique des gaz polyatomiques}
Une molécule de gaz monoatomique a trois degrés de liberté de translation.
Par degré de liberté, l'énergie d'agitation thermique vaut, par molécule
\[ \frac 13 \bar k = \frac 12 k_BT \]
et par mole
\[ \frac 13 N_A \bar k = \frac 12 RT \]
Pour une molécule diatomique on peut rajouter deux degrés de rotation et
un de vibration mais la rotation autour de l'axe de la molécule est négligée
car le moment d'inertie est négligeable par rapport aux deux autres.

Si le nombre de degrés de liberté est $\lambda$,
\[ C_{V,m} = \left(\fpart{U_m}{T}\right)_V =
\frac{(\lambda_\mathrm{translation} + \lambda_\mathrm{rotation} +
2\lambda_\mathrm{vibration})R}2 \]
Il faut tenir compte de l'énergie cinétique et potentielle
pour les degrés de liberté de vibration, ils comptent double.

\begin{figure}[h]
  \begin{center}
    \begin{tikzpicture}[x=3cm,y=1cm]
      \def\xmin{0}
      \def\xmax{3}
      \def\ymin{0}
      \def\ymax{4}

      \draw[style=help lines, ystep=4, xstep=1] (\xmin,\ymin) grid
      (\xmax,\ymax);
      \draw[style=help lines] (\xmin, 1.5) -- (\xmax, 1.5);
      \draw[style=help lines] (\xmin, 2.5) -- (\xmax, 2.5);
      \draw[style=help lines] (\xmin, 3.5) -- (\xmax, 3.5);

      %\draw[->] (\xmin,\ymin) -- (\xmax,\ymin) node[right] {$x$};
      %\draw[->] (\xmin,\ymin) -- (\xmin,\ymax) node[above] {$y$};

      \node at (0, \ymin) [below] {\si{10}{\kelvin}};
      \node at (1, \ymin) [below] {\si{100}{\kelvin}};
      \node at (2, \ymin) [below] {\si{1000}{\kelvin}};
      \node at (3, \ymin) [below] {\si{10000}{\kelvin}};
      \node at (\xmin, 1.5) [left] {$C_V = \frac 32R$};
      \node at (\xmin, 2.5) [left] {$C_V = \frac 52R$};
      \node at (\xmin, 3.5) [left] {$C_V = \frac 72R$};

      %\draw[color=blue] plot[smooth,mark=*,mark size=1pt] file {data.dat}
      %node [right] {data};

      \draw[color=red, thick, domain=\xmin:0.8] plot[id=parabola]
      (\x, 1.5);
      \draw[color=red, thick, domain=0.8:1.05] plot
      (\x, {8*(\x-0.8)^2+1.5});
      \draw[color=red, thick, domain=1.05:1.3] plot
      (\x, {-8*(\x-1.3)^2+2.5});
      \draw[color=red, thick, domain=1.3:1.7] plot
      (\x, 2.5);
      \draw[color=red, thick, domain=1.7:1.95] plot
      (\x, {4*(\x-1.7)^2+2.5});
      \draw[color=red, thick, domain=1.95:2.2] plot
      (\x, {2*(\x-1.95)+2.75});
      \draw[color=red, thick, domain=2.2:2.45] plot
      (\x, {-4*(\x-2.45)^2+3.5});

      \def\boxcol{blue}
      \draw[\boxcol, ->] (1.65, 2.3) to (1.5, 2.5);
      \fill[white] (1.2, 0.1) rectangle (2.1, 2.3);
      \draw[\boxcol] (1.2, 0.1) rectangle (2.1, 2.3);
      \draw[\boxcol] node[below] at (1.65, 2.3)
      {\begin{minipage}{2.6cm}\scriptsize À température ambiante,
        on doit rajouter les degrés de libertés rotationnels,
        ce qui fait 5 degré de liberté.
      \end{minipage}};

      \draw[\boxcol, ->] (0.5, 1.7) to (0.5, 1.5);
      \fill[white] (0.1, 1.7) rectangle (0.9, 4.1);
      \draw[\boxcol] (0.1, 1.7) rectangle (0.9, 4.9);
      \draw[\boxcol] node[above] at (0.5, 1.7)
      {\begin{minipage}{2.3cm}\scriptsize À basse température,
        les degrés de libertés rotationnels sont ``gelés''
        et la chaleur spécifique est comme celle
        d'un gaz parfait monoatomique.
      \end{minipage}};

      %\draw[\boxcol, ->] (2.6, 3.6) to (2.45, 3.5);
      \fill[white] (2.6, 2.3) rectangle (3.7, 4.9);
      \draw[\boxcol] (2.6, 2.3) rectangle (3.7, 4.9);
      \draw[\boxcol] node[right] at (2.6, 3.6)
      {\begin{minipage}{3cm}\scriptsize
        Proche de la température de dissociation, à \si{3200}{\kelvin},
        les degrés de liberté de vibration rentrent maintenant
        en compte dans la chaleur spécifique.
      \end{minipage}};

      \draw[\boxcol] (3.5, 2.3) to (3.5, 2.2);
      %\fill[white] (3, 2.2) rectangle (3.8, 4.9);
      \draw[\boxcol] (3.05, 0.35) rectangle (3.95, 2.2);
      \draw[\boxcol] node[below] at (3.5, 2.2)
      {\begin{minipage}{2.4cm}\scriptsize
        L'énergie cinétique de vibra\-tion et l'énergie potentielle
        ont chacune une énergie de $\frac{R}{2}$.
        %For vibration kinetic energy and potential energy
        %each get energy $\frac{R}{2}$ in \strong{equipartition}.
      \end{minipage}};

    \end{tikzpicture}
  \end{center}
  \caption{Cas de l'hydrogène moléculaire \ce{H2}}
  \label{fig:deqr}
\end{figure}

Expérimentalement, on observe pour $\ce{O2}$, $\ce{N2}$ et $\ce{H2}$
à température ambiante que
$C_{V,m} = \frac52 R$ au lieu de $C_{V,m} = \frac72 R$,
c'est le deuxième palier sur la Figure~\ref{fig:deqr}.
À basse température, la capacité calorifique de $H_2$ tombe même à $\frac32 R$,
des degrés de libertés sont bloqués,
c'est le premier palier sur la Figure~\ref{fig:deqr}.

\subsection{Déviations par rapport aux gaz parfait}
Pour un gaz réel on a $\frac{pV_m}{RT} = Z$,
si le gaz est parfait $Z$ vaut 1 mais généralement il dépend de $T$ et $p$.
$Z$ est appelé le facteur de compression.
En fonction de la pression, à haute température,
$Z$ est toujours plus grand que 1 et est croissant.
À basse température, $Z$ décroit un peu en dessous de 1 et finit par remonter.
\begin{figure}[h]
  \begin{center}
    \begin{tikzpicture}[x=2cm,y=1cm]
      \def\xmin{0}
      \def\xmax{4}
      \def\ymin{-2}
      \def\ymax{2}

      \draw[->, style=help lines] (\xmin, \ymin) -- (\xmin, \ymax);
      \node[left] at (\xmin, \ymax) {$Z$};
      \draw[->, style=help lines] (\xmin, 0) -- (\xmax, 0);
      \node[left] at (\xmin, 0) {1};
      \node[below] at (\xmax, 0) {$p$};

      %\draw[color=red, thick, domain=\xmin:\xmax] plot
      %function{x^2} node[right] {test};
      \draw[color=red, thick, domain=\xmin:1.5] plot
      (\x, \x^2) node[above] {Haute température};
      \draw[color=orange, thick, domain=\xmin:2.3] plot
      (\x, \x^3/6) node[right] {Température de Boyle};
      \draw[color=green, thick, domain=\xmin:3] plot
      (\x, 0) node[below] {Gaz parfait};
      \draw[color=blue, thick, domain=\xmin:3] plot
      (\x, {(\x-1)^2/3-0.33}) node[right] {Basse température};

    \end{tikzpicture}
  \end{center}
  \caption{Facteur de compression en fonction
  de la pression pour différentes températures}
  \label{fig:fact_comp}
\end{figure}

Comme on peut s'en apercevoir sur la Figure~\ref{fig:fact_comp},
il existe une température pour laquelle le gaz agit comme un gaz parfait
à basse pression, c'est la température de Boyle.

Un point critique est un état particulier où les phases liquide et
gazeuse commencent à coexister.
On ne peut plus négliger les attractions entre molécules et leur volume propre.
L'équation de Van der Waals tiens compte de cet état:
\[ (p+aV_m^{-2})(V_m-b) = RT \]
$a$ et $b$ dépendent de la nature du gaz mais pas de $T$ ou $p$.

\section{Distribution de Maxwell-Boltzmann}
\subsection{Nécessité}
La théorie de Bernoulli ne permet pas de connaitre la répartition des vitesses
des molécules elle ne donne que la vitesse moyenne.
Maxwell et Boltzmann ont déterminé statistiquement la distribution
des vitesses des molécules dans un volume de gaz.

On considérera pour toute cette section que
\begin{align}
  c & = ||(c_1, c_2, c_3)||\label{eq:cci}\\
  & = \sqrt{c_1^2 + c_2^2 + c_3^2}\nonumber
\end{align}

\subsection{Fonctions de probabilité et de densité}
%Pour un système isotrope, $\Phi_i(c_i) = \Phi_j(c_j) (\forall i, j)$
Soient $\Phi_k(c_k)$ la probabilité qu'une molécule ait la composante $i$
de sa vitesse inférieure ou égale à $c_k$.

De façon semblable, posons $\Phi(c)$.
$\Phi$ est la fonction de \strong{probabilité}.

En considérant un système isotrope,
$\Phi_1 = \Phi_2 = \Phi_3$.
Posons alors $\Phi_i \eqdef \Phi_1 = \Phi_2 = \Phi_3$.

On peut définir la fonction de \strong{densité}
\[ f_k(c_k) \eqdef \fdif{\Phi_k}{c_k}. \]
et $f(c)$ définit de la même manière.

En considérant un système isotrope,
on a $f_i \eqdef f_1 = f_2 = f_3$.

Pour illustrer la relation entre $\Phi$ et $f$, on peut se convaincre que
\[ \Phi(c_b) - \Phi(c_a) = \int_{c_a}^{c_b} f(c) \dif c \]

Autrement dit, l'intégrale de $f(c)$ sur un intervalle de vitesse donne
la probabilité que la vitesse soit comprise dans cet intervalle.

\subsection{Calculs de probabilité}
La probabilité $\dif p_i$ qu'une molécule ait
la composante $c_k$ de sa vitesse dans l'interval $[c_k; c_k + \dif c_k]$,
comme les $\Phi_k$ sont indépendants, peut être obtenue par
\begin{align*}
  \dif p_i & = \prod_{k=1}^3 \dif\Phi_i(c_k)\\
  & = \prod_{k=1}^3 f_i(c_k) \dif c_k.
\end{align*}

Soit $\dif p$ la probabilité que la vitesse d'une molécule soit
dans l'intervale $[c; c + \dif c]$.
En d'autres mots, $\dif p = f(c) \dif c$.
On a
\[ \dif p
= \iiint_{\eqref{eq:cci}} \dif p_i
= \iiint_{\eqref{eq:cci}} \prod_{k=1}^3 f_i(c_k) \dif c_k \]
C'est l'intégrale de $\prod_{i=1}^3 f_k(c_k)$ sur une coquille sphérique
d'épaisseur $\dif c$.

Pour un système isotrope, nous pouvons considérer que
la densité de deux vitesses de même intensité mais de direction
différentes est pareil.
$\prod_{k=1}^3 f_i(c_k)$ a donc la même valeur
quelle que soit les $c_k$ (tant que \eqref{eq:cci} est respectée bien entendu).

On peut donc le sortir de l'intégrale et calculer
\begin{align*}
  \dif p & = \prod_{k=1}^3 f_i(c_k)
  \iiint_{\eqref{eq:cci}} \dif c_1 \dif c_2 \dif c_3\\
  & = \prod_{k=1}^3 f_i(c_k) \cdot 4\pi c^2 \dif c
\end{align*}

Dès lors,
\[ f(c) = 4\pi c^2 \prod_{k=1}^3 f_i(c_k) \]

Il n'y a que\footnote{Voici un résultat un peu parachuté et pas trop rigoureux,
c'est la démonstration proposée par Maxwell.
Une démonstration plus rigoureuse est due à Boltzmann
mais elle fait appel à des concepts trop avancés.}
$f_i(c_k) = A \exp(-Bc_k^2)$ qui satisfasse les égalités. D'où
\[ f(c) = 4\pi c^2 A^3 \exp (-Bc^2). \]

% TODO : add graphe p.60
%On remarque que pour un volume donnée,
%la forme de la fonction dépend de la température.
%Plus elle sera basse plus la courbe sera proche de l'axe des ordonnées.

\subsubsection{Détermination des constantes $A$ et $B$}
Il faut maintenant calculer les paramètres $A$ et $B$.
\begin{itemize}
  \item Pour $B$ on utilise la moyenne de
    l'énergie cinétique grâce à la distribution
    \[ \frac{\int_0^{\infty}\frac 12 mc^2f(c)\dif c}
      {\int^{\infty}_0 f(c) \dif c}
    = \frac 32k_BT \]
    D'où on tire $B = \frac m{2_BT}$
  \item Pour $A$, on utilise le fait que
    \[ \int_0^{\infty} f(c) \dif c = 1 \]
\end{itemize}
Finalement on obtient
\[ f(c) = 4\pi \left(\frac{m}{2\pi k_BT}\right)^\frac 32
c^2 \exp\left(\frac{-mc^2}{2k_BT}\right) \]
On voit que la distribution est proportionnelle à
$\exp \left(\frac{-E}{k_BT}\right)$ avec $E$ l'énergie cinétique d'une molécule.
Cette exponentielle est appelé le \strong{facteur de Boltzmann}.

À partir de cette distribution on peut calculer :

\begin{itemize}
	\item $c_p = \sqrt{\frac{2k_BT}{m}}$, la vitesse la plus probable ;
	\item $\overline{c} = \sqrt{\frac{8k_BT}{\pi m}}$, la vitesse moyenne ;
	\item $\sqrt{\overline{c^2}} = \sqrt{\frac{3k_BT}{m}}$, la vitesse quadratique moyenne.
\end{itemize}

\section{Résumé}

Pour un gaz parfait,
\[ pV = Nk_BT = nRT = mR^*T. \]

Par définition, la capacité calorifique vaut
\[ C_V \eqdef \left(\fpart{U}{T}\right)_V
= \frac{(\lambda_\mathrm{translation} + \lambda_\mathrm{rotation} +
2\lambda_\mathrm{vibration})}{2}nR. \]

Soient $\lambda$ le nombre de degrés de liberté
et $\bar k$ l'énergie cinétique moyenne d'une molécule,
pour un gaz parfait, on a
\begin{align*}
  \bar k & = \frac{(\lambda_\mathrm{translation} + \lambda_\mathrm{rotation} +
  2\lambda_\mathrm{vibration})}{2}k_BT\\
  U & = \frac{(\lambda_\mathrm{translation} + \lambda_\mathrm{rotation} +
  2\lambda_\mathrm{vibration})}{2}nRT\\
  & = C_V T\\
  & = N \bar k.
\end{align*}

Pour un gaz parfait monoatomique,
\[ U = \frac 32 nRT. \]

Pour un gaz parfait diatomique,
\begin{center}
  \begin{tabular}{|l|l|}
    \hline
    À basse température, &
    \( U = \frac 32 nRT \)\\
    %\hline
    À température ambiante, &
    \( U = \frac 52 nRT \)\\
    %\hline
    À haute température, &
    \( U = \frac 72 nRT \)\\
    \hline
  \end{tabular}
\end{center}

\part{Thermodynamique}
% +-----------------+
% | Thermodynamique |
% +-----------------+
\section{Rappels sur le premier Principe}
\subsection{Définitions fondamentales}
\subsubsection{La notion de système}
Les éléments étudiés sont appelés le système,
il est délimité de l'environnement par des
frontières arbitraires physiques ou non.
Il y a trois types de systèmes distincts
\begin{itemize}
  \item Un système ouvert peut échanger de la matière et
    de la chaleur avec l'environnement;
  \item Un système fermé ne peut pas échanger de matière
    mais peut échanger de la chaleur ou du travail avec l'environnement;
  \item Un système isolé ne peut échanger ni matière,
    ni chaleur, ni travail avec l'environnement.
    On suppose aussi qu'il n'est pas soumis au champ
    d'une force à distance (gravité, \ldots).
\end{itemize}

\subsubsection{L'état d'un système}
Une variable d'état est une grandeur thermodynamique caractéristique
d'un système qui dépend uniquement de l'état actuel du système et
pas du cheminement qui a amené le système dans cet état.

Dans un système à l'équilibre, il suffit de connaitre deux variables d'états
pour caractériser entièrement le système.
Cela est vrai si il n'y a qu'une seule composante et qu'une seule phase.
Par exemple pour un gaz parfait, pour un nombre de mole connu $n$
\[ pV = nRT \]
On peut également écrire cette expression sous forme massique,
ce qui est plus utile pour des problèmes mécaniques.
\[ pV = (nM_m) \frac R {M_m} T \]
ou $M_m$ représente la masse molaire moyenne du gaz.
$\frac{R}{M_m} = R^*$ définit une nouvelle constante dépendant du gaz considéré
et vaut \si{287.1}{\joule\per\kilogram\kelvin} dans le cas de l'air.
Finalement,
\begin{align*}
  pV & = mR^*T\\
  pv & = R^*T.
\end{align*}

\subsubsection{Transformations d'un système}
Entre deux états distincts,
il existe une infinité de transformations possibles.

Une transformation est dite cyclique si l'état initial et l'état final du
système sont confondus.
Le système passe par une succession d'état intermédiaires mais le résultat
final en termes de variation des variables d'états est nul.

Lors des transformations, le système a pu interagir avec l'environnement,
c'est l'essence des cycles thermodynamiques!

Il existe des transformations pour lesquelles
une variable d'état du système est maintenue constante
\begin{itemize}
  \item Une transformation isotherme garde son température constante;
  \item Une transformation isobare garde sa pression constante;
  \item Une transformation isochore garde son volume constant.
  \item Une transformation isentrope garde son entropie constant.
\end{itemize}

\subsubsection{Détente d'un gaz}
Prenons comme exemple un cylindre fermé latéralement et dans
sa partie inférieure dont la partie supérieure est fermée par
un piston mobile mais étanche, c'est un système fermé.
Nous considérons qu'il n'y a pas de frottements
entre le piston et la paroi du cylindre.

La force résistante du piston est celle induite par
la pression exercée sur la face supérieur du piston,
c'est à dire par l'environnement
(ici, on a $\penv = \patm$ mais restons générique)
\[ F_{\mathrm{rés}} = \penv A \]
où $A$ est la section du piston.
Si le piston effectue un déplacement $d$ dans la direction opposée
à celle de la force, le travail effectué par le système vaut
\[ W = -\penv Ad = -\penv\Delta V \]
et pour un déplacement infinitésimal,
\[  \delta W = -\penv\dif V \]
Un travail est considéré comme positif lorsqu'il est fournit au système.
Si la pression externe est nulle,
le fluide ne produira pas de travail lors de sa détente.
Si la pression externe est supérieur à la pression interne,
le travail sera positif car le fluide sera comprimé.
Pour pouvoir augmenter le travail fourni lors de la détente
on peut avoir recourt à des artifices.

Imaginons un cylindre décrit comme précédemment
avec une masse posé sur sa partie supérieur.
Le piston est à l'équilibre car la pression externe
additionné à la pression due à la masse égale la pression interne.
Si on retire la masse d'un coup,
le piston va se détendre et fournir un travail d'expansion.
Si retire petit à petit un morceau de masse telle
que le piston reste toujours en équilibre, le travail sera maximum.

Comme le piston reste toujours en équilibre, on a $\psys = \penv$.
Dans ce cas, on pose souvent $p \eqdef \penv = \psys$.
On a alors
\[ \delta W = -\penv \dif V = -\psys \dif V = -p \dif V \]

Cette transformation est dite réversible car un changement infinitésimale
d'une grandeur caractéristique permet d'inverser le sens de la transformation,
elle passe par une succession d'états d'équilibre.

On peut aussi imaginer un échange de chaleur réversible lorsque
l'écart de température entre le système qui cède sa chaleur et
celui qui la reçoit est infinitésimal.

Reprenons
\[ \delta W = -p\dif V \]
Avec les suppositions suivantes
\begin{itemize}
  \item La transformation est réversible;
  \item Le système est constitué de gaz parfaits;
  \item La transformation est isotherme.
\end{itemize}
On peut écrire
\[ \delta W = -mR^*T\frac{\dif V}V \]
Et calculer
\begin{align*}
  W & = -mR^*T \ln\left(\frac{V_2}{V_1}\right)\\
    & = -mR^*T \ln\left(\frac{p_1}{p_2}\right)\\
    & = mR^*T \ln\left(\frac{p_2}{p_1}\right)
\end{align*}

\subsection{Le premier principe}
\subsubsection{Introduction et énergie interne}
Le premier principe, aussi appelé le principe d'équivalence est le suivant:

La chaleur et le travail sont deux formes équivalentes d'énergie.
Une forme peut être convertie en une autre et réciproquement.

La valeur de l'expérience de Joule vient du fait qu'il a pu
faire correspondre les unités de chaleur (calorie) et
les unité de travail (Joule), par la relation
$\si{1}{cal} = \si{4.184}{\joule}$.
Les apports de de ces deux grandeurs modifient
le contenu énergétique unique du système.
Nous noterons $U$, l'énergie interne qui est une nouvelle variable d'état.
\[ U = Q + W \]

L'énergie interne correspond à l'énergie totale d'un système fermé.
En effet, celui ci ne possède pas d'énergie cinétique ou
potentielle macroscopique.
Par contre $Q$ et $W$ ne sont pas des variables d'états,
le travail est dépendant du chemin parcouru par le système puisque
\[W = - \int_1^2p\dif V \]
Pour un système isolé
\[ \Delta U = 0 \]
et pour une transformation cyclique
\[ \Delta U_\mathrm{cycle} = 0 \]

\subsubsection{Expressions mathématiques de l'énergie interne}
Comme l'énergie interne est une variable d'état on peut l'exprimer en
fonction de deux autres variables d'états, on peut donc écrire
\begin{align*}
  \dif U & =
  \left(\fpart{U}{T}\right)_p \dif T +
  \left(\fpart{U}{p}\right)_T \dif p\\
  \dif U & =
  \left(\fpart{U}{T}\right)_V \dif T +
  \left(\fpart{U}{V}\right)_T \dif V\\
  \dif U & =
  \left(\fpart{U}{p}\right)_V \dif p +
  \left(\fpart{U}{V}\right)_p \dif V
\end{align*}

\subsection{Les capacités calorifiques}
\subsubsection{Définitions mathématiques}
Reprenons la différentielle de l'énergie interne suivante
\begin{equation}
  \dif U = \left(\fpart{U}{T}\right)_V \dif T +
  \left(\fpart{U}{V}\right)_T \dif V
  \label{eq:diffUdvdt}
\end{equation}
Si la transformation est isochore, cela devient
\[ \dif U = \left(\fpart{U}{T}\right)_V \dif T = C_V \dif T \]
La capacité calorifique (en \joule\per\kelvin) à volume constant est
la quantité de chaleur nécessaire pour
augmenter la température d'un degré du système.
Afin d'affranchir la capacité calorifique (grandeur extensive)
de la taille du système, on utilisera en pratique
la notion de chaleur massique en \joule\per\kilogram\cdot\kelvin.
\[ \dif u = c_V\dif T \]
Ou la chaleur molaire en \joule\per\mole\cdot\kelvin
\[ \dif U_m = C_{V,m}\dif T \]
Si on considère les capacités calorifiques invariables dans tout le système
\[ \Delta U = C_V \Delta T = mc_V\Delta T = nC_{V,m}\Delta T \]

\paragraph{Remarque}
En réalité, $C_V$ dépend de la température donc
\[ \int_{T_1}^{T_2} C_V \dif T \neq C_V(T_2 - T_1) \]
mais dans le cadre de ce cours, on peut considérer que $C_V$ est constant.

\subsubsection{Seconde expérience de Joule}
Le but de la deuxième expérience de Joule est de calculer le terme
\[ \pi_T = \left(\fpart{U}{V}\right)_T \]
(en $\joule\per\meter\cubed = \pascal$)
de l'équation \eqref{eq:diffUdvdt} pour une transformation qui n'est pas isochore.
Pour ce faire,
il augmente le volume d'un gaz et ajoute
un volume rempli de vide au volume initial.
Il remarque qu'il n'y a aucune variation de chaleur ($Q = 0$).

Comme le volume ajouté est vide, $p = 0$ et donc
$\int_1^2 p \dif V = 0$.

On a donc
\[ \Delta U = Q - \int_1^2 p \dif V  = 0 - 0 = 0 \]

Il n'y a donc pas de variation d'énergie interne alors que le volume a changé.
Dès lors,
\[\pi_T = 0 \]

Il s'avère en réalité que ce résultat n'est valable que pour les gaz parfaits.

\subsubsection{Pour un gaz parfait}
\begin{equation}
  \dif U = \left(\fpart{U}{T}\right)_V \dif T =
  \left(\fpart{U}{T}\right)_p \dif T = C_V\dif T \label{eq:peru}
\end{equation}

\subsubsection{Pour un gaz quelconque}
Pour un système quelconque (pour lequel l'énergie n'est
pas uniquement fonction de la température), on a
\begin{equation}
  \left(\fpart{U}{T}\right)_p =
  \pi_T \left(\fpart{V}{T}\right)_p + C_V \label{eq:dudtp}
\end{equation}

\subsubsection{Le point de vue de la mécanique statistique}
\[ U_m = \frac{N_A mc_q^2}{2} = \frac{N_A 3k_BT}{2} \]
Il n'y a pas de force d'interaction entre les molécules et
donc pas de contribution d'énergie potentielle,
seul l'énergie cinétique des molécules est présente.
On a donc
\[ \fpart{U_m}{T} = \frac{3N_A k_B}{2} = \frac 32R = C_{V,m} \]
dans le cas d'un gaz monoatomique.
Si le gaz est diatomique alors $C_{V,m} = \frac 52R$.

\subsubsection{L'enthalpie}
Dans le cas précédent, on a supposé que
l'apport de chaleur se faisait par un processus isochore.
Il y avait égalité entre $Q$ et $\Delta U$
car il n'y avait pas de travail échangé.
Considérons maintenant que le processus est isobare,
le système réalise alors un travail sur l'environnement pour maintenir
sa pression constante malgré l'apport de chaleur.
Cette apport de chaleur correspond à une nouvelle variable d'état notée $H$,
c'est l'enthalpie.
Pour un système fermé, pour un processus isobare,
\[ \dif H = \delta Q_{p = \constant} \]
On peut lier les deux variables $H$ et $U$ par
\[ H = U + pV \]
En effet, on a, comme $\dif p = 0$,
\[
  \dif H
  = \delta Q
  = \dif U + p \dif V
  = \dif U + p \dif V + V\dif p
  = \dif \left(U + pV\right) % left, right to get proper space before (
\]

Chauffer un système à pression constante demande plus d'énergie
que chauffer un système à volume constant.
La définition de l'enthalpie reflète cette réalité.
\[ \dif H = \left(\fpart{H}{T}\right)_p \dif T +
\left(\fpart{H}{p}\right)_T \dif p \]
Pour un processus isobare,
\[ \dif H = \left(\fpart{H}{T}\right)_p \dif T = C_p\dif T \]
où $C_p$ est est la capacité calorifique à pression constante.
C'est la quantité de chaleur à apporter à un système maintenu à pression
constante pour en augmenter la température d'un degré celsius.

%TODO: slide 14

\subsubsection{Relation de Mayer}
\[ C_p > C_V \]
Cela reflète l'apport supplémentaire de chaleur à fournir pour
compenser le travail effectué dans le cas à pression constante.
Dans le cas d'un gaz parfait,
\[ C_p - C_V = \left(\fpart{H}{T}\right)_p -
\left(\fpart{U}{T}\right)_V =
\left(\fpart{U}{T}\right)_p +
\left(\fpart{(pV)}{T}\right)_p -
\left(\fpart{U}{T}\right)_V =
\left(\fpart{(pV)}{T}\right)_p \]
Comme nous parlons d'un gaz parfait,
$pV = nRT$ et la relation devient
\[ C_p - C_V = nR = mR^* \]
Ce qui signifie qu'il suffit de connaitre
une des deux grandeurs pour connaitre l'autre.
Dans le cas d'une substance quelconque, en posant
\[ \alpha = \frac{1}{V} \left(\fpart{V}{T}\right)_p \]
On a
\begin{align*}
  C_p - C_V & =
  \left(\fpart{U}{T}\right)_p +
  \left(\fpart{(pV)}{T}\right)_p -
  \left(\fpart{U}{T}\right)_V\\
  & \stackrel{\eqref{eq:dudtp}}{=} \alpha \pi_T V + \alpha p V
\end{align*}
$\alpha$ est le coefficient de dilatation thermique à pression constante.

\subsubsection{Influence de la température}
Les capacités calorifiques sont fonction de la température
même dans le cas de l'hypothèse simplificatrice du gaz parfait.
les valeurs constantes prédites par la théorie cinétique des gaz sont
utiles car proches de la réalité à température ambiante.
Il existe des tables comprenant toute les valeurs de ces capacités
\begin{align*}
C_{v,m} & = \left(\fpart{U_m}{T}\right )_V\\
C_{p,m} & = \left(\fpart{H_m}{T}\right )_p
\end{align*}
Les valeurs moyennes sur un intervalle de température sont
\[ |C_{v,m}|_0^t =
\frac 1t \int_0^t \left(\fpart{U_m}{T}\right )_V \dif t \]

\subsection{Transformations particulières}
\subsubsection{Évolution adiabatique et réversible}
Une transformation d'un gaz qui évolue de manière réversible dans
un environnement dont il est parfaitement isolé thermiquement est
appelé une évolution adiabatique réversible.
Pour un gaz parfait, on a
\[ pV^{\gamma} = \constant \]
où
\[ \gamma \eqdef \frac{c_p}{c_V} = \frac{c_v+R^*}{c_v}
  = \frac{\frac{\lambda R^*}{2}+R^*}{\frac{\lambda R^*}{2}}
  = \frac{\lambda + 2}{\lambda} \]
où $\lambda$ est le nombre de degrés de liberté.
C'est à dire que
\begin{itemize}
  \item Pour un gaz composé de molécules monoatomiques
    ayant des chaleurs molaires répondant à la théorie cinétique des gaz,
    la valeur de $\gamma$ est $\frac{5}{3} \approx \si{1.67}{}$.
  \item Pour un gaz composé de molécules diatomiques
    ayant des chaleurs molaires répondant à la théorie cinétique des gaz,
    la valeur de $\gamma$ est $\frac{7}{5} = \si{1.4}{}$.
\end{itemize}
\begin{proof}
  Comme c'est un milieu adiabatique, $\delta Q = 0$ et donc
  \[ \dif U = \dif W = -p\dif V. \]
  Comme c'est un gaz parfait, on a \eqref{eq:peru}, on obtient alors
  \begin{align*}
    C_V \dif T & = -p \dif V\\
    mc_{V}\frac{\dif T}{T} & = -mR^*\frac{\dif V}{V}\\
    c_{V} \ln\left(\frac{T_2}{T_1}\right) & =
    R^* \ln\left(\frac{V_1}{V_2}\right)
  \end{align*}
  En prenant l'exponentielle des deux termes, on arrive à
  \[ T_1V_1^{\frac{R^*}{c_{V}}} = T_2V_2^{\frac{R^*}{c_{V}}} \]
  Par la loi des gaz parfait, il vient
  \[ p_1V_1^{\frac{R^*}{c_{V}}+1} = p_2V_2^{\frac{R^*}{c_{V}}+1} \]
  On peut aussi simplifier l'exposant grâce à la relation de Mayer
  \[ \frac{R^*}{c_{V}}+1 = \frac{c_{p}-c_{V}}{c_{V}}+1 =
  \frac{c_{p}}{c_{V}} \eqdef \gamma \]
  et on obtient bien ce qu'il fallait obtenir
  \[ pV^{\gamma} = \constant \]
\end{proof}
La loi des gaz parfaits nous permet aussi d'écrire
\begin{center}
  \begin{tabular}{p{0.3\textwidth}p{0.3\textwidth}p{0.3\textwidth}}
    \centering$\frac{T^\gamma}{p^{\gamma-1}} = \constant$ &
    \centering$TV^{\gamma-1} = \constant$ &
    \centering$pV^\gamma = \constant$
  \end{tabular}
\end{center}

%TODO: slide 22

\subsection{Le cycle de Carnot}
\subsubsection{définition du cycle: le cycle moteur}
Le cycle de Carnot n'échange de la chaleur avec
l'environnement que lors de transformations isothermes.
Donc à chaque échange de chaleur, le système effectue
un travail de signe contraire qui compense exactement l'échange.
Nous allons étudier un cycle de Carnot composé de quatre transformations:
\begin{itemize}
  \item Une compression isotherme réversible $(1 \Rightarrow 2)$;
  \item Une compression adiabatique réversible $(2 \Rightarrow 3)$;
  \item Une détente isotherme réversible $( 3 \Rightarrow 4)$;
  \item Une détente adiabatique réversible $( 4 \Rightarrow 1)$.
\end{itemize}

Une détente a tendance à faire chuter la température,
il faut donc un apport de chaleur pour la maintenir à un niveau constant.
Cet apport se calcule facilement
\[ \Delta U = Q + W = U_4-U_3 = Q_{34} - \int_3^4 p\dif V \]
Comme la température est constante et que l'énergie interne n'est que
fonction de la température pour un gaz parfait, $\Delta U = 0$ et donc
\[ Q_{34} = \int_3^4 p\dif V = mR^*T_I \ln \left(\frac{V_4}{V_3} \right) \]
Puisque le fluide se détend,
l'intégrale est toujours positive et le fluide reçoit bien de la chaleur.

Lors de la compression isotherme, le fluide rejette de la chaleur pour
maintenir sa température constante telle que
\[ Q_{12} = mR^*T_{II} \ln \left(\frac{V_2}{V_1} \right) \]
Ici, le terme est toujours négatif car $V_2 < V_1$.

Lors des transformations adiabatiques,
il n'y a pas d'échange de chaleur et le travail total réalisé lors du cycle est:
\[ -W_{1234} = Q_{34} + Q_{12} \]
\[ |W_{1234}| = Q_{34}-|Q_{12}| \]
On définit le rendement thermique comme étant la fraction de l'énergie
thermique reçue à haute température qui a pu être convertie en travail.
\[ \eta = \frac {|W_{1234}|}{Q_{34}} = 1-\frac {|Q_{12}|}{Q_{34}} =
1-\frac{T_{II} \ln \left(\frac{V_1}{V_2}\right)}{T_I
\ln \left(\frac{V_4}{V_3} \right)} \]
On peut encore simplifier cette expression en considérant
les transformations adiabatiques réversibles:
\[ \left(\frac{V_3}{V_2}\right)^{\gamma -1} =
\frac {T_{II}}{T_I} = \left(\frac{V_4}{V_1} \right)^{\gamma -1} \]
Finalement, on obtient pour le rendement:
\[  \eta = 1 -\frac{T_{II}}{T_I} \]

\subsubsection{Le cycle récepteur}
On peut inverser le cycle de Carnot pour en faire
un cycle récepteur au cours duquel le travail est transformé en chaleur.
On prélève donc de la chaleur à basse température pour
la rendre à plus haute température grâce au travail reçu.
On a cette fois ci
\[ W_{1234} = |Q_{34}|-Q_{12} \]
Si on regarde en basse température, on parleras de cycle frigorifique et
si on regarde les hautes températures on verra une pompe à chaleur.
On peut définir des coefficients de performances ($COP$) pour ces machines.
Pour la machine frigorifique:
\[  COP = \frac {Q_{12}}{W_{1234}} = \frac{1}{\frac{T_I}{T_{II}} -1} \]
et pour la pompe à chaleur:
\[  COP = \frac{|Q_{34}|}{W_{1234}} = \frac {1} {1-\frac{T_I}{T_{II}}} \]

\section{Le premier principe pour les systèmes ouverts}
\subsection{Conservation de la masse}
Considérons un système ouvert, par ses frontières entrent et
sortent des débits de masses du fluide (en \kilogram\per\second)
\[ \frac{\dif m_\mathrm{syst}}{\dif t} = \sum_{i = 1}^{n} \dot m_i \]
où $n$ représente le nombre de frontières ouvertes entre
le système et l'environnement.

Si le fluide est incompressible (masse volumique constante)
\[ \frac{\dif V_\mathrm{syst}}{\dif t} = \sum_{i = 1}^{n}\dot V_i \]
en $\meter\cubed\per\second$ et pour un système indéformable cela revient à
\[  \sum_{i = 1}^n \dot V_i = 0 \]
Sans oublier, pour un régime permanent, que le bilan de masse est nulle
\[ \sum_{i = 1}^n \dot m_i = 0 \]

\subsection{Conservation de l'énergie mécanique}
La variation d'énergie cinétique du fluide vaut
la somme des travaux sur le système.
Il y en a de 3 types.
\begin{itemize}
  \item $w_i$, travaux des forces internes
    (i.e. conversion d'énergie interne en énergie mécanique);
  \item $w_e$, travaux des forces exernes de contact;
  \item $w_d$, travaux des forces à distance.
\end{itemize}
\[ \Delta k = w_i + w_e + w_d \]
où $\Delta k = \frac{c_2^2}{2}-\frac{c_1^2}{2}$ représente la variation
d'énergie cinétique d'un kilogramme de fluide parcourant
le système ($\joule\per\kilogram = \meter\squared\per\second\squared$).

\subsubsection{Travail des forces internes}
Si il est compressible, le fluide peut rencontrer des pression différentes et
engendrer un travail de compression ou de détente mais une partie
de ce travail va être dissipé au sein du fluide suite aux
frottements internes dus à la viscosité du fluide
\[  w_i = \int_1^2 p \dif v - w_f \]

\subsubsection{Travail des forces externes}
Le travail des forces externes est la résultante de plusieurs composantes.
Il peut y avoir un apport d'un organe externe $w_m$,
il est positif si il est reçu par le système (pompe, ventilateur,...) ou
négatif comme le cas d'une turbine.
Il y aussi les travaux d'introduction et d'extraction du fluide.
Lorsque le fluide  entre dans le système il subit une force exercée
par le fluide se trouvant en dehors du système.
Pour cela, il faut vaincre la force d'entrée
(pression d'entrée multipliée par la section de passage)
sur une distance qui correspond au passage d'un \kilogram.
\[ Fd = p_1A_1\frac{v_1}{A_1} = p_1v_1. \]
On a donc
\[ w_e = w_m + p_1v_1 - p_2v_2 = w_m - \Delta(pv). \]

\subsubsection{Travail des forces à distance}
Le travail des forces à distance correspond au déplacement du fluide
dans un champ de force, comme celui de la gravité.
\[  w_d = -g \Delta z. \]
Le signe négatif est du au fait que ce gain d'énergie potentielle
se fait au détriment de l'énergie cinétique.

\subsubsection{Ensemble des travaux}
De manière générale, on peut écrire que
\[ \Delta k + g\Delta z = w_e + w_i =
w_m + p_1v_1 - p_2v_2 + \int_1^2 p \dif v - w_f \]
La somme des travaux sur le fluide correspond
à la variation d'énergie cinétique et potentielle.
$w_f$ représente l'ensemble des dissipations
d'énergie mécanique subie par le fluide, ce terme est toujours positif.

On peut constater que
\[  p_1v_1-p_2v_2 = -\int_1^2 \dif\left(pv\right) =
  -\int_1^2 v\dif p - \int_1^2 p\dif v \]
Du coup, on peut écrire
\[ w_m = \int_1^2v\dif p + \Delta k + g\Delta z + w_f \]
Cette équation fondamentale est l'expression mécanique
du travail moteur pour un système ouvert en régime permanent.
Si le fluide recoit de l'énergie alors $w_m$ est positif,
dans le cas contraire il est négatif.
On parle de machine motrice si le fluide donne son énergie
pour actionner un organe moteur, $w_m$ est négatif.
Si le fluide reçoit de l'énergie, on parlera de machine réceptrice.

\subsubsection{Illustration}
Considérons une pompe qui aspire de l'eau à pression constante
pour la relâcher à pression plus élevée.
Dans ce cas, $\Delta k$ et $\Delta z$ sont faibles et l'expression devient
\[ w_m = \int_1^2v\dif p +w_f \]
De plus, comme l'eau est incompressible,
$v$ est constant et $\int_1^2 v\dif p = v \Delta p$, on obtient alors
\[ w_m = v \Delta p + w_f \]
On voit alors que le travail fournit par la pompe correspond
aux travaux d'admission et de refoulement du
liquide majorée d'une perte due aux travaux de frottements.
Si on retire la pompe ($w_m = 0$), on a $-\Delta p = \rho w_f$,
on voit que le passage d'un fluide dans le système le fait passer
d'une pression supérieur à une pression inférieur diminuant
sa capacité à effectuer un travail de sortie.
Cela est à nouveau du aux travaux de frottements.

Dans ces deux exemples on remarque bien
l'importance des travaux d'admissions et d'extractions du fluide

\subsubsection{Équation de Bernoulli}
Si on prend un travail moteur nul et
des dissipations par frottement également nulles,
on obtient l'équation dite de Bernoulli
\[ \int_1^2 v\dif p + \Delta k + g \Delta z = 0 \]
Si le fluide se comporte de manière incompressible (comme pour tout liquide),
on a que $\int_1^2 v\dif p = v \Delta p$ et on obtient
\[ p_1 + \rho \frac{c_1^2}{2}+\rho gz_1 =
  p_2 + \rho \frac{c_2^2}{2}+\rho gz_2 \]
Celle relation est riche de conséquence pratique, par exemple,
dans un tube horizontal,
une accélération du fluide lui fait diminuer sa pression
\[  p_2 - p_1 = \rho \left (\frac{c_1^2-c_2^2}{2}\right) \]
Si le fluide accélère,
son énergie cinétique augmente au détriment de sa capacité à
effectuer un travail de sortie, sa pression diminue donc.

\subsection{Le premier principe pour les systèmes ouverts}
Prenons maintenant également en compte l'énergie interne
du fluide à la section d'entrée 1 et à la section de sortie 2.
\[ u_2 + \frac{c_2^2}{2} +gz_2 =
u_1 +\frac{c_1^2}{2} + gz_1 + p_1v_1- p_2v_2 + w_m + q \]
L'énergie totale à la section d'entrée est égale à
l'énergie totale de la section de sortie majorée du travail d'entrée et
de sortie du fluide, du travail moteur et de la chaleur reçue $q$.
Les termes des travaux de frottements ont disparus car
ils correspondent à un transfert d'énergie sans affecter le bilan global.
Pour la même raison, le terme $\int p \dif v$ a également disparu.
Cet aspect n'intervenait pas dans le bilan d'énergie mécanique
car la chaleur était stockée sous forme d'énergie interne.

On constate que $u_i + p_iv_i$ est l'enthalpie du fluide.
\[ w_m = -q + \Delta h + \Delta k +g \Delta z \]
À présent, nous pouvons trouver une expression
du premier principe identique à celle développée pour les systèmes fermés.
Que la forme soit mécanique ou énergétique,
le travail moteur est identique donc
\[ -q + \Delta h + \Delta k + g \Delta z = w_m =
\int_1^2v\dif p + \Delta k + g\Delta z + w_f \]
En simplifiant et en mettant l'enthalpie en évidence:
\[ \Delta h = \int_1^2v\dif p + q + w_f \]
Il s'agit du premier principe exprimé pour un système ouvert.

\section{L'entropie}
\subsection{La notion de différentielle exacte}
\subsubsection{Différentielles exactes et inexactes}
La notion de variable d'état est liée à celle de différentielle exacte.
On s'intéresse aux évolutions infinitésimales des systèmes thermodynamiques et
donc à des variations infinitésimales des grandeurs utiles.
Dans la plupart des cas,
c'est le seul moyen de déduire des coefficients empiriques d'intérêt.
\[ \dif U = df = \left(\fpart{f(T,V)}{T}\right)_V \dif T +
\left ( \fpart{f(T,V)}{V}\right ) _T \dif V \]
À partir de cette expression de la différentielle d'une variable d'état,
on vérifie l'égalité entre les dérivées secondes croisées
\[ \frac{\partial^2 f(T,V)}{\partial T \partial V} =
\frac{\partial^2 f(T,V)}{\partial V \partial T} \]
Le fait que les dérivés croisées sont identiques indique
qu'elles proviennent bien d'une même fonction.
Cela est moins évident si on ne dispose que d'une expression déjà dérivé:
\[ f_1(T,V) \dif T = f_2(T,V)\dif V \]
Donc si l'égalité
\[ \fpart{f_1}{V} = \fpart{f_2}{T} \]
est vérifiée alors il existe une fonction telle que:
\[ \dif F = f_1(T,V) \dif T = f_2(T,V)\dif V  \]
Si on s'intéresse non plus à une transformation infinitésimale mais finie,
\[ \int_1^2 dF = F(2)-F(1) \]
On constate que la fonction $F$ répond à la définition de variable d'état
puisque l'intégrale correspond simplement à la différence
de la grandeur entre le point final et le point initial.
Une variable est toujours associée à une différentielle exacte et
réciproquement, à chaque différentielle exacte correspond une variable d'état.

Comme il existe des différentielles exactes,
on suppose qu'il existe aussi des différentielles inexactes;
\[ \delta f = y\dif x - x\dif y \]
Les dérivées croisées étant différentes,
la différentielles $\delta f$ est inexacte.
Cette différentielle ne peut devenir exact que si
on spécifie un chemin pou réaliser l'intégration, une relation entre $x$ et $y$.
Dans le cas où $x = y^3$ on obtient:
\[  y\dif(y^3)-y^3\dif y = 3y^3\dif y-y^3\dif y
= 2y^3\dif y = \dif\left(\frac{y^4}{2}\right) \]
qui est bien une différentielle exacte.

En appliquant cette démarche à la notion de travail,
on peut affirmer que l'expression différentielle du travail
dans un cas réversible, $-p\dif V$ n'est pas une différentielle exacte.
\[ -p\dif V+0\dif p \]
Les dérivées croisées sont différentes,
il n'y a donc pas de fonction dont la différentielle est l'expression ci dessus.
Par contre si connait le chemin, la différentielle devient exacte!
\subsubsection{Le cas du gaz parfait}
Prenons la pression qui est une variable d'état
\begin{align*}
  p & = \frac{mR^*T}{V}\\
  \dif p & = \frac{mR^*}{V}\dif T-\frac{mR^*T}{V^2}\dif V.
\end{align*}
Remarquons que les dérivées secondes sont identiques
\[ \fpart{}{V} \left(\frac{mR^*}{V}\right) =
-\frac{mR^*}{V^2} = \fpart{}{T}
\left(\frac{-mR^*T}{V^2}\right). \]

\subsubsection{Le facteur intégrant}
Il est parfois souhaitable d'associer à une différentielle
inexacte une différentielle exacte.
On peut le faire sauf si la différentielle inexacte est nulle.
Dans ce cas, on multiplie les membre de l'égalité par un facteur intégrant
pour obtenir une différentielle exacte.
\[ \delta f = y\dif x-x\dif y = 0. \]
En multipliant par $g(x,y) = x^{-2}$ on obtient une différentielle exacte:
\[ \frac y {x^2} \dif x -\frac 1x\dif y = 0. \]
\subsection{La notion d'entropie}
Partons du premier principe:
\[ \delta Q = \dif U - \delta W = \dif U +p\dif V =
\left(\fpart{U}{p}\right)_V \dif p +
\left ( \left ( \fpart{U}{V}\right)_p + p \right)\dif V \]
Le facteur intégrant est ici l'inverse de la température
$\frac{1}{T}$ et la différentielle exacte s'écrit:
\[ \dif S = \frac{\delta Q}{T} = \frac{\dif U(p,V)+p\dif V }T \]
On définit $S$ comme l'entropie du système.
À ce stade, une variation positive de l'entropie
indique un apport d'énergie sous forme de chaleur et
une variation négative indique une perte d'énergie sous forme de chaleur.

On peut vérifier que les deux dérivées croisées sont égales et
que l'entropie est une variable d'état.
On obtiendra aussi comme relation:
\[ T\left(\fpart{p}{T}\right)_V =
\left(\fpart{U}{V}\right)_T+p \]
On définit un nouveau coefficient $\pi_T$ comme étant:
\[ \pi_T = T\left(\fpart{p}{T}\right)_V -p =
\left(\fpart{U}{V}\right)_T \]
Pour un gaz parfait, ce coefficient est bien nul.

Essayons de montrer l'indépendance du chemin parcouru
pour $\int_1^2 \delta Q/T$.
Considérons une courbe adiabatique,
comme il n'y a pas de transfert de chaleur on peut écrire:
\[ \dif S = \frac{\delta Q}{T} = 0 \]
On appelle aussi cette courbe une isentropique.

Prenons une séquence de transformation,
d'abord une transformation isobare puis une isochore.
Le long de l'isobare on a, pour un gaz parfait:
\[ \frac{\delta Q}{T} = \frac{\dif U+p\dif V}{T} =
C_V \frac{\dif T}{T}+\frac pT\dif V =
C_V \frac{\dif T}T+mR^*\frac{\dif T}T = C_p\frac{\dif T}T \]
Ensuite en intégrant on obtient:
\[ \int_1^2 \frac{\delta Q}T = C_p \ln \left (\frac{T_{II}}{T_{I}}\right) \]
Pour l'isochore, il n'y a pas de travail d'expansion et on a alors:
\[ \int_2^3 \frac{\delta Q}T = C_V \ln \left (\frac{T_{III}}{T_{II}}\right) \]
La somme des deux vaut:
\[ \int_1^3 \frac{\delta Q}T = C_p \ln \left (\frac{T_{II}}{T_{I}}\right)+
C_V \ln \left (\frac{T_{III}}{T_{II}}\right) \]
D'autre part on a aussi les relations d'état des transformations
\begin{align*}
  \frac{T_{II}}{T_I} = \frac{V_2}{V_1} =
  \left(\frac{p_1}{p_2}\right)^{1/\gamma}\\
  \frac{T_{III}}{T_{II}} = \frac {p_2}{p_1}
\end{align*}
Finalement on a pour la variation d'entropie:
\[ \Delta S = S_2-S_1 = \int_1^3 \frac{\delta Q}T =
-\frac{C_p}{\gamma} \ln \left(\frac{p_2}{p_1}\right) +
C_V \ln \left (\frac{p_2}{p_1}\right) = 0 \]
La conclusion sera identique pour le chemin composé d'abord
d'une isochore puis d'une isobare.
Cela illustre bien que l'entropie est une variable d'état.

\subsubsection{Relations fondamentales pour l'entropie}
En reprenant la définition de l'entropie
\[ \dif S = \frac{\dif U(p,V)+p\dif V}{T} \]
on constate que le membre de gauche est une différentielle exacte
car l'intégrale ne dépend pas du chemin.
On en tire deux relations extrêmement utiles
pour les calcules de thermodynamique
\[ T\dif S = \dif U+p\dif V \]
\[ T\dif S = \dif H-V\dif p \]
Alors que nous n'avons aucune interprétation physique de l'entropie,
on est à même de calculer son évolution.
Dans le cas d'une transformation isotherme
d'un gaz parfait les expressions deviennent
\[ \dif S = \frac{p\dif V}{T} = -\frac{V\dif p}{T} \]
puisque l'énergie interne et l'enthalpie ne dépendent que de la température.
\[ \dif S = mR^*\frac{\dif V}{V} = -mR^*\frac{\dif p}p \]
\[ \Delta S = mR^* \ln \left(\frac{V_2}{V_1}\right) =
-mR^* \ln \left(\frac{p_2}{p_1}\right) \]

\subsubsection{L'entropie et les cycles}
Comme l'entropie est une variable d'état,
\[ \oint \dif S = \oint \frac{\delta Q}{T} = 0 \]
Prenons un cas simple d'un cycle où le système n'échange de la chaleur
avec l'extérieur que lors de deux transformations, le bilan d'énergie est:
\[ -W = Q_I+Q_{II} \]
$W$ est une valeur négative (donc -$W$ est positif)
car le cycle produit du travail.
Admettons que $Q_I$ soit positive et $Q_{II}$
soit négative car cédée par le système.
Le rendement est:
\[  \eta = \frac{-W}{Q_I} = 1+\frac{Q_{II}}{Q_I} \]
Ce résultat avait déjà été obtenu lors du cycle de Carnot,
ici le résultat est général.
Tout cycle réversible dont l'objectif est de produire du travail
à partir de chaleur aura un rendement toujours inférieur à l'unité.
Dans le cas du cycle de Carnot, comme les échanges se font
à température constante, par bilan d'entropie on a
\[ \frac{Q_I}{T_I} = - \frac{Q_{II}}{T_{II}} \]
et on obtient:
\[ \eta = 1 - \frac {T_{II}}{T_I} \]
Cette particularité ne s'applique qu'aux cycles qui
échangent de la chaleur à température constante.

\subsubsection{Les énoncés du second principe}
On introduit le second principe par deux énoncés dont l'équivalence est prouvée.

\begin{description}
  \item[L'énoncé de Kelvin est]
    Il est impossible de concevoir un système cyclique dont le seul résultat net
    serait de convertir totalement de la chaleur reçue en travail.
  \item[L'énonce de Clausius est]
    Il est impossible de concevoir un système cyclique dont
    le seul résultat net serait de transférer de la chaleur
    d'une source froide vers une source plus chaude sans un apport externe.
\end{description}
L'énoncé de Kelvin est une forme générale que le rendement de la conversion
de chaleur en travail est toujours inférieur à l'unité.
Ce qui est vrai si l'entropie est une variable d'état.

Si l'énoncé de Clausius était faux, il serait possible de créer
un cycle dont le seul objet serait l'échange de chaleur
entre une source froide et une source chaude.
Le bilan d'entropie serait
\[ \oint \dif S = \frac{Q_{II}}{T_{II}}+\frac{Q_I}{T_I} =
Q\left(\frac 1 {T_{II}} - \frac 1 {T_I}\right) \]
L'intégrale n'est nulle,
ce qui est nécessaire selon la définition de l'entropie,
que si aucune chaleur n'est échangée ou si les deux températures sont
identiques (ce qui revient au même).

\section{Le second principe}
\subsection{Les irréversibilités}
\subsubsection{Irréversibilités dans la production de travail}
Dans un premier temps, considérons un cycle similaire à
celui de Carnot mais qui produit un travail inférieur à la valeur maximale.
Supposons également que la quantité de chaleur est
identique dans les deux cas et équivalente à $Q_I$.
Pour le cycle de Carnot on a
\[ \frac {Q_I}{T_I} + \frac{Q_{II}}{T_{II}} = 0 \]
et pour le cycle irréversible
\[  \frac {Q_I}{T_I} + \frac {Q^{'}_{II}}{T_{II}} < 0 \]
car la chaleur rejetée à la source froide $Q^{'}_{II}$ sera plus importante.

Comme les échange de chaleur sont réversible, la relation
ci-dessus revient à dire que le bilan d'entropie du cycle est négatif.
Il faut donc introduire une autre source d'entropie que
nous appellerons entropie interne $\dif S_i$ de telle manière que:
\[ \oint \dif_eS +\dif_iS = 0 \]
Il est surprenant que l'entropie qui a été définie par association
au transfert de chaleur réversible intervienne dans le cas
d'un travail qui inclut des irréversibilités purement interne.
Prenons un exemple assez parlant,
un volume divisé en deux dont une partie est
occupée par du gaz parfait et l'autre est vide.
On enlève la séparation entre les deux parties,
le travail sera nul de même que la variation d'énergie interne
puisque la température sera constante.
\[ \dif S = \frac{\dif U+p\dif V}{T} = \frac{p\dif V}T = mR^*\frac{\dif V}V \]
En intégrant entre l'état initial et l'état final, on trouve:
\[ \Delta S = mR^* \ln 2 \]
L'entropie a donc augmenté sans apport de chaleur!
L'explication est que l'expansion est irréversible et que dans ce cas,
l'entropie peut évoluer positivement sans échange de chaleur.
Maintenant essayons de trouver une même évolution mais
en utilisant seulement des transformations réversibles.
D'abord une transformation adiabatique réversible jusqu'à
un volume double puis un réchauffement isochore.
La première transformation n'induite aucune
variation d'entropie puisqu'elle est réversible et adiabatique.
Pour la seconde
\[ \dif S = \frac{\dif U+ p\dif V}{T} = \frac{\dif U}{T} = mc_v\frac {\dif T}{T} \]
La température finale du problème est $T$ et la température initiale
sera celle obtenue après la détente réversible c'est à dire $T/2^{\gamma -1}$
\[ \Delta S = mc_v \ln \left(\frac T{T/2^{\gamma-1}}\right) =
mc_v \ln (2^{\gamma-1}) = mc_v(\gamma -1) \ln 2 = mR^*  \ln 2 \]
Donc pour aboutir au même résultat de manière réversible,
il serait nécessaire d'apporter de la chaleur au système,
ce qui est donc bien cohérent.

\subsubsection{Irréversibilités de transfert de chaleur}
Prenons maintenant un cycle réversible

\subsection{L'énoncé du second principe}
L'évolution d'entropie lors d'une transformation est
donc associée d'une part aux échanges de chaleur réversibles et
d'autres part aux irréversibilités
\[ \dif S = \dif_iS+\dif_eS \]
et la génération d'entropie interne est toujours positive
\[ \int \dif_iS \geq 0  \]
L'inégalité donne d'une certaine manière une direction
aux changements possibles.

\subsubsection{Le théorème de Carnot}
Le théorème de Carnot montre qu'entre les deux sources de chaleurs
d'un cycle de Carnot, un rendement de conversion  maximal de chaleur
en travail est obtenu lorsque le cycle est réversible.

Considérons un cycle réversible
\[ \dif_iS = 0 \]
L'entropie n'évoluera que lors d'échanges de chaleur
\[ \dif S = \dif_eS = \frac{\delta Q}T \]
et pour le cycle complet:
\[ d\oint S = \oint \frac {\delta Q}{T} = 0 \]
Comme seulement deux échanges sont considérés
\[ \frac{Q_I}{T_1}+\frac{Q_{II}}{T_2} = 0 \]
Ce résultat à déjà été obtenu et n'apporte aucune preuve
de ce que nous avançons.
Considérons à présent un cycle irréversible pour lequel $\dif_iS > 0$.
Dans ce cas, un calcul similaire donne:
\[ \frac{Q_I}{T_1}+\frac{Q_{II}}{T_2} < 0 \]
ce qui indique que la perte à la source froide sera
plus grande dans le cycle irréversible.
Le rendement sera donc moins bon.

\subsubsection{Le théorème de Clausius}
On ne travaillera pas toujours avec des échanges à température constante.
Clausius à généraliser ce qui est décrit dans les sections précédentes.
Il affirme que l'entropie d'échange sera négative pour le système et
positive pour l'environnement.
\[ \oint \frac{\delta Q}T \leq 0  \]
On généralise ce concept de bilan négatif pour
une transformation quelconque d'un cycle ou non.
\[ N = S_2-S_1-\int \frac{\delta Q}T \geq 0 \]

\subsubsection{Échelle Kelvin de température}
A présent, on peut redéfinir le rendement thermique
des cycles sous une forme plus simple.
Le théorème de Carnot nous dit que tous les cycles réversibles travaillant
entre les même sources de chaleur ont le même rendement.
\[ \eta_{th} = 1-\frac{|Q_C|}{Q_H} = 1-f(T_H,T_C) \]
Prenons deux cycles en cascade on a
\[ \frac{|Q_3|}{Q_1} = \frac{|Q_3|}{Q_2}\frac{|Q_2|}{Q_1}
\Rightarrow f(T_1,T_3) = \frac{f(T_2,T_3)}{f(T_2,T_1)} \]
Cette relation ne peut dépendre de $T_2$ puisqu'elle est arbitraire,
on en déduit
\[ f(T_1,T_3) = \frac{\Theta (T_3)}{\Theta (T_1)} \]
Autrement dit, le rendement dépend de la température
au travers de la fonction $\Theta$.
Nous considérons cette nouvelle fonction comme une
nouvelle échelle de température, l'échelle absolue de température
qui n'est d'autre que l'échelle de Kelvin.

\subsubsection{Retour sur les deux énoncés historiques}

\subsection{Le diagramme entropique}
\subsubsection{Équivalence des aires}
Le diagramme $(p,V)$ n'est pas adapté aux problèmes
où l'entropie joue un rôle important.
Une isentropique est une hyperbole dans ce type
de diagramme puisque $pV^{\gamma} = \constant$.
Il est donc difficile de déterminer graphiquement une
augmentation ou diminution d'entropie sur ces diagrammes.
On utilise un diagramme entropique $(T,S)$ avec l'entropie en abcisse.
Un cycle reste un cycle et l'aire couverte par le cycle représente
également le travail accompli par cycle,
c'est la notion d'équivalence des aires.

Pour un cycle nous avons équivalence entre la chaleur
apportée et le travail fourni
\[ Q_\mathrm{cycle} = -W_\mathrm{cycle} \]
et comme
\[ W = -\int p\dif V \hspace{1.5cm} |Q| = \oint T\dif S \]
Les aires sont bien identiques et correspondent au travail fourni.

\subsubsection{Transformations particulière}
Dans le diagramme $(T,S)$ on peut essayer de trouver
la formes des isobares et des isochores.
Pour une isobare
\[ T\dif S = \dif H = C_p\dif T \]
pour un gaz parfait
\[ T = K \exp \left(\frac S{C_p}\right) \]
où $K$ est fonction de la pression.
Pour une isochore on obtient
\[ T = K \exp \left(\frac S{C_V}\right) \]
où $K$ est fonction du volume.

\section{Résumé}
\subsection{Systèmes fermés}
\begin{align*}
  U & = Q + W\\
  W & = -\int_1^2 p \dif V\\
  \oint U & = 0
\end{align*}

Soit $H$, l'entalpie
\begin{align*}
  \dif H & = \delta Q_{p = \constant}\\
  H & = U + pV
\end{align*}

Soit $C_p$ la capacité calorifique à pression constante
\begin{align*}
  C_p & \eqdef \left(\fpart{H}{T}\right)_p\\
  H & = C_p T % constant pressure
\end{align*}

Pour un gaz parfait
\[ C_p - C_V = nR \]

Pour un gaz quelconque
\[ C_p - C_V = \alpha V (\pi_T - p) \]
où $\alpha = \frac{1}{V} \left(\fpart{V}{T}\right)_p$
est le coefficient de dilatation thermique.

Pour un gaz parfait
\[ \Delta U = C_V \Delta T \]

Pour un gaz quelconque
\[ \Delta U = C_V \Delta T + \pi_T \Delta V \]


Pour une transformation adiabatique réversible d'un gaz parfait
\[ pV^\gamma = \constant \]
où $\gamma = \frac{c_p}{c_v} = \frac{c_v+R^*}{c_v} = \frac{\lambda+2}{\lambda}$
où $\lambda$ est le nombre de degrés de liberté.

Pour une transformation isothermique réversible d'un gaz parfait
\begin{align*}
  pv & = \constant\\
  \Delta u & = c_V \Delta T = 0\\
  w & = R^* T \ln\left(\frac{v_1}{v_2}\right)\\
  q & = -w
\end{align*}

\paragraph{Cycle}
\[ \oint \dif u = 0 \]
Pour un cycle moteur,
\[ \oint \dif w > 0 \]
Pour un cycle récepteur,
\[ \oint \dif w < 0 \]

Le rendement d'un cycle est le suivant
\[ \eta = \frac{Q_\mathrm{fourni} - Q_\mathrm{rejeté}}{Q_\mathrm{fourni}} \]
Pour le cycle de Carnot,
\begin{itemize}
  \item Pour une machine motrice,
    \[ \eta = 1 - \frac{T_{II}}{T_I} \]
  \item Pour un machine réceptrice frigorifique,
    \[ COP = \frac{1}{\frac{T_I}{T_{II}}-1} \]
  \item Pour une machine réceptrice pompe à chaleur,
    \[ COP = \frac{1}{1-\frac{T_I}{T_{II}}} \]
\end{itemize}

\subsection{Systèmes ouverts}
\[ w_m = \int_1^2v\dif p + \Delta k + g\Delta z + w_f \]

Pour un régime permanent (débit 1 = débit 2),
\[ A_1 c_1 = A_2 c_2 \]

La pression dynamique
\[ p_\mathrm{dyn} = \rho \frac{c^2}{2} \]

Des relations évidentes mais récurrentes
\begin{align*}
  \dot{m} & = \dot{V} \rho\\
  \dot{V} & = A c\\
  P & = w \dot{m}
\end{align*}

Premier principe exprimé pour un système ouvert
\[ \Delta h = \int_1^2 v \dif p + q + w_f \]

\part{Équilibres chimiques}
% +-----------------------------------------------------------------+
% | L'équilibre, l'équilibre chimique et les piles électrochimiques |
% +-----------------------------------------------------------------+
\section{Evolution spontanée et entropie}
Cette premier section a pour but de rappeler des notions déjà vues nécessaires
à l'établissement de la matière avenir.

\subsection{Transformations et réversibilités}
Dans un état d'équilibre, il n'y a pas de transformation spontanée du système
en l'absence d'action du milieu extérieur.

Une transformation \emph{réversible} est
constitué d'un continuum d'état d'équilibre
successifs, en pratique c'est une transformation fictive, irréalisable.
Par opposition, une transformation \emph{irréversible} est spontanée et
se déroule dans une certaine direction.

Une transformation est dite \emph{élémentaire} lorsque les variables d'états
du système ne subissent qu'un changement infinitésimal
au cours de la transformation.
Dans le cas contraire, elle est dite \emph{finie}.
Le changement d'une variable d'état d'un point à un autre est
indépendant du chemin:
$\Delta X = X_2 - X_1$.

La variation d'une grandeur d'état extensive est toujours
la somme d'un terme de transfert et d'un terme de création interne.
En général, pour une transformation élémentaire: $dX = \delta_eX+\delta_iX$
\subsection{Entropie d'un système}
\begin{itemize}
  \item L'entropie est une fonction d'état, pour une transformation élémentaire:
    $\dif S = \delta_eS+\delta_iS$ % not dif ?
  \item Pour un système fermé de température uniforme égale à $T$,
    on a
    \[ \delta_eS = \frac{\delta Q}T \]
  \item L'entropie créée au sein du système au cours d'une transformation est
    toujours positive ou nulle: $\delta_iS \geq 0$.
    L'entropie créée est \emph{nulle} pour une transformation \emph{réversible}
    et \emph{positive} pour une \emph{irréversible}.
\end{itemize}
En général on calcule l'entropie transférée en intégrant
le long du chemin de transformation mais il est très difficile de
calculer l'entropie créée au cours d'un processus irréversible.
Mais il est souvent possible d'imaginer une suite de processus réversible
entre deux états et trouver la différence entre ces deux états.

Comme l'entropie évolue toujours positivement dans un système isolé,
on peut dire qu'un système isolé atteint son équilibre
quand son entropie atteint son maximum.

À pression constante, la variation d'entropie est facile à calculer
\[ \dif S = \left(\frac{\delta Q_\mathrm{rév}}{T}\right)_p =
\left(\frac{\dif H}{T}\right)_p = C_p\frac{\dif T}{T} \]
Pour un changement d'état (solide à liquide)
\[ \Delta_\mathrm{tr}S = \frac{Q_\mathrm{rév}}{T_\mathrm{tr}}
= \frac{\Delta_\mathrm{tr}H}{T_\mathrm{tr}} \]
Lors d'une transformation endothermique, il y a augmentation d'entropie.

\subsection{Théorème de Nernst}
L'entropie de n'importe quelle substance peut être prise
comme égale à zéro à \si{0}{\kelvin}.
C'est le ``troisième principe'' de la thermodynamique.
Comme nous ne savons que calculer les variations d'entropie,
définir ce zéro absolu permet de calculer toute les valeurs d'entropie.

\section{Les potentiels thermodynamiques}
Commençons par définir deux autres fonctions d'états:

L'énergie libre de Helmholtz
\[ A = U - TS \]

L'enthalpie libre de Gibbs
\[ G = H - TS \]

\subsection{Conditions d'équilibre basée sur $G$}
Prenons un système fermé faisant une transformation isotherme et isobare
\begin{align*}
  \dif G & = \dif U+p\dif V+V\dif p-T\dif S-S\dif T\\
  & = \dif U+p\dif V-T\dif S.
\end{align*}
Comme
\begin{align*}
  \dif U & = \delta Q + \delta W\\
  \delta Q & \leq T\dif S.
\end{align*}
On a
\[ \dif G \leq \delta W + p\dif V. \]
Mais rappelons nous que $p\dif V$ est le travail fourni à l'\emph{environment}
et non au \emph{système}.
Dès lors, soit $\delta W'$ le travail ``hors $p \dif V$'', on a
\[ \delta W = \delta W' - p \dif V \]
donc l'inégalité devient
\[ \dif G \leq \delta W'. \]
En conséquence, le travail ``hors $p\dif V$'' maximum qui peut être fourni
par un système fermé au cours d'un processus isotherme isobare est égal
à la diminution de son enthalpie libre de Gibbs.
Pour cela, il faut que le processus soit réversible sinon une partie
du travail sera perdu sous forme de production interne d'entropie.

À pression et température constante et si le seul travail effectué par
ou sur le système est un travail de type $p\dif V$, on a
\[ \dif G \leq 0 \]
donc un système fermé évolue spontanément dans le sens
d'une diminution de son énergie libre de Gibbs.
Donc l'équilibre est atteint lorsque G atteint son minimum.

\subsection{Équations fondamentales d'un système de composition fixée}
\begin{align}
  \nonumber
  \dif H & = T\dif S+V\dif p\\
  \nonumber
  \dif A & = -S\dif T -p\dif V\\
  \label{eq:fondG}
  \dif G & = -S\dif T +V\dif p\\
  \nonumber
  \dif U & = T\dif S-p\dif V
\end{align}
Comme la pression et la température sont les plus faciles à contrôler,
l'enthalpie libre de Gibbs est importante dans les applications.

\subsection{Variation de l'enthalpie libre avec la pression}
A température constante
\[ \dif G = -S\dif T+V\dif p = V\dif p \]
En intégrant on obtient
\[ G(p_2,T) = G(p_1,T)+\int_{p_i}^{p_2}V(p,T)\dif p \]
Si on prend comme pression initiale 1 bar et pour une mole de ce corps
\[ G_m(p,T) = G_m\std(T)+\int_{p\std}^{p}V_m(p',T)\dif p' \]
Le symbole $\std$ est utilisé pour désigner une grandeur à l'état standard,
c'est à dire pour un corps pur à pression standard $p\std = \si{1}{bar}$.

Pour un gaz, la loi des gaz parfaits nous permet de simplifier l'expression
comme suit
\[ G_m(p,T) = G_m\std(T)+RT \ln \frac{p}{p\std} \]
L'augmentation de $G$ avec $p$ est donc le reflet de la diminution de $S$
avec l'augmentation de $p$ en conditions isothermes.

Pour un solide ou un liquide, à $T$ fixé,
comme le volume ne varie que très peu avec la pression, on a
\[ G_m(p,T) = G_m\std(T)+\int_{p\std}^{p}V_m(p',T)\dif p'
\approx G_m\std(T)+V_m(p-p\std) \]
Si une mole de solide ou de liquide coexiste avec une mole de gaz
($V_{m(\mathrm{gaz})} p\std = RT$ par la loi des gaz parfaits),
\[ G_m(p,T) - G_m\std(T) = V_{m(\mathrm{sol})}(p-p\std)+
V_{m(\mathrm{gaz})}\std p\std\ln \frac{p}{p\std}
\approx RT \ln \frac{p}{p\std} \]
Lorsqu'un système contient un gaz et une phase condensée
on peut souvent négliger la variation de $G(p)$ pour cette phase condensée
et considérer que seul le gaz contribue à la variation de $G$ avec $p$.
$V_m$ d'un solide ou d'un liquide varie beaucoup
moins avec la pression que pour un gaz.
On peut faire l'approximation qu'il ne dépend pas de $p$.

\section{Le potentiel chimique}
\subsection{Systèmes composés d'espèces chimiques différentes}
Intéressons nous maintenant a des systèmes dont la composition peut changer.
Soit par flux entrant dans un système ouvert
soit par réaction chimique dans un système fermé, ...
L'équation fondamentale~\eqref{eq:fondG} devient
\[ \dif G = \left(\fpart{G}{T}\right)_{p,n_i}\dif T +
\left(\fpart{G}{p}\right)_{T,n_i}\dif p +
\sum_i \left(\fpart{G}{n_i}\right)_{T,p,n_{j\neq i}}\dif n_i \]
On peut faire pareil avec l'énergie interne et trouver que
\[ \left(\fpart{G}{n_i}\right)_{T,p,n_{j\neq i}} =
\left(\fpart{U}{n_i}\right)_{S,V,n_{j\neq i}} \]
En général on démontre que
\[ \left(\fpart{G}{n_i}\right)_{T,p,n_{j\neq i}} =
\left(\fpart{U}{n_i}\right)_{S,V,n_{j\neq i}} =
\left(\fpart{H}{n_i}\right)_{S,p,n_{j\neq i}} =
\left(\fpart{A}{n_i}\right)_{T,V,n_{j\neq i}} \eqdef \mu_i \]

\subsection{Potentiel chimique}
Et on définit ce $\mu_i$ comme étant le potentiel chimique de l'espèce $i$,
c'est une grandeur intensive.
L'équation fondamentale de $G$ dans un système à composants multiples
\[ \dif G = -S\dif T+V\dif p+\sum \mu_i\dif n_i \]
On utilisera $i$ en indice quand l'espèce sera présente dans un mélange et
en exposant quand il est présent sous forme pur.

On définit aussi
\[ G_m^A \eqdef \mu^A \] % FIXME: so weird
Pour un gaz parfait pur, on a alors
\[ \mu^A = {G_m^A}\std+RT \ln \frac{p}{p\std}. \]
%\[ \mu^i = {\mu^i}\std+RT \ln \frac{p}{p\std}. \]
%\[ \mu_i = {\mu_i}\std+RT \ln \frac{p_i}{p_i\std}. \]
% TODO remove all ^A _A and G_m^A if not used later

Par la suite on aura besoin lors de mélange de la fraction molaire de
l'espèce $A$ qu'on définit comme suit
\[ x_A \eqdef \frac{n_A}{\sum_in_i}. \] % FIXME: it is maybe misplaced

On remarque que si on a une espèce $A$ séparé en deux parties par une membrane
avec un flux $\dif n_{A1} = \dif n_{A2}$ passant par la membrane,
on a
\[ \dif G = \dif G_1 + \dif G_2 = (\mu_{A1} - \mu_{A2})\dif n_{A1}. \]
Comme $\dif G \leq 0$, $\dif n_{A1}$ et $(\mu_{A1} - \mu_{A2})$ sont
de signe contraire.
Dès lors,
tout espèce migre vers l'endroit où son potentiel chimique est le plus faible:
le gradient de potentiel chimique est le moteur du déplacement des espèces.

Une condition d'équilibre est l'uniformité partout dans le système,
des potentiels chimiques $\mu_i$ de chaque espèce.

\section{Enthalpie libre standard de réaction et l'affinité chimique}
\subsection{Enthalpie libre standard}
L'enthalpie standard de réaction est la différence entre
l'enthalpie des produits et l'enthalpie des réactifs:
\[ Q_r\std = \Delta_rH\std =
\sum_{K'}\nu_{K'}{H^{K'}}\std-\sum_K\nu_K {H_m^K}\std \]
Avec $\nu_K$ et $\nu_{K'}$ les coefficients stœchiométriques
des réactifs et des produits.
Comme dans le cas de l'entropie,
l'enthalpie ne peut être connue qu'à une constante près.
On définit donc l'enthalpie standard de formation
à \si{298}{\kelvin}
d'un corps simple sous forme la plus stable à zéro.

On calcule de la même manière
l'enthalpie libre de Gibbs standard de formation par
\[ \Delta_rG\std = \sum_{K'}\nu_{K'}{G_m^{K'}}\std-\sum_K\nu_K {G_m^K}\std \]
et
\[ {G^A}\std = {H^A}\std-T{S^A}\std \]
Ne pas confondre avec l'enthalpie libre de Gibbs standard de réaction
\[ \Delta_rG\std = \Delta_rH\std-T\Delta_rS\std \]

\subsection{Avancement d'une réaction}
Avancement élémentaire $\dif\xi$ d'une réaction
\[ \dif\xi = -\frac{\dif m_K}{\nu_Km_{mK}} = -\frac{\dif n_K}{\nu_K} =
\frac{\dif m_{K'}}{\nu_{K'}m_{mK'}} = \frac{\dif n_{K'}}{\nu_{K'}} \]
L'unité de $\dif\xi$ est la mole et il est identique
à tout instant pour tous les constituants du système.
Il y a une seule variable de composition du système.

\subsection{Affinité chimique}
Pour un système réactionnel fermé
\[ \sum_i\mu_i\dif n_i = \sum_K\mu_K\dif n_K+\sum_{K'}\mu_{K'}\dif n_{K'} =
\left(-\sum_K\nu_K\mu_K+\sum_{K'}\nu_{K'}\mu_{K'}\right)\dif\xi =
-\mathcal{A}\dif\xi \]
et donc
\[ \dif G = V\dif p-S\dif T +\sum_i\mu_i\dif n_i
= V\dif p+S\dif T-\mathcal{A}\dif\xi \leq 0 \]
avec $\mathcal{A} = -\left(\fpart{G}{\xi}\right)_{T,p} =
\sum_K\nu_K\mu_K-\sum_{K'}\nu_{K'}\mu_{K'} = -\Delta_r \mu$
l'affinité chimique du système
(ne pas confondre avec l'énergie de Helmotz $A$).

Une réaction chimique n'atteint son équilibre que lorsque
l'affinité chimique du mélange réactionnel devint égale a zéro.

\section{L'équilibre chimique entre gaz parfaits}
\subsection{Loi de Dalton et pressions partielles}
La pression exercée par un mélange idéal de gaz parfaits est égale à la somme,
des pressions $p_i$ que les gaz individuels
$i$ exerceraient s'ils étaient seuls présents dans le volume occupé.

Par mélange idéal,
nous entendons un mélange dans lequel les forces qui lient chaque espèce
à ses voisines ne changent pas lorsque la composition du mélange change.
$p_i$ est la pression partielle du gaz $i$ dans un mélange et vaut
$p_i = \frac{n_iRT}{V}$ et donc si on définit la fraction molaire
du constituant $i$ par $x_i = \frac{n_i}{n}$ alors $p_i = x_ip$.

\subsection{Variations des potentiels chimiques $\mu_i$
en fonction des pressions partielles $p_i$}
\[ \mu_i = {G_m^i}\std+RT \ln \frac{p_i}{p\std} \]
On peut remarquer que $RT \ln \frac{p_i}{p\std} =
RT \left(\ln \frac{p_i}{p} + \ln \frac{p}{p\std}\right)$
où le premier terme est la contribution au mélange et
le deuxième est du au changement de pression.

\subsection{Affinité chimique d'une réaction entre gaz parfaits}
Si $K$ et $K'$ sont des gaz parfaits,
\begin{equation}
  \label{eq:AGRTln}
  \left(\frac {\partial G}{\partial \xi}\right)_{T,p} =
  \Delta_rG\std+RT\left(\sum_{K'}\nu_{K'} \ln \frac{p_{K'}}{p\std}-
  \sum_{K}\nu_{K} \ln \frac{p_{K}}{p\std}\right)
\end{equation}

Le premier terme est la contribution des constituants purs
dans leur état standard et le deuxième terme est
la contribution des changement de pression et de mélange.

La réaction chimique atteint son équilibre lorsque le milieu réactionnel
a atteint une composition pour laquelle, en cas de changement élémentaire
de composition $\dif\xi$, la contribution à $\dif G$ due au changement
d'entropie du mélange est opposée et égal en valeur absolue
à la contribution due à l'enthalpie libre standard de la réaction.

\subsection{Quotient réactionnel et constante d'équilibre}
On définit le quotient réactionnel par
\[ Q_p = \frac {\prod_{K'}\left(\frac{p_{K'}}{p\std}\right)^{\nu_{K'}}}
{\prod_{K}\left(\frac{p_{K}}{p\std}\right)^{\nu_{K}}} \]
L'indice $p$ signifie qu'on calcule ce quotient
à partie de pressions partielles des produits et réactifs.

À l'équilibre chimique, lorsque la fonction $G(\xi)$ atteint un minimum,
le quotient réactionnel $Q_p$ est appelé la constante d'équilibre $K_p$.
Lorsque $Q_p < K_p$, la réaction se déplace spontanément
vers la droite jusqu'a atteindre l'équilibre.

À cet équilibre, $\mathcal{A} = 0$ donc \eqref{eq:AGRTln} se simplifie
et donne
\[ \Delta_rG\std = -RT\ln K_p \]
Étant donné que $\Delta_rG\std$ ne dépend que la température et
non de la pression, c'est pareil pour $K_p$.

Il est parfois plus simple d'écrire la constante d'équilibre $K_x$
en fonction à partir des $x_i$ ou $K_n$ à partir des $n_i$,
la relation entre les trois est
\[ K_p = K_x\left(\frac{p}{p\std}\right)^{\Delta \nu} =
K_n \left(\frac{p}{p\std\sum_i n_i}\right)^{\Delta \nu} \]
où $\Delta \nu = \sum_{K'}\nu_{K'}-\sum_K \nu_K$ est la différence entre
le nombre de moles de produits et le nombre de moles des réactifs.

\subsection{Évolution de $K_p$ avec la température}
Les processus endothermiques sont favorisées
lorsque $T$ augmente et inversement.
\[ \fdif{(\ln K_p)}{T} = \frac{\Delta_rH\std}{RT^2} \]
c'est l'équation de van't Hoff.
Si $\Delta_rH\std$ est constant alors l'intégration donne
\[ \ln K_{p,T_2}-\ln K_{p,T_1} = -\frac{\Delta_rH\std}{R}\left(\frac{1}{T_2}-
\frac{1}{T_1}\right) \]

Par contre $K_p$ ne dépend pas de la pression!

Comme lorsque des solides ou des liquides sont en présence de gaz on peut
considérer que seuls les gaz contribuent à la variation de $G$ avec $p$,
c'est pareil pour $K_p$.
La présence de phase gazeuse permet de négliger
les phases condensées (liquides ou solides).

On peut aussi calculer
\[ \left(\fdif{(\ln K_x)}{p}\right)_T = -\frac{\Delta\nu}{p} \]
qui nous dit que si $\Delta\nu_\mathrm{gaz} > 0$, une augmentation
de pression provoque une diminution de $K_x$, c'est le principe de Le
Châtelier.

\section{L'équilibre chimique en solution liquide}
\subsection{Potentiel chimique}
L'expression générale du potentiel chimique du constituant $A$ d'un système est
\[ \mu_A = {\mu^A}\std+RT \ln a_A \]
Avec $a_A$ l'activité du constituant $A$
(dans un gaz parfait: $a_A = p_A/p\std$).

La concentration d'une espèce en solution diluée affecte son potentiel
chimique de la même manière que la pression partielle
d'un constituant dans un mélange de gaz parfaits.
Dans ce cas, $a_A$ devient $m_A/m_A\std$,
l'état standard étant à une mole de $A$ par \kilogram{} de solvant.
Comme \si{1}{\liter} d'eau pèse \si{1}{\kilogram},
on a $m_A/m_A\std \approx [A] / \si{1}{M}$ où $[A]$ est la concentration
de $A$ dans la solution.

\subsection{Produit de solubilité $K_s$}
Le produit de solubilité $K_s$ est la constante d'équilibre
de la réaction entre un solide immergé dans un liquide et
ses ions en solution dans le liquide.
\[ K_s = \frac{[\ce{A^+}]_\mathrm{sat}[\ce{B^-}]_\mathrm{sat}}{[\ce{AB}]}
\approx [\ce{A^+}]_\mathrm{sat}[\ce{B^-}]_\mathrm{sat}. \]
Le solide est supposé rester pur et n'apparait pas dans la constante d'équilibre.
Si la solution saturée est suffisamment diluée,
$K_s$ ne dépend pas de la présence éventuelle d'autres ions dans la solution
(sauf si ce sont aussi des $\ce{A+}$ ou des $\ce{B-}$).

\subsection{Solubilité}
La solubilité $s$ d'une substance est la quantité dissoute
dans la solution saturée en moles/litres.
On peut facilement lier $s$ et $K_s$ pour n'importe quelle réaction.
Par exemple, pour une réaction quelconque
\[ \ce{A}_x\ce{B}_y\ce{(s)} \ce{->}
x\ce{A}^{y+}\ce{(aq)} + y\ce{B}^{x-}\ce{(aq)} \]
on trouve $K_s = [\ce{A}^{y+}]^x_\mathrm{sat}
[\ce{B}^{x-}]^y_\mathrm{sat} = (xs)^x(ys)^y = x^xy^ys^{x+y}$ donc
$s = \sqrt[x+y]{\frac{K_s}{x^xy^y}}$.
En pratique il est plus utile de connaitre la solubilité
d'une substance plutôt que son produit de solubilité.

\subsection{Précipitation}
On peut prédire l'évolution d'une réaction grâce à son coefficient réactionnel
$Q_s$ en le comparant avec $K_s$.
La précipitation est la réaction inverse à la mise en solution.
Une substance précipite (devient solide, réaction déplacé vers la gauche)
lorsque $Q_s > K_s$.

\subsection{Effet d'ions communs}
Comme vu plus haut, les autres ions en solutions n'interviennent pas
seulement s'ils sont différent de $\ce{A}$ et $\ce{B}$.
Que se passerait-il si on avait déjà
une concentration $c$ de $\ce{A}$ en solution ?
On aurait $K_s = [\ce{A}^{y+}]^x_\mathrm{sat}
[\ce{B}^{x-}]^y_\mathrm{sat} = (xs+c)^x(ys)^y$.
Du coup, $s$ diminue.
Pour calculer $s$, on peut souvent éviter un calcul fastidieux
et considérer que $c \gg xs$.
Dès lors, on a
$K_s \approx c^xy^ys^y$ et donc
$s \approx \sqrt[y]{\frac{K_s}{c^xy^y}}$.

Souvent, on ajoutera un composé beaucoup plus soluble contenant
\ce{A} pour que l'ajout de \ce{AB} ne le fasse pas précipiter
et qu'on puisse bien utiliser la formule trouvée plus haut.

\subsection{Effet du $\ph$ sur la solubilité}
Supposons que $\ce{A}$ soit une acide ou une base.
la solubilité $s$ va changer car une partie du $\ce{A}$ va se transformer
et ne va donc plus faire partie du produit de solubilité.

Par exemple, si on veut dissoudre du \ce{ClOMg}.
Il est important de noter que même si \ce{ClO-} est une base,
on appelle ça un \emph{sel d'acide faible}.
Comme le \ce{ClO-} est une base,
il peut faire la réaction
\( \ce{ClO- + H3O+ -> ClOH + H2O}. \)
Sa concentration respecte donc
\[ K_b = \frac{[\ce{ClOH}]}{[\ce{ClO-}][\ce{H3O+}]} =
\frac{s-[\ce{ClO-}]}{[\ce{ClO-}][\ce{H3O+}]} =
\frac{\frac{s}{[\ce{ClO-}]}-1}{[\ce{H3O+}]}. \]
Mais on a aussi
\[ K_s = [\ce{ClO-}][\ce{Mg}] = [\ce{ClO-}]s \]
donc
\[ K_b = \frac{\frac{s^2}{K_s}-1}{[\ce{H3O+}]} \]
d'où
\[ s = \sqrt{K_s\left(1+K_b[\ce{H3O+}]\right)} =
\sqrt{K_s\left(1+\frac{[\ce{H3O+}]}{K_a}\right)}. \]
Si on considère que $\ce{ClO-}$ ne change pas le $\ph$ qu'il y avait,
on a $[\ce{H3O+}] = 10^{-\ph}$.

L'avantage ici, c'est que les différents composés n'ont pas tous la
solubilité qui chutent au même $\ph$ donc on peut faire
des précipitations sélectives.

\section{Les réactions d'oxydo-réduction}
\subsection{Degré, étage, nombre d'oxydation}
Le degré d'oxydation est une charge ionique réelle ou fictive attribuée
aux éléments constitutifs d'une substance chimique.
La somme des degrés d'oxydations des éléments constituant une espèce
doit être égale à la charge portée par l'espèce.
Pour connaitre le degré d'oxydation de n'importe quelle substance,
il y a des règles:
Le degré d'oxydation est
\begin{itemize}
  \item $0$ pour un élément pur (e.g. \ce{Cu} ou \ce{O} dans \ce{O_2});
  \item la charge de l'ion pour un ion monoatomique;
  \item +1 pour un Alcalin;
  \item $-1$ pour le Fluor;
  \item +2 pour les Alcalino-terreux ainsi que pour \ce{Zn} et \ce{Cd};
  \item $+1$ pour l'hydrogène lié à un non métal;
  \item $-1$ pour l'hydrogène lié à un métal;
  \item $-2$ pour l'oxygène sauf dans les peroxydes
    où c'est $-1$ (e.g. \ce{H_2O_2}).
\end{itemize}

Les règles ont été écrites avec une priorité décroissante.
Par exemple, dans \ce{H2}, c'est la première règle qui prime et
\ce{H} a un degré d'oxydation de 0.

\subsection{Réaction d'oxydo-réduction ``rédox''}
Une réaction d'oxydo-réduction est un réaction au cours de laquelle
le degré d'oxydation d'au moins un élément augmente tandis que celui
d'au moins un autre élément diminue.

Quelques définitions:

\begin{itemize}
  \item L'oxydation est la perte d'un ou plusieurs électrons,
    le degré d'oxydation augmente;
  \item La réduction est le gain d'un ou plusieurs électrons,
    le degré d'oxydation diminue;
  \item L'oxydant est un accepteur d'électrons,
    en réagissant l'oxydant prend un électron et se réduit;
  \item Le réducteur est un donneur d'électrons,
    en réagissant le réducteur perd un électron et s'oxyde.
\end{itemize}

Les réactions rédox peuvent être composées de deux demi-réactions.

$\ce{A_{ox}}+ze^- \ce{<-> A_{red}}$
est une demi-réaction de réduction.

$\ce{B_{red} <-> B_{ox} +}ze^-$
est une demi-réaction d'oxydation.

$\ce{A_{ox} + B_{red} <-> A_{red} + B_{ox}}$
est la réaction globale.

Les couples $\ce{A_{ox}}/\ce{A_{red}}$ sont appelés des couples rédox.
La somme des deux demi-réactions fait la réaction globale.
L'équilibre sera déplacé vers la droite si $\ce{A_{ox}}$ est
plus oxydant que $\ce{B_{ox}}$ et inversement.

\section{Les piles électrochimiques}
Une réaction électrochimique est une séparation de deux demi-réactions rédox.

Prenons deux solutions de \ce{ZnSO4} et de \ce{CuSO4},
trempons y une électrode de Zinc et de Cuivre.
Dans le premier cas, le zinc tend à s'oxyder,
il y aura excès d'électron sur l'électrode et une grande présence de
$\ce{Zn^{2+}}$
dans la solution et donc un déficit de \ce{SO4^{2-}}.
L'électrode se trouve alors à un potentiel électrique négatif
par rapport à la solution loin de l'électrode.
Dans l'autre solution, le cuivre tend à se réduire,
il y a alors un manque en électron sur l'électrode et
un excès de \ce{SO4^{2-}} dans la solution.
L'électrode se trouve à un potentiel positif
par rapport à la solution loin de l'électrode.

\subsection{Pile électrochimique}
Reprenons les deux solutions précédemment décrites et mettons les en contact
avec une membrane semi-perméable de telle sorte que
seul le passage des \ce{SO4-} soit permis.
Les ions \ce{SO4-} ne seront plus présent en excès et
en déficit de part et d'autre des solutions.
Il y aura une différence de potentiel entre les deux électrodes.
Si on relie les deux électrodes,
il y peut y avoir un transfert d'électrons et donc un courant.
La cellule électrochimique est un générateur de courant, une pile.
Elle permet de produire du travail électrique directement
à partir d'une réaction chimique.

\subsection{Anode et Cathode}
\begin{itemize}
  \item On appelle anode l'électrode où
    se déroule la demi-réaction d'oxydation;
  \item On appelle cathode l'électrode où
    e déroule la demi-réaction de réduction;
  \item Pour une pile qui fournis de l'énergie électrique,
    on dessine la cathode à droite et l'anode à gauche;
  \item Pour une batterie qui se recharge,
    on dessine l'anode à droite et la cathode à gauche.
\end{itemize}
En fait, on dessine toujours l'électrode pour laquelle la demi-réaction
de réduction a son potententiel standard de réaction le plus élevé à
droite car c'est là qu'il y aura la réduction et donc la borne positive.

$E$ est défini comme le potentiel de réduction à droite moins
le potentiel de réduction à gauche.
Dès lors, si $E\std$ est négatif, c'est qu'on a mal fait son dessin.

\subsection{Récepteur électrochimique}
Supposons maintenant qu'on applique à la pile une différence
de potentiel $\Delta \phi_\mathrm{ext}$ calculée dans le même sens
($\Delta \phi = \phi_\mathrm{droite} - \phi_\mathrm{gauche}$).

Si $\Delta \phi_\mathrm{ext}$ est supérieure à un certain seuil $E$
qui est appelée la force électromotrice de la cellule.
Les charges circulent alors dans le sens contraire au fonctionnement de la pile.
On a alors une cellule d'électrolyse, une batterie mise en charge.

\subsection{Réaction électrochimiques entre
deux électrodes à $p$ et $T$ constants}
L'équation~\eqref{eq:dGdW} nous donne
\[ -\dif G = -\left(\fpart{G}{\xi }\right)_{T,p}\dif\xi =
  -\left(\sum_{K'}\nu_{K'}\mu_{K'}-
\sum_{K}\nu_{K}\mu_{K}\right)\dif\xi \geq - \delta W' \]
donc $-\dif G$ est le travail électrique effectué par le passage des électrons
d'une électrode à l'autre dans des conditions réversibles,
c'est à dire lorsque le potentiel $\Delta \phi$ est équilibré par une source
de potentiel identique $\Delta \phi_\mathrm{ext}$
de manière à ce que le courant soit
infinitésimal dans un sens ou dans l'autre.

On peut trouver
\[ zFE = -\left(\fpart{G}{\xi }\right)_{T,p} \]
avec $F = eN_A = \si{96485}{\coulomb\per\mole}$
et $z$ vaut le nombre d'électrons échangés dans l'équation de la réaction
d'oxydo-réduction globale (il dépend donc de l'écriture de cette équation).

Si $\Delta \phi_\mathrm{ext}$ est exactement égale à $E$,
$\dif G$ est nulle et
il n'y a pas d'évolution du système.

Si $\Delta \phi_\mathrm{ext} < |E|$
alors $G$ diminue et on est en présence d'une pile
qui fournit de l'énergie électrique.

Si $\Delta \phi_\mathrm{ext} > |E|$
alors $G$ augmente et
on est en présence d'une batterie
qui se recharge.

\subsection{Equation de Nernst}
Le potentiel standard de la cellule vaut
$E\std = -\frac{\Delta_rG\std}{zF}$ et
\[ E = E\std-\frac{RT}{zF}\ln Q_a \]
où $Q_a$ est le quotient réactionnel de l'équation globale.
$E$ est fonction de la nature des électrodes et
de la concentrations des ions en solutions.
$E\std$ est fonction uniquement de la nature des électrodes,
de la différence entre la force oxydante du système oxydant et
la force réductrice du système réducteur.

À l'équilibre, on a $E = 0$ et $Q_a = K_a$ donc
\[ E\std = \frac{RT}{zF}\ln K_a. \]

\subsection{Exemple}
Supposons que l'on ai d'un côté du \ce{ZnSO4} et de l'autre du \ce{CuSO4}.
\ce{Cu^{2+}} est plus oxydant que \ce{Zn^{2+}}
car $E_{\ce{Cu^{2+}}}\std = \np{0.34}$ et $E_{\ce{Zn^{2+}}}\std = \np{-0.76}$
donc c'est lui qui fait la réduction.
On le dessine donc à droite et
on a comme demi réactions
\begin{align*}
  \ce{Zn} & = \ce{Zn^{2+}} + 2e^-\\
  \ce{Cu^{2+}} + 2e^- & = \ce{Cu}
\end{align*}
et comme réaction globale
\begin{align*}
  \ce{Zn} + \ce{Cu^{2+}} = \ce{Zn^{2+}} + \ce{Cu}.
\end{align*}
On peut maintenant calculer,
en se rappelant que l'activité d'un solide est 1
\begin{align*}
  E & = E_{\ce{Cu^{2+}}} - E_{\ce{Zn^{2+}}}\\
  & = E_{\ce{Cu^{2+}}}\std - \frac{RT}{2F}\ln\frac{\si{1}{M}}{[\ce{Cu^{2+}}]} -
  \left(E_{\ce{Zn^{2+}}}\std -
  \frac{RT}{2F}\ln\frac{\si{1}{M}}{[\ce{Zn^{2+}}]}\right)\\
  & = E\std - \frac{RT}{2F}\ln\frac{[\ce{Zn^{2+}}]}{[\ce{Cu^{2+}}]}\\
  & = E\std - \frac{RT}{2F}\ln\frac{a_\ce{Zn^{2+}}}{a_\ce{Cu^{2+}}}\\
  & = E\std - \frac{RT}{2F}\ln\frac{a_\ce{Zn^{2+}}a_\ce{Cu}}
  {a_\ce{Zn}a_\ce{Cu^{2+}}}\\
  & = E\std - \frac{RT}{2F}\ln Q_a
\end{align*}
où $E\std = \np{0.34} - (\np{-0.76}) = \np{1.1}$.

\section{Résumé}
L'entropie est une variable d'état qui est composée de deux parties
\[ \dif S = \delta S_i + \delta S_e \]
où
\[ \delta S_i \geq 0 \]
avec égalité pour une transformation \emph{réversible}
et
\[ \delta S_e = \frac{\delta Q}{T}. \]
Si bien que pour une transformation \emph{réversible} et \emph{isobare},
\[ \dif S = \left(\frac{\dif H}{T}\right)_p = \frac{C_p}{T} \dif T. \]

Le troisième principe de la thermodynamique (aka théorème de Nernst
ou encore postulat de Planck) stipule que
\[ S(\si{0}{\kelvin}) = 0. \]

L'entalpie libre de Gibbs est définie comme suit
\[ G = H - TS \]

Pour une transformation \emph{isotherme} et \emph{isobare}
en système \emph{fermé}, on a
\begin{equation}
  \label{eq:dGdW}
  \dif G \leq \delta W'
\end{equation}
où $W'$ est le travail ``hors $p\dif V$''.

Si ce travail est nul, on voit que $G$ ne peut que diminuer.
L'équilibre est donc atteint lorsque $G$ est minimum.

L'enthalpie libre à une certaine température peut être calculée
en fonction de l'enthalpie libre standard à cette même température
\[ G_m(p, T) = G_m\std(T) + RT \ln \frac{p}{p\std} \]
``standard'' signifie ``corps pur à pression $p\std$'' où
$p\std = \si{1}{bar}$ est la pression standard.

Dans un système composé de solides, liquides et gaz, la variation
de l'enthalpie libre due aux solides et aux liquides et négligeable par
rapport à celle due au gaz.

Posons
\[ \mu_i \eqdef \left(\fpart{G}{n_i}\right)_{T,p,n_{j\neq i}} =
\left(\fpart{U}{n_i}\right)_{S,V,n_{j\neq i}} =
\left(\fpart{H}{n_i}\right)_{S,p,n_{j\neq i}} =
\left(\fpart{A}{n_i}\right)_{T,V,n_{j\neq i}} \]
Où $\mu_i$ est potentiel chimique de $i$ dans un mélange
et $\mu^i$ sont potentiel pur.
On a
\[ \mu^i = {\mu^i}\std + RT \ln\frac{p}{p\std} \]
L'équilibre d'une réaction est atteint lorsque tous les potentiels
chimiques $\mu_i$ sont uniformes.

L'équation fondamentale pour $G$ dans un système à composantes
multiple est
\[ \dif G = -S\dif T + V\dif p + \sum \mu_i \dif n_i. \]

On a, pour une réaction,
\[ \Delta_r H\std =
\sum_{K'}\Delta_f{H_m^{K'}}\std - \sum_{K'}\Delta_f{H_m^{K'}}\std \]
avec $\Delta_f H\std$ dépendant de la température et
$\Delta_f H_{298}\std = 0$ par convention.
Aussi
\[ \Delta_r S\std =
\sum_{K'}{S_m^{K'}}\std - \sum_{K'}{S_m^{K'}}\std. \]
Et
\[ \Delta_r G\std =
\sum_{K'}\Delta_f{G_m^{K'}}\std - \sum_{K'}\Delta_f{G_m^{K'}}\std \]
avec $\Delta_r G\std$ dépendant de la température et
$\Delta_r G_{298}\std = 0$ par la convention sur $\Delta_r H\std$.

On définit une variable d'avancement de la réaction
\[ \dif \xi = \frac{\dif n_{K'}}{\nu_{K'}} = -\frac{\dif n_K}{\nu_K} \]
qui permet d'écrire
\[ \left(\fpart{G}{\xi}\right)_{T,p}\dif \xi = -\mathcal{A}\dif \xi \]
où $\mathcal{A} \eqdef
-\Delta_r \mu = \sum_K \nu_K\mu_K - \sum_{K'}\nu_{K'}\mu{K'}$.
L'équilibre est donc atteint quand $\mathcal{A} = 0$.

En définissant le quotient réactionnel $Q_p$
\[ Q_p = \frac{\prod_{K'}\left(\frac{p_{K'}}{p\std}\right)^{\nu_{K'}}}
{\prod_{K}\left(\frac{p_{K}}{p\std}\right)^{\nu_{K}}} \]
et $K_p$ valant $Q_p$ à l'équilibre, on a
\[ \Delta_r G\std = -RT\ln K_p \]
et l'équation de van't Hoff
\[ \fdif{(\ln K_p)}{T} = \frac{\Delta_r H\std}{RT^2} \]
qui, en supposant $\Delta_r C_{p,m}$ suffisamment faible
et donc $\Delta_r H\std$ indépendant
de la température, implique que
\[ \ln K_{p,T_2}-\ln K_{p,T_1} =
\frac{\Delta_r H\std}{R}\left(\frac{1}{T_2} - \frac{1}{T_1}\right). \]

Dans un liquide, les potentiels chimiques n'évoluent plus avec $p$,
on a
\[ \mu_A = {\mu^A}\std + RT \ln \frac{[A]}{\si{1}{M}}. \]
La constante ne s'écrit alors plus $K_p$ mais $K_a$.

Pour une dissolution, on l'écrit $K_s$
\[ K_s = [\ce{A+}][\ce{B-}]. \]
$s$ est appelé la solubilité et est défini comme le nombre de mole
du solide en solution qui se dissolvent.

Il faut faire attention car si un produit est déjà dans la réaction,
il faut le prendre en compte et la solubilité diminue.
Si un des produit est un acide ou une base,
il va réagir et la solubilité dépendra du $\ph$.

Dans une molécule,
qu'elle soit faite de liaisons ioniques ou covalentes,
on peut assigner à chaque atome un \emph{degré d'oxydation}
selon certaines conventions.
La somme des degré d'oxydation doit bien sûr valoir la charge totale
de la molécule.

Une réaction d'oxydo-réduction est une réaction au cours duquel
au moins un atome change de degré d'oxydation.

Si son degré d'oxydation augmente, c'est qu'un atome s'oxyde,
on dit que c'est un \emph{réducteur}.

Si son degré d'oxydation diminue, c'est qu'un atome se réduit,
on dit que c'est un \emph{oxydant}.

Une pile électrochimique est constituée
\begin{itemize}
  \item d'une électrode où se passe
    une oxydation, on l'appelle l'anode, on la dessine à gauche et
    on lui donne un signe $-$.
    C'est la réaction de réduction avec le plus petit $E\std$;
  \item et d'une électrode où se passe
    une réduction, on l'appelle la cathode, on la dessine à droite et
    on lui donne un signe $+$.
    C'est la réaction de réduction avec le plus grand $E\std$.
\end{itemize}
Pour qu'une pile marche, il faut une barrière ne laissant que passer
les ions et pas les électrons.
Des anions passent par cette barrière de la cathode vers l'anode
et des électrons passent de l'anode vers la cathode dans l'autre sens
sous forme de courrant.

On a
\[ E = E\std - \frac{RT}{zF} \ln Q_a \]
et
\[ E\std = \frac{RT}{zF} \ln K_a = -\frac{\Delta_r G\std}{zF}. \]

\part{Cinétique chimique}
% +--------------------+
% | Cinétique chimique |
% +--------------------+
\section{Introduction}
On distingue fondamentalement l'équilibre et la cinétique chimiques.
Le premier dérive de la thermodynamique et contrôle la composition vers
laquelle tend naturellement le mélange réactif.
Dans certaines réaction, les réactifs sont entièrement consommés.
Dans d'autres, au contraire, réactifs et produits coexistent à l'équilibre.

La cinétique décrit la vitesse à laquelle on tend vers l'équilibre.

Si un mécanisme est composé de deux étapes distinctes,
c'est l'étape la plus lente qui détermine la vitesse de réaction,
elle est déterminante de vitesse.
\section{Définition}
\subsection{Exemple}
Si on a une réaction
\[ \ce{A} \ce{->} 2\ce{B}. \]
On a alors à tout moment,
\[ \frac{\dif n_\ce{B}}{\dif t} = -2\frac{\dif n_\ce{A}}{\dif t} \]
et la vitesse de réaction $r$ est définie par
\[ r = \frac 1V \left(\frac 12 \right)\frac{\dif n_\ce{B}}{\dif t} =
-\frac 1V \left(\frac 11 \right) \frac{\dif n_\ce{A}}{\dif t}. \]
À tout moment, la variation instantanée du nombre de molécule d'un
réactif ou d'un produit est liée à celle de tous les autres
par la stœchiométrie de la réaction.
C'est pour cela qu'on peut définir une vitesse de réaction unique.

\subsection{Généralisation}
Soit la réaction
\[ a\ce{A}+b\ce{B}+\ldots \ce{->} m\ce{M} + n\ce{N} + \ldots. \]
À tout moment $t$ et dans un volume donné
\[ \frac 1m \frac {\dif n_\ce{M}}{\dif t} =
  \frac 1n \frac{\dif n_\ce{N}}{\dif t}
  = \ldots = -\frac 1a \frac{\dif n_\ce{A}}{\dif t}
  = -\frac 1b \frac{\dif n_\ce{B}}{\dif t} \]
La vitesse de réaction est donné par
\[ r = \frac 1V \frac 1m \frac {\dif n_\ce{M}}{\dif t} =
  \frac 1V \frac 1n \frac {\dif n_\ce{N}}{\dif t} = \ldots =
  -\frac 1V \frac 1a \frac {\dif n_\ce{A}}{\dif t}
  = -\frac 1V \frac 1b \frac {\dif n_\ce{B}}{\dif t}. \]
La vitesse d'apparition des produits et des réactifs est
\[ r_A = -\frac 1V \frac {\dif n_A}{\dif t} = ra \]
\[ r_P = \frac 1V \frac {\dif n_P}{\dif t} = rp. \]

\subsection{À volume constant}
\[ r = \frac 1V \frac 1m \frac {\dif n_M}{\dif t} =
  \frac 1m \frac {\dif \left(\frac {n_M}V\right)}{\dif t} =
  \frac 1m \frac {\dif [M]}{\dif t} = \ldots. \]
Ce n'est que lorsque le volume est constant que la vitesse de
réaction devient égale à la dérivée temporelle de la concentration.
Dans le cadre du cours, on utilisera presque toujours cette simplification.

\section{Ordre de réaction}
\subsection{Dégradation de l'Ozone}
\[ \ce{2O3} \ce{->} \ce{3O2} \]
On observe:
\[  r = k \frac{[\ce{O_3}]^2}{[\ce{O_2}]^2}. \]
$k$ est fonction de la température et de la concentration en $UV$.

\subsection{En général}
Pour une réaction quelconque,
\[ a\ce{A}+b\ce{B}+\ldots \ce{->} m\ce{M} + n\ce{N} + \ldots \]
La vitesse $r$ peut s'écrire
\[  r = k[\ce{A}]^{\alpha}[\ce{B}]^{\beta}
\ldots[\ce{M}]^{\mu}[\ce{N}]^{\eta}\ldots \]
$k$ est la constante de vitesse, elle ne dépend pas des concentrations.

$\alpha,\beta,\ldots$ sont les ordres partiels.

$\alpha+\beta+\ldots$ est l'ordre global.

Habituellement, les ordres partiels sont différents
des coefficients stœchiométriques,
ils sont positifs quand il s'agit des réactifs et
négatifs quand il s'agit des produits.
En fait, ils sont égaux dans le cas de réaction élémentaire,
ils sont égaux car la réaction élémentaire n'en cache pas d'autre.
Mais il faut alors faire attention à prendre la bonne stochiométrie
comme c'est détaillé plus loin.

On voit avec l'équation de $r$ que les unités de
la constante de vitesse dépendent de l'ordre de la réaction.

\subsection{Différents ordre}
\subsection{Dégénérescence d'ordre}
Supposons une réaction
\[ \ce{A + B -> P} \]
En supposant que l'ordre partiel de \ce{P} est nul,
on peut mesurer expériementalement l'ordre de \ce{A}
en prenant une concentration initiale de \ce{B} beaucoup plus
grande que celle de \ce{A}.
Du coup, la diminution de la concentration de \ce{B} ne se fera
presque pas sentir et on peut considérer \ce{[B]} constant.
On peut alors réécrire
\[ r = k[\ce{A}]^\alpha [\ce{B}]^\beta \]
en
\[ r = k'[\ce{A}]^\alpha \]
avec $k'\eqdef k[\ce{B}]^\beta$ en apparence constant.
On peut alors mesurer $\alpha$.

\subsubsection{Ordre 0}
Soit la réaction
\[ \ce{A -> P}. \]
Pour un ordre 0, on aura
\[ r = \frac{\dif [\ce{P}]}{\dif t} = -\frac{\dif [\ce{A}]}{\dif t} = k \]
et donc
\[ [\ce{P}] = [\ce{P}]_{t = 0}+kt \]
\[ [\ce{A}] = [\ce{A}]_{t = 0}-kt. \]

En fait, l'ordre zéro n'existe pas mais
si l'ordre de la réaction est 1, c'est à dire si $r = k_1[\ce{A}]$,
on aura en apparence un ordre 0 tant que la concentration $[\ce{A}]$
reste proche de la concentration initiale $[\ce{A}]_{t = 0}$.
C'est le même effet que pour la dégénerescence d'ordre.
On parle de cinétique initiale.

\subsection{Ordre 1}
\[ \ce{A -> P} \]
\[ r = \frac{\dif [\ce{P}]}{\dif t}
= -\frac{\dif [\ce{A}]}{\dif t} = k[\ce{A}]. \]
L'ordre 1 s'observe souvent lors de décomposition thermique ou
lorsque la concentration d'un seul réactif limite la vitesse.
\[ [\ce{A}] = [\ce{A}]_{t = 0} \exp(-kt) \]
\[ [\ce{P}] = [\ce{P}]_{t = 0}+[\ce{A}]_{t = 0}\ (1-\exp (-kt)). \]
La fraction convertie des produits vaut
\[ p_\ce{A} = 1-\frac{[\ce{A}]}{[\ce{A}]_{t = 0}} = 1-\exp (-kt). \]
En regardant cette fraction, on peut dire que la cinétique est
indépendante de la concentration initiale.

Le temps de demi vie d'une réaction est le temps nécessaire
pour convertir la moitié des réactifs et vaut pour l'ordre 1
\[ t_{1/2} = \frac {\ln 2}{k}. \]

\subsection{Ordre 2}
\[ \ce{A -> P} \]
\[ r = \frac{\dif [\ce{P}]}{\dif t}
= -\frac{\dif [\ce{A}]}{\dif t} = k[\ce{A}]^2. \]
L'ordre deux est souvent observé pour des mécanismes réactionnels faisant
intervenir deux molécules à la fois dans l'étape déterminante de vitesse.

On a
\[ [\ce{A}] = \frac{[\ce{A}]_0}{1+[\ce{A}]_0kt}\]

Le temps de réaction dépend de la concentration initiale
\[ t_{1/2} = \frac 1{k[\ce{A}]_0}. \]

\section{Influence de la température}
La plupart du temps, la vitesse augmente lorsque la température monte,
la vitesse de nombreuse réactions double si on augmente
la température de \si{10}{\celsius}.

On peut expliquer ce phénomène par deux théories,
celles des collisions, basée sur la théorie cinétique des gaz et
valable uniquement pour eux, ou celle du complexe activé.

\subsection{Théorie des collisions}
La réaction est le résultat de collisions de plusieurs molécules.
De temps en temps, lors d'une collision, les molécules entrent en réactions.
Seulement une petite partie de ces collisions donnent la réaction,
il y a un facteur limitant lié à la température.

La réaction ne se produit que si l'énergie cinétique des réactifs est
supérieure à $E_A$ parce que les molécules
doivent passer une barrière de hauteur $E_A$.
La fraction des molécules d'énergie supérieure à
$E_A$ est connue par la distribution de Maxwell Boltzmann
$\exp \left(-\frac{E_A}{RT}\right)$ et donc:
\begin{equation}
  \label{eq:kART}
  k(T) = A \exp \left(-\frac{E_A}{RT}\right).
\end{equation}
Le facteur $A$, appelée le facteur de fréquence tient compte du fait
que toutes les collisions d'énergie suffisante ne donne pas
forcément une réaction, il y a aussi d'autres facteurs.

\begin{myrem}
  Dans l'équation ci-dessus, on a $E_A$ qui a des unités de \joule\per\mole.
  C'est l'énergie qu'il faut pour tout une mole.
  Ce n'est pas l'énergie pour une molécule,
  sinon on aurait eu
  \[ k(T) = A \exp \left(-\frac{E'_A}{k_BT}\right). \]
  avec $E'_A \eqdef \frac{E_A}{N_A}$ l'énergie pour une molécule.
\end{myrem}

\subsection{Équation d'Arrhenius}
De \eqref{eq:kART}, on aboutit directement à
\[ \ln(k(T)) = \ln(A)-\frac{E_A}{RT} \]
\begin{itemize}
  \item Si on connait $k$ à diverses températures,
    on peut déterminer $A$ et $E_A$;
  \item Si on connait $A$ et $E_A$,
    on peut calculer $k$ à toutes les températures;
  \item Si on connait $k$ à une température et $E_A$,
    on peut calculer $k$ à d'autres températures.
\end{itemize}
Il faut bien voir que la constante de vitesse $k$
n'est constante qu'à température constante.
L'énergie d'activation, elle, est indépendante de la température.
Plus l'énergie d'activation est élevée,
plus vite la constante de vitesse augmente
avec la température.

On peut tirer deux équations intéressantes en dérivant
l'équation d'Arrhénius par $T$ ou par $1/T$
\begin{align*}
  \fdif{k}{T} & = \frac{E_A}{RT^2}\\
  \fdif{k}{\left(\frac{1}{T}\right)} & = -\frac{E_A}{R}.
\end{align*}

\subsection{Théorie du complexe activé}
Lors de collision, les molécules se déforment sous l'effet du choc et
une partie de l'énergie cinétique est transformée en énergie potentielle.
Les molécules déformées en contact forment un édifice transitoire:
le complexe est activé.
Le complexe a une énergie supérieur aux constituants:
les liaisons sont déformées.

Pour que la réaction se produise,
les chocs doivent conférer une énergie potentielle
suffisante pour passer la barrière

\section{Décomposition d'une réaction}
Les ordres partiels peuvent être expliqués si on décompose
la réaction on plusieurs réactions élémentaires.

Les vitesses de ces réactions sont reliées entre elles car souvent,
l'une a besoin des produits de l'autre.
On appelle ces molécules des radicaux.

La décomposition en réactions élémentaires permet de déterminer
la vitesse de la réaction globale en fonction des concentration
des molécules des réactions élémentaires.
Le problème c'est qu'on ne connait pas les concentrations
des radicaux donc on ne veut pas qu'ils apparaissent dans l'expression.

Il y a deux manières de s'en sortir,
\begin{itemize}
  \item L'hypothèse de quasi-stationnarité des radicaux (aka HQSR);
  \item L'hypothèse qu'une réaction est équilibrée.
\end{itemize}

\subsection{Réactions élémentaires}
Une réaction élémentaire est une réaction qui n'en cache pas d'autre.
Pour ces réactions, on a
\[ r = k \prod_K [K]^{\nu_K}. \]
Par exemple, pour la réaction
\[ \ce{O3 + O -> 2O2}. \]
on a
\[ r = k[\ce{O3}][\ce{O}] \]
qui peut se comprendre intuitivement avec les probabilité car
la probabilité de rencontrer une molécule est proportionnelle à
sa concentration et la probabilité que deux éléments se rencontrent,
c'est la probabilité de rencontrer les deux éléments et
\[ P(A \land B) = P(A) \cdot P(B) \]
où $\land$ est le ``et'' logique.

\begin{myrem}[Importance de la stoechiométrie]
  Il est important de remarquer que lorsqu'on veut utiliser
  le fait qu'une réaction est \emph{élémentaire} pour en déduire ses
  ordres partiels, il est primordial d'écrire une \emph{stoechiométrie}
  qui \emph{reflète la réalité}.
  Par exemple, si on a une réaction
  \[ \ce{A -> B}, \]
  il faut voir si la réaction est monomoléculaire, c'est à dire qu'une
  molécule \ce{A} réagit toute seule ou si elle est bimoléculaire,
  c'est à dire qu'il faut que deux molécules de \ce{A} se rencontrent
  pour qu'il y ait réaction,
  dans quel cas, il faudra plutôt écrire
  \[ \ce{2A -> 2B} \]
  car la réaction est alors d'ordre 2.

  L'idée de réaction monomoléculaire peut paraitre bizarre au premier
  abord mais la théorie des collisions permet de l'expliquer intuitivement.
  Dans une réaction monomoléculaire, après la réaction,
  l'énergie est redistribuée entre les molécules.
  Une partie de cette énergie peut servir à exciter une des deux molécules qui
  se décompose ensuite lentement dans une étape monomoléculaire.
\end{myrem}

\subsection{Hypothèse de quasi-stationnarité des radicaux}
La réaction de terminaison a une vitesse qui augmente avec l'augmentation
des radicaux.
Imaginons que la vitesse de disparition soit plus élevée que la vitesse
d'apparition. Il y aura de moins en moins de radicaux donc la vitesse
de disparition diminuera jusqu'à ce qu'elle soit égale à
la vitesse d'apparition.

Si la vitesse de disparition est moins élevée que la vitesse
d'apparition, l'effet inverse va se passer.
De telle sorte que les vitesse d'apparition des radicaux et de disparition
des radicaux vont être égales quelle que soit la situation de départ.

\begin{myrem}
  Ça ne veut absolument \emph{pas} dire
  que la concentration en radicaux est constante.
  Imaginons que la vitesse d'apparition augmente, la vitesse
  de disparition devra augmenter aussi et donc la concentration en
  radicaux aussi comme elle est liée à la vitesse de disparition.

  Par contre, il est vrai de dire que si la vitesse d'apparition
  est \emph{constante}, la concentration des radicaux sera constante aussi.
  Évidemment, une vitesse n'est théoriquement jamais vraiment constante,
  l'ordre 0 est un pseudo-ordre mais en pratique,
  si la concentration en réactif de la réaction qui crée les radicaux
  est très grande par rapport à la vitesse de la réaction que la
  diminution en réactif sera négligeable et on aura un ordre 0.
\end{myrem}

\subsection{Réactions équilibrées}
On peut supposer qu'une réaction est équilibrée,
c'est à dire que pendant la transformation,
les composés n'ont de cesse d'être à l'équilibre.

Autrement dit, si par exemple, on vous dit que la réaction
\[ \ce{2 A} + \ce{3 B} = \ce{C} \]
est à l'équilibre, on a la relation
\[ K_a = \frac{a_\ce{C}}{a_\ce{A}^2a_\ce{B}^3}. \]
qui est respectée tout au long de la réaction.
Bien sûr on l'utilisera la plus souvent sous la forme
\[ K_{\text{éq}} = \frac{[\ce{C}]}{[\ce{A}]^2[\ce{B}]^3}. \]

On peut remarquer que si la réaction est élémentaire,
\[ r_{\rightarrow} = k_{\rightarrow}[\ce{A}]^2[\ce{B}]^3 \]
et
\[ r_{\leftarrow} = k_{\leftarrow}[\ce{C}]. \]
Si la réaction est à l'équilibre,
on a $r_{\rightarrow} = r_{\leftarrow}$ donc
\[ \frac{k_\rightarrow}{k_\leftarrow}
  = \frac{[\ce{C}]}{[\ce{A}]^2[\ce{B}]^3} = K_{\text{éq}}. \]
Pour pouvoir retomber sur des formules connues,
appliquons un logarithme népérien puis dérivons le tout par $T$
\[ \fdif{\ln k_\rightarrow}{T} - \fdif{\ln k_\leftarrow}{T}
= \fdif{\ln K_{\text{éq}}}{T}. \]
En appliquant Arrhénius pour le membre de gauche et
van't Hoff pour le membre de droite, on obtient
\[ \frac{E_{A\rightarrow}}{RT^2} - \frac{E_{A\leftarrow}}{RT^2}
= \frac{\Delta_r H\std}{RT^2}. \]
Ce qui donne
\[ E_{A\rightarrow} - E_{A\leftarrow}
= \Delta_r H\std. \]
qui est, ma foi, un résultat fort intéressant.

\subsection{Réactions en chaines}
Il peut arriver qu'une réaction se fasse en chaine
comme pour la création d'un polymère
\begin{align*}
  \ce{R-O-(CH2-CH2)}_{n}-\ce{CH2-CH2^*} + \ce{H2C=CH2}\\
  \ce{-> R-O-(CH2-CH2)}_{n+1}-\ce{CH2-CH2^*}.
\end{align*}
Dans ce cas, on a besoin de l'hypothèse de Flory qui dit que
la vitesse de la réaction est indépendante de $n$.

\annexe

\section{L'effet Joule-Thomson}
\subsection{Le coefficient de Joule-Thomson}
Le coefficient $\pi_t$ défini par joule comme:
\[ \pi_t = \left(\fpart{U}{V}\right)_T \]
est relativement difficile à évaluer.
Le coefficient de Joule-Thomson mesure le lien entre la température et
la pression à enthalpie constante:
\[  u_{J-T} = \left(\fpart{T}{p}\right)_H \]
Pour illustrer cette dérivée, prenons deux cylindres connectés
entre eux par une zone poreuse.
Les deux cylindres possèdent un piston leur extrémité.
À la situation initiale on a le premier cylindre plein
à pression $p_1$ et le second vide.
Lentement on va pousser le piston et rendre le premier cylindre vide
et le deuxième rempli à une pression $p_2$.
Si le milieu est isolé thermiquement de l'environnement
\[ \Delta U = Q+ W = p_1V_1-p_2V_2 \]
ou plus simplement $\Delta H = 0$ l'enthalpie est donc conservée.

\subsection{Lien avec les grandeurs connues}
Repartons de
\[ \dif T = \left ( \fpart{T}{p}\right)_H \dif p+
\left ( \fpart{T}{H}\right)_p \dif H = u_{J-T}\dif p+\frac 1{C_p}\dif H \]
En dérivant par rapport au volume en maintenant
la température constante on obtient
\[ \left ( \fpart{T}{V} \right)_T =
  u_{J-T}\left(\fpart{p}{V}\right)_T+
  \frac 1{C_p}\left(\fpart{H}{V}\right)_T \]
Le terme de gauche est nulle et on peut remplacer
l'enthalpie par sa définition et obtenir
\[ u_{J-T}\left(\fpart{p}{V}\right)_T+
  \frac 1{C_p} \left( \left(\fpart{U}{V}\right)_T+p+
  V\left(\fpart{p}{V}\right)_T\right) = 0 \]
On retrouve le coefficient $\pi_T$ et une autre grandeur,
la compressibilité isotherme
\[  \kappa = - \frac 1V \left(\fpart{V}{p}\right)_T \]
qui est la variation de volume lorsque la pression change.
L'expression devient alors:
\[ \frac{u_{J-T}+V/C_p}{\kappa V} = \frac 1{C_p}(\pi_T+p) \]
En isolant $\pi_T$
\[  \pi_T = \frac {C_p u_{J-T}+V}{\kappa V}-p \]
On voit que le coefficient de Joule-Thomson et $\pi_T$ sont bien lié entre eux.

\section{Constantes}
\label{sec:const}
Beaucoup de constantes sont utilisées en chimie et si on ne les connait pas,
on a beau avoir tout compris, on ne sait rien calculer.

Elles ne seront pas données à l'examen donc il faut les connaitre.
\begin{align*}
  T[\kelvin] & = \theta[\celsius] + \si{273.15}{\kelvin}\\
  N_a & = \si{6e23}{\#\per\mole}\\
  k_B & = \si{1.38e-23}{\joule\per\kelvin}\\
  R & = \si{8.3145}{\joule\per\mole\cdot\kelvin} = k_B N_A\\
  \si{1}{bar} & = \si{e5}{Pa}\\
  \si{1}{atm} & = \si{101300}{\pascal}\\
  \si{1}{atm} & = \si{1.013}{bar}\\
  \si{1}{Torr} & = \si{1}{\milli\meter_{Hg}}\\
  \si{1}{Torr} & = \si{133,322}{\pascal}\\
  \si{760}{Torr} & = \si{1}{atm}\\
  M_\mathrm{air} & = \si{28.96}{\gram\per\mole}\\
  M_\mathrm{\ce{N2}} & = \si{28}{\gram\per\mole}\\
  M_\mathrm{\ce{O2}} & = \si{32}{\gram\per\mole}\\
  R & = \si{8.31451e-2}{\liter\cdot bar\per\mole\cdot\kelvin}\\
  R & = \si{8.20578e-2}{\liter\cdot atm\per\mole\cdot\kelvin}\\
  R & = \si{62.364}{\liter\cdot Torr\per\mole\cdot\kelvin}\\
  R^* & = \frac{R}{M}\\
  F & = \si{96485}{\coulomb\per\mole} = eN_A
\end{align*}

\begin{myrem}
$R^*$ n'est pas vraiment une constante puisqu'il dépend de $M$,
la masse molaire.
\end{myrem}

\section{Équations en vrac}
Voici, en vrac, les équations principales vues qu'il vaut mieux
s'assurer de connaitre par coeur.
Les hypothèses pour les utiliser ne sont pas notées,
ce n'est pas parce que ce n'est pas important,
loin de là, mais ce n'est pas le but de cette section.
\begin{align*}
  pV & = nRT\\
  H & = U + pV\\
  G & = H - TS\\
  \dif U & = C_V \dif T\\
  \dif H & = C_p \dif T\\
  \dif S & = \frac{C_p}{T} \dif T\\
  \dif S & = \frac{\dif U + p \dif V}{T} = \frac{\dif H - V \dif p}{T}\\
  C_V & = \frac 32 nR\\
  C_p & = C_V + nR\\
  w_m & = \int_1^2 v\dif p + \Delta k + g\Delta z + w_f\\
  \Delta h & = \int_1^2 v\dif p + q + w_f\\
  G_m & = G_m\std + RT \ln a_A\\
  \mu_i & = {\mu^i}\std + RT \ln a_A\\
  a_A & = \frac{p}{p\std}\\
  a_A & = \frac{[A]}{\si{1}{M}}\\
  \dif G & = -S\dif T + V\dif p + \sum \mu_i \dif n_i\\
  \Delta_r G\std & = -RT\ln K_p\\
  \fdif{(\ln K_p)}{T} & = \frac{\Delta_r H\std}{RT^2}\\
  E & = E\std - \frac{RT}{zF} \ln Q_a = -\frac{\Delta_r G}{zF}\\
  \ln(k(t)) & = \ln A - \frac{E_A}{RT}\\
  E_{A\rightarrow} - E_{A\leftarrow} & = \Delta_r H\std.
\end{align*}

\end{document}
