%De Nicolas aux lecteurs:
%%PREMIER jet, de nombreuses fautes de frappe surement mais je n'ai pas encore relu, je le ferai en ajoutant les graphs.
%% Je me suis basé sur mes notes des deux premiers cours et les slides

%%%%%%%%%%%%%%%%%%%%%%%%%%%%%%%%%%%%%%%%%%%%%%%%%%%%%%%%%%%%%%




\documentclass[11pt,a4paper]{article}

% French
\usepackage[utf8x]{inputenc}
\usepackage[frenchb]{babel}
\usepackage[T1]{fontenc}
\usepackage{lmodern}
\usepackage{ifthen}

% Color
% cfr http://en.wikibooks.org/wiki/LaTeX/Colors
\usepackage{color}
\usepackage[usenames,dvipsnames,svgnames,table]{xcolor}
\definecolor{dkgreen}{rgb}{0.25,0.7,0.35}
\definecolor{dkred}{rgb}{0.7,0,0}

% Floats and referencing
\newcommand{\sectionref}[1]{section~\ref{sec:#1}}
\newcommand{\annexeref}[1]{annexe~\ref{ann:#1}}
\newcommand{\figuref}[1]{figure~\ref{fig:#1}}
\newcommand{\tabref}[1]{table~\ref{tab:#1}}
\usepackage{xparse}
\NewDocumentEnvironment{myfig}{mm}
{\begin{figure}[!ht]\centering}
{\caption{#2}\label{fig:#1}\end{figure}}

% Listing
\usepackage{listings}
\lstset{
  numbers=left,
  numberstyle=\tiny\color{gray},
  basicstyle=\rm\small\ttfamily,
  keywordstyle=\bfseries\color{dkred},
  frame=single,
  commentstyle=\color{gray}=small,
  stringstyle=\color{dkgreen},
  %backgroundcolor=\color{gray!10},
  %tabsize=2,
  rulecolor=\color{black!30},
  %title=\lstname,
  breaklines=true,
  framextopmargin=2pt,
  framexbottommargin=2pt,
  extendedchars=true,
  inputencoding=utf8x
}

\newcommand{\matlab}{\textsc{Matlab}}
\newcommand{\octave}{\textsc{GNU/Octave}}
\newcommand{\qtoctave}{\textsc{QtOctave}}
\newcommand{\oz}{\textsc{Oz}}
\newcommand{\java}{\textsc{Java}}
\newcommand{\clang}{\textsc{C}}
\newcommand{\keyword}{mot clef}

% Math symbols
\usepackage{amsmath}
\usepackage{amssymb}
\usepackage{amsthm}
\DeclareMathOperator*{\argmin}{arg\,min}
\DeclareMathOperator*{\argmax}{arg\,max}

% Sets
\newcommand{\Z}{\mathbb{Z}}
\newcommand{\R}{\mathbb{R}}
\newcommand{\Rn}{\R^n}
\newcommand{\Rnn}{\R^{n \times n}}
\newcommand{\C}{\mathbb{C}}
\newcommand{\K}{\mathbb{K}}
\newcommand{\Kn}{\K^n}
\newcommand{\Knn}{\K^{n \times n}}

% Chemistry
\newcommand{\std}{\ensuremath{^{\circ}}}
\newcommand\ph{\ensuremath{\mathrm{pH}}}

% Theorem and definitions
\theoremstyle{definition}
\newtheorem{mydef}{Définition}
\newtheorem{mynota}[mydef]{Notation}
\newtheorem{myprop}[mydef]{Propriétés}
\newtheorem{myrem}[mydef]{Remarque}
\newtheorem{myform}[mydef]{Formules}
\newtheorem{mycorr}[mydef]{Corrolaire}
\newtheorem{mytheo}[mydef]{Théorème}
\newtheorem{mylem}[mydef]{Lemme}
\newtheorem{myexem}[mydef]{Exemple}
\newtheorem{myineg}[mydef]{Inégalité}

% Unit vectors
\usepackage{esint}
\usepackage{esvect}
\newcommand{\kmath}{k}
\newcommand{\xunit}{\hat{\imath}}
\newcommand{\yunit}{\hat{\jmath}}
\newcommand{\zunit}{\hat{\kmath}}

% rot & div & grad & lap
\DeclareMathOperator{\newdiv}{div}
\newcommand{\divn}[1]{\nabla \cdot #1}
\newcommand{\rotn}[1]{\nabla \times #1}
\newcommand{\grad}[1]{\nabla #1}
\newcommand{\gradn}[1]{\nabla #1}
\newcommand{\lap}[1]{\nabla^2 #1}


% Elec
\newcommand{\B}{\vec B}
\newcommand{\E}{\vec E}
\newcommand{\EMF}{\mathcal{E}}
\newcommand{\perm}{\varepsilon} % permittivity

\newcommand{\bigoh}{\mathcal{O}}
\newcommand\eqdef{\triangleq}

\DeclareMathOperator{\newdiff}{d} % use \dif instead
\newcommand{\dif}{\newdiff\!}
\newcommand{\fpart}[2]{\frac{\partial #1}{\partial #2}}
\newcommand{\ffpart}[2]{\frac{\partial^2 #1}{\partial #2^2}}
\newcommand{\fdpart}[3]{\frac{\partial^2 #1}{\partial #2\partial #3}}
\newcommand{\fdif}[2]{\frac{\dif #1}{\dif #2}}
\newcommand{\ffdif}[2]{\frac{\dif^2 #1}{\dif #2^2}}
\newcommand{\constant}{\ensuremath{\mathrm{cst}}}

% Numbers and units
\usepackage[squaren, Gray]{SIunits}
\usepackage{sistyle}
\usepackage[autolanguage]{numprint}
%\usepackage{numprint}
\newcommand\si[2]{\numprint[#2]{#1}}
\newcommand\np[1]{\numprint{#1}}

\newcommand\strong[1]{\textbf{#1}}
\newcommand{\annexe}{\part{Annexes}\appendix}

% Bibliography
\newcommand{\biblio}{\bibliographystyle{plain}\bibliography{biblio}}

\usepackage{fullpage}
% le `[e ]' rend le premier argument (#1) optionnel
% avec comme valeur par défaut `e `
\newcommand{\hypertitle}[7][e ]{
\usepackage{hyperref}
{\renewcommand{\and}{\unskip, }
\hypersetup{pdfauthor={#6},
            pdftitle={Synth\`ese d#1#2 Q#3 - L#4#5},
            pdfsubject={#2}}
}

\title{Synth\`ese d#1#2 Q#3 - L#4#5}
\author{#6}

\begin{document}

\ifthenelse{\isundefined{\skiptitlepage}}{
\begin{titlepage}
\maketitle

 \paragraph{Informations importantes}
   Ce document est grandement inspiré de l'excellent cours
   donné par #7 à l'EPL (École Polytechnique de Louvain),
   faculté de l'UCL (Université Catholique de Louvain).
   Il est écrit par les auteurs susnommés avec l'aide de tous
   les autres étudiants, la vôtre est donc la bienvenue.
   Il y a toujours moyen de l'améliorer, surtout si le cours
   change car la synthèse doit alors être modifiée en conséquence.
   On peut retrouver le code source à l'adresse suivante
   \begin{center}
     \url{https://github.com/Gp2mv3/Syntheses}.
   \end{center}
   On y trouve aussi le contenu du \texttt{README} qui contient de plus
   amples informations, vous êtes invité à le lire.

   Il y est indiqué que les questions, signalements d'erreurs,
   suggestions d'améliorations ou quelque discussion que ce soit
   relative au projet
   sont à spécifier de préférence à l'adresse suivante
   \begin{center}
     \url{https://github.com/Gp2mv3/Syntheses/issues}.
   \end{center}
   Ça permet à tout le monde de les voir, les commenter et agir
   en conséquence.
   Vous êtes d'ailleurs invité à participer aux discussions.

   Vous trouverez aussi des informations dans le wiki
   \begin{center}
     \url{https://github.com/Gp2mv3/Syntheses/wiki}.
   \end{center}
   comme le statut des synthèses pour chaque cours
   \begin{center}
     \url{https://github.com/Gp2mv3/Syntheses/wiki/Status}.
   \end{center}
   vous pouvez d'ailleurs remarquer qu'il en manque encore beaucoup,
   votre aide est la bienvenue.

   Pour contribuer au bug tracker et au wiki, il vous suffira de
   créer un compte sur Github.
   Pour interagir avec le code des synthèses,
   il vous faudra installer \LaTeX.
   Pour interagir directement avec le code sur Github,
   vous devez utiliser \texttt{git}.
   Si cela pose problème,
   nous sommes évidemment ouverts à des contributeurs envoyant leurs
   changements par mail ou n'importe quel autre moyen.
\end{titlepage}
}{}

\ifthenelse{\isundefined{\skiptableofcontents}}{
\tableofcontents
}{}
}


\usepackage[hideerrors]{xcolor}
\usepackage{array}
\usepackage{amsmath}
\usepackage{amssymb}
\usepackage{amsthm}
\usepackage{fancybox}
\usepackage{float}
\usepackage{colortbl}
%\usepackage{makecell}
\usepackage{graphicx}
%\usepackage{titlesec}
%\usepackage{qtree}
%\usepackage{tensor}

\DeclareMathOperator{\Sur}{Sur}
\DeclareMathOperator{\In}{In}
\DeclareMathOperator{\newnull}{null}
\DeclareMathOperator{\newim}{Im}
\DeclareMathOperator{\newker}{Ker}
\DeclareMathOperator{\cof}{cof}
\DeclareMathOperator{\rang}{rang}

%%\renewcommand{\def}{\buildrel\triangle\over =} ME DONNE UNE ERREUR DONC JE LE FAIS ``A LA MAIN''

%\hypertitle{Chimie}{3}{1302}{Nicolas Cognaux}{Nicolas Cognaux}
\begin{document}

\part{Gaz parfaits}
\section{Notions d'équation d'état}
Pour les corps purs mono atomiques, il existe une relation entre trois variables intensives (Pression $P$, Température $T$, Volume massique $V/m$), valable à l'équilibre. Celà signifie qu'en posant deux de ces variables, latroisième est déterminée par la relation.
Exemple: la relation de Van Der Waals.

Une variable intensive est une variable mesurable en un point d'un environnement. Par exemple la température dans une pièce est mesurable à tout point de la pièce.

\subsection{Boyle - Charles}
La loi de Boyle stipule que pour un volume donné, $pV = cste$.
La loi de Charles dit que le volume d'un gaz dépend de sa température: $V(\theta) = V(0^\circ C)(1 + \alpha \cdot \theta)$ ($\theta$ en degrés Celcius, $\alpha$ le coefficient d'expanson thermique).

$$\alpha_0 T[K] = (\frac{V(\theta)}{V(0^\circ C)})_{p \rightarrow 0}$$
Cette équation dit qu'à 0K, le volume d'un gaz est nulle et que le volume d'un gaz est proportionnel à la température.

Si on combine les deux equations, on retrouve $\frac{pV}{T} = A(m) = A_0 m$ donc la pression, le volume, la température et la masse sont toutes dépendantes à une constante près ($A_0$).

\section{Théorie de Bernoulli}
Hypothèses à prendre pour la suite de cette section:
\begin{itemize}
 \item Il n'y a aucunes pertes d'énergie dus à la friction des molécules;
 \item La parroie renvoit toujours une quantité d'énergie égale;
 \item les molécules sont très petites comparé à la distance qui les sépare.
\end{itemize}

\subsection{Estimation de la pression exercée par un gaz parfait}
Prenons un cube d'une longueur $L$ de coté, contenant une molécule de vitesse $\vec{c} = (c_x, c_y, c_z)$. Après collision avec une parroie (plan yz), la vitesse devient $\vec{c_+} = (-c_x, c_y, c_z)$.
Connaissant la vitesse et la distance entre les deux parroies opposées, on peut déterminer le temps qui sépare les deux collisions: $\Delta t = \frac{2L}{|c_x|}$.

Le changement de quantité de mouvement du à une collision avec la parroie est de $\Delta \vec{p} = m \cdot \vec{\Delta V} = m(-2c_x, 0, 0)$. On peut donc calculer la force exercée sur la parroie: $F_x = \frac{dp_x}{dt} \approx \frac{\Delta p_x}{\Delta t} = \frac{-m{c_x}^2}{L}$.

La pression exercée par {\bf une molécule} sur la parroie est donc la force exercée sur la surface totale de la parroie: $$p = \frac{m{c_x}^2}{L^2} = \frac{m{c_x}^2}{V}$$

Pour N molécules identiques, on peut donc faire la somme ce qui donne:
$$p = \frac{m}{V} \sum_{i=1}^{N}{({c_x}^2)_i}$$

A cause des collisions entre les molécules elles-mêmes, on doit passer par la vitesse moyenne, on définit donc: $$ \sum_{i=1}^{N}{({c_x}^2)_i} \buildrel\triangle\over = N \overline{{c_x}^2}$$ 


$$c^2 = {c_x}^2 + {c_y}^2 + {c_z}^2 \textrm{ et }\overline{c^2} = \overline{{c_x}^2} + \overline{{c_y}^2} + \overline{{c_z}^2}$$

Par le fait que le système est isotrope donc toutes les composantes de vitesses sont égales, on trouve: 
$$ \frac{1}{3}\overline{c^2} = \overline{{c_x}^2} $$

\subsubsection{Finalement}
Après tout ce développement, on retrouve:
$$pV = \frac{1}{3}Nm\overline{c^2}$$
On peut également poser $k = \frac{mc^2}{2}$ qui est l'énergie cinétique d'une molécule.
Par cette équation et l'équation d'état des gaz parfaits, on retrouve:
$$\overline{k} = \frac{3}{2}k_B T$$

\subsection{Energie interne d'un gaz}
Par l'hipothèse que l'énergie interne d'une molécule n'est constituée que de son énergie cinétique, on trouve que l'énergie totale d'un volume de gaz est:
$$U = N\overline{k} = \frac{3}{2}Nk_BT = \frac{3}{2}nRT$$
Cette énergie ne dépend donc que de {\bf la température} et $n$ est le nombre de moles de gaz.

\subsection{Lien entre énergie interne et capacité calorifique}
\subsubsection{Gaz mono-atomiques}
La capacité calorifique à volume constant est définie par: 
$$C_V \buildrel\triangle\over = (\frac{\partial Q}{\partial T})_V$$

Pour un gaz parfait, si le volume ne change pas, par le premier principe de thermodynamique, on a $(dU)_V = (dQ)_V$. On peut donc dire:

$$C_V \buildrel\triangle\over = (\frac{\partial Q}{\partial T})_V = (\frac{\partial U}{\partial T})_V$$

Comme on connait $U = \frac{3}{2}nRT$, on peut dire que $C_V = \frac{3}{2}nR$.

{\bf Attention, cette formule n'est valable que pour les gaz monoatomiques (gaz nobles).}

\subsubsection{Gaz diatomiques}
Une molécule diatomique a 5 degrés de liberté alors qu'une molécule diatomique n'en a que 3.
A chaque degré de liberté de translation ou de rotation, il y a une énergie d'agitation thermique donc une énergie interne. Cette énergie vaut: $\frac{1}{2}k_BT = \frac{1}{2}RT$ par mole.

Pour les degrés de liberté de vibration, il y a le double d'énergie car il y a l'énergie cinétique et potentielle.
On a alors, pour $X$ degrés de liberté:
$$ C_{Vm} = (\frac{\partial U_m}{\partial T})_V = \frac{(X_{translation} + X_{rotation} + 2X_{vibration})R}{2}$$ 

\subsection{Déviations par rapport aux gaz parfaits}
Pour un gaz réel, on a la relation: $\frac{pV_m}{RT} = Z$ où $Z$ est le facteur de compression, il dépend de $T$ et de $p$. Ce facteur est le facteur ``d'erreur'' de la relation des gaz parfaits.

Cette déviation est particulièrement importante près du point critique $(p_c, T_c, V_c)$ et à ce point, il faut des équations plus sophistiquées comme celle de Van der Waals.

\section{Distribution de Maxwell Boltzmann}
La théorie de Bernoulli ne permet pas de connaitre la répartition des vitesses des molécules elle ne donne que le vitesse moyenne. Boltzmann permet de déterminer statistiquement la distribution des vitesses des molécules dans un volume de gaz.

Soit une fonction $\Phi_i(c_i)$ la fraction des molécules dont la composante i de vitesse est inférieure à la vitesse $c_i$, $i$ peut donc représenter les trois dimensions.
Soit $\Phi(c)$ la fraction des molécules dont la vitesse totale $\in [0, c]$.
Pour un système isotrope, $\Phi_i(c_i) = \Phi_j(c_j) (\forall i, j)$

%%AJOUTER GRAPHES DU COURS slide 49

On doit donc utiliser une intégrale pour déterminer le nombre de molécules dont la vitesse est comprise entre $c$ et $c'$.
Par définition, $f_i(c_i) = \frac{d\Phi_i}{d(c_i)}$ et $f(c) = \frac{d\Phi}{d(c)}$. Pour un système isotrope c'est égal pour toutes les composantes.

On trouve donc: $$\Phi(c') - \Phi(c) = \int_{c}^{c'}{f(c)dc}$$

\subsection{Espace de phases}
Imaginons un repère virtuel dont les trois axes sont les composantes de vitesses. Ce repère est appelé espace de phases. Chaque vitesse de molécule est donc représentée par un vecteur.
La probabilité qu'une molécule ait ses composantes de vitesses comprises dans l'interval $(dc_1, dc_2, dc_3)$ qui est centré en $(c_1, c_2, c_3)$ est donc le produit des probabilités:
$$
dp_i = d\Phi_1 d\Phi_2 d\Phi_3 \\
dp_i = f_1(c_1)f_2(c_2)f_3(c_3)dc_1 dc_2 dc_3 \\
$$

Pour trouver la probabilité de retrouver une molécule dont $c$ est égal est donc l'intégrale en 3D d'une coquille sphérique centrée.

Comme le système est isotrope, on a une symétrie, ce qui permet de simplifier l'intégrale à:
\begin{eqnarray} 
dp &=& f_i(c_1)f_i(c_2)f_i(c_3) \iint_{c}{dc_1dc_2dc_3} \\
dp &=& f_i(c_1)f_i(c_2)f_i(c_3)\cdot4\pi c^2dc \\
dp &=& f(c)dc
\end{eqnarray}
Donc, on peut en conclure que 
$$ f_i(c_1)f_i(c_2)f_i(c_3)\cdot 4\pi c^2dc = f(c)dc$$
avec
$$ c^2 = c_x^2 + c_y^2 + c_z^2 $$
L'égalité n'est donc possible que si 
$$f_i(c_i) = A e^{-Bc_i^2} \Rightarrow f(c) = 4\pi c^2 A^3 e^{-Bc^2} $$

%%Ajouter graphes slide 60

\subsubsection{Déterminer les constantes $A$ et $B$}
Pour déterminer $B$, il suffit de calculer la moyenne de l'énergie cinétique grâce à la distribution:
$$ \frac{\int_{0}^{\infty}{\frac{1}{2}mc^2f(c)dc}}{\int_{0}^{\infty}{f(c)dc}} \buildrel\triangle\over = \frac{m\overline{c^2}}{2} = \frac{3}{2}k_BT \Rightarrow B = \frac{m}{2k_BT}$$

Pour déterminer $A$ il faut normaliser la distribution f intégrée sur toute les vitesses.

\subsubsection{Finalement}
Au final, on retrouve la fonction de distribution de Maxwell-Boltzmann:
$$ f(c) = 4\pi(\frac{m}{2\pi k_BT})^{3/2}c^2e^{\frac{-mc^2}{2k_BT}}$$

On remarque donc que cette distribution est proportionnelle à : $ f(c) \propto e^{\frac{-E}{k_BT}} $ où $E$ est l'énergie cinétique totale d'une molécule. Cette exponentielle est appellée {\bf facteur de Boltzmann}.

On remarque que pour un volume donnée, la forme de la fonction dépend de la température. Plus elle sera basse plus la courbe sera proche de l'axe des $Y$.
\end{document}
