\documentclass[fr]{../../../../../../eplexam}

\usepackage{../../../math-FSAB1103-exam}

\hypertitle{Math\'ematique}{3}{FSAB}{1103}{2016}{Ao\^ut}
{William André}
{Jean-François Remacle et Grégoire Winckelmans}

\section{}
Soit l'EDP suivante :
\[ P\fpart{u}{x} + Q\fpart{u}{y} = R\]
et la courbe $\Gamma(s)$ définie par $x(s)=r_0\cos(s)$ et $y(s)=r_0\sin(s)$ \\
On sait que l'équation des caractéristiques est de la forme $r = L\theta + B(r_0,s)$ avec $x = r\cos(\theta)$ et $y=r\sin(\theta)$
\begin{enumerate}
	\item Déterminez $B(r_0,s)$. Dessinez le réseau des caractéritiques pour $0\leq r \leq 2r_0$ et $s=-\frac{\pi}{4};0;\frac{\pi}{4};\frac{\pi}{2}$
	\item Dérivez l'équation des caractéristiques pour obtenir la forme de $P(x,y)$ et $Q(x,y)$
	\item En supposant $R=0$, déterminer la valeur de $u(2r_0,2r_0)$. En quelle(s) valeur(s) la solution n'est elle pas définie ?
\end{enumerate}


\begin{solution}
\end{solution}

\section{}
Soit l'EDP
\[ \fpart{u}{t} = \alpha\ffpart{u}{x} + Q_0\]
avec $\alpha > 0$. Le domaine est borné: $0 \leq x \leq L$.
La condition initiale est $u(x, 0) = \mu_0$ et $\fpart{u}{x}(L,0) = u_0$
avec $\mu_0 > 0$ constant.
Pour $t > 0$, on impose que $\fpart{u}{x}(0,t) = -\frac{\mu_0}{L}$
et que $u(L,t) = 0$.
\begin{enumerate}
  \item De quel type d'EDP s'agit-il mathématiquement et physiquement ? 
    Mathématiquement, qu'est-ce qu'un temps court pour cette EDP : $0<t<<...$?
  \item
    La solution est de la forme  $u(x,y) = R(x,t) + \Theta(x,t)$ avec $R(x,t)$ la solution en régime et $\Theta(x,t)$ la solution transitoire . Obtenez d'abord $R(x,y)$ et ensuite $\Theta(x,t)$, par la méthode de séparation des variables.
  \item
    Si on considère que $Q_0 = \frac{2\alpha\mu_0}{L^2}$, je ne sais plus la suite...

\end{enumerate}

\begin{solution}
\end{solution}


\section{}
On définit la fonction
\[ w = \frac{1}{i} \log{\left(\frac{\sqrt{z^2-1}+i}{z}\right)}\]
\begin{enumerate}
  \item Obtenez une expression pour $\sin(w)$ en fonction de $z$ (indice : l'expression est simple)
  \item Déterminez les points d'embranchement et les coupures. Faites le d'abord pour $\eta = \sqrt{z^2-1}+1$ puis pour $w$. Déterminez la coupure principale. On utilisera celle ci pour la suite de l'examen.
  \item Obtenez l'expression de $w$ pour $z=x$
  \item Obtenez l'expression de $w$ pour $z=iy$
  \item On peut montrer (ne le faites pas ici !) que $w = \int_0^Z{\frac{1}{1+\zeta^2}d\zeta}$ avec $Z=\frac{1}{z}$ \\Obtenez le développement en série de $w$ autour de $z_0 = 0$. Quel est le rayon de convergence de la série ?
    (\textbf{Aide}: $\frac{1}{\sqrt{1+\zeta}} = 1 - \frac{1}{2}\zeta^1 + \frac{1.3}{2.4}\zeta^2 - \frac{1.3.5}{2.4.6}\zeta^3 + ...$)
\end{enumerate}

\begin{solution}
\end{solution}

\section{}
Obtenez la valeur de l'intégrale suivante
\[\int_0^\infty{\frac{x^2\log(x)}{(x^2+1)^2}dx}\] en utilisant les lemmes de Jordan (donnés à l'examen)

\begin{solution}
\end{solution}

\end{document}
