\documentclass[fr]{../../../../../../eplexam}

\usepackage{../../../math-FSAB1103-exam}
\usepackage{../../../../../../eplunits}

\hypertitle{Math\'ematique}{3}{FSAB}{1103}{2015}{Janvier}
{Léa Paulus}
{Jean-François Remacle et Grégoire Winckelmans}

\section{(30$\%$)}
On considère l'EDP suivante pour $u(x,t)$ :
\begin{align*}
  \fpart{}{x}(cu) + \fpart{u}{t} &= S
\end{align*}
On a que $c=c(x) = c_0 \cos{(2 \pi \frac{x}{L})}$ avec $c_0$ et L constants. La condition initiale est $u(s,0) = u_0 \cdot \textrm{exp}(-\frac{s}{L_0})$ pour $s\geq 0$ avec $u_0$ une constante de même dimension que $u$ et $L_0$ une constante de longueur. La condition limite est $u(0, \tau) = u_0$ pour $\tau \geq 0$.

\begin{enumerate}
  \item
    Obtenez le réseau des caractéristiques  : équation pour la ``région A'' du quart de plan (i.e. relation entre $x$, $t$ et $s$) et équation  pour la ``région B'' du quart de plan (i.e. relation entre $x$, $t$ et $\tau$). Veuillez à finalement aussi exprimer explicitement $s$ en fonction de $x$ et $t$ (région A) et $\tau$ en fonction de $x$ et $t$ (région B).
  \item
    Faites un esquisse de $\frac{c(x)}{c_0}$. Faites ensuite une esquisse du réseau des caractéristiques (une esquisse propre, avec des axes adimensionnels et chiffrés !) en particulier esquissez celles qui émanent de $\frac{s}{L}=0,\frac{1}{4},\frac{1}{2},\frac{3}{4} \textrm{ et } 1$ et de $\frac{c_0 \tau}{L} = 0, \frac{1}{2}, 1$.
  \item
    Considérez le cas avec $S=0$.
    \begin{itemize}
      \item Obtenez la solution $u(x,t)$ dans la région A.
      \item Obtenez la solution $u(x,t)$ dans la région B.
      \item L'intégrale $I(t) = \int^{\infty}_0 u(x,t) dx$ est elle conservée au cours du temps et pourquoi ?
  \end{itemize}
  \item
    Considérez le cas avec $S=\frac{cu}{L}$
    \begin{itemize}
      \item Obtenez la solution $u(x,t)$ dans la région A.
      \item Obtenez la solution $u(x,t)$ dans la région B.
    \end{itemize}
\end{enumerate}

\subsection*{Aide}
\begin{align*}
\int \cos{(a \eta)} d\eta &= \dfrac{1}{a} \sin{(a\eta)}\\
\int \dfrac{d\eta}{\cos{(a \eta)}} &= \dfrac{1}{a} \cdot \ln\Big|\tan\Big(\dfrac{a \eta}{2} + \dfrac{\pi}{4}\Big)\Big|.
\end{align*}


\begin{solution}
On réécrit l'EDP comme suit
\begin{align*}
  c\fpart{u}{x} + \fpart{u}{t} &= -u\fpart{c}{x} + S
\end{align*}
\begin{enumerate}
\item
On a
\begin{align*}
  \dif t & = \frac{\dif x}{c_0 \cos\left(2 \pi \frac{x}{L}\right)}
\end{align*}
\begin{description}
\item[Région A]
On intègre de $(s,0)$ à $(x,t)$:
\begin{align*}
  \int_0^t \dif t & = \int_s^x \frac{\dif x}{c_0 \ln\left(2 \pi \frac{x}{L}\right)}\\
  t & = \frac{L}{2 \pi c_0}\left(\ln\Big|\tan\Big(\pi \frac{x}{L} + \frac{\pi}{4}\Big)\Big|-\ln\Big|\tan\Big(\pi \frac{s}{L} + \frac{\pi}{4}\Big)\Big|\right)\\
  \exp\Big(\frac{2 \pi tc_0}{L}\Big) & = \frac{\tan\Big(\pi \frac{x}{L} + \frac{\pi}{4}\Big)}{\tan\Big(\pi \frac{s}{L} + \frac{\pi}{4}\Big)}\\
  \tan\Big(\pi \frac{s}{L} + \frac{\pi}{4}\Big) & = \exp\Big(\frac{-2 \pi tc_0}{L}\Big)\tan\Big(\pi \frac{x}{L} + \frac{\pi}{4}\Big)\\
  s & = \frac{L}{\pi}\arctan\left(\exp\Big(\frac{-2 \pi tc_0}{L}\Big)\tan\Big(\pi \frac{x}{L} + \frac{\pi}{4}\Big)\right) - \frac{L}{4}.
\end{align*}

\item[Région B]
On intègre de $(0,\tau)$ à $(x,t)$:
\begin{align*}
  \int_\tau^t \dif t & = \int_0^x \frac{\dif x}{c_0 \ln\left(2 \pi \frac{x}{L}\right)}\\
  t-\tau & = \frac{L}{2 \pi c_0}\left(\ln\Big|\tan\Big(\pi \frac{x}{L} + \frac{\pi}{4}\Big)\Big|-\ln\Big|\tan\Big(\pi \frac{0}{L} + \frac{\pi}{4}\Big)\Big|\right)\\
  \tau & = t - \frac{L}{2 \pi c_0}\ln\Big|\tan\Big(\pi \frac{x}{L} + \frac{\pi}{4}\Big)\Big|.
\end{align*}
\end{description}
\item
\item
\item
\end{enumerate}
\end{solution}

\clearpage

\section{(25$\%$)}
On considère l'EDP suivante pour $u(x,t)$
\begin{align*}
\alpha \dfrac{\partial^2 u}{\partial x^2} = \dfrac{\partial u}{\partial t}
\end{align*}
avec $\alpha > 0$ constant. Le domaine est borné : $0 \leq x \leq L$. La condition initiale est $u(x,0) = u_0 (1-\frac{x}{L})$ avec $u_0 > 0$ constant. Pour $t>0$, les conditions aux limites sont $\frac{\partial u}{\partial x}(0,t) = 2 \frac{u_0}{L}$ et $u(L,t) = u_0$.
\begin{enumerate}
\item
De quel type d'EDP s'agit il, physiquement et mathématiquement ? Qu'est ce que $\alpha$ ? Qu'est ce qu'un temps ``court'' pour le problème présent, $0 < t \ll ...$ ?
\item
Sans encore résoudre mathématiquement le problème, esquissez la graphe attendu de $\frac{u}{u_0}$ en fonction de $\frac{x}{L}$ en des temps différents : $t=0$ (CI), temps court, temps moyen, temps long, temps très long (solution de régime).
\item
Obtenez ensuite, mathématiquement, la solution $u(x,t)$ du problème :
\begin{itemize}
\item Obtenez d'abord la solution en régime.
\item Obtenez ensuite la solution transitoire, avec séparation des variables.
\end{itemize}
\end{enumerate}

\subsection*{Aide}
\begin{align*}
\int \eta \sin{(a \eta)} d\eta &= \dfrac{1}{a^2} \sin{(a \eta)} - \dfrac{1}{a} \cdot \eta \cos{(a \eta)}\\
\int \eta \cos{(a \eta)} d\eta &= \dfrac{1}{a^2} \cos{(a \eta)} + \dfrac{1}{a} \cdot \eta \sin{(a \eta)}
\end{align*}


\nosolution
%\begin{solution}
%\end{solution}

\clearpage

\section{(25$\%$)}
On considère la fonction
\begin{align*}
w &= \arctan z\\
&= \dfrac{i}{2} \log{\dfrac{i}{i-z}}\\
&= \int^z_0 \dfrac{d\hat{z}}{1+\hat{z}^2} \quad \textrm{avec z $\neq \pm$ i}
\end{align*}

\begin{enumerate}
\item
Montrez que la définition de $w$ est bien telle que $\tan w = z$.
\item
Obtenez le(s) point(s) de branchement. Utilisez aussi une esquisse et proposez un choix de coupure(s) qui soit compatible avec le fait de pouvoir évaluer $w$ pour $z$ purement réel. Définissez aussi la branche principale, $D_0$. Pour la suite on n'utilisera plus que celle-ci.
\item
Obtenez l'expression de $w$ pour le cas $z = iy$ avec $-1 \leq y < 1$. (Esquisse). A partir de ce résultat, obtenez aussi l'expression de $\textrm{arctanh}( y)$
\item
Obtenez le développement en série de $w$ au tour de $z_0 = 0$. Quel est le rayon de convergence de cette série ? Pourquoi ?
\end{enumerate}

\subsection*{Aide}
\begin{align*}
\sin z &= \dfrac{1}{i} \sinh{(iz)} = \dfrac{1}{2i} (e^{iz}-e^{-iz})\\
\cos z &= \cosh{(iz)} = \dfrac{1}{2} (e^{iz}+e^{-iz})\\
\end{align*}
Pour $|Z| < 1$ on a aussi que $\dfrac{1}{1+Z} = 1-Z+Z^2-Z^3+...$ et que $\log{(1+Z)} = Z - \frac{1}{2} Z^2 + \frac{1}{3} Z^3$


\nosolution
%\begin{solution}
%\end{solution}

\clearpage

\section{(20$\%$)}
Résoudre
\begin{equation*}
\int^{\infty}_0 \dfrac{x^p dx}{1+x^2} \quad \textrm{avec }0 < p < 1
\end{equation*}
avec toutes les formes des lemmes de Jordan donnés.


\nosolution
%\begin{solution}
%\end{solution}

\end{document}
