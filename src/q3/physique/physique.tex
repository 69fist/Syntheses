\documentclass[11pt,a4paper]{article}

% French
\usepackage[utf8x]{inputenc}
\usepackage[frenchb]{babel}
\usepackage[T1]{fontenc}
\usepackage{lmodern}
\usepackage{ifthen}

% Color
% cfr http://en.wikibooks.org/wiki/LaTeX/Colors
\usepackage{color}
\usepackage[usenames,dvipsnames,svgnames,table]{xcolor}
\definecolor{dkgreen}{rgb}{0.25,0.7,0.35}
\definecolor{dkred}{rgb}{0.7,0,0}

% Floats and referencing
\newcommand{\sectionref}[1]{section~\ref{sec:#1}}
\newcommand{\annexeref}[1]{annexe~\ref{ann:#1}}
\newcommand{\figuref}[1]{figure~\ref{fig:#1}}
\newcommand{\tabref}[1]{table~\ref{tab:#1}}
\usepackage{xparse}
\NewDocumentEnvironment{myfig}{mm}
{\begin{figure}[!ht]\centering}
{\caption{#2}\label{fig:#1}\end{figure}}

% Listing
\usepackage{listings}
\lstset{
  numbers=left,
  numberstyle=\tiny\color{gray},
  basicstyle=\rm\small\ttfamily,
  keywordstyle=\bfseries\color{dkred},
  frame=single,
  commentstyle=\color{gray}=small,
  stringstyle=\color{dkgreen},
  %backgroundcolor=\color{gray!10},
  %tabsize=2,
  rulecolor=\color{black!30},
  %title=\lstname,
  breaklines=true,
  framextopmargin=2pt,
  framexbottommargin=2pt,
  extendedchars=true,
  inputencoding=utf8x
}

\newcommand{\matlab}{\textsc{Matlab}}
\newcommand{\octave}{\textsc{GNU/Octave}}
\newcommand{\qtoctave}{\textsc{QtOctave}}
\newcommand{\oz}{\textsc{Oz}}
\newcommand{\java}{\textsc{Java}}
\newcommand{\clang}{\textsc{C}}
\newcommand{\keyword}{mot clef}

% Math symbols
\usepackage{amsmath}
\usepackage{amssymb}
\usepackage{amsthm}
\DeclareMathOperator*{\argmin}{arg\,min}
\DeclareMathOperator*{\argmax}{arg\,max}

% Sets
\newcommand{\Z}{\mathbb{Z}}
\newcommand{\R}{\mathbb{R}}
\newcommand{\Rn}{\R^n}
\newcommand{\Rnn}{\R^{n \times n}}
\newcommand{\C}{\mathbb{C}}
\newcommand{\K}{\mathbb{K}}
\newcommand{\Kn}{\K^n}
\newcommand{\Knn}{\K^{n \times n}}

% Chemistry
\newcommand{\std}{\ensuremath{^{\circ}}}
\newcommand\ph{\ensuremath{\mathrm{pH}}}

% Theorem and definitions
\theoremstyle{definition}
\newtheorem{mydef}{Définition}
\newtheorem{mynota}[mydef]{Notation}
\newtheorem{myprop}[mydef]{Propriétés}
\newtheorem{myrem}[mydef]{Remarque}
\newtheorem{myform}[mydef]{Formules}
\newtheorem{mycorr}[mydef]{Corrolaire}
\newtheorem{mytheo}[mydef]{Théorème}
\newtheorem{mylem}[mydef]{Lemme}
\newtheorem{myexem}[mydef]{Exemple}
\newtheorem{myineg}[mydef]{Inégalité}

% Unit vectors
\usepackage{esint}
\usepackage{esvect}
\newcommand{\kmath}{k}
\newcommand{\xunit}{\hat{\imath}}
\newcommand{\yunit}{\hat{\jmath}}
\newcommand{\zunit}{\hat{\kmath}}

% rot & div & grad & lap
\DeclareMathOperator{\newdiv}{div}
\newcommand{\divn}[1]{\nabla \cdot #1}
\newcommand{\rotn}[1]{\nabla \times #1}
\newcommand{\grad}[1]{\nabla #1}
\newcommand{\gradn}[1]{\nabla #1}
\newcommand{\lap}[1]{\nabla^2 #1}


% Elec
\newcommand{\B}{\vec B}
\newcommand{\E}{\vec E}
\newcommand{\EMF}{\mathcal{E}}
\newcommand{\perm}{\varepsilon} % permittivity

\newcommand{\bigoh}{\mathcal{O}}
\newcommand\eqdef{\triangleq}

\DeclareMathOperator{\newdiff}{d} % use \dif instead
\newcommand{\dif}{\newdiff\!}
\newcommand{\fpart}[2]{\frac{\partial #1}{\partial #2}}
\newcommand{\ffpart}[2]{\frac{\partial^2 #1}{\partial #2^2}}
\newcommand{\fdpart}[3]{\frac{\partial^2 #1}{\partial #2\partial #3}}
\newcommand{\fdif}[2]{\frac{\dif #1}{\dif #2}}
\newcommand{\ffdif}[2]{\frac{\dif^2 #1}{\dif #2^2}}
\newcommand{\constant}{\ensuremath{\mathrm{cst}}}

% Numbers and units
\usepackage[squaren, Gray]{SIunits}
\usepackage{sistyle}
\usepackage[autolanguage]{numprint}
%\usepackage{numprint}
\newcommand\si[2]{\numprint[#2]{#1}}
\newcommand\np[1]{\numprint{#1}}

\newcommand\strong[1]{\textbf{#1}}
\newcommand{\annexe}{\part{Annexes}\appendix}

% Bibliography
\newcommand{\biblio}{\bibliographystyle{plain}\bibliography{biblio}}

\usepackage{fullpage}
% le `[e ]' rend le premier argument (#1) optionnel
% avec comme valeur par défaut `e `
\newcommand{\hypertitle}[7][e ]{
\usepackage{hyperref}
{\renewcommand{\and}{\unskip, }
\hypersetup{pdfauthor={#6},
            pdftitle={Synth\`ese d#1#2 Q#3 - L#4#5},
            pdfsubject={#2}}
}

\title{Synth\`ese d#1#2 Q#3 - L#4#5}
\author{#6}

\begin{document}

\ifthenelse{\isundefined{\skiptitlepage}}{
\begin{titlepage}
\maketitle

 \paragraph{Informations importantes}
   Ce document est grandement inspiré de l'excellent cours
   donné par #7 à l'EPL (École Polytechnique de Louvain),
   faculté de l'UCL (Université Catholique de Louvain).
   Il est écrit par les auteurs susnommés avec l'aide de tous
   les autres étudiants, la vôtre est donc la bienvenue.
   Il y a toujours moyen de l'améliorer, surtout si le cours
   change car la synthèse doit alors être modifiée en conséquence.
   On peut retrouver le code source à l'adresse suivante
   \begin{center}
     \url{https://github.com/Gp2mv3/Syntheses}.
   \end{center}
   On y trouve aussi le contenu du \texttt{README} qui contient de plus
   amples informations, vous êtes invité à le lire.

   Il y est indiqué que les questions, signalements d'erreurs,
   suggestions d'améliorations ou quelque discussion que ce soit
   relative au projet
   sont à spécifier de préférence à l'adresse suivante
   \begin{center}
     \url{https://github.com/Gp2mv3/Syntheses/issues}.
   \end{center}
   Ça permet à tout le monde de les voir, les commenter et agir
   en conséquence.
   Vous êtes d'ailleurs invité à participer aux discussions.

   Vous trouverez aussi des informations dans le wiki
   \begin{center}
     \url{https://github.com/Gp2mv3/Syntheses/wiki}.
   \end{center}
   comme le statut des synthèses pour chaque cours
   \begin{center}
     \url{https://github.com/Gp2mv3/Syntheses/wiki/Status}.
   \end{center}
   vous pouvez d'ailleurs remarquer qu'il en manque encore beaucoup,
   votre aide est la bienvenue.

   Pour contribuer au bug tracker et au wiki, il vous suffira de
   créer un compte sur Github.
   Pour interagir avec le code des synthèses,
   il vous faudra installer \LaTeX.
   Pour interagir directement avec le code sur Github,
   vous devez utiliser \texttt{git}.
   Si cela pose problème,
   nous sommes évidemment ouverts à des contributeurs envoyant leurs
   changements par mail ou n'importe quel autre moyen.
\end{titlepage}
}{}

\ifthenelse{\isundefined{\skiptableofcontents}}{
\tableofcontents
}{}
}


\usepackage{array}
\usepackage{fancybox}
\usepackage{float}
\usepackage{colortbl}
\usepackage{makecell}
\usepackage{graphicx}
\usepackage{titlesec}
\usepackage{qtree}
\usepackage{tensor}
\usepackage{circuitikz}

\hypertitle{Physique}{3}{1203}{Benoît Legat}{Benoît Legat}

\part{Ondes}
%   _     _     _     _     _     _     _...
%  / \   / \   / \   / \   / \   / \   / ...
% /   \_/   \_/   \_/   \_/   \_/   \_/  ...
\section{Courant de déplacement}
La loi d'ampère
\[ \oint \vec{B} \cdot \dif \vec{A} = \mu_0 \int \vec{J} \cdot \dif \vec{S} \]
est fausse dans certains cas.

Il faut rajouter un courant appelé courant de déplacement qui n'a pas
vraiment de sens physique pour qu'elle marche dans tous les cas.
Ce courant est déterminé par
\[ J_\mathrm{D} = \varepsilon_0\fpart{E}{t} \]
et l'équation d'Ampère correcte est alors
\[ \oint \vec{B} \cdot \dif \vec{A} =
\mu_0 \int \vec{J} \cdot \dif \vec{S} +
\mu_0\varepsilon_0 \int\fpart{\vec{E}}{t} \cdot \dif \vec{S} \]

\section{Équations de Maxwell}
On peut résumer toutes les lois qui gouvernent l'électromagnétisme
par 4 lois que les théorèmes intégraux nous permettent de réécrire
de 2 manière différentes
\begin{align*}
  \oint \E \cdot \dif \vec{A} & = \frac{Q}{\varepsilon}
  & \divn{\E} & = \frac{\rho}{\varepsilon}\\
  \oint \B \cdot \dif \vec{A} & = 0
  & \divn{\B} & = 0\\
  \oint \E \cdot \dif \vec{l} & = -\fpart{}{t} \int \B \cdot \dif \vec{A}
  & \rotn{\E} & = -\fpart{\B}{t}\\
  \oint \B \cdot \dif \vec{l} & = \mu \int \vec{J} \cdot \dif \vec{A}
  + \mu\epsilon\fpart{}{t} \int \E \dif \vec{A}
  & \rotn{\B} & = \mu\vec{J} + \mu\epsilon\fpart{\E}{t}
\end{align*}
avec
\begin{align*}
  \vec{E} & = -\gradn{V}\\
  Q & = \int \rho \dif V.
\end{align*}

\section{Ondes}
Une onde a l'équation suivante
\[ A(x, t) = A_0 \sin(\vec{k} \cdot \vec{x} - \omega t + \phi) \]
et les caractéristiques suivantes
\begin{itemize}
  \item $A_0$, son amplitude, c'est à dire son intensité maximale;
  \item $\omega$, sa vitesse angulaire;
  \item $f$, sa fréquence d'oscillation;
  \item $T$, sa période d'oscillation;
  \item $\vec{k}$, le vecteur d'onde;
  \item $\lambda$, la longueur d'onde;
  \item $v$, sa vitesse de propagation;
  \item $\phi$, une phase.
\end{itemize}
Elles sont liées entre elles par les relations suivantes
\begin{align*}
  T & = \frac{1}{f}\\
  \omega & = 2\pi f\\
  v & = \lambda f\\
  \vec{k} & = \frac{2\pi}{\lambda}\hat{v}.
\end{align*}

Il existe deux types d'ondes,
\begin{itemize}
  \item les ondes transverses où $\vec{A} \perp \vec{v}$;
  \item les ondes longitudinales où $\vec{A} \parallel \vec{v}$.
\end{itemize}

\subsection{Ondes électromagnétiques}
Les ondes électromagnétiques sont des ondes transverses composées
de $\vec{E}$ et $\vec{B}$.

Ces ondes voyagent à la vitesse de la lumière $c$
et on écrit leur fréquence $\nu$.
On a les relations supplémentaires
\begin{align*}
  \vec{E} & \perp \vec{B}\\
  c & = \frac{1}{\sqrt{\varepsilon\mu}}\\
  \frac{E}{B} & = c
\end{align*}

\subsection{Ondes mécaniques}
\subsubsection{Corde}
On peut appliquer une onde transverse dans une corde.
Si la corde est tendue avec une force $F$ et que
sa masse linéique $[\kilogram\per\meter]$ est $m_L$,
la vitesse de propagation d'une telle onde est
\[ v = \sqrt{\frac{F}{m_L}}. \]

\subsubsection{Son}
Le son est une onde longitudinale de compression de l'air.
La vitesse de propagation du son dans l'air est
$v = \si{344}{\meter\per\second}$.

Pour mesurer l'intensité d'une onde, on utilise une échelle logarithmique
\[ I[\deci\bel] = 10 \log_{10}\frac{I[\bel]}{I_0} \]

\subsection{Effet Doppler}
Soit une source se déplaçant à une vitesse $v_s$ et émettant
une onde de vitesse $v$ et de fréquence $f_s$.
Soit une observateur se déplaçant à une vitesse $v_o$ et
observant cette onde à une fréquence $f_o$.

On suppose que les trois vitesses sont parallèles.

\subsection{Pour une onde non-électromagnétique}
Si on a $v \ll c$, on peut dire que
\[ f_o = \frac{v \pm v_o}{v \pm v_s} f_s \]

Il faut utiliser le bon sens pour savoir si c'est un plus ou un moins.
Par exemple, si la source se rapproche de l'observateur, et que
l'observateur se rapproche de la source, les deux vitesses
tendent à augmenter la fréquence donc
\[ f_o = \frac{v + v_o}{v - v_s} f_s \]

\subsection{Pour une onde électromagnétique}
Si $v = c$, on a
\[ f_o = \sqrt{\frac{c + v_s}{c - v_s}} f_s
\approx \left(1 + \frac{v_s}{c}\right) f_s \]

D'où
\[ \frac{f_o - f_s}{f_s} \approx \frac{v_s}{c} \]

\section{Polarisation, réflexion et réfraction}
% \   /
%  \ /
% -----
%    \
%     \
\subsection{Polarisation}
Le type de polarisation indique la forme du lieu parcouru
par un champ électrique.

Supposons que la l'onde se déplace selon $\vec{z}$,
une polarisation générale serait
\[ \vec{E}(\vec{r}, t) = A_x \sin(\vec{k}\cdot\vec{r} - \omega t) \xunit
+ A_y \sin(\vec{k} \cdot \vec{r} - \omega t + \phi) \yunit \]
C'est ce qu'on appelle on polarisation elliptique.

\begin{itemize}
  \item Si $\phi = 0$ et $A_x = A_y$, c'est une polarisation circulaire;
  \item Si $A_x = 0$ ou $A_y = 0$, c'est une polarisation linéaire.
\end{itemize}

\subsection{Réflexion et réfraction}
Lorsqu'une onde passe d'un milieu à un autre,
elle se réfléchit en une onde \emph{réfléchie} et se réfracte en une onde
\emph{transmise} à la surface de séparation des deux milieu.

\paragraph{Attention}
Une onde électromagnétique ne passe pas à travers une membrane métallique.

\subsubsection{Réflexion}
La vitesse angulaire $\omega$, la fréquence $\nu$, la vitesse $v$ et la
longueur d'onde $\lambda$.

\begin{itemize}
  \item Si $n_1 > n_2$, elles sont en phase avec l'onde incidente;
  \item Si $n_1 < n_2$, elles sont déphasées de $\pi$ avec l'onde incidente.
\end{itemize}

L'angle réfléchi est égal à l'angle d'incidence.

\subsubsection{Réfraction}
Les ondes transmises ont la même vitesse angulaire $\omega$
et la même fréquence $\nu$ que l'onde incidente
mais pas la même vitesse $v$ ni la même longueur d'onde $\lambda$.

Elles sont en phase avec l'onde incidente.

\paragraph{Indice de réfraction}
On définit les indices de réfraction comme suit
\[ \frac{n_1}{n_2} \eqdef \frac{v_2}{v_1} = \frac{\lambda_2}{\lambda_1} \]

Dans le cas d'une onde électromagnétique,
$v = c = \frac{1}{\sqrt{\varepsilon\mu}}$ et donc
$n = \sqrt{\varepsilon_r\mu_r}$.
On voit d'ailleurs que $n_\mathrm{vide} = 1$.

\paragraph{Loi de Snell-Descartes}
L'angle réfléchi $\theta_2$ par rapport à l'angle d'incidence
$\theta_1$ nous est donné par la loi de \emph{Snell-Descartes}
\begin{equation}
  \label{eq:snell}
  n_1 \sin(\theta_i) = n_2 \sin(\theta_r)
\end{equation}

\subsubsection{Équations de Fresnel}
Pour trouver l'intensité de l'onde réfléchie et transmise, il nous
faut décomposer son intensité en une composante parallèle au plan d'incidence
$E^\parallel$ et une perpendiculaire $E^\perp$.

Soit $E_{1r}$ l'intensité du champ réfléchi
et $E_{2}$ l'intensité du champ transmi.
On a les formules suivantes
\begin{align*}
  E_{1r}^\parallel & = \frac{n_1\cos\theta_2 - n_2\cos\theta_1}
  {n_1\cos\theta_2 + n_2\cos\theta_1}E_{1}^\parallel
  \stackrel{\eqref{eq:snell}}{=}
  \frac{\tan(\theta_2 - \theta_1)}{\tan(\theta_2 + \theta_1)}E_1^\parallel
  & E_2^\parallel & = \frac{2n_1\cos(\theta_1)}
  {n_1\cos\theta_2 - n_2\cos\theta_1}E_{1}^\parallel\\
  E_{1r}^\perp & = \frac{n_1\cos\theta_1 - n_2\cos\theta_2}
  {n_1\cos\theta_1 + n_2\cos\theta_2}E_{1}^\perp
  \stackrel{\eqref{eq:snell}}{=}
  \frac{\sin(\theta_2 - \theta_1)}{\sin(\theta_2 + \theta_1)}E_1^\perp
  & E_2^\perp & = \frac{2n_1\cos(\theta_1)}
  {n_1\cos\theta_1 - n_2\cos\theta_2}E_{1}^\perp\\
\end{align*}

\paragraph{Angle de Brewster}
Il existe un angle pour lequel $E_{1r}^\parallel = 0$,
on l'appelle l'angle de Brewster et c'est l'angle $\theta_B$ tel que
\[ \tan\theta_B = \frac{n_2}{n_1} \]

\paragraph{Angle critique}
Si $n_1 > n_2$, il existe un angle critique $\theta_{1c}$ tel que
$E_{2} = 0$ pour tout $\theta_1 \geq \theta_{1c}$.
C'est l'angle qui respecte
\[ \sin\theta_{1c} = \frac{n_2}{n_1}. \]

On voit bien ici pourquoi, si $n_1 < n_2$, cet angle n'existe pas.

\section{Interférence et diffraction}
Quand une onde traverse une fente ou un objet, elle subit une diffraction.
Si c'est un objet et non une fente, l'effet est le même,
on ne traitera donc que le cas de fentes.
On considère aussi que le point où on mesure l'onde est loin des fentes
par rapport à leur distance entre elles.

\subsection{Interférence}
Si une onde arrive sur deux fentes de largeur négligeable séparées par
une distance $d$,
il y aura interférence constructive si $\exists n \in \mathbb{N}$
\[ d\sin\theta = n \lambda \]
et interférence destructive si $\exists n \in \mathbb{N}$
\[ d\sin\theta = \left(n+\frac{1}{2}\right) \lambda \]
où $\theta$ est l'angle entre la perpendiculaire aux deux fentes et
la distance entre le milieu des deux fentes et le point en question.

Si on considère $N$ fentes de largeur négligeable séparées par une
distance $d$,
\[ I(P) \propto \frac{A^2}{R^2} \times
  \frac{\sin^2\left(\frac{N \pi d \sin\theta}{\lambda}\right)}
{\sin^2\left(\frac{\pi d \sin\theta}{\lambda}\right)} \]
où $R$ est la distance entre $P$ et le milieu des fentes.

Ce qui donne des minima si
$\exists n \in \mathbb{N}_0$ non divisibles par $N$ tel que
\[ N d \sin \theta = n \lambda \]
et des maxima si $\exists n \in \mathbb{N}$ tel que
\[ d \sin \theta = n \lambda \]

\subsection{Diffraction}
Supposons maintenant que l'épaisseur $a$ d'une fente n'est plus
négligeable.
\begin{itemize}
  \item Pour une fente,
    \[ I(P) \propto I_0 \times
      \frac{\sin^2\left(\frac{\pi a \sin\theta}{\lambda}\right)}
    {\left(\frac{\pi a \sin\theta}{\lambda}\right)^2}; \]
  \item et pour $N$ fentes,
    \[ I(P) \propto I_0 \times
      \frac{\sin^2\left(\frac{\pi a \sin\theta}{\lambda}\right)}
      {\left(\frac{\pi a \sin\theta}{\lambda}\right)^2} \times
      \frac{\sin^2\left(\frac{N \pi d \sin\theta}{\lambda}\right)}
    {\sin^2\left(\frac{\pi d \sin\theta}{\lambda}\right)}. \]
\end{itemize}
On a des minima si $\exists n \in \mathbb{N}_0$ tel que
\[ a \sin \theta = n\pi \]

Si $N \neq 1$, on a aussi les minimas et maximas qu'avait
l'interférence à $N$ fentes.

\section{Ondes stationnaire}
Une onde stationnaire est une onde pour laquelle ses noeuds ne bougent pas.
Elle est obtenue en fixant des noeuds où des ventres à ses extrémités
Son équation est la suivante
\[ A \sin(kx) \sin(\omega t) \]
% modulation d'amplitude, vitesse de groupe
% rayonnement électromagnétique et antennes

\end{document}
