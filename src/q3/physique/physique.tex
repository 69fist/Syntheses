\documentclass[11pt,a4paper]{article}

% French
\usepackage[utf8x]{inputenc}
\usepackage[frenchb]{babel}
\usepackage[T1]{fontenc}
\usepackage{lmodern}
\usepackage{ifthen}

% Color
% cfr http://en.wikibooks.org/wiki/LaTeX/Colors
\usepackage{color}
\usepackage[usenames,dvipsnames,svgnames,table]{xcolor}
\definecolor{dkgreen}{rgb}{0.25,0.7,0.35}
\definecolor{dkred}{rgb}{0.7,0,0}

% Floats and referencing
\newcommand{\sectionref}[1]{section~\ref{sec:#1}}
\newcommand{\annexeref}[1]{annexe~\ref{ann:#1}}
\newcommand{\figuref}[1]{figure~\ref{fig:#1}}
\newcommand{\tabref}[1]{table~\ref{tab:#1}}
\usepackage{xparse}
\NewDocumentEnvironment{myfig}{mm}
{\begin{figure}[!ht]\centering}
{\caption{#2}\label{fig:#1}\end{figure}}

% Listing
\usepackage{listings}
\lstset{
  numbers=left,
  numberstyle=\tiny\color{gray},
  basicstyle=\rm\small\ttfamily,
  keywordstyle=\bfseries\color{dkred},
  frame=single,
  commentstyle=\color{gray}=small,
  stringstyle=\color{dkgreen},
  %backgroundcolor=\color{gray!10},
  %tabsize=2,
  rulecolor=\color{black!30},
  %title=\lstname,
  breaklines=true,
  framextopmargin=2pt,
  framexbottommargin=2pt,
  extendedchars=true,
  inputencoding=utf8x
}

\newcommand{\matlab}{\textsc{Matlab}}
\newcommand{\octave}{\textsc{GNU/Octave}}
\newcommand{\qtoctave}{\textsc{QtOctave}}
\newcommand{\oz}{\textsc{Oz}}
\newcommand{\java}{\textsc{Java}}
\newcommand{\clang}{\textsc{C}}
\newcommand{\keyword}{mot clef}

% Math symbols
\usepackage{amsmath}
\usepackage{amssymb}
\usepackage{amsthm}
\DeclareMathOperator*{\argmin}{arg\,min}
\DeclareMathOperator*{\argmax}{arg\,max}

% Sets
\newcommand{\Z}{\mathbb{Z}}
\newcommand{\R}{\mathbb{R}}
\newcommand{\Rn}{\R^n}
\newcommand{\Rnn}{\R^{n \times n}}
\newcommand{\C}{\mathbb{C}}
\newcommand{\K}{\mathbb{K}}
\newcommand{\Kn}{\K^n}
\newcommand{\Knn}{\K^{n \times n}}

% Chemistry
\newcommand{\std}{\ensuremath{^{\circ}}}
\newcommand\ph{\ensuremath{\mathrm{pH}}}

% Theorem and definitions
\theoremstyle{definition}
\newtheorem{mydef}{Définition}
\newtheorem{mynota}[mydef]{Notation}
\newtheorem{myprop}[mydef]{Propriétés}
\newtheorem{myrem}[mydef]{Remarque}
\newtheorem{myform}[mydef]{Formules}
\newtheorem{mycorr}[mydef]{Corrolaire}
\newtheorem{mytheo}[mydef]{Théorème}
\newtheorem{mylem}[mydef]{Lemme}
\newtheorem{myexem}[mydef]{Exemple}
\newtheorem{myineg}[mydef]{Inégalité}

% Unit vectors
\usepackage{esint}
\usepackage{esvect}
\newcommand{\kmath}{k}
\newcommand{\xunit}{\hat{\imath}}
\newcommand{\yunit}{\hat{\jmath}}
\newcommand{\zunit}{\hat{\kmath}}

% rot & div & grad & lap
\DeclareMathOperator{\newdiv}{div}
\newcommand{\divn}[1]{\nabla \cdot #1}
\newcommand{\rotn}[1]{\nabla \times #1}
\newcommand{\grad}[1]{\nabla #1}
\newcommand{\gradn}[1]{\nabla #1}
\newcommand{\lap}[1]{\nabla^2 #1}


% Elec
\newcommand{\B}{\vec B}
\newcommand{\E}{\vec E}
\newcommand{\EMF}{\mathcal{E}}
\newcommand{\perm}{\varepsilon} % permittivity

\newcommand{\bigoh}{\mathcal{O}}
\newcommand\eqdef{\triangleq}

\DeclareMathOperator{\newdiff}{d} % use \dif instead
\newcommand{\dif}{\newdiff\!}
\newcommand{\fpart}[2]{\frac{\partial #1}{\partial #2}}
\newcommand{\ffpart}[2]{\frac{\partial^2 #1}{\partial #2^2}}
\newcommand{\fdpart}[3]{\frac{\partial^2 #1}{\partial #2\partial #3}}
\newcommand{\fdif}[2]{\frac{\dif #1}{\dif #2}}
\newcommand{\ffdif}[2]{\frac{\dif^2 #1}{\dif #2^2}}
\newcommand{\constant}{\ensuremath{\mathrm{cst}}}

% Numbers and units
\usepackage[squaren, Gray]{SIunits}
\usepackage{sistyle}
\usepackage[autolanguage]{numprint}
%\usepackage{numprint}
\newcommand\si[2]{\numprint[#2]{#1}}
\newcommand\np[1]{\numprint{#1}}

\newcommand\strong[1]{\textbf{#1}}
\newcommand{\annexe}{\part{Annexes}\appendix}

% Bibliography
\newcommand{\biblio}{\bibliographystyle{plain}\bibliography{biblio}}

\usepackage{fullpage}
% le `[e ]' rend le premier argument (#1) optionnel
% avec comme valeur par défaut `e `
\newcommand{\hypertitle}[7][e ]{
\usepackage{hyperref}
{\renewcommand{\and}{\unskip, }
\hypersetup{pdfauthor={#6},
            pdftitle={Synth\`ese d#1#2 Q#3 - L#4#5},
            pdfsubject={#2}}
}

\title{Synth\`ese d#1#2 Q#3 - L#4#5}
\author{#6}

\begin{document}

\ifthenelse{\isundefined{\skiptitlepage}}{
\begin{titlepage}
\maketitle

 \paragraph{Informations importantes}
   Ce document est grandement inspiré de l'excellent cours
   donné par #7 à l'EPL (École Polytechnique de Louvain),
   faculté de l'UCL (Université Catholique de Louvain).
   Il est écrit par les auteurs susnommés avec l'aide de tous
   les autres étudiants, la vôtre est donc la bienvenue.
   Il y a toujours moyen de l'améliorer, surtout si le cours
   change car la synthèse doit alors être modifiée en conséquence.
   On peut retrouver le code source à l'adresse suivante
   \begin{center}
     \url{https://github.com/Gp2mv3/Syntheses}.
   \end{center}
   On y trouve aussi le contenu du \texttt{README} qui contient de plus
   amples informations, vous êtes invité à le lire.

   Il y est indiqué que les questions, signalements d'erreurs,
   suggestions d'améliorations ou quelque discussion que ce soit
   relative au projet
   sont à spécifier de préférence à l'adresse suivante
   \begin{center}
     \url{https://github.com/Gp2mv3/Syntheses/issues}.
   \end{center}
   Ça permet à tout le monde de les voir, les commenter et agir
   en conséquence.
   Vous êtes d'ailleurs invité à participer aux discussions.

   Vous trouverez aussi des informations dans le wiki
   \begin{center}
     \url{https://github.com/Gp2mv3/Syntheses/wiki}.
   \end{center}
   comme le statut des synthèses pour chaque cours
   \begin{center}
     \url{https://github.com/Gp2mv3/Syntheses/wiki/Status}.
   \end{center}
   vous pouvez d'ailleurs remarquer qu'il en manque encore beaucoup,
   votre aide est la bienvenue.

   Pour contribuer au bug tracker et au wiki, il vous suffira de
   créer un compte sur Github.
   Pour interagir avec le code des synthèses,
   il vous faudra installer \LaTeX.
   Pour interagir directement avec le code sur Github,
   vous devez utiliser \texttt{git}.
   Si cela pose problème,
   nous sommes évidemment ouverts à des contributeurs envoyant leurs
   changements par mail ou n'importe quel autre moyen.
\end{titlepage}
}{}

\ifthenelse{\isundefined{\skiptableofcontents}}{
\tableofcontents
}{}
}


\usepackage{pgfplots}
\usepackage{caption}
\usepackage{subcaption}

\hypertitle{Physique}{3}{FSAB}{1203}{Benoît Legat et Nicolas Cognaux}
{Jérômes Louveaux, Claude Oestges et Jean-Christophe Charlier}

\part{Ondes}
%   _     _     _     _     _     _     _...
%  / \   / \   / \   / \   / \   / \   / ...
% /   \_/   \_/   \_/   \_/   \_/   \_/  ...
\section{Courant de déplacement}
La loi d'ampère
\[ \oint \vec{B} \cdot \dif \vec{l} = \mu_0 \int \vec{J} \cdot \dif \vec{A} \]
est fausse dans certains cas.

Il faut rajouter un courant appelé courant de déplacement qui n'a pas
vraiment de sens physique pour qu'elle marche dans tous les cas.
Ce courant est déterminé par
\[ J_\mathrm{D} = \perm_0\fpart{E}{t} \]
et l'équation d'Ampère correcte est alors
\[ \oint \vec{B} \cdot \dif \vec{l} =
\mu_0 \int \vec{J} \cdot \dif \vec{A} +
\mu_0\perm_0 \int\fpart{\vec{E}}{t} \cdot \dif \vec{A}. \]

\section{Équations de Maxwell}
On peut résumer toutes les lois qui gouvernent l'électromagnétisme
par 4 lois que les théorèmes intégraux nous permettent de réécrire
de 2 manière différentes
\begin{align*}
  &\text{Gauss} & \oint \E \cdot \dif \vec{A} & = \frac{Q}{\perm}
  & \divn{\E} & = \frac{\rho}{\perm}\\
  &\text{No monopôle magn.} & \oint \B \cdot \dif \vec{A} & = 0
  & \divn{\B} & = 0\\
  &\text{Lenz-Faraday} & \oint \E \cdot \dif \vec{l} & = -\fpart{}{t} \int \B \cdot \dif \vec{A}~~~\footnote{Les physiciens qui aiment la symétrie cherchent un terme supplémentaire ici.}
  & \rotn{\E} & = -\fpart{\B}{t}\\
  &\text{Maxwell-Ampère} & \oint \B \cdot \dif \vec{l} & = \mu \int \vec{J} \cdot \dif \vec{A}
  + \mu\perm\fpart{}{t} \int \E \dif \vec{A}
  & \rotn{\B} & = \mu\vec{J} + \mu\perm\fpart{\E}{t}
\end{align*}
avec
\begin{align*}
  \vec{E} & = -\gradn{V}\\
  Q & = \int \rho \dif V.
\end{align*}

\section{Ondes}
\subsection{Equations de propagation et équation d'onde}
En émettant les hypothèses que $\vec{E}$ est suivant $\hat{y}$ et $\vec{H}$ suivant $\hat{z}$ et à l'aide des équations de Maxwell, on trouve que $\vec{E}$ et $\vec{H}$ ne dépendent que de leur abscisse $x$. Les champs électrique et magnétique sont donc constants le long d'un plan perpendiculaire à l'axe des abscisses. On obtient également les relations suivantes, appelées les \textit{équations de propagation} d'une onde électromagnétique
\begin{align*}
\fpart{E_y}{x} &= - \mu \fpart{H_z}{t} & \fpart{H_z}{x} &= - \perm \fpart{E_y}{t}.
\end{align*}
En appliquant $\fpart{}{t}$ et $\fpart{}{x}$ à ces relations, on obtient ce qu'on nomme \textit{l'équation d'onde} de $H_z$ et $E_y$ respectivement
\begin{align*}
\dfrac{\partial^2 H_z}{\partial t^2} &=
c^2 \dfrac{\partial^2 H_z}{\partial x^2}
& \dfrac{\partial^2 E_y}{\partial t^2} &=
c^2 \dfrac{\partial^2 E_y}{\partial x^2}.
\end{align*}
En fait, les équations susnommées ne sont pas particulières aux ondes électromagnétiques. Toutes les ondes respectent ces équations. En toute généralité, si 2 fonctions $A(x,t)$ et $B(x,t)$ sont liées par les équations de propagation
\begin{align*}
\fpart{A}{x} &= - a \fpart{B}{t} & \fpart{A}{t} &= - b \fpart{B}{x},
\end{align*}
alors elles obéissent toutes deux à la même équation d'onde
\begin{align*}
\dfrac{\partial^2 A}{\partial t^2} &=
v^2 \dfrac{\partial^2 A}{\partial x^2}
& \dfrac{\partial^2 B}{\partial t^2} &=
v^2 \dfrac{\partial^2 B}{\partial x^2}.
\end{align*}
On peut se demander s'il existe une fonction à une variable $C(w)$ vérifiant cette équation, c'est-à-dire $C(w) = A(x,t)$, et si oui quelle est l'expression de $w(x,t)$. La réponse est oui, si on prend\footnote{Cela fonctionne également si on prend $w_2(x,t) = t-\dfrac{x}{v}$, mais la fonction à une variable correspondante sera alors différente bien sûr, $D(w_2)$ par exemple.} $w(x,t) = x - vt$, on vérifie\footnote{à l'aide de la \textit{Chain rule}} bel et bien
$ \dfrac{\partial^2 C(w)}{\partial t^2} =
v^2 \dfrac{\partial^2 C(w)}{\partial x^2} $. Pour toute fonction $A(x,t)$ vérifiant l'équation d'onde, on peut donc trouver\footnote{Cela fonctionne également avec des + à la place des - (vous comprendrez pourquoi si vous utilisez la Chain rule), mais peu importe puisque $v$ peut être positif ou négatif.}
\begin{itemize}
\item Une fonction à une variable $C$ telle que $A(x,t) = C(x-vt)$ ;
\item Une fonction à une variable $D$ telle que $A(x,t) = D(t-\dfrac{x}{v})$.
\end{itemize}
Cela correspond bien à l'intuition. En effet, c'est comme si on avait une fonction $C(x)$ (p. ex. une sinusoïde ou un gaussienne) qui se déplaçait dans le temps à une vitesse $v$.

En l'absence de constante d'intégration\footnote{Le champ magnétique de la Terre dans le cas d'une onde EM, la hauteur de la corde au repos dans le cas d'une corde vibrante ...}, le rapport de $A$ sur $B$ vaut une constante particulière dénotée $Z$
\[
\dfrac{A(x-vt)}{B(x-vt)} = Z = \sqrt{ab}.
\]
Dans le cas d'une onde EM,
\[
Z = \sqrt{\dfrac{\mu}{\perm}} \overset{\text{si dans le vide}}{=} \sqrt{\dfrac{\mu_0}{\perm_0}} = \unit{120\pi}{\ohm}.
\]
Finalement, en prenant une sinusoïde pour $C(x)$ et en l'étoffant quelque peu, on trouve l'expression générale \eqref{eq:expr_gen_onde}.


\subsection{Expression générale}
Il est important, pour commencer, d'insister sur le fait
que les ondes transportent de l'énergie et non de la matière
d'un point à un autre (logique dans le cas d'une corde vibrante).

Une onde est exprimée de la façon suivante
\begin{equation} \label{eq:expr_gen_onde}
A(\vec{x}, t) = A_0 \sin(\vec{k} \cdot \vec{x} - \omega t + \phi)
\end{equation}
et les caractéristiques suivantes
\begin{itemize}
  \item $A_0$, son amplitude, c'est à dire son intensité maximale;
  \item $\vec{x}$, le vecteur position du point (cas général pour une onde 3D);
  \item $\omega$, sa vitesse angulaire;
  \item $f$, sa fréquence d'oscillation;
  \item $T$, sa période d'oscillation;
  \item $\vec{k}$, le vecteur d'onde;
  \item $\lambda$, la longueur d'onde;
  \item $v$, sa vitesse de propagation;
  \item $\phi$, une phase.
\end{itemize}
Elles sont liées entre elles par les relations suivantes
\begin{align*}
  T & = \frac{1}{f}\\
  \omega & = 2\pi f\\
  v & = \lambda f\\
  \vec{k} & = \frac{2\pi}{\lambda}\hat{v}.
\end{align*}

Il existe deux types d'ondes,
\begin{itemize}
  \item les ondes transverses où $\vec{A} \perp \vec{v}$ (par exemple une corde);
  \item les ondes longitudinales où $\vec{A} \parallel \vec{v}$ (ex: le son).
\end{itemize}

\subsection{Ondes électromagnétiques}
Les ondes électromagnétiques sont des ondes transverses composées
de $\vec{E}$ et $\vec{B}$.

Ces ondes voyagent à la vitesse de la lumière $c$
et on écrit leur fréquence $\nu$.
On a les relations supplémentaires
\begin{align*}
  \vec{E} & \perp \vec{B}\\
  c & = \frac{1}{\sqrt{\perm\mu}}\\
  \frac{E}{B} & = c.
\end{align*}

\subsection{Ondes mécaniques}

\subsubsection{Corde}
On peut appliquer une onde transverse dans une corde.
Si la corde est tendue avec une force $F$ et que
sa masse linéique $[\kilogram\per\meter]$ est $m_L$,
la vitesse de propagation d'une telle onde est
\[ v = \sqrt{\frac{F}{m_L}}. \]

\subsubsection{Son}
Le son est une onde longitudinale de compression de l'air.
La vitesse de propagation du son dans l'air à $298K$ est
$v = \si{344}{\meter\per\second}$.

La vitesse du son varie avec la température, suivant la loi des gaz parfaits.
Nous avons donc la relation $v = \sqrt{\frac{B}{\rho_0}} \simeq 20.1\sqrt{T}$

Pour mesurer l'intensité d'une onde, on utilise une échelle logarithmique
\[ I[\deci\bel] = 10 \log_{10}\frac{I[\watt\per\meter\squared]}{I_0} \]
où $I_0 = \si{1e-12}{\watt\per\meter\squared}$.

\subsection{Effet Doppler}
Soit une source se déplaçant à une vitesse $v_s$ et émettant
une onde de vitesse $v$ et de fréquence $f_s$.
Soit une observateur se déplaçant à une vitesse $v_o$ et
observant cette onde à une fréquence $f_o$.

On suppose que les trois vitesses sont parallèles.

\subsubsection{Pour une onde non-électromagnétique}
Si on a $v \ll c$, on peut dire que
\[ f_o = \frac{v \pm v_o}{v \pm v_s} f_s \]

Il faut utiliser le bon sens pour savoir si c'est un plus ou un moins.
Par exemple, si la source se rapproche de l'observateur, et que
l'observateur se rapproche de la source, les deux vitesses
tendent à augmenter la fréquence perçue par l'observateur donc
\[ f_o = \frac{v + v_o}{v - v_s} f_s. \]

\subsubsection{Pour une onde électromagnétique}
Lorsqu'on applique l'effet Doppler aux ondes électromagnétiques,
on ne s'intéresse plus à la vitesse de l'observateur et de la source séparément
mais à leur vitesse relative $u$ avec $u$ \emph{positif} s'ils se rapprochent
et \emph{négatif} s'ils s'éloignent.

La relativité nous permet de montrer que
\[ f_o = \sqrt{\frac{c + u}{c - u}}f_s. \]
Si $u \ll c$, on a
\[ \frac{f_o - f_s}{f_s} \approx \frac{u}{c}. \]

\paragraph{Attention} Parfois, il faut appliquer l'effet Doppler deux fois
comme pour un radar où l'objet est d'abord observateur et le radar la source
avant que les rôles ne s'inversent.
On a alors
\[ f_o = \frac{c + u}{c - u}f_s \]
et si $u \ll c$
\[ \frac{f_o - f_s}{f_s} \approx \frac{2u}{c}. \]

\section{Polarisation, réflexion et réfraction}

\subsection{Polarisation}
Le type de polarisation indique la forme du lieu parcouru
par un champ électrique.

Supposons que l'onde se déplace selon $\vec{z}$, 
une polarisation générale serait
\[ \vec{E}(\vec{r}, t) = A_x \sin(\vec{k}\cdot\vec{r} - \omega t + \phi_1)
  \xunit
+ A_y \sin(\vec{k} \cdot \vec{r} - \omega t + \phi_2) \yunit \]
C'est ce qu'on appelle on polarisation elliptique.
Il y a deux cas dégénérés:
\begin{itemize}
  \item Si $\phi_2-\phi_1 = \pm \frac{\pi}{2}$
    et $A_x = A_y$, c'est une polarisation circulaire;
  \item Si $\phi_2-\phi_1 = 0$ ou $\pm\pi$ ou si $A_x = 0$ ou $A_y = 0$,
    c'est une polarisation linéaire.
\end{itemize}

\subsection{Réflexion et réfraction}
Lorsqu'une onde passe d'un milieu à un autre,
elle se réfléchit en une onde \emph{réfléchie} et se réfracte en une onde
\emph{transmise} à la surface de séparation des deux milieu.

\paragraph{Attention}
Une onde électromagnétique ne passe pas à travers une membrane métallique, 
elle y est réfléchie.

\subsubsection{Réflexion}

Lors de la réflexion, les paramètres $\omega$, $f$, $v$ et $\lambda$
sont identiques à ceux de l'onde incidente.
L'angle de réflexion est égal à l'angle d'incidence.

L'onde réfléchie n'est pas toujours en phase avec l'onde incidente.
Elle est soit en phase, soit déphasée de $\pi$.

En fait, si le coefficient donné par l'équation de Fresnel est positif,
elle est en phase, sinon, elle est déphasée.
C'est à dire que:
\begin{itemize}
  \item Si $n_1 < n_2$, la réflexion de la partie de l'onde
    qui est perpendiculaire au plan
    d'incidence est déphasée de $\pi$ avec l'onde incidente
    et celle de la partie parallèle est déphasée de $\pi$
    pour $\theta < \theta_b$;
  \item Si $n_1 > n_2$, la réflexion de la partie perpendiculaire est
    en phase et celle de la partie parallèle est déphasée de $\pi$
    pour $\theta_b < \theta$.
\end{itemize}

% What's this ?
%\begin{figure}
%  \centering
%  \begin{subfigure}[b]{0.45\textwidth}
%    \begin{tikzpicture}%[x=2cm,y=3cm]
%      \begin{axis}
%        \addplot[smooth, color=blue, domain=0:90]
%        {(cos(sin(2*sin(x))) - 2*cos(x))/
%        (cos(asin(2*sin(x))) + 2*cos(x))};
%      \end{axis}
%    \end{tikzpicture}
%  \end{subfigure}
%  \begin{subfigure}[b]{0.45\textwidth}
%  \end{subfigure}
%\end{figure}

\subsubsection{Réfraction}
L'onde transmise a la même vitesse angulaire $\omega$
et la même fréquence $\nu$ que l'onde incidente
mais pas la même vitesse $v$ ni la même longueur d'onde $\lambda$.

Elle est en phase avec l'onde incidente.

\paragraph{Indice de réfraction}
On définit les indices de réfraction comme suit
\[ \frac{n_1}{n_2} \eqdef \frac{v_2}{v_1} = \frac{\lambda_2}{\lambda_1} \]

Dans le cas d'une onde électromagnétique,
$v = c = \frac{1}{\sqrt{\perm\mu}}$.
En définissant $n_\mathrm{vide} = 1$,
on a alors
$n = \sqrt{\perm_r\mu_r}$.

\paragraph{Loi de Snell-Descartes}
L'angle transmis $\theta_2$ par rapport à l'angle d'incidence
$\theta_1$ nous est donné par la loi de \emph{Snell-Descartes}
\begin{equation}
  \label{eq:snell}
  n_1 \sin(\theta_i) = n_2 \sin(\theta_r).
\end{equation}

\subsubsection{Équations de Fresnel}
Pour trouver l'intensité de l'onde réfléchie et transmise, il nous
faut décomposer son intensité en une composante parallèle au plan d'incidence
$E^\parallel$ et une perpendiculaire $E^\perp$.

Soit $E_{1r}$ l'intensité du champ réfléchi
et $E_{2}$ l'intensité du champ transmis.
On a les formules suivantes
\begin{align*}
  E_{1r}^\parallel & = \frac{n_1\cos\theta_2 - n_2\cos\theta_1}
  {n_1\cos\theta_2 + n_2\cos\theta_1}E_{1}^\parallel
  \stackrel{\eqref{eq:snell}}{=}
  \frac{\tan(\theta_2 - \theta_1)}{\tan(\theta_2 + \theta_1)}E_1^\parallel
  & E_2^\parallel & = \frac{2n_1\cos(\theta_1)}
  {n_1\cos\theta_2 - n_2\cos\theta_1}E_{1}^\parallel\\
  E_{1r}^\perp & = \frac{n_1\cos\theta_1 - n_2\cos\theta_2}
  {n_1\cos\theta_1 + n_2\cos\theta_2}E_{1}^\perp
  \stackrel{\eqref{eq:snell}}{=}
  \frac{\sin(\theta_2 - \theta_1)}{\sin(\theta_2 + \theta_1)}E_1^\perp
  & E_2^\perp & = \frac{2n_1\cos(\theta_1)}
  {n_1\cos\theta_1 - n_2\cos\theta_2}E_{1}^\perp\\
\end{align*}

\paragraph{Angle de Brewster}
Il existe un angle pour lequel $E_{1r}^\parallel = 0$,
on l'appelle l'angle de Brewster et c'est l'angle $\theta_B$ tel que
\[ \tan\theta_B = \frac{n_2}{n_1}. \]

\paragraph{Angle critique}
Si $n_1 > n_2$, il existe un angle critique $\theta_{1c}$ tel que
$E_{2} = 0$ pour tout $\theta_1 \geq \theta_{1c}$.
C'est l'angle qui respecte
\[ \sin\theta_{1c} = \frac{n_2}{n_1}. \]

On voit bien ici pourquoi, si $n_1 < n_2$, cet angle n'existe pas.

\section{Interférence et diffraction}
Quand une onde traverse une fente ou un objet, elle subit une diffraction.
Si c'est un objet et non une fente, l'effet est le même,
on ne traitera donc que le cas de fentes.
On considère aussi que le point où on mesure l'onde est loin des fentes
par rapport à leur distance entre elles.

\subsection{Approximation de Fraunhofer}
\label{sec:fraunhofer}
Lorsqu'il faut estimer la différence de distance entre différentes
fentes ou antennes situées aux points $P_i$ émettant des ondes
et un point $P$ loin d'elles,
il est nécessaire de faire l'approximation de Fraunhofer qui
consiste à considérer que tous les vecteurs $\vec{P_iP}$
sont parallèles.
La distance supplémentaire d'une fente à l'autre est donc $d\sin\theta$
où $d$ est la distance entre deux fentes et $\theta$ l'angle
formé avec la perpendiculaire aux deux fentes.


\subsection{Interférence}
Si une onde arrive sur deux fentes de largeur négligeable séparées par
une distance $d$,
il y aura interférence constructive si $\exists n \in \mathbb{N}$ tel que
\[ d\sin\theta = n \lambda \]
et interférence destructive si $\exists n \in \mathbb{N}$ tel que
\[ d\sin\theta = \left(n+\frac{1}{2}\right) \lambda \].

Si on considère $N$ fentes de largeur négligeable séparées par une
distance $d$,
\[ I(P) \propto \frac{A^2}{R^2} \cdot
  \frac{\sin^2\left(\frac{N \pi d \sin\theta}{\lambda}\right)}
{\sin^2\left(\frac{\pi d \sin\theta}{\lambda}\right)} \]
où $R$ est la distance entre $P$ et la fente la plus proche.

Ce qui donne des minimas si
$\exists n \in \mathbb{N}_0$ non divisibles par $N$ tel que
\[ N d \sin \theta = n \lambda \]
et des maxima si $\exists n \in \mathbb{N}$ tel que
\[ d \sin \theta = n \lambda. \]

\subsection{Diffraction}
Supposons maintenant que l'épaisseur $a$ des fentes n'est plus
négligeable.
\begin{itemize}
  \item Pour une fente,
    \[ I(P) \propto I_0 \times
      \frac{\sin^2\left(\frac{\pi a \sin\theta}{\lambda}\right)}
    {\left(\frac{\pi a \sin\theta}{\lambda}\right)^2}; \]
  \item et pour $N$ fentes,
    \[ I(P) \propto I_0 \times
      \frac{\sin^2\left(\frac{\pi a \sin\theta}{\lambda}\right)}
      {\left(\frac{\pi a \sin\theta}{\lambda}\right)^2} \times
      \frac{\sin^2\left(\frac{N \pi d \sin\theta}{\lambda}\right)}
    {\sin^2\left(\frac{\pi d \sin\theta}{\lambda}\right)}. \]
\end{itemize}
On a des minima si $\exists n \in \mathbb{N}_0$ tel que
\[ a \sin \theta = n\lambda. \]

Si $N \neq 1$, on a aussi les minimas et maximas qu'avait
l'interférence à $N$ fentes.
Sauf bien sûr, les maximas pour lesquels $a\sin\theta = n\lambda$.
Par exemple, si $a = \frac{d}{2}$, tous les maximas d'ordre pairs
seront \emph{éteints}.

%TODO Les fines couches d'huile

\section{Ondes stationnaire}
Une onde stationnaire est une onde pour laquelle ses nœuds ne bougent pas.
Elle est obtenue en fixant des nœuds où des ventres à ses extrémités
Son équation est la suivante
\[ A \sin(kx) \cos(\omega t). \]


%TODO modulation d'amplitude, vitesse de groupe CM7
%TODO Vitesse de propageation de l'information CM7


\section{Émission}
L'émission d'une onde EM est obtenue en accélérant deux charges.
Lorsque ces charges changent de vitesse, elles émettent une onde, qui sera
parallèle à leur accélération.

Ce principe est utilisé dans les antennes élémentaires.

\subsection{Antennes}
Les antennes sont modélisées par des segments dans lequel un courant
alternatif circule.
Supposons que le segment soit aligné avec l'axe $z$.
Il produit alors une onde électromagnétique partout dans le plan $xy$ mais
dont l'amplitude est inversement proportionnelle à $R$ et l'intensité
est inversement proportionnelles à $R^2$.

Ce type d'antenne est une antenne dite demi longueur d'onde, elle émet
donc une onde dont $\lambda = h/2$.

Pour analyser l'interférence de plusieurs antennes,
on fait l'approximation de Fraunhofer, voir section~\ref{sec:fraunhofer}.



%%%%%%%%%%%%%%%%%%%%%%%%%%%%
%	    QUANTIQUE          %
%%%%%%%%%%%%%%%%%%%%%%%%%%%%

\part{Physique quantique}
\section{Lumière et photons}
La lumière est composée de photons.
Chaque photon voyage à la vitesse de la lumière $c$.
Son énergie est appelée quanta et vaut
\[ E = h\nu = cp = mc^2. \]
Sa quantité de mouvement vaut
\[ p = \frac{h}{\lambda}. \]

Les photons sont \emph{indivisibles}.

\subsection{Particules et fonctions d'onde}
Chaque particule est caractérisée par une masse au repos $m > 0$
et une vitesse $v$.

Une onde de Broglie lui est associée et vaut
\[ \lambda = \frac{h}{mv} \]
Son énergie vaut
\[ E = hf = \hbar \omega \]

Une particule est liée à une fonction d'onde.
L'amplitude de cette fonction d'onde, i.e. $|\Psi|^2$, peut être interprété
comme la \emph{densité} de probabilité par unité de volume de trouver
la particule à l'endroit de l'espace et au moment où cette fonction
d'onde est calculée.
C'est à dire que la probabilité de trouver une particule
au temps $t$ dans un parallélépipède rectangle de côtés
$\dif x$, $\dif y$ et $\dif z$ au point de coordonnée $(x, y, z)$ est
\[ |\Psi(x, y, z, t)|^2 \dif x \dif y \dif z \]
Il est important de se rappeler que
\[ |z|^2 = z^{*} \cdot z \]
où $z^{*}$ est le conjugué de $z$.

On peut effectuer une séparation de variable sur la fonction d'onde,
\[ \Psi(\vec{r}, t) = \psi_1(\vec{r}) \cdot \psi_2(t). \]

$\psi_1$ a l'équation suivante
\[ \psi_1(\vec{r}) = \int_{k=-\infty}^{k=\infty}
C_1(k) \exp\left(i\vec{k}\cdot\vec{r}\right) \]
avec
\begin{align}
  \label{eq:pvk}
  \vec{p} & = m\vec{v} = \hbar \vec{k}\\
  \nonumber
  p & = \frac{h}{\lambda}.
\end{align}

$\psi_2$ a l'équation suivante
\[ \psi_2(t) = \int_{E=-\infty}^{E=\infty}
C_2(E) \exp\left(-i \frac{E}{\hbar} t\right) \]
avec
\begin{align}
  \label{eq:Ef}
  E & = hf = \hbar\omega.
\end{align}

Les intégrales sont présentes dans les expressions précédentes
à cause du principe d'incertitude d'Heisenberg.

\section{Principe d'incertitude d'Heisenberg}
\begin{mynota}
  Posons l'opérateur $\Delta$ comme
  la grandeur de l’intervalle d'incertitude
  d'une grandeur physique.
\end{mynota}

On remarque que lorsqu'on a la fonction d'onde $\Psi$,
on a toutes les grandeur physiques intéressantes de la particule.
Mais on remarque aussi qu'une même particule,
en plus d'avoir une incertitude sur la position et le temps,
peut avoir une énergie $E$, une fréquence $f$,
une longueur d'onde $\lambda$, une vitesse $v$,
une quantité de mouvement $p$ et un nombre d'onde $k$ incertains.

\subsection{Le principe d'incertitude d'Heisenberg pour la position}
En effet, si tous les $C_1(k)$ sont non nuls,
$k$ peut valoir la valeur qu'il veut,
c'est à dire, par \eqref{eq:pvk}, que
$\Delta k = \Delta p = \Delta v = \infty$.
Par contre, si $C_1(k)$ est non nul seulement
pour $k \in [-1;1]$, par \eqref{eq:pvk}, on a que
$\Delta k = 2$, $\Delta p = 2\hbar$ et $\Delta v = \frac{2\hbar}{m}$.

En fait, si $\Delta k = 0$, on remarque que
\[ |\psi_1(\vec{r})|
= \left|C_1(k) \exp\left(i\vec{r} \cdot \vec{k}\right)\right|
= \left|C_1(k)\right|\cdot\left|\exp\left(i\vec{r} \cdot \vec{k}\right)\right|
= C_1(k) \]
c'est à dire que la probabilité de trouver la particule est la même
quelle que soit la position.
On a donc $\Delta x = \infty$, $\Delta y = \infty$ et $\Delta z = \infty$.

De même, si on impose $\Delta x = 0$, comme $\psi_1(\vec{r})$ est une
série de Fourier, il faudra intégrer les sinus
sur un intervalle de longueur infinie pour obtenir
$\psi_1$ tel que $\psi_1(\vec{r})$ vaille 0 partout sauf en $x = 0$ donc
$\Delta p_x = \infty$.

Le principe d'incertitude d'Heisenberg pour la position nous dit que
\begin{align*}
  \Delta x \cdot \Delta p_x \geq \hbar\\
  \Delta y \cdot \Delta p_y \geq \hbar\\
  \Delta z \cdot \Delta p_z \geq \hbar.
\end{align*}

Ce n'est donc pas un hasard qu'on ne puisse pas connaître en même temps
la position de la particule avec précision ainsi que sa quantité de mouvement.
Il est important de bien comprendre que cette inégalité n'est pas due
à des défauts des appareils de mesures mais est liée à la nature
ondulatoire des particules.

\subsection{Le principe d'incertitude d'Heisenberg pour le temps}
Le même raisonnement marche aussi pour $\psi_2$.
Si on connaît $f$, c'est à dire, par \eqref{eq:Ef},
que $\Delta E = \Delta f = 0$, on a
\[ |\psi_2(\vec{r})|
= \left|C_2(E) \exp\left(-i\frac{E}{\hbar}t\right)\right|
= \left|C_2(E)\right|\cdot\left|\exp\left(-i\frac{E}{\hbar}t\right)\right|
= C_2(E). \]
C'est à dire que la particule n'est pas localisée dans le temps.

Si on veut que la probabilité que la particule soit là qu'à un temps précis,
c'est à dire que $\Delta t = 0$,
comme $\psi_2$ est une série de Fourier, il faudra intégrer les sinus
sur un intervalle de longueur infinie
pour obtenir $\psi_2$ tel qu'il soit non nul que pour un certain $t$.
Donc, on aura $\Delta E = \infty$.

Le principe d'incertitude d'Heisenberg pour le temps nous dit que
\begin{align*}
  \Delta t \cdot \Delta E \geq \hbar.
\end{align*}

La plupart du temps, on travaille dans le cas où
$\Delta t = \infty$ et $\Delta E = 0$.
On a alors
\[ \Psi(\vec{r}, t) = \psi_1(\vec{r}) \exp\left(-i\frac{E}{\hbar}t\right). \]

\section{L'équation de Schrödinger à une dimension}
Soit une particule de masse $m$ et d'énergie $E$ dans un potentiel $U(x)$,
on peut déterminer $E$ et $\psi_1$ grâce à l'équation de Schrödinger:
\[ -\frac{\hbar^2}{2m}\ffdif{\psi_1}{x} + U(x) \psi_1(x) = E\psi_1(x). \]

\subsection{Puits de potentiel infini}
En résolvant l'équation de Schrödinger dans un puits infini,
c'est à dire avec
\[ U(x) = \left\{
  \begin{aligned}
    0 & \text{ si } x \in [0; L]\\
    \infty & \text{ sinon}
  \end{aligned}
\right. \]
on obtient
\begin{align*}
  {\psi_1}_n(x) & =
  \sqrt{\frac{2}{L}}\sin\left(\frac{n\pi x}{L}\right) & n = 1, 2, \ldots
\end{align*}
et
\begin{align*}
  E_n & = \frac{n^2h^2}{8mL^2} & n = 1, 2, \ldots
\end{align*}

\subsection{Puits de potentiel fini}
En résolvant l'équation de Schrödinger dans un puits fini,
c'est à dire avec
\[ U(x) = \left\{
  \begin{aligned}
    0 & \text{ si } x \in [0; L]\\
    U_0 & \text{ sinon}
  \end{aligned}
\right. \]
on remarque que $\psi_1(x)$ est non nul pour $x < 0$ et $x > L$
même s'il tend vers 0 tel une exponentielle.
L'énergie vaut
\[
  E_n = -\frac{\hbar^2\alpha_n^2}{2m}
\]
où $\alpha_n$ ne peut être déterminé que numériquement.

\subsection{Barrière de potentiel}
En résolvant l'équation de Schrödinger
aux alentour d'une barrière de potentiel,
c'est à dire avec
\[ U(x) = \left\{
  \begin{aligned}
    U_0 & \text{ si } x \in [0; L]\\
    0 & \text{ sinon}
  \end{aligned}
\right. \]
on remarque que même si $E < U_0$,
$\psi_1(x)$ est non-nul pour $x > 0$ et aussi pour $x > L$.
C'est à dire que la particule peut passer la barrière.
On appelle ça l'\emph{effet tunnel}.

En analysant le comportement de $\Psi$ en fonction du temps,
on remarque qu'il y a
une partie qui est réfléchie par la barrière de potentiel et
une partie qui passe à travers la barrière par l'effet tunnel.
Bien évidemment, plus $\frac{U_0}{E}$ est grand,
moins il y a de probabilité que l'électron passe à travers la barrière
et plus il y en a qu'il soit réfléchi.

\subsection{Oscillateur harmonique}
Le potentiel autour d'un atome en fonction de la distance
par rapport au noyau ressemble à une parabole.
C'est le potentiel d'un oscillateur harmonique.

En résolvant l'équation de Schrödinger avec ce potentiel,
on trouve que
\[ E_n = \left(n+\frac{1}{2}\right)\hbar\omega. \]
où $\omega = \sqrt\frac{k'}{m}$ et
$k'$ est la constante de raideur.

On peut remarquer que ces niveaux d'énergie sont équidistants et que
le niveau zéro est d'énergie non nulle.

Les fonctions de probabilité de présence sont symétriques pour les
niveaux impaires et antisymétriques pour les niveaux pairs.

\section{Atome}
Tout électron dans un atome voyage dans une orbitale caractérisée par
4 nombres quantiques: $n$, $l$, $m_l$ et $m_s$.
Ces nombres quantiques respectent
\begin{align*}
  n, l, m_l & \in \mathbb{Z}\\
  n & \geq 1\\
  0 & \leq l < n\\
  |m_l| & \leq l\\
  m_s & = \pm \frac12
\end{align*}

L'énergie d'un électron est donc
\[ E_n = -\frac{Z_\mathrm{eff}^2}{(4\pi\perm_0)^2}\frac{m_ee^4}{2n^2\hbar^2}
= -Z_\mathrm{eff}^2\frac{\si{13.6}{\electronvolt}}{n^2} \]
où $m_e$ est la masse d'un électron, $e$ sa charge et
$Z_\mathrm{eff}$ est le nombre de charge vue par un électron.
Par exemple si c'est le 12\ieme{} électron d'un atome de 15 protons,
$Z_\mathrm{eff} = 15 - 11 = 4$.

Le moment angulaire de son orbite vaut
\[ L = \sqrt{l(l+1)} \hbar \]
avec une composante en $z$ valant
\[ L_z = m_l \hbar. \]

Son moment angulaire vaut
\[ S = \sqrt\frac34 \hbar \]
avec une composante en $z$ valant
\[ S_z = m_s \hbar. \]

\subsection{Probabilité de présence}
Pour trouver la probabilité de trouver un électron dans un atome
à une distance $r$ quand on connaît $\Psi(\vec{r})$,
il ne faut \emph{pas} faire $|\Psi(\vec{r})|^2 \dif r$
car comme on calcule maintenant $\Psi(\vec{r})$ en fonction des 3 dimension,
c'est la densité par volume.
Comme on est dans le cas d'une sphère, la formule de la probabilité
$P(r)$ de trouver l'électron à une distance entre $r$ et $r+\dif r$ est
\[ P(r) \dif r = |\Psi(\vec{r})|^2 \dif V
= |\Psi(\vec{r})|^2 4\pi r^2 \dif r. \]

\subsection{Rayon de l'orbitale}
Le rayon de l'orbitale est obtenu par la formule suivante
\[ r = \frac{4\pi\perm_0\hbar^2}{m_ee^2}n^2. \]
où $m_e$ est la masse d'un électron et $e$ sa charge.

\subsection{Principe d'exclusion de Pauli}
Deux électrons ne peuvent pas avoir les mêmes 4 nombres quantiques
$n$, $l$, $m_l$ et $m_s$.
Ce principe n'est donc utile que dans le cas de plusieurs électrons.

\subsection{Effet photoélectrique}
Un électron peut changer d'orbite à deux conditions
\begin{itemize}
  \item Le principe d'exclusion de Pauli le permet;
  \item $n$ et $l$ changent exactement de 1.
    C'est à dire que $\Delta n = \pm 1$ et $\Delta l = \pm 1$.
\end{itemize}

Si $\Delta n = 1$, c'est qu'un photon de fréquence
$\nu = \frac{\Delta E_n}{h}$ a été absorbé par l'électron.

Si $\Delta n = -1$, l'électron émet un photon de fréquence
$\nu = \frac{\Delta E_n}{h}$.

\section{Liaisons atomiques}
Lorsque deux atomes se lient,
ils vibrent et tournent.
Ça crée une énergie vibrationnelle
\[ E_n = \left(n + \frac{1}{2}\right)\hbar\sqrt\frac{k'}{m_r} \]
et une énergie rotationnelle
\[ E_l = l(l+1) \frac{\hbar}{2m_rr_0^2} \]
où $r_0$ est la distance entre les deux atomes,
$k'$ est la constante de raideur de la vibration et
$m_r$ est la masse réduite des masses des deux atomes $m_1$ et $m_2$ valant
\[ m_r = \frac{m_1m_2}{m_1+m_2}. \]

\section{Solides}
Dans un solides,
les électrons sont mis en communs entre les différents atomes.
On remarque que les électrons ne peuvent avoir leur énergie que dans certains
intervalles qu'on appelle bande.
Le principe d'exclusion de Pauli s'applique ici aussi et donc certaines
bandes peuvent être remplies et ne plus accepter d'électrons.
La dernière bande remplie est appelée la bande de valence et la suivante
bande de conduction.
La longueur de la bande interdite entre les deux est appelée
band gap et est notée $E_g$.

S'il n'y a pas exactement le bon nombre d'électron pour
que la bande de valence soit remplie et que la bande de conduction soit
vide, il y en a dans la bande de conduction.
La séparation entre les bandes pleines et vides, et égale à $E_F$,
l'énergie de Fermi.
Les électrons présents dans une bande partiellement remplie sont libres de se
déplacer facilement dans cette bande. Le courant peut donc passer facilement.
On dit que le solide est un \emph{conducteur}.

Sinon, si $E_g$ est élevé, de l'ordre de 2 à \si{6}{\electronvolt},
on dit que c'est un \emph{isolant}.
Si $E_g$ est plus faible, on dit que c'est un \emph{semi-conducteur}.

\subsection{Semi-conducteur}
Pour les semi-conducteurs,
si la température passe au dessus de \si{0}{\kelvin},
il y a une probabilité que les électrons
se trouvent sur la bande de conduction,
un petit courant peut donc passer.

La distribution des électrons est donnée par la distribution de Fermi-Dirac:
\[ f(E) = \frac{1}{1+\exp\left(\frac{E-E_F}{k_BT}\right)}. \]
où $E_F$ est l'énergie de Fermi et est défini telle que
$f(E_F) = \frac{1}{2}$.
%TODO graph

\appendix
\section{Unités}
En physique quantique, il y a deux unités souvent utilisées:
\begin{itemize}
\item le Hartree ($1Ha =  4,36\cdot 10^{-18} J$)
\item le Alchtreum ($1 \dot{A} = 10^{-10} m$)
\end{itemize}
\end{document}
