\documentclass[11pt,a4paper]{article} % french
\usepackage[utf8]{inputenc}
\usepackage{amsfonts}
\usepackage{amsmath}
\usepackage{bm}
\usepackage[french]{babel}

\newtheorem{defin}{Definition}[section]
\newtheorem{nota}[defin]{Notation}
\newtheorem{prop}[defin]{Propriete}

\newcommand{\B}{\vec B}
\newcommand{\EMF}{\varepsilon}

%%Tres Draft je sais... %%
\setlength{\parindent}{0cm} % Vire les alinéas
\setlength{\parskip}{0.5cm} %Agrandit l'espace entre les paragraphs

\title{Elec (Magnétisme) Q2}
\author{Nicolas Cognaux}


\begin{document}
\maketitle
Ce document reprend mes notes prises à la relecture des slides et de mes notes de cours. Il ne reprend pas toutes les formules puisqu'on les a dans le formulaire. Les démonstrations ne sont pas présentes (elles sont dans les slides pour la plupart). 
Comme ce n'est qu'un recopiage, le document sera sûrement brouillon. Je ne reprends que des points importants pour la compréhension de la matière.

\rule{\linewidth}{.5pt}


Un courant est un déplacement de charges. Si ce courant rencontre un champ magnétique ($\vec B$), il y a création d'une force:
$$ \vec F = (q \vec E) + q \vec v \times \B $$
Dans le cas d'un courant parcourant un champ, la formule devient:
$$ d\vec F = Id\vec l \times \B $$

On peut utiliser cette force pour créer un moment et donc un couple. C'est le cas dans un moteur électrique. Les spires doivent rester fixes pour annuler les forces latérales. Si elles sont déformables, elles tendront vers une forme circulaire.

Le moment dipolaire magnétique est noté $\vec \mu = I \vec S $

Dans un aimant, les lignes de champ vont du Nord au Sud. { \bf Il n'y a jamais de monopôle magnétique !}

Un courant crée un champ magnétique. Ce champ est la somme vectorielle de tous les champs générés par toutes les charges de ce courant:
$$ d\B = \frac{\mu_0}{4\pi}\frac{Id\vec l\times \vec r}{r^2} \textrm(Loi de Biot-Savart)$$

Pour déterminer le champ au centre d'une spire, on utilise Biot-Savart.
%% Alors ici la démo j'ai pas réellement compris son truc donc je vais pas me risquer à expliquer. Il y a un angle phi qui aparait sans raison... Si quelqu'un pouvait clarifier. ;)

Les bobines de Helmholtz sont des bobines espacées d'une distance $x = R$ où $R$ est le rayon des eeux bobines. Cette configuration permet de générer un champ magnétique constant sur toute la longueur séparant les bobines ($x$).

La loi d'Ampère dit que tout fil parcouru par un courant est toujours entouré d'un champ magnétique. Dans tout contour fermé,
$$\oint \B d\vec l = \mu_0 I$$
Cette loi permet de déterminer le champ dans un solénoïde long.

Pour déterminer le sens de $\B$ créé par un cournat $\vec I$, il faut utiliser la règle du tire-bouchon avec la main droite. Où le pouce est le sens du courant.

L'inductance: Chaque spire produit un champ magnétique et donc un flux. Ce flux est intercepté par chaque spire et produit donc un courant lors de la variation de ce flux (voir plus loin).
$$L = N\frac{\Phi(I)}{I} \textrm{(I s'annule en développant)}$$ 

Diamagnétisme, il crée un $\B$ inverse au champ magnétisant. Le paramagnétique s'alligne approximativement avec le champ magnétisant. Le Ferromagnétique s'aligne parfaitement avec le champ magnétisant. Il amplifie donc $\B$.

Dans la matière, il y a formation de domaines magnétiques. Dans ces domaines, les moments dipolaires sont tous alignés dans le même sens.

Les ferromagnétiques durs ont une courbe d'hystérésis plus large et nécessitent donc plus d'énergie pour etre magnétisés/démagnétisés que les ferromagnétiques doux.

Si on place une ferrite à l'intérieur d'une bobine, l'entièreté du $\B$ se retrouvera dans la ferrite, le $\B_{ext}$ est considéré comme nul.

Dans un électroaimant avec ente-fer, Le flux reste constant y compris dans l'entre-fer. Tout le champ se retrouve donc dans l'entre-fer. Cela explique la démagnétisation des aimants permanents ouverts.

Le travail fourni au système = l'énergie accumulée sous forme électrostatique dns un condensateur plat. L4énergie dans un condensateur: $U = \frac{CV^2}{2}$. Dans une inductance, l'énergie stoquées est $U = \frac{LI^2}{2}$.

La densité d'énergie électromagnétique est l'énergie dans une unité de volume. Elle dépend du $\B$ et de la permitivité du matériau. Si le $\mu$ est constant, elle est exprimée par:
$$E_M =  \frac{B^2}{2\mu} [J/m^3]$$
Pour déterminer l'énergie par les courbes d'hystérésis ($\mu$ non constant), on doit calculer l'air sous la première courbe et l'air sous la deuxième courbe. En les soustrayant on retrouve l'énergie nécessaire totale.

En faisant varier le champ dans une boucle (et donc le flux), on crée un courant.
$$\EMF = \oint \vec E_i d\vec l = -\frac{d\Phi}{dt} \textrm{(Loi de Farraday)}$$
Le courant créé s'oppose à la règle de la main droite.

Avec la loi d'induction, on peut conclure qu'il ne faut pas de conducteur pour transmettre un courant. Un simple contour fermé interceptant le champ suffit. (Important pour tout ce qui concerne la radio)




\end{document}
