\documentclass[11pt,a4paper]{article} % french
\usepackage[utf8]{inputenc}
\usepackage{amsfonts}
\usepackage{amsmath}
\usepackage{bm}
\usepackage[french]{babel}

\newtheorem{defin}{Definition}[section]
\newtheorem{nota}[defin]{Notation}
\newtheorem{prop}[defin]{Propriete}

%\newcommand{\sbullet}{\hbox{\fontfamily{lmr}\fontsize{4}{0}\selectfont\textbullet}}

\newcommand\dist{\mathrm{dist}}

\title{Elec (Magnétisme) Q2}
\author{Nicolas Cognaux}


\begin{document}
\maketitle
Un courant est un déplacement de charges. Si ce courant rencontre un champ magnétique ($\vec B$), il y a création d'une force:
$$
\vec F = (q \vec E) + q \vec v \times \vec B
$$
Dans le cas d'un courant parcourant un champ, la formule devient:
$$
d \vec F = Id\vec l \times \vec B
$$

\end{document}
