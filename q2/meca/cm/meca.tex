\documentclass[11pt,a4paper]{article} % french
\usepackage[utf8]{inputenc}
\usepackage{amsfonts}
\usepackage{amsmath}
\usepackage{tensor}
\usepackage{bm}
%\usepackage[french]{babel}

\newtheorem{defin}{Definition}[section]
\newtheorem{nota}[defin]{Notation}
\newtheorem{prop}[defin]{Propriete}

%\newcommand{\sbullet}{\hbox{\fontfamily{lmr}\fontsize{4}{0}\selectfont\textbullet}}

\newcommand\dist{\mathrm{dist}}
\newcommand\Ker{\mathrm{Ker}}
\newcommand\fv[1]{{\bf #1}} % free vector
\newcommand\fvd[1]{\dot{\bf #1}} % free vector derivated
\newcommand\fvdd[1]{\ddot{\bf #1}} % free vector derivated
\newcommand\fvr[1]{\mathring{\bf #1}} % free vector relatively derivated
\newcommand\fvrr[1]{\overset{\circ\circ}{\bf #1}} % free vector relatively derivated
\newcommand\uv[1]{{\bf\hat{ #1}}} % unit vector
\newcommand\ui{{\bf\hat{I}}} % unit vector I
\newcommand\uj{{\bf\hat{J}}} % unit vector J
\newcommand\uk{{\bf\hat{K}}} % unit vector K
\newcommand\eqdef{\stackrel{\Delta}{=}}
\newcommand\wrt[2]{\tensor*[_{ #1}]{ #2}{}} % With Respect To
\newcommand\wtr[3]{\tensor*[_{ #1}]{ #2}{^{ #3}}} % With Two Respect
\newcommand\omegaf{{\bm \omega}}
\newcommand\omegafr{\mathring{\bm \omega}}
\newcommand\omegaft{\tilde{\bm \omega}}
\newcommand\omegaftr{\mathring{\tilde{\bm \omega}}}
\newcommand\omegat{\tilde{\omega}}
\newcommand\omegatd{\tilde{\dot{\omega}}}
\newcommand\ine{{\bf I}}
\newcommand\st{{\bf L}}
\newcommand\stm{{\bf M}}
\newcommand\lm{{\bf N}}
\newcommand\am{{\bf H}}
\newcommand\fo{{\bf F}}

\title{Mécanique Q2}
\author{Benoît Legat}


\begin{document}
\maketitle

\section{Les vecteurs}

Il y a deux types de vecteurs
\begin{description}
	\item[Lié] Noté $\fv{u^v}$, vecteur de d'origine $\fv{v}$.
	\item[Libre] Noté $\fv{u}$, vecteur sans origine fixe.
\end{description}
La notation $\uv{u}$ indique que $||\uv{u}|| = 1$. $\uv{u}$ est appelé un vecteur unitaire.

\subsection{Base orthonormée}

La base orthonormée $\{\ui\}$ est une base composée de 3 vecteurs unitaires orthogonaux $\ui_1$, $\ui_2$ et $\ui_3$ respectivement arrangés dans l'ordre donné par la règle de la main droite.

Tout vecteur $\fv{u}$ a des coordonnées unique $u_1, u_2, u_3$ dans $\{\ui\}$.
On a
\[ \fv{u} = u_1\ui_1 + u_2\ui_2 + u_3\ui_3 = [\ui]^T u \]
	où $[\ui] = \begin{pmatrix}\ui_1\\\ui_2\\\ui_3\end{pmatrix}$ et $u = \begin{pmatrix} u_1 \\ u_2 \\ u_3 \end{pmatrix}$.
Il est important de noter que $[\ui]^T[\ui] = [\ui][\ui]^T = E$, où $E$ est le tenseur unitaire (neutre pour la multiplication).

\subsection{Produit scalaire}

Soient $\fv{u} = u_1\ui_1 + u_2\ui_2 + u_3\ui_3$ et $\fv{v} = v_1\ui_1 + v_2\ui_2 + v_3\ui_3$.
\begin{eqnarray*}
	\fv{u}.\fv{v} & \eqdef & ||\fv{u}|| ||\fv{v}|| \cos\theta\\
	& = & u_1v_1 + u_2v_2 + u_3v_3\\
	& = & u^T v
\end{eqnarray*}
où $\theta$ est l'angle entre $\fv{u}$ et $\fv{v}$.
Le produit scalaire est {\em commutatif}, {\em associatif} et {\em bilinéaire}.

\subsection{Produit vectoriel}

Soient $\fv{u} = u_1\ui_1 + u_2\ui_2 + u_3\ui_3$ et $\fv{v} = v_1\ui_1 + v_2\ui_2 + v_3\ui_3$.
\begin{eqnarray*}
	\fv{u}\times\fv{v} & \eqdef & (u_2v_3 - u_3v_2) \ui_1 + (u_3v_1 - u_1v_3) \ui_2 + (u_1v_2 - u_2v_1) \ui_3\\
	& = & [\ui]^T \begin{pmatrix}u_2v_3 - u_3v_2 \\ u_3v_1 - u_1v_3 \\ u_1v_2 - u_2v_1\end{pmatrix}\\
	& = & [\ui]^T \tilde{u}v\\
	& = & \tilde{\fv{u}}.\fv{v}
\end{eqnarray*}
où $\tilde{u} \eqdef \begin{pmatrix}0 & -u_3 & u_2\\ u_3 & 0 & -u_1\\ -u_2 & u_1 & 0\end{pmatrix}$
	et $\tilde{\fv{u}} \eqdef [\ui]^T \tilde{u} [\ui]$. On voit que $\tilde{u}^T = -\tilde{u}$.
Le produit vectoriel est {\em anticommutatif}, {\em pas associatif} et {\em bilinéaire}.

\section{Changement de base}

Pour tout changement de base de $\{\ui\}$ à $\{\uj\}$, il existe une matrice de rotation $A$ tel que
\[ [\uj] = A [\ui] \]
$A$ est orthogonale, c'est à dire que $AA^T = E$ ou encore $A^{-1} = A^T$. On a donc
\[ [\ui] = A^T [\uj] \]
Si $\{\uj\}$ respecte la règle de la main droite, on a aussi $\det A = 1$, on dit alors que $A$ est orthogonale directe.\\
On peut passer des coordonnées d'un vecteur d'une base à l'autre aisément. Soit $\fv{u} = [\uj]^T \wrt{J}{u}$, on a
\[ \fv{u} = [\uj]^T \wrt{J}{u} = (A[\ui])^T \wrt{J}{u} = [\ui]^TA^T \wrt{J}{u} \]

\section{Dérivées temporelles}

\subsection{Dérivée première}
On peut calculer la dérivée temporelle de $\fv{u}$ ainsi
\begin{eqnarray*}
	\fvd{u} &=& [\uj]^T \wrt{J}{\dot{u}} + [\uj]^T \omegat.\wrt{J}{u}\\
	&=& \fvr{u} + \omegaft . \fv{u}\\
	&=& \fvr{u} + \omegaf \times \fv{u}
\end{eqnarray*}
où $\omegat = A\dot{A}^T$.
On voit ici que, comme $\omegaft$ est multiplié à $\fv{u} = [\uj]^T \wrt{J}{u}$. Il nous faut $\omegat$ dans la base $\{\uj\}$, $\omegaft = [\uj]^T \omegat [\uj]$ et de même donc pour $\omega$, $\omegaf = [\uj]^T \omega$.

Quand on a plusieurs changement de base dans un même problème, on précise pour chaque $\omegaf$ à quel changement de base il correspond. Si le contexte n'avait pas été clair, j'aurais du appeler mon $\omegaf$ de tout à l'heure $\omegaf^{JI}$.

Cette notation nous permet d'énoncer une propriété fondamentale !
\[ \omegaf^{KI} = \omegaf^{KJ} + \omegaf^{JI} \]

Dans la pratique, il faudra parfois effectuer des changements de base. Souvent, on aime bien exprimer $\omegaf$ dans la base intermédiaire (ici $\{\uj\}$) car ça fait moins de changement de base successif.

\paragraph{Exemple}
Si on a $\omegaf^{KJ} = [\uk]^T \wtr{K}{\omega}{KJ}$, $\omegaf^{JI} = [\uj]^T \wtr{J}{\omega}{JI}$ et $[\uk] = A[\uj]$, on calcule
\begin{eqnarray*}
	[\uj]^T \wtr{J}{\omega}{KI} &=& [\uk]^T \wtr{K}{\omega}{KJ} + [\uj]^T \wtr{J}{\omega}{JI}\\
	&=& [\uj]^T A^T \wtr{K}{\omega}{KJ} + [\uj]^T \wtr{J}{\omega}{JI}
\end{eqnarray*}
D'où
\[ \wtr{J}{\omega}{KI} = A^T \wtr{K}{\omega}{KJ} + \wtr{J}{\omega}{JI} \]

\subsection{Dérivée seconde}
Il y a deux manière de calculer la dérivée seconde.
\begin{itemize}
	\item
		Si on a calculé $\fvd{u}$ précédemment, il suffit de le dériver avec la formule de la dérivée première pour avoir la dérivée seconde.
		\[ \fvdd{u} = \mathring{\fvd{u}} + \omegaf \times \fvd{u} \]
	\item
		Directement à partir de $\fv{u}$ et de la formule de la dérivée seconde
		\begin{eqnarray*}
			\fvdd{u} &=&  [\uj]^T \wrt{J}{\ddot{u}} + 2 [\uj]^T \omegat.\wrt{J}{\dot{u}} + [\uj]^T \dot{\omegat}.\wrt{J}{u} + [\uj]^T \omegat.\omegat.\wrt{J}{u}\\
			&=& \fvrr{u} + 2 \omegaft . \fvr{u} + \omegaftr . \fv{u} + \omegaft . \omegaft . \fv{u}\\
			&=& \fvrr{u} + 2 \omegaf \times \fvr{u} + \omegafr \times \fv{u} + \omegaf \times (\omegaf \times \fv{u})
		\end{eqnarray*}
\end{itemize}

\section{Roulement sans glissement}

Si on a une roue ou une boule se déplaçant sur un sol, on peut lui définir un roulement sans glissement.
La condition s'énonce assez simplement.
La vitesse d'un point de la roue est nulle lorsqu'elle est en contact avec le sol.
Il est important qu'elle soit nulle dans toutes les directions.
Il est aussi primordial de considérer la vitesse nulle selon un {\bf repère fixe} !

Dans la pratique, le plus simple est de prendre la position d'un point de la roue par rapport à un point fixe (l'origine est un choix judicieux comme point fixe).
Il faut ensuite dériver cette position en fonction du temps. L'expression qu'on obtient pour la vitesse ne doit pas spécialement être nulle pour tout $t$, il faut identifier les moments pour lesquels le point sera en contact avec le sol et s'assurer qu'à ce moment là la vitesse sera nulle selon toute direction.

\paragraph{Exemple}
Si on a une roue de rayon $R$ tournant autour de $\ui_1$ avec une position angulaire $\theta(t)$, le sol ayant l'équation $z = -R$.
On définit le repère cure-dent $\{\uj\}$ à la roue.
Comme elle tourne autour de $\ui_1$ avec un angle $\theta(t)$,
\[ \omegaf = \dot{\theta}(t) \ui_1 = \dot{\theta}(t) \uj_1 = [\uj]^T \begin{pmatrix}0\\0\\\dot{\theta}(t)\end{pmatrix} \]
Prenons, sans perte de généralité, notre point $X$ au bord de la roue sur l'axe formé par $\uj_2$ et posons $C$, le centre de la roue
\[ \vec{OX} = \vec{OC} + R \uj_2 = \vec{OC} + [\uj]^T \begin{pmatrix}0\\R\\0\end{pmatrix} \]
Calculons à présent la dérivée de $\vec{OX}$ en fonction du temps
\begin{eqnarray*}
	\vec{\dot{OX}} &=& \vec{\dot{OC}} + [\uj]^T \begin{pmatrix}0\\\dot{R}\\0\end{pmatrix} + [\uj]^T \omegat . \begin{pmatrix}0\\R\\0\end{pmatrix}\\
	&=& \vec{\dot{OC}} + [\uj]^T \begin{pmatrix}0\\0\\0\end{pmatrix} + (\dot{\theta}(t) \uj_1) \times (R \uj_2)\\
	&=& \vec{\dot{OC}} + \dot{\theta}(t) R (\uj_1 \times \uj_2)\\
	&=& \vec{\dot{OC}} + \dot{\theta}(t) R \uj_3\\
\end{eqnarray*}
Nous avons ici que la vitesse est tangeantielle à la roue et vaut la vitesse angulaire fois le rayon.
Ce qui est une formule évidente du mouvement rectiligne uniforme.
Il est important de remarquer ici que le repère $\{\uj\}$ n'est {\bf pas fixe} !
Il faudra donc d'abord l'exprimer selon $\{\ui\}$ avant d'imposer que la vitesse soit nulle en contact avec le sol.
On a
\[ [\uj] = A^{(1)} [\ui] \]
D'où ($A^{(1)}$ est dans le formulaire)
\[ \uj_3 = -\sin\theta(t) \ui_2 + \cos\theta(t) \ui_3 \]
On trouve donc enfin
\[ \vec{\dot{OX}} = \vec{\dot{OC}} + \dot{\theta}(t) R (-\sin\theta(t) \ui_2 + \cos\theta(t) \ui_3) \]
Vu la position du sol choisie, $X$ sera en contact avec le sol ssi $\exists k \in \mathbb{Z}$ tel que $\theta(t) = \frac{3\pi}{2} + 2k\pi$.
On aura alors
\[ \vec{\dot{OX}} = \vec{\dot{OC}} + \dot{\theta}(t) R \ui_2 \]
A ce moment là, la vitesse doit être nulle, on a donc
\[ \vec{\dot{OC}} = -\dot{\theta}(t) R \ui_2 \]
Ce qui est intuitivement pas si choquant que ça.


\section{Corps rigide}
Un corps rigide $C$ est un corps qui ne peut être déformé.
C'est à dire que quelles que soient le point $P$ $P$ et la base $\{\uj\}$ fixé au corps $C$, pour tout point $X$ du corps, les coordonnées de $PX$ dans $\{\uj\}$ sont constantes.
On a aussi que $\omegaf^{JI}$ ne dépend pas du choix de $\{\uj\}$.

Tout corps $C$ a un centre de masse unique $G$ définit comme suit
\[ \fv{x}^G = \frac{\int_C \fv{x} dm}{m(C)} \]
où $\fv{x}^G \eqdef \vec{OG}$ et $m(C) \eqdef \int_C dm$.

Il a la particularité que
\[ \int_C \fv{r} dm = \int_C \dot{\fv{r}} dm = 0 \]
où $\fv{r} \eqdef \vec{GX}$.

\subsection{Vitesse et accélération}
Soit $P$ et $X$, deux points fixe sur $C$, on a
\begin{eqnarray*}
	\vec{\dot{OX}} &=& \vec{\dot{OP}} + \omegaf \times \vec{PX}\\
	\vec{\dot{OX}} &=& \vec{\ddot{OP}} + \omegafr \times \vec{PX} + \omegaf \times (\omegaf \times \vec{PX})
\end{eqnarray*}

\subsection{Quantité de mouvement linéaire et angulaire}
Soit $\lm(C)$ la quantité de mouvement linéaire du corps $C$ et $\am^P(C)$ la quantité de mouvement angulaire du corps $C$ par rapport au point $P$.
On a
\begin{eqnarray*}
	\lm(C) &=& m(C) \vec{\dot{OG}}\\
	\am^O(C) &=& \vec{OG} \times m(C)\vec{\dot{OG}} + \am^G(C)\\
	\am^G(C) &=& \ine^G . \omegaf
\end{eqnarray*}
où $\ine^G \eqdef -\int_C \tilde{\fv{r}}.\tilde{\fv{r}} dm$, on l'appelle le tenseur d'inertie.

\begin{itemize}
	\item Il est constant lorsqu'il est exprimé dans une base fixe au corps $C$
	\item Il est symétrique
	\item Il est défini semi positif
	\item
		Il existe une base $\{\uk\}$ telle que $I_G$ ($\ine^G = [\uk]^T I^G$) est diagonale. On appelle les $\uk_i$ les axes principals d'inertie.
	\item
		On peut passer d'une matrice d'intertie de $G$ à un point fixe du corps $C$ quelconque avec la formule suivante
		\[ \ine^P = -m(C)\tilde{\fv{d}}.\tilde{\fv{d}} + \ine^G \]
		Il est primordial que $\ine^G$ soit par rapport au centre de masse du corps en question !
	\item
		Si $C = \cup_i \{C_i\}$,
		\[ \ine^G = \sum_i \ine_{C^i}^G \]
		Il faut faire attention néanmoins, toutes les matrices d'inerties doivent être exprimée par rapport au même point, il faut utiliser la formule précédente pour changer de point de référence.
\end{itemize}

\subsection{Forces et moments de force}
La somme des moments agissant sur un corps $C$ par rapport à un point $Q$ est noté $\st^Q$.
Si la somme des forces est nulle, $\st^Q$ ne dépend pas du choix de $Q$ tant qu'il est solidaire au corps $C$.
Elle est alors appelée $\stm$.

\subsection{\'Equation de Newton-Euler}
\begin{eqnarray*}
	\dot{\lm}(C) &=& \fo\\
	\dot{\am}^G(C) &=& \st^G
\end{eqnarray*}
La deuxième équation reste vraie si on remplace $G$ par n'importe quel point fixe dans l'espace.

\end{document}
