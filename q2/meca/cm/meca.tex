\documentclass[11pt,a4paper]{article} % french
\usepackage[utf8]{inputenc}
\usepackage{amsfonts}
\usepackage{amsmath}
\usepackage{tensor}
\usepackage{bm}
%\usepackage[french]{babel}

\newtheorem{defin}{Definition}[section]
\newtheorem{nota}[defin]{Notation}
\newtheorem{prop}[defin]{Propriete}

%\newcommand{\sbullet}{\hbox{\fontfamily{lmr}\fontsize{4}{0}\selectfont\textbullet}}

\newcommand\dist{\mathrm{dist}}
\newcommand\Ker{\mathrm{Ker}}
\newcommand\fv[1]{{\bf #1}} % free vector
\newcommand\fvd[1]{\dot{\bf #1}} % free vector derivated
\newcommand\fvdd[1]{\ddot{\bf #1}} % free vector derivated
\newcommand\fvr[1]{\mathring{\bf #1}} % free vector relatively derivated
\newcommand\fvrr[1]{\overset{\circ\circ}{\bf #1}} % free vector relatively derivated
\newcommand\uv[1]{{\bf\hat{ #1}}} % unit vector
\newcommand\ui{{\bf\hat{I}}} % unit vector I
\newcommand\uj{{\bf\hat{J}}} % unit vector J
\newcommand\uk{{\bf\hat{K}}} % unit vector K
\newcommand\eqdef{\stackrel{\Delta}{=}}
\newcommand\wrt[2]{\tensor*[_{ #1}]{ #2}{}} % With Respect To
\newcommand\wtr[3]{\tensor*[_{ #1}]{ #2}{^{ #3}}} % With Two Respect
\newcommand\omegaf{{\bm \omega}}
\newcommand\omegafr{\mathring{\bm \omega}}
\newcommand\omegafd{\dot{\bm \omega}}
\newcommand\omegaft{\tilde{\bm \omega}}
\newcommand\omegaftr{\mathring{\tilde{\bm \omega}}}
\newcommand\omegat{\tilde{\omega}}
\newcommand\omegatd{\tilde{\dot{\omega}}}
\newcommand\ine{{\bf I}}
\newcommand\st{{\bf L}}
\newcommand\pst{{\bf M}}
\newcommand\lm{{\bf N}}
\newcommand\am{{\bf H}}
\newcommand\fo{{\bf F}}
\newcommand\po{\mathcal{P}}
\newcommand\xg{\fv{x}^G}
\newcommand\xgd{\fvd{x}^G}
\newcommand\xgdd{\fvdd{x}^G}

\title{Mécanique Q2}
\author{Benoît Legat}


\begin{document}
\maketitle

\section{Les vecteurs}

Il y a deux types de vecteurs
\begin{description}
	\item[Lié] Noté $\fv{u^v}$, vecteur de d'origine $\fv{v}$.
	\item[Libre] Noté $\fv{u}$, vecteur sans origine fixe.
\end{description}
La notation $\uv{u}$ indique que $||\uv{u}|| = 1$. $\uv{u}$ est appelé un vecteur unitaire.

\subsection{Base orthonormée}

La base orthonormée $\{\ui\}$ est une base composée de 3 vecteurs unitaires orthogonaux $\ui_1$, $\ui_2$ et $\ui_3$ respectivement arrangés dans l'ordre donné par la règle de la main droite.

Tout vecteur $\fv{u}$ a des coordonnées unique $u_1, u_2, u_3$ dans $\{\ui\}$.
On a
\[ \fv{u} = u_1\ui_1 + u_2\ui_2 + u_3\ui_3 = [\ui]^T u \]
	où $[\ui] = \begin{pmatrix}\ui_1\\\ui_2\\\ui_3\end{pmatrix}$ et $u = \begin{pmatrix} u_1 \\ u_2 \\ u_3 \end{pmatrix}$.
Il est important de noter que $[\ui][\ui]^T = E$, où $E$ est le tenseur unitaire (neutre pour la multiplication).

\subsection{Produit scalaire}

Soient $\fv{u} = u_1\ui_1 + u_2\ui_2 + u_3\ui_3$ et $\fv{v} = v_1\ui_1 + v_2\ui_2 + v_3\ui_3$.
\begin{eqnarray*}
	\fv{u}.\fv{v} & \eqdef & ||\fv{u}|| ||\fv{v}|| \cos\theta\\
	& = & u_1v_1 + u_2v_2 + u_3v_3\\
	& = & u^T v
\end{eqnarray*}
où $\theta$ est l'angle entre $\fv{u}$ et $\fv{v}$.
Le produit scalaire est {\em commutatif}, {\em associatif} et {\em bilinéaire}.

\subsection{Produit vectoriel}

Soient $\fv{u} = u_1\ui_1 + u_2\ui_2 + u_3\ui_3$ et $\fv{v} = v_1\ui_1 + v_2\ui_2 + v_3\ui_3$.
\begin{eqnarray*}
	\fv{u}\times\fv{v} & \eqdef & (u_2v_3 - u_3v_2) \ui_1 + (u_3v_1 - u_1v_3) \ui_2 + (u_1v_2 - u_2v_1) \ui_3\\
	& = & [\ui]^T \begin{pmatrix}u_2v_3 - u_3v_2 \\ u_3v_1 - u_1v_3 \\ u_1v_2 - u_2v_1\end{pmatrix}\\
	& = & [\ui]^T \tilde{u}v\\
	& = & \tilde{\fv{u}}.\fv{v}
\end{eqnarray*}
où $\tilde{u} \eqdef \begin{pmatrix}0 & -u_3 & u_2\\ u_3 & 0 & -u_1\\ -u_2 & u_1 & 0\end{pmatrix}$
	et $\tilde{\fv{u}} \eqdef [\ui]^T \tilde{u} [\ui]$. On voit que $\tilde{u}^T = -\tilde{u}$.
Le produit vectoriel est {\em anticommutatif}, {\em pas associatif} et {\em bilinéaire}.

\section{Changement de base}

Pour tout changement de base de $\{\ui\}$ à $\{\uj\}$, il existe une matrice de rotation $A$ tel que
\[ [\uj] = A [\ui] \]
$A$ est orthogonale, c'est à dire que $AA^T = E = A^TA$ ou encore $A^{-1} = A^T$. On a donc
\[ [\ui] = A^T [\uj] \]
Si $\{\uj\}$ respecte la règle de la main droite, on a aussi $\det A = 1$, on dit alors que $A$ est orthogonale directe.\\
On peut faire passer les coordonnées d'un vecteur aisément d'une base à l'autre. Soit $\fv{u} = [\uj]^T \wrt{J}{u}$, on a
\[ \fv{u} = [\uj]^T \wrt{J}{u} = (A[\ui])^T \wrt{J}{u} = [\ui]^TA^T \wrt{J}{u} \]

\section{Dérivées temporelles}

\subsection{Dérivée première}
On peut calculer la dérivée temporelle de $\fv{u}$ ainsi
\begin{eqnarray*}
	\fvd{u} &=& [\uj]^T \wrt{J}{\dot{u}} + [\uj]^T \omegat.\wrt{J}{u}\\
	&=& \fvr{u} + \omegaft . \fv{u}\\
	&=& \fvr{u} + \omegaf \times \fv{u}
\end{eqnarray*}
où $\omegat = A\dot{A}^T$.
On voit ici que, comme $\omegaft$ est multiplié à $\fv{u} = [\uj]^T \wrt{J}{u}$. Il nous faut $\omegat$ dans la base $\{\uj\}$, $\omegaft = [\uj]^T \omegat [\uj]$ et de même donc pour $\omega$, $\omegaf = [\uj]^T \omega$.

Quand on a plusieurs changement de base dans un même problème, on précise pour chaque $\omegaf$ à quel changement de base il correspond. Si le contexte n'avait pas été clair, j'aurais du appeler mon $\omegaf$ de tout à l'heure $\omegaf^{JI}$.

Cette notation nous permet d'énoncer une propriété fondamentale !
\[ \omegaf^{KI} = \omegaf^{KJ} + \omegaf^{JI} \]

Dans la pratique, il faudra parfois effectuer des changements de base. Souvent, on aime bien exprimer $\omegaf$ dans la base intermédiaire (ici $\{\uj\}$) car ça fait moins de changement de base successif.

\paragraph{Exemple}
Si on a $\omegaf^{KJ} = [\uk]^T \wtr{K}{\omega}{KJ}$, $\omegaf^{JI} = [\uj]^T \wtr{J}{\omega}{JI}$ et $[\uk] = A[\uj]$, on calcule
\begin{eqnarray*}
	[\uj]^T \wtr{J}{\omega}{KI} &=& [\uk]^T \wtr{K}{\omega}{KJ} + [\uj]^T \wtr{J}{\omega}{JI}\\
	&=& [\uj]^T A^T \wtr{K}{\omega}{KJ} + [\uj]^T \wtr{J}{\omega}{JI}
\end{eqnarray*}
D'où
\[ \wtr{J}{\omega}{KI} = A^T \wtr{K}{\omega}{KJ} + \wtr{J}{\omega}{JI} \]

\subsection{Dérivée seconde}
Il y a deux manières de calculer la dérivée seconde.
\begin{itemize}
	\item
		Si on a calculé $\fvd{u}$ précédemment, il suffit de le dériver avec la formule de la dérivée première pour avoir la dérivée seconde.
		\[ \fvdd{u} = \mathring{\fvd{u}} + \omegaf \times \fvd{u} \]
	\item
		Directement à partir de $\fv{u}$ et de la formule de la dérivée seconde
		\begin{eqnarray*}
			\fvdd{u} &=&  [\uj]^T \wrt{J}{\ddot{u}} + 2 [\uj]^T \omegat.\wrt{J}{\dot{u}} + [\uj]^T \dot{\omegat}.\wrt{J}{u} + [\uj]^T \omegat.\omegat.\wrt{J}{u}\\
			&=& \fvrr{u} + 2 \omegaft . \fvr{u} + \omegaftr . \fv{u} + \omegaft . \omegaft . \fv{u}\\
			&=& \fvrr{u} + 2 \omegaf \times \fvr{u} + \omegafr \times \fv{u} + \omegaf \times (\omegaf \times \fv{u})
		\end{eqnarray*}
\end{itemize}

\section{Roulement sans glissement}

Si on a une roue ou une boule se déplaçant sur un sol, on peut lui définir un roulement sans glissement.
La condition s'énonce assez simplement.
La vitesse d'un point de la roue est nulle lorsque ce point est en contact avec le sol.
Il est important qu'elle soit nulle dans toutes les directions.
Il est aussi primordial de considérer la vitesse nulle selon un {\bf repère fixe} !

Dans la pratique, le plus simple est de prendre la position d'un point de la roue par rapport à un point fixe (l'origine est un choix judicieux comme point fixe).
Il faut ensuite dériver cette position en fonction du temps. L'expression qu'on obtient pour la vitesse ne doit pas spécialement être nulle pour tout $t$, il faut identifier les moments pour lesquels le point sera en contact avec le sol et s'assurer qu'à ce moment là la vitesse sera nulle selon toute direction.

\paragraph{Exemple}
Si on a une roue de rayon $R$ tournant autour de $\ui_1$ avec une position angulaire $\theta(t)$, le sol ayant l'équation $z = -R$.
On définit le repère cure-dent $\{\uj\}$ à la roue.
Comme elle tourne autour de $\ui_1$ avec un angle $\theta(t)$,
\[ \omegaf = \dot{\theta}(t) \ui_1 = \dot{\theta}(t) \uj_1 = [\uj]^T \begin{pmatrix}\dot{\theta}(t)\\0\\0\end{pmatrix} \]
Prenons, sans perte de généralité, notre point $X$ au bord de la roue sur l'axe formé par $\uj_2$ et posons $C$, le centre de la roue
\[ \vec{OX} = \vec{OC} + R \uj_2 = \vec{OC} + [\uj]^T \begin{pmatrix}0\\R\\0\end{pmatrix} \]
Calculons à présent la dérivée de $\vec{OX}$ en fonction du temps
\begin{eqnarray*}
	\dot{\vec{OX}} &=& \dot{\vec{OC}} + [\uj]^T \begin{pmatrix}0\\\dot{R}\\0\end{pmatrix} + [\uj]^T \omegat . \begin{pmatrix}0\\R\\0\end{pmatrix}\\
	&=& \dot{\vec{OC}} + [\uj]^T \begin{pmatrix}0\\0\\0\end{pmatrix} + (\dot{\theta}(t) \uj_1) \times (R \uj_2)\\
	&=& \dot{\vec{OC}} + \dot{\theta}(t) R (\uj_1 \times \uj_2)\\
	&=& \dot{\vec{OC}} + \dot{\theta}(t) R \uj_3\\
\end{eqnarray*}
Nous avons ici que la vitesse est tangentielle à la roue et vaut la vitesse angulaire fois le rayon.
Ce qui est une formule évidente du mouvement rectiligne uniforme.
Il est important de remarquer ici que le repère $\{\uj\}$ n'est {\bf pas fixe} !
Il faudra donc d'abord l'exprimer selon $\{\ui\}$ avant d'imposer que la vitesse soit nulle en contact avec le sol.
On a
\[ [\uj] = A^{(1)} [\ui] \]
D'où ($A^{(1)}$ est dans le formulaire)
\[ \uj_3 = -\sin\theta(t) \ui_2 + \cos\theta(t) \ui_3 \]
On trouve donc enfin
\[ \dot{\vec{OX}} = \dot{\vec{OC}} + \dot{\theta}(t) R (-\sin\theta(t) \ui_2 + \cos\theta(t) \ui_3) \]
Vu la position du sol choisie, $X$ sera en contact avec le sol ssi $\exists k \in \mathbb{Z}$ tel que $\theta(t) = \frac{3\pi}{2} + 2k\pi$.
On aura alors
\[ \dot{\vec{OX}} = \dot{\vec{OC}} + \dot{\theta}(t) R \ui_2 \]
A ce moment là, la vitesse doit être nulle, on a donc
\[ \dot{\vec{OC}} = -\dot{\theta}(t) R \ui_2 \]
Ce qui est intuitivement pas si choquant que ça.


\section{Corps rigide}
Un corps rigide $C$ est un corps qui ne peut être déformé.
C'est à dire que quelles que soient le point $P$ et la base $\{\uj\}$ solidaires au corps $C$, pour tout point $X$ solidaire au corps $C$, les coordonnées de $\vec{PX}$ dans $\{\uj\}$ sont constantes.
On a aussi que $\omegaf^{JI}$ ne dépend pas du choix de $\{\uj\}$.

\paragraph{Remarque}
Par la suite, pour plus de concision, on utilisera les notations suivantes
\begin{eqnarray*}
	\int_C &\eqdef& \int_{X \in C}\\
	\fv{x}^Q &\eqdef& \vec{OQ}\\
	\fv{r} &\eqdef& \vec{GX}\\
	m(C) &\eqdef& \int_C dm
\end{eqnarray*}
Dans le cadre de cette synthèse on respectera aussi la convention suivante
\begin{itemize}
	\item $G$ sera toujours le centre de masse
	\item $P$ sera toujours un point quelconque solidaire à $C$
	\item $Q$ sera toujours un point quelconque de l'espace
\end{itemize}

Tout corps $C$ a un centre de masse unique $G$ définit comme suit
\[ \xg = \frac{\int_C \fv{x}^X dm}{m(C)} \]

$G$ est aussi l'unique point respectant la propriété suivante
\[ \int_C \fv{r} dm = \int_C \fvd{r} dm = 0 \]

\subsection{Vitesse et accélération}
Soit $X$, un point solidaire au corps $C$, on a
\begin{eqnarray*}
	\fvd{x}^X &=& \fvd{x}^P + \omegaf \times \vec{PX}\\
	\fvdd{x}^X &=& \fvdd{x}^P + \omegafr \times \vec{PX} + \omegaf \times (\omegaf \times \vec{PX})
\end{eqnarray*}

\subsection{Quantité de mouvement linéaire et angulaire}
Soient $\lm(C)$ la quantité de mouvement linéaire du corps $C$ et $\am^P(C)$ la quantité de mouvement angulaire du corps $C$ par rapport au point $P$.
On a
\begin{eqnarray*}
	\lm(C) &=& m(C) \xgd\\
	\am^O(C) &=& \xg \times m(C)\xgd + \am^G(C)\\
	\am^G(C) &=& \ine^G . \omegaf
\end{eqnarray*}
où $\ine^G \eqdef -\int_C \tilde{\fv{r}}.\tilde{\fv{r}} dm$, on l'appelle le tenseur d'inertie. Il respecte les propriétés suivantes

\begin{itemize}
	\item Il est constant lorsqu'il est exprimé dans une base solidaire au corps $C$
	\item Il est symétrique
	\item Il est semi-défini positif
	\item
		Il existe une base $\{\uk\}$ telle que $\wtr{K}{I}{G}$ ($\ine^G = [\uk]^T \wtr{K}{I}{G}$) est diagonale. On appelle les $\uk_i$ les axes principaux d'inertie.
	\item
		On peut passer d'un tenseur d'inertie par rapport au point $G$ à ce même tenseur par rapport à un point $P$ avec la formule suivante
		\[ \ine^P = -m(C)\tilde{\fv{d}}.\tilde{\fv{d}} + \ine^G \]
		où $\fv{d} = \vec{PG}$ (ou $\vec{GP}$, ça revient au même car $\tilde{-\fv{d}}.\tilde{-\fv{d}} = (-\tilde{\fv{d}}).(-\tilde{\fv{d}}) = \tilde{\fv{d}}.\tilde{\fv{d}}$).
		Il est primordial que $G$ soit le centre de masse du corps en question pour que cette formule soit correcte !
		C'est la {\em formule de Steiner}.
	\item
		Si $C = \cup_i \{C_i\}$, alors
		\[ \ine^G = \sum_i \ine_{C^i}^G \]
		Il faut faire attention néanmoins, tous les tenseurs d'inerties doivent être exprimée par rapport au même point pour pouvoir être sommés, il faut utiliser la formule de Steiner pour changer de point de référence.
\end{itemize}

\subsection{Forces et moments de force}
La somme des moments agissant sur un corps $C$ par rapport à un point $Q$ est noté $\st^Q$.

Si la somme des forces est nulle, $\st^Q$ ne dépend pas du choix de $Q$.
On appelle alors $\st^Q$ un pure moment de force et on le note $\pst$.

Si sur un corps $C$ est appliqué des forces $F^i$ aux points $A^i$ et des pures moments de force $M^j$, on a,% pour n'importe quel point de l'espace $Q$ % Point `fixe' de l'espace ?
\begin{eqnarray*}
	\fo &=& \sum_i \fo^i\\
	\st^Q &=&  \sum_i \vec{QA^i}\times\fo^i + \sum_j \pst^j
\end{eqnarray*}

\subsection{Puissance}
Soit $\mathcal{P}$ la puissance correspondantes au forces et au pures moments de force appliqués au corps $C$. Elle peut être calculée par la formule suivante
\[ \mathcal{P} = \fo.\fvd{x}^P + \st^P.\omegaf \]
%où $P$ est un point quelconque solidaire au corps $C$.

\subsection{Les équations de Newton et Euler}
Comme $\{\ui\}$ est une base inertielle, les 3 lois de Newtons s'appliquent
\begin{enumerate}
	\item Si la résultante des forces externes agissant sur un corps $C$ est nulle, alors $\lm(C)$ est constant.
	\item La résultante des forces externes agissant sur un corps $C$ est égale à la dérivée temporelle de sa quantité de mouvement linéaire
		\begin{eqnarray*}
			\dot{\lm}(C) &=& \fo\\
			&=& m(C) \xgdd
		\end{eqnarray*}
	\item Soient $\fo^{1,2}$, une force appliquée au corps $C^2$ par le corps $C^1$.
		La force de réaction résultante $\fo^{2,1}$ appliquée au corps $C^1$ par le corps $C^2$ est donnée par
		\[ \fo^{2,1} = -\fo^{1,2} \]
\end{enumerate}

Les mêmes lois se transposent très bien pour les quantités de mouvement angulaire et les moment de forces
\begin{enumerate}
	\item Si la résultante des moments de force externes agissant sur un corps $C$ est nulle, alors $\am(C)$ est constant.
	\item La résultante des moments de force externes agissant sur un corps $C$ est égale à la dérivée temporelle de sa quantité de mouvement angulaire
		\begin{eqnarray*}
			\dot{\am}^Q(C) &=& \st^Q\\
			&=& \ine^Q.\omegafd + \omegaft.\ine^Q.\omegaf
		\end{eqnarray*}
		La première des deux équations est connue sous le nom d'{\em équation rotationelle} ou d'{\em équation d'Euler}.
	\item Soient $\st^{1,2}$, un moment de force appliqué au corps $C^2$ par le corps $C^1$.
		Le moment de force de réaction résultant $\st^{2,1}$ appliquée au corps $C^1$ par le corps $C^2$ est donnée par
		\[ \st^{2,1} = -\st^{1,2} \]
\end{enumerate}

\end{document}
