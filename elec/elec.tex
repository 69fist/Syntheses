%%% elec.tex --- 

%% Author: gp2mv3
%% Version: $Id: elec.tex,v 0.0 2011/11/26 11:58:14 gp2mv3 Exp$


\documentclass[11pt,a4paper,french]{article}
\usepackage[utf8]{inputenc}
%\usepackage[T1]{fontenc}
\usepackage{lmodern}
\usepackage[french]{babel}
%\usepackage[debugshow,final]{graphics}

%%\revision$Header: elec.tex,v 0.0 2011/11/26 11:58:14 gp2mv3 Exp$

\pagestyle{plain}

\title{Electricity}
\author{Nicolas Cognaux \and Benoit Legat}
\date{\today}


\begin{document}
\maketitle

\section{Electric field and electric charge}
Two charges with the same sign are attracted. Two opposite charges are repeled.

\subsection{Coulomb's law}
The magnitude of the force between two charges is:
$$ F = \frac{1}{4\pi\epsilon_0}\frac{|q_1q_2|}{r^2} $$
$$ \frac{1}{4\pi\epsilon_0} = k \cong 9.0x10^9N.m^2/C^2$$


\subsection{Electric Field and Electric Forces}
The electric field is the electric Force per unit of charge
$$ \vec{E} = \frac{\vec{F}}{q} $$

So $\vec{E}$ can be written as:
$$ \vec{E} = \frac{1}{4\pi\epsilon_0}\frac{q}{r^2}\hat{r} $$

\subsection{Electric Field Lines}
Electric field lines starts at positive charge and go to negative charge.

\begin{itemize}
\item They never intersect
\item The number of field lines can give an idea of the magnitude of the Electric Field.
\item The electric field is tangeant with the electric field lines.
\end{itemize}

\subsection{Electric Dipole}
If two opposit charges are linked, they can form an electric Dipole. There is a rotation and so a couple.

The magnitude of the torque can be written as:
$$ \tau = pE\sin{\theta} $$
$$\vec{\tau} = \vec{P}x\vec{E} $$

The potential energy of an Electric Dipole is:
$$ U = -\vec{p}. \vec{E} $$

\section{Gauss's Law}
\subsection{Electric Flux}
The Electric flux is the quantity of electric charges passing throug a defined surface.
$$ \frac{dV}{dt} = vA\cos{\theta} = v \perp A = \vec{v}.\vec{A} $$


\end{document}
